
%%%%%%%%%%%%%%%%%%%%%%%%%%%%%%%%%%%%%%%%%%%%%%%%%%%%%%%%%%%%%%%%%%%%%%%%%%%%%
%
% Package pgfplots.sty documentation.
%
% Copyright 2007/2008 by Christian Feuersaenger.
%
% This program is free software: you can redistribute it and/or modify
% it under the terms of the GNU General Public License as published by
% the Free Software Foundation, either version 3 of the License, or
% (at your option) any later version.
%
% This program is distributed in the hope that it will be useful,
% but WITHOUT ANY WARRANTY; without even the implied warranty of
% MERCHANTABILITY or FITNESS FOR A PARTICULAR PURPOSE.  See the
% GNU General Public License for more details.
%
% You should have received a copy of the GNU General Public License
% along with this program.  If not, see <http://www.gnu.org/licenses/>.
%
%
%%%%%%%%%%%%%%%%%%%%%%%%%%%%%%%%%%%%%%%%%%%%%%%%%%%%%%%%%%%%%%%%%%%%%%%%%%%%%
% SEE pgfplots-macros.tex as well!
%\pdfminorversion=5 % to allow compression
%\pdfobjcompresslevel=2
\documentclass[a4paper,openany]{book}

\let\bookmaketitle=\maketitle

% -----------------------------------
% this here is from ltxdoc:
\usepackage{doc}[=v2]
\AtBeginDocument{\MakeShortVerb{\|}}
%\setlength{\textwidth}{355pt}
%\addtolength\marginparwidth{30pt}
%\addtolength\oddsidemargin{20pt}
%\addtolength\evensidemargin{20pt}
\def\cmd#1{\cs{\expandafter\cmd@to@cs\string#1}}
\def\cmd@to@cs#1#2{\char\number`#2\relax}
\DeclareRobustCommand\cs[1]{\texttt{\char`\\#1}}
\providecommand\marg[1]{%
  {\ttfamily\char`\{}\meta{#1}{\ttfamily\char`\}}}
\providecommand\oarg[1]{%
  {\ttfamily[}\meta{#1}{\ttfamily]}}
\providecommand\parg[1]{%
  {\ttfamily(}\meta{#1}{\ttfamily)}}
\raggedbottom

% -----------------------------------
\input{pgfplots.preamble.tex}

\makeatletter
% I want a two-column index just as in pgfmanual styles. This here
% was the best way to get one:
\def\index@prologue{\section*{Index}\addcontentsline{toc}{chapter}{Index}
}
\makeatother

%\RequirePackage[german,english,francais]{babel}

\def\matlabcolormaptext{This colormap is similar to one shipped with Matlab$^\text{\textregistered}$ under a similar name.}

\IfFileExists{tikzlibraryspy.code.tex}{%
\usetikzlibrary{spy}
}{%
    \message{ERROR: tikz SPY library NOT available. The manual will only compile partially.^^J}%
}%

\usepackage{xparse}% for colorbrewer manual

\usetikzlibrary{
    decorations.markings,
    decorations.footprints,
    shapes.arrows,
    matrix,
    positioning,
}

\usepgfplotslibrary{
    fillbetween,
    ternary,
    smithchart,
    patchplots,
    polar,
    colormaps,
    colorbrewer,
    colortol,           % docu not ready yet
}
\pgfqkeys{/codeexample}{%
    every codeexample/.append style={
        /pgfplots/every ternary axis/.append style={
            /pgfplots/legend style={fill=graphicbackground},
        }
    },
    tabsize=4,
}

\pgfplotsmanualenableexternalizationofexpensive

%\usetikzlibrary{external}
%\tikzexternalize[prefix=figures/]{pgfplots}

\title{%
    Manual for Package \PGFPlots{}\\
    {\small 2D/3D Plots in \LaTeX{}, Version \pgfplotsversion}\\
    {\small\url{https://github.com/pgf-tikz/pgfplots}}
    %\\{\small Attention: you are using an unstable development version.}
}

\makeatletter
\long\def\abstractsmuggle{%
    \centering
    \textbf{Abstract}\\[0.5cm]

    \begin{minipage}{12cm}
        \PGFPlots{} draws high-quality function plots in normal or logarithmic
        scaling with a user-friendly interface directly in \TeX{}. The user
        supplies axis labels, legend entries and the plot coordinates for one or
        more plots and \PGFPlots{} applies axis scaling, computes any logarithms
        and axis ticks and draws the plots. It supports line plots, scatter
        plots, piecewise constant plots, bar plots, area plots, mesh and surface
        plots, patch plots, contour plots, quiver plots, histogram plots, box
        plots, polar axes, ternary diagrams, smith charts and some more. It is
        based on Till Tantau's package \PGF{}/\Tikz{}.
    \end{minipage}

}%

\expandafter\date\expandafter{\@date\\[2cm]
    \abstractsmuggle
}%
\makeatother

%\includeonly{pgfplots.reference}


\begin{document}

\def\plotcoords{%
\addplot coordinates {
(5,8.312e-02)    (17,2.547e-02)   (49,7.407e-03)
(129,2.102e-03)  (321,5.874e-04)  (769,1.623e-04)
(1793,4.442e-05) (4097,1.207e-05) (9217,3.261e-06)
};

\addplot coordinates{
(7,8.472e-02)    (31,3.044e-02)    (111,1.022e-02)
(351,3.303e-03)  (1023,1.039e-03)  (2815,3.196e-04)
(7423,9.658e-05) (18943,2.873e-05) (47103,8.437e-06)
};

\addplot coordinates{
(9,7.881e-02)     (49,3.243e-02)    (209,1.232e-02)
(769,4.454e-03)   (2561,1.551e-03)  (7937,5.236e-04)
(23297,1.723e-04) (65537,5.545e-05) (178177,1.751e-05)
};

\addplot coordinates{
(11,6.887e-02)    (71,3.177e-02)     (351,1.341e-02)
(1471,5.334e-03)  (5503,2.027e-03)   (18943,7.415e-04)
(61183,2.628e-04) (187903,9.063e-05) (553983,3.053e-05)
};

\addplot coordinates{
(13,5.755e-02)     (97,2.925e-02)     (545,1.351e-02)
(2561,5.842e-03)   (10625,2.397e-03)  (40193,9.414e-04)
(141569,3.564e-04) (471041,1.308e-04)
(1496065,4.670e-05)
};
}%


\bookmaketitle
\tableofcontents

%%%%%%%%%%%%%%%%%%%%%%%%%%%%%%%%%%%%%%%%%%%%%%%%%%%%%%%%%%%%%%%%%%%%%%%%%%%%%
%
% Package pgfplots.sty documentation.
%
% Copyright 2007/2008 by Christian Feuersaenger.
%
% This program is free software: you can redistribute it and/or modify
% it under the terms of the GNU General Public License as published by
% the Free Software Foundation, either version 3 of the License, or
% (at your option) any later version.
%
% This program is distributed in the hope that it will be useful,
% but WITHOUT ANY WARRANTY; without even the implied warranty of
% MERCHANTABILITY or FITNESS FOR A PARTICULAR PURPOSE.  See the
% GNU General Public License for more details.
%
% You should have received a copy of the GNU General Public License
% along with this program.  If not, see <http://www.gnu.org/licenses/>.
%
%
%%%%%%%%%%%%%%%%%%%%%%%%%%%%%%%%%%%%%%%%%%%%%%%%%%%%%%%%%%%%%%%%%%%%%%%%%%%%%
% SEE pgfplots-macros.tex as well!
%\pdfminorversion=5 % to allow compression
%\pdfobjcompresslevel=2
\documentclass[a4paper,openany]{book}

\let\bookmaketitle=\maketitle

% -----------------------------------
% this here is from ltxdoc:
\usepackage{doc}[=v2]
\AtBeginDocument{\MakeShortVerb{\|}}
%\setlength{\textwidth}{355pt}
%\addtolength\marginparwidth{30pt}
%\addtolength\oddsidemargin{20pt}
%\addtolength\evensidemargin{20pt}
\def\cmd#1{\cs{\expandafter\cmd@to@cs\string#1}}
\def\cmd@to@cs#1#2{\char\number`#2\relax}
\DeclareRobustCommand\cs[1]{\texttt{\char`\\#1}}
\providecommand\marg[1]{%
  {\ttfamily\char`\{}\meta{#1}{\ttfamily\char`\}}}
\providecommand\oarg[1]{%
  {\ttfamily[}\meta{#1}{\ttfamily]}}
\providecommand\parg[1]{%
  {\ttfamily(}\meta{#1}{\ttfamily)}}
\raggedbottom

% -----------------------------------
\input{pgfplots.preamble.tex}

\makeatletter
% I want a two-column index just as in pgfmanual styles. This here
% was the best way to get one:
\def\index@prologue{\section*{Index}\addcontentsline{toc}{chapter}{Index}
}
\makeatother

%\RequirePackage[german,english,francais]{babel}

\def\matlabcolormaptext{This colormap is similar to one shipped with Matlab$^\text{\textregistered}$ under a similar name.}

\IfFileExists{tikzlibraryspy.code.tex}{%
\usetikzlibrary{spy}
}{%
    \message{ERROR: tikz SPY library NOT available. The manual will only compile partially.^^J}%
}%

\usepackage{xparse}% for colorbrewer manual

\usetikzlibrary{
    decorations.markings,
    decorations.footprints,
    shapes.arrows,
    matrix,
    positioning,
}

\usepgfplotslibrary{
    fillbetween,
    ternary,
    smithchart,
    patchplots,
    polar,
    colormaps,
    colorbrewer,
    colortol,           % docu not ready yet
}
\pgfqkeys{/codeexample}{%
    every codeexample/.append style={
        /pgfplots/every ternary axis/.append style={
            /pgfplots/legend style={fill=graphicbackground},
        }
    },
    tabsize=4,
}

\pgfplotsmanualenableexternalizationofexpensive

%\usetikzlibrary{external}
%\tikzexternalize[prefix=figures/]{pgfplots}

\title{%
    Manual for Package \PGFPlots{}\\
    {\small 2D/3D Plots in \LaTeX{}, Version \pgfplotsversion}\\
    {\small\url{https://github.com/pgf-tikz/pgfplots}}
    %\\{\small Attention: you are using an unstable development version.}
}

\makeatletter
\long\def\abstractsmuggle{%
    \centering
    \textbf{Abstract}\\[0.5cm]

    \begin{minipage}{12cm}
        \PGFPlots{} draws high-quality function plots in normal or logarithmic
        scaling with a user-friendly interface directly in \TeX{}. The user
        supplies axis labels, legend entries and the plot coordinates for one or
        more plots and \PGFPlots{} applies axis scaling, computes any logarithms
        and axis ticks and draws the plots. It supports line plots, scatter
        plots, piecewise constant plots, bar plots, area plots, mesh and surface
        plots, patch plots, contour plots, quiver plots, histogram plots, box
        plots, polar axes, ternary diagrams, smith charts and some more. It is
        based on Till Tantau's package \PGF{}/\Tikz{}.
    \end{minipage}

}%

\expandafter\date\expandafter{\@date\\[2cm]
    \abstractsmuggle
}%
\makeatother

%\includeonly{pgfplots.reference}


\begin{document}

\def\plotcoords{%
\addplot coordinates {
(5,8.312e-02)    (17,2.547e-02)   (49,7.407e-03)
(129,2.102e-03)  (321,5.874e-04)  (769,1.623e-04)
(1793,4.442e-05) (4097,1.207e-05) (9217,3.261e-06)
};

\addplot coordinates{
(7,8.472e-02)    (31,3.044e-02)    (111,1.022e-02)
(351,3.303e-03)  (1023,1.039e-03)  (2815,3.196e-04)
(7423,9.658e-05) (18943,2.873e-05) (47103,8.437e-06)
};

\addplot coordinates{
(9,7.881e-02)     (49,3.243e-02)    (209,1.232e-02)
(769,4.454e-03)   (2561,1.551e-03)  (7937,5.236e-04)
(23297,1.723e-04) (65537,5.545e-05) (178177,1.751e-05)
};

\addplot coordinates{
(11,6.887e-02)    (71,3.177e-02)     (351,1.341e-02)
(1471,5.334e-03)  (5503,2.027e-03)   (18943,7.415e-04)
(61183,2.628e-04) (187903,9.063e-05) (553983,3.053e-05)
};

\addplot coordinates{
(13,5.755e-02)     (97,2.925e-02)     (545,1.351e-02)
(2561,5.842e-03)   (10625,2.397e-03)  (40193,9.414e-04)
(141569,3.564e-04) (471041,1.308e-04)
(1496065,4.670e-05)
};
}%


\bookmaketitle
\tableofcontents

%%%%%%%%%%%%%%%%%%%%%%%%%%%%%%%%%%%%%%%%%%%%%%%%%%%%%%%%%%%%%%%%%%%%%%%%%%%%%
%
% Package pgfplots.sty documentation.
%
% Copyright 2007/2008 by Christian Feuersaenger.
%
% This program is free software: you can redistribute it and/or modify
% it under the terms of the GNU General Public License as published by
% the Free Software Foundation, either version 3 of the License, or
% (at your option) any later version.
%
% This program is distributed in the hope that it will be useful,
% but WITHOUT ANY WARRANTY; without even the implied warranty of
% MERCHANTABILITY or FITNESS FOR A PARTICULAR PURPOSE.  See the
% GNU General Public License for more details.
%
% You should have received a copy of the GNU General Public License
% along with this program.  If not, see <http://www.gnu.org/licenses/>.
%
%
%%%%%%%%%%%%%%%%%%%%%%%%%%%%%%%%%%%%%%%%%%%%%%%%%%%%%%%%%%%%%%%%%%%%%%%%%%%%%
% SEE pgfplots-macros.tex as well!
%\pdfminorversion=5 % to allow compression
%\pdfobjcompresslevel=2
\documentclass[a4paper,openany]{book}

\let\bookmaketitle=\maketitle

% -----------------------------------
% this here is from ltxdoc:
\usepackage{doc}[=v2]
\AtBeginDocument{\MakeShortVerb{\|}}
%\setlength{\textwidth}{355pt}
%\addtolength\marginparwidth{30pt}
%\addtolength\oddsidemargin{20pt}
%\addtolength\evensidemargin{20pt}
\def\cmd#1{\cs{\expandafter\cmd@to@cs\string#1}}
\def\cmd@to@cs#1#2{\char\number`#2\relax}
\DeclareRobustCommand\cs[1]{\texttt{\char`\\#1}}
\providecommand\marg[1]{%
  {\ttfamily\char`\{}\meta{#1}{\ttfamily\char`\}}}
\providecommand\oarg[1]{%
  {\ttfamily[}\meta{#1}{\ttfamily]}}
\providecommand\parg[1]{%
  {\ttfamily(}\meta{#1}{\ttfamily)}}
\raggedbottom

% -----------------------------------
\input{pgfplots.preamble.tex}

\makeatletter
% I want a two-column index just as in pgfmanual styles. This here
% was the best way to get one:
\def\index@prologue{\section*{Index}\addcontentsline{toc}{chapter}{Index}
}
\makeatother

%\RequirePackage[german,english,francais]{babel}

\def\matlabcolormaptext{This colormap is similar to one shipped with Matlab$^\text{\textregistered}$ under a similar name.}

\IfFileExists{tikzlibraryspy.code.tex}{%
\usetikzlibrary{spy}
}{%
    \message{ERROR: tikz SPY library NOT available. The manual will only compile partially.^^J}%
}%

\usepackage{xparse}% for colorbrewer manual

\usetikzlibrary{
    decorations.markings,
    decorations.footprints,
    shapes.arrows,
    matrix,
    positioning,
}

\usepgfplotslibrary{
    fillbetween,
    ternary,
    smithchart,
    patchplots,
    polar,
    colormaps,
    colorbrewer,
    colortol,           % docu not ready yet
}
\pgfqkeys{/codeexample}{%
    every codeexample/.append style={
        /pgfplots/every ternary axis/.append style={
            /pgfplots/legend style={fill=graphicbackground},
        }
    },
    tabsize=4,
}

\pgfplotsmanualenableexternalizationofexpensive

%\usetikzlibrary{external}
%\tikzexternalize[prefix=figures/]{pgfplots}

\title{%
    Manual for Package \PGFPlots{}\\
    {\small 2D/3D Plots in \LaTeX{}, Version \pgfplotsversion}\\
    {\small\url{https://github.com/pgf-tikz/pgfplots}}
    %\\{\small Attention: you are using an unstable development version.}
}

\makeatletter
\long\def\abstractsmuggle{%
    \centering
    \textbf{Abstract}\\[0.5cm]

    \begin{minipage}{12cm}
        \PGFPlots{} draws high-quality function plots in normal or logarithmic
        scaling with a user-friendly interface directly in \TeX{}. The user
        supplies axis labels, legend entries and the plot coordinates for one or
        more plots and \PGFPlots{} applies axis scaling, computes any logarithms
        and axis ticks and draws the plots. It supports line plots, scatter
        plots, piecewise constant plots, bar plots, area plots, mesh and surface
        plots, patch plots, contour plots, quiver plots, histogram plots, box
        plots, polar axes, ternary diagrams, smith charts and some more. It is
        based on Till Tantau's package \PGF{}/\Tikz{}.
    \end{minipage}

}%

\expandafter\date\expandafter{\@date\\[2cm]
    \abstractsmuggle
}%
\makeatother

%\includeonly{pgfplots.reference}


\begin{document}

\def\plotcoords{%
\addplot coordinates {
(5,8.312e-02)    (17,2.547e-02)   (49,7.407e-03)
(129,2.102e-03)  (321,5.874e-04)  (769,1.623e-04)
(1793,4.442e-05) (4097,1.207e-05) (9217,3.261e-06)
};

\addplot coordinates{
(7,8.472e-02)    (31,3.044e-02)    (111,1.022e-02)
(351,3.303e-03)  (1023,1.039e-03)  (2815,3.196e-04)
(7423,9.658e-05) (18943,2.873e-05) (47103,8.437e-06)
};

\addplot coordinates{
(9,7.881e-02)     (49,3.243e-02)    (209,1.232e-02)
(769,4.454e-03)   (2561,1.551e-03)  (7937,5.236e-04)
(23297,1.723e-04) (65537,5.545e-05) (178177,1.751e-05)
};

\addplot coordinates{
(11,6.887e-02)    (71,3.177e-02)     (351,1.341e-02)
(1471,5.334e-03)  (5503,2.027e-03)   (18943,7.415e-04)
(61183,2.628e-04) (187903,9.063e-05) (553983,3.053e-05)
};

\addplot coordinates{
(13,5.755e-02)     (97,2.925e-02)     (545,1.351e-02)
(2561,5.842e-03)   (10625,2.397e-03)  (40193,9.414e-04)
(141569,3.564e-04) (471041,1.308e-04)
(1496065,4.670e-05)
};
}%


\bookmaketitle
\tableofcontents

%%%%%%%%%%%%%%%%%%%%%%%%%%%%%%%%%%%%%%%%%%%%%%%%%%%%%%%%%%%%%%%%%%%%%%%%%%%%%
%
% Package pgfplots.sty documentation.
%
% Copyright 2007/2008 by Christian Feuersaenger.
%
% This program is free software: you can redistribute it and/or modify
% it under the terms of the GNU General Public License as published by
% the Free Software Foundation, either version 3 of the License, or
% (at your option) any later version.
%
% This program is distributed in the hope that it will be useful,
% but WITHOUT ANY WARRANTY; without even the implied warranty of
% MERCHANTABILITY or FITNESS FOR A PARTICULAR PURPOSE.  See the
% GNU General Public License for more details.
%
% You should have received a copy of the GNU General Public License
% along with this program.  If not, see <http://www.gnu.org/licenses/>.
%
%
%%%%%%%%%%%%%%%%%%%%%%%%%%%%%%%%%%%%%%%%%%%%%%%%%%%%%%%%%%%%%%%%%%%%%%%%%%%%%
% SEE pgfplots-macros.tex as well!
%\pdfminorversion=5 % to allow compression
%\pdfobjcompresslevel=2
\documentclass[a4paper,openany]{book}

\let\bookmaketitle=\maketitle

% -----------------------------------
% this here is from ltxdoc:
\usepackage{doc}[=v2]
\AtBeginDocument{\MakeShortVerb{\|}}
%\setlength{\textwidth}{355pt}
%\addtolength\marginparwidth{30pt}
%\addtolength\oddsidemargin{20pt}
%\addtolength\evensidemargin{20pt}
\def\cmd#1{\cs{\expandafter\cmd@to@cs\string#1}}
\def\cmd@to@cs#1#2{\char\number`#2\relax}
\DeclareRobustCommand\cs[1]{\texttt{\char`\\#1}}
\providecommand\marg[1]{%
  {\ttfamily\char`\{}\meta{#1}{\ttfamily\char`\}}}
\providecommand\oarg[1]{%
  {\ttfamily[}\meta{#1}{\ttfamily]}}
\providecommand\parg[1]{%
  {\ttfamily(}\meta{#1}{\ttfamily)}}
\raggedbottom

% -----------------------------------
\input{pgfplots.preamble.tex}

\makeatletter
% I want a two-column index just as in pgfmanual styles. This here
% was the best way to get one:
\def\index@prologue{\section*{Index}\addcontentsline{toc}{chapter}{Index}
}
\makeatother

%\RequirePackage[german,english,francais]{babel}

\def\matlabcolormaptext{This colormap is similar to one shipped with Matlab$^\text{\textregistered}$ under a similar name.}

\IfFileExists{tikzlibraryspy.code.tex}{%
\usetikzlibrary{spy}
}{%
    \message{ERROR: tikz SPY library NOT available. The manual will only compile partially.^^J}%
}%

\usepackage{xparse}% for colorbrewer manual

\usetikzlibrary{
    decorations.markings,
    decorations.footprints,
    shapes.arrows,
    matrix,
    positioning,
}

\usepgfplotslibrary{
    fillbetween,
    ternary,
    smithchart,
    patchplots,
    polar,
    colormaps,
    colorbrewer,
    colortol,           % docu not ready yet
}
\pgfqkeys{/codeexample}{%
    every codeexample/.append style={
        /pgfplots/every ternary axis/.append style={
            /pgfplots/legend style={fill=graphicbackground},
        }
    },
    tabsize=4,
}

\pgfplotsmanualenableexternalizationofexpensive

%\usetikzlibrary{external}
%\tikzexternalize[prefix=figures/]{pgfplots}

\title{%
    Manual for Package \PGFPlots{}\\
    {\small 2D/3D Plots in \LaTeX{}, Version \pgfplotsversion}\\
    {\small\url{https://github.com/pgf-tikz/pgfplots}}
    %\\{\small Attention: you are using an unstable development version.}
}

\makeatletter
\long\def\abstractsmuggle{%
    \centering
    \textbf{Abstract}\\[0.5cm]

    \begin{minipage}{12cm}
        \PGFPlots{} draws high-quality function plots in normal or logarithmic
        scaling with a user-friendly interface directly in \TeX{}. The user
        supplies axis labels, legend entries and the plot coordinates for one or
        more plots and \PGFPlots{} applies axis scaling, computes any logarithms
        and axis ticks and draws the plots. It supports line plots, scatter
        plots, piecewise constant plots, bar plots, area plots, mesh and surface
        plots, patch plots, contour plots, quiver plots, histogram plots, box
        plots, polar axes, ternary diagrams, smith charts and some more. It is
        based on Till Tantau's package \PGF{}/\Tikz{}.
    \end{minipage}

}%

\expandafter\date\expandafter{\@date\\[2cm]
    \abstractsmuggle
}%
\makeatother

%\includeonly{pgfplots.reference}


\begin{document}

\def\plotcoords{%
\addplot coordinates {
(5,8.312e-02)    (17,2.547e-02)   (49,7.407e-03)
(129,2.102e-03)  (321,5.874e-04)  (769,1.623e-04)
(1793,4.442e-05) (4097,1.207e-05) (9217,3.261e-06)
};

\addplot coordinates{
(7,8.472e-02)    (31,3.044e-02)    (111,1.022e-02)
(351,3.303e-03)  (1023,1.039e-03)  (2815,3.196e-04)
(7423,9.658e-05) (18943,2.873e-05) (47103,8.437e-06)
};

\addplot coordinates{
(9,7.881e-02)     (49,3.243e-02)    (209,1.232e-02)
(769,4.454e-03)   (2561,1.551e-03)  (7937,5.236e-04)
(23297,1.723e-04) (65537,5.545e-05) (178177,1.751e-05)
};

\addplot coordinates{
(11,6.887e-02)    (71,3.177e-02)     (351,1.341e-02)
(1471,5.334e-03)  (5503,2.027e-03)   (18943,7.415e-04)
(61183,2.628e-04) (187903,9.063e-05) (553983,3.053e-05)
};

\addplot coordinates{
(13,5.755e-02)     (97,2.925e-02)     (545,1.351e-02)
(2561,5.842e-03)   (10625,2.397e-03)  (40193,9.414e-04)
(141569,3.564e-04) (471041,1.308e-04)
(1496065,4.670e-05)
};
}%


\bookmaketitle
\tableofcontents
\include{pgfplots.title_abstract_intro}
\include{pgfplots.preliminaries}
\include{pgfplots.intro}
\include{pgfplots.reference}
\include{pgfplots.libs}
\include{pgfplots.resources}
\include{pgfplots.importexport}
\include{pgfplots.basic.reference}

\printindex

\bibliographystyle{abbrv} %gerapali} %gerabbrv} %gerunsrt.bst} %gerabbrv}% gerplain}
\nocite{pgfplotstable}
\nocite{programmingnotes}
\bibliography{pgfplots}
\end{document}

% -----------------------------------------------------------------------------
% For Stefan Pinnow as reminder on what to look for when editing the manual
% -----------------------------------------------------------------------------
% There should be no line breaks in the following environments
% - |...|
% - \declareandlabel{...}
% - \verbpdfref{...}
% If "MakeTikzPictures" isn't running through check one of the externalized
% LOG files what is the cause of that.
% -----------------------------------------------------------------------------


%%%%%%%%%%%%%%%%%%%%%%%%%%%%%%%%%%%%%%%%%%%%%%%%%%%%%%%%%%%%%%%%%%%%%%%%%%%%%
%
% Package pgfplots.sty documentation.
%
% Copyright 2007/2008 by Christian Feuersaenger.
%
% This program is free software: you can redistribute it and/or modify
% it under the terms of the GNU General Public License as published by
% the Free Software Foundation, either version 3 of the License, or
% (at your option) any later version.
%
% This program is distributed in the hope that it will be useful,
% but WITHOUT ANY WARRANTY; without even the implied warranty of
% MERCHANTABILITY or FITNESS FOR A PARTICULAR PURPOSE.  See the
% GNU General Public License for more details.
%
% You should have received a copy of the GNU General Public License
% along with this program.  If not, see <http://www.gnu.org/licenses/>.
%
%
%%%%%%%%%%%%%%%%%%%%%%%%%%%%%%%%%%%%%%%%%%%%%%%%%%%%%%%%%%%%%%%%%%%%%%%%%%%%%
% SEE pgfplots-macros.tex as well!
%\pdfminorversion=5 % to allow compression
%\pdfobjcompresslevel=2
\documentclass[a4paper,openany]{book}

\let\bookmaketitle=\maketitle

% -----------------------------------
% this here is from ltxdoc:
\usepackage{doc}[=v2]
\AtBeginDocument{\MakeShortVerb{\|}}
%\setlength{\textwidth}{355pt}
%\addtolength\marginparwidth{30pt}
%\addtolength\oddsidemargin{20pt}
%\addtolength\evensidemargin{20pt}
\def\cmd#1{\cs{\expandafter\cmd@to@cs\string#1}}
\def\cmd@to@cs#1#2{\char\number`#2\relax}
\DeclareRobustCommand\cs[1]{\texttt{\char`\\#1}}
\providecommand\marg[1]{%
  {\ttfamily\char`\{}\meta{#1}{\ttfamily\char`\}}}
\providecommand\oarg[1]{%
  {\ttfamily[}\meta{#1}{\ttfamily]}}
\providecommand\parg[1]{%
  {\ttfamily(}\meta{#1}{\ttfamily)}}
\raggedbottom

% -----------------------------------
\input{pgfplots.preamble.tex}

\makeatletter
% I want a two-column index just as in pgfmanual styles. This here
% was the best way to get one:
\def\index@prologue{\section*{Index}\addcontentsline{toc}{chapter}{Index}
}
\makeatother

%\RequirePackage[german,english,francais]{babel}

\def\matlabcolormaptext{This colormap is similar to one shipped with Matlab$^\text{\textregistered}$ under a similar name.}

\IfFileExists{tikzlibraryspy.code.tex}{%
\usetikzlibrary{spy}
}{%
    \message{ERROR: tikz SPY library NOT available. The manual will only compile partially.^^J}%
}%

\usepackage{xparse}% for colorbrewer manual

\usetikzlibrary{
    decorations.markings,
    decorations.footprints,
    shapes.arrows,
    matrix,
    positioning,
}

\usepgfplotslibrary{
    fillbetween,
    ternary,
    smithchart,
    patchplots,
    polar,
    colormaps,
    colorbrewer,
    colortol,           % docu not ready yet
}
\pgfqkeys{/codeexample}{%
    every codeexample/.append style={
        /pgfplots/every ternary axis/.append style={
            /pgfplots/legend style={fill=graphicbackground},
        }
    },
    tabsize=4,
}

\pgfplotsmanualenableexternalizationofexpensive

%\usetikzlibrary{external}
%\tikzexternalize[prefix=figures/]{pgfplots}

\title{%
    Manual for Package \PGFPlots{}\\
    {\small 2D/3D Plots in \LaTeX{}, Version \pgfplotsversion}\\
    {\small\url{https://github.com/pgf-tikz/pgfplots}}
    %\\{\small Attention: you are using an unstable development version.}
}

\makeatletter
\long\def\abstractsmuggle{%
    \centering
    \textbf{Abstract}\\[0.5cm]

    \begin{minipage}{12cm}
        \PGFPlots{} draws high-quality function plots in normal or logarithmic
        scaling with a user-friendly interface directly in \TeX{}. The user
        supplies axis labels, legend entries and the plot coordinates for one or
        more plots and \PGFPlots{} applies axis scaling, computes any logarithms
        and axis ticks and draws the plots. It supports line plots, scatter
        plots, piecewise constant plots, bar plots, area plots, mesh and surface
        plots, patch plots, contour plots, quiver plots, histogram plots, box
        plots, polar axes, ternary diagrams, smith charts and some more. It is
        based on Till Tantau's package \PGF{}/\Tikz{}.
    \end{minipage}

}%

\expandafter\date\expandafter{\@date\\[2cm]
    \abstractsmuggle
}%
\makeatother

%\includeonly{pgfplots.reference}


\begin{document}

\def\plotcoords{%
\addplot coordinates {
(5,8.312e-02)    (17,2.547e-02)   (49,7.407e-03)
(129,2.102e-03)  (321,5.874e-04)  (769,1.623e-04)
(1793,4.442e-05) (4097,1.207e-05) (9217,3.261e-06)
};

\addplot coordinates{
(7,8.472e-02)    (31,3.044e-02)    (111,1.022e-02)
(351,3.303e-03)  (1023,1.039e-03)  (2815,3.196e-04)
(7423,9.658e-05) (18943,2.873e-05) (47103,8.437e-06)
};

\addplot coordinates{
(9,7.881e-02)     (49,3.243e-02)    (209,1.232e-02)
(769,4.454e-03)   (2561,1.551e-03)  (7937,5.236e-04)
(23297,1.723e-04) (65537,5.545e-05) (178177,1.751e-05)
};

\addplot coordinates{
(11,6.887e-02)    (71,3.177e-02)     (351,1.341e-02)
(1471,5.334e-03)  (5503,2.027e-03)   (18943,7.415e-04)
(61183,2.628e-04) (187903,9.063e-05) (553983,3.053e-05)
};

\addplot coordinates{
(13,5.755e-02)     (97,2.925e-02)     (545,1.351e-02)
(2561,5.842e-03)   (10625,2.397e-03)  (40193,9.414e-04)
(141569,3.564e-04) (471041,1.308e-04)
(1496065,4.670e-05)
};
}%


\bookmaketitle
\tableofcontents
\include{pgfplots.title_abstract_intro}
\include{pgfplots.preliminaries}
\include{pgfplots.intro}
\include{pgfplots.reference}
\include{pgfplots.libs}
\include{pgfplots.resources}
\include{pgfplots.importexport}
\include{pgfplots.basic.reference}

\printindex

\bibliographystyle{abbrv} %gerapali} %gerabbrv} %gerunsrt.bst} %gerabbrv}% gerplain}
\nocite{pgfplotstable}
\nocite{programmingnotes}
\bibliography{pgfplots}
\end{document}

% -----------------------------------------------------------------------------
% For Stefan Pinnow as reminder on what to look for when editing the manual
% -----------------------------------------------------------------------------
% There should be no line breaks in the following environments
% - |...|
% - \declareandlabel{...}
% - \verbpdfref{...}
% If "MakeTikzPictures" isn't running through check one of the externalized
% LOG files what is the cause of that.
% -----------------------------------------------------------------------------


%%%%%%%%%%%%%%%%%%%%%%%%%%%%%%%%%%%%%%%%%%%%%%%%%%%%%%%%%%%%%%%%%%%%%%%%%%%%%
%
% Package pgfplots.sty documentation.
%
% Copyright 2007/2008 by Christian Feuersaenger.
%
% This program is free software: you can redistribute it and/or modify
% it under the terms of the GNU General Public License as published by
% the Free Software Foundation, either version 3 of the License, or
% (at your option) any later version.
%
% This program is distributed in the hope that it will be useful,
% but WITHOUT ANY WARRANTY; without even the implied warranty of
% MERCHANTABILITY or FITNESS FOR A PARTICULAR PURPOSE.  See the
% GNU General Public License for more details.
%
% You should have received a copy of the GNU General Public License
% along with this program.  If not, see <http://www.gnu.org/licenses/>.
%
%
%%%%%%%%%%%%%%%%%%%%%%%%%%%%%%%%%%%%%%%%%%%%%%%%%%%%%%%%%%%%%%%%%%%%%%%%%%%%%
% SEE pgfplots-macros.tex as well!
%\pdfminorversion=5 % to allow compression
%\pdfobjcompresslevel=2
\documentclass[a4paper,openany]{book}

\let\bookmaketitle=\maketitle

% -----------------------------------
% this here is from ltxdoc:
\usepackage{doc}[=v2]
\AtBeginDocument{\MakeShortVerb{\|}}
%\setlength{\textwidth}{355pt}
%\addtolength\marginparwidth{30pt}
%\addtolength\oddsidemargin{20pt}
%\addtolength\evensidemargin{20pt}
\def\cmd#1{\cs{\expandafter\cmd@to@cs\string#1}}
\def\cmd@to@cs#1#2{\char\number`#2\relax}
\DeclareRobustCommand\cs[1]{\texttt{\char`\\#1}}
\providecommand\marg[1]{%
  {\ttfamily\char`\{}\meta{#1}{\ttfamily\char`\}}}
\providecommand\oarg[1]{%
  {\ttfamily[}\meta{#1}{\ttfamily]}}
\providecommand\parg[1]{%
  {\ttfamily(}\meta{#1}{\ttfamily)}}
\raggedbottom

% -----------------------------------
\input{pgfplots.preamble.tex}

\makeatletter
% I want a two-column index just as in pgfmanual styles. This here
% was the best way to get one:
\def\index@prologue{\section*{Index}\addcontentsline{toc}{chapter}{Index}
}
\makeatother

%\RequirePackage[german,english,francais]{babel}

\def\matlabcolormaptext{This colormap is similar to one shipped with Matlab$^\text{\textregistered}$ under a similar name.}

\IfFileExists{tikzlibraryspy.code.tex}{%
\usetikzlibrary{spy}
}{%
    \message{ERROR: tikz SPY library NOT available. The manual will only compile partially.^^J}%
}%

\usepackage{xparse}% for colorbrewer manual

\usetikzlibrary{
    decorations.markings,
    decorations.footprints,
    shapes.arrows,
    matrix,
    positioning,
}

\usepgfplotslibrary{
    fillbetween,
    ternary,
    smithchart,
    patchplots,
    polar,
    colormaps,
    colorbrewer,
    colortol,           % docu not ready yet
}
\pgfqkeys{/codeexample}{%
    every codeexample/.append style={
        /pgfplots/every ternary axis/.append style={
            /pgfplots/legend style={fill=graphicbackground},
        }
    },
    tabsize=4,
}

\pgfplotsmanualenableexternalizationofexpensive

%\usetikzlibrary{external}
%\tikzexternalize[prefix=figures/]{pgfplots}

\title{%
    Manual for Package \PGFPlots{}\\
    {\small 2D/3D Plots in \LaTeX{}, Version \pgfplotsversion}\\
    {\small\url{https://github.com/pgf-tikz/pgfplots}}
    %\\{\small Attention: you are using an unstable development version.}
}

\makeatletter
\long\def\abstractsmuggle{%
    \centering
    \textbf{Abstract}\\[0.5cm]

    \begin{minipage}{12cm}
        \PGFPlots{} draws high-quality function plots in normal or logarithmic
        scaling with a user-friendly interface directly in \TeX{}. The user
        supplies axis labels, legend entries and the plot coordinates for one or
        more plots and \PGFPlots{} applies axis scaling, computes any logarithms
        and axis ticks and draws the plots. It supports line plots, scatter
        plots, piecewise constant plots, bar plots, area plots, mesh and surface
        plots, patch plots, contour plots, quiver plots, histogram plots, box
        plots, polar axes, ternary diagrams, smith charts and some more. It is
        based on Till Tantau's package \PGF{}/\Tikz{}.
    \end{minipage}

}%

\expandafter\date\expandafter{\@date\\[2cm]
    \abstractsmuggle
}%
\makeatother

%\includeonly{pgfplots.reference}


\begin{document}

\def\plotcoords{%
\addplot coordinates {
(5,8.312e-02)    (17,2.547e-02)   (49,7.407e-03)
(129,2.102e-03)  (321,5.874e-04)  (769,1.623e-04)
(1793,4.442e-05) (4097,1.207e-05) (9217,3.261e-06)
};

\addplot coordinates{
(7,8.472e-02)    (31,3.044e-02)    (111,1.022e-02)
(351,3.303e-03)  (1023,1.039e-03)  (2815,3.196e-04)
(7423,9.658e-05) (18943,2.873e-05) (47103,8.437e-06)
};

\addplot coordinates{
(9,7.881e-02)     (49,3.243e-02)    (209,1.232e-02)
(769,4.454e-03)   (2561,1.551e-03)  (7937,5.236e-04)
(23297,1.723e-04) (65537,5.545e-05) (178177,1.751e-05)
};

\addplot coordinates{
(11,6.887e-02)    (71,3.177e-02)     (351,1.341e-02)
(1471,5.334e-03)  (5503,2.027e-03)   (18943,7.415e-04)
(61183,2.628e-04) (187903,9.063e-05) (553983,3.053e-05)
};

\addplot coordinates{
(13,5.755e-02)     (97,2.925e-02)     (545,1.351e-02)
(2561,5.842e-03)   (10625,2.397e-03)  (40193,9.414e-04)
(141569,3.564e-04) (471041,1.308e-04)
(1496065,4.670e-05)
};
}%


\bookmaketitle
\tableofcontents
\include{pgfplots.title_abstract_intro}
\include{pgfplots.preliminaries}
\include{pgfplots.intro}
\include{pgfplots.reference}
\include{pgfplots.libs}
\include{pgfplots.resources}
\include{pgfplots.importexport}
\include{pgfplots.basic.reference}

\printindex

\bibliographystyle{abbrv} %gerapali} %gerabbrv} %gerunsrt.bst} %gerabbrv}% gerplain}
\nocite{pgfplotstable}
\nocite{programmingnotes}
\bibliography{pgfplots}
\end{document}

% -----------------------------------------------------------------------------
% For Stefan Pinnow as reminder on what to look for when editing the manual
% -----------------------------------------------------------------------------
% There should be no line breaks in the following environments
% - |...|
% - \declareandlabel{...}
% - \verbpdfref{...}
% If "MakeTikzPictures" isn't running through check one of the externalized
% LOG files what is the cause of that.
% -----------------------------------------------------------------------------


%%%%%%%%%%%%%%%%%%%%%%%%%%%%%%%%%%%%%%%%%%%%%%%%%%%%%%%%%%%%%%%%%%%%%%%%%%%%%
%
% Package pgfplots.sty documentation.
%
% Copyright 2007/2008 by Christian Feuersaenger.
%
% This program is free software: you can redistribute it and/or modify
% it under the terms of the GNU General Public License as published by
% the Free Software Foundation, either version 3 of the License, or
% (at your option) any later version.
%
% This program is distributed in the hope that it will be useful,
% but WITHOUT ANY WARRANTY; without even the implied warranty of
% MERCHANTABILITY or FITNESS FOR A PARTICULAR PURPOSE.  See the
% GNU General Public License for more details.
%
% You should have received a copy of the GNU General Public License
% along with this program.  If not, see <http://www.gnu.org/licenses/>.
%
%
%%%%%%%%%%%%%%%%%%%%%%%%%%%%%%%%%%%%%%%%%%%%%%%%%%%%%%%%%%%%%%%%%%%%%%%%%%%%%
% SEE pgfplots-macros.tex as well!
%\pdfminorversion=5 % to allow compression
%\pdfobjcompresslevel=2
\documentclass[a4paper,openany]{book}

\let\bookmaketitle=\maketitle

% -----------------------------------
% this here is from ltxdoc:
\usepackage{doc}[=v2]
\AtBeginDocument{\MakeShortVerb{\|}}
%\setlength{\textwidth}{355pt}
%\addtolength\marginparwidth{30pt}
%\addtolength\oddsidemargin{20pt}
%\addtolength\evensidemargin{20pt}
\def\cmd#1{\cs{\expandafter\cmd@to@cs\string#1}}
\def\cmd@to@cs#1#2{\char\number`#2\relax}
\DeclareRobustCommand\cs[1]{\texttt{\char`\\#1}}
\providecommand\marg[1]{%
  {\ttfamily\char`\{}\meta{#1}{\ttfamily\char`\}}}
\providecommand\oarg[1]{%
  {\ttfamily[}\meta{#1}{\ttfamily]}}
\providecommand\parg[1]{%
  {\ttfamily(}\meta{#1}{\ttfamily)}}
\raggedbottom

% -----------------------------------
\input{pgfplots.preamble.tex}

\makeatletter
% I want a two-column index just as in pgfmanual styles. This here
% was the best way to get one:
\def\index@prologue{\section*{Index}\addcontentsline{toc}{chapter}{Index}
}
\makeatother

%\RequirePackage[german,english,francais]{babel}

\def\matlabcolormaptext{This colormap is similar to one shipped with Matlab$^\text{\textregistered}$ under a similar name.}

\IfFileExists{tikzlibraryspy.code.tex}{%
\usetikzlibrary{spy}
}{%
    \message{ERROR: tikz SPY library NOT available. The manual will only compile partially.^^J}%
}%

\usepackage{xparse}% for colorbrewer manual

\usetikzlibrary{
    decorations.markings,
    decorations.footprints,
    shapes.arrows,
    matrix,
    positioning,
}

\usepgfplotslibrary{
    fillbetween,
    ternary,
    smithchart,
    patchplots,
    polar,
    colormaps,
    colorbrewer,
    colortol,           % docu not ready yet
}
\pgfqkeys{/codeexample}{%
    every codeexample/.append style={
        /pgfplots/every ternary axis/.append style={
            /pgfplots/legend style={fill=graphicbackground},
        }
    },
    tabsize=4,
}

\pgfplotsmanualenableexternalizationofexpensive

%\usetikzlibrary{external}
%\tikzexternalize[prefix=figures/]{pgfplots}

\title{%
    Manual for Package \PGFPlots{}\\
    {\small 2D/3D Plots in \LaTeX{}, Version \pgfplotsversion}\\
    {\small\url{https://github.com/pgf-tikz/pgfplots}}
    %\\{\small Attention: you are using an unstable development version.}
}

\makeatletter
\long\def\abstractsmuggle{%
    \centering
    \textbf{Abstract}\\[0.5cm]

    \begin{minipage}{12cm}
        \PGFPlots{} draws high-quality function plots in normal or logarithmic
        scaling with a user-friendly interface directly in \TeX{}. The user
        supplies axis labels, legend entries and the plot coordinates for one or
        more plots and \PGFPlots{} applies axis scaling, computes any logarithms
        and axis ticks and draws the plots. It supports line plots, scatter
        plots, piecewise constant plots, bar plots, area plots, mesh and surface
        plots, patch plots, contour plots, quiver plots, histogram plots, box
        plots, polar axes, ternary diagrams, smith charts and some more. It is
        based on Till Tantau's package \PGF{}/\Tikz{}.
    \end{minipage}

}%

\expandafter\date\expandafter{\@date\\[2cm]
    \abstractsmuggle
}%
\makeatother

%\includeonly{pgfplots.reference}


\begin{document}

\def\plotcoords{%
\addplot coordinates {
(5,8.312e-02)    (17,2.547e-02)   (49,7.407e-03)
(129,2.102e-03)  (321,5.874e-04)  (769,1.623e-04)
(1793,4.442e-05) (4097,1.207e-05) (9217,3.261e-06)
};

\addplot coordinates{
(7,8.472e-02)    (31,3.044e-02)    (111,1.022e-02)
(351,3.303e-03)  (1023,1.039e-03)  (2815,3.196e-04)
(7423,9.658e-05) (18943,2.873e-05) (47103,8.437e-06)
};

\addplot coordinates{
(9,7.881e-02)     (49,3.243e-02)    (209,1.232e-02)
(769,4.454e-03)   (2561,1.551e-03)  (7937,5.236e-04)
(23297,1.723e-04) (65537,5.545e-05) (178177,1.751e-05)
};

\addplot coordinates{
(11,6.887e-02)    (71,3.177e-02)     (351,1.341e-02)
(1471,5.334e-03)  (5503,2.027e-03)   (18943,7.415e-04)
(61183,2.628e-04) (187903,9.063e-05) (553983,3.053e-05)
};

\addplot coordinates{
(13,5.755e-02)     (97,2.925e-02)     (545,1.351e-02)
(2561,5.842e-03)   (10625,2.397e-03)  (40193,9.414e-04)
(141569,3.564e-04) (471041,1.308e-04)
(1496065,4.670e-05)
};
}%


\bookmaketitle
\tableofcontents
\include{pgfplots.title_abstract_intro}
\include{pgfplots.preliminaries}
\include{pgfplots.intro}
\include{pgfplots.reference}
\include{pgfplots.libs}
\include{pgfplots.resources}
\include{pgfplots.importexport}
\include{pgfplots.basic.reference}

\printindex

\bibliographystyle{abbrv} %gerapali} %gerabbrv} %gerunsrt.bst} %gerabbrv}% gerplain}
\nocite{pgfplotstable}
\nocite{programmingnotes}
\bibliography{pgfplots}
\end{document}

% -----------------------------------------------------------------------------
% For Stefan Pinnow as reminder on what to look for when editing the manual
% -----------------------------------------------------------------------------
% There should be no line breaks in the following environments
% - |...|
% - \declareandlabel{...}
% - \verbpdfref{...}
% If "MakeTikzPictures" isn't running through check one of the externalized
% LOG files what is the cause of that.
% -----------------------------------------------------------------------------


%%%%%%%%%%%%%%%%%%%%%%%%%%%%%%%%%%%%%%%%%%%%%%%%%%%%%%%%%%%%%%%%%%%%%%%%%%%%%
%
% Package pgfplots.sty documentation.
%
% Copyright 2007/2008 by Christian Feuersaenger.
%
% This program is free software: you can redistribute it and/or modify
% it under the terms of the GNU General Public License as published by
% the Free Software Foundation, either version 3 of the License, or
% (at your option) any later version.
%
% This program is distributed in the hope that it will be useful,
% but WITHOUT ANY WARRANTY; without even the implied warranty of
% MERCHANTABILITY or FITNESS FOR A PARTICULAR PURPOSE.  See the
% GNU General Public License for more details.
%
% You should have received a copy of the GNU General Public License
% along with this program.  If not, see <http://www.gnu.org/licenses/>.
%
%
%%%%%%%%%%%%%%%%%%%%%%%%%%%%%%%%%%%%%%%%%%%%%%%%%%%%%%%%%%%%%%%%%%%%%%%%%%%%%
% SEE pgfplots-macros.tex as well!
%\pdfminorversion=5 % to allow compression
%\pdfobjcompresslevel=2
\documentclass[a4paper,openany]{book}

\let\bookmaketitle=\maketitle

% -----------------------------------
% this here is from ltxdoc:
\usepackage{doc}[=v2]
\AtBeginDocument{\MakeShortVerb{\|}}
%\setlength{\textwidth}{355pt}
%\addtolength\marginparwidth{30pt}
%\addtolength\oddsidemargin{20pt}
%\addtolength\evensidemargin{20pt}
\def\cmd#1{\cs{\expandafter\cmd@to@cs\string#1}}
\def\cmd@to@cs#1#2{\char\number`#2\relax}
\DeclareRobustCommand\cs[1]{\texttt{\char`\\#1}}
\providecommand\marg[1]{%
  {\ttfamily\char`\{}\meta{#1}{\ttfamily\char`\}}}
\providecommand\oarg[1]{%
  {\ttfamily[}\meta{#1}{\ttfamily]}}
\providecommand\parg[1]{%
  {\ttfamily(}\meta{#1}{\ttfamily)}}
\raggedbottom

% -----------------------------------
\input{pgfplots.preamble.tex}

\makeatletter
% I want a two-column index just as in pgfmanual styles. This here
% was the best way to get one:
\def\index@prologue{\section*{Index}\addcontentsline{toc}{chapter}{Index}
}
\makeatother

%\RequirePackage[german,english,francais]{babel}

\def\matlabcolormaptext{This colormap is similar to one shipped with Matlab$^\text{\textregistered}$ under a similar name.}

\IfFileExists{tikzlibraryspy.code.tex}{%
\usetikzlibrary{spy}
}{%
    \message{ERROR: tikz SPY library NOT available. The manual will only compile partially.^^J}%
}%

\usepackage{xparse}% for colorbrewer manual

\usetikzlibrary{
    decorations.markings,
    decorations.footprints,
    shapes.arrows,
    matrix,
    positioning,
}

\usepgfplotslibrary{
    fillbetween,
    ternary,
    smithchart,
    patchplots,
    polar,
    colormaps,
    colorbrewer,
    colortol,           % docu not ready yet
}
\pgfqkeys{/codeexample}{%
    every codeexample/.append style={
        /pgfplots/every ternary axis/.append style={
            /pgfplots/legend style={fill=graphicbackground},
        }
    },
    tabsize=4,
}

\pgfplotsmanualenableexternalizationofexpensive

%\usetikzlibrary{external}
%\tikzexternalize[prefix=figures/]{pgfplots}

\title{%
    Manual for Package \PGFPlots{}\\
    {\small 2D/3D Plots in \LaTeX{}, Version \pgfplotsversion}\\
    {\small\url{https://github.com/pgf-tikz/pgfplots}}
    %\\{\small Attention: you are using an unstable development version.}
}

\makeatletter
\long\def\abstractsmuggle{%
    \centering
    \textbf{Abstract}\\[0.5cm]

    \begin{minipage}{12cm}
        \PGFPlots{} draws high-quality function plots in normal or logarithmic
        scaling with a user-friendly interface directly in \TeX{}. The user
        supplies axis labels, legend entries and the plot coordinates for one or
        more plots and \PGFPlots{} applies axis scaling, computes any logarithms
        and axis ticks and draws the plots. It supports line plots, scatter
        plots, piecewise constant plots, bar plots, area plots, mesh and surface
        plots, patch plots, contour plots, quiver plots, histogram plots, box
        plots, polar axes, ternary diagrams, smith charts and some more. It is
        based on Till Tantau's package \PGF{}/\Tikz{}.
    \end{minipage}

}%

\expandafter\date\expandafter{\@date\\[2cm]
    \abstractsmuggle
}%
\makeatother

%\includeonly{pgfplots.reference}


\begin{document}

\def\plotcoords{%
\addplot coordinates {
(5,8.312e-02)    (17,2.547e-02)   (49,7.407e-03)
(129,2.102e-03)  (321,5.874e-04)  (769,1.623e-04)
(1793,4.442e-05) (4097,1.207e-05) (9217,3.261e-06)
};

\addplot coordinates{
(7,8.472e-02)    (31,3.044e-02)    (111,1.022e-02)
(351,3.303e-03)  (1023,1.039e-03)  (2815,3.196e-04)
(7423,9.658e-05) (18943,2.873e-05) (47103,8.437e-06)
};

\addplot coordinates{
(9,7.881e-02)     (49,3.243e-02)    (209,1.232e-02)
(769,4.454e-03)   (2561,1.551e-03)  (7937,5.236e-04)
(23297,1.723e-04) (65537,5.545e-05) (178177,1.751e-05)
};

\addplot coordinates{
(11,6.887e-02)    (71,3.177e-02)     (351,1.341e-02)
(1471,5.334e-03)  (5503,2.027e-03)   (18943,7.415e-04)
(61183,2.628e-04) (187903,9.063e-05) (553983,3.053e-05)
};

\addplot coordinates{
(13,5.755e-02)     (97,2.925e-02)     (545,1.351e-02)
(2561,5.842e-03)   (10625,2.397e-03)  (40193,9.414e-04)
(141569,3.564e-04) (471041,1.308e-04)
(1496065,4.670e-05)
};
}%


\bookmaketitle
\tableofcontents
\include{pgfplots.title_abstract_intro}
\include{pgfplots.preliminaries}
\include{pgfplots.intro}
\include{pgfplots.reference}
\include{pgfplots.libs}
\include{pgfplots.resources}
\include{pgfplots.importexport}
\include{pgfplots.basic.reference}

\printindex

\bibliographystyle{abbrv} %gerapali} %gerabbrv} %gerunsrt.bst} %gerabbrv}% gerplain}
\nocite{pgfplotstable}
\nocite{programmingnotes}
\bibliography{pgfplots}
\end{document}

% -----------------------------------------------------------------------------
% For Stefan Pinnow as reminder on what to look for when editing the manual
% -----------------------------------------------------------------------------
% There should be no line breaks in the following environments
% - |...|
% - \declareandlabel{...}
% - \verbpdfref{...}
% If "MakeTikzPictures" isn't running through check one of the externalized
% LOG files what is the cause of that.
% -----------------------------------------------------------------------------


%%%%%%%%%%%%%%%%%%%%%%%%%%%%%%%%%%%%%%%%%%%%%%%%%%%%%%%%%%%%%%%%%%%%%%%%%%%%%
%
% Package pgfplots.sty documentation.
%
% Copyright 2007/2008 by Christian Feuersaenger.
%
% This program is free software: you can redistribute it and/or modify
% it under the terms of the GNU General Public License as published by
% the Free Software Foundation, either version 3 of the License, or
% (at your option) any later version.
%
% This program is distributed in the hope that it will be useful,
% but WITHOUT ANY WARRANTY; without even the implied warranty of
% MERCHANTABILITY or FITNESS FOR A PARTICULAR PURPOSE.  See the
% GNU General Public License for more details.
%
% You should have received a copy of the GNU General Public License
% along with this program.  If not, see <http://www.gnu.org/licenses/>.
%
%
%%%%%%%%%%%%%%%%%%%%%%%%%%%%%%%%%%%%%%%%%%%%%%%%%%%%%%%%%%%%%%%%%%%%%%%%%%%%%
% SEE pgfplots-macros.tex as well!
%\pdfminorversion=5 % to allow compression
%\pdfobjcompresslevel=2
\documentclass[a4paper,openany]{book}

\let\bookmaketitle=\maketitle

% -----------------------------------
% this here is from ltxdoc:
\usepackage{doc}[=v2]
\AtBeginDocument{\MakeShortVerb{\|}}
%\setlength{\textwidth}{355pt}
%\addtolength\marginparwidth{30pt}
%\addtolength\oddsidemargin{20pt}
%\addtolength\evensidemargin{20pt}
\def\cmd#1{\cs{\expandafter\cmd@to@cs\string#1}}
\def\cmd@to@cs#1#2{\char\number`#2\relax}
\DeclareRobustCommand\cs[1]{\texttt{\char`\\#1}}
\providecommand\marg[1]{%
  {\ttfamily\char`\{}\meta{#1}{\ttfamily\char`\}}}
\providecommand\oarg[1]{%
  {\ttfamily[}\meta{#1}{\ttfamily]}}
\providecommand\parg[1]{%
  {\ttfamily(}\meta{#1}{\ttfamily)}}
\raggedbottom

% -----------------------------------
\input{pgfplots.preamble.tex}

\makeatletter
% I want a two-column index just as in pgfmanual styles. This here
% was the best way to get one:
\def\index@prologue{\section*{Index}\addcontentsline{toc}{chapter}{Index}
}
\makeatother

%\RequirePackage[german,english,francais]{babel}

\def\matlabcolormaptext{This colormap is similar to one shipped with Matlab$^\text{\textregistered}$ under a similar name.}

\IfFileExists{tikzlibraryspy.code.tex}{%
\usetikzlibrary{spy}
}{%
    \message{ERROR: tikz SPY library NOT available. The manual will only compile partially.^^J}%
}%

\usepackage{xparse}% for colorbrewer manual

\usetikzlibrary{
    decorations.markings,
    decorations.footprints,
    shapes.arrows,
    matrix,
    positioning,
}

\usepgfplotslibrary{
    fillbetween,
    ternary,
    smithchart,
    patchplots,
    polar,
    colormaps,
    colorbrewer,
    colortol,           % docu not ready yet
}
\pgfqkeys{/codeexample}{%
    every codeexample/.append style={
        /pgfplots/every ternary axis/.append style={
            /pgfplots/legend style={fill=graphicbackground},
        }
    },
    tabsize=4,
}

\pgfplotsmanualenableexternalizationofexpensive

%\usetikzlibrary{external}
%\tikzexternalize[prefix=figures/]{pgfplots}

\title{%
    Manual for Package \PGFPlots{}\\
    {\small 2D/3D Plots in \LaTeX{}, Version \pgfplotsversion}\\
    {\small\url{https://github.com/pgf-tikz/pgfplots}}
    %\\{\small Attention: you are using an unstable development version.}
}

\makeatletter
\long\def\abstractsmuggle{%
    \centering
    \textbf{Abstract}\\[0.5cm]

    \begin{minipage}{12cm}
        \PGFPlots{} draws high-quality function plots in normal or logarithmic
        scaling with a user-friendly interface directly in \TeX{}. The user
        supplies axis labels, legend entries and the plot coordinates for one or
        more plots and \PGFPlots{} applies axis scaling, computes any logarithms
        and axis ticks and draws the plots. It supports line plots, scatter
        plots, piecewise constant plots, bar plots, area plots, mesh and surface
        plots, patch plots, contour plots, quiver plots, histogram plots, box
        plots, polar axes, ternary diagrams, smith charts and some more. It is
        based on Till Tantau's package \PGF{}/\Tikz{}.
    \end{minipage}

}%

\expandafter\date\expandafter{\@date\\[2cm]
    \abstractsmuggle
}%
\makeatother

%\includeonly{pgfplots.reference}


\begin{document}

\def\plotcoords{%
\addplot coordinates {
(5,8.312e-02)    (17,2.547e-02)   (49,7.407e-03)
(129,2.102e-03)  (321,5.874e-04)  (769,1.623e-04)
(1793,4.442e-05) (4097,1.207e-05) (9217,3.261e-06)
};

\addplot coordinates{
(7,8.472e-02)    (31,3.044e-02)    (111,1.022e-02)
(351,3.303e-03)  (1023,1.039e-03)  (2815,3.196e-04)
(7423,9.658e-05) (18943,2.873e-05) (47103,8.437e-06)
};

\addplot coordinates{
(9,7.881e-02)     (49,3.243e-02)    (209,1.232e-02)
(769,4.454e-03)   (2561,1.551e-03)  (7937,5.236e-04)
(23297,1.723e-04) (65537,5.545e-05) (178177,1.751e-05)
};

\addplot coordinates{
(11,6.887e-02)    (71,3.177e-02)     (351,1.341e-02)
(1471,5.334e-03)  (5503,2.027e-03)   (18943,7.415e-04)
(61183,2.628e-04) (187903,9.063e-05) (553983,3.053e-05)
};

\addplot coordinates{
(13,5.755e-02)     (97,2.925e-02)     (545,1.351e-02)
(2561,5.842e-03)   (10625,2.397e-03)  (40193,9.414e-04)
(141569,3.564e-04) (471041,1.308e-04)
(1496065,4.670e-05)
};
}%


\bookmaketitle
\tableofcontents
\include{pgfplots.title_abstract_intro}
\include{pgfplots.preliminaries}
\include{pgfplots.intro}
\include{pgfplots.reference}
\include{pgfplots.libs}
\include{pgfplots.resources}
\include{pgfplots.importexport}
\include{pgfplots.basic.reference}

\printindex

\bibliographystyle{abbrv} %gerapali} %gerabbrv} %gerunsrt.bst} %gerabbrv}% gerplain}
\nocite{pgfplotstable}
\nocite{programmingnotes}
\bibliography{pgfplots}
\end{document}

% -----------------------------------------------------------------------------
% For Stefan Pinnow as reminder on what to look for when editing the manual
% -----------------------------------------------------------------------------
% There should be no line breaks in the following environments
% - |...|
% - \declareandlabel{...}
% - \verbpdfref{...}
% If "MakeTikzPictures" isn't running through check one of the externalized
% LOG files what is the cause of that.
% -----------------------------------------------------------------------------


%%%%%%%%%%%%%%%%%%%%%%%%%%%%%%%%%%%%%%%%%%%%%%%%%%%%%%%%%%%%%%%%%%%%%%%%%%%%%
%
% Package pgfplots.sty documentation.
%
% Copyright 2007/2008 by Christian Feuersaenger.
%
% This program is free software: you can redistribute it and/or modify
% it under the terms of the GNU General Public License as published by
% the Free Software Foundation, either version 3 of the License, or
% (at your option) any later version.
%
% This program is distributed in the hope that it will be useful,
% but WITHOUT ANY WARRANTY; without even the implied warranty of
% MERCHANTABILITY or FITNESS FOR A PARTICULAR PURPOSE.  See the
% GNU General Public License for more details.
%
% You should have received a copy of the GNU General Public License
% along with this program.  If not, see <http://www.gnu.org/licenses/>.
%
%
%%%%%%%%%%%%%%%%%%%%%%%%%%%%%%%%%%%%%%%%%%%%%%%%%%%%%%%%%%%%%%%%%%%%%%%%%%%%%
% SEE pgfplots-macros.tex as well!
%\pdfminorversion=5 % to allow compression
%\pdfobjcompresslevel=2
\documentclass[a4paper,openany]{book}

\let\bookmaketitle=\maketitle

% -----------------------------------
% this here is from ltxdoc:
\usepackage{doc}[=v2]
\AtBeginDocument{\MakeShortVerb{\|}}
%\setlength{\textwidth}{355pt}
%\addtolength\marginparwidth{30pt}
%\addtolength\oddsidemargin{20pt}
%\addtolength\evensidemargin{20pt}
\def\cmd#1{\cs{\expandafter\cmd@to@cs\string#1}}
\def\cmd@to@cs#1#2{\char\number`#2\relax}
\DeclareRobustCommand\cs[1]{\texttt{\char`\\#1}}
\providecommand\marg[1]{%
  {\ttfamily\char`\{}\meta{#1}{\ttfamily\char`\}}}
\providecommand\oarg[1]{%
  {\ttfamily[}\meta{#1}{\ttfamily]}}
\providecommand\parg[1]{%
  {\ttfamily(}\meta{#1}{\ttfamily)}}
\raggedbottom

% -----------------------------------
\input{pgfplots.preamble.tex}

\makeatletter
% I want a two-column index just as in pgfmanual styles. This here
% was the best way to get one:
\def\index@prologue{\section*{Index}\addcontentsline{toc}{chapter}{Index}
}
\makeatother

%\RequirePackage[german,english,francais]{babel}

\def\matlabcolormaptext{This colormap is similar to one shipped with Matlab$^\text{\textregistered}$ under a similar name.}

\IfFileExists{tikzlibraryspy.code.tex}{%
\usetikzlibrary{spy}
}{%
    \message{ERROR: tikz SPY library NOT available. The manual will only compile partially.^^J}%
}%

\usepackage{xparse}% for colorbrewer manual

\usetikzlibrary{
    decorations.markings,
    decorations.footprints,
    shapes.arrows,
    matrix,
    positioning,
}

\usepgfplotslibrary{
    fillbetween,
    ternary,
    smithchart,
    patchplots,
    polar,
    colormaps,
    colorbrewer,
    colortol,           % docu not ready yet
}
\pgfqkeys{/codeexample}{%
    every codeexample/.append style={
        /pgfplots/every ternary axis/.append style={
            /pgfplots/legend style={fill=graphicbackground},
        }
    },
    tabsize=4,
}

\pgfplotsmanualenableexternalizationofexpensive

%\usetikzlibrary{external}
%\tikzexternalize[prefix=figures/]{pgfplots}

\title{%
    Manual for Package \PGFPlots{}\\
    {\small 2D/3D Plots in \LaTeX{}, Version \pgfplotsversion}\\
    {\small\url{https://github.com/pgf-tikz/pgfplots}}
    %\\{\small Attention: you are using an unstable development version.}
}

\makeatletter
\long\def\abstractsmuggle{%
    \centering
    \textbf{Abstract}\\[0.5cm]

    \begin{minipage}{12cm}
        \PGFPlots{} draws high-quality function plots in normal or logarithmic
        scaling with a user-friendly interface directly in \TeX{}. The user
        supplies axis labels, legend entries and the plot coordinates for one or
        more plots and \PGFPlots{} applies axis scaling, computes any logarithms
        and axis ticks and draws the plots. It supports line plots, scatter
        plots, piecewise constant plots, bar plots, area plots, mesh and surface
        plots, patch plots, contour plots, quiver plots, histogram plots, box
        plots, polar axes, ternary diagrams, smith charts and some more. It is
        based on Till Tantau's package \PGF{}/\Tikz{}.
    \end{minipage}

}%

\expandafter\date\expandafter{\@date\\[2cm]
    \abstractsmuggle
}%
\makeatother

%\includeonly{pgfplots.reference}


\begin{document}

\def\plotcoords{%
\addplot coordinates {
(5,8.312e-02)    (17,2.547e-02)   (49,7.407e-03)
(129,2.102e-03)  (321,5.874e-04)  (769,1.623e-04)
(1793,4.442e-05) (4097,1.207e-05) (9217,3.261e-06)
};

\addplot coordinates{
(7,8.472e-02)    (31,3.044e-02)    (111,1.022e-02)
(351,3.303e-03)  (1023,1.039e-03)  (2815,3.196e-04)
(7423,9.658e-05) (18943,2.873e-05) (47103,8.437e-06)
};

\addplot coordinates{
(9,7.881e-02)     (49,3.243e-02)    (209,1.232e-02)
(769,4.454e-03)   (2561,1.551e-03)  (7937,5.236e-04)
(23297,1.723e-04) (65537,5.545e-05) (178177,1.751e-05)
};

\addplot coordinates{
(11,6.887e-02)    (71,3.177e-02)     (351,1.341e-02)
(1471,5.334e-03)  (5503,2.027e-03)   (18943,7.415e-04)
(61183,2.628e-04) (187903,9.063e-05) (553983,3.053e-05)
};

\addplot coordinates{
(13,5.755e-02)     (97,2.925e-02)     (545,1.351e-02)
(2561,5.842e-03)   (10625,2.397e-03)  (40193,9.414e-04)
(141569,3.564e-04) (471041,1.308e-04)
(1496065,4.670e-05)
};
}%


\bookmaketitle
\tableofcontents
\include{pgfplots.title_abstract_intro}
\include{pgfplots.preliminaries}
\include{pgfplots.intro}
\include{pgfplots.reference}
\include{pgfplots.libs}
\include{pgfplots.resources}
\include{pgfplots.importexport}
\include{pgfplots.basic.reference}

\printindex

\bibliographystyle{abbrv} %gerapali} %gerabbrv} %gerunsrt.bst} %gerabbrv}% gerplain}
\nocite{pgfplotstable}
\nocite{programmingnotes}
\bibliography{pgfplots}
\end{document}

% -----------------------------------------------------------------------------
% For Stefan Pinnow as reminder on what to look for when editing the manual
% -----------------------------------------------------------------------------
% There should be no line breaks in the following environments
% - |...|
% - \declareandlabel{...}
% - \verbpdfref{...}
% If "MakeTikzPictures" isn't running through check one of the externalized
% LOG files what is the cause of that.
% -----------------------------------------------------------------------------

\include{pgfplots.basic.reference}

\printindex

\bibliographystyle{abbrv} %gerapali} %gerabbrv} %gerunsrt.bst} %gerabbrv}% gerplain}
\nocite{pgfplotstable}
\nocite{programmingnotes}
\bibliography{pgfplots}
\end{document}

% -----------------------------------------------------------------------------
% For Stefan Pinnow as reminder on what to look for when editing the manual
% -----------------------------------------------------------------------------
% There should be no line breaks in the following environments
% - |...|
% - \declareandlabel{...}
% - \verbpdfref{...}
% If "MakeTikzPictures" isn't running through check one of the externalized
% LOG files what is the cause of that.
% -----------------------------------------------------------------------------


%%%%%%%%%%%%%%%%%%%%%%%%%%%%%%%%%%%%%%%%%%%%%%%%%%%%%%%%%%%%%%%%%%%%%%%%%%%%%
%
% Package pgfplots.sty documentation.
%
% Copyright 2007/2008 by Christian Feuersaenger.
%
% This program is free software: you can redistribute it and/or modify
% it under the terms of the GNU General Public License as published by
% the Free Software Foundation, either version 3 of the License, or
% (at your option) any later version.
%
% This program is distributed in the hope that it will be useful,
% but WITHOUT ANY WARRANTY; without even the implied warranty of
% MERCHANTABILITY or FITNESS FOR A PARTICULAR PURPOSE.  See the
% GNU General Public License for more details.
%
% You should have received a copy of the GNU General Public License
% along with this program.  If not, see <http://www.gnu.org/licenses/>.
%
%
%%%%%%%%%%%%%%%%%%%%%%%%%%%%%%%%%%%%%%%%%%%%%%%%%%%%%%%%%%%%%%%%%%%%%%%%%%%%%
% SEE pgfplots-macros.tex as well!
%\pdfminorversion=5 % to allow compression
%\pdfobjcompresslevel=2
\documentclass[a4paper,openany]{book}

\let\bookmaketitle=\maketitle

% -----------------------------------
% this here is from ltxdoc:
\usepackage{doc}[=v2]
\AtBeginDocument{\MakeShortVerb{\|}}
%\setlength{\textwidth}{355pt}
%\addtolength\marginparwidth{30pt}
%\addtolength\oddsidemargin{20pt}
%\addtolength\evensidemargin{20pt}
\def\cmd#1{\cs{\expandafter\cmd@to@cs\string#1}}
\def\cmd@to@cs#1#2{\char\number`#2\relax}
\DeclareRobustCommand\cs[1]{\texttt{\char`\\#1}}
\providecommand\marg[1]{%
  {\ttfamily\char`\{}\meta{#1}{\ttfamily\char`\}}}
\providecommand\oarg[1]{%
  {\ttfamily[}\meta{#1}{\ttfamily]}}
\providecommand\parg[1]{%
  {\ttfamily(}\meta{#1}{\ttfamily)}}
\raggedbottom

% -----------------------------------
\input{pgfplots.preamble.tex}

\makeatletter
% I want a two-column index just as in pgfmanual styles. This here
% was the best way to get one:
\def\index@prologue{\section*{Index}\addcontentsline{toc}{chapter}{Index}
}
\makeatother

%\RequirePackage[german,english,francais]{babel}

\def\matlabcolormaptext{This colormap is similar to one shipped with Matlab$^\text{\textregistered}$ under a similar name.}

\IfFileExists{tikzlibraryspy.code.tex}{%
\usetikzlibrary{spy}
}{%
    \message{ERROR: tikz SPY library NOT available. The manual will only compile partially.^^J}%
}%

\usepackage{xparse}% for colorbrewer manual

\usetikzlibrary{
    decorations.markings,
    decorations.footprints,
    shapes.arrows,
    matrix,
    positioning,
}

\usepgfplotslibrary{
    fillbetween,
    ternary,
    smithchart,
    patchplots,
    polar,
    colormaps,
    colorbrewer,
    colortol,           % docu not ready yet
}
\pgfqkeys{/codeexample}{%
    every codeexample/.append style={
        /pgfplots/every ternary axis/.append style={
            /pgfplots/legend style={fill=graphicbackground},
        }
    },
    tabsize=4,
}

\pgfplotsmanualenableexternalizationofexpensive

%\usetikzlibrary{external}
%\tikzexternalize[prefix=figures/]{pgfplots}

\title{%
    Manual for Package \PGFPlots{}\\
    {\small 2D/3D Plots in \LaTeX{}, Version \pgfplotsversion}\\
    {\small\url{https://github.com/pgf-tikz/pgfplots}}
    %\\{\small Attention: you are using an unstable development version.}
}

\makeatletter
\long\def\abstractsmuggle{%
    \centering
    \textbf{Abstract}\\[0.5cm]

    \begin{minipage}{12cm}
        \PGFPlots{} draws high-quality function plots in normal or logarithmic
        scaling with a user-friendly interface directly in \TeX{}. The user
        supplies axis labels, legend entries and the plot coordinates for one or
        more plots and \PGFPlots{} applies axis scaling, computes any logarithms
        and axis ticks and draws the plots. It supports line plots, scatter
        plots, piecewise constant plots, bar plots, area plots, mesh and surface
        plots, patch plots, contour plots, quiver plots, histogram plots, box
        plots, polar axes, ternary diagrams, smith charts and some more. It is
        based on Till Tantau's package \PGF{}/\Tikz{}.
    \end{minipage}

}%

\expandafter\date\expandafter{\@date\\[2cm]
    \abstractsmuggle
}%
\makeatother

%\includeonly{pgfplots.reference}


\begin{document}

\def\plotcoords{%
\addplot coordinates {
(5,8.312e-02)    (17,2.547e-02)   (49,7.407e-03)
(129,2.102e-03)  (321,5.874e-04)  (769,1.623e-04)
(1793,4.442e-05) (4097,1.207e-05) (9217,3.261e-06)
};

\addplot coordinates{
(7,8.472e-02)    (31,3.044e-02)    (111,1.022e-02)
(351,3.303e-03)  (1023,1.039e-03)  (2815,3.196e-04)
(7423,9.658e-05) (18943,2.873e-05) (47103,8.437e-06)
};

\addplot coordinates{
(9,7.881e-02)     (49,3.243e-02)    (209,1.232e-02)
(769,4.454e-03)   (2561,1.551e-03)  (7937,5.236e-04)
(23297,1.723e-04) (65537,5.545e-05) (178177,1.751e-05)
};

\addplot coordinates{
(11,6.887e-02)    (71,3.177e-02)     (351,1.341e-02)
(1471,5.334e-03)  (5503,2.027e-03)   (18943,7.415e-04)
(61183,2.628e-04) (187903,9.063e-05) (553983,3.053e-05)
};

\addplot coordinates{
(13,5.755e-02)     (97,2.925e-02)     (545,1.351e-02)
(2561,5.842e-03)   (10625,2.397e-03)  (40193,9.414e-04)
(141569,3.564e-04) (471041,1.308e-04)
(1496065,4.670e-05)
};
}%


\bookmaketitle
\tableofcontents

%%%%%%%%%%%%%%%%%%%%%%%%%%%%%%%%%%%%%%%%%%%%%%%%%%%%%%%%%%%%%%%%%%%%%%%%%%%%%
%
% Package pgfplots.sty documentation.
%
% Copyright 2007/2008 by Christian Feuersaenger.
%
% This program is free software: you can redistribute it and/or modify
% it under the terms of the GNU General Public License as published by
% the Free Software Foundation, either version 3 of the License, or
% (at your option) any later version.
%
% This program is distributed in the hope that it will be useful,
% but WITHOUT ANY WARRANTY; without even the implied warranty of
% MERCHANTABILITY or FITNESS FOR A PARTICULAR PURPOSE.  See the
% GNU General Public License for more details.
%
% You should have received a copy of the GNU General Public License
% along with this program.  If not, see <http://www.gnu.org/licenses/>.
%
%
%%%%%%%%%%%%%%%%%%%%%%%%%%%%%%%%%%%%%%%%%%%%%%%%%%%%%%%%%%%%%%%%%%%%%%%%%%%%%
% SEE pgfplots-macros.tex as well!
%\pdfminorversion=5 % to allow compression
%\pdfobjcompresslevel=2
\documentclass[a4paper,openany]{book}

\let\bookmaketitle=\maketitle

% -----------------------------------
% this here is from ltxdoc:
\usepackage{doc}[=v2]
\AtBeginDocument{\MakeShortVerb{\|}}
%\setlength{\textwidth}{355pt}
%\addtolength\marginparwidth{30pt}
%\addtolength\oddsidemargin{20pt}
%\addtolength\evensidemargin{20pt}
\def\cmd#1{\cs{\expandafter\cmd@to@cs\string#1}}
\def\cmd@to@cs#1#2{\char\number`#2\relax}
\DeclareRobustCommand\cs[1]{\texttt{\char`\\#1}}
\providecommand\marg[1]{%
  {\ttfamily\char`\{}\meta{#1}{\ttfamily\char`\}}}
\providecommand\oarg[1]{%
  {\ttfamily[}\meta{#1}{\ttfamily]}}
\providecommand\parg[1]{%
  {\ttfamily(}\meta{#1}{\ttfamily)}}
\raggedbottom

% -----------------------------------
\input{pgfplots.preamble.tex}

\makeatletter
% I want a two-column index just as in pgfmanual styles. This here
% was the best way to get one:
\def\index@prologue{\section*{Index}\addcontentsline{toc}{chapter}{Index}
}
\makeatother

%\RequirePackage[german,english,francais]{babel}

\def\matlabcolormaptext{This colormap is similar to one shipped with Matlab$^\text{\textregistered}$ under a similar name.}

\IfFileExists{tikzlibraryspy.code.tex}{%
\usetikzlibrary{spy}
}{%
    \message{ERROR: tikz SPY library NOT available. The manual will only compile partially.^^J}%
}%

\usepackage{xparse}% for colorbrewer manual

\usetikzlibrary{
    decorations.markings,
    decorations.footprints,
    shapes.arrows,
    matrix,
    positioning,
}

\usepgfplotslibrary{
    fillbetween,
    ternary,
    smithchart,
    patchplots,
    polar,
    colormaps,
    colorbrewer,
    colortol,           % docu not ready yet
}
\pgfqkeys{/codeexample}{%
    every codeexample/.append style={
        /pgfplots/every ternary axis/.append style={
            /pgfplots/legend style={fill=graphicbackground},
        }
    },
    tabsize=4,
}

\pgfplotsmanualenableexternalizationofexpensive

%\usetikzlibrary{external}
%\tikzexternalize[prefix=figures/]{pgfplots}

\title{%
    Manual for Package \PGFPlots{}\\
    {\small 2D/3D Plots in \LaTeX{}, Version \pgfplotsversion}\\
    {\small\url{https://github.com/pgf-tikz/pgfplots}}
    %\\{\small Attention: you are using an unstable development version.}
}

\makeatletter
\long\def\abstractsmuggle{%
    \centering
    \textbf{Abstract}\\[0.5cm]

    \begin{minipage}{12cm}
        \PGFPlots{} draws high-quality function plots in normal or logarithmic
        scaling with a user-friendly interface directly in \TeX{}. The user
        supplies axis labels, legend entries and the plot coordinates for one or
        more plots and \PGFPlots{} applies axis scaling, computes any logarithms
        and axis ticks and draws the plots. It supports line plots, scatter
        plots, piecewise constant plots, bar plots, area plots, mesh and surface
        plots, patch plots, contour plots, quiver plots, histogram plots, box
        plots, polar axes, ternary diagrams, smith charts and some more. It is
        based on Till Tantau's package \PGF{}/\Tikz{}.
    \end{minipage}

}%

\expandafter\date\expandafter{\@date\\[2cm]
    \abstractsmuggle
}%
\makeatother

%\includeonly{pgfplots.reference}


\begin{document}

\def\plotcoords{%
\addplot coordinates {
(5,8.312e-02)    (17,2.547e-02)   (49,7.407e-03)
(129,2.102e-03)  (321,5.874e-04)  (769,1.623e-04)
(1793,4.442e-05) (4097,1.207e-05) (9217,3.261e-06)
};

\addplot coordinates{
(7,8.472e-02)    (31,3.044e-02)    (111,1.022e-02)
(351,3.303e-03)  (1023,1.039e-03)  (2815,3.196e-04)
(7423,9.658e-05) (18943,2.873e-05) (47103,8.437e-06)
};

\addplot coordinates{
(9,7.881e-02)     (49,3.243e-02)    (209,1.232e-02)
(769,4.454e-03)   (2561,1.551e-03)  (7937,5.236e-04)
(23297,1.723e-04) (65537,5.545e-05) (178177,1.751e-05)
};

\addplot coordinates{
(11,6.887e-02)    (71,3.177e-02)     (351,1.341e-02)
(1471,5.334e-03)  (5503,2.027e-03)   (18943,7.415e-04)
(61183,2.628e-04) (187903,9.063e-05) (553983,3.053e-05)
};

\addplot coordinates{
(13,5.755e-02)     (97,2.925e-02)     (545,1.351e-02)
(2561,5.842e-03)   (10625,2.397e-03)  (40193,9.414e-04)
(141569,3.564e-04) (471041,1.308e-04)
(1496065,4.670e-05)
};
}%


\bookmaketitle
\tableofcontents
\include{pgfplots.title_abstract_intro}
\include{pgfplots.preliminaries}
\include{pgfplots.intro}
\include{pgfplots.reference}
\include{pgfplots.libs}
\include{pgfplots.resources}
\include{pgfplots.importexport}
\include{pgfplots.basic.reference}

\printindex

\bibliographystyle{abbrv} %gerapali} %gerabbrv} %gerunsrt.bst} %gerabbrv}% gerplain}
\nocite{pgfplotstable}
\nocite{programmingnotes}
\bibliography{pgfplots}
\end{document}

% -----------------------------------------------------------------------------
% For Stefan Pinnow as reminder on what to look for when editing the manual
% -----------------------------------------------------------------------------
% There should be no line breaks in the following environments
% - |...|
% - \declareandlabel{...}
% - \verbpdfref{...}
% If "MakeTikzPictures" isn't running through check one of the externalized
% LOG files what is the cause of that.
% -----------------------------------------------------------------------------


%%%%%%%%%%%%%%%%%%%%%%%%%%%%%%%%%%%%%%%%%%%%%%%%%%%%%%%%%%%%%%%%%%%%%%%%%%%%%
%
% Package pgfplots.sty documentation.
%
% Copyright 2007/2008 by Christian Feuersaenger.
%
% This program is free software: you can redistribute it and/or modify
% it under the terms of the GNU General Public License as published by
% the Free Software Foundation, either version 3 of the License, or
% (at your option) any later version.
%
% This program is distributed in the hope that it will be useful,
% but WITHOUT ANY WARRANTY; without even the implied warranty of
% MERCHANTABILITY or FITNESS FOR A PARTICULAR PURPOSE.  See the
% GNU General Public License for more details.
%
% You should have received a copy of the GNU General Public License
% along with this program.  If not, see <http://www.gnu.org/licenses/>.
%
%
%%%%%%%%%%%%%%%%%%%%%%%%%%%%%%%%%%%%%%%%%%%%%%%%%%%%%%%%%%%%%%%%%%%%%%%%%%%%%
% SEE pgfplots-macros.tex as well!
%\pdfminorversion=5 % to allow compression
%\pdfobjcompresslevel=2
\documentclass[a4paper,openany]{book}

\let\bookmaketitle=\maketitle

% -----------------------------------
% this here is from ltxdoc:
\usepackage{doc}[=v2]
\AtBeginDocument{\MakeShortVerb{\|}}
%\setlength{\textwidth}{355pt}
%\addtolength\marginparwidth{30pt}
%\addtolength\oddsidemargin{20pt}
%\addtolength\evensidemargin{20pt}
\def\cmd#1{\cs{\expandafter\cmd@to@cs\string#1}}
\def\cmd@to@cs#1#2{\char\number`#2\relax}
\DeclareRobustCommand\cs[1]{\texttt{\char`\\#1}}
\providecommand\marg[1]{%
  {\ttfamily\char`\{}\meta{#1}{\ttfamily\char`\}}}
\providecommand\oarg[1]{%
  {\ttfamily[}\meta{#1}{\ttfamily]}}
\providecommand\parg[1]{%
  {\ttfamily(}\meta{#1}{\ttfamily)}}
\raggedbottom

% -----------------------------------
\input{pgfplots.preamble.tex}

\makeatletter
% I want a two-column index just as in pgfmanual styles. This here
% was the best way to get one:
\def\index@prologue{\section*{Index}\addcontentsline{toc}{chapter}{Index}
}
\makeatother

%\RequirePackage[german,english,francais]{babel}

\def\matlabcolormaptext{This colormap is similar to one shipped with Matlab$^\text{\textregistered}$ under a similar name.}

\IfFileExists{tikzlibraryspy.code.tex}{%
\usetikzlibrary{spy}
}{%
    \message{ERROR: tikz SPY library NOT available. The manual will only compile partially.^^J}%
}%

\usepackage{xparse}% for colorbrewer manual

\usetikzlibrary{
    decorations.markings,
    decorations.footprints,
    shapes.arrows,
    matrix,
    positioning,
}

\usepgfplotslibrary{
    fillbetween,
    ternary,
    smithchart,
    patchplots,
    polar,
    colormaps,
    colorbrewer,
    colortol,           % docu not ready yet
}
\pgfqkeys{/codeexample}{%
    every codeexample/.append style={
        /pgfplots/every ternary axis/.append style={
            /pgfplots/legend style={fill=graphicbackground},
        }
    },
    tabsize=4,
}

\pgfplotsmanualenableexternalizationofexpensive

%\usetikzlibrary{external}
%\tikzexternalize[prefix=figures/]{pgfplots}

\title{%
    Manual for Package \PGFPlots{}\\
    {\small 2D/3D Plots in \LaTeX{}, Version \pgfplotsversion}\\
    {\small\url{https://github.com/pgf-tikz/pgfplots}}
    %\\{\small Attention: you are using an unstable development version.}
}

\makeatletter
\long\def\abstractsmuggle{%
    \centering
    \textbf{Abstract}\\[0.5cm]

    \begin{minipage}{12cm}
        \PGFPlots{} draws high-quality function plots in normal or logarithmic
        scaling with a user-friendly interface directly in \TeX{}. The user
        supplies axis labels, legend entries and the plot coordinates for one or
        more plots and \PGFPlots{} applies axis scaling, computes any logarithms
        and axis ticks and draws the plots. It supports line plots, scatter
        plots, piecewise constant plots, bar plots, area plots, mesh and surface
        plots, patch plots, contour plots, quiver plots, histogram plots, box
        plots, polar axes, ternary diagrams, smith charts and some more. It is
        based on Till Tantau's package \PGF{}/\Tikz{}.
    \end{minipage}

}%

\expandafter\date\expandafter{\@date\\[2cm]
    \abstractsmuggle
}%
\makeatother

%\includeonly{pgfplots.reference}


\begin{document}

\def\plotcoords{%
\addplot coordinates {
(5,8.312e-02)    (17,2.547e-02)   (49,7.407e-03)
(129,2.102e-03)  (321,5.874e-04)  (769,1.623e-04)
(1793,4.442e-05) (4097,1.207e-05) (9217,3.261e-06)
};

\addplot coordinates{
(7,8.472e-02)    (31,3.044e-02)    (111,1.022e-02)
(351,3.303e-03)  (1023,1.039e-03)  (2815,3.196e-04)
(7423,9.658e-05) (18943,2.873e-05) (47103,8.437e-06)
};

\addplot coordinates{
(9,7.881e-02)     (49,3.243e-02)    (209,1.232e-02)
(769,4.454e-03)   (2561,1.551e-03)  (7937,5.236e-04)
(23297,1.723e-04) (65537,5.545e-05) (178177,1.751e-05)
};

\addplot coordinates{
(11,6.887e-02)    (71,3.177e-02)     (351,1.341e-02)
(1471,5.334e-03)  (5503,2.027e-03)   (18943,7.415e-04)
(61183,2.628e-04) (187903,9.063e-05) (553983,3.053e-05)
};

\addplot coordinates{
(13,5.755e-02)     (97,2.925e-02)     (545,1.351e-02)
(2561,5.842e-03)   (10625,2.397e-03)  (40193,9.414e-04)
(141569,3.564e-04) (471041,1.308e-04)
(1496065,4.670e-05)
};
}%


\bookmaketitle
\tableofcontents
\include{pgfplots.title_abstract_intro}
\include{pgfplots.preliminaries}
\include{pgfplots.intro}
\include{pgfplots.reference}
\include{pgfplots.libs}
\include{pgfplots.resources}
\include{pgfplots.importexport}
\include{pgfplots.basic.reference}

\printindex

\bibliographystyle{abbrv} %gerapali} %gerabbrv} %gerunsrt.bst} %gerabbrv}% gerplain}
\nocite{pgfplotstable}
\nocite{programmingnotes}
\bibliography{pgfplots}
\end{document}

% -----------------------------------------------------------------------------
% For Stefan Pinnow as reminder on what to look for when editing the manual
% -----------------------------------------------------------------------------
% There should be no line breaks in the following environments
% - |...|
% - \declareandlabel{...}
% - \verbpdfref{...}
% If "MakeTikzPictures" isn't running through check one of the externalized
% LOG files what is the cause of that.
% -----------------------------------------------------------------------------


%%%%%%%%%%%%%%%%%%%%%%%%%%%%%%%%%%%%%%%%%%%%%%%%%%%%%%%%%%%%%%%%%%%%%%%%%%%%%
%
% Package pgfplots.sty documentation.
%
% Copyright 2007/2008 by Christian Feuersaenger.
%
% This program is free software: you can redistribute it and/or modify
% it under the terms of the GNU General Public License as published by
% the Free Software Foundation, either version 3 of the License, or
% (at your option) any later version.
%
% This program is distributed in the hope that it will be useful,
% but WITHOUT ANY WARRANTY; without even the implied warranty of
% MERCHANTABILITY or FITNESS FOR A PARTICULAR PURPOSE.  See the
% GNU General Public License for more details.
%
% You should have received a copy of the GNU General Public License
% along with this program.  If not, see <http://www.gnu.org/licenses/>.
%
%
%%%%%%%%%%%%%%%%%%%%%%%%%%%%%%%%%%%%%%%%%%%%%%%%%%%%%%%%%%%%%%%%%%%%%%%%%%%%%
% SEE pgfplots-macros.tex as well!
%\pdfminorversion=5 % to allow compression
%\pdfobjcompresslevel=2
\documentclass[a4paper,openany]{book}

\let\bookmaketitle=\maketitle

% -----------------------------------
% this here is from ltxdoc:
\usepackage{doc}[=v2]
\AtBeginDocument{\MakeShortVerb{\|}}
%\setlength{\textwidth}{355pt}
%\addtolength\marginparwidth{30pt}
%\addtolength\oddsidemargin{20pt}
%\addtolength\evensidemargin{20pt}
\def\cmd#1{\cs{\expandafter\cmd@to@cs\string#1}}
\def\cmd@to@cs#1#2{\char\number`#2\relax}
\DeclareRobustCommand\cs[1]{\texttt{\char`\\#1}}
\providecommand\marg[1]{%
  {\ttfamily\char`\{}\meta{#1}{\ttfamily\char`\}}}
\providecommand\oarg[1]{%
  {\ttfamily[}\meta{#1}{\ttfamily]}}
\providecommand\parg[1]{%
  {\ttfamily(}\meta{#1}{\ttfamily)}}
\raggedbottom

% -----------------------------------
\input{pgfplots.preamble.tex}

\makeatletter
% I want a two-column index just as in pgfmanual styles. This here
% was the best way to get one:
\def\index@prologue{\section*{Index}\addcontentsline{toc}{chapter}{Index}
}
\makeatother

%\RequirePackage[german,english,francais]{babel}

\def\matlabcolormaptext{This colormap is similar to one shipped with Matlab$^\text{\textregistered}$ under a similar name.}

\IfFileExists{tikzlibraryspy.code.tex}{%
\usetikzlibrary{spy}
}{%
    \message{ERROR: tikz SPY library NOT available. The manual will only compile partially.^^J}%
}%

\usepackage{xparse}% for colorbrewer manual

\usetikzlibrary{
    decorations.markings,
    decorations.footprints,
    shapes.arrows,
    matrix,
    positioning,
}

\usepgfplotslibrary{
    fillbetween,
    ternary,
    smithchart,
    patchplots,
    polar,
    colormaps,
    colorbrewer,
    colortol,           % docu not ready yet
}
\pgfqkeys{/codeexample}{%
    every codeexample/.append style={
        /pgfplots/every ternary axis/.append style={
            /pgfplots/legend style={fill=graphicbackground},
        }
    },
    tabsize=4,
}

\pgfplotsmanualenableexternalizationofexpensive

%\usetikzlibrary{external}
%\tikzexternalize[prefix=figures/]{pgfplots}

\title{%
    Manual for Package \PGFPlots{}\\
    {\small 2D/3D Plots in \LaTeX{}, Version \pgfplotsversion}\\
    {\small\url{https://github.com/pgf-tikz/pgfplots}}
    %\\{\small Attention: you are using an unstable development version.}
}

\makeatletter
\long\def\abstractsmuggle{%
    \centering
    \textbf{Abstract}\\[0.5cm]

    \begin{minipage}{12cm}
        \PGFPlots{} draws high-quality function plots in normal or logarithmic
        scaling with a user-friendly interface directly in \TeX{}. The user
        supplies axis labels, legend entries and the plot coordinates for one or
        more plots and \PGFPlots{} applies axis scaling, computes any logarithms
        and axis ticks and draws the plots. It supports line plots, scatter
        plots, piecewise constant plots, bar plots, area plots, mesh and surface
        plots, patch plots, contour plots, quiver plots, histogram plots, box
        plots, polar axes, ternary diagrams, smith charts and some more. It is
        based on Till Tantau's package \PGF{}/\Tikz{}.
    \end{minipage}

}%

\expandafter\date\expandafter{\@date\\[2cm]
    \abstractsmuggle
}%
\makeatother

%\includeonly{pgfplots.reference}


\begin{document}

\def\plotcoords{%
\addplot coordinates {
(5,8.312e-02)    (17,2.547e-02)   (49,7.407e-03)
(129,2.102e-03)  (321,5.874e-04)  (769,1.623e-04)
(1793,4.442e-05) (4097,1.207e-05) (9217,3.261e-06)
};

\addplot coordinates{
(7,8.472e-02)    (31,3.044e-02)    (111,1.022e-02)
(351,3.303e-03)  (1023,1.039e-03)  (2815,3.196e-04)
(7423,9.658e-05) (18943,2.873e-05) (47103,8.437e-06)
};

\addplot coordinates{
(9,7.881e-02)     (49,3.243e-02)    (209,1.232e-02)
(769,4.454e-03)   (2561,1.551e-03)  (7937,5.236e-04)
(23297,1.723e-04) (65537,5.545e-05) (178177,1.751e-05)
};

\addplot coordinates{
(11,6.887e-02)    (71,3.177e-02)     (351,1.341e-02)
(1471,5.334e-03)  (5503,2.027e-03)   (18943,7.415e-04)
(61183,2.628e-04) (187903,9.063e-05) (553983,3.053e-05)
};

\addplot coordinates{
(13,5.755e-02)     (97,2.925e-02)     (545,1.351e-02)
(2561,5.842e-03)   (10625,2.397e-03)  (40193,9.414e-04)
(141569,3.564e-04) (471041,1.308e-04)
(1496065,4.670e-05)
};
}%


\bookmaketitle
\tableofcontents
\include{pgfplots.title_abstract_intro}
\include{pgfplots.preliminaries}
\include{pgfplots.intro}
\include{pgfplots.reference}
\include{pgfplots.libs}
\include{pgfplots.resources}
\include{pgfplots.importexport}
\include{pgfplots.basic.reference}

\printindex

\bibliographystyle{abbrv} %gerapali} %gerabbrv} %gerunsrt.bst} %gerabbrv}% gerplain}
\nocite{pgfplotstable}
\nocite{programmingnotes}
\bibliography{pgfplots}
\end{document}

% -----------------------------------------------------------------------------
% For Stefan Pinnow as reminder on what to look for when editing the manual
% -----------------------------------------------------------------------------
% There should be no line breaks in the following environments
% - |...|
% - \declareandlabel{...}
% - \verbpdfref{...}
% If "MakeTikzPictures" isn't running through check one of the externalized
% LOG files what is the cause of that.
% -----------------------------------------------------------------------------


%%%%%%%%%%%%%%%%%%%%%%%%%%%%%%%%%%%%%%%%%%%%%%%%%%%%%%%%%%%%%%%%%%%%%%%%%%%%%
%
% Package pgfplots.sty documentation.
%
% Copyright 2007/2008 by Christian Feuersaenger.
%
% This program is free software: you can redistribute it and/or modify
% it under the terms of the GNU General Public License as published by
% the Free Software Foundation, either version 3 of the License, or
% (at your option) any later version.
%
% This program is distributed in the hope that it will be useful,
% but WITHOUT ANY WARRANTY; without even the implied warranty of
% MERCHANTABILITY or FITNESS FOR A PARTICULAR PURPOSE.  See the
% GNU General Public License for more details.
%
% You should have received a copy of the GNU General Public License
% along with this program.  If not, see <http://www.gnu.org/licenses/>.
%
%
%%%%%%%%%%%%%%%%%%%%%%%%%%%%%%%%%%%%%%%%%%%%%%%%%%%%%%%%%%%%%%%%%%%%%%%%%%%%%
% SEE pgfplots-macros.tex as well!
%\pdfminorversion=5 % to allow compression
%\pdfobjcompresslevel=2
\documentclass[a4paper,openany]{book}

\let\bookmaketitle=\maketitle

% -----------------------------------
% this here is from ltxdoc:
\usepackage{doc}[=v2]
\AtBeginDocument{\MakeShortVerb{\|}}
%\setlength{\textwidth}{355pt}
%\addtolength\marginparwidth{30pt}
%\addtolength\oddsidemargin{20pt}
%\addtolength\evensidemargin{20pt}
\def\cmd#1{\cs{\expandafter\cmd@to@cs\string#1}}
\def\cmd@to@cs#1#2{\char\number`#2\relax}
\DeclareRobustCommand\cs[1]{\texttt{\char`\\#1}}
\providecommand\marg[1]{%
  {\ttfamily\char`\{}\meta{#1}{\ttfamily\char`\}}}
\providecommand\oarg[1]{%
  {\ttfamily[}\meta{#1}{\ttfamily]}}
\providecommand\parg[1]{%
  {\ttfamily(}\meta{#1}{\ttfamily)}}
\raggedbottom

% -----------------------------------
\input{pgfplots.preamble.tex}

\makeatletter
% I want a two-column index just as in pgfmanual styles. This here
% was the best way to get one:
\def\index@prologue{\section*{Index}\addcontentsline{toc}{chapter}{Index}
}
\makeatother

%\RequirePackage[german,english,francais]{babel}

\def\matlabcolormaptext{This colormap is similar to one shipped with Matlab$^\text{\textregistered}$ under a similar name.}

\IfFileExists{tikzlibraryspy.code.tex}{%
\usetikzlibrary{spy}
}{%
    \message{ERROR: tikz SPY library NOT available. The manual will only compile partially.^^J}%
}%

\usepackage{xparse}% for colorbrewer manual

\usetikzlibrary{
    decorations.markings,
    decorations.footprints,
    shapes.arrows,
    matrix,
    positioning,
}

\usepgfplotslibrary{
    fillbetween,
    ternary,
    smithchart,
    patchplots,
    polar,
    colormaps,
    colorbrewer,
    colortol,           % docu not ready yet
}
\pgfqkeys{/codeexample}{%
    every codeexample/.append style={
        /pgfplots/every ternary axis/.append style={
            /pgfplots/legend style={fill=graphicbackground},
        }
    },
    tabsize=4,
}

\pgfplotsmanualenableexternalizationofexpensive

%\usetikzlibrary{external}
%\tikzexternalize[prefix=figures/]{pgfplots}

\title{%
    Manual for Package \PGFPlots{}\\
    {\small 2D/3D Plots in \LaTeX{}, Version \pgfplotsversion}\\
    {\small\url{https://github.com/pgf-tikz/pgfplots}}
    %\\{\small Attention: you are using an unstable development version.}
}

\makeatletter
\long\def\abstractsmuggle{%
    \centering
    \textbf{Abstract}\\[0.5cm]

    \begin{minipage}{12cm}
        \PGFPlots{} draws high-quality function plots in normal or logarithmic
        scaling with a user-friendly interface directly in \TeX{}. The user
        supplies axis labels, legend entries and the plot coordinates for one or
        more plots and \PGFPlots{} applies axis scaling, computes any logarithms
        and axis ticks and draws the plots. It supports line plots, scatter
        plots, piecewise constant plots, bar plots, area plots, mesh and surface
        plots, patch plots, contour plots, quiver plots, histogram plots, box
        plots, polar axes, ternary diagrams, smith charts and some more. It is
        based on Till Tantau's package \PGF{}/\Tikz{}.
    \end{minipage}

}%

\expandafter\date\expandafter{\@date\\[2cm]
    \abstractsmuggle
}%
\makeatother

%\includeonly{pgfplots.reference}


\begin{document}

\def\plotcoords{%
\addplot coordinates {
(5,8.312e-02)    (17,2.547e-02)   (49,7.407e-03)
(129,2.102e-03)  (321,5.874e-04)  (769,1.623e-04)
(1793,4.442e-05) (4097,1.207e-05) (9217,3.261e-06)
};

\addplot coordinates{
(7,8.472e-02)    (31,3.044e-02)    (111,1.022e-02)
(351,3.303e-03)  (1023,1.039e-03)  (2815,3.196e-04)
(7423,9.658e-05) (18943,2.873e-05) (47103,8.437e-06)
};

\addplot coordinates{
(9,7.881e-02)     (49,3.243e-02)    (209,1.232e-02)
(769,4.454e-03)   (2561,1.551e-03)  (7937,5.236e-04)
(23297,1.723e-04) (65537,5.545e-05) (178177,1.751e-05)
};

\addplot coordinates{
(11,6.887e-02)    (71,3.177e-02)     (351,1.341e-02)
(1471,5.334e-03)  (5503,2.027e-03)   (18943,7.415e-04)
(61183,2.628e-04) (187903,9.063e-05) (553983,3.053e-05)
};

\addplot coordinates{
(13,5.755e-02)     (97,2.925e-02)     (545,1.351e-02)
(2561,5.842e-03)   (10625,2.397e-03)  (40193,9.414e-04)
(141569,3.564e-04) (471041,1.308e-04)
(1496065,4.670e-05)
};
}%


\bookmaketitle
\tableofcontents
\include{pgfplots.title_abstract_intro}
\include{pgfplots.preliminaries}
\include{pgfplots.intro}
\include{pgfplots.reference}
\include{pgfplots.libs}
\include{pgfplots.resources}
\include{pgfplots.importexport}
\include{pgfplots.basic.reference}

\printindex

\bibliographystyle{abbrv} %gerapali} %gerabbrv} %gerunsrt.bst} %gerabbrv}% gerplain}
\nocite{pgfplotstable}
\nocite{programmingnotes}
\bibliography{pgfplots}
\end{document}

% -----------------------------------------------------------------------------
% For Stefan Pinnow as reminder on what to look for when editing the manual
% -----------------------------------------------------------------------------
% There should be no line breaks in the following environments
% - |...|
% - \declareandlabel{...}
% - \verbpdfref{...}
% If "MakeTikzPictures" isn't running through check one of the externalized
% LOG files what is the cause of that.
% -----------------------------------------------------------------------------


%%%%%%%%%%%%%%%%%%%%%%%%%%%%%%%%%%%%%%%%%%%%%%%%%%%%%%%%%%%%%%%%%%%%%%%%%%%%%
%
% Package pgfplots.sty documentation.
%
% Copyright 2007/2008 by Christian Feuersaenger.
%
% This program is free software: you can redistribute it and/or modify
% it under the terms of the GNU General Public License as published by
% the Free Software Foundation, either version 3 of the License, or
% (at your option) any later version.
%
% This program is distributed in the hope that it will be useful,
% but WITHOUT ANY WARRANTY; without even the implied warranty of
% MERCHANTABILITY or FITNESS FOR A PARTICULAR PURPOSE.  See the
% GNU General Public License for more details.
%
% You should have received a copy of the GNU General Public License
% along with this program.  If not, see <http://www.gnu.org/licenses/>.
%
%
%%%%%%%%%%%%%%%%%%%%%%%%%%%%%%%%%%%%%%%%%%%%%%%%%%%%%%%%%%%%%%%%%%%%%%%%%%%%%
% SEE pgfplots-macros.tex as well!
%\pdfminorversion=5 % to allow compression
%\pdfobjcompresslevel=2
\documentclass[a4paper,openany]{book}

\let\bookmaketitle=\maketitle

% -----------------------------------
% this here is from ltxdoc:
\usepackage{doc}[=v2]
\AtBeginDocument{\MakeShortVerb{\|}}
%\setlength{\textwidth}{355pt}
%\addtolength\marginparwidth{30pt}
%\addtolength\oddsidemargin{20pt}
%\addtolength\evensidemargin{20pt}
\def\cmd#1{\cs{\expandafter\cmd@to@cs\string#1}}
\def\cmd@to@cs#1#2{\char\number`#2\relax}
\DeclareRobustCommand\cs[1]{\texttt{\char`\\#1}}
\providecommand\marg[1]{%
  {\ttfamily\char`\{}\meta{#1}{\ttfamily\char`\}}}
\providecommand\oarg[1]{%
  {\ttfamily[}\meta{#1}{\ttfamily]}}
\providecommand\parg[1]{%
  {\ttfamily(}\meta{#1}{\ttfamily)}}
\raggedbottom

% -----------------------------------
\input{pgfplots.preamble.tex}

\makeatletter
% I want a two-column index just as in pgfmanual styles. This here
% was the best way to get one:
\def\index@prologue{\section*{Index}\addcontentsline{toc}{chapter}{Index}
}
\makeatother

%\RequirePackage[german,english,francais]{babel}

\def\matlabcolormaptext{This colormap is similar to one shipped with Matlab$^\text{\textregistered}$ under a similar name.}

\IfFileExists{tikzlibraryspy.code.tex}{%
\usetikzlibrary{spy}
}{%
    \message{ERROR: tikz SPY library NOT available. The manual will only compile partially.^^J}%
}%

\usepackage{xparse}% for colorbrewer manual

\usetikzlibrary{
    decorations.markings,
    decorations.footprints,
    shapes.arrows,
    matrix,
    positioning,
}

\usepgfplotslibrary{
    fillbetween,
    ternary,
    smithchart,
    patchplots,
    polar,
    colormaps,
    colorbrewer,
    colortol,           % docu not ready yet
}
\pgfqkeys{/codeexample}{%
    every codeexample/.append style={
        /pgfplots/every ternary axis/.append style={
            /pgfplots/legend style={fill=graphicbackground},
        }
    },
    tabsize=4,
}

\pgfplotsmanualenableexternalizationofexpensive

%\usetikzlibrary{external}
%\tikzexternalize[prefix=figures/]{pgfplots}

\title{%
    Manual for Package \PGFPlots{}\\
    {\small 2D/3D Plots in \LaTeX{}, Version \pgfplotsversion}\\
    {\small\url{https://github.com/pgf-tikz/pgfplots}}
    %\\{\small Attention: you are using an unstable development version.}
}

\makeatletter
\long\def\abstractsmuggle{%
    \centering
    \textbf{Abstract}\\[0.5cm]

    \begin{minipage}{12cm}
        \PGFPlots{} draws high-quality function plots in normal or logarithmic
        scaling with a user-friendly interface directly in \TeX{}. The user
        supplies axis labels, legend entries and the plot coordinates for one or
        more plots and \PGFPlots{} applies axis scaling, computes any logarithms
        and axis ticks and draws the plots. It supports line plots, scatter
        plots, piecewise constant plots, bar plots, area plots, mesh and surface
        plots, patch plots, contour plots, quiver plots, histogram plots, box
        plots, polar axes, ternary diagrams, smith charts and some more. It is
        based on Till Tantau's package \PGF{}/\Tikz{}.
    \end{minipage}

}%

\expandafter\date\expandafter{\@date\\[2cm]
    \abstractsmuggle
}%
\makeatother

%\includeonly{pgfplots.reference}


\begin{document}

\def\plotcoords{%
\addplot coordinates {
(5,8.312e-02)    (17,2.547e-02)   (49,7.407e-03)
(129,2.102e-03)  (321,5.874e-04)  (769,1.623e-04)
(1793,4.442e-05) (4097,1.207e-05) (9217,3.261e-06)
};

\addplot coordinates{
(7,8.472e-02)    (31,3.044e-02)    (111,1.022e-02)
(351,3.303e-03)  (1023,1.039e-03)  (2815,3.196e-04)
(7423,9.658e-05) (18943,2.873e-05) (47103,8.437e-06)
};

\addplot coordinates{
(9,7.881e-02)     (49,3.243e-02)    (209,1.232e-02)
(769,4.454e-03)   (2561,1.551e-03)  (7937,5.236e-04)
(23297,1.723e-04) (65537,5.545e-05) (178177,1.751e-05)
};

\addplot coordinates{
(11,6.887e-02)    (71,3.177e-02)     (351,1.341e-02)
(1471,5.334e-03)  (5503,2.027e-03)   (18943,7.415e-04)
(61183,2.628e-04) (187903,9.063e-05) (553983,3.053e-05)
};

\addplot coordinates{
(13,5.755e-02)     (97,2.925e-02)     (545,1.351e-02)
(2561,5.842e-03)   (10625,2.397e-03)  (40193,9.414e-04)
(141569,3.564e-04) (471041,1.308e-04)
(1496065,4.670e-05)
};
}%


\bookmaketitle
\tableofcontents
\include{pgfplots.title_abstract_intro}
\include{pgfplots.preliminaries}
\include{pgfplots.intro}
\include{pgfplots.reference}
\include{pgfplots.libs}
\include{pgfplots.resources}
\include{pgfplots.importexport}
\include{pgfplots.basic.reference}

\printindex

\bibliographystyle{abbrv} %gerapali} %gerabbrv} %gerunsrt.bst} %gerabbrv}% gerplain}
\nocite{pgfplotstable}
\nocite{programmingnotes}
\bibliography{pgfplots}
\end{document}

% -----------------------------------------------------------------------------
% For Stefan Pinnow as reminder on what to look for when editing the manual
% -----------------------------------------------------------------------------
% There should be no line breaks in the following environments
% - |...|
% - \declareandlabel{...}
% - \verbpdfref{...}
% If "MakeTikzPictures" isn't running through check one of the externalized
% LOG files what is the cause of that.
% -----------------------------------------------------------------------------


%%%%%%%%%%%%%%%%%%%%%%%%%%%%%%%%%%%%%%%%%%%%%%%%%%%%%%%%%%%%%%%%%%%%%%%%%%%%%
%
% Package pgfplots.sty documentation.
%
% Copyright 2007/2008 by Christian Feuersaenger.
%
% This program is free software: you can redistribute it and/or modify
% it under the terms of the GNU General Public License as published by
% the Free Software Foundation, either version 3 of the License, or
% (at your option) any later version.
%
% This program is distributed in the hope that it will be useful,
% but WITHOUT ANY WARRANTY; without even the implied warranty of
% MERCHANTABILITY or FITNESS FOR A PARTICULAR PURPOSE.  See the
% GNU General Public License for more details.
%
% You should have received a copy of the GNU General Public License
% along with this program.  If not, see <http://www.gnu.org/licenses/>.
%
%
%%%%%%%%%%%%%%%%%%%%%%%%%%%%%%%%%%%%%%%%%%%%%%%%%%%%%%%%%%%%%%%%%%%%%%%%%%%%%
% SEE pgfplots-macros.tex as well!
%\pdfminorversion=5 % to allow compression
%\pdfobjcompresslevel=2
\documentclass[a4paper,openany]{book}

\let\bookmaketitle=\maketitle

% -----------------------------------
% this here is from ltxdoc:
\usepackage{doc}[=v2]
\AtBeginDocument{\MakeShortVerb{\|}}
%\setlength{\textwidth}{355pt}
%\addtolength\marginparwidth{30pt}
%\addtolength\oddsidemargin{20pt}
%\addtolength\evensidemargin{20pt}
\def\cmd#1{\cs{\expandafter\cmd@to@cs\string#1}}
\def\cmd@to@cs#1#2{\char\number`#2\relax}
\DeclareRobustCommand\cs[1]{\texttt{\char`\\#1}}
\providecommand\marg[1]{%
  {\ttfamily\char`\{}\meta{#1}{\ttfamily\char`\}}}
\providecommand\oarg[1]{%
  {\ttfamily[}\meta{#1}{\ttfamily]}}
\providecommand\parg[1]{%
  {\ttfamily(}\meta{#1}{\ttfamily)}}
\raggedbottom

% -----------------------------------
\input{pgfplots.preamble.tex}

\makeatletter
% I want a two-column index just as in pgfmanual styles. This here
% was the best way to get one:
\def\index@prologue{\section*{Index}\addcontentsline{toc}{chapter}{Index}
}
\makeatother

%\RequirePackage[german,english,francais]{babel}

\def\matlabcolormaptext{This colormap is similar to one shipped with Matlab$^\text{\textregistered}$ under a similar name.}

\IfFileExists{tikzlibraryspy.code.tex}{%
\usetikzlibrary{spy}
}{%
    \message{ERROR: tikz SPY library NOT available. The manual will only compile partially.^^J}%
}%

\usepackage{xparse}% for colorbrewer manual

\usetikzlibrary{
    decorations.markings,
    decorations.footprints,
    shapes.arrows,
    matrix,
    positioning,
}

\usepgfplotslibrary{
    fillbetween,
    ternary,
    smithchart,
    patchplots,
    polar,
    colormaps,
    colorbrewer,
    colortol,           % docu not ready yet
}
\pgfqkeys{/codeexample}{%
    every codeexample/.append style={
        /pgfplots/every ternary axis/.append style={
            /pgfplots/legend style={fill=graphicbackground},
        }
    },
    tabsize=4,
}

\pgfplotsmanualenableexternalizationofexpensive

%\usetikzlibrary{external}
%\tikzexternalize[prefix=figures/]{pgfplots}

\title{%
    Manual for Package \PGFPlots{}\\
    {\small 2D/3D Plots in \LaTeX{}, Version \pgfplotsversion}\\
    {\small\url{https://github.com/pgf-tikz/pgfplots}}
    %\\{\small Attention: you are using an unstable development version.}
}

\makeatletter
\long\def\abstractsmuggle{%
    \centering
    \textbf{Abstract}\\[0.5cm]

    \begin{minipage}{12cm}
        \PGFPlots{} draws high-quality function plots in normal or logarithmic
        scaling with a user-friendly interface directly in \TeX{}. The user
        supplies axis labels, legend entries and the plot coordinates for one or
        more plots and \PGFPlots{} applies axis scaling, computes any logarithms
        and axis ticks and draws the plots. It supports line plots, scatter
        plots, piecewise constant plots, bar plots, area plots, mesh and surface
        plots, patch plots, contour plots, quiver plots, histogram plots, box
        plots, polar axes, ternary diagrams, smith charts and some more. It is
        based on Till Tantau's package \PGF{}/\Tikz{}.
    \end{minipage}

}%

\expandafter\date\expandafter{\@date\\[2cm]
    \abstractsmuggle
}%
\makeatother

%\includeonly{pgfplots.reference}


\begin{document}

\def\plotcoords{%
\addplot coordinates {
(5,8.312e-02)    (17,2.547e-02)   (49,7.407e-03)
(129,2.102e-03)  (321,5.874e-04)  (769,1.623e-04)
(1793,4.442e-05) (4097,1.207e-05) (9217,3.261e-06)
};

\addplot coordinates{
(7,8.472e-02)    (31,3.044e-02)    (111,1.022e-02)
(351,3.303e-03)  (1023,1.039e-03)  (2815,3.196e-04)
(7423,9.658e-05) (18943,2.873e-05) (47103,8.437e-06)
};

\addplot coordinates{
(9,7.881e-02)     (49,3.243e-02)    (209,1.232e-02)
(769,4.454e-03)   (2561,1.551e-03)  (7937,5.236e-04)
(23297,1.723e-04) (65537,5.545e-05) (178177,1.751e-05)
};

\addplot coordinates{
(11,6.887e-02)    (71,3.177e-02)     (351,1.341e-02)
(1471,5.334e-03)  (5503,2.027e-03)   (18943,7.415e-04)
(61183,2.628e-04) (187903,9.063e-05) (553983,3.053e-05)
};

\addplot coordinates{
(13,5.755e-02)     (97,2.925e-02)     (545,1.351e-02)
(2561,5.842e-03)   (10625,2.397e-03)  (40193,9.414e-04)
(141569,3.564e-04) (471041,1.308e-04)
(1496065,4.670e-05)
};
}%


\bookmaketitle
\tableofcontents
\include{pgfplots.title_abstract_intro}
\include{pgfplots.preliminaries}
\include{pgfplots.intro}
\include{pgfplots.reference}
\include{pgfplots.libs}
\include{pgfplots.resources}
\include{pgfplots.importexport}
\include{pgfplots.basic.reference}

\printindex

\bibliographystyle{abbrv} %gerapali} %gerabbrv} %gerunsrt.bst} %gerabbrv}% gerplain}
\nocite{pgfplotstable}
\nocite{programmingnotes}
\bibliography{pgfplots}
\end{document}

% -----------------------------------------------------------------------------
% For Stefan Pinnow as reminder on what to look for when editing the manual
% -----------------------------------------------------------------------------
% There should be no line breaks in the following environments
% - |...|
% - \declareandlabel{...}
% - \verbpdfref{...}
% If "MakeTikzPictures" isn't running through check one of the externalized
% LOG files what is the cause of that.
% -----------------------------------------------------------------------------


%%%%%%%%%%%%%%%%%%%%%%%%%%%%%%%%%%%%%%%%%%%%%%%%%%%%%%%%%%%%%%%%%%%%%%%%%%%%%
%
% Package pgfplots.sty documentation.
%
% Copyright 2007/2008 by Christian Feuersaenger.
%
% This program is free software: you can redistribute it and/or modify
% it under the terms of the GNU General Public License as published by
% the Free Software Foundation, either version 3 of the License, or
% (at your option) any later version.
%
% This program is distributed in the hope that it will be useful,
% but WITHOUT ANY WARRANTY; without even the implied warranty of
% MERCHANTABILITY or FITNESS FOR A PARTICULAR PURPOSE.  See the
% GNU General Public License for more details.
%
% You should have received a copy of the GNU General Public License
% along with this program.  If not, see <http://www.gnu.org/licenses/>.
%
%
%%%%%%%%%%%%%%%%%%%%%%%%%%%%%%%%%%%%%%%%%%%%%%%%%%%%%%%%%%%%%%%%%%%%%%%%%%%%%
% SEE pgfplots-macros.tex as well!
%\pdfminorversion=5 % to allow compression
%\pdfobjcompresslevel=2
\documentclass[a4paper,openany]{book}

\let\bookmaketitle=\maketitle

% -----------------------------------
% this here is from ltxdoc:
\usepackage{doc}[=v2]
\AtBeginDocument{\MakeShortVerb{\|}}
%\setlength{\textwidth}{355pt}
%\addtolength\marginparwidth{30pt}
%\addtolength\oddsidemargin{20pt}
%\addtolength\evensidemargin{20pt}
\def\cmd#1{\cs{\expandafter\cmd@to@cs\string#1}}
\def\cmd@to@cs#1#2{\char\number`#2\relax}
\DeclareRobustCommand\cs[1]{\texttt{\char`\\#1}}
\providecommand\marg[1]{%
  {\ttfamily\char`\{}\meta{#1}{\ttfamily\char`\}}}
\providecommand\oarg[1]{%
  {\ttfamily[}\meta{#1}{\ttfamily]}}
\providecommand\parg[1]{%
  {\ttfamily(}\meta{#1}{\ttfamily)}}
\raggedbottom

% -----------------------------------
\input{pgfplots.preamble.tex}

\makeatletter
% I want a two-column index just as in pgfmanual styles. This here
% was the best way to get one:
\def\index@prologue{\section*{Index}\addcontentsline{toc}{chapter}{Index}
}
\makeatother

%\RequirePackage[german,english,francais]{babel}

\def\matlabcolormaptext{This colormap is similar to one shipped with Matlab$^\text{\textregistered}$ under a similar name.}

\IfFileExists{tikzlibraryspy.code.tex}{%
\usetikzlibrary{spy}
}{%
    \message{ERROR: tikz SPY library NOT available. The manual will only compile partially.^^J}%
}%

\usepackage{xparse}% for colorbrewer manual

\usetikzlibrary{
    decorations.markings,
    decorations.footprints,
    shapes.arrows,
    matrix,
    positioning,
}

\usepgfplotslibrary{
    fillbetween,
    ternary,
    smithchart,
    patchplots,
    polar,
    colormaps,
    colorbrewer,
    colortol,           % docu not ready yet
}
\pgfqkeys{/codeexample}{%
    every codeexample/.append style={
        /pgfplots/every ternary axis/.append style={
            /pgfplots/legend style={fill=graphicbackground},
        }
    },
    tabsize=4,
}

\pgfplotsmanualenableexternalizationofexpensive

%\usetikzlibrary{external}
%\tikzexternalize[prefix=figures/]{pgfplots}

\title{%
    Manual for Package \PGFPlots{}\\
    {\small 2D/3D Plots in \LaTeX{}, Version \pgfplotsversion}\\
    {\small\url{https://github.com/pgf-tikz/pgfplots}}
    %\\{\small Attention: you are using an unstable development version.}
}

\makeatletter
\long\def\abstractsmuggle{%
    \centering
    \textbf{Abstract}\\[0.5cm]

    \begin{minipage}{12cm}
        \PGFPlots{} draws high-quality function plots in normal or logarithmic
        scaling with a user-friendly interface directly in \TeX{}. The user
        supplies axis labels, legend entries and the plot coordinates for one or
        more plots and \PGFPlots{} applies axis scaling, computes any logarithms
        and axis ticks and draws the plots. It supports line plots, scatter
        plots, piecewise constant plots, bar plots, area plots, mesh and surface
        plots, patch plots, contour plots, quiver plots, histogram plots, box
        plots, polar axes, ternary diagrams, smith charts and some more. It is
        based on Till Tantau's package \PGF{}/\Tikz{}.
    \end{minipage}

}%

\expandafter\date\expandafter{\@date\\[2cm]
    \abstractsmuggle
}%
\makeatother

%\includeonly{pgfplots.reference}


\begin{document}

\def\plotcoords{%
\addplot coordinates {
(5,8.312e-02)    (17,2.547e-02)   (49,7.407e-03)
(129,2.102e-03)  (321,5.874e-04)  (769,1.623e-04)
(1793,4.442e-05) (4097,1.207e-05) (9217,3.261e-06)
};

\addplot coordinates{
(7,8.472e-02)    (31,3.044e-02)    (111,1.022e-02)
(351,3.303e-03)  (1023,1.039e-03)  (2815,3.196e-04)
(7423,9.658e-05) (18943,2.873e-05) (47103,8.437e-06)
};

\addplot coordinates{
(9,7.881e-02)     (49,3.243e-02)    (209,1.232e-02)
(769,4.454e-03)   (2561,1.551e-03)  (7937,5.236e-04)
(23297,1.723e-04) (65537,5.545e-05) (178177,1.751e-05)
};

\addplot coordinates{
(11,6.887e-02)    (71,3.177e-02)     (351,1.341e-02)
(1471,5.334e-03)  (5503,2.027e-03)   (18943,7.415e-04)
(61183,2.628e-04) (187903,9.063e-05) (553983,3.053e-05)
};

\addplot coordinates{
(13,5.755e-02)     (97,2.925e-02)     (545,1.351e-02)
(2561,5.842e-03)   (10625,2.397e-03)  (40193,9.414e-04)
(141569,3.564e-04) (471041,1.308e-04)
(1496065,4.670e-05)
};
}%


\bookmaketitle
\tableofcontents
\include{pgfplots.title_abstract_intro}
\include{pgfplots.preliminaries}
\include{pgfplots.intro}
\include{pgfplots.reference}
\include{pgfplots.libs}
\include{pgfplots.resources}
\include{pgfplots.importexport}
\include{pgfplots.basic.reference}

\printindex

\bibliographystyle{abbrv} %gerapali} %gerabbrv} %gerunsrt.bst} %gerabbrv}% gerplain}
\nocite{pgfplotstable}
\nocite{programmingnotes}
\bibliography{pgfplots}
\end{document}

% -----------------------------------------------------------------------------
% For Stefan Pinnow as reminder on what to look for when editing the manual
% -----------------------------------------------------------------------------
% There should be no line breaks in the following environments
% - |...|
% - \declareandlabel{...}
% - \verbpdfref{...}
% If "MakeTikzPictures" isn't running through check one of the externalized
% LOG files what is the cause of that.
% -----------------------------------------------------------------------------

\include{pgfplots.basic.reference}

\printindex

\bibliographystyle{abbrv} %gerapali} %gerabbrv} %gerunsrt.bst} %gerabbrv}% gerplain}
\nocite{pgfplotstable}
\nocite{programmingnotes}
\bibliography{pgfplots}
\end{document}

% -----------------------------------------------------------------------------
% For Stefan Pinnow as reminder on what to look for when editing the manual
% -----------------------------------------------------------------------------
% There should be no line breaks in the following environments
% - |...|
% - \declareandlabel{...}
% - \verbpdfref{...}
% If "MakeTikzPictures" isn't running through check one of the externalized
% LOG files what is the cause of that.
% -----------------------------------------------------------------------------


%%%%%%%%%%%%%%%%%%%%%%%%%%%%%%%%%%%%%%%%%%%%%%%%%%%%%%%%%%%%%%%%%%%%%%%%%%%%%
%
% Package pgfplots.sty documentation.
%
% Copyright 2007/2008 by Christian Feuersaenger.
%
% This program is free software: you can redistribute it and/or modify
% it under the terms of the GNU General Public License as published by
% the Free Software Foundation, either version 3 of the License, or
% (at your option) any later version.
%
% This program is distributed in the hope that it will be useful,
% but WITHOUT ANY WARRANTY; without even the implied warranty of
% MERCHANTABILITY or FITNESS FOR A PARTICULAR PURPOSE.  See the
% GNU General Public License for more details.
%
% You should have received a copy of the GNU General Public License
% along with this program.  If not, see <http://www.gnu.org/licenses/>.
%
%
%%%%%%%%%%%%%%%%%%%%%%%%%%%%%%%%%%%%%%%%%%%%%%%%%%%%%%%%%%%%%%%%%%%%%%%%%%%%%
% SEE pgfplots-macros.tex as well!
%\pdfminorversion=5 % to allow compression
%\pdfobjcompresslevel=2
\documentclass[a4paper,openany]{book}

\let\bookmaketitle=\maketitle

% -----------------------------------
% this here is from ltxdoc:
\usepackage{doc}[=v2]
\AtBeginDocument{\MakeShortVerb{\|}}
%\setlength{\textwidth}{355pt}
%\addtolength\marginparwidth{30pt}
%\addtolength\oddsidemargin{20pt}
%\addtolength\evensidemargin{20pt}
\def\cmd#1{\cs{\expandafter\cmd@to@cs\string#1}}
\def\cmd@to@cs#1#2{\char\number`#2\relax}
\DeclareRobustCommand\cs[1]{\texttt{\char`\\#1}}
\providecommand\marg[1]{%
  {\ttfamily\char`\{}\meta{#1}{\ttfamily\char`\}}}
\providecommand\oarg[1]{%
  {\ttfamily[}\meta{#1}{\ttfamily]}}
\providecommand\parg[1]{%
  {\ttfamily(}\meta{#1}{\ttfamily)}}
\raggedbottom

% -----------------------------------
\input{pgfplots.preamble.tex}

\makeatletter
% I want a two-column index just as in pgfmanual styles. This here
% was the best way to get one:
\def\index@prologue{\section*{Index}\addcontentsline{toc}{chapter}{Index}
}
\makeatother

%\RequirePackage[german,english,francais]{babel}

\def\matlabcolormaptext{This colormap is similar to one shipped with Matlab$^\text{\textregistered}$ under a similar name.}

\IfFileExists{tikzlibraryspy.code.tex}{%
\usetikzlibrary{spy}
}{%
    \message{ERROR: tikz SPY library NOT available. The manual will only compile partially.^^J}%
}%

\usepackage{xparse}% for colorbrewer manual

\usetikzlibrary{
    decorations.markings,
    decorations.footprints,
    shapes.arrows,
    matrix,
    positioning,
}

\usepgfplotslibrary{
    fillbetween,
    ternary,
    smithchart,
    patchplots,
    polar,
    colormaps,
    colorbrewer,
    colortol,           % docu not ready yet
}
\pgfqkeys{/codeexample}{%
    every codeexample/.append style={
        /pgfplots/every ternary axis/.append style={
            /pgfplots/legend style={fill=graphicbackground},
        }
    },
    tabsize=4,
}

\pgfplotsmanualenableexternalizationofexpensive

%\usetikzlibrary{external}
%\tikzexternalize[prefix=figures/]{pgfplots}

\title{%
    Manual for Package \PGFPlots{}\\
    {\small 2D/3D Plots in \LaTeX{}, Version \pgfplotsversion}\\
    {\small\url{https://github.com/pgf-tikz/pgfplots}}
    %\\{\small Attention: you are using an unstable development version.}
}

\makeatletter
\long\def\abstractsmuggle{%
    \centering
    \textbf{Abstract}\\[0.5cm]

    \begin{minipage}{12cm}
        \PGFPlots{} draws high-quality function plots in normal or logarithmic
        scaling with a user-friendly interface directly in \TeX{}. The user
        supplies axis labels, legend entries and the plot coordinates for one or
        more plots and \PGFPlots{} applies axis scaling, computes any logarithms
        and axis ticks and draws the plots. It supports line plots, scatter
        plots, piecewise constant plots, bar plots, area plots, mesh and surface
        plots, patch plots, contour plots, quiver plots, histogram plots, box
        plots, polar axes, ternary diagrams, smith charts and some more. It is
        based on Till Tantau's package \PGF{}/\Tikz{}.
    \end{minipage}

}%

\expandafter\date\expandafter{\@date\\[2cm]
    \abstractsmuggle
}%
\makeatother

%\includeonly{pgfplots.reference}


\begin{document}

\def\plotcoords{%
\addplot coordinates {
(5,8.312e-02)    (17,2.547e-02)   (49,7.407e-03)
(129,2.102e-03)  (321,5.874e-04)  (769,1.623e-04)
(1793,4.442e-05) (4097,1.207e-05) (9217,3.261e-06)
};

\addplot coordinates{
(7,8.472e-02)    (31,3.044e-02)    (111,1.022e-02)
(351,3.303e-03)  (1023,1.039e-03)  (2815,3.196e-04)
(7423,9.658e-05) (18943,2.873e-05) (47103,8.437e-06)
};

\addplot coordinates{
(9,7.881e-02)     (49,3.243e-02)    (209,1.232e-02)
(769,4.454e-03)   (2561,1.551e-03)  (7937,5.236e-04)
(23297,1.723e-04) (65537,5.545e-05) (178177,1.751e-05)
};

\addplot coordinates{
(11,6.887e-02)    (71,3.177e-02)     (351,1.341e-02)
(1471,5.334e-03)  (5503,2.027e-03)   (18943,7.415e-04)
(61183,2.628e-04) (187903,9.063e-05) (553983,3.053e-05)
};

\addplot coordinates{
(13,5.755e-02)     (97,2.925e-02)     (545,1.351e-02)
(2561,5.842e-03)   (10625,2.397e-03)  (40193,9.414e-04)
(141569,3.564e-04) (471041,1.308e-04)
(1496065,4.670e-05)
};
}%


\bookmaketitle
\tableofcontents

%%%%%%%%%%%%%%%%%%%%%%%%%%%%%%%%%%%%%%%%%%%%%%%%%%%%%%%%%%%%%%%%%%%%%%%%%%%%%
%
% Package pgfplots.sty documentation.
%
% Copyright 2007/2008 by Christian Feuersaenger.
%
% This program is free software: you can redistribute it and/or modify
% it under the terms of the GNU General Public License as published by
% the Free Software Foundation, either version 3 of the License, or
% (at your option) any later version.
%
% This program is distributed in the hope that it will be useful,
% but WITHOUT ANY WARRANTY; without even the implied warranty of
% MERCHANTABILITY or FITNESS FOR A PARTICULAR PURPOSE.  See the
% GNU General Public License for more details.
%
% You should have received a copy of the GNU General Public License
% along with this program.  If not, see <http://www.gnu.org/licenses/>.
%
%
%%%%%%%%%%%%%%%%%%%%%%%%%%%%%%%%%%%%%%%%%%%%%%%%%%%%%%%%%%%%%%%%%%%%%%%%%%%%%
% SEE pgfplots-macros.tex as well!
%\pdfminorversion=5 % to allow compression
%\pdfobjcompresslevel=2
\documentclass[a4paper,openany]{book}

\let\bookmaketitle=\maketitle

% -----------------------------------
% this here is from ltxdoc:
\usepackage{doc}[=v2]
\AtBeginDocument{\MakeShortVerb{\|}}
%\setlength{\textwidth}{355pt}
%\addtolength\marginparwidth{30pt}
%\addtolength\oddsidemargin{20pt}
%\addtolength\evensidemargin{20pt}
\def\cmd#1{\cs{\expandafter\cmd@to@cs\string#1}}
\def\cmd@to@cs#1#2{\char\number`#2\relax}
\DeclareRobustCommand\cs[1]{\texttt{\char`\\#1}}
\providecommand\marg[1]{%
  {\ttfamily\char`\{}\meta{#1}{\ttfamily\char`\}}}
\providecommand\oarg[1]{%
  {\ttfamily[}\meta{#1}{\ttfamily]}}
\providecommand\parg[1]{%
  {\ttfamily(}\meta{#1}{\ttfamily)}}
\raggedbottom

% -----------------------------------
\input{pgfplots.preamble.tex}

\makeatletter
% I want a two-column index just as in pgfmanual styles. This here
% was the best way to get one:
\def\index@prologue{\section*{Index}\addcontentsline{toc}{chapter}{Index}
}
\makeatother

%\RequirePackage[german,english,francais]{babel}

\def\matlabcolormaptext{This colormap is similar to one shipped with Matlab$^\text{\textregistered}$ under a similar name.}

\IfFileExists{tikzlibraryspy.code.tex}{%
\usetikzlibrary{spy}
}{%
    \message{ERROR: tikz SPY library NOT available. The manual will only compile partially.^^J}%
}%

\usepackage{xparse}% for colorbrewer manual

\usetikzlibrary{
    decorations.markings,
    decorations.footprints,
    shapes.arrows,
    matrix,
    positioning,
}

\usepgfplotslibrary{
    fillbetween,
    ternary,
    smithchart,
    patchplots,
    polar,
    colormaps,
    colorbrewer,
    colortol,           % docu not ready yet
}
\pgfqkeys{/codeexample}{%
    every codeexample/.append style={
        /pgfplots/every ternary axis/.append style={
            /pgfplots/legend style={fill=graphicbackground},
        }
    },
    tabsize=4,
}

\pgfplotsmanualenableexternalizationofexpensive

%\usetikzlibrary{external}
%\tikzexternalize[prefix=figures/]{pgfplots}

\title{%
    Manual for Package \PGFPlots{}\\
    {\small 2D/3D Plots in \LaTeX{}, Version \pgfplotsversion}\\
    {\small\url{https://github.com/pgf-tikz/pgfplots}}
    %\\{\small Attention: you are using an unstable development version.}
}

\makeatletter
\long\def\abstractsmuggle{%
    \centering
    \textbf{Abstract}\\[0.5cm]

    \begin{minipage}{12cm}
        \PGFPlots{} draws high-quality function plots in normal or logarithmic
        scaling with a user-friendly interface directly in \TeX{}. The user
        supplies axis labels, legend entries and the plot coordinates for one or
        more plots and \PGFPlots{} applies axis scaling, computes any logarithms
        and axis ticks and draws the plots. It supports line plots, scatter
        plots, piecewise constant plots, bar plots, area plots, mesh and surface
        plots, patch plots, contour plots, quiver plots, histogram plots, box
        plots, polar axes, ternary diagrams, smith charts and some more. It is
        based on Till Tantau's package \PGF{}/\Tikz{}.
    \end{minipage}

}%

\expandafter\date\expandafter{\@date\\[2cm]
    \abstractsmuggle
}%
\makeatother

%\includeonly{pgfplots.reference}


\begin{document}

\def\plotcoords{%
\addplot coordinates {
(5,8.312e-02)    (17,2.547e-02)   (49,7.407e-03)
(129,2.102e-03)  (321,5.874e-04)  (769,1.623e-04)
(1793,4.442e-05) (4097,1.207e-05) (9217,3.261e-06)
};

\addplot coordinates{
(7,8.472e-02)    (31,3.044e-02)    (111,1.022e-02)
(351,3.303e-03)  (1023,1.039e-03)  (2815,3.196e-04)
(7423,9.658e-05) (18943,2.873e-05) (47103,8.437e-06)
};

\addplot coordinates{
(9,7.881e-02)     (49,3.243e-02)    (209,1.232e-02)
(769,4.454e-03)   (2561,1.551e-03)  (7937,5.236e-04)
(23297,1.723e-04) (65537,5.545e-05) (178177,1.751e-05)
};

\addplot coordinates{
(11,6.887e-02)    (71,3.177e-02)     (351,1.341e-02)
(1471,5.334e-03)  (5503,2.027e-03)   (18943,7.415e-04)
(61183,2.628e-04) (187903,9.063e-05) (553983,3.053e-05)
};

\addplot coordinates{
(13,5.755e-02)     (97,2.925e-02)     (545,1.351e-02)
(2561,5.842e-03)   (10625,2.397e-03)  (40193,9.414e-04)
(141569,3.564e-04) (471041,1.308e-04)
(1496065,4.670e-05)
};
}%


\bookmaketitle
\tableofcontents
\include{pgfplots.title_abstract_intro}
\include{pgfplots.preliminaries}
\include{pgfplots.intro}
\include{pgfplots.reference}
\include{pgfplots.libs}
\include{pgfplots.resources}
\include{pgfplots.importexport}
\include{pgfplots.basic.reference}

\printindex

\bibliographystyle{abbrv} %gerapali} %gerabbrv} %gerunsrt.bst} %gerabbrv}% gerplain}
\nocite{pgfplotstable}
\nocite{programmingnotes}
\bibliography{pgfplots}
\end{document}

% -----------------------------------------------------------------------------
% For Stefan Pinnow as reminder on what to look for when editing the manual
% -----------------------------------------------------------------------------
% There should be no line breaks in the following environments
% - |...|
% - \declareandlabel{...}
% - \verbpdfref{...}
% If "MakeTikzPictures" isn't running through check one of the externalized
% LOG files what is the cause of that.
% -----------------------------------------------------------------------------


%%%%%%%%%%%%%%%%%%%%%%%%%%%%%%%%%%%%%%%%%%%%%%%%%%%%%%%%%%%%%%%%%%%%%%%%%%%%%
%
% Package pgfplots.sty documentation.
%
% Copyright 2007/2008 by Christian Feuersaenger.
%
% This program is free software: you can redistribute it and/or modify
% it under the terms of the GNU General Public License as published by
% the Free Software Foundation, either version 3 of the License, or
% (at your option) any later version.
%
% This program is distributed in the hope that it will be useful,
% but WITHOUT ANY WARRANTY; without even the implied warranty of
% MERCHANTABILITY or FITNESS FOR A PARTICULAR PURPOSE.  See the
% GNU General Public License for more details.
%
% You should have received a copy of the GNU General Public License
% along with this program.  If not, see <http://www.gnu.org/licenses/>.
%
%
%%%%%%%%%%%%%%%%%%%%%%%%%%%%%%%%%%%%%%%%%%%%%%%%%%%%%%%%%%%%%%%%%%%%%%%%%%%%%
% SEE pgfplots-macros.tex as well!
%\pdfminorversion=5 % to allow compression
%\pdfobjcompresslevel=2
\documentclass[a4paper,openany]{book}

\let\bookmaketitle=\maketitle

% -----------------------------------
% this here is from ltxdoc:
\usepackage{doc}[=v2]
\AtBeginDocument{\MakeShortVerb{\|}}
%\setlength{\textwidth}{355pt}
%\addtolength\marginparwidth{30pt}
%\addtolength\oddsidemargin{20pt}
%\addtolength\evensidemargin{20pt}
\def\cmd#1{\cs{\expandafter\cmd@to@cs\string#1}}
\def\cmd@to@cs#1#2{\char\number`#2\relax}
\DeclareRobustCommand\cs[1]{\texttt{\char`\\#1}}
\providecommand\marg[1]{%
  {\ttfamily\char`\{}\meta{#1}{\ttfamily\char`\}}}
\providecommand\oarg[1]{%
  {\ttfamily[}\meta{#1}{\ttfamily]}}
\providecommand\parg[1]{%
  {\ttfamily(}\meta{#1}{\ttfamily)}}
\raggedbottom

% -----------------------------------
\input{pgfplots.preamble.tex}

\makeatletter
% I want a two-column index just as in pgfmanual styles. This here
% was the best way to get one:
\def\index@prologue{\section*{Index}\addcontentsline{toc}{chapter}{Index}
}
\makeatother

%\RequirePackage[german,english,francais]{babel}

\def\matlabcolormaptext{This colormap is similar to one shipped with Matlab$^\text{\textregistered}$ under a similar name.}

\IfFileExists{tikzlibraryspy.code.tex}{%
\usetikzlibrary{spy}
}{%
    \message{ERROR: tikz SPY library NOT available. The manual will only compile partially.^^J}%
}%

\usepackage{xparse}% for colorbrewer manual

\usetikzlibrary{
    decorations.markings,
    decorations.footprints,
    shapes.arrows,
    matrix,
    positioning,
}

\usepgfplotslibrary{
    fillbetween,
    ternary,
    smithchart,
    patchplots,
    polar,
    colormaps,
    colorbrewer,
    colortol,           % docu not ready yet
}
\pgfqkeys{/codeexample}{%
    every codeexample/.append style={
        /pgfplots/every ternary axis/.append style={
            /pgfplots/legend style={fill=graphicbackground},
        }
    },
    tabsize=4,
}

\pgfplotsmanualenableexternalizationofexpensive

%\usetikzlibrary{external}
%\tikzexternalize[prefix=figures/]{pgfplots}

\title{%
    Manual for Package \PGFPlots{}\\
    {\small 2D/3D Plots in \LaTeX{}, Version \pgfplotsversion}\\
    {\small\url{https://github.com/pgf-tikz/pgfplots}}
    %\\{\small Attention: you are using an unstable development version.}
}

\makeatletter
\long\def\abstractsmuggle{%
    \centering
    \textbf{Abstract}\\[0.5cm]

    \begin{minipage}{12cm}
        \PGFPlots{} draws high-quality function plots in normal or logarithmic
        scaling with a user-friendly interface directly in \TeX{}. The user
        supplies axis labels, legend entries and the plot coordinates for one or
        more plots and \PGFPlots{} applies axis scaling, computes any logarithms
        and axis ticks and draws the plots. It supports line plots, scatter
        plots, piecewise constant plots, bar plots, area plots, mesh and surface
        plots, patch plots, contour plots, quiver plots, histogram plots, box
        plots, polar axes, ternary diagrams, smith charts and some more. It is
        based on Till Tantau's package \PGF{}/\Tikz{}.
    \end{minipage}

}%

\expandafter\date\expandafter{\@date\\[2cm]
    \abstractsmuggle
}%
\makeatother

%\includeonly{pgfplots.reference}


\begin{document}

\def\plotcoords{%
\addplot coordinates {
(5,8.312e-02)    (17,2.547e-02)   (49,7.407e-03)
(129,2.102e-03)  (321,5.874e-04)  (769,1.623e-04)
(1793,4.442e-05) (4097,1.207e-05) (9217,3.261e-06)
};

\addplot coordinates{
(7,8.472e-02)    (31,3.044e-02)    (111,1.022e-02)
(351,3.303e-03)  (1023,1.039e-03)  (2815,3.196e-04)
(7423,9.658e-05) (18943,2.873e-05) (47103,8.437e-06)
};

\addplot coordinates{
(9,7.881e-02)     (49,3.243e-02)    (209,1.232e-02)
(769,4.454e-03)   (2561,1.551e-03)  (7937,5.236e-04)
(23297,1.723e-04) (65537,5.545e-05) (178177,1.751e-05)
};

\addplot coordinates{
(11,6.887e-02)    (71,3.177e-02)     (351,1.341e-02)
(1471,5.334e-03)  (5503,2.027e-03)   (18943,7.415e-04)
(61183,2.628e-04) (187903,9.063e-05) (553983,3.053e-05)
};

\addplot coordinates{
(13,5.755e-02)     (97,2.925e-02)     (545,1.351e-02)
(2561,5.842e-03)   (10625,2.397e-03)  (40193,9.414e-04)
(141569,3.564e-04) (471041,1.308e-04)
(1496065,4.670e-05)
};
}%


\bookmaketitle
\tableofcontents
\include{pgfplots.title_abstract_intro}
\include{pgfplots.preliminaries}
\include{pgfplots.intro}
\include{pgfplots.reference}
\include{pgfplots.libs}
\include{pgfplots.resources}
\include{pgfplots.importexport}
\include{pgfplots.basic.reference}

\printindex

\bibliographystyle{abbrv} %gerapali} %gerabbrv} %gerunsrt.bst} %gerabbrv}% gerplain}
\nocite{pgfplotstable}
\nocite{programmingnotes}
\bibliography{pgfplots}
\end{document}

% -----------------------------------------------------------------------------
% For Stefan Pinnow as reminder on what to look for when editing the manual
% -----------------------------------------------------------------------------
% There should be no line breaks in the following environments
% - |...|
% - \declareandlabel{...}
% - \verbpdfref{...}
% If "MakeTikzPictures" isn't running through check one of the externalized
% LOG files what is the cause of that.
% -----------------------------------------------------------------------------


%%%%%%%%%%%%%%%%%%%%%%%%%%%%%%%%%%%%%%%%%%%%%%%%%%%%%%%%%%%%%%%%%%%%%%%%%%%%%
%
% Package pgfplots.sty documentation.
%
% Copyright 2007/2008 by Christian Feuersaenger.
%
% This program is free software: you can redistribute it and/or modify
% it under the terms of the GNU General Public License as published by
% the Free Software Foundation, either version 3 of the License, or
% (at your option) any later version.
%
% This program is distributed in the hope that it will be useful,
% but WITHOUT ANY WARRANTY; without even the implied warranty of
% MERCHANTABILITY or FITNESS FOR A PARTICULAR PURPOSE.  See the
% GNU General Public License for more details.
%
% You should have received a copy of the GNU General Public License
% along with this program.  If not, see <http://www.gnu.org/licenses/>.
%
%
%%%%%%%%%%%%%%%%%%%%%%%%%%%%%%%%%%%%%%%%%%%%%%%%%%%%%%%%%%%%%%%%%%%%%%%%%%%%%
% SEE pgfplots-macros.tex as well!
%\pdfminorversion=5 % to allow compression
%\pdfobjcompresslevel=2
\documentclass[a4paper,openany]{book}

\let\bookmaketitle=\maketitle

% -----------------------------------
% this here is from ltxdoc:
\usepackage{doc}[=v2]
\AtBeginDocument{\MakeShortVerb{\|}}
%\setlength{\textwidth}{355pt}
%\addtolength\marginparwidth{30pt}
%\addtolength\oddsidemargin{20pt}
%\addtolength\evensidemargin{20pt}
\def\cmd#1{\cs{\expandafter\cmd@to@cs\string#1}}
\def\cmd@to@cs#1#2{\char\number`#2\relax}
\DeclareRobustCommand\cs[1]{\texttt{\char`\\#1}}
\providecommand\marg[1]{%
  {\ttfamily\char`\{}\meta{#1}{\ttfamily\char`\}}}
\providecommand\oarg[1]{%
  {\ttfamily[}\meta{#1}{\ttfamily]}}
\providecommand\parg[1]{%
  {\ttfamily(}\meta{#1}{\ttfamily)}}
\raggedbottom

% -----------------------------------
\input{pgfplots.preamble.tex}

\makeatletter
% I want a two-column index just as in pgfmanual styles. This here
% was the best way to get one:
\def\index@prologue{\section*{Index}\addcontentsline{toc}{chapter}{Index}
}
\makeatother

%\RequirePackage[german,english,francais]{babel}

\def\matlabcolormaptext{This colormap is similar to one shipped with Matlab$^\text{\textregistered}$ under a similar name.}

\IfFileExists{tikzlibraryspy.code.tex}{%
\usetikzlibrary{spy}
}{%
    \message{ERROR: tikz SPY library NOT available. The manual will only compile partially.^^J}%
}%

\usepackage{xparse}% for colorbrewer manual

\usetikzlibrary{
    decorations.markings,
    decorations.footprints,
    shapes.arrows,
    matrix,
    positioning,
}

\usepgfplotslibrary{
    fillbetween,
    ternary,
    smithchart,
    patchplots,
    polar,
    colormaps,
    colorbrewer,
    colortol,           % docu not ready yet
}
\pgfqkeys{/codeexample}{%
    every codeexample/.append style={
        /pgfplots/every ternary axis/.append style={
            /pgfplots/legend style={fill=graphicbackground},
        }
    },
    tabsize=4,
}

\pgfplotsmanualenableexternalizationofexpensive

%\usetikzlibrary{external}
%\tikzexternalize[prefix=figures/]{pgfplots}

\title{%
    Manual for Package \PGFPlots{}\\
    {\small 2D/3D Plots in \LaTeX{}, Version \pgfplotsversion}\\
    {\small\url{https://github.com/pgf-tikz/pgfplots}}
    %\\{\small Attention: you are using an unstable development version.}
}

\makeatletter
\long\def\abstractsmuggle{%
    \centering
    \textbf{Abstract}\\[0.5cm]

    \begin{minipage}{12cm}
        \PGFPlots{} draws high-quality function plots in normal or logarithmic
        scaling with a user-friendly interface directly in \TeX{}. The user
        supplies axis labels, legend entries and the plot coordinates for one or
        more plots and \PGFPlots{} applies axis scaling, computes any logarithms
        and axis ticks and draws the plots. It supports line plots, scatter
        plots, piecewise constant plots, bar plots, area plots, mesh and surface
        plots, patch plots, contour plots, quiver plots, histogram plots, box
        plots, polar axes, ternary diagrams, smith charts and some more. It is
        based on Till Tantau's package \PGF{}/\Tikz{}.
    \end{minipage}

}%

\expandafter\date\expandafter{\@date\\[2cm]
    \abstractsmuggle
}%
\makeatother

%\includeonly{pgfplots.reference}


\begin{document}

\def\plotcoords{%
\addplot coordinates {
(5,8.312e-02)    (17,2.547e-02)   (49,7.407e-03)
(129,2.102e-03)  (321,5.874e-04)  (769,1.623e-04)
(1793,4.442e-05) (4097,1.207e-05) (9217,3.261e-06)
};

\addplot coordinates{
(7,8.472e-02)    (31,3.044e-02)    (111,1.022e-02)
(351,3.303e-03)  (1023,1.039e-03)  (2815,3.196e-04)
(7423,9.658e-05) (18943,2.873e-05) (47103,8.437e-06)
};

\addplot coordinates{
(9,7.881e-02)     (49,3.243e-02)    (209,1.232e-02)
(769,4.454e-03)   (2561,1.551e-03)  (7937,5.236e-04)
(23297,1.723e-04) (65537,5.545e-05) (178177,1.751e-05)
};

\addplot coordinates{
(11,6.887e-02)    (71,3.177e-02)     (351,1.341e-02)
(1471,5.334e-03)  (5503,2.027e-03)   (18943,7.415e-04)
(61183,2.628e-04) (187903,9.063e-05) (553983,3.053e-05)
};

\addplot coordinates{
(13,5.755e-02)     (97,2.925e-02)     (545,1.351e-02)
(2561,5.842e-03)   (10625,2.397e-03)  (40193,9.414e-04)
(141569,3.564e-04) (471041,1.308e-04)
(1496065,4.670e-05)
};
}%


\bookmaketitle
\tableofcontents
\include{pgfplots.title_abstract_intro}
\include{pgfplots.preliminaries}
\include{pgfplots.intro}
\include{pgfplots.reference}
\include{pgfplots.libs}
\include{pgfplots.resources}
\include{pgfplots.importexport}
\include{pgfplots.basic.reference}

\printindex

\bibliographystyle{abbrv} %gerapali} %gerabbrv} %gerunsrt.bst} %gerabbrv}% gerplain}
\nocite{pgfplotstable}
\nocite{programmingnotes}
\bibliography{pgfplots}
\end{document}

% -----------------------------------------------------------------------------
% For Stefan Pinnow as reminder on what to look for when editing the manual
% -----------------------------------------------------------------------------
% There should be no line breaks in the following environments
% - |...|
% - \declareandlabel{...}
% - \verbpdfref{...}
% If "MakeTikzPictures" isn't running through check one of the externalized
% LOG files what is the cause of that.
% -----------------------------------------------------------------------------


%%%%%%%%%%%%%%%%%%%%%%%%%%%%%%%%%%%%%%%%%%%%%%%%%%%%%%%%%%%%%%%%%%%%%%%%%%%%%
%
% Package pgfplots.sty documentation.
%
% Copyright 2007/2008 by Christian Feuersaenger.
%
% This program is free software: you can redistribute it and/or modify
% it under the terms of the GNU General Public License as published by
% the Free Software Foundation, either version 3 of the License, or
% (at your option) any later version.
%
% This program is distributed in the hope that it will be useful,
% but WITHOUT ANY WARRANTY; without even the implied warranty of
% MERCHANTABILITY or FITNESS FOR A PARTICULAR PURPOSE.  See the
% GNU General Public License for more details.
%
% You should have received a copy of the GNU General Public License
% along with this program.  If not, see <http://www.gnu.org/licenses/>.
%
%
%%%%%%%%%%%%%%%%%%%%%%%%%%%%%%%%%%%%%%%%%%%%%%%%%%%%%%%%%%%%%%%%%%%%%%%%%%%%%
% SEE pgfplots-macros.tex as well!
%\pdfminorversion=5 % to allow compression
%\pdfobjcompresslevel=2
\documentclass[a4paper,openany]{book}

\let\bookmaketitle=\maketitle

% -----------------------------------
% this here is from ltxdoc:
\usepackage{doc}[=v2]
\AtBeginDocument{\MakeShortVerb{\|}}
%\setlength{\textwidth}{355pt}
%\addtolength\marginparwidth{30pt}
%\addtolength\oddsidemargin{20pt}
%\addtolength\evensidemargin{20pt}
\def\cmd#1{\cs{\expandafter\cmd@to@cs\string#1}}
\def\cmd@to@cs#1#2{\char\number`#2\relax}
\DeclareRobustCommand\cs[1]{\texttt{\char`\\#1}}
\providecommand\marg[1]{%
  {\ttfamily\char`\{}\meta{#1}{\ttfamily\char`\}}}
\providecommand\oarg[1]{%
  {\ttfamily[}\meta{#1}{\ttfamily]}}
\providecommand\parg[1]{%
  {\ttfamily(}\meta{#1}{\ttfamily)}}
\raggedbottom

% -----------------------------------
\input{pgfplots.preamble.tex}

\makeatletter
% I want a two-column index just as in pgfmanual styles. This here
% was the best way to get one:
\def\index@prologue{\section*{Index}\addcontentsline{toc}{chapter}{Index}
}
\makeatother

%\RequirePackage[german,english,francais]{babel}

\def\matlabcolormaptext{This colormap is similar to one shipped with Matlab$^\text{\textregistered}$ under a similar name.}

\IfFileExists{tikzlibraryspy.code.tex}{%
\usetikzlibrary{spy}
}{%
    \message{ERROR: tikz SPY library NOT available. The manual will only compile partially.^^J}%
}%

\usepackage{xparse}% for colorbrewer manual

\usetikzlibrary{
    decorations.markings,
    decorations.footprints,
    shapes.arrows,
    matrix,
    positioning,
}

\usepgfplotslibrary{
    fillbetween,
    ternary,
    smithchart,
    patchplots,
    polar,
    colormaps,
    colorbrewer,
    colortol,           % docu not ready yet
}
\pgfqkeys{/codeexample}{%
    every codeexample/.append style={
        /pgfplots/every ternary axis/.append style={
            /pgfplots/legend style={fill=graphicbackground},
        }
    },
    tabsize=4,
}

\pgfplotsmanualenableexternalizationofexpensive

%\usetikzlibrary{external}
%\tikzexternalize[prefix=figures/]{pgfplots}

\title{%
    Manual for Package \PGFPlots{}\\
    {\small 2D/3D Plots in \LaTeX{}, Version \pgfplotsversion}\\
    {\small\url{https://github.com/pgf-tikz/pgfplots}}
    %\\{\small Attention: you are using an unstable development version.}
}

\makeatletter
\long\def\abstractsmuggle{%
    \centering
    \textbf{Abstract}\\[0.5cm]

    \begin{minipage}{12cm}
        \PGFPlots{} draws high-quality function plots in normal or logarithmic
        scaling with a user-friendly interface directly in \TeX{}. The user
        supplies axis labels, legend entries and the plot coordinates for one or
        more plots and \PGFPlots{} applies axis scaling, computes any logarithms
        and axis ticks and draws the plots. It supports line plots, scatter
        plots, piecewise constant plots, bar plots, area plots, mesh and surface
        plots, patch plots, contour plots, quiver plots, histogram plots, box
        plots, polar axes, ternary diagrams, smith charts and some more. It is
        based on Till Tantau's package \PGF{}/\Tikz{}.
    \end{minipage}

}%

\expandafter\date\expandafter{\@date\\[2cm]
    \abstractsmuggle
}%
\makeatother

%\includeonly{pgfplots.reference}


\begin{document}

\def\plotcoords{%
\addplot coordinates {
(5,8.312e-02)    (17,2.547e-02)   (49,7.407e-03)
(129,2.102e-03)  (321,5.874e-04)  (769,1.623e-04)
(1793,4.442e-05) (4097,1.207e-05) (9217,3.261e-06)
};

\addplot coordinates{
(7,8.472e-02)    (31,3.044e-02)    (111,1.022e-02)
(351,3.303e-03)  (1023,1.039e-03)  (2815,3.196e-04)
(7423,9.658e-05) (18943,2.873e-05) (47103,8.437e-06)
};

\addplot coordinates{
(9,7.881e-02)     (49,3.243e-02)    (209,1.232e-02)
(769,4.454e-03)   (2561,1.551e-03)  (7937,5.236e-04)
(23297,1.723e-04) (65537,5.545e-05) (178177,1.751e-05)
};

\addplot coordinates{
(11,6.887e-02)    (71,3.177e-02)     (351,1.341e-02)
(1471,5.334e-03)  (5503,2.027e-03)   (18943,7.415e-04)
(61183,2.628e-04) (187903,9.063e-05) (553983,3.053e-05)
};

\addplot coordinates{
(13,5.755e-02)     (97,2.925e-02)     (545,1.351e-02)
(2561,5.842e-03)   (10625,2.397e-03)  (40193,9.414e-04)
(141569,3.564e-04) (471041,1.308e-04)
(1496065,4.670e-05)
};
}%


\bookmaketitle
\tableofcontents
\include{pgfplots.title_abstract_intro}
\include{pgfplots.preliminaries}
\include{pgfplots.intro}
\include{pgfplots.reference}
\include{pgfplots.libs}
\include{pgfplots.resources}
\include{pgfplots.importexport}
\include{pgfplots.basic.reference}

\printindex

\bibliographystyle{abbrv} %gerapali} %gerabbrv} %gerunsrt.bst} %gerabbrv}% gerplain}
\nocite{pgfplotstable}
\nocite{programmingnotes}
\bibliography{pgfplots}
\end{document}

% -----------------------------------------------------------------------------
% For Stefan Pinnow as reminder on what to look for when editing the manual
% -----------------------------------------------------------------------------
% There should be no line breaks in the following environments
% - |...|
% - \declareandlabel{...}
% - \verbpdfref{...}
% If "MakeTikzPictures" isn't running through check one of the externalized
% LOG files what is the cause of that.
% -----------------------------------------------------------------------------


%%%%%%%%%%%%%%%%%%%%%%%%%%%%%%%%%%%%%%%%%%%%%%%%%%%%%%%%%%%%%%%%%%%%%%%%%%%%%
%
% Package pgfplots.sty documentation.
%
% Copyright 2007/2008 by Christian Feuersaenger.
%
% This program is free software: you can redistribute it and/or modify
% it under the terms of the GNU General Public License as published by
% the Free Software Foundation, either version 3 of the License, or
% (at your option) any later version.
%
% This program is distributed in the hope that it will be useful,
% but WITHOUT ANY WARRANTY; without even the implied warranty of
% MERCHANTABILITY or FITNESS FOR A PARTICULAR PURPOSE.  See the
% GNU General Public License for more details.
%
% You should have received a copy of the GNU General Public License
% along with this program.  If not, see <http://www.gnu.org/licenses/>.
%
%
%%%%%%%%%%%%%%%%%%%%%%%%%%%%%%%%%%%%%%%%%%%%%%%%%%%%%%%%%%%%%%%%%%%%%%%%%%%%%
% SEE pgfplots-macros.tex as well!
%\pdfminorversion=5 % to allow compression
%\pdfobjcompresslevel=2
\documentclass[a4paper,openany]{book}

\let\bookmaketitle=\maketitle

% -----------------------------------
% this here is from ltxdoc:
\usepackage{doc}[=v2]
\AtBeginDocument{\MakeShortVerb{\|}}
%\setlength{\textwidth}{355pt}
%\addtolength\marginparwidth{30pt}
%\addtolength\oddsidemargin{20pt}
%\addtolength\evensidemargin{20pt}
\def\cmd#1{\cs{\expandafter\cmd@to@cs\string#1}}
\def\cmd@to@cs#1#2{\char\number`#2\relax}
\DeclareRobustCommand\cs[1]{\texttt{\char`\\#1}}
\providecommand\marg[1]{%
  {\ttfamily\char`\{}\meta{#1}{\ttfamily\char`\}}}
\providecommand\oarg[1]{%
  {\ttfamily[}\meta{#1}{\ttfamily]}}
\providecommand\parg[1]{%
  {\ttfamily(}\meta{#1}{\ttfamily)}}
\raggedbottom

% -----------------------------------
\input{pgfplots.preamble.tex}

\makeatletter
% I want a two-column index just as in pgfmanual styles. This here
% was the best way to get one:
\def\index@prologue{\section*{Index}\addcontentsline{toc}{chapter}{Index}
}
\makeatother

%\RequirePackage[german,english,francais]{babel}

\def\matlabcolormaptext{This colormap is similar to one shipped with Matlab$^\text{\textregistered}$ under a similar name.}

\IfFileExists{tikzlibraryspy.code.tex}{%
\usetikzlibrary{spy}
}{%
    \message{ERROR: tikz SPY library NOT available. The manual will only compile partially.^^J}%
}%

\usepackage{xparse}% for colorbrewer manual

\usetikzlibrary{
    decorations.markings,
    decorations.footprints,
    shapes.arrows,
    matrix,
    positioning,
}

\usepgfplotslibrary{
    fillbetween,
    ternary,
    smithchart,
    patchplots,
    polar,
    colormaps,
    colorbrewer,
    colortol,           % docu not ready yet
}
\pgfqkeys{/codeexample}{%
    every codeexample/.append style={
        /pgfplots/every ternary axis/.append style={
            /pgfplots/legend style={fill=graphicbackground},
        }
    },
    tabsize=4,
}

\pgfplotsmanualenableexternalizationofexpensive

%\usetikzlibrary{external}
%\tikzexternalize[prefix=figures/]{pgfplots}

\title{%
    Manual for Package \PGFPlots{}\\
    {\small 2D/3D Plots in \LaTeX{}, Version \pgfplotsversion}\\
    {\small\url{https://github.com/pgf-tikz/pgfplots}}
    %\\{\small Attention: you are using an unstable development version.}
}

\makeatletter
\long\def\abstractsmuggle{%
    \centering
    \textbf{Abstract}\\[0.5cm]

    \begin{minipage}{12cm}
        \PGFPlots{} draws high-quality function plots in normal or logarithmic
        scaling with a user-friendly interface directly in \TeX{}. The user
        supplies axis labels, legend entries and the plot coordinates for one or
        more plots and \PGFPlots{} applies axis scaling, computes any logarithms
        and axis ticks and draws the plots. It supports line plots, scatter
        plots, piecewise constant plots, bar plots, area plots, mesh and surface
        plots, patch plots, contour plots, quiver plots, histogram plots, box
        plots, polar axes, ternary diagrams, smith charts and some more. It is
        based on Till Tantau's package \PGF{}/\Tikz{}.
    \end{minipage}

}%

\expandafter\date\expandafter{\@date\\[2cm]
    \abstractsmuggle
}%
\makeatother

%\includeonly{pgfplots.reference}


\begin{document}

\def\plotcoords{%
\addplot coordinates {
(5,8.312e-02)    (17,2.547e-02)   (49,7.407e-03)
(129,2.102e-03)  (321,5.874e-04)  (769,1.623e-04)
(1793,4.442e-05) (4097,1.207e-05) (9217,3.261e-06)
};

\addplot coordinates{
(7,8.472e-02)    (31,3.044e-02)    (111,1.022e-02)
(351,3.303e-03)  (1023,1.039e-03)  (2815,3.196e-04)
(7423,9.658e-05) (18943,2.873e-05) (47103,8.437e-06)
};

\addplot coordinates{
(9,7.881e-02)     (49,3.243e-02)    (209,1.232e-02)
(769,4.454e-03)   (2561,1.551e-03)  (7937,5.236e-04)
(23297,1.723e-04) (65537,5.545e-05) (178177,1.751e-05)
};

\addplot coordinates{
(11,6.887e-02)    (71,3.177e-02)     (351,1.341e-02)
(1471,5.334e-03)  (5503,2.027e-03)   (18943,7.415e-04)
(61183,2.628e-04) (187903,9.063e-05) (553983,3.053e-05)
};

\addplot coordinates{
(13,5.755e-02)     (97,2.925e-02)     (545,1.351e-02)
(2561,5.842e-03)   (10625,2.397e-03)  (40193,9.414e-04)
(141569,3.564e-04) (471041,1.308e-04)
(1496065,4.670e-05)
};
}%


\bookmaketitle
\tableofcontents
\include{pgfplots.title_abstract_intro}
\include{pgfplots.preliminaries}
\include{pgfplots.intro}
\include{pgfplots.reference}
\include{pgfplots.libs}
\include{pgfplots.resources}
\include{pgfplots.importexport}
\include{pgfplots.basic.reference}

\printindex

\bibliographystyle{abbrv} %gerapali} %gerabbrv} %gerunsrt.bst} %gerabbrv}% gerplain}
\nocite{pgfplotstable}
\nocite{programmingnotes}
\bibliography{pgfplots}
\end{document}

% -----------------------------------------------------------------------------
% For Stefan Pinnow as reminder on what to look for when editing the manual
% -----------------------------------------------------------------------------
% There should be no line breaks in the following environments
% - |...|
% - \declareandlabel{...}
% - \verbpdfref{...}
% If "MakeTikzPictures" isn't running through check one of the externalized
% LOG files what is the cause of that.
% -----------------------------------------------------------------------------


%%%%%%%%%%%%%%%%%%%%%%%%%%%%%%%%%%%%%%%%%%%%%%%%%%%%%%%%%%%%%%%%%%%%%%%%%%%%%
%
% Package pgfplots.sty documentation.
%
% Copyright 2007/2008 by Christian Feuersaenger.
%
% This program is free software: you can redistribute it and/or modify
% it under the terms of the GNU General Public License as published by
% the Free Software Foundation, either version 3 of the License, or
% (at your option) any later version.
%
% This program is distributed in the hope that it will be useful,
% but WITHOUT ANY WARRANTY; without even the implied warranty of
% MERCHANTABILITY or FITNESS FOR A PARTICULAR PURPOSE.  See the
% GNU General Public License for more details.
%
% You should have received a copy of the GNU General Public License
% along with this program.  If not, see <http://www.gnu.org/licenses/>.
%
%
%%%%%%%%%%%%%%%%%%%%%%%%%%%%%%%%%%%%%%%%%%%%%%%%%%%%%%%%%%%%%%%%%%%%%%%%%%%%%
% SEE pgfplots-macros.tex as well!
%\pdfminorversion=5 % to allow compression
%\pdfobjcompresslevel=2
\documentclass[a4paper,openany]{book}

\let\bookmaketitle=\maketitle

% -----------------------------------
% this here is from ltxdoc:
\usepackage{doc}[=v2]
\AtBeginDocument{\MakeShortVerb{\|}}
%\setlength{\textwidth}{355pt}
%\addtolength\marginparwidth{30pt}
%\addtolength\oddsidemargin{20pt}
%\addtolength\evensidemargin{20pt}
\def\cmd#1{\cs{\expandafter\cmd@to@cs\string#1}}
\def\cmd@to@cs#1#2{\char\number`#2\relax}
\DeclareRobustCommand\cs[1]{\texttt{\char`\\#1}}
\providecommand\marg[1]{%
  {\ttfamily\char`\{}\meta{#1}{\ttfamily\char`\}}}
\providecommand\oarg[1]{%
  {\ttfamily[}\meta{#1}{\ttfamily]}}
\providecommand\parg[1]{%
  {\ttfamily(}\meta{#1}{\ttfamily)}}
\raggedbottom

% -----------------------------------
\input{pgfplots.preamble.tex}

\makeatletter
% I want a two-column index just as in pgfmanual styles. This here
% was the best way to get one:
\def\index@prologue{\section*{Index}\addcontentsline{toc}{chapter}{Index}
}
\makeatother

%\RequirePackage[german,english,francais]{babel}

\def\matlabcolormaptext{This colormap is similar to one shipped with Matlab$^\text{\textregistered}$ under a similar name.}

\IfFileExists{tikzlibraryspy.code.tex}{%
\usetikzlibrary{spy}
}{%
    \message{ERROR: tikz SPY library NOT available. The manual will only compile partially.^^J}%
}%

\usepackage{xparse}% for colorbrewer manual

\usetikzlibrary{
    decorations.markings,
    decorations.footprints,
    shapes.arrows,
    matrix,
    positioning,
}

\usepgfplotslibrary{
    fillbetween,
    ternary,
    smithchart,
    patchplots,
    polar,
    colormaps,
    colorbrewer,
    colortol,           % docu not ready yet
}
\pgfqkeys{/codeexample}{%
    every codeexample/.append style={
        /pgfplots/every ternary axis/.append style={
            /pgfplots/legend style={fill=graphicbackground},
        }
    },
    tabsize=4,
}

\pgfplotsmanualenableexternalizationofexpensive

%\usetikzlibrary{external}
%\tikzexternalize[prefix=figures/]{pgfplots}

\title{%
    Manual for Package \PGFPlots{}\\
    {\small 2D/3D Plots in \LaTeX{}, Version \pgfplotsversion}\\
    {\small\url{https://github.com/pgf-tikz/pgfplots}}
    %\\{\small Attention: you are using an unstable development version.}
}

\makeatletter
\long\def\abstractsmuggle{%
    \centering
    \textbf{Abstract}\\[0.5cm]

    \begin{minipage}{12cm}
        \PGFPlots{} draws high-quality function plots in normal or logarithmic
        scaling with a user-friendly interface directly in \TeX{}. The user
        supplies axis labels, legend entries and the plot coordinates for one or
        more plots and \PGFPlots{} applies axis scaling, computes any logarithms
        and axis ticks and draws the plots. It supports line plots, scatter
        plots, piecewise constant plots, bar plots, area plots, mesh and surface
        plots, patch plots, contour plots, quiver plots, histogram plots, box
        plots, polar axes, ternary diagrams, smith charts and some more. It is
        based on Till Tantau's package \PGF{}/\Tikz{}.
    \end{minipage}

}%

\expandafter\date\expandafter{\@date\\[2cm]
    \abstractsmuggle
}%
\makeatother

%\includeonly{pgfplots.reference}


\begin{document}

\def\plotcoords{%
\addplot coordinates {
(5,8.312e-02)    (17,2.547e-02)   (49,7.407e-03)
(129,2.102e-03)  (321,5.874e-04)  (769,1.623e-04)
(1793,4.442e-05) (4097,1.207e-05) (9217,3.261e-06)
};

\addplot coordinates{
(7,8.472e-02)    (31,3.044e-02)    (111,1.022e-02)
(351,3.303e-03)  (1023,1.039e-03)  (2815,3.196e-04)
(7423,9.658e-05) (18943,2.873e-05) (47103,8.437e-06)
};

\addplot coordinates{
(9,7.881e-02)     (49,3.243e-02)    (209,1.232e-02)
(769,4.454e-03)   (2561,1.551e-03)  (7937,5.236e-04)
(23297,1.723e-04) (65537,5.545e-05) (178177,1.751e-05)
};

\addplot coordinates{
(11,6.887e-02)    (71,3.177e-02)     (351,1.341e-02)
(1471,5.334e-03)  (5503,2.027e-03)   (18943,7.415e-04)
(61183,2.628e-04) (187903,9.063e-05) (553983,3.053e-05)
};

\addplot coordinates{
(13,5.755e-02)     (97,2.925e-02)     (545,1.351e-02)
(2561,5.842e-03)   (10625,2.397e-03)  (40193,9.414e-04)
(141569,3.564e-04) (471041,1.308e-04)
(1496065,4.670e-05)
};
}%


\bookmaketitle
\tableofcontents
\include{pgfplots.title_abstract_intro}
\include{pgfplots.preliminaries}
\include{pgfplots.intro}
\include{pgfplots.reference}
\include{pgfplots.libs}
\include{pgfplots.resources}
\include{pgfplots.importexport}
\include{pgfplots.basic.reference}

\printindex

\bibliographystyle{abbrv} %gerapali} %gerabbrv} %gerunsrt.bst} %gerabbrv}% gerplain}
\nocite{pgfplotstable}
\nocite{programmingnotes}
\bibliography{pgfplots}
\end{document}

% -----------------------------------------------------------------------------
% For Stefan Pinnow as reminder on what to look for when editing the manual
% -----------------------------------------------------------------------------
% There should be no line breaks in the following environments
% - |...|
% - \declareandlabel{...}
% - \verbpdfref{...}
% If "MakeTikzPictures" isn't running through check one of the externalized
% LOG files what is the cause of that.
% -----------------------------------------------------------------------------


%%%%%%%%%%%%%%%%%%%%%%%%%%%%%%%%%%%%%%%%%%%%%%%%%%%%%%%%%%%%%%%%%%%%%%%%%%%%%
%
% Package pgfplots.sty documentation.
%
% Copyright 2007/2008 by Christian Feuersaenger.
%
% This program is free software: you can redistribute it and/or modify
% it under the terms of the GNU General Public License as published by
% the Free Software Foundation, either version 3 of the License, or
% (at your option) any later version.
%
% This program is distributed in the hope that it will be useful,
% but WITHOUT ANY WARRANTY; without even the implied warranty of
% MERCHANTABILITY or FITNESS FOR A PARTICULAR PURPOSE.  See the
% GNU General Public License for more details.
%
% You should have received a copy of the GNU General Public License
% along with this program.  If not, see <http://www.gnu.org/licenses/>.
%
%
%%%%%%%%%%%%%%%%%%%%%%%%%%%%%%%%%%%%%%%%%%%%%%%%%%%%%%%%%%%%%%%%%%%%%%%%%%%%%
% SEE pgfplots-macros.tex as well!
%\pdfminorversion=5 % to allow compression
%\pdfobjcompresslevel=2
\documentclass[a4paper,openany]{book}

\let\bookmaketitle=\maketitle

% -----------------------------------
% this here is from ltxdoc:
\usepackage{doc}[=v2]
\AtBeginDocument{\MakeShortVerb{\|}}
%\setlength{\textwidth}{355pt}
%\addtolength\marginparwidth{30pt}
%\addtolength\oddsidemargin{20pt}
%\addtolength\evensidemargin{20pt}
\def\cmd#1{\cs{\expandafter\cmd@to@cs\string#1}}
\def\cmd@to@cs#1#2{\char\number`#2\relax}
\DeclareRobustCommand\cs[1]{\texttt{\char`\\#1}}
\providecommand\marg[1]{%
  {\ttfamily\char`\{}\meta{#1}{\ttfamily\char`\}}}
\providecommand\oarg[1]{%
  {\ttfamily[}\meta{#1}{\ttfamily]}}
\providecommand\parg[1]{%
  {\ttfamily(}\meta{#1}{\ttfamily)}}
\raggedbottom

% -----------------------------------
\input{pgfplots.preamble.tex}

\makeatletter
% I want a two-column index just as in pgfmanual styles. This here
% was the best way to get one:
\def\index@prologue{\section*{Index}\addcontentsline{toc}{chapter}{Index}
}
\makeatother

%\RequirePackage[german,english,francais]{babel}

\def\matlabcolormaptext{This colormap is similar to one shipped with Matlab$^\text{\textregistered}$ under a similar name.}

\IfFileExists{tikzlibraryspy.code.tex}{%
\usetikzlibrary{spy}
}{%
    \message{ERROR: tikz SPY library NOT available. The manual will only compile partially.^^J}%
}%

\usepackage{xparse}% for colorbrewer manual

\usetikzlibrary{
    decorations.markings,
    decorations.footprints,
    shapes.arrows,
    matrix,
    positioning,
}

\usepgfplotslibrary{
    fillbetween,
    ternary,
    smithchart,
    patchplots,
    polar,
    colormaps,
    colorbrewer,
    colortol,           % docu not ready yet
}
\pgfqkeys{/codeexample}{%
    every codeexample/.append style={
        /pgfplots/every ternary axis/.append style={
            /pgfplots/legend style={fill=graphicbackground},
        }
    },
    tabsize=4,
}

\pgfplotsmanualenableexternalizationofexpensive

%\usetikzlibrary{external}
%\tikzexternalize[prefix=figures/]{pgfplots}

\title{%
    Manual for Package \PGFPlots{}\\
    {\small 2D/3D Plots in \LaTeX{}, Version \pgfplotsversion}\\
    {\small\url{https://github.com/pgf-tikz/pgfplots}}
    %\\{\small Attention: you are using an unstable development version.}
}

\makeatletter
\long\def\abstractsmuggle{%
    \centering
    \textbf{Abstract}\\[0.5cm]

    \begin{minipage}{12cm}
        \PGFPlots{} draws high-quality function plots in normal or logarithmic
        scaling with a user-friendly interface directly in \TeX{}. The user
        supplies axis labels, legend entries and the plot coordinates for one or
        more plots and \PGFPlots{} applies axis scaling, computes any logarithms
        and axis ticks and draws the plots. It supports line plots, scatter
        plots, piecewise constant plots, bar plots, area plots, mesh and surface
        plots, patch plots, contour plots, quiver plots, histogram plots, box
        plots, polar axes, ternary diagrams, smith charts and some more. It is
        based on Till Tantau's package \PGF{}/\Tikz{}.
    \end{minipage}

}%

\expandafter\date\expandafter{\@date\\[2cm]
    \abstractsmuggle
}%
\makeatother

%\includeonly{pgfplots.reference}


\begin{document}

\def\plotcoords{%
\addplot coordinates {
(5,8.312e-02)    (17,2.547e-02)   (49,7.407e-03)
(129,2.102e-03)  (321,5.874e-04)  (769,1.623e-04)
(1793,4.442e-05) (4097,1.207e-05) (9217,3.261e-06)
};

\addplot coordinates{
(7,8.472e-02)    (31,3.044e-02)    (111,1.022e-02)
(351,3.303e-03)  (1023,1.039e-03)  (2815,3.196e-04)
(7423,9.658e-05) (18943,2.873e-05) (47103,8.437e-06)
};

\addplot coordinates{
(9,7.881e-02)     (49,3.243e-02)    (209,1.232e-02)
(769,4.454e-03)   (2561,1.551e-03)  (7937,5.236e-04)
(23297,1.723e-04) (65537,5.545e-05) (178177,1.751e-05)
};

\addplot coordinates{
(11,6.887e-02)    (71,3.177e-02)     (351,1.341e-02)
(1471,5.334e-03)  (5503,2.027e-03)   (18943,7.415e-04)
(61183,2.628e-04) (187903,9.063e-05) (553983,3.053e-05)
};

\addplot coordinates{
(13,5.755e-02)     (97,2.925e-02)     (545,1.351e-02)
(2561,5.842e-03)   (10625,2.397e-03)  (40193,9.414e-04)
(141569,3.564e-04) (471041,1.308e-04)
(1496065,4.670e-05)
};
}%


\bookmaketitle
\tableofcontents
\include{pgfplots.title_abstract_intro}
\include{pgfplots.preliminaries}
\include{pgfplots.intro}
\include{pgfplots.reference}
\include{pgfplots.libs}
\include{pgfplots.resources}
\include{pgfplots.importexport}
\include{pgfplots.basic.reference}

\printindex

\bibliographystyle{abbrv} %gerapali} %gerabbrv} %gerunsrt.bst} %gerabbrv}% gerplain}
\nocite{pgfplotstable}
\nocite{programmingnotes}
\bibliography{pgfplots}
\end{document}

% -----------------------------------------------------------------------------
% For Stefan Pinnow as reminder on what to look for when editing the manual
% -----------------------------------------------------------------------------
% There should be no line breaks in the following environments
% - |...|
% - \declareandlabel{...}
% - \verbpdfref{...}
% If "MakeTikzPictures" isn't running through check one of the externalized
% LOG files what is the cause of that.
% -----------------------------------------------------------------------------

\include{pgfplots.basic.reference}

\printindex

\bibliographystyle{abbrv} %gerapali} %gerabbrv} %gerunsrt.bst} %gerabbrv}% gerplain}
\nocite{pgfplotstable}
\nocite{programmingnotes}
\bibliography{pgfplots}
\end{document}

% -----------------------------------------------------------------------------
% For Stefan Pinnow as reminder on what to look for when editing the manual
% -----------------------------------------------------------------------------
% There should be no line breaks in the following environments
% - |...|
% - \declareandlabel{...}
% - \verbpdfref{...}
% If "MakeTikzPictures" isn't running through check one of the externalized
% LOG files what is the cause of that.
% -----------------------------------------------------------------------------


%%%%%%%%%%%%%%%%%%%%%%%%%%%%%%%%%%%%%%%%%%%%%%%%%%%%%%%%%%%%%%%%%%%%%%%%%%%%%
%
% Package pgfplots.sty documentation.
%
% Copyright 2007/2008 by Christian Feuersaenger.
%
% This program is free software: you can redistribute it and/or modify
% it under the terms of the GNU General Public License as published by
% the Free Software Foundation, either version 3 of the License, or
% (at your option) any later version.
%
% This program is distributed in the hope that it will be useful,
% but WITHOUT ANY WARRANTY; without even the implied warranty of
% MERCHANTABILITY or FITNESS FOR A PARTICULAR PURPOSE.  See the
% GNU General Public License for more details.
%
% You should have received a copy of the GNU General Public License
% along with this program.  If not, see <http://www.gnu.org/licenses/>.
%
%
%%%%%%%%%%%%%%%%%%%%%%%%%%%%%%%%%%%%%%%%%%%%%%%%%%%%%%%%%%%%%%%%%%%%%%%%%%%%%
% SEE pgfplots-macros.tex as well!
%\pdfminorversion=5 % to allow compression
%\pdfobjcompresslevel=2
\documentclass[a4paper,openany]{book}

\let\bookmaketitle=\maketitle

% -----------------------------------
% this here is from ltxdoc:
\usepackage{doc}[=v2]
\AtBeginDocument{\MakeShortVerb{\|}}
%\setlength{\textwidth}{355pt}
%\addtolength\marginparwidth{30pt}
%\addtolength\oddsidemargin{20pt}
%\addtolength\evensidemargin{20pt}
\def\cmd#1{\cs{\expandafter\cmd@to@cs\string#1}}
\def\cmd@to@cs#1#2{\char\number`#2\relax}
\DeclareRobustCommand\cs[1]{\texttt{\char`\\#1}}
\providecommand\marg[1]{%
  {\ttfamily\char`\{}\meta{#1}{\ttfamily\char`\}}}
\providecommand\oarg[1]{%
  {\ttfamily[}\meta{#1}{\ttfamily]}}
\providecommand\parg[1]{%
  {\ttfamily(}\meta{#1}{\ttfamily)}}
\raggedbottom

% -----------------------------------
\input{pgfplots.preamble.tex}

\makeatletter
% I want a two-column index just as in pgfmanual styles. This here
% was the best way to get one:
\def\index@prologue{\section*{Index}\addcontentsline{toc}{chapter}{Index}
}
\makeatother

%\RequirePackage[german,english,francais]{babel}

\def\matlabcolormaptext{This colormap is similar to one shipped with Matlab$^\text{\textregistered}$ under a similar name.}

\IfFileExists{tikzlibraryspy.code.tex}{%
\usetikzlibrary{spy}
}{%
    \message{ERROR: tikz SPY library NOT available. The manual will only compile partially.^^J}%
}%

\usepackage{xparse}% for colorbrewer manual

\usetikzlibrary{
    decorations.markings,
    decorations.footprints,
    shapes.arrows,
    matrix,
    positioning,
}

\usepgfplotslibrary{
    fillbetween,
    ternary,
    smithchart,
    patchplots,
    polar,
    colormaps,
    colorbrewer,
    colortol,           % docu not ready yet
}
\pgfqkeys{/codeexample}{%
    every codeexample/.append style={
        /pgfplots/every ternary axis/.append style={
            /pgfplots/legend style={fill=graphicbackground},
        }
    },
    tabsize=4,
}

\pgfplotsmanualenableexternalizationofexpensive

%\usetikzlibrary{external}
%\tikzexternalize[prefix=figures/]{pgfplots}

\title{%
    Manual for Package \PGFPlots{}\\
    {\small 2D/3D Plots in \LaTeX{}, Version \pgfplotsversion}\\
    {\small\url{https://github.com/pgf-tikz/pgfplots}}
    %\\{\small Attention: you are using an unstable development version.}
}

\makeatletter
\long\def\abstractsmuggle{%
    \centering
    \textbf{Abstract}\\[0.5cm]

    \begin{minipage}{12cm}
        \PGFPlots{} draws high-quality function plots in normal or logarithmic
        scaling with a user-friendly interface directly in \TeX{}. The user
        supplies axis labels, legend entries and the plot coordinates for one or
        more plots and \PGFPlots{} applies axis scaling, computes any logarithms
        and axis ticks and draws the plots. It supports line plots, scatter
        plots, piecewise constant plots, bar plots, area plots, mesh and surface
        plots, patch plots, contour plots, quiver plots, histogram plots, box
        plots, polar axes, ternary diagrams, smith charts and some more. It is
        based on Till Tantau's package \PGF{}/\Tikz{}.
    \end{minipage}

}%

\expandafter\date\expandafter{\@date\\[2cm]
    \abstractsmuggle
}%
\makeatother

%\includeonly{pgfplots.reference}


\begin{document}

\def\plotcoords{%
\addplot coordinates {
(5,8.312e-02)    (17,2.547e-02)   (49,7.407e-03)
(129,2.102e-03)  (321,5.874e-04)  (769,1.623e-04)
(1793,4.442e-05) (4097,1.207e-05) (9217,3.261e-06)
};

\addplot coordinates{
(7,8.472e-02)    (31,3.044e-02)    (111,1.022e-02)
(351,3.303e-03)  (1023,1.039e-03)  (2815,3.196e-04)
(7423,9.658e-05) (18943,2.873e-05) (47103,8.437e-06)
};

\addplot coordinates{
(9,7.881e-02)     (49,3.243e-02)    (209,1.232e-02)
(769,4.454e-03)   (2561,1.551e-03)  (7937,5.236e-04)
(23297,1.723e-04) (65537,5.545e-05) (178177,1.751e-05)
};

\addplot coordinates{
(11,6.887e-02)    (71,3.177e-02)     (351,1.341e-02)
(1471,5.334e-03)  (5503,2.027e-03)   (18943,7.415e-04)
(61183,2.628e-04) (187903,9.063e-05) (553983,3.053e-05)
};

\addplot coordinates{
(13,5.755e-02)     (97,2.925e-02)     (545,1.351e-02)
(2561,5.842e-03)   (10625,2.397e-03)  (40193,9.414e-04)
(141569,3.564e-04) (471041,1.308e-04)
(1496065,4.670e-05)
};
}%


\bookmaketitle
\tableofcontents

%%%%%%%%%%%%%%%%%%%%%%%%%%%%%%%%%%%%%%%%%%%%%%%%%%%%%%%%%%%%%%%%%%%%%%%%%%%%%
%
% Package pgfplots.sty documentation.
%
% Copyright 2007/2008 by Christian Feuersaenger.
%
% This program is free software: you can redistribute it and/or modify
% it under the terms of the GNU General Public License as published by
% the Free Software Foundation, either version 3 of the License, or
% (at your option) any later version.
%
% This program is distributed in the hope that it will be useful,
% but WITHOUT ANY WARRANTY; without even the implied warranty of
% MERCHANTABILITY or FITNESS FOR A PARTICULAR PURPOSE.  See the
% GNU General Public License for more details.
%
% You should have received a copy of the GNU General Public License
% along with this program.  If not, see <http://www.gnu.org/licenses/>.
%
%
%%%%%%%%%%%%%%%%%%%%%%%%%%%%%%%%%%%%%%%%%%%%%%%%%%%%%%%%%%%%%%%%%%%%%%%%%%%%%
% SEE pgfplots-macros.tex as well!
%\pdfminorversion=5 % to allow compression
%\pdfobjcompresslevel=2
\documentclass[a4paper,openany]{book}

\let\bookmaketitle=\maketitle

% -----------------------------------
% this here is from ltxdoc:
\usepackage{doc}[=v2]
\AtBeginDocument{\MakeShortVerb{\|}}
%\setlength{\textwidth}{355pt}
%\addtolength\marginparwidth{30pt}
%\addtolength\oddsidemargin{20pt}
%\addtolength\evensidemargin{20pt}
\def\cmd#1{\cs{\expandafter\cmd@to@cs\string#1}}
\def\cmd@to@cs#1#2{\char\number`#2\relax}
\DeclareRobustCommand\cs[1]{\texttt{\char`\\#1}}
\providecommand\marg[1]{%
  {\ttfamily\char`\{}\meta{#1}{\ttfamily\char`\}}}
\providecommand\oarg[1]{%
  {\ttfamily[}\meta{#1}{\ttfamily]}}
\providecommand\parg[1]{%
  {\ttfamily(}\meta{#1}{\ttfamily)}}
\raggedbottom

% -----------------------------------
\input{pgfplots.preamble.tex}

\makeatletter
% I want a two-column index just as in pgfmanual styles. This here
% was the best way to get one:
\def\index@prologue{\section*{Index}\addcontentsline{toc}{chapter}{Index}
}
\makeatother

%\RequirePackage[german,english,francais]{babel}

\def\matlabcolormaptext{This colormap is similar to one shipped with Matlab$^\text{\textregistered}$ under a similar name.}

\IfFileExists{tikzlibraryspy.code.tex}{%
\usetikzlibrary{spy}
}{%
    \message{ERROR: tikz SPY library NOT available. The manual will only compile partially.^^J}%
}%

\usepackage{xparse}% for colorbrewer manual

\usetikzlibrary{
    decorations.markings,
    decorations.footprints,
    shapes.arrows,
    matrix,
    positioning,
}

\usepgfplotslibrary{
    fillbetween,
    ternary,
    smithchart,
    patchplots,
    polar,
    colormaps,
    colorbrewer,
    colortol,           % docu not ready yet
}
\pgfqkeys{/codeexample}{%
    every codeexample/.append style={
        /pgfplots/every ternary axis/.append style={
            /pgfplots/legend style={fill=graphicbackground},
        }
    },
    tabsize=4,
}

\pgfplotsmanualenableexternalizationofexpensive

%\usetikzlibrary{external}
%\tikzexternalize[prefix=figures/]{pgfplots}

\title{%
    Manual for Package \PGFPlots{}\\
    {\small 2D/3D Plots in \LaTeX{}, Version \pgfplotsversion}\\
    {\small\url{https://github.com/pgf-tikz/pgfplots}}
    %\\{\small Attention: you are using an unstable development version.}
}

\makeatletter
\long\def\abstractsmuggle{%
    \centering
    \textbf{Abstract}\\[0.5cm]

    \begin{minipage}{12cm}
        \PGFPlots{} draws high-quality function plots in normal or logarithmic
        scaling with a user-friendly interface directly in \TeX{}. The user
        supplies axis labels, legend entries and the plot coordinates for one or
        more plots and \PGFPlots{} applies axis scaling, computes any logarithms
        and axis ticks and draws the plots. It supports line plots, scatter
        plots, piecewise constant plots, bar plots, area plots, mesh and surface
        plots, patch plots, contour plots, quiver plots, histogram plots, box
        plots, polar axes, ternary diagrams, smith charts and some more. It is
        based on Till Tantau's package \PGF{}/\Tikz{}.
    \end{minipage}

}%

\expandafter\date\expandafter{\@date\\[2cm]
    \abstractsmuggle
}%
\makeatother

%\includeonly{pgfplots.reference}


\begin{document}

\def\plotcoords{%
\addplot coordinates {
(5,8.312e-02)    (17,2.547e-02)   (49,7.407e-03)
(129,2.102e-03)  (321,5.874e-04)  (769,1.623e-04)
(1793,4.442e-05) (4097,1.207e-05) (9217,3.261e-06)
};

\addplot coordinates{
(7,8.472e-02)    (31,3.044e-02)    (111,1.022e-02)
(351,3.303e-03)  (1023,1.039e-03)  (2815,3.196e-04)
(7423,9.658e-05) (18943,2.873e-05) (47103,8.437e-06)
};

\addplot coordinates{
(9,7.881e-02)     (49,3.243e-02)    (209,1.232e-02)
(769,4.454e-03)   (2561,1.551e-03)  (7937,5.236e-04)
(23297,1.723e-04) (65537,5.545e-05) (178177,1.751e-05)
};

\addplot coordinates{
(11,6.887e-02)    (71,3.177e-02)     (351,1.341e-02)
(1471,5.334e-03)  (5503,2.027e-03)   (18943,7.415e-04)
(61183,2.628e-04) (187903,9.063e-05) (553983,3.053e-05)
};

\addplot coordinates{
(13,5.755e-02)     (97,2.925e-02)     (545,1.351e-02)
(2561,5.842e-03)   (10625,2.397e-03)  (40193,9.414e-04)
(141569,3.564e-04) (471041,1.308e-04)
(1496065,4.670e-05)
};
}%


\bookmaketitle
\tableofcontents
\include{pgfplots.title_abstract_intro}
\include{pgfplots.preliminaries}
\include{pgfplots.intro}
\include{pgfplots.reference}
\include{pgfplots.libs}
\include{pgfplots.resources}
\include{pgfplots.importexport}
\include{pgfplots.basic.reference}

\printindex

\bibliographystyle{abbrv} %gerapali} %gerabbrv} %gerunsrt.bst} %gerabbrv}% gerplain}
\nocite{pgfplotstable}
\nocite{programmingnotes}
\bibliography{pgfplots}
\end{document}

% -----------------------------------------------------------------------------
% For Stefan Pinnow as reminder on what to look for when editing the manual
% -----------------------------------------------------------------------------
% There should be no line breaks in the following environments
% - |...|
% - \declareandlabel{...}
% - \verbpdfref{...}
% If "MakeTikzPictures" isn't running through check one of the externalized
% LOG files what is the cause of that.
% -----------------------------------------------------------------------------


%%%%%%%%%%%%%%%%%%%%%%%%%%%%%%%%%%%%%%%%%%%%%%%%%%%%%%%%%%%%%%%%%%%%%%%%%%%%%
%
% Package pgfplots.sty documentation.
%
% Copyright 2007/2008 by Christian Feuersaenger.
%
% This program is free software: you can redistribute it and/or modify
% it under the terms of the GNU General Public License as published by
% the Free Software Foundation, either version 3 of the License, or
% (at your option) any later version.
%
% This program is distributed in the hope that it will be useful,
% but WITHOUT ANY WARRANTY; without even the implied warranty of
% MERCHANTABILITY or FITNESS FOR A PARTICULAR PURPOSE.  See the
% GNU General Public License for more details.
%
% You should have received a copy of the GNU General Public License
% along with this program.  If not, see <http://www.gnu.org/licenses/>.
%
%
%%%%%%%%%%%%%%%%%%%%%%%%%%%%%%%%%%%%%%%%%%%%%%%%%%%%%%%%%%%%%%%%%%%%%%%%%%%%%
% SEE pgfplots-macros.tex as well!
%\pdfminorversion=5 % to allow compression
%\pdfobjcompresslevel=2
\documentclass[a4paper,openany]{book}

\let\bookmaketitle=\maketitle

% -----------------------------------
% this here is from ltxdoc:
\usepackage{doc}[=v2]
\AtBeginDocument{\MakeShortVerb{\|}}
%\setlength{\textwidth}{355pt}
%\addtolength\marginparwidth{30pt}
%\addtolength\oddsidemargin{20pt}
%\addtolength\evensidemargin{20pt}
\def\cmd#1{\cs{\expandafter\cmd@to@cs\string#1}}
\def\cmd@to@cs#1#2{\char\number`#2\relax}
\DeclareRobustCommand\cs[1]{\texttt{\char`\\#1}}
\providecommand\marg[1]{%
  {\ttfamily\char`\{}\meta{#1}{\ttfamily\char`\}}}
\providecommand\oarg[1]{%
  {\ttfamily[}\meta{#1}{\ttfamily]}}
\providecommand\parg[1]{%
  {\ttfamily(}\meta{#1}{\ttfamily)}}
\raggedbottom

% -----------------------------------
\input{pgfplots.preamble.tex}

\makeatletter
% I want a two-column index just as in pgfmanual styles. This here
% was the best way to get one:
\def\index@prologue{\section*{Index}\addcontentsline{toc}{chapter}{Index}
}
\makeatother

%\RequirePackage[german,english,francais]{babel}

\def\matlabcolormaptext{This colormap is similar to one shipped with Matlab$^\text{\textregistered}$ under a similar name.}

\IfFileExists{tikzlibraryspy.code.tex}{%
\usetikzlibrary{spy}
}{%
    \message{ERROR: tikz SPY library NOT available. The manual will only compile partially.^^J}%
}%

\usepackage{xparse}% for colorbrewer manual

\usetikzlibrary{
    decorations.markings,
    decorations.footprints,
    shapes.arrows,
    matrix,
    positioning,
}

\usepgfplotslibrary{
    fillbetween,
    ternary,
    smithchart,
    patchplots,
    polar,
    colormaps,
    colorbrewer,
    colortol,           % docu not ready yet
}
\pgfqkeys{/codeexample}{%
    every codeexample/.append style={
        /pgfplots/every ternary axis/.append style={
            /pgfplots/legend style={fill=graphicbackground},
        }
    },
    tabsize=4,
}

\pgfplotsmanualenableexternalizationofexpensive

%\usetikzlibrary{external}
%\tikzexternalize[prefix=figures/]{pgfplots}

\title{%
    Manual for Package \PGFPlots{}\\
    {\small 2D/3D Plots in \LaTeX{}, Version \pgfplotsversion}\\
    {\small\url{https://github.com/pgf-tikz/pgfplots}}
    %\\{\small Attention: you are using an unstable development version.}
}

\makeatletter
\long\def\abstractsmuggle{%
    \centering
    \textbf{Abstract}\\[0.5cm]

    \begin{minipage}{12cm}
        \PGFPlots{} draws high-quality function plots in normal or logarithmic
        scaling with a user-friendly interface directly in \TeX{}. The user
        supplies axis labels, legend entries and the plot coordinates for one or
        more plots and \PGFPlots{} applies axis scaling, computes any logarithms
        and axis ticks and draws the plots. It supports line plots, scatter
        plots, piecewise constant plots, bar plots, area plots, mesh and surface
        plots, patch plots, contour plots, quiver plots, histogram plots, box
        plots, polar axes, ternary diagrams, smith charts and some more. It is
        based on Till Tantau's package \PGF{}/\Tikz{}.
    \end{minipage}

}%

\expandafter\date\expandafter{\@date\\[2cm]
    \abstractsmuggle
}%
\makeatother

%\includeonly{pgfplots.reference}


\begin{document}

\def\plotcoords{%
\addplot coordinates {
(5,8.312e-02)    (17,2.547e-02)   (49,7.407e-03)
(129,2.102e-03)  (321,5.874e-04)  (769,1.623e-04)
(1793,4.442e-05) (4097,1.207e-05) (9217,3.261e-06)
};

\addplot coordinates{
(7,8.472e-02)    (31,3.044e-02)    (111,1.022e-02)
(351,3.303e-03)  (1023,1.039e-03)  (2815,3.196e-04)
(7423,9.658e-05) (18943,2.873e-05) (47103,8.437e-06)
};

\addplot coordinates{
(9,7.881e-02)     (49,3.243e-02)    (209,1.232e-02)
(769,4.454e-03)   (2561,1.551e-03)  (7937,5.236e-04)
(23297,1.723e-04) (65537,5.545e-05) (178177,1.751e-05)
};

\addplot coordinates{
(11,6.887e-02)    (71,3.177e-02)     (351,1.341e-02)
(1471,5.334e-03)  (5503,2.027e-03)   (18943,7.415e-04)
(61183,2.628e-04) (187903,9.063e-05) (553983,3.053e-05)
};

\addplot coordinates{
(13,5.755e-02)     (97,2.925e-02)     (545,1.351e-02)
(2561,5.842e-03)   (10625,2.397e-03)  (40193,9.414e-04)
(141569,3.564e-04) (471041,1.308e-04)
(1496065,4.670e-05)
};
}%


\bookmaketitle
\tableofcontents
\include{pgfplots.title_abstract_intro}
\include{pgfplots.preliminaries}
\include{pgfplots.intro}
\include{pgfplots.reference}
\include{pgfplots.libs}
\include{pgfplots.resources}
\include{pgfplots.importexport}
\include{pgfplots.basic.reference}

\printindex

\bibliographystyle{abbrv} %gerapali} %gerabbrv} %gerunsrt.bst} %gerabbrv}% gerplain}
\nocite{pgfplotstable}
\nocite{programmingnotes}
\bibliography{pgfplots}
\end{document}

% -----------------------------------------------------------------------------
% For Stefan Pinnow as reminder on what to look for when editing the manual
% -----------------------------------------------------------------------------
% There should be no line breaks in the following environments
% - |...|
% - \declareandlabel{...}
% - \verbpdfref{...}
% If "MakeTikzPictures" isn't running through check one of the externalized
% LOG files what is the cause of that.
% -----------------------------------------------------------------------------


%%%%%%%%%%%%%%%%%%%%%%%%%%%%%%%%%%%%%%%%%%%%%%%%%%%%%%%%%%%%%%%%%%%%%%%%%%%%%
%
% Package pgfplots.sty documentation.
%
% Copyright 2007/2008 by Christian Feuersaenger.
%
% This program is free software: you can redistribute it and/or modify
% it under the terms of the GNU General Public License as published by
% the Free Software Foundation, either version 3 of the License, or
% (at your option) any later version.
%
% This program is distributed in the hope that it will be useful,
% but WITHOUT ANY WARRANTY; without even the implied warranty of
% MERCHANTABILITY or FITNESS FOR A PARTICULAR PURPOSE.  See the
% GNU General Public License for more details.
%
% You should have received a copy of the GNU General Public License
% along with this program.  If not, see <http://www.gnu.org/licenses/>.
%
%
%%%%%%%%%%%%%%%%%%%%%%%%%%%%%%%%%%%%%%%%%%%%%%%%%%%%%%%%%%%%%%%%%%%%%%%%%%%%%
% SEE pgfplots-macros.tex as well!
%\pdfminorversion=5 % to allow compression
%\pdfobjcompresslevel=2
\documentclass[a4paper,openany]{book}

\let\bookmaketitle=\maketitle

% -----------------------------------
% this here is from ltxdoc:
\usepackage{doc}[=v2]
\AtBeginDocument{\MakeShortVerb{\|}}
%\setlength{\textwidth}{355pt}
%\addtolength\marginparwidth{30pt}
%\addtolength\oddsidemargin{20pt}
%\addtolength\evensidemargin{20pt}
\def\cmd#1{\cs{\expandafter\cmd@to@cs\string#1}}
\def\cmd@to@cs#1#2{\char\number`#2\relax}
\DeclareRobustCommand\cs[1]{\texttt{\char`\\#1}}
\providecommand\marg[1]{%
  {\ttfamily\char`\{}\meta{#1}{\ttfamily\char`\}}}
\providecommand\oarg[1]{%
  {\ttfamily[}\meta{#1}{\ttfamily]}}
\providecommand\parg[1]{%
  {\ttfamily(}\meta{#1}{\ttfamily)}}
\raggedbottom

% -----------------------------------
\input{pgfplots.preamble.tex}

\makeatletter
% I want a two-column index just as in pgfmanual styles. This here
% was the best way to get one:
\def\index@prologue{\section*{Index}\addcontentsline{toc}{chapter}{Index}
}
\makeatother

%\RequirePackage[german,english,francais]{babel}

\def\matlabcolormaptext{This colormap is similar to one shipped with Matlab$^\text{\textregistered}$ under a similar name.}

\IfFileExists{tikzlibraryspy.code.tex}{%
\usetikzlibrary{spy}
}{%
    \message{ERROR: tikz SPY library NOT available. The manual will only compile partially.^^J}%
}%

\usepackage{xparse}% for colorbrewer manual

\usetikzlibrary{
    decorations.markings,
    decorations.footprints,
    shapes.arrows,
    matrix,
    positioning,
}

\usepgfplotslibrary{
    fillbetween,
    ternary,
    smithchart,
    patchplots,
    polar,
    colormaps,
    colorbrewer,
    colortol,           % docu not ready yet
}
\pgfqkeys{/codeexample}{%
    every codeexample/.append style={
        /pgfplots/every ternary axis/.append style={
            /pgfplots/legend style={fill=graphicbackground},
        }
    },
    tabsize=4,
}

\pgfplotsmanualenableexternalizationofexpensive

%\usetikzlibrary{external}
%\tikzexternalize[prefix=figures/]{pgfplots}

\title{%
    Manual for Package \PGFPlots{}\\
    {\small 2D/3D Plots in \LaTeX{}, Version \pgfplotsversion}\\
    {\small\url{https://github.com/pgf-tikz/pgfplots}}
    %\\{\small Attention: you are using an unstable development version.}
}

\makeatletter
\long\def\abstractsmuggle{%
    \centering
    \textbf{Abstract}\\[0.5cm]

    \begin{minipage}{12cm}
        \PGFPlots{} draws high-quality function plots in normal or logarithmic
        scaling with a user-friendly interface directly in \TeX{}. The user
        supplies axis labels, legend entries and the plot coordinates for one or
        more plots and \PGFPlots{} applies axis scaling, computes any logarithms
        and axis ticks and draws the plots. It supports line plots, scatter
        plots, piecewise constant plots, bar plots, area plots, mesh and surface
        plots, patch plots, contour plots, quiver plots, histogram plots, box
        plots, polar axes, ternary diagrams, smith charts and some more. It is
        based on Till Tantau's package \PGF{}/\Tikz{}.
    \end{minipage}

}%

\expandafter\date\expandafter{\@date\\[2cm]
    \abstractsmuggle
}%
\makeatother

%\includeonly{pgfplots.reference}


\begin{document}

\def\plotcoords{%
\addplot coordinates {
(5,8.312e-02)    (17,2.547e-02)   (49,7.407e-03)
(129,2.102e-03)  (321,5.874e-04)  (769,1.623e-04)
(1793,4.442e-05) (4097,1.207e-05) (9217,3.261e-06)
};

\addplot coordinates{
(7,8.472e-02)    (31,3.044e-02)    (111,1.022e-02)
(351,3.303e-03)  (1023,1.039e-03)  (2815,3.196e-04)
(7423,9.658e-05) (18943,2.873e-05) (47103,8.437e-06)
};

\addplot coordinates{
(9,7.881e-02)     (49,3.243e-02)    (209,1.232e-02)
(769,4.454e-03)   (2561,1.551e-03)  (7937,5.236e-04)
(23297,1.723e-04) (65537,5.545e-05) (178177,1.751e-05)
};

\addplot coordinates{
(11,6.887e-02)    (71,3.177e-02)     (351,1.341e-02)
(1471,5.334e-03)  (5503,2.027e-03)   (18943,7.415e-04)
(61183,2.628e-04) (187903,9.063e-05) (553983,3.053e-05)
};

\addplot coordinates{
(13,5.755e-02)     (97,2.925e-02)     (545,1.351e-02)
(2561,5.842e-03)   (10625,2.397e-03)  (40193,9.414e-04)
(141569,3.564e-04) (471041,1.308e-04)
(1496065,4.670e-05)
};
}%


\bookmaketitle
\tableofcontents
\include{pgfplots.title_abstract_intro}
\include{pgfplots.preliminaries}
\include{pgfplots.intro}
\include{pgfplots.reference}
\include{pgfplots.libs}
\include{pgfplots.resources}
\include{pgfplots.importexport}
\include{pgfplots.basic.reference}

\printindex

\bibliographystyle{abbrv} %gerapali} %gerabbrv} %gerunsrt.bst} %gerabbrv}% gerplain}
\nocite{pgfplotstable}
\nocite{programmingnotes}
\bibliography{pgfplots}
\end{document}

% -----------------------------------------------------------------------------
% For Stefan Pinnow as reminder on what to look for when editing the manual
% -----------------------------------------------------------------------------
% There should be no line breaks in the following environments
% - |...|
% - \declareandlabel{...}
% - \verbpdfref{...}
% If "MakeTikzPictures" isn't running through check one of the externalized
% LOG files what is the cause of that.
% -----------------------------------------------------------------------------


%%%%%%%%%%%%%%%%%%%%%%%%%%%%%%%%%%%%%%%%%%%%%%%%%%%%%%%%%%%%%%%%%%%%%%%%%%%%%
%
% Package pgfplots.sty documentation.
%
% Copyright 2007/2008 by Christian Feuersaenger.
%
% This program is free software: you can redistribute it and/or modify
% it under the terms of the GNU General Public License as published by
% the Free Software Foundation, either version 3 of the License, or
% (at your option) any later version.
%
% This program is distributed in the hope that it will be useful,
% but WITHOUT ANY WARRANTY; without even the implied warranty of
% MERCHANTABILITY or FITNESS FOR A PARTICULAR PURPOSE.  See the
% GNU General Public License for more details.
%
% You should have received a copy of the GNU General Public License
% along with this program.  If not, see <http://www.gnu.org/licenses/>.
%
%
%%%%%%%%%%%%%%%%%%%%%%%%%%%%%%%%%%%%%%%%%%%%%%%%%%%%%%%%%%%%%%%%%%%%%%%%%%%%%
% SEE pgfplots-macros.tex as well!
%\pdfminorversion=5 % to allow compression
%\pdfobjcompresslevel=2
\documentclass[a4paper,openany]{book}

\let\bookmaketitle=\maketitle

% -----------------------------------
% this here is from ltxdoc:
\usepackage{doc}[=v2]
\AtBeginDocument{\MakeShortVerb{\|}}
%\setlength{\textwidth}{355pt}
%\addtolength\marginparwidth{30pt}
%\addtolength\oddsidemargin{20pt}
%\addtolength\evensidemargin{20pt}
\def\cmd#1{\cs{\expandafter\cmd@to@cs\string#1}}
\def\cmd@to@cs#1#2{\char\number`#2\relax}
\DeclareRobustCommand\cs[1]{\texttt{\char`\\#1}}
\providecommand\marg[1]{%
  {\ttfamily\char`\{}\meta{#1}{\ttfamily\char`\}}}
\providecommand\oarg[1]{%
  {\ttfamily[}\meta{#1}{\ttfamily]}}
\providecommand\parg[1]{%
  {\ttfamily(}\meta{#1}{\ttfamily)}}
\raggedbottom

% -----------------------------------
\input{pgfplots.preamble.tex}

\makeatletter
% I want a two-column index just as in pgfmanual styles. This here
% was the best way to get one:
\def\index@prologue{\section*{Index}\addcontentsline{toc}{chapter}{Index}
}
\makeatother

%\RequirePackage[german,english,francais]{babel}

\def\matlabcolormaptext{This colormap is similar to one shipped with Matlab$^\text{\textregistered}$ under a similar name.}

\IfFileExists{tikzlibraryspy.code.tex}{%
\usetikzlibrary{spy}
}{%
    \message{ERROR: tikz SPY library NOT available. The manual will only compile partially.^^J}%
}%

\usepackage{xparse}% for colorbrewer manual

\usetikzlibrary{
    decorations.markings,
    decorations.footprints,
    shapes.arrows,
    matrix,
    positioning,
}

\usepgfplotslibrary{
    fillbetween,
    ternary,
    smithchart,
    patchplots,
    polar,
    colormaps,
    colorbrewer,
    colortol,           % docu not ready yet
}
\pgfqkeys{/codeexample}{%
    every codeexample/.append style={
        /pgfplots/every ternary axis/.append style={
            /pgfplots/legend style={fill=graphicbackground},
        }
    },
    tabsize=4,
}

\pgfplotsmanualenableexternalizationofexpensive

%\usetikzlibrary{external}
%\tikzexternalize[prefix=figures/]{pgfplots}

\title{%
    Manual for Package \PGFPlots{}\\
    {\small 2D/3D Plots in \LaTeX{}, Version \pgfplotsversion}\\
    {\small\url{https://github.com/pgf-tikz/pgfplots}}
    %\\{\small Attention: you are using an unstable development version.}
}

\makeatletter
\long\def\abstractsmuggle{%
    \centering
    \textbf{Abstract}\\[0.5cm]

    \begin{minipage}{12cm}
        \PGFPlots{} draws high-quality function plots in normal or logarithmic
        scaling with a user-friendly interface directly in \TeX{}. The user
        supplies axis labels, legend entries and the plot coordinates for one or
        more plots and \PGFPlots{} applies axis scaling, computes any logarithms
        and axis ticks and draws the plots. It supports line plots, scatter
        plots, piecewise constant plots, bar plots, area plots, mesh and surface
        plots, patch plots, contour plots, quiver plots, histogram plots, box
        plots, polar axes, ternary diagrams, smith charts and some more. It is
        based on Till Tantau's package \PGF{}/\Tikz{}.
    \end{minipage}

}%

\expandafter\date\expandafter{\@date\\[2cm]
    \abstractsmuggle
}%
\makeatother

%\includeonly{pgfplots.reference}


\begin{document}

\def\plotcoords{%
\addplot coordinates {
(5,8.312e-02)    (17,2.547e-02)   (49,7.407e-03)
(129,2.102e-03)  (321,5.874e-04)  (769,1.623e-04)
(1793,4.442e-05) (4097,1.207e-05) (9217,3.261e-06)
};

\addplot coordinates{
(7,8.472e-02)    (31,3.044e-02)    (111,1.022e-02)
(351,3.303e-03)  (1023,1.039e-03)  (2815,3.196e-04)
(7423,9.658e-05) (18943,2.873e-05) (47103,8.437e-06)
};

\addplot coordinates{
(9,7.881e-02)     (49,3.243e-02)    (209,1.232e-02)
(769,4.454e-03)   (2561,1.551e-03)  (7937,5.236e-04)
(23297,1.723e-04) (65537,5.545e-05) (178177,1.751e-05)
};

\addplot coordinates{
(11,6.887e-02)    (71,3.177e-02)     (351,1.341e-02)
(1471,5.334e-03)  (5503,2.027e-03)   (18943,7.415e-04)
(61183,2.628e-04) (187903,9.063e-05) (553983,3.053e-05)
};

\addplot coordinates{
(13,5.755e-02)     (97,2.925e-02)     (545,1.351e-02)
(2561,5.842e-03)   (10625,2.397e-03)  (40193,9.414e-04)
(141569,3.564e-04) (471041,1.308e-04)
(1496065,4.670e-05)
};
}%


\bookmaketitle
\tableofcontents
\include{pgfplots.title_abstract_intro}
\include{pgfplots.preliminaries}
\include{pgfplots.intro}
\include{pgfplots.reference}
\include{pgfplots.libs}
\include{pgfplots.resources}
\include{pgfplots.importexport}
\include{pgfplots.basic.reference}

\printindex

\bibliographystyle{abbrv} %gerapali} %gerabbrv} %gerunsrt.bst} %gerabbrv}% gerplain}
\nocite{pgfplotstable}
\nocite{programmingnotes}
\bibliography{pgfplots}
\end{document}

% -----------------------------------------------------------------------------
% For Stefan Pinnow as reminder on what to look for when editing the manual
% -----------------------------------------------------------------------------
% There should be no line breaks in the following environments
% - |...|
% - \declareandlabel{...}
% - \verbpdfref{...}
% If "MakeTikzPictures" isn't running through check one of the externalized
% LOG files what is the cause of that.
% -----------------------------------------------------------------------------


%%%%%%%%%%%%%%%%%%%%%%%%%%%%%%%%%%%%%%%%%%%%%%%%%%%%%%%%%%%%%%%%%%%%%%%%%%%%%
%
% Package pgfplots.sty documentation.
%
% Copyright 2007/2008 by Christian Feuersaenger.
%
% This program is free software: you can redistribute it and/or modify
% it under the terms of the GNU General Public License as published by
% the Free Software Foundation, either version 3 of the License, or
% (at your option) any later version.
%
% This program is distributed in the hope that it will be useful,
% but WITHOUT ANY WARRANTY; without even the implied warranty of
% MERCHANTABILITY or FITNESS FOR A PARTICULAR PURPOSE.  See the
% GNU General Public License for more details.
%
% You should have received a copy of the GNU General Public License
% along with this program.  If not, see <http://www.gnu.org/licenses/>.
%
%
%%%%%%%%%%%%%%%%%%%%%%%%%%%%%%%%%%%%%%%%%%%%%%%%%%%%%%%%%%%%%%%%%%%%%%%%%%%%%
% SEE pgfplots-macros.tex as well!
%\pdfminorversion=5 % to allow compression
%\pdfobjcompresslevel=2
\documentclass[a4paper,openany]{book}

\let\bookmaketitle=\maketitle

% -----------------------------------
% this here is from ltxdoc:
\usepackage{doc}[=v2]
\AtBeginDocument{\MakeShortVerb{\|}}
%\setlength{\textwidth}{355pt}
%\addtolength\marginparwidth{30pt}
%\addtolength\oddsidemargin{20pt}
%\addtolength\evensidemargin{20pt}
\def\cmd#1{\cs{\expandafter\cmd@to@cs\string#1}}
\def\cmd@to@cs#1#2{\char\number`#2\relax}
\DeclareRobustCommand\cs[1]{\texttt{\char`\\#1}}
\providecommand\marg[1]{%
  {\ttfamily\char`\{}\meta{#1}{\ttfamily\char`\}}}
\providecommand\oarg[1]{%
  {\ttfamily[}\meta{#1}{\ttfamily]}}
\providecommand\parg[1]{%
  {\ttfamily(}\meta{#1}{\ttfamily)}}
\raggedbottom

% -----------------------------------
\input{pgfplots.preamble.tex}

\makeatletter
% I want a two-column index just as in pgfmanual styles. This here
% was the best way to get one:
\def\index@prologue{\section*{Index}\addcontentsline{toc}{chapter}{Index}
}
\makeatother

%\RequirePackage[german,english,francais]{babel}

\def\matlabcolormaptext{This colormap is similar to one shipped with Matlab$^\text{\textregistered}$ under a similar name.}

\IfFileExists{tikzlibraryspy.code.tex}{%
\usetikzlibrary{spy}
}{%
    \message{ERROR: tikz SPY library NOT available. The manual will only compile partially.^^J}%
}%

\usepackage{xparse}% for colorbrewer manual

\usetikzlibrary{
    decorations.markings,
    decorations.footprints,
    shapes.arrows,
    matrix,
    positioning,
}

\usepgfplotslibrary{
    fillbetween,
    ternary,
    smithchart,
    patchplots,
    polar,
    colormaps,
    colorbrewer,
    colortol,           % docu not ready yet
}
\pgfqkeys{/codeexample}{%
    every codeexample/.append style={
        /pgfplots/every ternary axis/.append style={
            /pgfplots/legend style={fill=graphicbackground},
        }
    },
    tabsize=4,
}

\pgfplotsmanualenableexternalizationofexpensive

%\usetikzlibrary{external}
%\tikzexternalize[prefix=figures/]{pgfplots}

\title{%
    Manual for Package \PGFPlots{}\\
    {\small 2D/3D Plots in \LaTeX{}, Version \pgfplotsversion}\\
    {\small\url{https://github.com/pgf-tikz/pgfplots}}
    %\\{\small Attention: you are using an unstable development version.}
}

\makeatletter
\long\def\abstractsmuggle{%
    \centering
    \textbf{Abstract}\\[0.5cm]

    \begin{minipage}{12cm}
        \PGFPlots{} draws high-quality function plots in normal or logarithmic
        scaling with a user-friendly interface directly in \TeX{}. The user
        supplies axis labels, legend entries and the plot coordinates for one or
        more plots and \PGFPlots{} applies axis scaling, computes any logarithms
        and axis ticks and draws the plots. It supports line plots, scatter
        plots, piecewise constant plots, bar plots, area plots, mesh and surface
        plots, patch plots, contour plots, quiver plots, histogram plots, box
        plots, polar axes, ternary diagrams, smith charts and some more. It is
        based on Till Tantau's package \PGF{}/\Tikz{}.
    \end{minipage}

}%

\expandafter\date\expandafter{\@date\\[2cm]
    \abstractsmuggle
}%
\makeatother

%\includeonly{pgfplots.reference}


\begin{document}

\def\plotcoords{%
\addplot coordinates {
(5,8.312e-02)    (17,2.547e-02)   (49,7.407e-03)
(129,2.102e-03)  (321,5.874e-04)  (769,1.623e-04)
(1793,4.442e-05) (4097,1.207e-05) (9217,3.261e-06)
};

\addplot coordinates{
(7,8.472e-02)    (31,3.044e-02)    (111,1.022e-02)
(351,3.303e-03)  (1023,1.039e-03)  (2815,3.196e-04)
(7423,9.658e-05) (18943,2.873e-05) (47103,8.437e-06)
};

\addplot coordinates{
(9,7.881e-02)     (49,3.243e-02)    (209,1.232e-02)
(769,4.454e-03)   (2561,1.551e-03)  (7937,5.236e-04)
(23297,1.723e-04) (65537,5.545e-05) (178177,1.751e-05)
};

\addplot coordinates{
(11,6.887e-02)    (71,3.177e-02)     (351,1.341e-02)
(1471,5.334e-03)  (5503,2.027e-03)   (18943,7.415e-04)
(61183,2.628e-04) (187903,9.063e-05) (553983,3.053e-05)
};

\addplot coordinates{
(13,5.755e-02)     (97,2.925e-02)     (545,1.351e-02)
(2561,5.842e-03)   (10625,2.397e-03)  (40193,9.414e-04)
(141569,3.564e-04) (471041,1.308e-04)
(1496065,4.670e-05)
};
}%


\bookmaketitle
\tableofcontents
\include{pgfplots.title_abstract_intro}
\include{pgfplots.preliminaries}
\include{pgfplots.intro}
\include{pgfplots.reference}
\include{pgfplots.libs}
\include{pgfplots.resources}
\include{pgfplots.importexport}
\include{pgfplots.basic.reference}

\printindex

\bibliographystyle{abbrv} %gerapali} %gerabbrv} %gerunsrt.bst} %gerabbrv}% gerplain}
\nocite{pgfplotstable}
\nocite{programmingnotes}
\bibliography{pgfplots}
\end{document}

% -----------------------------------------------------------------------------
% For Stefan Pinnow as reminder on what to look for when editing the manual
% -----------------------------------------------------------------------------
% There should be no line breaks in the following environments
% - |...|
% - \declareandlabel{...}
% - \verbpdfref{...}
% If "MakeTikzPictures" isn't running through check one of the externalized
% LOG files what is the cause of that.
% -----------------------------------------------------------------------------


%%%%%%%%%%%%%%%%%%%%%%%%%%%%%%%%%%%%%%%%%%%%%%%%%%%%%%%%%%%%%%%%%%%%%%%%%%%%%
%
% Package pgfplots.sty documentation.
%
% Copyright 2007/2008 by Christian Feuersaenger.
%
% This program is free software: you can redistribute it and/or modify
% it under the terms of the GNU General Public License as published by
% the Free Software Foundation, either version 3 of the License, or
% (at your option) any later version.
%
% This program is distributed in the hope that it will be useful,
% but WITHOUT ANY WARRANTY; without even the implied warranty of
% MERCHANTABILITY or FITNESS FOR A PARTICULAR PURPOSE.  See the
% GNU General Public License for more details.
%
% You should have received a copy of the GNU General Public License
% along with this program.  If not, see <http://www.gnu.org/licenses/>.
%
%
%%%%%%%%%%%%%%%%%%%%%%%%%%%%%%%%%%%%%%%%%%%%%%%%%%%%%%%%%%%%%%%%%%%%%%%%%%%%%
% SEE pgfplots-macros.tex as well!
%\pdfminorversion=5 % to allow compression
%\pdfobjcompresslevel=2
\documentclass[a4paper,openany]{book}

\let\bookmaketitle=\maketitle

% -----------------------------------
% this here is from ltxdoc:
\usepackage{doc}[=v2]
\AtBeginDocument{\MakeShortVerb{\|}}
%\setlength{\textwidth}{355pt}
%\addtolength\marginparwidth{30pt}
%\addtolength\oddsidemargin{20pt}
%\addtolength\evensidemargin{20pt}
\def\cmd#1{\cs{\expandafter\cmd@to@cs\string#1}}
\def\cmd@to@cs#1#2{\char\number`#2\relax}
\DeclareRobustCommand\cs[1]{\texttt{\char`\\#1}}
\providecommand\marg[1]{%
  {\ttfamily\char`\{}\meta{#1}{\ttfamily\char`\}}}
\providecommand\oarg[1]{%
  {\ttfamily[}\meta{#1}{\ttfamily]}}
\providecommand\parg[1]{%
  {\ttfamily(}\meta{#1}{\ttfamily)}}
\raggedbottom

% -----------------------------------
\input{pgfplots.preamble.tex}

\makeatletter
% I want a two-column index just as in pgfmanual styles. This here
% was the best way to get one:
\def\index@prologue{\section*{Index}\addcontentsline{toc}{chapter}{Index}
}
\makeatother

%\RequirePackage[german,english,francais]{babel}

\def\matlabcolormaptext{This colormap is similar to one shipped with Matlab$^\text{\textregistered}$ under a similar name.}

\IfFileExists{tikzlibraryspy.code.tex}{%
\usetikzlibrary{spy}
}{%
    \message{ERROR: tikz SPY library NOT available. The manual will only compile partially.^^J}%
}%

\usepackage{xparse}% for colorbrewer manual

\usetikzlibrary{
    decorations.markings,
    decorations.footprints,
    shapes.arrows,
    matrix,
    positioning,
}

\usepgfplotslibrary{
    fillbetween,
    ternary,
    smithchart,
    patchplots,
    polar,
    colormaps,
    colorbrewer,
    colortol,           % docu not ready yet
}
\pgfqkeys{/codeexample}{%
    every codeexample/.append style={
        /pgfplots/every ternary axis/.append style={
            /pgfplots/legend style={fill=graphicbackground},
        }
    },
    tabsize=4,
}

\pgfplotsmanualenableexternalizationofexpensive

%\usetikzlibrary{external}
%\tikzexternalize[prefix=figures/]{pgfplots}

\title{%
    Manual for Package \PGFPlots{}\\
    {\small 2D/3D Plots in \LaTeX{}, Version \pgfplotsversion}\\
    {\small\url{https://github.com/pgf-tikz/pgfplots}}
    %\\{\small Attention: you are using an unstable development version.}
}

\makeatletter
\long\def\abstractsmuggle{%
    \centering
    \textbf{Abstract}\\[0.5cm]

    \begin{minipage}{12cm}
        \PGFPlots{} draws high-quality function plots in normal or logarithmic
        scaling with a user-friendly interface directly in \TeX{}. The user
        supplies axis labels, legend entries and the plot coordinates for one or
        more plots and \PGFPlots{} applies axis scaling, computes any logarithms
        and axis ticks and draws the plots. It supports line plots, scatter
        plots, piecewise constant plots, bar plots, area plots, mesh and surface
        plots, patch plots, contour plots, quiver plots, histogram plots, box
        plots, polar axes, ternary diagrams, smith charts and some more. It is
        based on Till Tantau's package \PGF{}/\Tikz{}.
    \end{minipage}

}%

\expandafter\date\expandafter{\@date\\[2cm]
    \abstractsmuggle
}%
\makeatother

%\includeonly{pgfplots.reference}


\begin{document}

\def\plotcoords{%
\addplot coordinates {
(5,8.312e-02)    (17,2.547e-02)   (49,7.407e-03)
(129,2.102e-03)  (321,5.874e-04)  (769,1.623e-04)
(1793,4.442e-05) (4097,1.207e-05) (9217,3.261e-06)
};

\addplot coordinates{
(7,8.472e-02)    (31,3.044e-02)    (111,1.022e-02)
(351,3.303e-03)  (1023,1.039e-03)  (2815,3.196e-04)
(7423,9.658e-05) (18943,2.873e-05) (47103,8.437e-06)
};

\addplot coordinates{
(9,7.881e-02)     (49,3.243e-02)    (209,1.232e-02)
(769,4.454e-03)   (2561,1.551e-03)  (7937,5.236e-04)
(23297,1.723e-04) (65537,5.545e-05) (178177,1.751e-05)
};

\addplot coordinates{
(11,6.887e-02)    (71,3.177e-02)     (351,1.341e-02)
(1471,5.334e-03)  (5503,2.027e-03)   (18943,7.415e-04)
(61183,2.628e-04) (187903,9.063e-05) (553983,3.053e-05)
};

\addplot coordinates{
(13,5.755e-02)     (97,2.925e-02)     (545,1.351e-02)
(2561,5.842e-03)   (10625,2.397e-03)  (40193,9.414e-04)
(141569,3.564e-04) (471041,1.308e-04)
(1496065,4.670e-05)
};
}%


\bookmaketitle
\tableofcontents
\include{pgfplots.title_abstract_intro}
\include{pgfplots.preliminaries}
\include{pgfplots.intro}
\include{pgfplots.reference}
\include{pgfplots.libs}
\include{pgfplots.resources}
\include{pgfplots.importexport}
\include{pgfplots.basic.reference}

\printindex

\bibliographystyle{abbrv} %gerapali} %gerabbrv} %gerunsrt.bst} %gerabbrv}% gerplain}
\nocite{pgfplotstable}
\nocite{programmingnotes}
\bibliography{pgfplots}
\end{document}

% -----------------------------------------------------------------------------
% For Stefan Pinnow as reminder on what to look for when editing the manual
% -----------------------------------------------------------------------------
% There should be no line breaks in the following environments
% - |...|
% - \declareandlabel{...}
% - \verbpdfref{...}
% If "MakeTikzPictures" isn't running through check one of the externalized
% LOG files what is the cause of that.
% -----------------------------------------------------------------------------


%%%%%%%%%%%%%%%%%%%%%%%%%%%%%%%%%%%%%%%%%%%%%%%%%%%%%%%%%%%%%%%%%%%%%%%%%%%%%
%
% Package pgfplots.sty documentation.
%
% Copyright 2007/2008 by Christian Feuersaenger.
%
% This program is free software: you can redistribute it and/or modify
% it under the terms of the GNU General Public License as published by
% the Free Software Foundation, either version 3 of the License, or
% (at your option) any later version.
%
% This program is distributed in the hope that it will be useful,
% but WITHOUT ANY WARRANTY; without even the implied warranty of
% MERCHANTABILITY or FITNESS FOR A PARTICULAR PURPOSE.  See the
% GNU General Public License for more details.
%
% You should have received a copy of the GNU General Public License
% along with this program.  If not, see <http://www.gnu.org/licenses/>.
%
%
%%%%%%%%%%%%%%%%%%%%%%%%%%%%%%%%%%%%%%%%%%%%%%%%%%%%%%%%%%%%%%%%%%%%%%%%%%%%%
% SEE pgfplots-macros.tex as well!
%\pdfminorversion=5 % to allow compression
%\pdfobjcompresslevel=2
\documentclass[a4paper,openany]{book}

\let\bookmaketitle=\maketitle

% -----------------------------------
% this here is from ltxdoc:
\usepackage{doc}[=v2]
\AtBeginDocument{\MakeShortVerb{\|}}
%\setlength{\textwidth}{355pt}
%\addtolength\marginparwidth{30pt}
%\addtolength\oddsidemargin{20pt}
%\addtolength\evensidemargin{20pt}
\def\cmd#1{\cs{\expandafter\cmd@to@cs\string#1}}
\def\cmd@to@cs#1#2{\char\number`#2\relax}
\DeclareRobustCommand\cs[1]{\texttt{\char`\\#1}}
\providecommand\marg[1]{%
  {\ttfamily\char`\{}\meta{#1}{\ttfamily\char`\}}}
\providecommand\oarg[1]{%
  {\ttfamily[}\meta{#1}{\ttfamily]}}
\providecommand\parg[1]{%
  {\ttfamily(}\meta{#1}{\ttfamily)}}
\raggedbottom

% -----------------------------------
\input{pgfplots.preamble.tex}

\makeatletter
% I want a two-column index just as in pgfmanual styles. This here
% was the best way to get one:
\def\index@prologue{\section*{Index}\addcontentsline{toc}{chapter}{Index}
}
\makeatother

%\RequirePackage[german,english,francais]{babel}

\def\matlabcolormaptext{This colormap is similar to one shipped with Matlab$^\text{\textregistered}$ under a similar name.}

\IfFileExists{tikzlibraryspy.code.tex}{%
\usetikzlibrary{spy}
}{%
    \message{ERROR: tikz SPY library NOT available. The manual will only compile partially.^^J}%
}%

\usepackage{xparse}% for colorbrewer manual

\usetikzlibrary{
    decorations.markings,
    decorations.footprints,
    shapes.arrows,
    matrix,
    positioning,
}

\usepgfplotslibrary{
    fillbetween,
    ternary,
    smithchart,
    patchplots,
    polar,
    colormaps,
    colorbrewer,
    colortol,           % docu not ready yet
}
\pgfqkeys{/codeexample}{%
    every codeexample/.append style={
        /pgfplots/every ternary axis/.append style={
            /pgfplots/legend style={fill=graphicbackground},
        }
    },
    tabsize=4,
}

\pgfplotsmanualenableexternalizationofexpensive

%\usetikzlibrary{external}
%\tikzexternalize[prefix=figures/]{pgfplots}

\title{%
    Manual for Package \PGFPlots{}\\
    {\small 2D/3D Plots in \LaTeX{}, Version \pgfplotsversion}\\
    {\small\url{https://github.com/pgf-tikz/pgfplots}}
    %\\{\small Attention: you are using an unstable development version.}
}

\makeatletter
\long\def\abstractsmuggle{%
    \centering
    \textbf{Abstract}\\[0.5cm]

    \begin{minipage}{12cm}
        \PGFPlots{} draws high-quality function plots in normal or logarithmic
        scaling with a user-friendly interface directly in \TeX{}. The user
        supplies axis labels, legend entries and the plot coordinates for one or
        more plots and \PGFPlots{} applies axis scaling, computes any logarithms
        and axis ticks and draws the plots. It supports line plots, scatter
        plots, piecewise constant plots, bar plots, area plots, mesh and surface
        plots, patch plots, contour plots, quiver plots, histogram plots, box
        plots, polar axes, ternary diagrams, smith charts and some more. It is
        based on Till Tantau's package \PGF{}/\Tikz{}.
    \end{minipage}

}%

\expandafter\date\expandafter{\@date\\[2cm]
    \abstractsmuggle
}%
\makeatother

%\includeonly{pgfplots.reference}


\begin{document}

\def\plotcoords{%
\addplot coordinates {
(5,8.312e-02)    (17,2.547e-02)   (49,7.407e-03)
(129,2.102e-03)  (321,5.874e-04)  (769,1.623e-04)
(1793,4.442e-05) (4097,1.207e-05) (9217,3.261e-06)
};

\addplot coordinates{
(7,8.472e-02)    (31,3.044e-02)    (111,1.022e-02)
(351,3.303e-03)  (1023,1.039e-03)  (2815,3.196e-04)
(7423,9.658e-05) (18943,2.873e-05) (47103,8.437e-06)
};

\addplot coordinates{
(9,7.881e-02)     (49,3.243e-02)    (209,1.232e-02)
(769,4.454e-03)   (2561,1.551e-03)  (7937,5.236e-04)
(23297,1.723e-04) (65537,5.545e-05) (178177,1.751e-05)
};

\addplot coordinates{
(11,6.887e-02)    (71,3.177e-02)     (351,1.341e-02)
(1471,5.334e-03)  (5503,2.027e-03)   (18943,7.415e-04)
(61183,2.628e-04) (187903,9.063e-05) (553983,3.053e-05)
};

\addplot coordinates{
(13,5.755e-02)     (97,2.925e-02)     (545,1.351e-02)
(2561,5.842e-03)   (10625,2.397e-03)  (40193,9.414e-04)
(141569,3.564e-04) (471041,1.308e-04)
(1496065,4.670e-05)
};
}%


\bookmaketitle
\tableofcontents
\include{pgfplots.title_abstract_intro}
\include{pgfplots.preliminaries}
\include{pgfplots.intro}
\include{pgfplots.reference}
\include{pgfplots.libs}
\include{pgfplots.resources}
\include{pgfplots.importexport}
\include{pgfplots.basic.reference}

\printindex

\bibliographystyle{abbrv} %gerapali} %gerabbrv} %gerunsrt.bst} %gerabbrv}% gerplain}
\nocite{pgfplotstable}
\nocite{programmingnotes}
\bibliography{pgfplots}
\end{document}

% -----------------------------------------------------------------------------
% For Stefan Pinnow as reminder on what to look for when editing the manual
% -----------------------------------------------------------------------------
% There should be no line breaks in the following environments
% - |...|
% - \declareandlabel{...}
% - \verbpdfref{...}
% If "MakeTikzPictures" isn't running through check one of the externalized
% LOG files what is the cause of that.
% -----------------------------------------------------------------------------

\include{pgfplots.basic.reference}

\printindex

\bibliographystyle{abbrv} %gerapali} %gerabbrv} %gerunsrt.bst} %gerabbrv}% gerplain}
\nocite{pgfplotstable}
\nocite{programmingnotes}
\bibliography{pgfplots}
\end{document}

% -----------------------------------------------------------------------------
% For Stefan Pinnow as reminder on what to look for when editing the manual
% -----------------------------------------------------------------------------
% There should be no line breaks in the following environments
% - |...|
% - \declareandlabel{...}
% - \verbpdfref{...}
% If "MakeTikzPictures" isn't running through check one of the externalized
% LOG files what is the cause of that.
% -----------------------------------------------------------------------------


%%%%%%%%%%%%%%%%%%%%%%%%%%%%%%%%%%%%%%%%%%%%%%%%%%%%%%%%%%%%%%%%%%%%%%%%%%%%%
%
% Package pgfplots.sty documentation.
%
% Copyright 2007/2008 by Christian Feuersaenger.
%
% This program is free software: you can redistribute it and/or modify
% it under the terms of the GNU General Public License as published by
% the Free Software Foundation, either version 3 of the License, or
% (at your option) any later version.
%
% This program is distributed in the hope that it will be useful,
% but WITHOUT ANY WARRANTY; without even the implied warranty of
% MERCHANTABILITY or FITNESS FOR A PARTICULAR PURPOSE.  See the
% GNU General Public License for more details.
%
% You should have received a copy of the GNU General Public License
% along with this program.  If not, see <http://www.gnu.org/licenses/>.
%
%
%%%%%%%%%%%%%%%%%%%%%%%%%%%%%%%%%%%%%%%%%%%%%%%%%%%%%%%%%%%%%%%%%%%%%%%%%%%%%
% SEE pgfplots-macros.tex as well!
%\pdfminorversion=5 % to allow compression
%\pdfobjcompresslevel=2
\documentclass[a4paper,openany]{book}

\let\bookmaketitle=\maketitle

% -----------------------------------
% this here is from ltxdoc:
\usepackage{doc}[=v2]
\AtBeginDocument{\MakeShortVerb{\|}}
%\setlength{\textwidth}{355pt}
%\addtolength\marginparwidth{30pt}
%\addtolength\oddsidemargin{20pt}
%\addtolength\evensidemargin{20pt}
\def\cmd#1{\cs{\expandafter\cmd@to@cs\string#1}}
\def\cmd@to@cs#1#2{\char\number`#2\relax}
\DeclareRobustCommand\cs[1]{\texttt{\char`\\#1}}
\providecommand\marg[1]{%
  {\ttfamily\char`\{}\meta{#1}{\ttfamily\char`\}}}
\providecommand\oarg[1]{%
  {\ttfamily[}\meta{#1}{\ttfamily]}}
\providecommand\parg[1]{%
  {\ttfamily(}\meta{#1}{\ttfamily)}}
\raggedbottom

% -----------------------------------
\input{pgfplots.preamble.tex}

\makeatletter
% I want a two-column index just as in pgfmanual styles. This here
% was the best way to get one:
\def\index@prologue{\section*{Index}\addcontentsline{toc}{chapter}{Index}
}
\makeatother

%\RequirePackage[german,english,francais]{babel}

\def\matlabcolormaptext{This colormap is similar to one shipped with Matlab$^\text{\textregistered}$ under a similar name.}

\IfFileExists{tikzlibraryspy.code.tex}{%
\usetikzlibrary{spy}
}{%
    \message{ERROR: tikz SPY library NOT available. The manual will only compile partially.^^J}%
}%

\usepackage{xparse}% for colorbrewer manual

\usetikzlibrary{
    decorations.markings,
    decorations.footprints,
    shapes.arrows,
    matrix,
    positioning,
}

\usepgfplotslibrary{
    fillbetween,
    ternary,
    smithchart,
    patchplots,
    polar,
    colormaps,
    colorbrewer,
    colortol,           % docu not ready yet
}
\pgfqkeys{/codeexample}{%
    every codeexample/.append style={
        /pgfplots/every ternary axis/.append style={
            /pgfplots/legend style={fill=graphicbackground},
        }
    },
    tabsize=4,
}

\pgfplotsmanualenableexternalizationofexpensive

%\usetikzlibrary{external}
%\tikzexternalize[prefix=figures/]{pgfplots}

\title{%
    Manual for Package \PGFPlots{}\\
    {\small 2D/3D Plots in \LaTeX{}, Version \pgfplotsversion}\\
    {\small\url{https://github.com/pgf-tikz/pgfplots}}
    %\\{\small Attention: you are using an unstable development version.}
}

\makeatletter
\long\def\abstractsmuggle{%
    \centering
    \textbf{Abstract}\\[0.5cm]

    \begin{minipage}{12cm}
        \PGFPlots{} draws high-quality function plots in normal or logarithmic
        scaling with a user-friendly interface directly in \TeX{}. The user
        supplies axis labels, legend entries and the plot coordinates for one or
        more plots and \PGFPlots{} applies axis scaling, computes any logarithms
        and axis ticks and draws the plots. It supports line plots, scatter
        plots, piecewise constant plots, bar plots, area plots, mesh and surface
        plots, patch plots, contour plots, quiver plots, histogram plots, box
        plots, polar axes, ternary diagrams, smith charts and some more. It is
        based on Till Tantau's package \PGF{}/\Tikz{}.
    \end{minipage}

}%

\expandafter\date\expandafter{\@date\\[2cm]
    \abstractsmuggle
}%
\makeatother

%\includeonly{pgfplots.reference}


\begin{document}

\def\plotcoords{%
\addplot coordinates {
(5,8.312e-02)    (17,2.547e-02)   (49,7.407e-03)
(129,2.102e-03)  (321,5.874e-04)  (769,1.623e-04)
(1793,4.442e-05) (4097,1.207e-05) (9217,3.261e-06)
};

\addplot coordinates{
(7,8.472e-02)    (31,3.044e-02)    (111,1.022e-02)
(351,3.303e-03)  (1023,1.039e-03)  (2815,3.196e-04)
(7423,9.658e-05) (18943,2.873e-05) (47103,8.437e-06)
};

\addplot coordinates{
(9,7.881e-02)     (49,3.243e-02)    (209,1.232e-02)
(769,4.454e-03)   (2561,1.551e-03)  (7937,5.236e-04)
(23297,1.723e-04) (65537,5.545e-05) (178177,1.751e-05)
};

\addplot coordinates{
(11,6.887e-02)    (71,3.177e-02)     (351,1.341e-02)
(1471,5.334e-03)  (5503,2.027e-03)   (18943,7.415e-04)
(61183,2.628e-04) (187903,9.063e-05) (553983,3.053e-05)
};

\addplot coordinates{
(13,5.755e-02)     (97,2.925e-02)     (545,1.351e-02)
(2561,5.842e-03)   (10625,2.397e-03)  (40193,9.414e-04)
(141569,3.564e-04) (471041,1.308e-04)
(1496065,4.670e-05)
};
}%


\bookmaketitle
\tableofcontents

%%%%%%%%%%%%%%%%%%%%%%%%%%%%%%%%%%%%%%%%%%%%%%%%%%%%%%%%%%%%%%%%%%%%%%%%%%%%%
%
% Package pgfplots.sty documentation.
%
% Copyright 2007/2008 by Christian Feuersaenger.
%
% This program is free software: you can redistribute it and/or modify
% it under the terms of the GNU General Public License as published by
% the Free Software Foundation, either version 3 of the License, or
% (at your option) any later version.
%
% This program is distributed in the hope that it will be useful,
% but WITHOUT ANY WARRANTY; without even the implied warranty of
% MERCHANTABILITY or FITNESS FOR A PARTICULAR PURPOSE.  See the
% GNU General Public License for more details.
%
% You should have received a copy of the GNU General Public License
% along with this program.  If not, see <http://www.gnu.org/licenses/>.
%
%
%%%%%%%%%%%%%%%%%%%%%%%%%%%%%%%%%%%%%%%%%%%%%%%%%%%%%%%%%%%%%%%%%%%%%%%%%%%%%
% SEE pgfplots-macros.tex as well!
%\pdfminorversion=5 % to allow compression
%\pdfobjcompresslevel=2
\documentclass[a4paper,openany]{book}

\let\bookmaketitle=\maketitle

% -----------------------------------
% this here is from ltxdoc:
\usepackage{doc}[=v2]
\AtBeginDocument{\MakeShortVerb{\|}}
%\setlength{\textwidth}{355pt}
%\addtolength\marginparwidth{30pt}
%\addtolength\oddsidemargin{20pt}
%\addtolength\evensidemargin{20pt}
\def\cmd#1{\cs{\expandafter\cmd@to@cs\string#1}}
\def\cmd@to@cs#1#2{\char\number`#2\relax}
\DeclareRobustCommand\cs[1]{\texttt{\char`\\#1}}
\providecommand\marg[1]{%
  {\ttfamily\char`\{}\meta{#1}{\ttfamily\char`\}}}
\providecommand\oarg[1]{%
  {\ttfamily[}\meta{#1}{\ttfamily]}}
\providecommand\parg[1]{%
  {\ttfamily(}\meta{#1}{\ttfamily)}}
\raggedbottom

% -----------------------------------
\input{pgfplots.preamble.tex}

\makeatletter
% I want a two-column index just as in pgfmanual styles. This here
% was the best way to get one:
\def\index@prologue{\section*{Index}\addcontentsline{toc}{chapter}{Index}
}
\makeatother

%\RequirePackage[german,english,francais]{babel}

\def\matlabcolormaptext{This colormap is similar to one shipped with Matlab$^\text{\textregistered}$ under a similar name.}

\IfFileExists{tikzlibraryspy.code.tex}{%
\usetikzlibrary{spy}
}{%
    \message{ERROR: tikz SPY library NOT available. The manual will only compile partially.^^J}%
}%

\usepackage{xparse}% for colorbrewer manual

\usetikzlibrary{
    decorations.markings,
    decorations.footprints,
    shapes.arrows,
    matrix,
    positioning,
}

\usepgfplotslibrary{
    fillbetween,
    ternary,
    smithchart,
    patchplots,
    polar,
    colormaps,
    colorbrewer,
    colortol,           % docu not ready yet
}
\pgfqkeys{/codeexample}{%
    every codeexample/.append style={
        /pgfplots/every ternary axis/.append style={
            /pgfplots/legend style={fill=graphicbackground},
        }
    },
    tabsize=4,
}

\pgfplotsmanualenableexternalizationofexpensive

%\usetikzlibrary{external}
%\tikzexternalize[prefix=figures/]{pgfplots}

\title{%
    Manual for Package \PGFPlots{}\\
    {\small 2D/3D Plots in \LaTeX{}, Version \pgfplotsversion}\\
    {\small\url{https://github.com/pgf-tikz/pgfplots}}
    %\\{\small Attention: you are using an unstable development version.}
}

\makeatletter
\long\def\abstractsmuggle{%
    \centering
    \textbf{Abstract}\\[0.5cm]

    \begin{minipage}{12cm}
        \PGFPlots{} draws high-quality function plots in normal or logarithmic
        scaling with a user-friendly interface directly in \TeX{}. The user
        supplies axis labels, legend entries and the plot coordinates for one or
        more plots and \PGFPlots{} applies axis scaling, computes any logarithms
        and axis ticks and draws the plots. It supports line plots, scatter
        plots, piecewise constant plots, bar plots, area plots, mesh and surface
        plots, patch plots, contour plots, quiver plots, histogram plots, box
        plots, polar axes, ternary diagrams, smith charts and some more. It is
        based on Till Tantau's package \PGF{}/\Tikz{}.
    \end{minipage}

}%

\expandafter\date\expandafter{\@date\\[2cm]
    \abstractsmuggle
}%
\makeatother

%\includeonly{pgfplots.reference}


\begin{document}

\def\plotcoords{%
\addplot coordinates {
(5,8.312e-02)    (17,2.547e-02)   (49,7.407e-03)
(129,2.102e-03)  (321,5.874e-04)  (769,1.623e-04)
(1793,4.442e-05) (4097,1.207e-05) (9217,3.261e-06)
};

\addplot coordinates{
(7,8.472e-02)    (31,3.044e-02)    (111,1.022e-02)
(351,3.303e-03)  (1023,1.039e-03)  (2815,3.196e-04)
(7423,9.658e-05) (18943,2.873e-05) (47103,8.437e-06)
};

\addplot coordinates{
(9,7.881e-02)     (49,3.243e-02)    (209,1.232e-02)
(769,4.454e-03)   (2561,1.551e-03)  (7937,5.236e-04)
(23297,1.723e-04) (65537,5.545e-05) (178177,1.751e-05)
};

\addplot coordinates{
(11,6.887e-02)    (71,3.177e-02)     (351,1.341e-02)
(1471,5.334e-03)  (5503,2.027e-03)   (18943,7.415e-04)
(61183,2.628e-04) (187903,9.063e-05) (553983,3.053e-05)
};

\addplot coordinates{
(13,5.755e-02)     (97,2.925e-02)     (545,1.351e-02)
(2561,5.842e-03)   (10625,2.397e-03)  (40193,9.414e-04)
(141569,3.564e-04) (471041,1.308e-04)
(1496065,4.670e-05)
};
}%


\bookmaketitle
\tableofcontents
\include{pgfplots.title_abstract_intro}
\include{pgfplots.preliminaries}
\include{pgfplots.intro}
\include{pgfplots.reference}
\include{pgfplots.libs}
\include{pgfplots.resources}
\include{pgfplots.importexport}
\include{pgfplots.basic.reference}

\printindex

\bibliographystyle{abbrv} %gerapali} %gerabbrv} %gerunsrt.bst} %gerabbrv}% gerplain}
\nocite{pgfplotstable}
\nocite{programmingnotes}
\bibliography{pgfplots}
\end{document}

% -----------------------------------------------------------------------------
% For Stefan Pinnow as reminder on what to look for when editing the manual
% -----------------------------------------------------------------------------
% There should be no line breaks in the following environments
% - |...|
% - \declareandlabel{...}
% - \verbpdfref{...}
% If "MakeTikzPictures" isn't running through check one of the externalized
% LOG files what is the cause of that.
% -----------------------------------------------------------------------------


%%%%%%%%%%%%%%%%%%%%%%%%%%%%%%%%%%%%%%%%%%%%%%%%%%%%%%%%%%%%%%%%%%%%%%%%%%%%%
%
% Package pgfplots.sty documentation.
%
% Copyright 2007/2008 by Christian Feuersaenger.
%
% This program is free software: you can redistribute it and/or modify
% it under the terms of the GNU General Public License as published by
% the Free Software Foundation, either version 3 of the License, or
% (at your option) any later version.
%
% This program is distributed in the hope that it will be useful,
% but WITHOUT ANY WARRANTY; without even the implied warranty of
% MERCHANTABILITY or FITNESS FOR A PARTICULAR PURPOSE.  See the
% GNU General Public License for more details.
%
% You should have received a copy of the GNU General Public License
% along with this program.  If not, see <http://www.gnu.org/licenses/>.
%
%
%%%%%%%%%%%%%%%%%%%%%%%%%%%%%%%%%%%%%%%%%%%%%%%%%%%%%%%%%%%%%%%%%%%%%%%%%%%%%
% SEE pgfplots-macros.tex as well!
%\pdfminorversion=5 % to allow compression
%\pdfobjcompresslevel=2
\documentclass[a4paper,openany]{book}

\let\bookmaketitle=\maketitle

% -----------------------------------
% this here is from ltxdoc:
\usepackage{doc}[=v2]
\AtBeginDocument{\MakeShortVerb{\|}}
%\setlength{\textwidth}{355pt}
%\addtolength\marginparwidth{30pt}
%\addtolength\oddsidemargin{20pt}
%\addtolength\evensidemargin{20pt}
\def\cmd#1{\cs{\expandafter\cmd@to@cs\string#1}}
\def\cmd@to@cs#1#2{\char\number`#2\relax}
\DeclareRobustCommand\cs[1]{\texttt{\char`\\#1}}
\providecommand\marg[1]{%
  {\ttfamily\char`\{}\meta{#1}{\ttfamily\char`\}}}
\providecommand\oarg[1]{%
  {\ttfamily[}\meta{#1}{\ttfamily]}}
\providecommand\parg[1]{%
  {\ttfamily(}\meta{#1}{\ttfamily)}}
\raggedbottom

% -----------------------------------
\input{pgfplots.preamble.tex}

\makeatletter
% I want a two-column index just as in pgfmanual styles. This here
% was the best way to get one:
\def\index@prologue{\section*{Index}\addcontentsline{toc}{chapter}{Index}
}
\makeatother

%\RequirePackage[german,english,francais]{babel}

\def\matlabcolormaptext{This colormap is similar to one shipped with Matlab$^\text{\textregistered}$ under a similar name.}

\IfFileExists{tikzlibraryspy.code.tex}{%
\usetikzlibrary{spy}
}{%
    \message{ERROR: tikz SPY library NOT available. The manual will only compile partially.^^J}%
}%

\usepackage{xparse}% for colorbrewer manual

\usetikzlibrary{
    decorations.markings,
    decorations.footprints,
    shapes.arrows,
    matrix,
    positioning,
}

\usepgfplotslibrary{
    fillbetween,
    ternary,
    smithchart,
    patchplots,
    polar,
    colormaps,
    colorbrewer,
    colortol,           % docu not ready yet
}
\pgfqkeys{/codeexample}{%
    every codeexample/.append style={
        /pgfplots/every ternary axis/.append style={
            /pgfplots/legend style={fill=graphicbackground},
        }
    },
    tabsize=4,
}

\pgfplotsmanualenableexternalizationofexpensive

%\usetikzlibrary{external}
%\tikzexternalize[prefix=figures/]{pgfplots}

\title{%
    Manual for Package \PGFPlots{}\\
    {\small 2D/3D Plots in \LaTeX{}, Version \pgfplotsversion}\\
    {\small\url{https://github.com/pgf-tikz/pgfplots}}
    %\\{\small Attention: you are using an unstable development version.}
}

\makeatletter
\long\def\abstractsmuggle{%
    \centering
    \textbf{Abstract}\\[0.5cm]

    \begin{minipage}{12cm}
        \PGFPlots{} draws high-quality function plots in normal or logarithmic
        scaling with a user-friendly interface directly in \TeX{}. The user
        supplies axis labels, legend entries and the plot coordinates for one or
        more plots and \PGFPlots{} applies axis scaling, computes any logarithms
        and axis ticks and draws the plots. It supports line plots, scatter
        plots, piecewise constant plots, bar plots, area plots, mesh and surface
        plots, patch plots, contour plots, quiver plots, histogram plots, box
        plots, polar axes, ternary diagrams, smith charts and some more. It is
        based on Till Tantau's package \PGF{}/\Tikz{}.
    \end{minipage}

}%

\expandafter\date\expandafter{\@date\\[2cm]
    \abstractsmuggle
}%
\makeatother

%\includeonly{pgfplots.reference}


\begin{document}

\def\plotcoords{%
\addplot coordinates {
(5,8.312e-02)    (17,2.547e-02)   (49,7.407e-03)
(129,2.102e-03)  (321,5.874e-04)  (769,1.623e-04)
(1793,4.442e-05) (4097,1.207e-05) (9217,3.261e-06)
};

\addplot coordinates{
(7,8.472e-02)    (31,3.044e-02)    (111,1.022e-02)
(351,3.303e-03)  (1023,1.039e-03)  (2815,3.196e-04)
(7423,9.658e-05) (18943,2.873e-05) (47103,8.437e-06)
};

\addplot coordinates{
(9,7.881e-02)     (49,3.243e-02)    (209,1.232e-02)
(769,4.454e-03)   (2561,1.551e-03)  (7937,5.236e-04)
(23297,1.723e-04) (65537,5.545e-05) (178177,1.751e-05)
};

\addplot coordinates{
(11,6.887e-02)    (71,3.177e-02)     (351,1.341e-02)
(1471,5.334e-03)  (5503,2.027e-03)   (18943,7.415e-04)
(61183,2.628e-04) (187903,9.063e-05) (553983,3.053e-05)
};

\addplot coordinates{
(13,5.755e-02)     (97,2.925e-02)     (545,1.351e-02)
(2561,5.842e-03)   (10625,2.397e-03)  (40193,9.414e-04)
(141569,3.564e-04) (471041,1.308e-04)
(1496065,4.670e-05)
};
}%


\bookmaketitle
\tableofcontents
\include{pgfplots.title_abstract_intro}
\include{pgfplots.preliminaries}
\include{pgfplots.intro}
\include{pgfplots.reference}
\include{pgfplots.libs}
\include{pgfplots.resources}
\include{pgfplots.importexport}
\include{pgfplots.basic.reference}

\printindex

\bibliographystyle{abbrv} %gerapali} %gerabbrv} %gerunsrt.bst} %gerabbrv}% gerplain}
\nocite{pgfplotstable}
\nocite{programmingnotes}
\bibliography{pgfplots}
\end{document}

% -----------------------------------------------------------------------------
% For Stefan Pinnow as reminder on what to look for when editing the manual
% -----------------------------------------------------------------------------
% There should be no line breaks in the following environments
% - |...|
% - \declareandlabel{...}
% - \verbpdfref{...}
% If "MakeTikzPictures" isn't running through check one of the externalized
% LOG files what is the cause of that.
% -----------------------------------------------------------------------------


%%%%%%%%%%%%%%%%%%%%%%%%%%%%%%%%%%%%%%%%%%%%%%%%%%%%%%%%%%%%%%%%%%%%%%%%%%%%%
%
% Package pgfplots.sty documentation.
%
% Copyright 2007/2008 by Christian Feuersaenger.
%
% This program is free software: you can redistribute it and/or modify
% it under the terms of the GNU General Public License as published by
% the Free Software Foundation, either version 3 of the License, or
% (at your option) any later version.
%
% This program is distributed in the hope that it will be useful,
% but WITHOUT ANY WARRANTY; without even the implied warranty of
% MERCHANTABILITY or FITNESS FOR A PARTICULAR PURPOSE.  See the
% GNU General Public License for more details.
%
% You should have received a copy of the GNU General Public License
% along with this program.  If not, see <http://www.gnu.org/licenses/>.
%
%
%%%%%%%%%%%%%%%%%%%%%%%%%%%%%%%%%%%%%%%%%%%%%%%%%%%%%%%%%%%%%%%%%%%%%%%%%%%%%
% SEE pgfplots-macros.tex as well!
%\pdfminorversion=5 % to allow compression
%\pdfobjcompresslevel=2
\documentclass[a4paper,openany]{book}

\let\bookmaketitle=\maketitle

% -----------------------------------
% this here is from ltxdoc:
\usepackage{doc}[=v2]
\AtBeginDocument{\MakeShortVerb{\|}}
%\setlength{\textwidth}{355pt}
%\addtolength\marginparwidth{30pt}
%\addtolength\oddsidemargin{20pt}
%\addtolength\evensidemargin{20pt}
\def\cmd#1{\cs{\expandafter\cmd@to@cs\string#1}}
\def\cmd@to@cs#1#2{\char\number`#2\relax}
\DeclareRobustCommand\cs[1]{\texttt{\char`\\#1}}
\providecommand\marg[1]{%
  {\ttfamily\char`\{}\meta{#1}{\ttfamily\char`\}}}
\providecommand\oarg[1]{%
  {\ttfamily[}\meta{#1}{\ttfamily]}}
\providecommand\parg[1]{%
  {\ttfamily(}\meta{#1}{\ttfamily)}}
\raggedbottom

% -----------------------------------
\input{pgfplots.preamble.tex}

\makeatletter
% I want a two-column index just as in pgfmanual styles. This here
% was the best way to get one:
\def\index@prologue{\section*{Index}\addcontentsline{toc}{chapter}{Index}
}
\makeatother

%\RequirePackage[german,english,francais]{babel}

\def\matlabcolormaptext{This colormap is similar to one shipped with Matlab$^\text{\textregistered}$ under a similar name.}

\IfFileExists{tikzlibraryspy.code.tex}{%
\usetikzlibrary{spy}
}{%
    \message{ERROR: tikz SPY library NOT available. The manual will only compile partially.^^J}%
}%

\usepackage{xparse}% for colorbrewer manual

\usetikzlibrary{
    decorations.markings,
    decorations.footprints,
    shapes.arrows,
    matrix,
    positioning,
}

\usepgfplotslibrary{
    fillbetween,
    ternary,
    smithchart,
    patchplots,
    polar,
    colormaps,
    colorbrewer,
    colortol,           % docu not ready yet
}
\pgfqkeys{/codeexample}{%
    every codeexample/.append style={
        /pgfplots/every ternary axis/.append style={
            /pgfplots/legend style={fill=graphicbackground},
        }
    },
    tabsize=4,
}

\pgfplotsmanualenableexternalizationofexpensive

%\usetikzlibrary{external}
%\tikzexternalize[prefix=figures/]{pgfplots}

\title{%
    Manual for Package \PGFPlots{}\\
    {\small 2D/3D Plots in \LaTeX{}, Version \pgfplotsversion}\\
    {\small\url{https://github.com/pgf-tikz/pgfplots}}
    %\\{\small Attention: you are using an unstable development version.}
}

\makeatletter
\long\def\abstractsmuggle{%
    \centering
    \textbf{Abstract}\\[0.5cm]

    \begin{minipage}{12cm}
        \PGFPlots{} draws high-quality function plots in normal or logarithmic
        scaling with a user-friendly interface directly in \TeX{}. The user
        supplies axis labels, legend entries and the plot coordinates for one or
        more plots and \PGFPlots{} applies axis scaling, computes any logarithms
        and axis ticks and draws the plots. It supports line plots, scatter
        plots, piecewise constant plots, bar plots, area plots, mesh and surface
        plots, patch plots, contour plots, quiver plots, histogram plots, box
        plots, polar axes, ternary diagrams, smith charts and some more. It is
        based on Till Tantau's package \PGF{}/\Tikz{}.
    \end{minipage}

}%

\expandafter\date\expandafter{\@date\\[2cm]
    \abstractsmuggle
}%
\makeatother

%\includeonly{pgfplots.reference}


\begin{document}

\def\plotcoords{%
\addplot coordinates {
(5,8.312e-02)    (17,2.547e-02)   (49,7.407e-03)
(129,2.102e-03)  (321,5.874e-04)  (769,1.623e-04)
(1793,4.442e-05) (4097,1.207e-05) (9217,3.261e-06)
};

\addplot coordinates{
(7,8.472e-02)    (31,3.044e-02)    (111,1.022e-02)
(351,3.303e-03)  (1023,1.039e-03)  (2815,3.196e-04)
(7423,9.658e-05) (18943,2.873e-05) (47103,8.437e-06)
};

\addplot coordinates{
(9,7.881e-02)     (49,3.243e-02)    (209,1.232e-02)
(769,4.454e-03)   (2561,1.551e-03)  (7937,5.236e-04)
(23297,1.723e-04) (65537,5.545e-05) (178177,1.751e-05)
};

\addplot coordinates{
(11,6.887e-02)    (71,3.177e-02)     (351,1.341e-02)
(1471,5.334e-03)  (5503,2.027e-03)   (18943,7.415e-04)
(61183,2.628e-04) (187903,9.063e-05) (553983,3.053e-05)
};

\addplot coordinates{
(13,5.755e-02)     (97,2.925e-02)     (545,1.351e-02)
(2561,5.842e-03)   (10625,2.397e-03)  (40193,9.414e-04)
(141569,3.564e-04) (471041,1.308e-04)
(1496065,4.670e-05)
};
}%


\bookmaketitle
\tableofcontents
\include{pgfplots.title_abstract_intro}
\include{pgfplots.preliminaries}
\include{pgfplots.intro}
\include{pgfplots.reference}
\include{pgfplots.libs}
\include{pgfplots.resources}
\include{pgfplots.importexport}
\include{pgfplots.basic.reference}

\printindex

\bibliographystyle{abbrv} %gerapali} %gerabbrv} %gerunsrt.bst} %gerabbrv}% gerplain}
\nocite{pgfplotstable}
\nocite{programmingnotes}
\bibliography{pgfplots}
\end{document}

% -----------------------------------------------------------------------------
% For Stefan Pinnow as reminder on what to look for when editing the manual
% -----------------------------------------------------------------------------
% There should be no line breaks in the following environments
% - |...|
% - \declareandlabel{...}
% - \verbpdfref{...}
% If "MakeTikzPictures" isn't running through check one of the externalized
% LOG files what is the cause of that.
% -----------------------------------------------------------------------------


%%%%%%%%%%%%%%%%%%%%%%%%%%%%%%%%%%%%%%%%%%%%%%%%%%%%%%%%%%%%%%%%%%%%%%%%%%%%%
%
% Package pgfplots.sty documentation.
%
% Copyright 2007/2008 by Christian Feuersaenger.
%
% This program is free software: you can redistribute it and/or modify
% it under the terms of the GNU General Public License as published by
% the Free Software Foundation, either version 3 of the License, or
% (at your option) any later version.
%
% This program is distributed in the hope that it will be useful,
% but WITHOUT ANY WARRANTY; without even the implied warranty of
% MERCHANTABILITY or FITNESS FOR A PARTICULAR PURPOSE.  See the
% GNU General Public License for more details.
%
% You should have received a copy of the GNU General Public License
% along with this program.  If not, see <http://www.gnu.org/licenses/>.
%
%
%%%%%%%%%%%%%%%%%%%%%%%%%%%%%%%%%%%%%%%%%%%%%%%%%%%%%%%%%%%%%%%%%%%%%%%%%%%%%
% SEE pgfplots-macros.tex as well!
%\pdfminorversion=5 % to allow compression
%\pdfobjcompresslevel=2
\documentclass[a4paper,openany]{book}

\let\bookmaketitle=\maketitle

% -----------------------------------
% this here is from ltxdoc:
\usepackage{doc}[=v2]
\AtBeginDocument{\MakeShortVerb{\|}}
%\setlength{\textwidth}{355pt}
%\addtolength\marginparwidth{30pt}
%\addtolength\oddsidemargin{20pt}
%\addtolength\evensidemargin{20pt}
\def\cmd#1{\cs{\expandafter\cmd@to@cs\string#1}}
\def\cmd@to@cs#1#2{\char\number`#2\relax}
\DeclareRobustCommand\cs[1]{\texttt{\char`\\#1}}
\providecommand\marg[1]{%
  {\ttfamily\char`\{}\meta{#1}{\ttfamily\char`\}}}
\providecommand\oarg[1]{%
  {\ttfamily[}\meta{#1}{\ttfamily]}}
\providecommand\parg[1]{%
  {\ttfamily(}\meta{#1}{\ttfamily)}}
\raggedbottom

% -----------------------------------
\input{pgfplots.preamble.tex}

\makeatletter
% I want a two-column index just as in pgfmanual styles. This here
% was the best way to get one:
\def\index@prologue{\section*{Index}\addcontentsline{toc}{chapter}{Index}
}
\makeatother

%\RequirePackage[german,english,francais]{babel}

\def\matlabcolormaptext{This colormap is similar to one shipped with Matlab$^\text{\textregistered}$ under a similar name.}

\IfFileExists{tikzlibraryspy.code.tex}{%
\usetikzlibrary{spy}
}{%
    \message{ERROR: tikz SPY library NOT available. The manual will only compile partially.^^J}%
}%

\usepackage{xparse}% for colorbrewer manual

\usetikzlibrary{
    decorations.markings,
    decorations.footprints,
    shapes.arrows,
    matrix,
    positioning,
}

\usepgfplotslibrary{
    fillbetween,
    ternary,
    smithchart,
    patchplots,
    polar,
    colormaps,
    colorbrewer,
    colortol,           % docu not ready yet
}
\pgfqkeys{/codeexample}{%
    every codeexample/.append style={
        /pgfplots/every ternary axis/.append style={
            /pgfplots/legend style={fill=graphicbackground},
        }
    },
    tabsize=4,
}

\pgfplotsmanualenableexternalizationofexpensive

%\usetikzlibrary{external}
%\tikzexternalize[prefix=figures/]{pgfplots}

\title{%
    Manual for Package \PGFPlots{}\\
    {\small 2D/3D Plots in \LaTeX{}, Version \pgfplotsversion}\\
    {\small\url{https://github.com/pgf-tikz/pgfplots}}
    %\\{\small Attention: you are using an unstable development version.}
}

\makeatletter
\long\def\abstractsmuggle{%
    \centering
    \textbf{Abstract}\\[0.5cm]

    \begin{minipage}{12cm}
        \PGFPlots{} draws high-quality function plots in normal or logarithmic
        scaling with a user-friendly interface directly in \TeX{}. The user
        supplies axis labels, legend entries and the plot coordinates for one or
        more plots and \PGFPlots{} applies axis scaling, computes any logarithms
        and axis ticks and draws the plots. It supports line plots, scatter
        plots, piecewise constant plots, bar plots, area plots, mesh and surface
        plots, patch plots, contour plots, quiver plots, histogram plots, box
        plots, polar axes, ternary diagrams, smith charts and some more. It is
        based on Till Tantau's package \PGF{}/\Tikz{}.
    \end{minipage}

}%

\expandafter\date\expandafter{\@date\\[2cm]
    \abstractsmuggle
}%
\makeatother

%\includeonly{pgfplots.reference}


\begin{document}

\def\plotcoords{%
\addplot coordinates {
(5,8.312e-02)    (17,2.547e-02)   (49,7.407e-03)
(129,2.102e-03)  (321,5.874e-04)  (769,1.623e-04)
(1793,4.442e-05) (4097,1.207e-05) (9217,3.261e-06)
};

\addplot coordinates{
(7,8.472e-02)    (31,3.044e-02)    (111,1.022e-02)
(351,3.303e-03)  (1023,1.039e-03)  (2815,3.196e-04)
(7423,9.658e-05) (18943,2.873e-05) (47103,8.437e-06)
};

\addplot coordinates{
(9,7.881e-02)     (49,3.243e-02)    (209,1.232e-02)
(769,4.454e-03)   (2561,1.551e-03)  (7937,5.236e-04)
(23297,1.723e-04) (65537,5.545e-05) (178177,1.751e-05)
};

\addplot coordinates{
(11,6.887e-02)    (71,3.177e-02)     (351,1.341e-02)
(1471,5.334e-03)  (5503,2.027e-03)   (18943,7.415e-04)
(61183,2.628e-04) (187903,9.063e-05) (553983,3.053e-05)
};

\addplot coordinates{
(13,5.755e-02)     (97,2.925e-02)     (545,1.351e-02)
(2561,5.842e-03)   (10625,2.397e-03)  (40193,9.414e-04)
(141569,3.564e-04) (471041,1.308e-04)
(1496065,4.670e-05)
};
}%


\bookmaketitle
\tableofcontents
\include{pgfplots.title_abstract_intro}
\include{pgfplots.preliminaries}
\include{pgfplots.intro}
\include{pgfplots.reference}
\include{pgfplots.libs}
\include{pgfplots.resources}
\include{pgfplots.importexport}
\include{pgfplots.basic.reference}

\printindex

\bibliographystyle{abbrv} %gerapali} %gerabbrv} %gerunsrt.bst} %gerabbrv}% gerplain}
\nocite{pgfplotstable}
\nocite{programmingnotes}
\bibliography{pgfplots}
\end{document}

% -----------------------------------------------------------------------------
% For Stefan Pinnow as reminder on what to look for when editing the manual
% -----------------------------------------------------------------------------
% There should be no line breaks in the following environments
% - |...|
% - \declareandlabel{...}
% - \verbpdfref{...}
% If "MakeTikzPictures" isn't running through check one of the externalized
% LOG files what is the cause of that.
% -----------------------------------------------------------------------------


%%%%%%%%%%%%%%%%%%%%%%%%%%%%%%%%%%%%%%%%%%%%%%%%%%%%%%%%%%%%%%%%%%%%%%%%%%%%%
%
% Package pgfplots.sty documentation.
%
% Copyright 2007/2008 by Christian Feuersaenger.
%
% This program is free software: you can redistribute it and/or modify
% it under the terms of the GNU General Public License as published by
% the Free Software Foundation, either version 3 of the License, or
% (at your option) any later version.
%
% This program is distributed in the hope that it will be useful,
% but WITHOUT ANY WARRANTY; without even the implied warranty of
% MERCHANTABILITY or FITNESS FOR A PARTICULAR PURPOSE.  See the
% GNU General Public License for more details.
%
% You should have received a copy of the GNU General Public License
% along with this program.  If not, see <http://www.gnu.org/licenses/>.
%
%
%%%%%%%%%%%%%%%%%%%%%%%%%%%%%%%%%%%%%%%%%%%%%%%%%%%%%%%%%%%%%%%%%%%%%%%%%%%%%
% SEE pgfplots-macros.tex as well!
%\pdfminorversion=5 % to allow compression
%\pdfobjcompresslevel=2
\documentclass[a4paper,openany]{book}

\let\bookmaketitle=\maketitle

% -----------------------------------
% this here is from ltxdoc:
\usepackage{doc}[=v2]
\AtBeginDocument{\MakeShortVerb{\|}}
%\setlength{\textwidth}{355pt}
%\addtolength\marginparwidth{30pt}
%\addtolength\oddsidemargin{20pt}
%\addtolength\evensidemargin{20pt}
\def\cmd#1{\cs{\expandafter\cmd@to@cs\string#1}}
\def\cmd@to@cs#1#2{\char\number`#2\relax}
\DeclareRobustCommand\cs[1]{\texttt{\char`\\#1}}
\providecommand\marg[1]{%
  {\ttfamily\char`\{}\meta{#1}{\ttfamily\char`\}}}
\providecommand\oarg[1]{%
  {\ttfamily[}\meta{#1}{\ttfamily]}}
\providecommand\parg[1]{%
  {\ttfamily(}\meta{#1}{\ttfamily)}}
\raggedbottom

% -----------------------------------
\input{pgfplots.preamble.tex}

\makeatletter
% I want a two-column index just as in pgfmanual styles. This here
% was the best way to get one:
\def\index@prologue{\section*{Index}\addcontentsline{toc}{chapter}{Index}
}
\makeatother

%\RequirePackage[german,english,francais]{babel}

\def\matlabcolormaptext{This colormap is similar to one shipped with Matlab$^\text{\textregistered}$ under a similar name.}

\IfFileExists{tikzlibraryspy.code.tex}{%
\usetikzlibrary{spy}
}{%
    \message{ERROR: tikz SPY library NOT available. The manual will only compile partially.^^J}%
}%

\usepackage{xparse}% for colorbrewer manual

\usetikzlibrary{
    decorations.markings,
    decorations.footprints,
    shapes.arrows,
    matrix,
    positioning,
}

\usepgfplotslibrary{
    fillbetween,
    ternary,
    smithchart,
    patchplots,
    polar,
    colormaps,
    colorbrewer,
    colortol,           % docu not ready yet
}
\pgfqkeys{/codeexample}{%
    every codeexample/.append style={
        /pgfplots/every ternary axis/.append style={
            /pgfplots/legend style={fill=graphicbackground},
        }
    },
    tabsize=4,
}

\pgfplotsmanualenableexternalizationofexpensive

%\usetikzlibrary{external}
%\tikzexternalize[prefix=figures/]{pgfplots}

\title{%
    Manual for Package \PGFPlots{}\\
    {\small 2D/3D Plots in \LaTeX{}, Version \pgfplotsversion}\\
    {\small\url{https://github.com/pgf-tikz/pgfplots}}
    %\\{\small Attention: you are using an unstable development version.}
}

\makeatletter
\long\def\abstractsmuggle{%
    \centering
    \textbf{Abstract}\\[0.5cm]

    \begin{minipage}{12cm}
        \PGFPlots{} draws high-quality function plots in normal or logarithmic
        scaling with a user-friendly interface directly in \TeX{}. The user
        supplies axis labels, legend entries and the plot coordinates for one or
        more plots and \PGFPlots{} applies axis scaling, computes any logarithms
        and axis ticks and draws the plots. It supports line plots, scatter
        plots, piecewise constant plots, bar plots, area plots, mesh and surface
        plots, patch plots, contour plots, quiver plots, histogram plots, box
        plots, polar axes, ternary diagrams, smith charts and some more. It is
        based on Till Tantau's package \PGF{}/\Tikz{}.
    \end{minipage}

}%

\expandafter\date\expandafter{\@date\\[2cm]
    \abstractsmuggle
}%
\makeatother

%\includeonly{pgfplots.reference}


\begin{document}

\def\plotcoords{%
\addplot coordinates {
(5,8.312e-02)    (17,2.547e-02)   (49,7.407e-03)
(129,2.102e-03)  (321,5.874e-04)  (769,1.623e-04)
(1793,4.442e-05) (4097,1.207e-05) (9217,3.261e-06)
};

\addplot coordinates{
(7,8.472e-02)    (31,3.044e-02)    (111,1.022e-02)
(351,3.303e-03)  (1023,1.039e-03)  (2815,3.196e-04)
(7423,9.658e-05) (18943,2.873e-05) (47103,8.437e-06)
};

\addplot coordinates{
(9,7.881e-02)     (49,3.243e-02)    (209,1.232e-02)
(769,4.454e-03)   (2561,1.551e-03)  (7937,5.236e-04)
(23297,1.723e-04) (65537,5.545e-05) (178177,1.751e-05)
};

\addplot coordinates{
(11,6.887e-02)    (71,3.177e-02)     (351,1.341e-02)
(1471,5.334e-03)  (5503,2.027e-03)   (18943,7.415e-04)
(61183,2.628e-04) (187903,9.063e-05) (553983,3.053e-05)
};

\addplot coordinates{
(13,5.755e-02)     (97,2.925e-02)     (545,1.351e-02)
(2561,5.842e-03)   (10625,2.397e-03)  (40193,9.414e-04)
(141569,3.564e-04) (471041,1.308e-04)
(1496065,4.670e-05)
};
}%


\bookmaketitle
\tableofcontents
\include{pgfplots.title_abstract_intro}
\include{pgfplots.preliminaries}
\include{pgfplots.intro}
\include{pgfplots.reference}
\include{pgfplots.libs}
\include{pgfplots.resources}
\include{pgfplots.importexport}
\include{pgfplots.basic.reference}

\printindex

\bibliographystyle{abbrv} %gerapali} %gerabbrv} %gerunsrt.bst} %gerabbrv}% gerplain}
\nocite{pgfplotstable}
\nocite{programmingnotes}
\bibliography{pgfplots}
\end{document}

% -----------------------------------------------------------------------------
% For Stefan Pinnow as reminder on what to look for when editing the manual
% -----------------------------------------------------------------------------
% There should be no line breaks in the following environments
% - |...|
% - \declareandlabel{...}
% - \verbpdfref{...}
% If "MakeTikzPictures" isn't running through check one of the externalized
% LOG files what is the cause of that.
% -----------------------------------------------------------------------------


%%%%%%%%%%%%%%%%%%%%%%%%%%%%%%%%%%%%%%%%%%%%%%%%%%%%%%%%%%%%%%%%%%%%%%%%%%%%%
%
% Package pgfplots.sty documentation.
%
% Copyright 2007/2008 by Christian Feuersaenger.
%
% This program is free software: you can redistribute it and/or modify
% it under the terms of the GNU General Public License as published by
% the Free Software Foundation, either version 3 of the License, or
% (at your option) any later version.
%
% This program is distributed in the hope that it will be useful,
% but WITHOUT ANY WARRANTY; without even the implied warranty of
% MERCHANTABILITY or FITNESS FOR A PARTICULAR PURPOSE.  See the
% GNU General Public License for more details.
%
% You should have received a copy of the GNU General Public License
% along with this program.  If not, see <http://www.gnu.org/licenses/>.
%
%
%%%%%%%%%%%%%%%%%%%%%%%%%%%%%%%%%%%%%%%%%%%%%%%%%%%%%%%%%%%%%%%%%%%%%%%%%%%%%
% SEE pgfplots-macros.tex as well!
%\pdfminorversion=5 % to allow compression
%\pdfobjcompresslevel=2
\documentclass[a4paper,openany]{book}

\let\bookmaketitle=\maketitle

% -----------------------------------
% this here is from ltxdoc:
\usepackage{doc}[=v2]
\AtBeginDocument{\MakeShortVerb{\|}}
%\setlength{\textwidth}{355pt}
%\addtolength\marginparwidth{30pt}
%\addtolength\oddsidemargin{20pt}
%\addtolength\evensidemargin{20pt}
\def\cmd#1{\cs{\expandafter\cmd@to@cs\string#1}}
\def\cmd@to@cs#1#2{\char\number`#2\relax}
\DeclareRobustCommand\cs[1]{\texttt{\char`\\#1}}
\providecommand\marg[1]{%
  {\ttfamily\char`\{}\meta{#1}{\ttfamily\char`\}}}
\providecommand\oarg[1]{%
  {\ttfamily[}\meta{#1}{\ttfamily]}}
\providecommand\parg[1]{%
  {\ttfamily(}\meta{#1}{\ttfamily)}}
\raggedbottom

% -----------------------------------
\input{pgfplots.preamble.tex}

\makeatletter
% I want a two-column index just as in pgfmanual styles. This here
% was the best way to get one:
\def\index@prologue{\section*{Index}\addcontentsline{toc}{chapter}{Index}
}
\makeatother

%\RequirePackage[german,english,francais]{babel}

\def\matlabcolormaptext{This colormap is similar to one shipped with Matlab$^\text{\textregistered}$ under a similar name.}

\IfFileExists{tikzlibraryspy.code.tex}{%
\usetikzlibrary{spy}
}{%
    \message{ERROR: tikz SPY library NOT available. The manual will only compile partially.^^J}%
}%

\usepackage{xparse}% for colorbrewer manual

\usetikzlibrary{
    decorations.markings,
    decorations.footprints,
    shapes.arrows,
    matrix,
    positioning,
}

\usepgfplotslibrary{
    fillbetween,
    ternary,
    smithchart,
    patchplots,
    polar,
    colormaps,
    colorbrewer,
    colortol,           % docu not ready yet
}
\pgfqkeys{/codeexample}{%
    every codeexample/.append style={
        /pgfplots/every ternary axis/.append style={
            /pgfplots/legend style={fill=graphicbackground},
        }
    },
    tabsize=4,
}

\pgfplotsmanualenableexternalizationofexpensive

%\usetikzlibrary{external}
%\tikzexternalize[prefix=figures/]{pgfplots}

\title{%
    Manual for Package \PGFPlots{}\\
    {\small 2D/3D Plots in \LaTeX{}, Version \pgfplotsversion}\\
    {\small\url{https://github.com/pgf-tikz/pgfplots}}
    %\\{\small Attention: you are using an unstable development version.}
}

\makeatletter
\long\def\abstractsmuggle{%
    \centering
    \textbf{Abstract}\\[0.5cm]

    \begin{minipage}{12cm}
        \PGFPlots{} draws high-quality function plots in normal or logarithmic
        scaling with a user-friendly interface directly in \TeX{}. The user
        supplies axis labels, legend entries and the plot coordinates for one or
        more plots and \PGFPlots{} applies axis scaling, computes any logarithms
        and axis ticks and draws the plots. It supports line plots, scatter
        plots, piecewise constant plots, bar plots, area plots, mesh and surface
        plots, patch plots, contour plots, quiver plots, histogram plots, box
        plots, polar axes, ternary diagrams, smith charts and some more. It is
        based on Till Tantau's package \PGF{}/\Tikz{}.
    \end{minipage}

}%

\expandafter\date\expandafter{\@date\\[2cm]
    \abstractsmuggle
}%
\makeatother

%\includeonly{pgfplots.reference}


\begin{document}

\def\plotcoords{%
\addplot coordinates {
(5,8.312e-02)    (17,2.547e-02)   (49,7.407e-03)
(129,2.102e-03)  (321,5.874e-04)  (769,1.623e-04)
(1793,4.442e-05) (4097,1.207e-05) (9217,3.261e-06)
};

\addplot coordinates{
(7,8.472e-02)    (31,3.044e-02)    (111,1.022e-02)
(351,3.303e-03)  (1023,1.039e-03)  (2815,3.196e-04)
(7423,9.658e-05) (18943,2.873e-05) (47103,8.437e-06)
};

\addplot coordinates{
(9,7.881e-02)     (49,3.243e-02)    (209,1.232e-02)
(769,4.454e-03)   (2561,1.551e-03)  (7937,5.236e-04)
(23297,1.723e-04) (65537,5.545e-05) (178177,1.751e-05)
};

\addplot coordinates{
(11,6.887e-02)    (71,3.177e-02)     (351,1.341e-02)
(1471,5.334e-03)  (5503,2.027e-03)   (18943,7.415e-04)
(61183,2.628e-04) (187903,9.063e-05) (553983,3.053e-05)
};

\addplot coordinates{
(13,5.755e-02)     (97,2.925e-02)     (545,1.351e-02)
(2561,5.842e-03)   (10625,2.397e-03)  (40193,9.414e-04)
(141569,3.564e-04) (471041,1.308e-04)
(1496065,4.670e-05)
};
}%


\bookmaketitle
\tableofcontents
\include{pgfplots.title_abstract_intro}
\include{pgfplots.preliminaries}
\include{pgfplots.intro}
\include{pgfplots.reference}
\include{pgfplots.libs}
\include{pgfplots.resources}
\include{pgfplots.importexport}
\include{pgfplots.basic.reference}

\printindex

\bibliographystyle{abbrv} %gerapali} %gerabbrv} %gerunsrt.bst} %gerabbrv}% gerplain}
\nocite{pgfplotstable}
\nocite{programmingnotes}
\bibliography{pgfplots}
\end{document}

% -----------------------------------------------------------------------------
% For Stefan Pinnow as reminder on what to look for when editing the manual
% -----------------------------------------------------------------------------
% There should be no line breaks in the following environments
% - |...|
% - \declareandlabel{...}
% - \verbpdfref{...}
% If "MakeTikzPictures" isn't running through check one of the externalized
% LOG files what is the cause of that.
% -----------------------------------------------------------------------------


%%%%%%%%%%%%%%%%%%%%%%%%%%%%%%%%%%%%%%%%%%%%%%%%%%%%%%%%%%%%%%%%%%%%%%%%%%%%%
%
% Package pgfplots.sty documentation.
%
% Copyright 2007/2008 by Christian Feuersaenger.
%
% This program is free software: you can redistribute it and/or modify
% it under the terms of the GNU General Public License as published by
% the Free Software Foundation, either version 3 of the License, or
% (at your option) any later version.
%
% This program is distributed in the hope that it will be useful,
% but WITHOUT ANY WARRANTY; without even the implied warranty of
% MERCHANTABILITY or FITNESS FOR A PARTICULAR PURPOSE.  See the
% GNU General Public License for more details.
%
% You should have received a copy of the GNU General Public License
% along with this program.  If not, see <http://www.gnu.org/licenses/>.
%
%
%%%%%%%%%%%%%%%%%%%%%%%%%%%%%%%%%%%%%%%%%%%%%%%%%%%%%%%%%%%%%%%%%%%%%%%%%%%%%
% SEE pgfplots-macros.tex as well!
%\pdfminorversion=5 % to allow compression
%\pdfobjcompresslevel=2
\documentclass[a4paper,openany]{book}

\let\bookmaketitle=\maketitle

% -----------------------------------
% this here is from ltxdoc:
\usepackage{doc}[=v2]
\AtBeginDocument{\MakeShortVerb{\|}}
%\setlength{\textwidth}{355pt}
%\addtolength\marginparwidth{30pt}
%\addtolength\oddsidemargin{20pt}
%\addtolength\evensidemargin{20pt}
\def\cmd#1{\cs{\expandafter\cmd@to@cs\string#1}}
\def\cmd@to@cs#1#2{\char\number`#2\relax}
\DeclareRobustCommand\cs[1]{\texttt{\char`\\#1}}
\providecommand\marg[1]{%
  {\ttfamily\char`\{}\meta{#1}{\ttfamily\char`\}}}
\providecommand\oarg[1]{%
  {\ttfamily[}\meta{#1}{\ttfamily]}}
\providecommand\parg[1]{%
  {\ttfamily(}\meta{#1}{\ttfamily)}}
\raggedbottom

% -----------------------------------
\input{pgfplots.preamble.tex}

\makeatletter
% I want a two-column index just as in pgfmanual styles. This here
% was the best way to get one:
\def\index@prologue{\section*{Index}\addcontentsline{toc}{chapter}{Index}
}
\makeatother

%\RequirePackage[german,english,francais]{babel}

\def\matlabcolormaptext{This colormap is similar to one shipped with Matlab$^\text{\textregistered}$ under a similar name.}

\IfFileExists{tikzlibraryspy.code.tex}{%
\usetikzlibrary{spy}
}{%
    \message{ERROR: tikz SPY library NOT available. The manual will only compile partially.^^J}%
}%

\usepackage{xparse}% for colorbrewer manual

\usetikzlibrary{
    decorations.markings,
    decorations.footprints,
    shapes.arrows,
    matrix,
    positioning,
}

\usepgfplotslibrary{
    fillbetween,
    ternary,
    smithchart,
    patchplots,
    polar,
    colormaps,
    colorbrewer,
    colortol,           % docu not ready yet
}
\pgfqkeys{/codeexample}{%
    every codeexample/.append style={
        /pgfplots/every ternary axis/.append style={
            /pgfplots/legend style={fill=graphicbackground},
        }
    },
    tabsize=4,
}

\pgfplotsmanualenableexternalizationofexpensive

%\usetikzlibrary{external}
%\tikzexternalize[prefix=figures/]{pgfplots}

\title{%
    Manual for Package \PGFPlots{}\\
    {\small 2D/3D Plots in \LaTeX{}, Version \pgfplotsversion}\\
    {\small\url{https://github.com/pgf-tikz/pgfplots}}
    %\\{\small Attention: you are using an unstable development version.}
}

\makeatletter
\long\def\abstractsmuggle{%
    \centering
    \textbf{Abstract}\\[0.5cm]

    \begin{minipage}{12cm}
        \PGFPlots{} draws high-quality function plots in normal or logarithmic
        scaling with a user-friendly interface directly in \TeX{}. The user
        supplies axis labels, legend entries and the plot coordinates for one or
        more plots and \PGFPlots{} applies axis scaling, computes any logarithms
        and axis ticks and draws the plots. It supports line plots, scatter
        plots, piecewise constant plots, bar plots, area plots, mesh and surface
        plots, patch plots, contour plots, quiver plots, histogram plots, box
        plots, polar axes, ternary diagrams, smith charts and some more. It is
        based on Till Tantau's package \PGF{}/\Tikz{}.
    \end{minipage}

}%

\expandafter\date\expandafter{\@date\\[2cm]
    \abstractsmuggle
}%
\makeatother

%\includeonly{pgfplots.reference}


\begin{document}

\def\plotcoords{%
\addplot coordinates {
(5,8.312e-02)    (17,2.547e-02)   (49,7.407e-03)
(129,2.102e-03)  (321,5.874e-04)  (769,1.623e-04)
(1793,4.442e-05) (4097,1.207e-05) (9217,3.261e-06)
};

\addplot coordinates{
(7,8.472e-02)    (31,3.044e-02)    (111,1.022e-02)
(351,3.303e-03)  (1023,1.039e-03)  (2815,3.196e-04)
(7423,9.658e-05) (18943,2.873e-05) (47103,8.437e-06)
};

\addplot coordinates{
(9,7.881e-02)     (49,3.243e-02)    (209,1.232e-02)
(769,4.454e-03)   (2561,1.551e-03)  (7937,5.236e-04)
(23297,1.723e-04) (65537,5.545e-05) (178177,1.751e-05)
};

\addplot coordinates{
(11,6.887e-02)    (71,3.177e-02)     (351,1.341e-02)
(1471,5.334e-03)  (5503,2.027e-03)   (18943,7.415e-04)
(61183,2.628e-04) (187903,9.063e-05) (553983,3.053e-05)
};

\addplot coordinates{
(13,5.755e-02)     (97,2.925e-02)     (545,1.351e-02)
(2561,5.842e-03)   (10625,2.397e-03)  (40193,9.414e-04)
(141569,3.564e-04) (471041,1.308e-04)
(1496065,4.670e-05)
};
}%


\bookmaketitle
\tableofcontents
\include{pgfplots.title_abstract_intro}
\include{pgfplots.preliminaries}
\include{pgfplots.intro}
\include{pgfplots.reference}
\include{pgfplots.libs}
\include{pgfplots.resources}
\include{pgfplots.importexport}
\include{pgfplots.basic.reference}

\printindex

\bibliographystyle{abbrv} %gerapali} %gerabbrv} %gerunsrt.bst} %gerabbrv}% gerplain}
\nocite{pgfplotstable}
\nocite{programmingnotes}
\bibliography{pgfplots}
\end{document}

% -----------------------------------------------------------------------------
% For Stefan Pinnow as reminder on what to look for when editing the manual
% -----------------------------------------------------------------------------
% There should be no line breaks in the following environments
% - |...|
% - \declareandlabel{...}
% - \verbpdfref{...}
% If "MakeTikzPictures" isn't running through check one of the externalized
% LOG files what is the cause of that.
% -----------------------------------------------------------------------------

\include{pgfplots.basic.reference}

\printindex

\bibliographystyle{abbrv} %gerapali} %gerabbrv} %gerunsrt.bst} %gerabbrv}% gerplain}
\nocite{pgfplotstable}
\nocite{programmingnotes}
\bibliography{pgfplots}
\end{document}

% -----------------------------------------------------------------------------
% For Stefan Pinnow as reminder on what to look for when editing the manual
% -----------------------------------------------------------------------------
% There should be no line breaks in the following environments
% - |...|
% - \declareandlabel{...}
% - \verbpdfref{...}
% If "MakeTikzPictures" isn't running through check one of the externalized
% LOG files what is the cause of that.
% -----------------------------------------------------------------------------


%%%%%%%%%%%%%%%%%%%%%%%%%%%%%%%%%%%%%%%%%%%%%%%%%%%%%%%%%%%%%%%%%%%%%%%%%%%%%
%
% Package pgfplots.sty documentation.
%
% Copyright 2007/2008 by Christian Feuersaenger.
%
% This program is free software: you can redistribute it and/or modify
% it under the terms of the GNU General Public License as published by
% the Free Software Foundation, either version 3 of the License, or
% (at your option) any later version.
%
% This program is distributed in the hope that it will be useful,
% but WITHOUT ANY WARRANTY; without even the implied warranty of
% MERCHANTABILITY or FITNESS FOR A PARTICULAR PURPOSE.  See the
% GNU General Public License for more details.
%
% You should have received a copy of the GNU General Public License
% along with this program.  If not, see <http://www.gnu.org/licenses/>.
%
%
%%%%%%%%%%%%%%%%%%%%%%%%%%%%%%%%%%%%%%%%%%%%%%%%%%%%%%%%%%%%%%%%%%%%%%%%%%%%%
% SEE pgfplots-macros.tex as well!
%\pdfminorversion=5 % to allow compression
%\pdfobjcompresslevel=2
\documentclass[a4paper,openany]{book}

\let\bookmaketitle=\maketitle

% -----------------------------------
% this here is from ltxdoc:
\usepackage{doc}[=v2]
\AtBeginDocument{\MakeShortVerb{\|}}
%\setlength{\textwidth}{355pt}
%\addtolength\marginparwidth{30pt}
%\addtolength\oddsidemargin{20pt}
%\addtolength\evensidemargin{20pt}
\def\cmd#1{\cs{\expandafter\cmd@to@cs\string#1}}
\def\cmd@to@cs#1#2{\char\number`#2\relax}
\DeclareRobustCommand\cs[1]{\texttt{\char`\\#1}}
\providecommand\marg[1]{%
  {\ttfamily\char`\{}\meta{#1}{\ttfamily\char`\}}}
\providecommand\oarg[1]{%
  {\ttfamily[}\meta{#1}{\ttfamily]}}
\providecommand\parg[1]{%
  {\ttfamily(}\meta{#1}{\ttfamily)}}
\raggedbottom

% -----------------------------------
\input{pgfplots.preamble.tex}

\makeatletter
% I want a two-column index just as in pgfmanual styles. This here
% was the best way to get one:
\def\index@prologue{\section*{Index}\addcontentsline{toc}{chapter}{Index}
}
\makeatother

%\RequirePackage[german,english,francais]{babel}

\def\matlabcolormaptext{This colormap is similar to one shipped with Matlab$^\text{\textregistered}$ under a similar name.}

\IfFileExists{tikzlibraryspy.code.tex}{%
\usetikzlibrary{spy}
}{%
    \message{ERROR: tikz SPY library NOT available. The manual will only compile partially.^^J}%
}%

\usepackage{xparse}% for colorbrewer manual

\usetikzlibrary{
    decorations.markings,
    decorations.footprints,
    shapes.arrows,
    matrix,
    positioning,
}

\usepgfplotslibrary{
    fillbetween,
    ternary,
    smithchart,
    patchplots,
    polar,
    colormaps,
    colorbrewer,
    colortol,           % docu not ready yet
}
\pgfqkeys{/codeexample}{%
    every codeexample/.append style={
        /pgfplots/every ternary axis/.append style={
            /pgfplots/legend style={fill=graphicbackground},
        }
    },
    tabsize=4,
}

\pgfplotsmanualenableexternalizationofexpensive

%\usetikzlibrary{external}
%\tikzexternalize[prefix=figures/]{pgfplots}

\title{%
    Manual for Package \PGFPlots{}\\
    {\small 2D/3D Plots in \LaTeX{}, Version \pgfplotsversion}\\
    {\small\url{https://github.com/pgf-tikz/pgfplots}}
    %\\{\small Attention: you are using an unstable development version.}
}

\makeatletter
\long\def\abstractsmuggle{%
    \centering
    \textbf{Abstract}\\[0.5cm]

    \begin{minipage}{12cm}
        \PGFPlots{} draws high-quality function plots in normal or logarithmic
        scaling with a user-friendly interface directly in \TeX{}. The user
        supplies axis labels, legend entries and the plot coordinates for one or
        more plots and \PGFPlots{} applies axis scaling, computes any logarithms
        and axis ticks and draws the plots. It supports line plots, scatter
        plots, piecewise constant plots, bar plots, area plots, mesh and surface
        plots, patch plots, contour plots, quiver plots, histogram plots, box
        plots, polar axes, ternary diagrams, smith charts and some more. It is
        based on Till Tantau's package \PGF{}/\Tikz{}.
    \end{minipage}

}%

\expandafter\date\expandafter{\@date\\[2cm]
    \abstractsmuggle
}%
\makeatother

%\includeonly{pgfplots.reference}


\begin{document}

\def\plotcoords{%
\addplot coordinates {
(5,8.312e-02)    (17,2.547e-02)   (49,7.407e-03)
(129,2.102e-03)  (321,5.874e-04)  (769,1.623e-04)
(1793,4.442e-05) (4097,1.207e-05) (9217,3.261e-06)
};

\addplot coordinates{
(7,8.472e-02)    (31,3.044e-02)    (111,1.022e-02)
(351,3.303e-03)  (1023,1.039e-03)  (2815,3.196e-04)
(7423,9.658e-05) (18943,2.873e-05) (47103,8.437e-06)
};

\addplot coordinates{
(9,7.881e-02)     (49,3.243e-02)    (209,1.232e-02)
(769,4.454e-03)   (2561,1.551e-03)  (7937,5.236e-04)
(23297,1.723e-04) (65537,5.545e-05) (178177,1.751e-05)
};

\addplot coordinates{
(11,6.887e-02)    (71,3.177e-02)     (351,1.341e-02)
(1471,5.334e-03)  (5503,2.027e-03)   (18943,7.415e-04)
(61183,2.628e-04) (187903,9.063e-05) (553983,3.053e-05)
};

\addplot coordinates{
(13,5.755e-02)     (97,2.925e-02)     (545,1.351e-02)
(2561,5.842e-03)   (10625,2.397e-03)  (40193,9.414e-04)
(141569,3.564e-04) (471041,1.308e-04)
(1496065,4.670e-05)
};
}%


\bookmaketitle
\tableofcontents

%%%%%%%%%%%%%%%%%%%%%%%%%%%%%%%%%%%%%%%%%%%%%%%%%%%%%%%%%%%%%%%%%%%%%%%%%%%%%
%
% Package pgfplots.sty documentation.
%
% Copyright 2007/2008 by Christian Feuersaenger.
%
% This program is free software: you can redistribute it and/or modify
% it under the terms of the GNU General Public License as published by
% the Free Software Foundation, either version 3 of the License, or
% (at your option) any later version.
%
% This program is distributed in the hope that it will be useful,
% but WITHOUT ANY WARRANTY; without even the implied warranty of
% MERCHANTABILITY or FITNESS FOR A PARTICULAR PURPOSE.  See the
% GNU General Public License for more details.
%
% You should have received a copy of the GNU General Public License
% along with this program.  If not, see <http://www.gnu.org/licenses/>.
%
%
%%%%%%%%%%%%%%%%%%%%%%%%%%%%%%%%%%%%%%%%%%%%%%%%%%%%%%%%%%%%%%%%%%%%%%%%%%%%%
% SEE pgfplots-macros.tex as well!
%\pdfminorversion=5 % to allow compression
%\pdfobjcompresslevel=2
\documentclass[a4paper,openany]{book}

\let\bookmaketitle=\maketitle

% -----------------------------------
% this here is from ltxdoc:
\usepackage{doc}[=v2]
\AtBeginDocument{\MakeShortVerb{\|}}
%\setlength{\textwidth}{355pt}
%\addtolength\marginparwidth{30pt}
%\addtolength\oddsidemargin{20pt}
%\addtolength\evensidemargin{20pt}
\def\cmd#1{\cs{\expandafter\cmd@to@cs\string#1}}
\def\cmd@to@cs#1#2{\char\number`#2\relax}
\DeclareRobustCommand\cs[1]{\texttt{\char`\\#1}}
\providecommand\marg[1]{%
  {\ttfamily\char`\{}\meta{#1}{\ttfamily\char`\}}}
\providecommand\oarg[1]{%
  {\ttfamily[}\meta{#1}{\ttfamily]}}
\providecommand\parg[1]{%
  {\ttfamily(}\meta{#1}{\ttfamily)}}
\raggedbottom

% -----------------------------------
\input{pgfplots.preamble.tex}

\makeatletter
% I want a two-column index just as in pgfmanual styles. This here
% was the best way to get one:
\def\index@prologue{\section*{Index}\addcontentsline{toc}{chapter}{Index}
}
\makeatother

%\RequirePackage[german,english,francais]{babel}

\def\matlabcolormaptext{This colormap is similar to one shipped with Matlab$^\text{\textregistered}$ under a similar name.}

\IfFileExists{tikzlibraryspy.code.tex}{%
\usetikzlibrary{spy}
}{%
    \message{ERROR: tikz SPY library NOT available. The manual will only compile partially.^^J}%
}%

\usepackage{xparse}% for colorbrewer manual

\usetikzlibrary{
    decorations.markings,
    decorations.footprints,
    shapes.arrows,
    matrix,
    positioning,
}

\usepgfplotslibrary{
    fillbetween,
    ternary,
    smithchart,
    patchplots,
    polar,
    colormaps,
    colorbrewer,
    colortol,           % docu not ready yet
}
\pgfqkeys{/codeexample}{%
    every codeexample/.append style={
        /pgfplots/every ternary axis/.append style={
            /pgfplots/legend style={fill=graphicbackground},
        }
    },
    tabsize=4,
}

\pgfplotsmanualenableexternalizationofexpensive

%\usetikzlibrary{external}
%\tikzexternalize[prefix=figures/]{pgfplots}

\title{%
    Manual for Package \PGFPlots{}\\
    {\small 2D/3D Plots in \LaTeX{}, Version \pgfplotsversion}\\
    {\small\url{https://github.com/pgf-tikz/pgfplots}}
    %\\{\small Attention: you are using an unstable development version.}
}

\makeatletter
\long\def\abstractsmuggle{%
    \centering
    \textbf{Abstract}\\[0.5cm]

    \begin{minipage}{12cm}
        \PGFPlots{} draws high-quality function plots in normal or logarithmic
        scaling with a user-friendly interface directly in \TeX{}. The user
        supplies axis labels, legend entries and the plot coordinates for one or
        more plots and \PGFPlots{} applies axis scaling, computes any logarithms
        and axis ticks and draws the plots. It supports line plots, scatter
        plots, piecewise constant plots, bar plots, area plots, mesh and surface
        plots, patch plots, contour plots, quiver plots, histogram plots, box
        plots, polar axes, ternary diagrams, smith charts and some more. It is
        based on Till Tantau's package \PGF{}/\Tikz{}.
    \end{minipage}

}%

\expandafter\date\expandafter{\@date\\[2cm]
    \abstractsmuggle
}%
\makeatother

%\includeonly{pgfplots.reference}


\begin{document}

\def\plotcoords{%
\addplot coordinates {
(5,8.312e-02)    (17,2.547e-02)   (49,7.407e-03)
(129,2.102e-03)  (321,5.874e-04)  (769,1.623e-04)
(1793,4.442e-05) (4097,1.207e-05) (9217,3.261e-06)
};

\addplot coordinates{
(7,8.472e-02)    (31,3.044e-02)    (111,1.022e-02)
(351,3.303e-03)  (1023,1.039e-03)  (2815,3.196e-04)
(7423,9.658e-05) (18943,2.873e-05) (47103,8.437e-06)
};

\addplot coordinates{
(9,7.881e-02)     (49,3.243e-02)    (209,1.232e-02)
(769,4.454e-03)   (2561,1.551e-03)  (7937,5.236e-04)
(23297,1.723e-04) (65537,5.545e-05) (178177,1.751e-05)
};

\addplot coordinates{
(11,6.887e-02)    (71,3.177e-02)     (351,1.341e-02)
(1471,5.334e-03)  (5503,2.027e-03)   (18943,7.415e-04)
(61183,2.628e-04) (187903,9.063e-05) (553983,3.053e-05)
};

\addplot coordinates{
(13,5.755e-02)     (97,2.925e-02)     (545,1.351e-02)
(2561,5.842e-03)   (10625,2.397e-03)  (40193,9.414e-04)
(141569,3.564e-04) (471041,1.308e-04)
(1496065,4.670e-05)
};
}%


\bookmaketitle
\tableofcontents
\include{pgfplots.title_abstract_intro}
\include{pgfplots.preliminaries}
\include{pgfplots.intro}
\include{pgfplots.reference}
\include{pgfplots.libs}
\include{pgfplots.resources}
\include{pgfplots.importexport}
\include{pgfplots.basic.reference}

\printindex

\bibliographystyle{abbrv} %gerapali} %gerabbrv} %gerunsrt.bst} %gerabbrv}% gerplain}
\nocite{pgfplotstable}
\nocite{programmingnotes}
\bibliography{pgfplots}
\end{document}

% -----------------------------------------------------------------------------
% For Stefan Pinnow as reminder on what to look for when editing the manual
% -----------------------------------------------------------------------------
% There should be no line breaks in the following environments
% - |...|
% - \declareandlabel{...}
% - \verbpdfref{...}
% If "MakeTikzPictures" isn't running through check one of the externalized
% LOG files what is the cause of that.
% -----------------------------------------------------------------------------


%%%%%%%%%%%%%%%%%%%%%%%%%%%%%%%%%%%%%%%%%%%%%%%%%%%%%%%%%%%%%%%%%%%%%%%%%%%%%
%
% Package pgfplots.sty documentation.
%
% Copyright 2007/2008 by Christian Feuersaenger.
%
% This program is free software: you can redistribute it and/or modify
% it under the terms of the GNU General Public License as published by
% the Free Software Foundation, either version 3 of the License, or
% (at your option) any later version.
%
% This program is distributed in the hope that it will be useful,
% but WITHOUT ANY WARRANTY; without even the implied warranty of
% MERCHANTABILITY or FITNESS FOR A PARTICULAR PURPOSE.  See the
% GNU General Public License for more details.
%
% You should have received a copy of the GNU General Public License
% along with this program.  If not, see <http://www.gnu.org/licenses/>.
%
%
%%%%%%%%%%%%%%%%%%%%%%%%%%%%%%%%%%%%%%%%%%%%%%%%%%%%%%%%%%%%%%%%%%%%%%%%%%%%%
% SEE pgfplots-macros.tex as well!
%\pdfminorversion=5 % to allow compression
%\pdfobjcompresslevel=2
\documentclass[a4paper,openany]{book}

\let\bookmaketitle=\maketitle

% -----------------------------------
% this here is from ltxdoc:
\usepackage{doc}[=v2]
\AtBeginDocument{\MakeShortVerb{\|}}
%\setlength{\textwidth}{355pt}
%\addtolength\marginparwidth{30pt}
%\addtolength\oddsidemargin{20pt}
%\addtolength\evensidemargin{20pt}
\def\cmd#1{\cs{\expandafter\cmd@to@cs\string#1}}
\def\cmd@to@cs#1#2{\char\number`#2\relax}
\DeclareRobustCommand\cs[1]{\texttt{\char`\\#1}}
\providecommand\marg[1]{%
  {\ttfamily\char`\{}\meta{#1}{\ttfamily\char`\}}}
\providecommand\oarg[1]{%
  {\ttfamily[}\meta{#1}{\ttfamily]}}
\providecommand\parg[1]{%
  {\ttfamily(}\meta{#1}{\ttfamily)}}
\raggedbottom

% -----------------------------------
\input{pgfplots.preamble.tex}

\makeatletter
% I want a two-column index just as in pgfmanual styles. This here
% was the best way to get one:
\def\index@prologue{\section*{Index}\addcontentsline{toc}{chapter}{Index}
}
\makeatother

%\RequirePackage[german,english,francais]{babel}

\def\matlabcolormaptext{This colormap is similar to one shipped with Matlab$^\text{\textregistered}$ under a similar name.}

\IfFileExists{tikzlibraryspy.code.tex}{%
\usetikzlibrary{spy}
}{%
    \message{ERROR: tikz SPY library NOT available. The manual will only compile partially.^^J}%
}%

\usepackage{xparse}% for colorbrewer manual

\usetikzlibrary{
    decorations.markings,
    decorations.footprints,
    shapes.arrows,
    matrix,
    positioning,
}

\usepgfplotslibrary{
    fillbetween,
    ternary,
    smithchart,
    patchplots,
    polar,
    colormaps,
    colorbrewer,
    colortol,           % docu not ready yet
}
\pgfqkeys{/codeexample}{%
    every codeexample/.append style={
        /pgfplots/every ternary axis/.append style={
            /pgfplots/legend style={fill=graphicbackground},
        }
    },
    tabsize=4,
}

\pgfplotsmanualenableexternalizationofexpensive

%\usetikzlibrary{external}
%\tikzexternalize[prefix=figures/]{pgfplots}

\title{%
    Manual for Package \PGFPlots{}\\
    {\small 2D/3D Plots in \LaTeX{}, Version \pgfplotsversion}\\
    {\small\url{https://github.com/pgf-tikz/pgfplots}}
    %\\{\small Attention: you are using an unstable development version.}
}

\makeatletter
\long\def\abstractsmuggle{%
    \centering
    \textbf{Abstract}\\[0.5cm]

    \begin{minipage}{12cm}
        \PGFPlots{} draws high-quality function plots in normal or logarithmic
        scaling with a user-friendly interface directly in \TeX{}. The user
        supplies axis labels, legend entries and the plot coordinates for one or
        more plots and \PGFPlots{} applies axis scaling, computes any logarithms
        and axis ticks and draws the plots. It supports line plots, scatter
        plots, piecewise constant plots, bar plots, area plots, mesh and surface
        plots, patch plots, contour plots, quiver plots, histogram plots, box
        plots, polar axes, ternary diagrams, smith charts and some more. It is
        based on Till Tantau's package \PGF{}/\Tikz{}.
    \end{minipage}

}%

\expandafter\date\expandafter{\@date\\[2cm]
    \abstractsmuggle
}%
\makeatother

%\includeonly{pgfplots.reference}


\begin{document}

\def\plotcoords{%
\addplot coordinates {
(5,8.312e-02)    (17,2.547e-02)   (49,7.407e-03)
(129,2.102e-03)  (321,5.874e-04)  (769,1.623e-04)
(1793,4.442e-05) (4097,1.207e-05) (9217,3.261e-06)
};

\addplot coordinates{
(7,8.472e-02)    (31,3.044e-02)    (111,1.022e-02)
(351,3.303e-03)  (1023,1.039e-03)  (2815,3.196e-04)
(7423,9.658e-05) (18943,2.873e-05) (47103,8.437e-06)
};

\addplot coordinates{
(9,7.881e-02)     (49,3.243e-02)    (209,1.232e-02)
(769,4.454e-03)   (2561,1.551e-03)  (7937,5.236e-04)
(23297,1.723e-04) (65537,5.545e-05) (178177,1.751e-05)
};

\addplot coordinates{
(11,6.887e-02)    (71,3.177e-02)     (351,1.341e-02)
(1471,5.334e-03)  (5503,2.027e-03)   (18943,7.415e-04)
(61183,2.628e-04) (187903,9.063e-05) (553983,3.053e-05)
};

\addplot coordinates{
(13,5.755e-02)     (97,2.925e-02)     (545,1.351e-02)
(2561,5.842e-03)   (10625,2.397e-03)  (40193,9.414e-04)
(141569,3.564e-04) (471041,1.308e-04)
(1496065,4.670e-05)
};
}%


\bookmaketitle
\tableofcontents
\include{pgfplots.title_abstract_intro}
\include{pgfplots.preliminaries}
\include{pgfplots.intro}
\include{pgfplots.reference}
\include{pgfplots.libs}
\include{pgfplots.resources}
\include{pgfplots.importexport}
\include{pgfplots.basic.reference}

\printindex

\bibliographystyle{abbrv} %gerapali} %gerabbrv} %gerunsrt.bst} %gerabbrv}% gerplain}
\nocite{pgfplotstable}
\nocite{programmingnotes}
\bibliography{pgfplots}
\end{document}

% -----------------------------------------------------------------------------
% For Stefan Pinnow as reminder on what to look for when editing the manual
% -----------------------------------------------------------------------------
% There should be no line breaks in the following environments
% - |...|
% - \declareandlabel{...}
% - \verbpdfref{...}
% If "MakeTikzPictures" isn't running through check one of the externalized
% LOG files what is the cause of that.
% -----------------------------------------------------------------------------


%%%%%%%%%%%%%%%%%%%%%%%%%%%%%%%%%%%%%%%%%%%%%%%%%%%%%%%%%%%%%%%%%%%%%%%%%%%%%
%
% Package pgfplots.sty documentation.
%
% Copyright 2007/2008 by Christian Feuersaenger.
%
% This program is free software: you can redistribute it and/or modify
% it under the terms of the GNU General Public License as published by
% the Free Software Foundation, either version 3 of the License, or
% (at your option) any later version.
%
% This program is distributed in the hope that it will be useful,
% but WITHOUT ANY WARRANTY; without even the implied warranty of
% MERCHANTABILITY or FITNESS FOR A PARTICULAR PURPOSE.  See the
% GNU General Public License for more details.
%
% You should have received a copy of the GNU General Public License
% along with this program.  If not, see <http://www.gnu.org/licenses/>.
%
%
%%%%%%%%%%%%%%%%%%%%%%%%%%%%%%%%%%%%%%%%%%%%%%%%%%%%%%%%%%%%%%%%%%%%%%%%%%%%%
% SEE pgfplots-macros.tex as well!
%\pdfminorversion=5 % to allow compression
%\pdfobjcompresslevel=2
\documentclass[a4paper,openany]{book}

\let\bookmaketitle=\maketitle

% -----------------------------------
% this here is from ltxdoc:
\usepackage{doc}[=v2]
\AtBeginDocument{\MakeShortVerb{\|}}
%\setlength{\textwidth}{355pt}
%\addtolength\marginparwidth{30pt}
%\addtolength\oddsidemargin{20pt}
%\addtolength\evensidemargin{20pt}
\def\cmd#1{\cs{\expandafter\cmd@to@cs\string#1}}
\def\cmd@to@cs#1#2{\char\number`#2\relax}
\DeclareRobustCommand\cs[1]{\texttt{\char`\\#1}}
\providecommand\marg[1]{%
  {\ttfamily\char`\{}\meta{#1}{\ttfamily\char`\}}}
\providecommand\oarg[1]{%
  {\ttfamily[}\meta{#1}{\ttfamily]}}
\providecommand\parg[1]{%
  {\ttfamily(}\meta{#1}{\ttfamily)}}
\raggedbottom

% -----------------------------------
\input{pgfplots.preamble.tex}

\makeatletter
% I want a two-column index just as in pgfmanual styles. This here
% was the best way to get one:
\def\index@prologue{\section*{Index}\addcontentsline{toc}{chapter}{Index}
}
\makeatother

%\RequirePackage[german,english,francais]{babel}

\def\matlabcolormaptext{This colormap is similar to one shipped with Matlab$^\text{\textregistered}$ under a similar name.}

\IfFileExists{tikzlibraryspy.code.tex}{%
\usetikzlibrary{spy}
}{%
    \message{ERROR: tikz SPY library NOT available. The manual will only compile partially.^^J}%
}%

\usepackage{xparse}% for colorbrewer manual

\usetikzlibrary{
    decorations.markings,
    decorations.footprints,
    shapes.arrows,
    matrix,
    positioning,
}

\usepgfplotslibrary{
    fillbetween,
    ternary,
    smithchart,
    patchplots,
    polar,
    colormaps,
    colorbrewer,
    colortol,           % docu not ready yet
}
\pgfqkeys{/codeexample}{%
    every codeexample/.append style={
        /pgfplots/every ternary axis/.append style={
            /pgfplots/legend style={fill=graphicbackground},
        }
    },
    tabsize=4,
}

\pgfplotsmanualenableexternalizationofexpensive

%\usetikzlibrary{external}
%\tikzexternalize[prefix=figures/]{pgfplots}

\title{%
    Manual for Package \PGFPlots{}\\
    {\small 2D/3D Plots in \LaTeX{}, Version \pgfplotsversion}\\
    {\small\url{https://github.com/pgf-tikz/pgfplots}}
    %\\{\small Attention: you are using an unstable development version.}
}

\makeatletter
\long\def\abstractsmuggle{%
    \centering
    \textbf{Abstract}\\[0.5cm]

    \begin{minipage}{12cm}
        \PGFPlots{} draws high-quality function plots in normal or logarithmic
        scaling with a user-friendly interface directly in \TeX{}. The user
        supplies axis labels, legend entries and the plot coordinates for one or
        more plots and \PGFPlots{} applies axis scaling, computes any logarithms
        and axis ticks and draws the plots. It supports line plots, scatter
        plots, piecewise constant plots, bar plots, area plots, mesh and surface
        plots, patch plots, contour plots, quiver plots, histogram plots, box
        plots, polar axes, ternary diagrams, smith charts and some more. It is
        based on Till Tantau's package \PGF{}/\Tikz{}.
    \end{minipage}

}%

\expandafter\date\expandafter{\@date\\[2cm]
    \abstractsmuggle
}%
\makeatother

%\includeonly{pgfplots.reference}


\begin{document}

\def\plotcoords{%
\addplot coordinates {
(5,8.312e-02)    (17,2.547e-02)   (49,7.407e-03)
(129,2.102e-03)  (321,5.874e-04)  (769,1.623e-04)
(1793,4.442e-05) (4097,1.207e-05) (9217,3.261e-06)
};

\addplot coordinates{
(7,8.472e-02)    (31,3.044e-02)    (111,1.022e-02)
(351,3.303e-03)  (1023,1.039e-03)  (2815,3.196e-04)
(7423,9.658e-05) (18943,2.873e-05) (47103,8.437e-06)
};

\addplot coordinates{
(9,7.881e-02)     (49,3.243e-02)    (209,1.232e-02)
(769,4.454e-03)   (2561,1.551e-03)  (7937,5.236e-04)
(23297,1.723e-04) (65537,5.545e-05) (178177,1.751e-05)
};

\addplot coordinates{
(11,6.887e-02)    (71,3.177e-02)     (351,1.341e-02)
(1471,5.334e-03)  (5503,2.027e-03)   (18943,7.415e-04)
(61183,2.628e-04) (187903,9.063e-05) (553983,3.053e-05)
};

\addplot coordinates{
(13,5.755e-02)     (97,2.925e-02)     (545,1.351e-02)
(2561,5.842e-03)   (10625,2.397e-03)  (40193,9.414e-04)
(141569,3.564e-04) (471041,1.308e-04)
(1496065,4.670e-05)
};
}%


\bookmaketitle
\tableofcontents
\include{pgfplots.title_abstract_intro}
\include{pgfplots.preliminaries}
\include{pgfplots.intro}
\include{pgfplots.reference}
\include{pgfplots.libs}
\include{pgfplots.resources}
\include{pgfplots.importexport}
\include{pgfplots.basic.reference}

\printindex

\bibliographystyle{abbrv} %gerapali} %gerabbrv} %gerunsrt.bst} %gerabbrv}% gerplain}
\nocite{pgfplotstable}
\nocite{programmingnotes}
\bibliography{pgfplots}
\end{document}

% -----------------------------------------------------------------------------
% For Stefan Pinnow as reminder on what to look for when editing the manual
% -----------------------------------------------------------------------------
% There should be no line breaks in the following environments
% - |...|
% - \declareandlabel{...}
% - \verbpdfref{...}
% If "MakeTikzPictures" isn't running through check one of the externalized
% LOG files what is the cause of that.
% -----------------------------------------------------------------------------


%%%%%%%%%%%%%%%%%%%%%%%%%%%%%%%%%%%%%%%%%%%%%%%%%%%%%%%%%%%%%%%%%%%%%%%%%%%%%
%
% Package pgfplots.sty documentation.
%
% Copyright 2007/2008 by Christian Feuersaenger.
%
% This program is free software: you can redistribute it and/or modify
% it under the terms of the GNU General Public License as published by
% the Free Software Foundation, either version 3 of the License, or
% (at your option) any later version.
%
% This program is distributed in the hope that it will be useful,
% but WITHOUT ANY WARRANTY; without even the implied warranty of
% MERCHANTABILITY or FITNESS FOR A PARTICULAR PURPOSE.  See the
% GNU General Public License for more details.
%
% You should have received a copy of the GNU General Public License
% along with this program.  If not, see <http://www.gnu.org/licenses/>.
%
%
%%%%%%%%%%%%%%%%%%%%%%%%%%%%%%%%%%%%%%%%%%%%%%%%%%%%%%%%%%%%%%%%%%%%%%%%%%%%%
% SEE pgfplots-macros.tex as well!
%\pdfminorversion=5 % to allow compression
%\pdfobjcompresslevel=2
\documentclass[a4paper,openany]{book}

\let\bookmaketitle=\maketitle

% -----------------------------------
% this here is from ltxdoc:
\usepackage{doc}[=v2]
\AtBeginDocument{\MakeShortVerb{\|}}
%\setlength{\textwidth}{355pt}
%\addtolength\marginparwidth{30pt}
%\addtolength\oddsidemargin{20pt}
%\addtolength\evensidemargin{20pt}
\def\cmd#1{\cs{\expandafter\cmd@to@cs\string#1}}
\def\cmd@to@cs#1#2{\char\number`#2\relax}
\DeclareRobustCommand\cs[1]{\texttt{\char`\\#1}}
\providecommand\marg[1]{%
  {\ttfamily\char`\{}\meta{#1}{\ttfamily\char`\}}}
\providecommand\oarg[1]{%
  {\ttfamily[}\meta{#1}{\ttfamily]}}
\providecommand\parg[1]{%
  {\ttfamily(}\meta{#1}{\ttfamily)}}
\raggedbottom

% -----------------------------------
\input{pgfplots.preamble.tex}

\makeatletter
% I want a two-column index just as in pgfmanual styles. This here
% was the best way to get one:
\def\index@prologue{\section*{Index}\addcontentsline{toc}{chapter}{Index}
}
\makeatother

%\RequirePackage[german,english,francais]{babel}

\def\matlabcolormaptext{This colormap is similar to one shipped with Matlab$^\text{\textregistered}$ under a similar name.}

\IfFileExists{tikzlibraryspy.code.tex}{%
\usetikzlibrary{spy}
}{%
    \message{ERROR: tikz SPY library NOT available. The manual will only compile partially.^^J}%
}%

\usepackage{xparse}% for colorbrewer manual

\usetikzlibrary{
    decorations.markings,
    decorations.footprints,
    shapes.arrows,
    matrix,
    positioning,
}

\usepgfplotslibrary{
    fillbetween,
    ternary,
    smithchart,
    patchplots,
    polar,
    colormaps,
    colorbrewer,
    colortol,           % docu not ready yet
}
\pgfqkeys{/codeexample}{%
    every codeexample/.append style={
        /pgfplots/every ternary axis/.append style={
            /pgfplots/legend style={fill=graphicbackground},
        }
    },
    tabsize=4,
}

\pgfplotsmanualenableexternalizationofexpensive

%\usetikzlibrary{external}
%\tikzexternalize[prefix=figures/]{pgfplots}

\title{%
    Manual for Package \PGFPlots{}\\
    {\small 2D/3D Plots in \LaTeX{}, Version \pgfplotsversion}\\
    {\small\url{https://github.com/pgf-tikz/pgfplots}}
    %\\{\small Attention: you are using an unstable development version.}
}

\makeatletter
\long\def\abstractsmuggle{%
    \centering
    \textbf{Abstract}\\[0.5cm]

    \begin{minipage}{12cm}
        \PGFPlots{} draws high-quality function plots in normal or logarithmic
        scaling with a user-friendly interface directly in \TeX{}. The user
        supplies axis labels, legend entries and the plot coordinates for one or
        more plots and \PGFPlots{} applies axis scaling, computes any logarithms
        and axis ticks and draws the plots. It supports line plots, scatter
        plots, piecewise constant plots, bar plots, area plots, mesh and surface
        plots, patch plots, contour plots, quiver plots, histogram plots, box
        plots, polar axes, ternary diagrams, smith charts and some more. It is
        based on Till Tantau's package \PGF{}/\Tikz{}.
    \end{minipage}

}%

\expandafter\date\expandafter{\@date\\[2cm]
    \abstractsmuggle
}%
\makeatother

%\includeonly{pgfplots.reference}


\begin{document}

\def\plotcoords{%
\addplot coordinates {
(5,8.312e-02)    (17,2.547e-02)   (49,7.407e-03)
(129,2.102e-03)  (321,5.874e-04)  (769,1.623e-04)
(1793,4.442e-05) (4097,1.207e-05) (9217,3.261e-06)
};

\addplot coordinates{
(7,8.472e-02)    (31,3.044e-02)    (111,1.022e-02)
(351,3.303e-03)  (1023,1.039e-03)  (2815,3.196e-04)
(7423,9.658e-05) (18943,2.873e-05) (47103,8.437e-06)
};

\addplot coordinates{
(9,7.881e-02)     (49,3.243e-02)    (209,1.232e-02)
(769,4.454e-03)   (2561,1.551e-03)  (7937,5.236e-04)
(23297,1.723e-04) (65537,5.545e-05) (178177,1.751e-05)
};

\addplot coordinates{
(11,6.887e-02)    (71,3.177e-02)     (351,1.341e-02)
(1471,5.334e-03)  (5503,2.027e-03)   (18943,7.415e-04)
(61183,2.628e-04) (187903,9.063e-05) (553983,3.053e-05)
};

\addplot coordinates{
(13,5.755e-02)     (97,2.925e-02)     (545,1.351e-02)
(2561,5.842e-03)   (10625,2.397e-03)  (40193,9.414e-04)
(141569,3.564e-04) (471041,1.308e-04)
(1496065,4.670e-05)
};
}%


\bookmaketitle
\tableofcontents
\include{pgfplots.title_abstract_intro}
\include{pgfplots.preliminaries}
\include{pgfplots.intro}
\include{pgfplots.reference}
\include{pgfplots.libs}
\include{pgfplots.resources}
\include{pgfplots.importexport}
\include{pgfplots.basic.reference}

\printindex

\bibliographystyle{abbrv} %gerapali} %gerabbrv} %gerunsrt.bst} %gerabbrv}% gerplain}
\nocite{pgfplotstable}
\nocite{programmingnotes}
\bibliography{pgfplots}
\end{document}

% -----------------------------------------------------------------------------
% For Stefan Pinnow as reminder on what to look for when editing the manual
% -----------------------------------------------------------------------------
% There should be no line breaks in the following environments
% - |...|
% - \declareandlabel{...}
% - \verbpdfref{...}
% If "MakeTikzPictures" isn't running through check one of the externalized
% LOG files what is the cause of that.
% -----------------------------------------------------------------------------


%%%%%%%%%%%%%%%%%%%%%%%%%%%%%%%%%%%%%%%%%%%%%%%%%%%%%%%%%%%%%%%%%%%%%%%%%%%%%
%
% Package pgfplots.sty documentation.
%
% Copyright 2007/2008 by Christian Feuersaenger.
%
% This program is free software: you can redistribute it and/or modify
% it under the terms of the GNU General Public License as published by
% the Free Software Foundation, either version 3 of the License, or
% (at your option) any later version.
%
% This program is distributed in the hope that it will be useful,
% but WITHOUT ANY WARRANTY; without even the implied warranty of
% MERCHANTABILITY or FITNESS FOR A PARTICULAR PURPOSE.  See the
% GNU General Public License for more details.
%
% You should have received a copy of the GNU General Public License
% along with this program.  If not, see <http://www.gnu.org/licenses/>.
%
%
%%%%%%%%%%%%%%%%%%%%%%%%%%%%%%%%%%%%%%%%%%%%%%%%%%%%%%%%%%%%%%%%%%%%%%%%%%%%%
% SEE pgfplots-macros.tex as well!
%\pdfminorversion=5 % to allow compression
%\pdfobjcompresslevel=2
\documentclass[a4paper,openany]{book}

\let\bookmaketitle=\maketitle

% -----------------------------------
% this here is from ltxdoc:
\usepackage{doc}[=v2]
\AtBeginDocument{\MakeShortVerb{\|}}
%\setlength{\textwidth}{355pt}
%\addtolength\marginparwidth{30pt}
%\addtolength\oddsidemargin{20pt}
%\addtolength\evensidemargin{20pt}
\def\cmd#1{\cs{\expandafter\cmd@to@cs\string#1}}
\def\cmd@to@cs#1#2{\char\number`#2\relax}
\DeclareRobustCommand\cs[1]{\texttt{\char`\\#1}}
\providecommand\marg[1]{%
  {\ttfamily\char`\{}\meta{#1}{\ttfamily\char`\}}}
\providecommand\oarg[1]{%
  {\ttfamily[}\meta{#1}{\ttfamily]}}
\providecommand\parg[1]{%
  {\ttfamily(}\meta{#1}{\ttfamily)}}
\raggedbottom

% -----------------------------------
\input{pgfplots.preamble.tex}

\makeatletter
% I want a two-column index just as in pgfmanual styles. This here
% was the best way to get one:
\def\index@prologue{\section*{Index}\addcontentsline{toc}{chapter}{Index}
}
\makeatother

%\RequirePackage[german,english,francais]{babel}

\def\matlabcolormaptext{This colormap is similar to one shipped with Matlab$^\text{\textregistered}$ under a similar name.}

\IfFileExists{tikzlibraryspy.code.tex}{%
\usetikzlibrary{spy}
}{%
    \message{ERROR: tikz SPY library NOT available. The manual will only compile partially.^^J}%
}%

\usepackage{xparse}% for colorbrewer manual

\usetikzlibrary{
    decorations.markings,
    decorations.footprints,
    shapes.arrows,
    matrix,
    positioning,
}

\usepgfplotslibrary{
    fillbetween,
    ternary,
    smithchart,
    patchplots,
    polar,
    colormaps,
    colorbrewer,
    colortol,           % docu not ready yet
}
\pgfqkeys{/codeexample}{%
    every codeexample/.append style={
        /pgfplots/every ternary axis/.append style={
            /pgfplots/legend style={fill=graphicbackground},
        }
    },
    tabsize=4,
}

\pgfplotsmanualenableexternalizationofexpensive

%\usetikzlibrary{external}
%\tikzexternalize[prefix=figures/]{pgfplots}

\title{%
    Manual for Package \PGFPlots{}\\
    {\small 2D/3D Plots in \LaTeX{}, Version \pgfplotsversion}\\
    {\small\url{https://github.com/pgf-tikz/pgfplots}}
    %\\{\small Attention: you are using an unstable development version.}
}

\makeatletter
\long\def\abstractsmuggle{%
    \centering
    \textbf{Abstract}\\[0.5cm]

    \begin{minipage}{12cm}
        \PGFPlots{} draws high-quality function plots in normal or logarithmic
        scaling with a user-friendly interface directly in \TeX{}. The user
        supplies axis labels, legend entries and the plot coordinates for one or
        more plots and \PGFPlots{} applies axis scaling, computes any logarithms
        and axis ticks and draws the plots. It supports line plots, scatter
        plots, piecewise constant plots, bar plots, area plots, mesh and surface
        plots, patch plots, contour plots, quiver plots, histogram plots, box
        plots, polar axes, ternary diagrams, smith charts and some more. It is
        based on Till Tantau's package \PGF{}/\Tikz{}.
    \end{minipage}

}%

\expandafter\date\expandafter{\@date\\[2cm]
    \abstractsmuggle
}%
\makeatother

%\includeonly{pgfplots.reference}


\begin{document}

\def\plotcoords{%
\addplot coordinates {
(5,8.312e-02)    (17,2.547e-02)   (49,7.407e-03)
(129,2.102e-03)  (321,5.874e-04)  (769,1.623e-04)
(1793,4.442e-05) (4097,1.207e-05) (9217,3.261e-06)
};

\addplot coordinates{
(7,8.472e-02)    (31,3.044e-02)    (111,1.022e-02)
(351,3.303e-03)  (1023,1.039e-03)  (2815,3.196e-04)
(7423,9.658e-05) (18943,2.873e-05) (47103,8.437e-06)
};

\addplot coordinates{
(9,7.881e-02)     (49,3.243e-02)    (209,1.232e-02)
(769,4.454e-03)   (2561,1.551e-03)  (7937,5.236e-04)
(23297,1.723e-04) (65537,5.545e-05) (178177,1.751e-05)
};

\addplot coordinates{
(11,6.887e-02)    (71,3.177e-02)     (351,1.341e-02)
(1471,5.334e-03)  (5503,2.027e-03)   (18943,7.415e-04)
(61183,2.628e-04) (187903,9.063e-05) (553983,3.053e-05)
};

\addplot coordinates{
(13,5.755e-02)     (97,2.925e-02)     (545,1.351e-02)
(2561,5.842e-03)   (10625,2.397e-03)  (40193,9.414e-04)
(141569,3.564e-04) (471041,1.308e-04)
(1496065,4.670e-05)
};
}%


\bookmaketitle
\tableofcontents
\include{pgfplots.title_abstract_intro}
\include{pgfplots.preliminaries}
\include{pgfplots.intro}
\include{pgfplots.reference}
\include{pgfplots.libs}
\include{pgfplots.resources}
\include{pgfplots.importexport}
\include{pgfplots.basic.reference}

\printindex

\bibliographystyle{abbrv} %gerapali} %gerabbrv} %gerunsrt.bst} %gerabbrv}% gerplain}
\nocite{pgfplotstable}
\nocite{programmingnotes}
\bibliography{pgfplots}
\end{document}

% -----------------------------------------------------------------------------
% For Stefan Pinnow as reminder on what to look for when editing the manual
% -----------------------------------------------------------------------------
% There should be no line breaks in the following environments
% - |...|
% - \declareandlabel{...}
% - \verbpdfref{...}
% If "MakeTikzPictures" isn't running through check one of the externalized
% LOG files what is the cause of that.
% -----------------------------------------------------------------------------


%%%%%%%%%%%%%%%%%%%%%%%%%%%%%%%%%%%%%%%%%%%%%%%%%%%%%%%%%%%%%%%%%%%%%%%%%%%%%
%
% Package pgfplots.sty documentation.
%
% Copyright 2007/2008 by Christian Feuersaenger.
%
% This program is free software: you can redistribute it and/or modify
% it under the terms of the GNU General Public License as published by
% the Free Software Foundation, either version 3 of the License, or
% (at your option) any later version.
%
% This program is distributed in the hope that it will be useful,
% but WITHOUT ANY WARRANTY; without even the implied warranty of
% MERCHANTABILITY or FITNESS FOR A PARTICULAR PURPOSE.  See the
% GNU General Public License for more details.
%
% You should have received a copy of the GNU General Public License
% along with this program.  If not, see <http://www.gnu.org/licenses/>.
%
%
%%%%%%%%%%%%%%%%%%%%%%%%%%%%%%%%%%%%%%%%%%%%%%%%%%%%%%%%%%%%%%%%%%%%%%%%%%%%%
% SEE pgfplots-macros.tex as well!
%\pdfminorversion=5 % to allow compression
%\pdfobjcompresslevel=2
\documentclass[a4paper,openany]{book}

\let\bookmaketitle=\maketitle

% -----------------------------------
% this here is from ltxdoc:
\usepackage{doc}[=v2]
\AtBeginDocument{\MakeShortVerb{\|}}
%\setlength{\textwidth}{355pt}
%\addtolength\marginparwidth{30pt}
%\addtolength\oddsidemargin{20pt}
%\addtolength\evensidemargin{20pt}
\def\cmd#1{\cs{\expandafter\cmd@to@cs\string#1}}
\def\cmd@to@cs#1#2{\char\number`#2\relax}
\DeclareRobustCommand\cs[1]{\texttt{\char`\\#1}}
\providecommand\marg[1]{%
  {\ttfamily\char`\{}\meta{#1}{\ttfamily\char`\}}}
\providecommand\oarg[1]{%
  {\ttfamily[}\meta{#1}{\ttfamily]}}
\providecommand\parg[1]{%
  {\ttfamily(}\meta{#1}{\ttfamily)}}
\raggedbottom

% -----------------------------------
\input{pgfplots.preamble.tex}

\makeatletter
% I want a two-column index just as in pgfmanual styles. This here
% was the best way to get one:
\def\index@prologue{\section*{Index}\addcontentsline{toc}{chapter}{Index}
}
\makeatother

%\RequirePackage[german,english,francais]{babel}

\def\matlabcolormaptext{This colormap is similar to one shipped with Matlab$^\text{\textregistered}$ under a similar name.}

\IfFileExists{tikzlibraryspy.code.tex}{%
\usetikzlibrary{spy}
}{%
    \message{ERROR: tikz SPY library NOT available. The manual will only compile partially.^^J}%
}%

\usepackage{xparse}% for colorbrewer manual

\usetikzlibrary{
    decorations.markings,
    decorations.footprints,
    shapes.arrows,
    matrix,
    positioning,
}

\usepgfplotslibrary{
    fillbetween,
    ternary,
    smithchart,
    patchplots,
    polar,
    colormaps,
    colorbrewer,
    colortol,           % docu not ready yet
}
\pgfqkeys{/codeexample}{%
    every codeexample/.append style={
        /pgfplots/every ternary axis/.append style={
            /pgfplots/legend style={fill=graphicbackground},
        }
    },
    tabsize=4,
}

\pgfplotsmanualenableexternalizationofexpensive

%\usetikzlibrary{external}
%\tikzexternalize[prefix=figures/]{pgfplots}

\title{%
    Manual for Package \PGFPlots{}\\
    {\small 2D/3D Plots in \LaTeX{}, Version \pgfplotsversion}\\
    {\small\url{https://github.com/pgf-tikz/pgfplots}}
    %\\{\small Attention: you are using an unstable development version.}
}

\makeatletter
\long\def\abstractsmuggle{%
    \centering
    \textbf{Abstract}\\[0.5cm]

    \begin{minipage}{12cm}
        \PGFPlots{} draws high-quality function plots in normal or logarithmic
        scaling with a user-friendly interface directly in \TeX{}. The user
        supplies axis labels, legend entries and the plot coordinates for one or
        more plots and \PGFPlots{} applies axis scaling, computes any logarithms
        and axis ticks and draws the plots. It supports line plots, scatter
        plots, piecewise constant plots, bar plots, area plots, mesh and surface
        plots, patch plots, contour plots, quiver plots, histogram plots, box
        plots, polar axes, ternary diagrams, smith charts and some more. It is
        based on Till Tantau's package \PGF{}/\Tikz{}.
    \end{minipage}

}%

\expandafter\date\expandafter{\@date\\[2cm]
    \abstractsmuggle
}%
\makeatother

%\includeonly{pgfplots.reference}


\begin{document}

\def\plotcoords{%
\addplot coordinates {
(5,8.312e-02)    (17,2.547e-02)   (49,7.407e-03)
(129,2.102e-03)  (321,5.874e-04)  (769,1.623e-04)
(1793,4.442e-05) (4097,1.207e-05) (9217,3.261e-06)
};

\addplot coordinates{
(7,8.472e-02)    (31,3.044e-02)    (111,1.022e-02)
(351,3.303e-03)  (1023,1.039e-03)  (2815,3.196e-04)
(7423,9.658e-05) (18943,2.873e-05) (47103,8.437e-06)
};

\addplot coordinates{
(9,7.881e-02)     (49,3.243e-02)    (209,1.232e-02)
(769,4.454e-03)   (2561,1.551e-03)  (7937,5.236e-04)
(23297,1.723e-04) (65537,5.545e-05) (178177,1.751e-05)
};

\addplot coordinates{
(11,6.887e-02)    (71,3.177e-02)     (351,1.341e-02)
(1471,5.334e-03)  (5503,2.027e-03)   (18943,7.415e-04)
(61183,2.628e-04) (187903,9.063e-05) (553983,3.053e-05)
};

\addplot coordinates{
(13,5.755e-02)     (97,2.925e-02)     (545,1.351e-02)
(2561,5.842e-03)   (10625,2.397e-03)  (40193,9.414e-04)
(141569,3.564e-04) (471041,1.308e-04)
(1496065,4.670e-05)
};
}%


\bookmaketitle
\tableofcontents
\include{pgfplots.title_abstract_intro}
\include{pgfplots.preliminaries}
\include{pgfplots.intro}
\include{pgfplots.reference}
\include{pgfplots.libs}
\include{pgfplots.resources}
\include{pgfplots.importexport}
\include{pgfplots.basic.reference}

\printindex

\bibliographystyle{abbrv} %gerapali} %gerabbrv} %gerunsrt.bst} %gerabbrv}% gerplain}
\nocite{pgfplotstable}
\nocite{programmingnotes}
\bibliography{pgfplots}
\end{document}

% -----------------------------------------------------------------------------
% For Stefan Pinnow as reminder on what to look for when editing the manual
% -----------------------------------------------------------------------------
% There should be no line breaks in the following environments
% - |...|
% - \declareandlabel{...}
% - \verbpdfref{...}
% If "MakeTikzPictures" isn't running through check one of the externalized
% LOG files what is the cause of that.
% -----------------------------------------------------------------------------


%%%%%%%%%%%%%%%%%%%%%%%%%%%%%%%%%%%%%%%%%%%%%%%%%%%%%%%%%%%%%%%%%%%%%%%%%%%%%
%
% Package pgfplots.sty documentation.
%
% Copyright 2007/2008 by Christian Feuersaenger.
%
% This program is free software: you can redistribute it and/or modify
% it under the terms of the GNU General Public License as published by
% the Free Software Foundation, either version 3 of the License, or
% (at your option) any later version.
%
% This program is distributed in the hope that it will be useful,
% but WITHOUT ANY WARRANTY; without even the implied warranty of
% MERCHANTABILITY or FITNESS FOR A PARTICULAR PURPOSE.  See the
% GNU General Public License for more details.
%
% You should have received a copy of the GNU General Public License
% along with this program.  If not, see <http://www.gnu.org/licenses/>.
%
%
%%%%%%%%%%%%%%%%%%%%%%%%%%%%%%%%%%%%%%%%%%%%%%%%%%%%%%%%%%%%%%%%%%%%%%%%%%%%%
% SEE pgfplots-macros.tex as well!
%\pdfminorversion=5 % to allow compression
%\pdfobjcompresslevel=2
\documentclass[a4paper,openany]{book}

\let\bookmaketitle=\maketitle

% -----------------------------------
% this here is from ltxdoc:
\usepackage{doc}[=v2]
\AtBeginDocument{\MakeShortVerb{\|}}
%\setlength{\textwidth}{355pt}
%\addtolength\marginparwidth{30pt}
%\addtolength\oddsidemargin{20pt}
%\addtolength\evensidemargin{20pt}
\def\cmd#1{\cs{\expandafter\cmd@to@cs\string#1}}
\def\cmd@to@cs#1#2{\char\number`#2\relax}
\DeclareRobustCommand\cs[1]{\texttt{\char`\\#1}}
\providecommand\marg[1]{%
  {\ttfamily\char`\{}\meta{#1}{\ttfamily\char`\}}}
\providecommand\oarg[1]{%
  {\ttfamily[}\meta{#1}{\ttfamily]}}
\providecommand\parg[1]{%
  {\ttfamily(}\meta{#1}{\ttfamily)}}
\raggedbottom

% -----------------------------------
\input{pgfplots.preamble.tex}

\makeatletter
% I want a two-column index just as in pgfmanual styles. This here
% was the best way to get one:
\def\index@prologue{\section*{Index}\addcontentsline{toc}{chapter}{Index}
}
\makeatother

%\RequirePackage[german,english,francais]{babel}

\def\matlabcolormaptext{This colormap is similar to one shipped with Matlab$^\text{\textregistered}$ under a similar name.}

\IfFileExists{tikzlibraryspy.code.tex}{%
\usetikzlibrary{spy}
}{%
    \message{ERROR: tikz SPY library NOT available. The manual will only compile partially.^^J}%
}%

\usepackage{xparse}% for colorbrewer manual

\usetikzlibrary{
    decorations.markings,
    decorations.footprints,
    shapes.arrows,
    matrix,
    positioning,
}

\usepgfplotslibrary{
    fillbetween,
    ternary,
    smithchart,
    patchplots,
    polar,
    colormaps,
    colorbrewer,
    colortol,           % docu not ready yet
}
\pgfqkeys{/codeexample}{%
    every codeexample/.append style={
        /pgfplots/every ternary axis/.append style={
            /pgfplots/legend style={fill=graphicbackground},
        }
    },
    tabsize=4,
}

\pgfplotsmanualenableexternalizationofexpensive

%\usetikzlibrary{external}
%\tikzexternalize[prefix=figures/]{pgfplots}

\title{%
    Manual for Package \PGFPlots{}\\
    {\small 2D/3D Plots in \LaTeX{}, Version \pgfplotsversion}\\
    {\small\url{https://github.com/pgf-tikz/pgfplots}}
    %\\{\small Attention: you are using an unstable development version.}
}

\makeatletter
\long\def\abstractsmuggle{%
    \centering
    \textbf{Abstract}\\[0.5cm]

    \begin{minipage}{12cm}
        \PGFPlots{} draws high-quality function plots in normal or logarithmic
        scaling with a user-friendly interface directly in \TeX{}. The user
        supplies axis labels, legend entries and the plot coordinates for one or
        more plots and \PGFPlots{} applies axis scaling, computes any logarithms
        and axis ticks and draws the plots. It supports line plots, scatter
        plots, piecewise constant plots, bar plots, area plots, mesh and surface
        plots, patch plots, contour plots, quiver plots, histogram plots, box
        plots, polar axes, ternary diagrams, smith charts and some more. It is
        based on Till Tantau's package \PGF{}/\Tikz{}.
    \end{minipage}

}%

\expandafter\date\expandafter{\@date\\[2cm]
    \abstractsmuggle
}%
\makeatother

%\includeonly{pgfplots.reference}


\begin{document}

\def\plotcoords{%
\addplot coordinates {
(5,8.312e-02)    (17,2.547e-02)   (49,7.407e-03)
(129,2.102e-03)  (321,5.874e-04)  (769,1.623e-04)
(1793,4.442e-05) (4097,1.207e-05) (9217,3.261e-06)
};

\addplot coordinates{
(7,8.472e-02)    (31,3.044e-02)    (111,1.022e-02)
(351,3.303e-03)  (1023,1.039e-03)  (2815,3.196e-04)
(7423,9.658e-05) (18943,2.873e-05) (47103,8.437e-06)
};

\addplot coordinates{
(9,7.881e-02)     (49,3.243e-02)    (209,1.232e-02)
(769,4.454e-03)   (2561,1.551e-03)  (7937,5.236e-04)
(23297,1.723e-04) (65537,5.545e-05) (178177,1.751e-05)
};

\addplot coordinates{
(11,6.887e-02)    (71,3.177e-02)     (351,1.341e-02)
(1471,5.334e-03)  (5503,2.027e-03)   (18943,7.415e-04)
(61183,2.628e-04) (187903,9.063e-05) (553983,3.053e-05)
};

\addplot coordinates{
(13,5.755e-02)     (97,2.925e-02)     (545,1.351e-02)
(2561,5.842e-03)   (10625,2.397e-03)  (40193,9.414e-04)
(141569,3.564e-04) (471041,1.308e-04)
(1496065,4.670e-05)
};
}%


\bookmaketitle
\tableofcontents
\include{pgfplots.title_abstract_intro}
\include{pgfplots.preliminaries}
\include{pgfplots.intro}
\include{pgfplots.reference}
\include{pgfplots.libs}
\include{pgfplots.resources}
\include{pgfplots.importexport}
\include{pgfplots.basic.reference}

\printindex

\bibliographystyle{abbrv} %gerapali} %gerabbrv} %gerunsrt.bst} %gerabbrv}% gerplain}
\nocite{pgfplotstable}
\nocite{programmingnotes}
\bibliography{pgfplots}
\end{document}

% -----------------------------------------------------------------------------
% For Stefan Pinnow as reminder on what to look for when editing the manual
% -----------------------------------------------------------------------------
% There should be no line breaks in the following environments
% - |...|
% - \declareandlabel{...}
% - \verbpdfref{...}
% If "MakeTikzPictures" isn't running through check one of the externalized
% LOG files what is the cause of that.
% -----------------------------------------------------------------------------

\include{pgfplots.basic.reference}

\printindex

\bibliographystyle{abbrv} %gerapali} %gerabbrv} %gerunsrt.bst} %gerabbrv}% gerplain}
\nocite{pgfplotstable}
\nocite{programmingnotes}
\bibliography{pgfplots}
\end{document}

% -----------------------------------------------------------------------------
% For Stefan Pinnow as reminder on what to look for when editing the manual
% -----------------------------------------------------------------------------
% There should be no line breaks in the following environments
% - |...|
% - \declareandlabel{...}
% - \verbpdfref{...}
% If "MakeTikzPictures" isn't running through check one of the externalized
% LOG files what is the cause of that.
% -----------------------------------------------------------------------------


%%%%%%%%%%%%%%%%%%%%%%%%%%%%%%%%%%%%%%%%%%%%%%%%%%%%%%%%%%%%%%%%%%%%%%%%%%%%%
%
% Package pgfplots.sty documentation.
%
% Copyright 2007/2008 by Christian Feuersaenger.
%
% This program is free software: you can redistribute it and/or modify
% it under the terms of the GNU General Public License as published by
% the Free Software Foundation, either version 3 of the License, or
% (at your option) any later version.
%
% This program is distributed in the hope that it will be useful,
% but WITHOUT ANY WARRANTY; without even the implied warranty of
% MERCHANTABILITY or FITNESS FOR A PARTICULAR PURPOSE.  See the
% GNU General Public License for more details.
%
% You should have received a copy of the GNU General Public License
% along with this program.  If not, see <http://www.gnu.org/licenses/>.
%
%
%%%%%%%%%%%%%%%%%%%%%%%%%%%%%%%%%%%%%%%%%%%%%%%%%%%%%%%%%%%%%%%%%%%%%%%%%%%%%
% SEE pgfplots-macros.tex as well!
%\pdfminorversion=5 % to allow compression
%\pdfobjcompresslevel=2
\documentclass[a4paper,openany]{book}

\let\bookmaketitle=\maketitle

% -----------------------------------
% this here is from ltxdoc:
\usepackage{doc}[=v2]
\AtBeginDocument{\MakeShortVerb{\|}}
%\setlength{\textwidth}{355pt}
%\addtolength\marginparwidth{30pt}
%\addtolength\oddsidemargin{20pt}
%\addtolength\evensidemargin{20pt}
\def\cmd#1{\cs{\expandafter\cmd@to@cs\string#1}}
\def\cmd@to@cs#1#2{\char\number`#2\relax}
\DeclareRobustCommand\cs[1]{\texttt{\char`\\#1}}
\providecommand\marg[1]{%
  {\ttfamily\char`\{}\meta{#1}{\ttfamily\char`\}}}
\providecommand\oarg[1]{%
  {\ttfamily[}\meta{#1}{\ttfamily]}}
\providecommand\parg[1]{%
  {\ttfamily(}\meta{#1}{\ttfamily)}}
\raggedbottom

% -----------------------------------
\input{pgfplots.preamble.tex}

\makeatletter
% I want a two-column index just as in pgfmanual styles. This here
% was the best way to get one:
\def\index@prologue{\section*{Index}\addcontentsline{toc}{chapter}{Index}
}
\makeatother

%\RequirePackage[german,english,francais]{babel}

\def\matlabcolormaptext{This colormap is similar to one shipped with Matlab$^\text{\textregistered}$ under a similar name.}

\IfFileExists{tikzlibraryspy.code.tex}{%
\usetikzlibrary{spy}
}{%
    \message{ERROR: tikz SPY library NOT available. The manual will only compile partially.^^J}%
}%

\usepackage{xparse}% for colorbrewer manual

\usetikzlibrary{
    decorations.markings,
    decorations.footprints,
    shapes.arrows,
    matrix,
    positioning,
}

\usepgfplotslibrary{
    fillbetween,
    ternary,
    smithchart,
    patchplots,
    polar,
    colormaps,
    colorbrewer,
    colortol,           % docu not ready yet
}
\pgfqkeys{/codeexample}{%
    every codeexample/.append style={
        /pgfplots/every ternary axis/.append style={
            /pgfplots/legend style={fill=graphicbackground},
        }
    },
    tabsize=4,
}

\pgfplotsmanualenableexternalizationofexpensive

%\usetikzlibrary{external}
%\tikzexternalize[prefix=figures/]{pgfplots}

\title{%
    Manual for Package \PGFPlots{}\\
    {\small 2D/3D Plots in \LaTeX{}, Version \pgfplotsversion}\\
    {\small\url{https://github.com/pgf-tikz/pgfplots}}
    %\\{\small Attention: you are using an unstable development version.}
}

\makeatletter
\long\def\abstractsmuggle{%
    \centering
    \textbf{Abstract}\\[0.5cm]

    \begin{minipage}{12cm}
        \PGFPlots{} draws high-quality function plots in normal or logarithmic
        scaling with a user-friendly interface directly in \TeX{}. The user
        supplies axis labels, legend entries and the plot coordinates for one or
        more plots and \PGFPlots{} applies axis scaling, computes any logarithms
        and axis ticks and draws the plots. It supports line plots, scatter
        plots, piecewise constant plots, bar plots, area plots, mesh and surface
        plots, patch plots, contour plots, quiver plots, histogram plots, box
        plots, polar axes, ternary diagrams, smith charts and some more. It is
        based on Till Tantau's package \PGF{}/\Tikz{}.
    \end{minipage}

}%

\expandafter\date\expandafter{\@date\\[2cm]
    \abstractsmuggle
}%
\makeatother

%\includeonly{pgfplots.reference}


\begin{document}

\def\plotcoords{%
\addplot coordinates {
(5,8.312e-02)    (17,2.547e-02)   (49,7.407e-03)
(129,2.102e-03)  (321,5.874e-04)  (769,1.623e-04)
(1793,4.442e-05) (4097,1.207e-05) (9217,3.261e-06)
};

\addplot coordinates{
(7,8.472e-02)    (31,3.044e-02)    (111,1.022e-02)
(351,3.303e-03)  (1023,1.039e-03)  (2815,3.196e-04)
(7423,9.658e-05) (18943,2.873e-05) (47103,8.437e-06)
};

\addplot coordinates{
(9,7.881e-02)     (49,3.243e-02)    (209,1.232e-02)
(769,4.454e-03)   (2561,1.551e-03)  (7937,5.236e-04)
(23297,1.723e-04) (65537,5.545e-05) (178177,1.751e-05)
};

\addplot coordinates{
(11,6.887e-02)    (71,3.177e-02)     (351,1.341e-02)
(1471,5.334e-03)  (5503,2.027e-03)   (18943,7.415e-04)
(61183,2.628e-04) (187903,9.063e-05) (553983,3.053e-05)
};

\addplot coordinates{
(13,5.755e-02)     (97,2.925e-02)     (545,1.351e-02)
(2561,5.842e-03)   (10625,2.397e-03)  (40193,9.414e-04)
(141569,3.564e-04) (471041,1.308e-04)
(1496065,4.670e-05)
};
}%


\bookmaketitle
\tableofcontents

%%%%%%%%%%%%%%%%%%%%%%%%%%%%%%%%%%%%%%%%%%%%%%%%%%%%%%%%%%%%%%%%%%%%%%%%%%%%%
%
% Package pgfplots.sty documentation.
%
% Copyright 2007/2008 by Christian Feuersaenger.
%
% This program is free software: you can redistribute it and/or modify
% it under the terms of the GNU General Public License as published by
% the Free Software Foundation, either version 3 of the License, or
% (at your option) any later version.
%
% This program is distributed in the hope that it will be useful,
% but WITHOUT ANY WARRANTY; without even the implied warranty of
% MERCHANTABILITY or FITNESS FOR A PARTICULAR PURPOSE.  See the
% GNU General Public License for more details.
%
% You should have received a copy of the GNU General Public License
% along with this program.  If not, see <http://www.gnu.org/licenses/>.
%
%
%%%%%%%%%%%%%%%%%%%%%%%%%%%%%%%%%%%%%%%%%%%%%%%%%%%%%%%%%%%%%%%%%%%%%%%%%%%%%
% SEE pgfplots-macros.tex as well!
%\pdfminorversion=5 % to allow compression
%\pdfobjcompresslevel=2
\documentclass[a4paper,openany]{book}

\let\bookmaketitle=\maketitle

% -----------------------------------
% this here is from ltxdoc:
\usepackage{doc}[=v2]
\AtBeginDocument{\MakeShortVerb{\|}}
%\setlength{\textwidth}{355pt}
%\addtolength\marginparwidth{30pt}
%\addtolength\oddsidemargin{20pt}
%\addtolength\evensidemargin{20pt}
\def\cmd#1{\cs{\expandafter\cmd@to@cs\string#1}}
\def\cmd@to@cs#1#2{\char\number`#2\relax}
\DeclareRobustCommand\cs[1]{\texttt{\char`\\#1}}
\providecommand\marg[1]{%
  {\ttfamily\char`\{}\meta{#1}{\ttfamily\char`\}}}
\providecommand\oarg[1]{%
  {\ttfamily[}\meta{#1}{\ttfamily]}}
\providecommand\parg[1]{%
  {\ttfamily(}\meta{#1}{\ttfamily)}}
\raggedbottom

% -----------------------------------
\input{pgfplots.preamble.tex}

\makeatletter
% I want a two-column index just as in pgfmanual styles. This here
% was the best way to get one:
\def\index@prologue{\section*{Index}\addcontentsline{toc}{chapter}{Index}
}
\makeatother

%\RequirePackage[german,english,francais]{babel}

\def\matlabcolormaptext{This colormap is similar to one shipped with Matlab$^\text{\textregistered}$ under a similar name.}

\IfFileExists{tikzlibraryspy.code.tex}{%
\usetikzlibrary{spy}
}{%
    \message{ERROR: tikz SPY library NOT available. The manual will only compile partially.^^J}%
}%

\usepackage{xparse}% for colorbrewer manual

\usetikzlibrary{
    decorations.markings,
    decorations.footprints,
    shapes.arrows,
    matrix,
    positioning,
}

\usepgfplotslibrary{
    fillbetween,
    ternary,
    smithchart,
    patchplots,
    polar,
    colormaps,
    colorbrewer,
    colortol,           % docu not ready yet
}
\pgfqkeys{/codeexample}{%
    every codeexample/.append style={
        /pgfplots/every ternary axis/.append style={
            /pgfplots/legend style={fill=graphicbackground},
        }
    },
    tabsize=4,
}

\pgfplotsmanualenableexternalizationofexpensive

%\usetikzlibrary{external}
%\tikzexternalize[prefix=figures/]{pgfplots}

\title{%
    Manual for Package \PGFPlots{}\\
    {\small 2D/3D Plots in \LaTeX{}, Version \pgfplotsversion}\\
    {\small\url{https://github.com/pgf-tikz/pgfplots}}
    %\\{\small Attention: you are using an unstable development version.}
}

\makeatletter
\long\def\abstractsmuggle{%
    \centering
    \textbf{Abstract}\\[0.5cm]

    \begin{minipage}{12cm}
        \PGFPlots{} draws high-quality function plots in normal or logarithmic
        scaling with a user-friendly interface directly in \TeX{}. The user
        supplies axis labels, legend entries and the plot coordinates for one or
        more plots and \PGFPlots{} applies axis scaling, computes any logarithms
        and axis ticks and draws the plots. It supports line plots, scatter
        plots, piecewise constant plots, bar plots, area plots, mesh and surface
        plots, patch plots, contour plots, quiver plots, histogram plots, box
        plots, polar axes, ternary diagrams, smith charts and some more. It is
        based on Till Tantau's package \PGF{}/\Tikz{}.
    \end{minipage}

}%

\expandafter\date\expandafter{\@date\\[2cm]
    \abstractsmuggle
}%
\makeatother

%\includeonly{pgfplots.reference}


\begin{document}

\def\plotcoords{%
\addplot coordinates {
(5,8.312e-02)    (17,2.547e-02)   (49,7.407e-03)
(129,2.102e-03)  (321,5.874e-04)  (769,1.623e-04)
(1793,4.442e-05) (4097,1.207e-05) (9217,3.261e-06)
};

\addplot coordinates{
(7,8.472e-02)    (31,3.044e-02)    (111,1.022e-02)
(351,3.303e-03)  (1023,1.039e-03)  (2815,3.196e-04)
(7423,9.658e-05) (18943,2.873e-05) (47103,8.437e-06)
};

\addplot coordinates{
(9,7.881e-02)     (49,3.243e-02)    (209,1.232e-02)
(769,4.454e-03)   (2561,1.551e-03)  (7937,5.236e-04)
(23297,1.723e-04) (65537,5.545e-05) (178177,1.751e-05)
};

\addplot coordinates{
(11,6.887e-02)    (71,3.177e-02)     (351,1.341e-02)
(1471,5.334e-03)  (5503,2.027e-03)   (18943,7.415e-04)
(61183,2.628e-04) (187903,9.063e-05) (553983,3.053e-05)
};

\addplot coordinates{
(13,5.755e-02)     (97,2.925e-02)     (545,1.351e-02)
(2561,5.842e-03)   (10625,2.397e-03)  (40193,9.414e-04)
(141569,3.564e-04) (471041,1.308e-04)
(1496065,4.670e-05)
};
}%


\bookmaketitle
\tableofcontents
\include{pgfplots.title_abstract_intro}
\include{pgfplots.preliminaries}
\include{pgfplots.intro}
\include{pgfplots.reference}
\include{pgfplots.libs}
\include{pgfplots.resources}
\include{pgfplots.importexport}
\include{pgfplots.basic.reference}

\printindex

\bibliographystyle{abbrv} %gerapali} %gerabbrv} %gerunsrt.bst} %gerabbrv}% gerplain}
\nocite{pgfplotstable}
\nocite{programmingnotes}
\bibliography{pgfplots}
\end{document}

% -----------------------------------------------------------------------------
% For Stefan Pinnow as reminder on what to look for when editing the manual
% -----------------------------------------------------------------------------
% There should be no line breaks in the following environments
% - |...|
% - \declareandlabel{...}
% - \verbpdfref{...}
% If "MakeTikzPictures" isn't running through check one of the externalized
% LOG files what is the cause of that.
% -----------------------------------------------------------------------------


%%%%%%%%%%%%%%%%%%%%%%%%%%%%%%%%%%%%%%%%%%%%%%%%%%%%%%%%%%%%%%%%%%%%%%%%%%%%%
%
% Package pgfplots.sty documentation.
%
% Copyright 2007/2008 by Christian Feuersaenger.
%
% This program is free software: you can redistribute it and/or modify
% it under the terms of the GNU General Public License as published by
% the Free Software Foundation, either version 3 of the License, or
% (at your option) any later version.
%
% This program is distributed in the hope that it will be useful,
% but WITHOUT ANY WARRANTY; without even the implied warranty of
% MERCHANTABILITY or FITNESS FOR A PARTICULAR PURPOSE.  See the
% GNU General Public License for more details.
%
% You should have received a copy of the GNU General Public License
% along with this program.  If not, see <http://www.gnu.org/licenses/>.
%
%
%%%%%%%%%%%%%%%%%%%%%%%%%%%%%%%%%%%%%%%%%%%%%%%%%%%%%%%%%%%%%%%%%%%%%%%%%%%%%
% SEE pgfplots-macros.tex as well!
%\pdfminorversion=5 % to allow compression
%\pdfobjcompresslevel=2
\documentclass[a4paper,openany]{book}

\let\bookmaketitle=\maketitle

% -----------------------------------
% this here is from ltxdoc:
\usepackage{doc}[=v2]
\AtBeginDocument{\MakeShortVerb{\|}}
%\setlength{\textwidth}{355pt}
%\addtolength\marginparwidth{30pt}
%\addtolength\oddsidemargin{20pt}
%\addtolength\evensidemargin{20pt}
\def\cmd#1{\cs{\expandafter\cmd@to@cs\string#1}}
\def\cmd@to@cs#1#2{\char\number`#2\relax}
\DeclareRobustCommand\cs[1]{\texttt{\char`\\#1}}
\providecommand\marg[1]{%
  {\ttfamily\char`\{}\meta{#1}{\ttfamily\char`\}}}
\providecommand\oarg[1]{%
  {\ttfamily[}\meta{#1}{\ttfamily]}}
\providecommand\parg[1]{%
  {\ttfamily(}\meta{#1}{\ttfamily)}}
\raggedbottom

% -----------------------------------
\input{pgfplots.preamble.tex}

\makeatletter
% I want a two-column index just as in pgfmanual styles. This here
% was the best way to get one:
\def\index@prologue{\section*{Index}\addcontentsline{toc}{chapter}{Index}
}
\makeatother

%\RequirePackage[german,english,francais]{babel}

\def\matlabcolormaptext{This colormap is similar to one shipped with Matlab$^\text{\textregistered}$ under a similar name.}

\IfFileExists{tikzlibraryspy.code.tex}{%
\usetikzlibrary{spy}
}{%
    \message{ERROR: tikz SPY library NOT available. The manual will only compile partially.^^J}%
}%

\usepackage{xparse}% for colorbrewer manual

\usetikzlibrary{
    decorations.markings,
    decorations.footprints,
    shapes.arrows,
    matrix,
    positioning,
}

\usepgfplotslibrary{
    fillbetween,
    ternary,
    smithchart,
    patchplots,
    polar,
    colormaps,
    colorbrewer,
    colortol,           % docu not ready yet
}
\pgfqkeys{/codeexample}{%
    every codeexample/.append style={
        /pgfplots/every ternary axis/.append style={
            /pgfplots/legend style={fill=graphicbackground},
        }
    },
    tabsize=4,
}

\pgfplotsmanualenableexternalizationofexpensive

%\usetikzlibrary{external}
%\tikzexternalize[prefix=figures/]{pgfplots}

\title{%
    Manual for Package \PGFPlots{}\\
    {\small 2D/3D Plots in \LaTeX{}, Version \pgfplotsversion}\\
    {\small\url{https://github.com/pgf-tikz/pgfplots}}
    %\\{\small Attention: you are using an unstable development version.}
}

\makeatletter
\long\def\abstractsmuggle{%
    \centering
    \textbf{Abstract}\\[0.5cm]

    \begin{minipage}{12cm}
        \PGFPlots{} draws high-quality function plots in normal or logarithmic
        scaling with a user-friendly interface directly in \TeX{}. The user
        supplies axis labels, legend entries and the plot coordinates for one or
        more plots and \PGFPlots{} applies axis scaling, computes any logarithms
        and axis ticks and draws the plots. It supports line plots, scatter
        plots, piecewise constant plots, bar plots, area plots, mesh and surface
        plots, patch plots, contour plots, quiver plots, histogram plots, box
        plots, polar axes, ternary diagrams, smith charts and some more. It is
        based on Till Tantau's package \PGF{}/\Tikz{}.
    \end{minipage}

}%

\expandafter\date\expandafter{\@date\\[2cm]
    \abstractsmuggle
}%
\makeatother

%\includeonly{pgfplots.reference}


\begin{document}

\def\plotcoords{%
\addplot coordinates {
(5,8.312e-02)    (17,2.547e-02)   (49,7.407e-03)
(129,2.102e-03)  (321,5.874e-04)  (769,1.623e-04)
(1793,4.442e-05) (4097,1.207e-05) (9217,3.261e-06)
};

\addplot coordinates{
(7,8.472e-02)    (31,3.044e-02)    (111,1.022e-02)
(351,3.303e-03)  (1023,1.039e-03)  (2815,3.196e-04)
(7423,9.658e-05) (18943,2.873e-05) (47103,8.437e-06)
};

\addplot coordinates{
(9,7.881e-02)     (49,3.243e-02)    (209,1.232e-02)
(769,4.454e-03)   (2561,1.551e-03)  (7937,5.236e-04)
(23297,1.723e-04) (65537,5.545e-05) (178177,1.751e-05)
};

\addplot coordinates{
(11,6.887e-02)    (71,3.177e-02)     (351,1.341e-02)
(1471,5.334e-03)  (5503,2.027e-03)   (18943,7.415e-04)
(61183,2.628e-04) (187903,9.063e-05) (553983,3.053e-05)
};

\addplot coordinates{
(13,5.755e-02)     (97,2.925e-02)     (545,1.351e-02)
(2561,5.842e-03)   (10625,2.397e-03)  (40193,9.414e-04)
(141569,3.564e-04) (471041,1.308e-04)
(1496065,4.670e-05)
};
}%


\bookmaketitle
\tableofcontents
\include{pgfplots.title_abstract_intro}
\include{pgfplots.preliminaries}
\include{pgfplots.intro}
\include{pgfplots.reference}
\include{pgfplots.libs}
\include{pgfplots.resources}
\include{pgfplots.importexport}
\include{pgfplots.basic.reference}

\printindex

\bibliographystyle{abbrv} %gerapali} %gerabbrv} %gerunsrt.bst} %gerabbrv}% gerplain}
\nocite{pgfplotstable}
\nocite{programmingnotes}
\bibliography{pgfplots}
\end{document}

% -----------------------------------------------------------------------------
% For Stefan Pinnow as reminder on what to look for when editing the manual
% -----------------------------------------------------------------------------
% There should be no line breaks in the following environments
% - |...|
% - \declareandlabel{...}
% - \verbpdfref{...}
% If "MakeTikzPictures" isn't running through check one of the externalized
% LOG files what is the cause of that.
% -----------------------------------------------------------------------------


%%%%%%%%%%%%%%%%%%%%%%%%%%%%%%%%%%%%%%%%%%%%%%%%%%%%%%%%%%%%%%%%%%%%%%%%%%%%%
%
% Package pgfplots.sty documentation.
%
% Copyright 2007/2008 by Christian Feuersaenger.
%
% This program is free software: you can redistribute it and/or modify
% it under the terms of the GNU General Public License as published by
% the Free Software Foundation, either version 3 of the License, or
% (at your option) any later version.
%
% This program is distributed in the hope that it will be useful,
% but WITHOUT ANY WARRANTY; without even the implied warranty of
% MERCHANTABILITY or FITNESS FOR A PARTICULAR PURPOSE.  See the
% GNU General Public License for more details.
%
% You should have received a copy of the GNU General Public License
% along with this program.  If not, see <http://www.gnu.org/licenses/>.
%
%
%%%%%%%%%%%%%%%%%%%%%%%%%%%%%%%%%%%%%%%%%%%%%%%%%%%%%%%%%%%%%%%%%%%%%%%%%%%%%
% SEE pgfplots-macros.tex as well!
%\pdfminorversion=5 % to allow compression
%\pdfobjcompresslevel=2
\documentclass[a4paper,openany]{book}

\let\bookmaketitle=\maketitle

% -----------------------------------
% this here is from ltxdoc:
\usepackage{doc}[=v2]
\AtBeginDocument{\MakeShortVerb{\|}}
%\setlength{\textwidth}{355pt}
%\addtolength\marginparwidth{30pt}
%\addtolength\oddsidemargin{20pt}
%\addtolength\evensidemargin{20pt}
\def\cmd#1{\cs{\expandafter\cmd@to@cs\string#1}}
\def\cmd@to@cs#1#2{\char\number`#2\relax}
\DeclareRobustCommand\cs[1]{\texttt{\char`\\#1}}
\providecommand\marg[1]{%
  {\ttfamily\char`\{}\meta{#1}{\ttfamily\char`\}}}
\providecommand\oarg[1]{%
  {\ttfamily[}\meta{#1}{\ttfamily]}}
\providecommand\parg[1]{%
  {\ttfamily(}\meta{#1}{\ttfamily)}}
\raggedbottom

% -----------------------------------
\input{pgfplots.preamble.tex}

\makeatletter
% I want a two-column index just as in pgfmanual styles. This here
% was the best way to get one:
\def\index@prologue{\section*{Index}\addcontentsline{toc}{chapter}{Index}
}
\makeatother

%\RequirePackage[german,english,francais]{babel}

\def\matlabcolormaptext{This colormap is similar to one shipped with Matlab$^\text{\textregistered}$ under a similar name.}

\IfFileExists{tikzlibraryspy.code.tex}{%
\usetikzlibrary{spy}
}{%
    \message{ERROR: tikz SPY library NOT available. The manual will only compile partially.^^J}%
}%

\usepackage{xparse}% for colorbrewer manual

\usetikzlibrary{
    decorations.markings,
    decorations.footprints,
    shapes.arrows,
    matrix,
    positioning,
}

\usepgfplotslibrary{
    fillbetween,
    ternary,
    smithchart,
    patchplots,
    polar,
    colormaps,
    colorbrewer,
    colortol,           % docu not ready yet
}
\pgfqkeys{/codeexample}{%
    every codeexample/.append style={
        /pgfplots/every ternary axis/.append style={
            /pgfplots/legend style={fill=graphicbackground},
        }
    },
    tabsize=4,
}

\pgfplotsmanualenableexternalizationofexpensive

%\usetikzlibrary{external}
%\tikzexternalize[prefix=figures/]{pgfplots}

\title{%
    Manual for Package \PGFPlots{}\\
    {\small 2D/3D Plots in \LaTeX{}, Version \pgfplotsversion}\\
    {\small\url{https://github.com/pgf-tikz/pgfplots}}
    %\\{\small Attention: you are using an unstable development version.}
}

\makeatletter
\long\def\abstractsmuggle{%
    \centering
    \textbf{Abstract}\\[0.5cm]

    \begin{minipage}{12cm}
        \PGFPlots{} draws high-quality function plots in normal or logarithmic
        scaling with a user-friendly interface directly in \TeX{}. The user
        supplies axis labels, legend entries and the plot coordinates for one or
        more plots and \PGFPlots{} applies axis scaling, computes any logarithms
        and axis ticks and draws the plots. It supports line plots, scatter
        plots, piecewise constant plots, bar plots, area plots, mesh and surface
        plots, patch plots, contour plots, quiver plots, histogram plots, box
        plots, polar axes, ternary diagrams, smith charts and some more. It is
        based on Till Tantau's package \PGF{}/\Tikz{}.
    \end{minipage}

}%

\expandafter\date\expandafter{\@date\\[2cm]
    \abstractsmuggle
}%
\makeatother

%\includeonly{pgfplots.reference}


\begin{document}

\def\plotcoords{%
\addplot coordinates {
(5,8.312e-02)    (17,2.547e-02)   (49,7.407e-03)
(129,2.102e-03)  (321,5.874e-04)  (769,1.623e-04)
(1793,4.442e-05) (4097,1.207e-05) (9217,3.261e-06)
};

\addplot coordinates{
(7,8.472e-02)    (31,3.044e-02)    (111,1.022e-02)
(351,3.303e-03)  (1023,1.039e-03)  (2815,3.196e-04)
(7423,9.658e-05) (18943,2.873e-05) (47103,8.437e-06)
};

\addplot coordinates{
(9,7.881e-02)     (49,3.243e-02)    (209,1.232e-02)
(769,4.454e-03)   (2561,1.551e-03)  (7937,5.236e-04)
(23297,1.723e-04) (65537,5.545e-05) (178177,1.751e-05)
};

\addplot coordinates{
(11,6.887e-02)    (71,3.177e-02)     (351,1.341e-02)
(1471,5.334e-03)  (5503,2.027e-03)   (18943,7.415e-04)
(61183,2.628e-04) (187903,9.063e-05) (553983,3.053e-05)
};

\addplot coordinates{
(13,5.755e-02)     (97,2.925e-02)     (545,1.351e-02)
(2561,5.842e-03)   (10625,2.397e-03)  (40193,9.414e-04)
(141569,3.564e-04) (471041,1.308e-04)
(1496065,4.670e-05)
};
}%


\bookmaketitle
\tableofcontents
\include{pgfplots.title_abstract_intro}
\include{pgfplots.preliminaries}
\include{pgfplots.intro}
\include{pgfplots.reference}
\include{pgfplots.libs}
\include{pgfplots.resources}
\include{pgfplots.importexport}
\include{pgfplots.basic.reference}

\printindex

\bibliographystyle{abbrv} %gerapali} %gerabbrv} %gerunsrt.bst} %gerabbrv}% gerplain}
\nocite{pgfplotstable}
\nocite{programmingnotes}
\bibliography{pgfplots}
\end{document}

% -----------------------------------------------------------------------------
% For Stefan Pinnow as reminder on what to look for when editing the manual
% -----------------------------------------------------------------------------
% There should be no line breaks in the following environments
% - |...|
% - \declareandlabel{...}
% - \verbpdfref{...}
% If "MakeTikzPictures" isn't running through check one of the externalized
% LOG files what is the cause of that.
% -----------------------------------------------------------------------------


%%%%%%%%%%%%%%%%%%%%%%%%%%%%%%%%%%%%%%%%%%%%%%%%%%%%%%%%%%%%%%%%%%%%%%%%%%%%%
%
% Package pgfplots.sty documentation.
%
% Copyright 2007/2008 by Christian Feuersaenger.
%
% This program is free software: you can redistribute it and/or modify
% it under the terms of the GNU General Public License as published by
% the Free Software Foundation, either version 3 of the License, or
% (at your option) any later version.
%
% This program is distributed in the hope that it will be useful,
% but WITHOUT ANY WARRANTY; without even the implied warranty of
% MERCHANTABILITY or FITNESS FOR A PARTICULAR PURPOSE.  See the
% GNU General Public License for more details.
%
% You should have received a copy of the GNU General Public License
% along with this program.  If not, see <http://www.gnu.org/licenses/>.
%
%
%%%%%%%%%%%%%%%%%%%%%%%%%%%%%%%%%%%%%%%%%%%%%%%%%%%%%%%%%%%%%%%%%%%%%%%%%%%%%
% SEE pgfplots-macros.tex as well!
%\pdfminorversion=5 % to allow compression
%\pdfobjcompresslevel=2
\documentclass[a4paper,openany]{book}

\let\bookmaketitle=\maketitle

% -----------------------------------
% this here is from ltxdoc:
\usepackage{doc}[=v2]
\AtBeginDocument{\MakeShortVerb{\|}}
%\setlength{\textwidth}{355pt}
%\addtolength\marginparwidth{30pt}
%\addtolength\oddsidemargin{20pt}
%\addtolength\evensidemargin{20pt}
\def\cmd#1{\cs{\expandafter\cmd@to@cs\string#1}}
\def\cmd@to@cs#1#2{\char\number`#2\relax}
\DeclareRobustCommand\cs[1]{\texttt{\char`\\#1}}
\providecommand\marg[1]{%
  {\ttfamily\char`\{}\meta{#1}{\ttfamily\char`\}}}
\providecommand\oarg[1]{%
  {\ttfamily[}\meta{#1}{\ttfamily]}}
\providecommand\parg[1]{%
  {\ttfamily(}\meta{#1}{\ttfamily)}}
\raggedbottom

% -----------------------------------
\input{pgfplots.preamble.tex}

\makeatletter
% I want a two-column index just as in pgfmanual styles. This here
% was the best way to get one:
\def\index@prologue{\section*{Index}\addcontentsline{toc}{chapter}{Index}
}
\makeatother

%\RequirePackage[german,english,francais]{babel}

\def\matlabcolormaptext{This colormap is similar to one shipped with Matlab$^\text{\textregistered}$ under a similar name.}

\IfFileExists{tikzlibraryspy.code.tex}{%
\usetikzlibrary{spy}
}{%
    \message{ERROR: tikz SPY library NOT available. The manual will only compile partially.^^J}%
}%

\usepackage{xparse}% for colorbrewer manual

\usetikzlibrary{
    decorations.markings,
    decorations.footprints,
    shapes.arrows,
    matrix,
    positioning,
}

\usepgfplotslibrary{
    fillbetween,
    ternary,
    smithchart,
    patchplots,
    polar,
    colormaps,
    colorbrewer,
    colortol,           % docu not ready yet
}
\pgfqkeys{/codeexample}{%
    every codeexample/.append style={
        /pgfplots/every ternary axis/.append style={
            /pgfplots/legend style={fill=graphicbackground},
        }
    },
    tabsize=4,
}

\pgfplotsmanualenableexternalizationofexpensive

%\usetikzlibrary{external}
%\tikzexternalize[prefix=figures/]{pgfplots}

\title{%
    Manual for Package \PGFPlots{}\\
    {\small 2D/3D Plots in \LaTeX{}, Version \pgfplotsversion}\\
    {\small\url{https://github.com/pgf-tikz/pgfplots}}
    %\\{\small Attention: you are using an unstable development version.}
}

\makeatletter
\long\def\abstractsmuggle{%
    \centering
    \textbf{Abstract}\\[0.5cm]

    \begin{minipage}{12cm}
        \PGFPlots{} draws high-quality function plots in normal or logarithmic
        scaling with a user-friendly interface directly in \TeX{}. The user
        supplies axis labels, legend entries and the plot coordinates for one or
        more plots and \PGFPlots{} applies axis scaling, computes any logarithms
        and axis ticks and draws the plots. It supports line plots, scatter
        plots, piecewise constant plots, bar plots, area plots, mesh and surface
        plots, patch plots, contour plots, quiver plots, histogram plots, box
        plots, polar axes, ternary diagrams, smith charts and some more. It is
        based on Till Tantau's package \PGF{}/\Tikz{}.
    \end{minipage}

}%

\expandafter\date\expandafter{\@date\\[2cm]
    \abstractsmuggle
}%
\makeatother

%\includeonly{pgfplots.reference}


\begin{document}

\def\plotcoords{%
\addplot coordinates {
(5,8.312e-02)    (17,2.547e-02)   (49,7.407e-03)
(129,2.102e-03)  (321,5.874e-04)  (769,1.623e-04)
(1793,4.442e-05) (4097,1.207e-05) (9217,3.261e-06)
};

\addplot coordinates{
(7,8.472e-02)    (31,3.044e-02)    (111,1.022e-02)
(351,3.303e-03)  (1023,1.039e-03)  (2815,3.196e-04)
(7423,9.658e-05) (18943,2.873e-05) (47103,8.437e-06)
};

\addplot coordinates{
(9,7.881e-02)     (49,3.243e-02)    (209,1.232e-02)
(769,4.454e-03)   (2561,1.551e-03)  (7937,5.236e-04)
(23297,1.723e-04) (65537,5.545e-05) (178177,1.751e-05)
};

\addplot coordinates{
(11,6.887e-02)    (71,3.177e-02)     (351,1.341e-02)
(1471,5.334e-03)  (5503,2.027e-03)   (18943,7.415e-04)
(61183,2.628e-04) (187903,9.063e-05) (553983,3.053e-05)
};

\addplot coordinates{
(13,5.755e-02)     (97,2.925e-02)     (545,1.351e-02)
(2561,5.842e-03)   (10625,2.397e-03)  (40193,9.414e-04)
(141569,3.564e-04) (471041,1.308e-04)
(1496065,4.670e-05)
};
}%


\bookmaketitle
\tableofcontents
\include{pgfplots.title_abstract_intro}
\include{pgfplots.preliminaries}
\include{pgfplots.intro}
\include{pgfplots.reference}
\include{pgfplots.libs}
\include{pgfplots.resources}
\include{pgfplots.importexport}
\include{pgfplots.basic.reference}

\printindex

\bibliographystyle{abbrv} %gerapali} %gerabbrv} %gerunsrt.bst} %gerabbrv}% gerplain}
\nocite{pgfplotstable}
\nocite{programmingnotes}
\bibliography{pgfplots}
\end{document}

% -----------------------------------------------------------------------------
% For Stefan Pinnow as reminder on what to look for when editing the manual
% -----------------------------------------------------------------------------
% There should be no line breaks in the following environments
% - |...|
% - \declareandlabel{...}
% - \verbpdfref{...}
% If "MakeTikzPictures" isn't running through check one of the externalized
% LOG files what is the cause of that.
% -----------------------------------------------------------------------------


%%%%%%%%%%%%%%%%%%%%%%%%%%%%%%%%%%%%%%%%%%%%%%%%%%%%%%%%%%%%%%%%%%%%%%%%%%%%%
%
% Package pgfplots.sty documentation.
%
% Copyright 2007/2008 by Christian Feuersaenger.
%
% This program is free software: you can redistribute it and/or modify
% it under the terms of the GNU General Public License as published by
% the Free Software Foundation, either version 3 of the License, or
% (at your option) any later version.
%
% This program is distributed in the hope that it will be useful,
% but WITHOUT ANY WARRANTY; without even the implied warranty of
% MERCHANTABILITY or FITNESS FOR A PARTICULAR PURPOSE.  See the
% GNU General Public License for more details.
%
% You should have received a copy of the GNU General Public License
% along with this program.  If not, see <http://www.gnu.org/licenses/>.
%
%
%%%%%%%%%%%%%%%%%%%%%%%%%%%%%%%%%%%%%%%%%%%%%%%%%%%%%%%%%%%%%%%%%%%%%%%%%%%%%
% SEE pgfplots-macros.tex as well!
%\pdfminorversion=5 % to allow compression
%\pdfobjcompresslevel=2
\documentclass[a4paper,openany]{book}

\let\bookmaketitle=\maketitle

% -----------------------------------
% this here is from ltxdoc:
\usepackage{doc}[=v2]
\AtBeginDocument{\MakeShortVerb{\|}}
%\setlength{\textwidth}{355pt}
%\addtolength\marginparwidth{30pt}
%\addtolength\oddsidemargin{20pt}
%\addtolength\evensidemargin{20pt}
\def\cmd#1{\cs{\expandafter\cmd@to@cs\string#1}}
\def\cmd@to@cs#1#2{\char\number`#2\relax}
\DeclareRobustCommand\cs[1]{\texttt{\char`\\#1}}
\providecommand\marg[1]{%
  {\ttfamily\char`\{}\meta{#1}{\ttfamily\char`\}}}
\providecommand\oarg[1]{%
  {\ttfamily[}\meta{#1}{\ttfamily]}}
\providecommand\parg[1]{%
  {\ttfamily(}\meta{#1}{\ttfamily)}}
\raggedbottom

% -----------------------------------
\input{pgfplots.preamble.tex}

\makeatletter
% I want a two-column index just as in pgfmanual styles. This here
% was the best way to get one:
\def\index@prologue{\section*{Index}\addcontentsline{toc}{chapter}{Index}
}
\makeatother

%\RequirePackage[german,english,francais]{babel}

\def\matlabcolormaptext{This colormap is similar to one shipped with Matlab$^\text{\textregistered}$ under a similar name.}

\IfFileExists{tikzlibraryspy.code.tex}{%
\usetikzlibrary{spy}
}{%
    \message{ERROR: tikz SPY library NOT available. The manual will only compile partially.^^J}%
}%

\usepackage{xparse}% for colorbrewer manual

\usetikzlibrary{
    decorations.markings,
    decorations.footprints,
    shapes.arrows,
    matrix,
    positioning,
}

\usepgfplotslibrary{
    fillbetween,
    ternary,
    smithchart,
    patchplots,
    polar,
    colormaps,
    colorbrewer,
    colortol,           % docu not ready yet
}
\pgfqkeys{/codeexample}{%
    every codeexample/.append style={
        /pgfplots/every ternary axis/.append style={
            /pgfplots/legend style={fill=graphicbackground},
        }
    },
    tabsize=4,
}

\pgfplotsmanualenableexternalizationofexpensive

%\usetikzlibrary{external}
%\tikzexternalize[prefix=figures/]{pgfplots}

\title{%
    Manual for Package \PGFPlots{}\\
    {\small 2D/3D Plots in \LaTeX{}, Version \pgfplotsversion}\\
    {\small\url{https://github.com/pgf-tikz/pgfplots}}
    %\\{\small Attention: you are using an unstable development version.}
}

\makeatletter
\long\def\abstractsmuggle{%
    \centering
    \textbf{Abstract}\\[0.5cm]

    \begin{minipage}{12cm}
        \PGFPlots{} draws high-quality function plots in normal or logarithmic
        scaling with a user-friendly interface directly in \TeX{}. The user
        supplies axis labels, legend entries and the plot coordinates for one or
        more plots and \PGFPlots{} applies axis scaling, computes any logarithms
        and axis ticks and draws the plots. It supports line plots, scatter
        plots, piecewise constant plots, bar plots, area plots, mesh and surface
        plots, patch plots, contour plots, quiver plots, histogram plots, box
        plots, polar axes, ternary diagrams, smith charts and some more. It is
        based on Till Tantau's package \PGF{}/\Tikz{}.
    \end{minipage}

}%

\expandafter\date\expandafter{\@date\\[2cm]
    \abstractsmuggle
}%
\makeatother

%\includeonly{pgfplots.reference}


\begin{document}

\def\plotcoords{%
\addplot coordinates {
(5,8.312e-02)    (17,2.547e-02)   (49,7.407e-03)
(129,2.102e-03)  (321,5.874e-04)  (769,1.623e-04)
(1793,4.442e-05) (4097,1.207e-05) (9217,3.261e-06)
};

\addplot coordinates{
(7,8.472e-02)    (31,3.044e-02)    (111,1.022e-02)
(351,3.303e-03)  (1023,1.039e-03)  (2815,3.196e-04)
(7423,9.658e-05) (18943,2.873e-05) (47103,8.437e-06)
};

\addplot coordinates{
(9,7.881e-02)     (49,3.243e-02)    (209,1.232e-02)
(769,4.454e-03)   (2561,1.551e-03)  (7937,5.236e-04)
(23297,1.723e-04) (65537,5.545e-05) (178177,1.751e-05)
};

\addplot coordinates{
(11,6.887e-02)    (71,3.177e-02)     (351,1.341e-02)
(1471,5.334e-03)  (5503,2.027e-03)   (18943,7.415e-04)
(61183,2.628e-04) (187903,9.063e-05) (553983,3.053e-05)
};

\addplot coordinates{
(13,5.755e-02)     (97,2.925e-02)     (545,1.351e-02)
(2561,5.842e-03)   (10625,2.397e-03)  (40193,9.414e-04)
(141569,3.564e-04) (471041,1.308e-04)
(1496065,4.670e-05)
};
}%


\bookmaketitle
\tableofcontents
\include{pgfplots.title_abstract_intro}
\include{pgfplots.preliminaries}
\include{pgfplots.intro}
\include{pgfplots.reference}
\include{pgfplots.libs}
\include{pgfplots.resources}
\include{pgfplots.importexport}
\include{pgfplots.basic.reference}

\printindex

\bibliographystyle{abbrv} %gerapali} %gerabbrv} %gerunsrt.bst} %gerabbrv}% gerplain}
\nocite{pgfplotstable}
\nocite{programmingnotes}
\bibliography{pgfplots}
\end{document}

% -----------------------------------------------------------------------------
% For Stefan Pinnow as reminder on what to look for when editing the manual
% -----------------------------------------------------------------------------
% There should be no line breaks in the following environments
% - |...|
% - \declareandlabel{...}
% - \verbpdfref{...}
% If "MakeTikzPictures" isn't running through check one of the externalized
% LOG files what is the cause of that.
% -----------------------------------------------------------------------------


%%%%%%%%%%%%%%%%%%%%%%%%%%%%%%%%%%%%%%%%%%%%%%%%%%%%%%%%%%%%%%%%%%%%%%%%%%%%%
%
% Package pgfplots.sty documentation.
%
% Copyright 2007/2008 by Christian Feuersaenger.
%
% This program is free software: you can redistribute it and/or modify
% it under the terms of the GNU General Public License as published by
% the Free Software Foundation, either version 3 of the License, or
% (at your option) any later version.
%
% This program is distributed in the hope that it will be useful,
% but WITHOUT ANY WARRANTY; without even the implied warranty of
% MERCHANTABILITY or FITNESS FOR A PARTICULAR PURPOSE.  See the
% GNU General Public License for more details.
%
% You should have received a copy of the GNU General Public License
% along with this program.  If not, see <http://www.gnu.org/licenses/>.
%
%
%%%%%%%%%%%%%%%%%%%%%%%%%%%%%%%%%%%%%%%%%%%%%%%%%%%%%%%%%%%%%%%%%%%%%%%%%%%%%
% SEE pgfplots-macros.tex as well!
%\pdfminorversion=5 % to allow compression
%\pdfobjcompresslevel=2
\documentclass[a4paper,openany]{book}

\let\bookmaketitle=\maketitle

% -----------------------------------
% this here is from ltxdoc:
\usepackage{doc}[=v2]
\AtBeginDocument{\MakeShortVerb{\|}}
%\setlength{\textwidth}{355pt}
%\addtolength\marginparwidth{30pt}
%\addtolength\oddsidemargin{20pt}
%\addtolength\evensidemargin{20pt}
\def\cmd#1{\cs{\expandafter\cmd@to@cs\string#1}}
\def\cmd@to@cs#1#2{\char\number`#2\relax}
\DeclareRobustCommand\cs[1]{\texttt{\char`\\#1}}
\providecommand\marg[1]{%
  {\ttfamily\char`\{}\meta{#1}{\ttfamily\char`\}}}
\providecommand\oarg[1]{%
  {\ttfamily[}\meta{#1}{\ttfamily]}}
\providecommand\parg[1]{%
  {\ttfamily(}\meta{#1}{\ttfamily)}}
\raggedbottom

% -----------------------------------
\input{pgfplots.preamble.tex}

\makeatletter
% I want a two-column index just as in pgfmanual styles. This here
% was the best way to get one:
\def\index@prologue{\section*{Index}\addcontentsline{toc}{chapter}{Index}
}
\makeatother

%\RequirePackage[german,english,francais]{babel}

\def\matlabcolormaptext{This colormap is similar to one shipped with Matlab$^\text{\textregistered}$ under a similar name.}

\IfFileExists{tikzlibraryspy.code.tex}{%
\usetikzlibrary{spy}
}{%
    \message{ERROR: tikz SPY library NOT available. The manual will only compile partially.^^J}%
}%

\usepackage{xparse}% for colorbrewer manual

\usetikzlibrary{
    decorations.markings,
    decorations.footprints,
    shapes.arrows,
    matrix,
    positioning,
}

\usepgfplotslibrary{
    fillbetween,
    ternary,
    smithchart,
    patchplots,
    polar,
    colormaps,
    colorbrewer,
    colortol,           % docu not ready yet
}
\pgfqkeys{/codeexample}{%
    every codeexample/.append style={
        /pgfplots/every ternary axis/.append style={
            /pgfplots/legend style={fill=graphicbackground},
        }
    },
    tabsize=4,
}

\pgfplotsmanualenableexternalizationofexpensive

%\usetikzlibrary{external}
%\tikzexternalize[prefix=figures/]{pgfplots}

\title{%
    Manual for Package \PGFPlots{}\\
    {\small 2D/3D Plots in \LaTeX{}, Version \pgfplotsversion}\\
    {\small\url{https://github.com/pgf-tikz/pgfplots}}
    %\\{\small Attention: you are using an unstable development version.}
}

\makeatletter
\long\def\abstractsmuggle{%
    \centering
    \textbf{Abstract}\\[0.5cm]

    \begin{minipage}{12cm}
        \PGFPlots{} draws high-quality function plots in normal or logarithmic
        scaling with a user-friendly interface directly in \TeX{}. The user
        supplies axis labels, legend entries and the plot coordinates for one or
        more plots and \PGFPlots{} applies axis scaling, computes any logarithms
        and axis ticks and draws the plots. It supports line plots, scatter
        plots, piecewise constant plots, bar plots, area plots, mesh and surface
        plots, patch plots, contour plots, quiver plots, histogram plots, box
        plots, polar axes, ternary diagrams, smith charts and some more. It is
        based on Till Tantau's package \PGF{}/\Tikz{}.
    \end{minipage}

}%

\expandafter\date\expandafter{\@date\\[2cm]
    \abstractsmuggle
}%
\makeatother

%\includeonly{pgfplots.reference}


\begin{document}

\def\plotcoords{%
\addplot coordinates {
(5,8.312e-02)    (17,2.547e-02)   (49,7.407e-03)
(129,2.102e-03)  (321,5.874e-04)  (769,1.623e-04)
(1793,4.442e-05) (4097,1.207e-05) (9217,3.261e-06)
};

\addplot coordinates{
(7,8.472e-02)    (31,3.044e-02)    (111,1.022e-02)
(351,3.303e-03)  (1023,1.039e-03)  (2815,3.196e-04)
(7423,9.658e-05) (18943,2.873e-05) (47103,8.437e-06)
};

\addplot coordinates{
(9,7.881e-02)     (49,3.243e-02)    (209,1.232e-02)
(769,4.454e-03)   (2561,1.551e-03)  (7937,5.236e-04)
(23297,1.723e-04) (65537,5.545e-05) (178177,1.751e-05)
};

\addplot coordinates{
(11,6.887e-02)    (71,3.177e-02)     (351,1.341e-02)
(1471,5.334e-03)  (5503,2.027e-03)   (18943,7.415e-04)
(61183,2.628e-04) (187903,9.063e-05) (553983,3.053e-05)
};

\addplot coordinates{
(13,5.755e-02)     (97,2.925e-02)     (545,1.351e-02)
(2561,5.842e-03)   (10625,2.397e-03)  (40193,9.414e-04)
(141569,3.564e-04) (471041,1.308e-04)
(1496065,4.670e-05)
};
}%


\bookmaketitle
\tableofcontents
\include{pgfplots.title_abstract_intro}
\include{pgfplots.preliminaries}
\include{pgfplots.intro}
\include{pgfplots.reference}
\include{pgfplots.libs}
\include{pgfplots.resources}
\include{pgfplots.importexport}
\include{pgfplots.basic.reference}

\printindex

\bibliographystyle{abbrv} %gerapali} %gerabbrv} %gerunsrt.bst} %gerabbrv}% gerplain}
\nocite{pgfplotstable}
\nocite{programmingnotes}
\bibliography{pgfplots}
\end{document}

% -----------------------------------------------------------------------------
% For Stefan Pinnow as reminder on what to look for when editing the manual
% -----------------------------------------------------------------------------
% There should be no line breaks in the following environments
% - |...|
% - \declareandlabel{...}
% - \verbpdfref{...}
% If "MakeTikzPictures" isn't running through check one of the externalized
% LOG files what is the cause of that.
% -----------------------------------------------------------------------------


%%%%%%%%%%%%%%%%%%%%%%%%%%%%%%%%%%%%%%%%%%%%%%%%%%%%%%%%%%%%%%%%%%%%%%%%%%%%%
%
% Package pgfplots.sty documentation.
%
% Copyright 2007/2008 by Christian Feuersaenger.
%
% This program is free software: you can redistribute it and/or modify
% it under the terms of the GNU General Public License as published by
% the Free Software Foundation, either version 3 of the License, or
% (at your option) any later version.
%
% This program is distributed in the hope that it will be useful,
% but WITHOUT ANY WARRANTY; without even the implied warranty of
% MERCHANTABILITY or FITNESS FOR A PARTICULAR PURPOSE.  See the
% GNU General Public License for more details.
%
% You should have received a copy of the GNU General Public License
% along with this program.  If not, see <http://www.gnu.org/licenses/>.
%
%
%%%%%%%%%%%%%%%%%%%%%%%%%%%%%%%%%%%%%%%%%%%%%%%%%%%%%%%%%%%%%%%%%%%%%%%%%%%%%
% SEE pgfplots-macros.tex as well!
%\pdfminorversion=5 % to allow compression
%\pdfobjcompresslevel=2
\documentclass[a4paper,openany]{book}

\let\bookmaketitle=\maketitle

% -----------------------------------
% this here is from ltxdoc:
\usepackage{doc}[=v2]
\AtBeginDocument{\MakeShortVerb{\|}}
%\setlength{\textwidth}{355pt}
%\addtolength\marginparwidth{30pt}
%\addtolength\oddsidemargin{20pt}
%\addtolength\evensidemargin{20pt}
\def\cmd#1{\cs{\expandafter\cmd@to@cs\string#1}}
\def\cmd@to@cs#1#2{\char\number`#2\relax}
\DeclareRobustCommand\cs[1]{\texttt{\char`\\#1}}
\providecommand\marg[1]{%
  {\ttfamily\char`\{}\meta{#1}{\ttfamily\char`\}}}
\providecommand\oarg[1]{%
  {\ttfamily[}\meta{#1}{\ttfamily]}}
\providecommand\parg[1]{%
  {\ttfamily(}\meta{#1}{\ttfamily)}}
\raggedbottom

% -----------------------------------
\input{pgfplots.preamble.tex}

\makeatletter
% I want a two-column index just as in pgfmanual styles. This here
% was the best way to get one:
\def\index@prologue{\section*{Index}\addcontentsline{toc}{chapter}{Index}
}
\makeatother

%\RequirePackage[german,english,francais]{babel}

\def\matlabcolormaptext{This colormap is similar to one shipped with Matlab$^\text{\textregistered}$ under a similar name.}

\IfFileExists{tikzlibraryspy.code.tex}{%
\usetikzlibrary{spy}
}{%
    \message{ERROR: tikz SPY library NOT available. The manual will only compile partially.^^J}%
}%

\usepackage{xparse}% for colorbrewer manual

\usetikzlibrary{
    decorations.markings,
    decorations.footprints,
    shapes.arrows,
    matrix,
    positioning,
}

\usepgfplotslibrary{
    fillbetween,
    ternary,
    smithchart,
    patchplots,
    polar,
    colormaps,
    colorbrewer,
    colortol,           % docu not ready yet
}
\pgfqkeys{/codeexample}{%
    every codeexample/.append style={
        /pgfplots/every ternary axis/.append style={
            /pgfplots/legend style={fill=graphicbackground},
        }
    },
    tabsize=4,
}

\pgfplotsmanualenableexternalizationofexpensive

%\usetikzlibrary{external}
%\tikzexternalize[prefix=figures/]{pgfplots}

\title{%
    Manual for Package \PGFPlots{}\\
    {\small 2D/3D Plots in \LaTeX{}, Version \pgfplotsversion}\\
    {\small\url{https://github.com/pgf-tikz/pgfplots}}
    %\\{\small Attention: you are using an unstable development version.}
}

\makeatletter
\long\def\abstractsmuggle{%
    \centering
    \textbf{Abstract}\\[0.5cm]

    \begin{minipage}{12cm}
        \PGFPlots{} draws high-quality function plots in normal or logarithmic
        scaling with a user-friendly interface directly in \TeX{}. The user
        supplies axis labels, legend entries and the plot coordinates for one or
        more plots and \PGFPlots{} applies axis scaling, computes any logarithms
        and axis ticks and draws the plots. It supports line plots, scatter
        plots, piecewise constant plots, bar plots, area plots, mesh and surface
        plots, patch plots, contour plots, quiver plots, histogram plots, box
        plots, polar axes, ternary diagrams, smith charts and some more. It is
        based on Till Tantau's package \PGF{}/\Tikz{}.
    \end{minipage}

}%

\expandafter\date\expandafter{\@date\\[2cm]
    \abstractsmuggle
}%
\makeatother

%\includeonly{pgfplots.reference}


\begin{document}

\def\plotcoords{%
\addplot coordinates {
(5,8.312e-02)    (17,2.547e-02)   (49,7.407e-03)
(129,2.102e-03)  (321,5.874e-04)  (769,1.623e-04)
(1793,4.442e-05) (4097,1.207e-05) (9217,3.261e-06)
};

\addplot coordinates{
(7,8.472e-02)    (31,3.044e-02)    (111,1.022e-02)
(351,3.303e-03)  (1023,1.039e-03)  (2815,3.196e-04)
(7423,9.658e-05) (18943,2.873e-05) (47103,8.437e-06)
};

\addplot coordinates{
(9,7.881e-02)     (49,3.243e-02)    (209,1.232e-02)
(769,4.454e-03)   (2561,1.551e-03)  (7937,5.236e-04)
(23297,1.723e-04) (65537,5.545e-05) (178177,1.751e-05)
};

\addplot coordinates{
(11,6.887e-02)    (71,3.177e-02)     (351,1.341e-02)
(1471,5.334e-03)  (5503,2.027e-03)   (18943,7.415e-04)
(61183,2.628e-04) (187903,9.063e-05) (553983,3.053e-05)
};

\addplot coordinates{
(13,5.755e-02)     (97,2.925e-02)     (545,1.351e-02)
(2561,5.842e-03)   (10625,2.397e-03)  (40193,9.414e-04)
(141569,3.564e-04) (471041,1.308e-04)
(1496065,4.670e-05)
};
}%


\bookmaketitle
\tableofcontents
\include{pgfplots.title_abstract_intro}
\include{pgfplots.preliminaries}
\include{pgfplots.intro}
\include{pgfplots.reference}
\include{pgfplots.libs}
\include{pgfplots.resources}
\include{pgfplots.importexport}
\include{pgfplots.basic.reference}

\printindex

\bibliographystyle{abbrv} %gerapali} %gerabbrv} %gerunsrt.bst} %gerabbrv}% gerplain}
\nocite{pgfplotstable}
\nocite{programmingnotes}
\bibliography{pgfplots}
\end{document}

% -----------------------------------------------------------------------------
% For Stefan Pinnow as reminder on what to look for when editing the manual
% -----------------------------------------------------------------------------
% There should be no line breaks in the following environments
% - |...|
% - \declareandlabel{...}
% - \verbpdfref{...}
% If "MakeTikzPictures" isn't running through check one of the externalized
% LOG files what is the cause of that.
% -----------------------------------------------------------------------------

\include{pgfplots.basic.reference}

\printindex

\bibliographystyle{abbrv} %gerapali} %gerabbrv} %gerunsrt.bst} %gerabbrv}% gerplain}
\nocite{pgfplotstable}
\nocite{programmingnotes}
\bibliography{pgfplots}
\end{document}

% -----------------------------------------------------------------------------
% For Stefan Pinnow as reminder on what to look for when editing the manual
% -----------------------------------------------------------------------------
% There should be no line breaks in the following environments
% - |...|
% - \declareandlabel{...}
% - \verbpdfref{...}
% If "MakeTikzPictures" isn't running through check one of the externalized
% LOG files what is the cause of that.
% -----------------------------------------------------------------------------

\include{pgfplots.basic.reference}

\printindex

\bibliographystyle{abbrv} %gerapali} %gerabbrv} %gerunsrt.bst} %gerabbrv}% gerplain}
\nocite{pgfplotstable}
\nocite{programmingnotes}
\bibliography{pgfplots}
\end{document}

% -----------------------------------------------------------------------------
% For Stefan Pinnow as reminder on what to look for when editing the manual
% -----------------------------------------------------------------------------
% There should be no line breaks in the following environments
% - |...|
% - \declareandlabel{...}
% - \verbpdfref{...}
% If "MakeTikzPictures" isn't running through check one of the externalized
% LOG files what is the cause of that.
% -----------------------------------------------------------------------------


%%%%%%%%%%%%%%%%%%%%%%%%%%%%%%%%%%%%%%%%%%%%%%%%%%%%%%%%%%%%%%%%%%%%%%%%%%%%%
%
% Package pgfplots.sty documentation.
%
% Copyright 2007/2008 by Christian Feuersaenger.
%
% This program is free software: you can redistribute it and/or modify
% it under the terms of the GNU General Public License as published by
% the Free Software Foundation, either version 3 of the License, or
% (at your option) any later version.
%
% This program is distributed in the hope that it will be useful,
% but WITHOUT ANY WARRANTY; without even the implied warranty of
% MERCHANTABILITY or FITNESS FOR A PARTICULAR PURPOSE.  See the
% GNU General Public License for more details.
%
% You should have received a copy of the GNU General Public License
% along with this program.  If not, see <http://www.gnu.org/licenses/>.
%
%
%%%%%%%%%%%%%%%%%%%%%%%%%%%%%%%%%%%%%%%%%%%%%%%%%%%%%%%%%%%%%%%%%%%%%%%%%%%%%
% SEE pgfplots-macros.tex as well!
%\pdfminorversion=5 % to allow compression
%\pdfobjcompresslevel=2
\documentclass[a4paper,openany]{book}

\let\bookmaketitle=\maketitle

% -----------------------------------
% this here is from ltxdoc:
\usepackage{doc}[=v2]
\AtBeginDocument{\MakeShortVerb{\|}}
%\setlength{\textwidth}{355pt}
%\addtolength\marginparwidth{30pt}
%\addtolength\oddsidemargin{20pt}
%\addtolength\evensidemargin{20pt}
\def\cmd#1{\cs{\expandafter\cmd@to@cs\string#1}}
\def\cmd@to@cs#1#2{\char\number`#2\relax}
\DeclareRobustCommand\cs[1]{\texttt{\char`\\#1}}
\providecommand\marg[1]{%
  {\ttfamily\char`\{}\meta{#1}{\ttfamily\char`\}}}
\providecommand\oarg[1]{%
  {\ttfamily[}\meta{#1}{\ttfamily]}}
\providecommand\parg[1]{%
  {\ttfamily(}\meta{#1}{\ttfamily)}}
\raggedbottom

% -----------------------------------
\input{pgfplots.preamble.tex}

\makeatletter
% I want a two-column index just as in pgfmanual styles. This here
% was the best way to get one:
\def\index@prologue{\section*{Index}\addcontentsline{toc}{chapter}{Index}
}
\makeatother

%\RequirePackage[german,english,francais]{babel}

\def\matlabcolormaptext{This colormap is similar to one shipped with Matlab$^\text{\textregistered}$ under a similar name.}

\IfFileExists{tikzlibraryspy.code.tex}{%
\usetikzlibrary{spy}
}{%
    \message{ERROR: tikz SPY library NOT available. The manual will only compile partially.^^J}%
}%

\usepackage{xparse}% for colorbrewer manual

\usetikzlibrary{
    decorations.markings,
    decorations.footprints,
    shapes.arrows,
    matrix,
    positioning,
}

\usepgfplotslibrary{
    fillbetween,
    ternary,
    smithchart,
    patchplots,
    polar,
    colormaps,
    colorbrewer,
    colortol,           % docu not ready yet
}
\pgfqkeys{/codeexample}{%
    every codeexample/.append style={
        /pgfplots/every ternary axis/.append style={
            /pgfplots/legend style={fill=graphicbackground},
        }
    },
    tabsize=4,
}

\pgfplotsmanualenableexternalizationofexpensive

%\usetikzlibrary{external}
%\tikzexternalize[prefix=figures/]{pgfplots}

\title{%
    Manual for Package \PGFPlots{}\\
    {\small 2D/3D Plots in \LaTeX{}, Version \pgfplotsversion}\\
    {\small\url{https://github.com/pgf-tikz/pgfplots}}
    %\\{\small Attention: you are using an unstable development version.}
}

\makeatletter
\long\def\abstractsmuggle{%
    \centering
    \textbf{Abstract}\\[0.5cm]

    \begin{minipage}{12cm}
        \PGFPlots{} draws high-quality function plots in normal or logarithmic
        scaling with a user-friendly interface directly in \TeX{}. The user
        supplies axis labels, legend entries and the plot coordinates for one or
        more plots and \PGFPlots{} applies axis scaling, computes any logarithms
        and axis ticks and draws the plots. It supports line plots, scatter
        plots, piecewise constant plots, bar plots, area plots, mesh and surface
        plots, patch plots, contour plots, quiver plots, histogram plots, box
        plots, polar axes, ternary diagrams, smith charts and some more. It is
        based on Till Tantau's package \PGF{}/\Tikz{}.
    \end{minipage}

}%

\expandafter\date\expandafter{\@date\\[2cm]
    \abstractsmuggle
}%
\makeatother

%\includeonly{pgfplots.reference}


\begin{document}

\def\plotcoords{%
\addplot coordinates {
(5,8.312e-02)    (17,2.547e-02)   (49,7.407e-03)
(129,2.102e-03)  (321,5.874e-04)  (769,1.623e-04)
(1793,4.442e-05) (4097,1.207e-05) (9217,3.261e-06)
};

\addplot coordinates{
(7,8.472e-02)    (31,3.044e-02)    (111,1.022e-02)
(351,3.303e-03)  (1023,1.039e-03)  (2815,3.196e-04)
(7423,9.658e-05) (18943,2.873e-05) (47103,8.437e-06)
};

\addplot coordinates{
(9,7.881e-02)     (49,3.243e-02)    (209,1.232e-02)
(769,4.454e-03)   (2561,1.551e-03)  (7937,5.236e-04)
(23297,1.723e-04) (65537,5.545e-05) (178177,1.751e-05)
};

\addplot coordinates{
(11,6.887e-02)    (71,3.177e-02)     (351,1.341e-02)
(1471,5.334e-03)  (5503,2.027e-03)   (18943,7.415e-04)
(61183,2.628e-04) (187903,9.063e-05) (553983,3.053e-05)
};

\addplot coordinates{
(13,5.755e-02)     (97,2.925e-02)     (545,1.351e-02)
(2561,5.842e-03)   (10625,2.397e-03)  (40193,9.414e-04)
(141569,3.564e-04) (471041,1.308e-04)
(1496065,4.670e-05)
};
}%


\bookmaketitle
\tableofcontents

%%%%%%%%%%%%%%%%%%%%%%%%%%%%%%%%%%%%%%%%%%%%%%%%%%%%%%%%%%%%%%%%%%%%%%%%%%%%%
%
% Package pgfplots.sty documentation.
%
% Copyright 2007/2008 by Christian Feuersaenger.
%
% This program is free software: you can redistribute it and/or modify
% it under the terms of the GNU General Public License as published by
% the Free Software Foundation, either version 3 of the License, or
% (at your option) any later version.
%
% This program is distributed in the hope that it will be useful,
% but WITHOUT ANY WARRANTY; without even the implied warranty of
% MERCHANTABILITY or FITNESS FOR A PARTICULAR PURPOSE.  See the
% GNU General Public License for more details.
%
% You should have received a copy of the GNU General Public License
% along with this program.  If not, see <http://www.gnu.org/licenses/>.
%
%
%%%%%%%%%%%%%%%%%%%%%%%%%%%%%%%%%%%%%%%%%%%%%%%%%%%%%%%%%%%%%%%%%%%%%%%%%%%%%
% SEE pgfplots-macros.tex as well!
%\pdfminorversion=5 % to allow compression
%\pdfobjcompresslevel=2
\documentclass[a4paper,openany]{book}

\let\bookmaketitle=\maketitle

% -----------------------------------
% this here is from ltxdoc:
\usepackage{doc}[=v2]
\AtBeginDocument{\MakeShortVerb{\|}}
%\setlength{\textwidth}{355pt}
%\addtolength\marginparwidth{30pt}
%\addtolength\oddsidemargin{20pt}
%\addtolength\evensidemargin{20pt}
\def\cmd#1{\cs{\expandafter\cmd@to@cs\string#1}}
\def\cmd@to@cs#1#2{\char\number`#2\relax}
\DeclareRobustCommand\cs[1]{\texttt{\char`\\#1}}
\providecommand\marg[1]{%
  {\ttfamily\char`\{}\meta{#1}{\ttfamily\char`\}}}
\providecommand\oarg[1]{%
  {\ttfamily[}\meta{#1}{\ttfamily]}}
\providecommand\parg[1]{%
  {\ttfamily(}\meta{#1}{\ttfamily)}}
\raggedbottom

% -----------------------------------
\input{pgfplots.preamble.tex}

\makeatletter
% I want a two-column index just as in pgfmanual styles. This here
% was the best way to get one:
\def\index@prologue{\section*{Index}\addcontentsline{toc}{chapter}{Index}
}
\makeatother

%\RequirePackage[german,english,francais]{babel}

\def\matlabcolormaptext{This colormap is similar to one shipped with Matlab$^\text{\textregistered}$ under a similar name.}

\IfFileExists{tikzlibraryspy.code.tex}{%
\usetikzlibrary{spy}
}{%
    \message{ERROR: tikz SPY library NOT available. The manual will only compile partially.^^J}%
}%

\usepackage{xparse}% for colorbrewer manual

\usetikzlibrary{
    decorations.markings,
    decorations.footprints,
    shapes.arrows,
    matrix,
    positioning,
}

\usepgfplotslibrary{
    fillbetween,
    ternary,
    smithchart,
    patchplots,
    polar,
    colormaps,
    colorbrewer,
    colortol,           % docu not ready yet
}
\pgfqkeys{/codeexample}{%
    every codeexample/.append style={
        /pgfplots/every ternary axis/.append style={
            /pgfplots/legend style={fill=graphicbackground},
        }
    },
    tabsize=4,
}

\pgfplotsmanualenableexternalizationofexpensive

%\usetikzlibrary{external}
%\tikzexternalize[prefix=figures/]{pgfplots}

\title{%
    Manual for Package \PGFPlots{}\\
    {\small 2D/3D Plots in \LaTeX{}, Version \pgfplotsversion}\\
    {\small\url{https://github.com/pgf-tikz/pgfplots}}
    %\\{\small Attention: you are using an unstable development version.}
}

\makeatletter
\long\def\abstractsmuggle{%
    \centering
    \textbf{Abstract}\\[0.5cm]

    \begin{minipage}{12cm}
        \PGFPlots{} draws high-quality function plots in normal or logarithmic
        scaling with a user-friendly interface directly in \TeX{}. The user
        supplies axis labels, legend entries and the plot coordinates for one or
        more plots and \PGFPlots{} applies axis scaling, computes any logarithms
        and axis ticks and draws the plots. It supports line plots, scatter
        plots, piecewise constant plots, bar plots, area plots, mesh and surface
        plots, patch plots, contour plots, quiver plots, histogram plots, box
        plots, polar axes, ternary diagrams, smith charts and some more. It is
        based on Till Tantau's package \PGF{}/\Tikz{}.
    \end{minipage}

}%

\expandafter\date\expandafter{\@date\\[2cm]
    \abstractsmuggle
}%
\makeatother

%\includeonly{pgfplots.reference}


\begin{document}

\def\plotcoords{%
\addplot coordinates {
(5,8.312e-02)    (17,2.547e-02)   (49,7.407e-03)
(129,2.102e-03)  (321,5.874e-04)  (769,1.623e-04)
(1793,4.442e-05) (4097,1.207e-05) (9217,3.261e-06)
};

\addplot coordinates{
(7,8.472e-02)    (31,3.044e-02)    (111,1.022e-02)
(351,3.303e-03)  (1023,1.039e-03)  (2815,3.196e-04)
(7423,9.658e-05) (18943,2.873e-05) (47103,8.437e-06)
};

\addplot coordinates{
(9,7.881e-02)     (49,3.243e-02)    (209,1.232e-02)
(769,4.454e-03)   (2561,1.551e-03)  (7937,5.236e-04)
(23297,1.723e-04) (65537,5.545e-05) (178177,1.751e-05)
};

\addplot coordinates{
(11,6.887e-02)    (71,3.177e-02)     (351,1.341e-02)
(1471,5.334e-03)  (5503,2.027e-03)   (18943,7.415e-04)
(61183,2.628e-04) (187903,9.063e-05) (553983,3.053e-05)
};

\addplot coordinates{
(13,5.755e-02)     (97,2.925e-02)     (545,1.351e-02)
(2561,5.842e-03)   (10625,2.397e-03)  (40193,9.414e-04)
(141569,3.564e-04) (471041,1.308e-04)
(1496065,4.670e-05)
};
}%


\bookmaketitle
\tableofcontents

%%%%%%%%%%%%%%%%%%%%%%%%%%%%%%%%%%%%%%%%%%%%%%%%%%%%%%%%%%%%%%%%%%%%%%%%%%%%%
%
% Package pgfplots.sty documentation.
%
% Copyright 2007/2008 by Christian Feuersaenger.
%
% This program is free software: you can redistribute it and/or modify
% it under the terms of the GNU General Public License as published by
% the Free Software Foundation, either version 3 of the License, or
% (at your option) any later version.
%
% This program is distributed in the hope that it will be useful,
% but WITHOUT ANY WARRANTY; without even the implied warranty of
% MERCHANTABILITY or FITNESS FOR A PARTICULAR PURPOSE.  See the
% GNU General Public License for more details.
%
% You should have received a copy of the GNU General Public License
% along with this program.  If not, see <http://www.gnu.org/licenses/>.
%
%
%%%%%%%%%%%%%%%%%%%%%%%%%%%%%%%%%%%%%%%%%%%%%%%%%%%%%%%%%%%%%%%%%%%%%%%%%%%%%
% SEE pgfplots-macros.tex as well!
%\pdfminorversion=5 % to allow compression
%\pdfobjcompresslevel=2
\documentclass[a4paper,openany]{book}

\let\bookmaketitle=\maketitle

% -----------------------------------
% this here is from ltxdoc:
\usepackage{doc}[=v2]
\AtBeginDocument{\MakeShortVerb{\|}}
%\setlength{\textwidth}{355pt}
%\addtolength\marginparwidth{30pt}
%\addtolength\oddsidemargin{20pt}
%\addtolength\evensidemargin{20pt}
\def\cmd#1{\cs{\expandafter\cmd@to@cs\string#1}}
\def\cmd@to@cs#1#2{\char\number`#2\relax}
\DeclareRobustCommand\cs[1]{\texttt{\char`\\#1}}
\providecommand\marg[1]{%
  {\ttfamily\char`\{}\meta{#1}{\ttfamily\char`\}}}
\providecommand\oarg[1]{%
  {\ttfamily[}\meta{#1}{\ttfamily]}}
\providecommand\parg[1]{%
  {\ttfamily(}\meta{#1}{\ttfamily)}}
\raggedbottom

% -----------------------------------
\input{pgfplots.preamble.tex}

\makeatletter
% I want a two-column index just as in pgfmanual styles. This here
% was the best way to get one:
\def\index@prologue{\section*{Index}\addcontentsline{toc}{chapter}{Index}
}
\makeatother

%\RequirePackage[german,english,francais]{babel}

\def\matlabcolormaptext{This colormap is similar to one shipped with Matlab$^\text{\textregistered}$ under a similar name.}

\IfFileExists{tikzlibraryspy.code.tex}{%
\usetikzlibrary{spy}
}{%
    \message{ERROR: tikz SPY library NOT available. The manual will only compile partially.^^J}%
}%

\usepackage{xparse}% for colorbrewer manual

\usetikzlibrary{
    decorations.markings,
    decorations.footprints,
    shapes.arrows,
    matrix,
    positioning,
}

\usepgfplotslibrary{
    fillbetween,
    ternary,
    smithchart,
    patchplots,
    polar,
    colormaps,
    colorbrewer,
    colortol,           % docu not ready yet
}
\pgfqkeys{/codeexample}{%
    every codeexample/.append style={
        /pgfplots/every ternary axis/.append style={
            /pgfplots/legend style={fill=graphicbackground},
        }
    },
    tabsize=4,
}

\pgfplotsmanualenableexternalizationofexpensive

%\usetikzlibrary{external}
%\tikzexternalize[prefix=figures/]{pgfplots}

\title{%
    Manual for Package \PGFPlots{}\\
    {\small 2D/3D Plots in \LaTeX{}, Version \pgfplotsversion}\\
    {\small\url{https://github.com/pgf-tikz/pgfplots}}
    %\\{\small Attention: you are using an unstable development version.}
}

\makeatletter
\long\def\abstractsmuggle{%
    \centering
    \textbf{Abstract}\\[0.5cm]

    \begin{minipage}{12cm}
        \PGFPlots{} draws high-quality function plots in normal or logarithmic
        scaling with a user-friendly interface directly in \TeX{}. The user
        supplies axis labels, legend entries and the plot coordinates for one or
        more plots and \PGFPlots{} applies axis scaling, computes any logarithms
        and axis ticks and draws the plots. It supports line plots, scatter
        plots, piecewise constant plots, bar plots, area plots, mesh and surface
        plots, patch plots, contour plots, quiver plots, histogram plots, box
        plots, polar axes, ternary diagrams, smith charts and some more. It is
        based on Till Tantau's package \PGF{}/\Tikz{}.
    \end{minipage}

}%

\expandafter\date\expandafter{\@date\\[2cm]
    \abstractsmuggle
}%
\makeatother

%\includeonly{pgfplots.reference}


\begin{document}

\def\plotcoords{%
\addplot coordinates {
(5,8.312e-02)    (17,2.547e-02)   (49,7.407e-03)
(129,2.102e-03)  (321,5.874e-04)  (769,1.623e-04)
(1793,4.442e-05) (4097,1.207e-05) (9217,3.261e-06)
};

\addplot coordinates{
(7,8.472e-02)    (31,3.044e-02)    (111,1.022e-02)
(351,3.303e-03)  (1023,1.039e-03)  (2815,3.196e-04)
(7423,9.658e-05) (18943,2.873e-05) (47103,8.437e-06)
};

\addplot coordinates{
(9,7.881e-02)     (49,3.243e-02)    (209,1.232e-02)
(769,4.454e-03)   (2561,1.551e-03)  (7937,5.236e-04)
(23297,1.723e-04) (65537,5.545e-05) (178177,1.751e-05)
};

\addplot coordinates{
(11,6.887e-02)    (71,3.177e-02)     (351,1.341e-02)
(1471,5.334e-03)  (5503,2.027e-03)   (18943,7.415e-04)
(61183,2.628e-04) (187903,9.063e-05) (553983,3.053e-05)
};

\addplot coordinates{
(13,5.755e-02)     (97,2.925e-02)     (545,1.351e-02)
(2561,5.842e-03)   (10625,2.397e-03)  (40193,9.414e-04)
(141569,3.564e-04) (471041,1.308e-04)
(1496065,4.670e-05)
};
}%


\bookmaketitle
\tableofcontents
\include{pgfplots.title_abstract_intro}
\include{pgfplots.preliminaries}
\include{pgfplots.intro}
\include{pgfplots.reference}
\include{pgfplots.libs}
\include{pgfplots.resources}
\include{pgfplots.importexport}
\include{pgfplots.basic.reference}

\printindex

\bibliographystyle{abbrv} %gerapali} %gerabbrv} %gerunsrt.bst} %gerabbrv}% gerplain}
\nocite{pgfplotstable}
\nocite{programmingnotes}
\bibliography{pgfplots}
\end{document}

% -----------------------------------------------------------------------------
% For Stefan Pinnow as reminder on what to look for when editing the manual
% -----------------------------------------------------------------------------
% There should be no line breaks in the following environments
% - |...|
% - \declareandlabel{...}
% - \verbpdfref{...}
% If "MakeTikzPictures" isn't running through check one of the externalized
% LOG files what is the cause of that.
% -----------------------------------------------------------------------------


%%%%%%%%%%%%%%%%%%%%%%%%%%%%%%%%%%%%%%%%%%%%%%%%%%%%%%%%%%%%%%%%%%%%%%%%%%%%%
%
% Package pgfplots.sty documentation.
%
% Copyright 2007/2008 by Christian Feuersaenger.
%
% This program is free software: you can redistribute it and/or modify
% it under the terms of the GNU General Public License as published by
% the Free Software Foundation, either version 3 of the License, or
% (at your option) any later version.
%
% This program is distributed in the hope that it will be useful,
% but WITHOUT ANY WARRANTY; without even the implied warranty of
% MERCHANTABILITY or FITNESS FOR A PARTICULAR PURPOSE.  See the
% GNU General Public License for more details.
%
% You should have received a copy of the GNU General Public License
% along with this program.  If not, see <http://www.gnu.org/licenses/>.
%
%
%%%%%%%%%%%%%%%%%%%%%%%%%%%%%%%%%%%%%%%%%%%%%%%%%%%%%%%%%%%%%%%%%%%%%%%%%%%%%
% SEE pgfplots-macros.tex as well!
%\pdfminorversion=5 % to allow compression
%\pdfobjcompresslevel=2
\documentclass[a4paper,openany]{book}

\let\bookmaketitle=\maketitle

% -----------------------------------
% this here is from ltxdoc:
\usepackage{doc}[=v2]
\AtBeginDocument{\MakeShortVerb{\|}}
%\setlength{\textwidth}{355pt}
%\addtolength\marginparwidth{30pt}
%\addtolength\oddsidemargin{20pt}
%\addtolength\evensidemargin{20pt}
\def\cmd#1{\cs{\expandafter\cmd@to@cs\string#1}}
\def\cmd@to@cs#1#2{\char\number`#2\relax}
\DeclareRobustCommand\cs[1]{\texttt{\char`\\#1}}
\providecommand\marg[1]{%
  {\ttfamily\char`\{}\meta{#1}{\ttfamily\char`\}}}
\providecommand\oarg[1]{%
  {\ttfamily[}\meta{#1}{\ttfamily]}}
\providecommand\parg[1]{%
  {\ttfamily(}\meta{#1}{\ttfamily)}}
\raggedbottom

% -----------------------------------
\input{pgfplots.preamble.tex}

\makeatletter
% I want a two-column index just as in pgfmanual styles. This here
% was the best way to get one:
\def\index@prologue{\section*{Index}\addcontentsline{toc}{chapter}{Index}
}
\makeatother

%\RequirePackage[german,english,francais]{babel}

\def\matlabcolormaptext{This colormap is similar to one shipped with Matlab$^\text{\textregistered}$ under a similar name.}

\IfFileExists{tikzlibraryspy.code.tex}{%
\usetikzlibrary{spy}
}{%
    \message{ERROR: tikz SPY library NOT available. The manual will only compile partially.^^J}%
}%

\usepackage{xparse}% for colorbrewer manual

\usetikzlibrary{
    decorations.markings,
    decorations.footprints,
    shapes.arrows,
    matrix,
    positioning,
}

\usepgfplotslibrary{
    fillbetween,
    ternary,
    smithchart,
    patchplots,
    polar,
    colormaps,
    colorbrewer,
    colortol,           % docu not ready yet
}
\pgfqkeys{/codeexample}{%
    every codeexample/.append style={
        /pgfplots/every ternary axis/.append style={
            /pgfplots/legend style={fill=graphicbackground},
        }
    },
    tabsize=4,
}

\pgfplotsmanualenableexternalizationofexpensive

%\usetikzlibrary{external}
%\tikzexternalize[prefix=figures/]{pgfplots}

\title{%
    Manual for Package \PGFPlots{}\\
    {\small 2D/3D Plots in \LaTeX{}, Version \pgfplotsversion}\\
    {\small\url{https://github.com/pgf-tikz/pgfplots}}
    %\\{\small Attention: you are using an unstable development version.}
}

\makeatletter
\long\def\abstractsmuggle{%
    \centering
    \textbf{Abstract}\\[0.5cm]

    \begin{minipage}{12cm}
        \PGFPlots{} draws high-quality function plots in normal or logarithmic
        scaling with a user-friendly interface directly in \TeX{}. The user
        supplies axis labels, legend entries and the plot coordinates for one or
        more plots and \PGFPlots{} applies axis scaling, computes any logarithms
        and axis ticks and draws the plots. It supports line plots, scatter
        plots, piecewise constant plots, bar plots, area plots, mesh and surface
        plots, patch plots, contour plots, quiver plots, histogram plots, box
        plots, polar axes, ternary diagrams, smith charts and some more. It is
        based on Till Tantau's package \PGF{}/\Tikz{}.
    \end{minipage}

}%

\expandafter\date\expandafter{\@date\\[2cm]
    \abstractsmuggle
}%
\makeatother

%\includeonly{pgfplots.reference}


\begin{document}

\def\plotcoords{%
\addplot coordinates {
(5,8.312e-02)    (17,2.547e-02)   (49,7.407e-03)
(129,2.102e-03)  (321,5.874e-04)  (769,1.623e-04)
(1793,4.442e-05) (4097,1.207e-05) (9217,3.261e-06)
};

\addplot coordinates{
(7,8.472e-02)    (31,3.044e-02)    (111,1.022e-02)
(351,3.303e-03)  (1023,1.039e-03)  (2815,3.196e-04)
(7423,9.658e-05) (18943,2.873e-05) (47103,8.437e-06)
};

\addplot coordinates{
(9,7.881e-02)     (49,3.243e-02)    (209,1.232e-02)
(769,4.454e-03)   (2561,1.551e-03)  (7937,5.236e-04)
(23297,1.723e-04) (65537,5.545e-05) (178177,1.751e-05)
};

\addplot coordinates{
(11,6.887e-02)    (71,3.177e-02)     (351,1.341e-02)
(1471,5.334e-03)  (5503,2.027e-03)   (18943,7.415e-04)
(61183,2.628e-04) (187903,9.063e-05) (553983,3.053e-05)
};

\addplot coordinates{
(13,5.755e-02)     (97,2.925e-02)     (545,1.351e-02)
(2561,5.842e-03)   (10625,2.397e-03)  (40193,9.414e-04)
(141569,3.564e-04) (471041,1.308e-04)
(1496065,4.670e-05)
};
}%


\bookmaketitle
\tableofcontents
\include{pgfplots.title_abstract_intro}
\include{pgfplots.preliminaries}
\include{pgfplots.intro}
\include{pgfplots.reference}
\include{pgfplots.libs}
\include{pgfplots.resources}
\include{pgfplots.importexport}
\include{pgfplots.basic.reference}

\printindex

\bibliographystyle{abbrv} %gerapali} %gerabbrv} %gerunsrt.bst} %gerabbrv}% gerplain}
\nocite{pgfplotstable}
\nocite{programmingnotes}
\bibliography{pgfplots}
\end{document}

% -----------------------------------------------------------------------------
% For Stefan Pinnow as reminder on what to look for when editing the manual
% -----------------------------------------------------------------------------
% There should be no line breaks in the following environments
% - |...|
% - \declareandlabel{...}
% - \verbpdfref{...}
% If "MakeTikzPictures" isn't running through check one of the externalized
% LOG files what is the cause of that.
% -----------------------------------------------------------------------------


%%%%%%%%%%%%%%%%%%%%%%%%%%%%%%%%%%%%%%%%%%%%%%%%%%%%%%%%%%%%%%%%%%%%%%%%%%%%%
%
% Package pgfplots.sty documentation.
%
% Copyright 2007/2008 by Christian Feuersaenger.
%
% This program is free software: you can redistribute it and/or modify
% it under the terms of the GNU General Public License as published by
% the Free Software Foundation, either version 3 of the License, or
% (at your option) any later version.
%
% This program is distributed in the hope that it will be useful,
% but WITHOUT ANY WARRANTY; without even the implied warranty of
% MERCHANTABILITY or FITNESS FOR A PARTICULAR PURPOSE.  See the
% GNU General Public License for more details.
%
% You should have received a copy of the GNU General Public License
% along with this program.  If not, see <http://www.gnu.org/licenses/>.
%
%
%%%%%%%%%%%%%%%%%%%%%%%%%%%%%%%%%%%%%%%%%%%%%%%%%%%%%%%%%%%%%%%%%%%%%%%%%%%%%
% SEE pgfplots-macros.tex as well!
%\pdfminorversion=5 % to allow compression
%\pdfobjcompresslevel=2
\documentclass[a4paper,openany]{book}

\let\bookmaketitle=\maketitle

% -----------------------------------
% this here is from ltxdoc:
\usepackage{doc}[=v2]
\AtBeginDocument{\MakeShortVerb{\|}}
%\setlength{\textwidth}{355pt}
%\addtolength\marginparwidth{30pt}
%\addtolength\oddsidemargin{20pt}
%\addtolength\evensidemargin{20pt}
\def\cmd#1{\cs{\expandafter\cmd@to@cs\string#1}}
\def\cmd@to@cs#1#2{\char\number`#2\relax}
\DeclareRobustCommand\cs[1]{\texttt{\char`\\#1}}
\providecommand\marg[1]{%
  {\ttfamily\char`\{}\meta{#1}{\ttfamily\char`\}}}
\providecommand\oarg[1]{%
  {\ttfamily[}\meta{#1}{\ttfamily]}}
\providecommand\parg[1]{%
  {\ttfamily(}\meta{#1}{\ttfamily)}}
\raggedbottom

% -----------------------------------
\input{pgfplots.preamble.tex}

\makeatletter
% I want a two-column index just as in pgfmanual styles. This here
% was the best way to get one:
\def\index@prologue{\section*{Index}\addcontentsline{toc}{chapter}{Index}
}
\makeatother

%\RequirePackage[german,english,francais]{babel}

\def\matlabcolormaptext{This colormap is similar to one shipped with Matlab$^\text{\textregistered}$ under a similar name.}

\IfFileExists{tikzlibraryspy.code.tex}{%
\usetikzlibrary{spy}
}{%
    \message{ERROR: tikz SPY library NOT available. The manual will only compile partially.^^J}%
}%

\usepackage{xparse}% for colorbrewer manual

\usetikzlibrary{
    decorations.markings,
    decorations.footprints,
    shapes.arrows,
    matrix,
    positioning,
}

\usepgfplotslibrary{
    fillbetween,
    ternary,
    smithchart,
    patchplots,
    polar,
    colormaps,
    colorbrewer,
    colortol,           % docu not ready yet
}
\pgfqkeys{/codeexample}{%
    every codeexample/.append style={
        /pgfplots/every ternary axis/.append style={
            /pgfplots/legend style={fill=graphicbackground},
        }
    },
    tabsize=4,
}

\pgfplotsmanualenableexternalizationofexpensive

%\usetikzlibrary{external}
%\tikzexternalize[prefix=figures/]{pgfplots}

\title{%
    Manual for Package \PGFPlots{}\\
    {\small 2D/3D Plots in \LaTeX{}, Version \pgfplotsversion}\\
    {\small\url{https://github.com/pgf-tikz/pgfplots}}
    %\\{\small Attention: you are using an unstable development version.}
}

\makeatletter
\long\def\abstractsmuggle{%
    \centering
    \textbf{Abstract}\\[0.5cm]

    \begin{minipage}{12cm}
        \PGFPlots{} draws high-quality function plots in normal or logarithmic
        scaling with a user-friendly interface directly in \TeX{}. The user
        supplies axis labels, legend entries and the plot coordinates for one or
        more plots and \PGFPlots{} applies axis scaling, computes any logarithms
        and axis ticks and draws the plots. It supports line plots, scatter
        plots, piecewise constant plots, bar plots, area plots, mesh and surface
        plots, patch plots, contour plots, quiver plots, histogram plots, box
        plots, polar axes, ternary diagrams, smith charts and some more. It is
        based on Till Tantau's package \PGF{}/\Tikz{}.
    \end{minipage}

}%

\expandafter\date\expandafter{\@date\\[2cm]
    \abstractsmuggle
}%
\makeatother

%\includeonly{pgfplots.reference}


\begin{document}

\def\plotcoords{%
\addplot coordinates {
(5,8.312e-02)    (17,2.547e-02)   (49,7.407e-03)
(129,2.102e-03)  (321,5.874e-04)  (769,1.623e-04)
(1793,4.442e-05) (4097,1.207e-05) (9217,3.261e-06)
};

\addplot coordinates{
(7,8.472e-02)    (31,3.044e-02)    (111,1.022e-02)
(351,3.303e-03)  (1023,1.039e-03)  (2815,3.196e-04)
(7423,9.658e-05) (18943,2.873e-05) (47103,8.437e-06)
};

\addplot coordinates{
(9,7.881e-02)     (49,3.243e-02)    (209,1.232e-02)
(769,4.454e-03)   (2561,1.551e-03)  (7937,5.236e-04)
(23297,1.723e-04) (65537,5.545e-05) (178177,1.751e-05)
};

\addplot coordinates{
(11,6.887e-02)    (71,3.177e-02)     (351,1.341e-02)
(1471,5.334e-03)  (5503,2.027e-03)   (18943,7.415e-04)
(61183,2.628e-04) (187903,9.063e-05) (553983,3.053e-05)
};

\addplot coordinates{
(13,5.755e-02)     (97,2.925e-02)     (545,1.351e-02)
(2561,5.842e-03)   (10625,2.397e-03)  (40193,9.414e-04)
(141569,3.564e-04) (471041,1.308e-04)
(1496065,4.670e-05)
};
}%


\bookmaketitle
\tableofcontents
\include{pgfplots.title_abstract_intro}
\include{pgfplots.preliminaries}
\include{pgfplots.intro}
\include{pgfplots.reference}
\include{pgfplots.libs}
\include{pgfplots.resources}
\include{pgfplots.importexport}
\include{pgfplots.basic.reference}

\printindex

\bibliographystyle{abbrv} %gerapali} %gerabbrv} %gerunsrt.bst} %gerabbrv}% gerplain}
\nocite{pgfplotstable}
\nocite{programmingnotes}
\bibliography{pgfplots}
\end{document}

% -----------------------------------------------------------------------------
% For Stefan Pinnow as reminder on what to look for when editing the manual
% -----------------------------------------------------------------------------
% There should be no line breaks in the following environments
% - |...|
% - \declareandlabel{...}
% - \verbpdfref{...}
% If "MakeTikzPictures" isn't running through check one of the externalized
% LOG files what is the cause of that.
% -----------------------------------------------------------------------------


%%%%%%%%%%%%%%%%%%%%%%%%%%%%%%%%%%%%%%%%%%%%%%%%%%%%%%%%%%%%%%%%%%%%%%%%%%%%%
%
% Package pgfplots.sty documentation.
%
% Copyright 2007/2008 by Christian Feuersaenger.
%
% This program is free software: you can redistribute it and/or modify
% it under the terms of the GNU General Public License as published by
% the Free Software Foundation, either version 3 of the License, or
% (at your option) any later version.
%
% This program is distributed in the hope that it will be useful,
% but WITHOUT ANY WARRANTY; without even the implied warranty of
% MERCHANTABILITY or FITNESS FOR A PARTICULAR PURPOSE.  See the
% GNU General Public License for more details.
%
% You should have received a copy of the GNU General Public License
% along with this program.  If not, see <http://www.gnu.org/licenses/>.
%
%
%%%%%%%%%%%%%%%%%%%%%%%%%%%%%%%%%%%%%%%%%%%%%%%%%%%%%%%%%%%%%%%%%%%%%%%%%%%%%
% SEE pgfplots-macros.tex as well!
%\pdfminorversion=5 % to allow compression
%\pdfobjcompresslevel=2
\documentclass[a4paper,openany]{book}

\let\bookmaketitle=\maketitle

% -----------------------------------
% this here is from ltxdoc:
\usepackage{doc}[=v2]
\AtBeginDocument{\MakeShortVerb{\|}}
%\setlength{\textwidth}{355pt}
%\addtolength\marginparwidth{30pt}
%\addtolength\oddsidemargin{20pt}
%\addtolength\evensidemargin{20pt}
\def\cmd#1{\cs{\expandafter\cmd@to@cs\string#1}}
\def\cmd@to@cs#1#2{\char\number`#2\relax}
\DeclareRobustCommand\cs[1]{\texttt{\char`\\#1}}
\providecommand\marg[1]{%
  {\ttfamily\char`\{}\meta{#1}{\ttfamily\char`\}}}
\providecommand\oarg[1]{%
  {\ttfamily[}\meta{#1}{\ttfamily]}}
\providecommand\parg[1]{%
  {\ttfamily(}\meta{#1}{\ttfamily)}}
\raggedbottom

% -----------------------------------
\input{pgfplots.preamble.tex}

\makeatletter
% I want a two-column index just as in pgfmanual styles. This here
% was the best way to get one:
\def\index@prologue{\section*{Index}\addcontentsline{toc}{chapter}{Index}
}
\makeatother

%\RequirePackage[german,english,francais]{babel}

\def\matlabcolormaptext{This colormap is similar to one shipped with Matlab$^\text{\textregistered}$ under a similar name.}

\IfFileExists{tikzlibraryspy.code.tex}{%
\usetikzlibrary{spy}
}{%
    \message{ERROR: tikz SPY library NOT available. The manual will only compile partially.^^J}%
}%

\usepackage{xparse}% for colorbrewer manual

\usetikzlibrary{
    decorations.markings,
    decorations.footprints,
    shapes.arrows,
    matrix,
    positioning,
}

\usepgfplotslibrary{
    fillbetween,
    ternary,
    smithchart,
    patchplots,
    polar,
    colormaps,
    colorbrewer,
    colortol,           % docu not ready yet
}
\pgfqkeys{/codeexample}{%
    every codeexample/.append style={
        /pgfplots/every ternary axis/.append style={
            /pgfplots/legend style={fill=graphicbackground},
        }
    },
    tabsize=4,
}

\pgfplotsmanualenableexternalizationofexpensive

%\usetikzlibrary{external}
%\tikzexternalize[prefix=figures/]{pgfplots}

\title{%
    Manual for Package \PGFPlots{}\\
    {\small 2D/3D Plots in \LaTeX{}, Version \pgfplotsversion}\\
    {\small\url{https://github.com/pgf-tikz/pgfplots}}
    %\\{\small Attention: you are using an unstable development version.}
}

\makeatletter
\long\def\abstractsmuggle{%
    \centering
    \textbf{Abstract}\\[0.5cm]

    \begin{minipage}{12cm}
        \PGFPlots{} draws high-quality function plots in normal or logarithmic
        scaling with a user-friendly interface directly in \TeX{}. The user
        supplies axis labels, legend entries and the plot coordinates for one or
        more plots and \PGFPlots{} applies axis scaling, computes any logarithms
        and axis ticks and draws the plots. It supports line plots, scatter
        plots, piecewise constant plots, bar plots, area plots, mesh and surface
        plots, patch plots, contour plots, quiver plots, histogram plots, box
        plots, polar axes, ternary diagrams, smith charts and some more. It is
        based on Till Tantau's package \PGF{}/\Tikz{}.
    \end{minipage}

}%

\expandafter\date\expandafter{\@date\\[2cm]
    \abstractsmuggle
}%
\makeatother

%\includeonly{pgfplots.reference}


\begin{document}

\def\plotcoords{%
\addplot coordinates {
(5,8.312e-02)    (17,2.547e-02)   (49,7.407e-03)
(129,2.102e-03)  (321,5.874e-04)  (769,1.623e-04)
(1793,4.442e-05) (4097,1.207e-05) (9217,3.261e-06)
};

\addplot coordinates{
(7,8.472e-02)    (31,3.044e-02)    (111,1.022e-02)
(351,3.303e-03)  (1023,1.039e-03)  (2815,3.196e-04)
(7423,9.658e-05) (18943,2.873e-05) (47103,8.437e-06)
};

\addplot coordinates{
(9,7.881e-02)     (49,3.243e-02)    (209,1.232e-02)
(769,4.454e-03)   (2561,1.551e-03)  (7937,5.236e-04)
(23297,1.723e-04) (65537,5.545e-05) (178177,1.751e-05)
};

\addplot coordinates{
(11,6.887e-02)    (71,3.177e-02)     (351,1.341e-02)
(1471,5.334e-03)  (5503,2.027e-03)   (18943,7.415e-04)
(61183,2.628e-04) (187903,9.063e-05) (553983,3.053e-05)
};

\addplot coordinates{
(13,5.755e-02)     (97,2.925e-02)     (545,1.351e-02)
(2561,5.842e-03)   (10625,2.397e-03)  (40193,9.414e-04)
(141569,3.564e-04) (471041,1.308e-04)
(1496065,4.670e-05)
};
}%


\bookmaketitle
\tableofcontents
\include{pgfplots.title_abstract_intro}
\include{pgfplots.preliminaries}
\include{pgfplots.intro}
\include{pgfplots.reference}
\include{pgfplots.libs}
\include{pgfplots.resources}
\include{pgfplots.importexport}
\include{pgfplots.basic.reference}

\printindex

\bibliographystyle{abbrv} %gerapali} %gerabbrv} %gerunsrt.bst} %gerabbrv}% gerplain}
\nocite{pgfplotstable}
\nocite{programmingnotes}
\bibliography{pgfplots}
\end{document}

% -----------------------------------------------------------------------------
% For Stefan Pinnow as reminder on what to look for when editing the manual
% -----------------------------------------------------------------------------
% There should be no line breaks in the following environments
% - |...|
% - \declareandlabel{...}
% - \verbpdfref{...}
% If "MakeTikzPictures" isn't running through check one of the externalized
% LOG files what is the cause of that.
% -----------------------------------------------------------------------------


%%%%%%%%%%%%%%%%%%%%%%%%%%%%%%%%%%%%%%%%%%%%%%%%%%%%%%%%%%%%%%%%%%%%%%%%%%%%%
%
% Package pgfplots.sty documentation.
%
% Copyright 2007/2008 by Christian Feuersaenger.
%
% This program is free software: you can redistribute it and/or modify
% it under the terms of the GNU General Public License as published by
% the Free Software Foundation, either version 3 of the License, or
% (at your option) any later version.
%
% This program is distributed in the hope that it will be useful,
% but WITHOUT ANY WARRANTY; without even the implied warranty of
% MERCHANTABILITY or FITNESS FOR A PARTICULAR PURPOSE.  See the
% GNU General Public License for more details.
%
% You should have received a copy of the GNU General Public License
% along with this program.  If not, see <http://www.gnu.org/licenses/>.
%
%
%%%%%%%%%%%%%%%%%%%%%%%%%%%%%%%%%%%%%%%%%%%%%%%%%%%%%%%%%%%%%%%%%%%%%%%%%%%%%
% SEE pgfplots-macros.tex as well!
%\pdfminorversion=5 % to allow compression
%\pdfobjcompresslevel=2
\documentclass[a4paper,openany]{book}

\let\bookmaketitle=\maketitle

% -----------------------------------
% this here is from ltxdoc:
\usepackage{doc}[=v2]
\AtBeginDocument{\MakeShortVerb{\|}}
%\setlength{\textwidth}{355pt}
%\addtolength\marginparwidth{30pt}
%\addtolength\oddsidemargin{20pt}
%\addtolength\evensidemargin{20pt}
\def\cmd#1{\cs{\expandafter\cmd@to@cs\string#1}}
\def\cmd@to@cs#1#2{\char\number`#2\relax}
\DeclareRobustCommand\cs[1]{\texttt{\char`\\#1}}
\providecommand\marg[1]{%
  {\ttfamily\char`\{}\meta{#1}{\ttfamily\char`\}}}
\providecommand\oarg[1]{%
  {\ttfamily[}\meta{#1}{\ttfamily]}}
\providecommand\parg[1]{%
  {\ttfamily(}\meta{#1}{\ttfamily)}}
\raggedbottom

% -----------------------------------
\input{pgfplots.preamble.tex}

\makeatletter
% I want a two-column index just as in pgfmanual styles. This here
% was the best way to get one:
\def\index@prologue{\section*{Index}\addcontentsline{toc}{chapter}{Index}
}
\makeatother

%\RequirePackage[german,english,francais]{babel}

\def\matlabcolormaptext{This colormap is similar to one shipped with Matlab$^\text{\textregistered}$ under a similar name.}

\IfFileExists{tikzlibraryspy.code.tex}{%
\usetikzlibrary{spy}
}{%
    \message{ERROR: tikz SPY library NOT available. The manual will only compile partially.^^J}%
}%

\usepackage{xparse}% for colorbrewer manual

\usetikzlibrary{
    decorations.markings,
    decorations.footprints,
    shapes.arrows,
    matrix,
    positioning,
}

\usepgfplotslibrary{
    fillbetween,
    ternary,
    smithchart,
    patchplots,
    polar,
    colormaps,
    colorbrewer,
    colortol,           % docu not ready yet
}
\pgfqkeys{/codeexample}{%
    every codeexample/.append style={
        /pgfplots/every ternary axis/.append style={
            /pgfplots/legend style={fill=graphicbackground},
        }
    },
    tabsize=4,
}

\pgfplotsmanualenableexternalizationofexpensive

%\usetikzlibrary{external}
%\tikzexternalize[prefix=figures/]{pgfplots}

\title{%
    Manual for Package \PGFPlots{}\\
    {\small 2D/3D Plots in \LaTeX{}, Version \pgfplotsversion}\\
    {\small\url{https://github.com/pgf-tikz/pgfplots}}
    %\\{\small Attention: you are using an unstable development version.}
}

\makeatletter
\long\def\abstractsmuggle{%
    \centering
    \textbf{Abstract}\\[0.5cm]

    \begin{minipage}{12cm}
        \PGFPlots{} draws high-quality function plots in normal or logarithmic
        scaling with a user-friendly interface directly in \TeX{}. The user
        supplies axis labels, legend entries and the plot coordinates for one or
        more plots and \PGFPlots{} applies axis scaling, computes any logarithms
        and axis ticks and draws the plots. It supports line plots, scatter
        plots, piecewise constant plots, bar plots, area plots, mesh and surface
        plots, patch plots, contour plots, quiver plots, histogram plots, box
        plots, polar axes, ternary diagrams, smith charts and some more. It is
        based on Till Tantau's package \PGF{}/\Tikz{}.
    \end{minipage}

}%

\expandafter\date\expandafter{\@date\\[2cm]
    \abstractsmuggle
}%
\makeatother

%\includeonly{pgfplots.reference}


\begin{document}

\def\plotcoords{%
\addplot coordinates {
(5,8.312e-02)    (17,2.547e-02)   (49,7.407e-03)
(129,2.102e-03)  (321,5.874e-04)  (769,1.623e-04)
(1793,4.442e-05) (4097,1.207e-05) (9217,3.261e-06)
};

\addplot coordinates{
(7,8.472e-02)    (31,3.044e-02)    (111,1.022e-02)
(351,3.303e-03)  (1023,1.039e-03)  (2815,3.196e-04)
(7423,9.658e-05) (18943,2.873e-05) (47103,8.437e-06)
};

\addplot coordinates{
(9,7.881e-02)     (49,3.243e-02)    (209,1.232e-02)
(769,4.454e-03)   (2561,1.551e-03)  (7937,5.236e-04)
(23297,1.723e-04) (65537,5.545e-05) (178177,1.751e-05)
};

\addplot coordinates{
(11,6.887e-02)    (71,3.177e-02)     (351,1.341e-02)
(1471,5.334e-03)  (5503,2.027e-03)   (18943,7.415e-04)
(61183,2.628e-04) (187903,9.063e-05) (553983,3.053e-05)
};

\addplot coordinates{
(13,5.755e-02)     (97,2.925e-02)     (545,1.351e-02)
(2561,5.842e-03)   (10625,2.397e-03)  (40193,9.414e-04)
(141569,3.564e-04) (471041,1.308e-04)
(1496065,4.670e-05)
};
}%


\bookmaketitle
\tableofcontents
\include{pgfplots.title_abstract_intro}
\include{pgfplots.preliminaries}
\include{pgfplots.intro}
\include{pgfplots.reference}
\include{pgfplots.libs}
\include{pgfplots.resources}
\include{pgfplots.importexport}
\include{pgfplots.basic.reference}

\printindex

\bibliographystyle{abbrv} %gerapali} %gerabbrv} %gerunsrt.bst} %gerabbrv}% gerplain}
\nocite{pgfplotstable}
\nocite{programmingnotes}
\bibliography{pgfplots}
\end{document}

% -----------------------------------------------------------------------------
% For Stefan Pinnow as reminder on what to look for when editing the manual
% -----------------------------------------------------------------------------
% There should be no line breaks in the following environments
% - |...|
% - \declareandlabel{...}
% - \verbpdfref{...}
% If "MakeTikzPictures" isn't running through check one of the externalized
% LOG files what is the cause of that.
% -----------------------------------------------------------------------------


%%%%%%%%%%%%%%%%%%%%%%%%%%%%%%%%%%%%%%%%%%%%%%%%%%%%%%%%%%%%%%%%%%%%%%%%%%%%%
%
% Package pgfplots.sty documentation.
%
% Copyright 2007/2008 by Christian Feuersaenger.
%
% This program is free software: you can redistribute it and/or modify
% it under the terms of the GNU General Public License as published by
% the Free Software Foundation, either version 3 of the License, or
% (at your option) any later version.
%
% This program is distributed in the hope that it will be useful,
% but WITHOUT ANY WARRANTY; without even the implied warranty of
% MERCHANTABILITY or FITNESS FOR A PARTICULAR PURPOSE.  See the
% GNU General Public License for more details.
%
% You should have received a copy of the GNU General Public License
% along with this program.  If not, see <http://www.gnu.org/licenses/>.
%
%
%%%%%%%%%%%%%%%%%%%%%%%%%%%%%%%%%%%%%%%%%%%%%%%%%%%%%%%%%%%%%%%%%%%%%%%%%%%%%
% SEE pgfplots-macros.tex as well!
%\pdfminorversion=5 % to allow compression
%\pdfobjcompresslevel=2
\documentclass[a4paper,openany]{book}

\let\bookmaketitle=\maketitle

% -----------------------------------
% this here is from ltxdoc:
\usepackage{doc}[=v2]
\AtBeginDocument{\MakeShortVerb{\|}}
%\setlength{\textwidth}{355pt}
%\addtolength\marginparwidth{30pt}
%\addtolength\oddsidemargin{20pt}
%\addtolength\evensidemargin{20pt}
\def\cmd#1{\cs{\expandafter\cmd@to@cs\string#1}}
\def\cmd@to@cs#1#2{\char\number`#2\relax}
\DeclareRobustCommand\cs[1]{\texttt{\char`\\#1}}
\providecommand\marg[1]{%
  {\ttfamily\char`\{}\meta{#1}{\ttfamily\char`\}}}
\providecommand\oarg[1]{%
  {\ttfamily[}\meta{#1}{\ttfamily]}}
\providecommand\parg[1]{%
  {\ttfamily(}\meta{#1}{\ttfamily)}}
\raggedbottom

% -----------------------------------
\input{pgfplots.preamble.tex}

\makeatletter
% I want a two-column index just as in pgfmanual styles. This here
% was the best way to get one:
\def\index@prologue{\section*{Index}\addcontentsline{toc}{chapter}{Index}
}
\makeatother

%\RequirePackage[german,english,francais]{babel}

\def\matlabcolormaptext{This colormap is similar to one shipped with Matlab$^\text{\textregistered}$ under a similar name.}

\IfFileExists{tikzlibraryspy.code.tex}{%
\usetikzlibrary{spy}
}{%
    \message{ERROR: tikz SPY library NOT available. The manual will only compile partially.^^J}%
}%

\usepackage{xparse}% for colorbrewer manual

\usetikzlibrary{
    decorations.markings,
    decorations.footprints,
    shapes.arrows,
    matrix,
    positioning,
}

\usepgfplotslibrary{
    fillbetween,
    ternary,
    smithchart,
    patchplots,
    polar,
    colormaps,
    colorbrewer,
    colortol,           % docu not ready yet
}
\pgfqkeys{/codeexample}{%
    every codeexample/.append style={
        /pgfplots/every ternary axis/.append style={
            /pgfplots/legend style={fill=graphicbackground},
        }
    },
    tabsize=4,
}

\pgfplotsmanualenableexternalizationofexpensive

%\usetikzlibrary{external}
%\tikzexternalize[prefix=figures/]{pgfplots}

\title{%
    Manual for Package \PGFPlots{}\\
    {\small 2D/3D Plots in \LaTeX{}, Version \pgfplotsversion}\\
    {\small\url{https://github.com/pgf-tikz/pgfplots}}
    %\\{\small Attention: you are using an unstable development version.}
}

\makeatletter
\long\def\abstractsmuggle{%
    \centering
    \textbf{Abstract}\\[0.5cm]

    \begin{minipage}{12cm}
        \PGFPlots{} draws high-quality function plots in normal or logarithmic
        scaling with a user-friendly interface directly in \TeX{}. The user
        supplies axis labels, legend entries and the plot coordinates for one or
        more plots and \PGFPlots{} applies axis scaling, computes any logarithms
        and axis ticks and draws the plots. It supports line plots, scatter
        plots, piecewise constant plots, bar plots, area plots, mesh and surface
        plots, patch plots, contour plots, quiver plots, histogram plots, box
        plots, polar axes, ternary diagrams, smith charts and some more. It is
        based on Till Tantau's package \PGF{}/\Tikz{}.
    \end{minipage}

}%

\expandafter\date\expandafter{\@date\\[2cm]
    \abstractsmuggle
}%
\makeatother

%\includeonly{pgfplots.reference}


\begin{document}

\def\plotcoords{%
\addplot coordinates {
(5,8.312e-02)    (17,2.547e-02)   (49,7.407e-03)
(129,2.102e-03)  (321,5.874e-04)  (769,1.623e-04)
(1793,4.442e-05) (4097,1.207e-05) (9217,3.261e-06)
};

\addplot coordinates{
(7,8.472e-02)    (31,3.044e-02)    (111,1.022e-02)
(351,3.303e-03)  (1023,1.039e-03)  (2815,3.196e-04)
(7423,9.658e-05) (18943,2.873e-05) (47103,8.437e-06)
};

\addplot coordinates{
(9,7.881e-02)     (49,3.243e-02)    (209,1.232e-02)
(769,4.454e-03)   (2561,1.551e-03)  (7937,5.236e-04)
(23297,1.723e-04) (65537,5.545e-05) (178177,1.751e-05)
};

\addplot coordinates{
(11,6.887e-02)    (71,3.177e-02)     (351,1.341e-02)
(1471,5.334e-03)  (5503,2.027e-03)   (18943,7.415e-04)
(61183,2.628e-04) (187903,9.063e-05) (553983,3.053e-05)
};

\addplot coordinates{
(13,5.755e-02)     (97,2.925e-02)     (545,1.351e-02)
(2561,5.842e-03)   (10625,2.397e-03)  (40193,9.414e-04)
(141569,3.564e-04) (471041,1.308e-04)
(1496065,4.670e-05)
};
}%


\bookmaketitle
\tableofcontents
\include{pgfplots.title_abstract_intro}
\include{pgfplots.preliminaries}
\include{pgfplots.intro}
\include{pgfplots.reference}
\include{pgfplots.libs}
\include{pgfplots.resources}
\include{pgfplots.importexport}
\include{pgfplots.basic.reference}

\printindex

\bibliographystyle{abbrv} %gerapali} %gerabbrv} %gerunsrt.bst} %gerabbrv}% gerplain}
\nocite{pgfplotstable}
\nocite{programmingnotes}
\bibliography{pgfplots}
\end{document}

% -----------------------------------------------------------------------------
% For Stefan Pinnow as reminder on what to look for when editing the manual
% -----------------------------------------------------------------------------
% There should be no line breaks in the following environments
% - |...|
% - \declareandlabel{...}
% - \verbpdfref{...}
% If "MakeTikzPictures" isn't running through check one of the externalized
% LOG files what is the cause of that.
% -----------------------------------------------------------------------------


%%%%%%%%%%%%%%%%%%%%%%%%%%%%%%%%%%%%%%%%%%%%%%%%%%%%%%%%%%%%%%%%%%%%%%%%%%%%%
%
% Package pgfplots.sty documentation.
%
% Copyright 2007/2008 by Christian Feuersaenger.
%
% This program is free software: you can redistribute it and/or modify
% it under the terms of the GNU General Public License as published by
% the Free Software Foundation, either version 3 of the License, or
% (at your option) any later version.
%
% This program is distributed in the hope that it will be useful,
% but WITHOUT ANY WARRANTY; without even the implied warranty of
% MERCHANTABILITY or FITNESS FOR A PARTICULAR PURPOSE.  See the
% GNU General Public License for more details.
%
% You should have received a copy of the GNU General Public License
% along with this program.  If not, see <http://www.gnu.org/licenses/>.
%
%
%%%%%%%%%%%%%%%%%%%%%%%%%%%%%%%%%%%%%%%%%%%%%%%%%%%%%%%%%%%%%%%%%%%%%%%%%%%%%
% SEE pgfplots-macros.tex as well!
%\pdfminorversion=5 % to allow compression
%\pdfobjcompresslevel=2
\documentclass[a4paper,openany]{book}

\let\bookmaketitle=\maketitle

% -----------------------------------
% this here is from ltxdoc:
\usepackage{doc}[=v2]
\AtBeginDocument{\MakeShortVerb{\|}}
%\setlength{\textwidth}{355pt}
%\addtolength\marginparwidth{30pt}
%\addtolength\oddsidemargin{20pt}
%\addtolength\evensidemargin{20pt}
\def\cmd#1{\cs{\expandafter\cmd@to@cs\string#1}}
\def\cmd@to@cs#1#2{\char\number`#2\relax}
\DeclareRobustCommand\cs[1]{\texttt{\char`\\#1}}
\providecommand\marg[1]{%
  {\ttfamily\char`\{}\meta{#1}{\ttfamily\char`\}}}
\providecommand\oarg[1]{%
  {\ttfamily[}\meta{#1}{\ttfamily]}}
\providecommand\parg[1]{%
  {\ttfamily(}\meta{#1}{\ttfamily)}}
\raggedbottom

% -----------------------------------
\input{pgfplots.preamble.tex}

\makeatletter
% I want a two-column index just as in pgfmanual styles. This here
% was the best way to get one:
\def\index@prologue{\section*{Index}\addcontentsline{toc}{chapter}{Index}
}
\makeatother

%\RequirePackage[german,english,francais]{babel}

\def\matlabcolormaptext{This colormap is similar to one shipped with Matlab$^\text{\textregistered}$ under a similar name.}

\IfFileExists{tikzlibraryspy.code.tex}{%
\usetikzlibrary{spy}
}{%
    \message{ERROR: tikz SPY library NOT available. The manual will only compile partially.^^J}%
}%

\usepackage{xparse}% for colorbrewer manual

\usetikzlibrary{
    decorations.markings,
    decorations.footprints,
    shapes.arrows,
    matrix,
    positioning,
}

\usepgfplotslibrary{
    fillbetween,
    ternary,
    smithchart,
    patchplots,
    polar,
    colormaps,
    colorbrewer,
    colortol,           % docu not ready yet
}
\pgfqkeys{/codeexample}{%
    every codeexample/.append style={
        /pgfplots/every ternary axis/.append style={
            /pgfplots/legend style={fill=graphicbackground},
        }
    },
    tabsize=4,
}

\pgfplotsmanualenableexternalizationofexpensive

%\usetikzlibrary{external}
%\tikzexternalize[prefix=figures/]{pgfplots}

\title{%
    Manual for Package \PGFPlots{}\\
    {\small 2D/3D Plots in \LaTeX{}, Version \pgfplotsversion}\\
    {\small\url{https://github.com/pgf-tikz/pgfplots}}
    %\\{\small Attention: you are using an unstable development version.}
}

\makeatletter
\long\def\abstractsmuggle{%
    \centering
    \textbf{Abstract}\\[0.5cm]

    \begin{minipage}{12cm}
        \PGFPlots{} draws high-quality function plots in normal or logarithmic
        scaling with a user-friendly interface directly in \TeX{}. The user
        supplies axis labels, legend entries and the plot coordinates for one or
        more plots and \PGFPlots{} applies axis scaling, computes any logarithms
        and axis ticks and draws the plots. It supports line plots, scatter
        plots, piecewise constant plots, bar plots, area plots, mesh and surface
        plots, patch plots, contour plots, quiver plots, histogram plots, box
        plots, polar axes, ternary diagrams, smith charts and some more. It is
        based on Till Tantau's package \PGF{}/\Tikz{}.
    \end{minipage}

}%

\expandafter\date\expandafter{\@date\\[2cm]
    \abstractsmuggle
}%
\makeatother

%\includeonly{pgfplots.reference}


\begin{document}

\def\plotcoords{%
\addplot coordinates {
(5,8.312e-02)    (17,2.547e-02)   (49,7.407e-03)
(129,2.102e-03)  (321,5.874e-04)  (769,1.623e-04)
(1793,4.442e-05) (4097,1.207e-05) (9217,3.261e-06)
};

\addplot coordinates{
(7,8.472e-02)    (31,3.044e-02)    (111,1.022e-02)
(351,3.303e-03)  (1023,1.039e-03)  (2815,3.196e-04)
(7423,9.658e-05) (18943,2.873e-05) (47103,8.437e-06)
};

\addplot coordinates{
(9,7.881e-02)     (49,3.243e-02)    (209,1.232e-02)
(769,4.454e-03)   (2561,1.551e-03)  (7937,5.236e-04)
(23297,1.723e-04) (65537,5.545e-05) (178177,1.751e-05)
};

\addplot coordinates{
(11,6.887e-02)    (71,3.177e-02)     (351,1.341e-02)
(1471,5.334e-03)  (5503,2.027e-03)   (18943,7.415e-04)
(61183,2.628e-04) (187903,9.063e-05) (553983,3.053e-05)
};

\addplot coordinates{
(13,5.755e-02)     (97,2.925e-02)     (545,1.351e-02)
(2561,5.842e-03)   (10625,2.397e-03)  (40193,9.414e-04)
(141569,3.564e-04) (471041,1.308e-04)
(1496065,4.670e-05)
};
}%


\bookmaketitle
\tableofcontents
\include{pgfplots.title_abstract_intro}
\include{pgfplots.preliminaries}
\include{pgfplots.intro}
\include{pgfplots.reference}
\include{pgfplots.libs}
\include{pgfplots.resources}
\include{pgfplots.importexport}
\include{pgfplots.basic.reference}

\printindex

\bibliographystyle{abbrv} %gerapali} %gerabbrv} %gerunsrt.bst} %gerabbrv}% gerplain}
\nocite{pgfplotstable}
\nocite{programmingnotes}
\bibliography{pgfplots}
\end{document}

% -----------------------------------------------------------------------------
% For Stefan Pinnow as reminder on what to look for when editing the manual
% -----------------------------------------------------------------------------
% There should be no line breaks in the following environments
% - |...|
% - \declareandlabel{...}
% - \verbpdfref{...}
% If "MakeTikzPictures" isn't running through check one of the externalized
% LOG files what is the cause of that.
% -----------------------------------------------------------------------------

\include{pgfplots.basic.reference}

\printindex

\bibliographystyle{abbrv} %gerapali} %gerabbrv} %gerunsrt.bst} %gerabbrv}% gerplain}
\nocite{pgfplotstable}
\nocite{programmingnotes}
\bibliography{pgfplots}
\end{document}

% -----------------------------------------------------------------------------
% For Stefan Pinnow as reminder on what to look for when editing the manual
% -----------------------------------------------------------------------------
% There should be no line breaks in the following environments
% - |...|
% - \declareandlabel{...}
% - \verbpdfref{...}
% If "MakeTikzPictures" isn't running through check one of the externalized
% LOG files what is the cause of that.
% -----------------------------------------------------------------------------


%%%%%%%%%%%%%%%%%%%%%%%%%%%%%%%%%%%%%%%%%%%%%%%%%%%%%%%%%%%%%%%%%%%%%%%%%%%%%
%
% Package pgfplots.sty documentation.
%
% Copyright 2007/2008 by Christian Feuersaenger.
%
% This program is free software: you can redistribute it and/or modify
% it under the terms of the GNU General Public License as published by
% the Free Software Foundation, either version 3 of the License, or
% (at your option) any later version.
%
% This program is distributed in the hope that it will be useful,
% but WITHOUT ANY WARRANTY; without even the implied warranty of
% MERCHANTABILITY or FITNESS FOR A PARTICULAR PURPOSE.  See the
% GNU General Public License for more details.
%
% You should have received a copy of the GNU General Public License
% along with this program.  If not, see <http://www.gnu.org/licenses/>.
%
%
%%%%%%%%%%%%%%%%%%%%%%%%%%%%%%%%%%%%%%%%%%%%%%%%%%%%%%%%%%%%%%%%%%%%%%%%%%%%%
% SEE pgfplots-macros.tex as well!
%\pdfminorversion=5 % to allow compression
%\pdfobjcompresslevel=2
\documentclass[a4paper,openany]{book}

\let\bookmaketitle=\maketitle

% -----------------------------------
% this here is from ltxdoc:
\usepackage{doc}[=v2]
\AtBeginDocument{\MakeShortVerb{\|}}
%\setlength{\textwidth}{355pt}
%\addtolength\marginparwidth{30pt}
%\addtolength\oddsidemargin{20pt}
%\addtolength\evensidemargin{20pt}
\def\cmd#1{\cs{\expandafter\cmd@to@cs\string#1}}
\def\cmd@to@cs#1#2{\char\number`#2\relax}
\DeclareRobustCommand\cs[1]{\texttt{\char`\\#1}}
\providecommand\marg[1]{%
  {\ttfamily\char`\{}\meta{#1}{\ttfamily\char`\}}}
\providecommand\oarg[1]{%
  {\ttfamily[}\meta{#1}{\ttfamily]}}
\providecommand\parg[1]{%
  {\ttfamily(}\meta{#1}{\ttfamily)}}
\raggedbottom

% -----------------------------------
\input{pgfplots.preamble.tex}

\makeatletter
% I want a two-column index just as in pgfmanual styles. This here
% was the best way to get one:
\def\index@prologue{\section*{Index}\addcontentsline{toc}{chapter}{Index}
}
\makeatother

%\RequirePackage[german,english,francais]{babel}

\def\matlabcolormaptext{This colormap is similar to one shipped with Matlab$^\text{\textregistered}$ under a similar name.}

\IfFileExists{tikzlibraryspy.code.tex}{%
\usetikzlibrary{spy}
}{%
    \message{ERROR: tikz SPY library NOT available. The manual will only compile partially.^^J}%
}%

\usepackage{xparse}% for colorbrewer manual

\usetikzlibrary{
    decorations.markings,
    decorations.footprints,
    shapes.arrows,
    matrix,
    positioning,
}

\usepgfplotslibrary{
    fillbetween,
    ternary,
    smithchart,
    patchplots,
    polar,
    colormaps,
    colorbrewer,
    colortol,           % docu not ready yet
}
\pgfqkeys{/codeexample}{%
    every codeexample/.append style={
        /pgfplots/every ternary axis/.append style={
            /pgfplots/legend style={fill=graphicbackground},
        }
    },
    tabsize=4,
}

\pgfplotsmanualenableexternalizationofexpensive

%\usetikzlibrary{external}
%\tikzexternalize[prefix=figures/]{pgfplots}

\title{%
    Manual for Package \PGFPlots{}\\
    {\small 2D/3D Plots in \LaTeX{}, Version \pgfplotsversion}\\
    {\small\url{https://github.com/pgf-tikz/pgfplots}}
    %\\{\small Attention: you are using an unstable development version.}
}

\makeatletter
\long\def\abstractsmuggle{%
    \centering
    \textbf{Abstract}\\[0.5cm]

    \begin{minipage}{12cm}
        \PGFPlots{} draws high-quality function plots in normal or logarithmic
        scaling with a user-friendly interface directly in \TeX{}. The user
        supplies axis labels, legend entries and the plot coordinates for one or
        more plots and \PGFPlots{} applies axis scaling, computes any logarithms
        and axis ticks and draws the plots. It supports line plots, scatter
        plots, piecewise constant plots, bar plots, area plots, mesh and surface
        plots, patch plots, contour plots, quiver plots, histogram plots, box
        plots, polar axes, ternary diagrams, smith charts and some more. It is
        based on Till Tantau's package \PGF{}/\Tikz{}.
    \end{minipage}

}%

\expandafter\date\expandafter{\@date\\[2cm]
    \abstractsmuggle
}%
\makeatother

%\includeonly{pgfplots.reference}


\begin{document}

\def\plotcoords{%
\addplot coordinates {
(5,8.312e-02)    (17,2.547e-02)   (49,7.407e-03)
(129,2.102e-03)  (321,5.874e-04)  (769,1.623e-04)
(1793,4.442e-05) (4097,1.207e-05) (9217,3.261e-06)
};

\addplot coordinates{
(7,8.472e-02)    (31,3.044e-02)    (111,1.022e-02)
(351,3.303e-03)  (1023,1.039e-03)  (2815,3.196e-04)
(7423,9.658e-05) (18943,2.873e-05) (47103,8.437e-06)
};

\addplot coordinates{
(9,7.881e-02)     (49,3.243e-02)    (209,1.232e-02)
(769,4.454e-03)   (2561,1.551e-03)  (7937,5.236e-04)
(23297,1.723e-04) (65537,5.545e-05) (178177,1.751e-05)
};

\addplot coordinates{
(11,6.887e-02)    (71,3.177e-02)     (351,1.341e-02)
(1471,5.334e-03)  (5503,2.027e-03)   (18943,7.415e-04)
(61183,2.628e-04) (187903,9.063e-05) (553983,3.053e-05)
};

\addplot coordinates{
(13,5.755e-02)     (97,2.925e-02)     (545,1.351e-02)
(2561,5.842e-03)   (10625,2.397e-03)  (40193,9.414e-04)
(141569,3.564e-04) (471041,1.308e-04)
(1496065,4.670e-05)
};
}%


\bookmaketitle
\tableofcontents

%%%%%%%%%%%%%%%%%%%%%%%%%%%%%%%%%%%%%%%%%%%%%%%%%%%%%%%%%%%%%%%%%%%%%%%%%%%%%
%
% Package pgfplots.sty documentation.
%
% Copyright 2007/2008 by Christian Feuersaenger.
%
% This program is free software: you can redistribute it and/or modify
% it under the terms of the GNU General Public License as published by
% the Free Software Foundation, either version 3 of the License, or
% (at your option) any later version.
%
% This program is distributed in the hope that it will be useful,
% but WITHOUT ANY WARRANTY; without even the implied warranty of
% MERCHANTABILITY or FITNESS FOR A PARTICULAR PURPOSE.  See the
% GNU General Public License for more details.
%
% You should have received a copy of the GNU General Public License
% along with this program.  If not, see <http://www.gnu.org/licenses/>.
%
%
%%%%%%%%%%%%%%%%%%%%%%%%%%%%%%%%%%%%%%%%%%%%%%%%%%%%%%%%%%%%%%%%%%%%%%%%%%%%%
% SEE pgfplots-macros.tex as well!
%\pdfminorversion=5 % to allow compression
%\pdfobjcompresslevel=2
\documentclass[a4paper,openany]{book}

\let\bookmaketitle=\maketitle

% -----------------------------------
% this here is from ltxdoc:
\usepackage{doc}[=v2]
\AtBeginDocument{\MakeShortVerb{\|}}
%\setlength{\textwidth}{355pt}
%\addtolength\marginparwidth{30pt}
%\addtolength\oddsidemargin{20pt}
%\addtolength\evensidemargin{20pt}
\def\cmd#1{\cs{\expandafter\cmd@to@cs\string#1}}
\def\cmd@to@cs#1#2{\char\number`#2\relax}
\DeclareRobustCommand\cs[1]{\texttt{\char`\\#1}}
\providecommand\marg[1]{%
  {\ttfamily\char`\{}\meta{#1}{\ttfamily\char`\}}}
\providecommand\oarg[1]{%
  {\ttfamily[}\meta{#1}{\ttfamily]}}
\providecommand\parg[1]{%
  {\ttfamily(}\meta{#1}{\ttfamily)}}
\raggedbottom

% -----------------------------------
\input{pgfplots.preamble.tex}

\makeatletter
% I want a two-column index just as in pgfmanual styles. This here
% was the best way to get one:
\def\index@prologue{\section*{Index}\addcontentsline{toc}{chapter}{Index}
}
\makeatother

%\RequirePackage[german,english,francais]{babel}

\def\matlabcolormaptext{This colormap is similar to one shipped with Matlab$^\text{\textregistered}$ under a similar name.}

\IfFileExists{tikzlibraryspy.code.tex}{%
\usetikzlibrary{spy}
}{%
    \message{ERROR: tikz SPY library NOT available. The manual will only compile partially.^^J}%
}%

\usepackage{xparse}% for colorbrewer manual

\usetikzlibrary{
    decorations.markings,
    decorations.footprints,
    shapes.arrows,
    matrix,
    positioning,
}

\usepgfplotslibrary{
    fillbetween,
    ternary,
    smithchart,
    patchplots,
    polar,
    colormaps,
    colorbrewer,
    colortol,           % docu not ready yet
}
\pgfqkeys{/codeexample}{%
    every codeexample/.append style={
        /pgfplots/every ternary axis/.append style={
            /pgfplots/legend style={fill=graphicbackground},
        }
    },
    tabsize=4,
}

\pgfplotsmanualenableexternalizationofexpensive

%\usetikzlibrary{external}
%\tikzexternalize[prefix=figures/]{pgfplots}

\title{%
    Manual for Package \PGFPlots{}\\
    {\small 2D/3D Plots in \LaTeX{}, Version \pgfplotsversion}\\
    {\small\url{https://github.com/pgf-tikz/pgfplots}}
    %\\{\small Attention: you are using an unstable development version.}
}

\makeatletter
\long\def\abstractsmuggle{%
    \centering
    \textbf{Abstract}\\[0.5cm]

    \begin{minipage}{12cm}
        \PGFPlots{} draws high-quality function plots in normal or logarithmic
        scaling with a user-friendly interface directly in \TeX{}. The user
        supplies axis labels, legend entries and the plot coordinates for one or
        more plots and \PGFPlots{} applies axis scaling, computes any logarithms
        and axis ticks and draws the plots. It supports line plots, scatter
        plots, piecewise constant plots, bar plots, area plots, mesh and surface
        plots, patch plots, contour plots, quiver plots, histogram plots, box
        plots, polar axes, ternary diagrams, smith charts and some more. It is
        based on Till Tantau's package \PGF{}/\Tikz{}.
    \end{minipage}

}%

\expandafter\date\expandafter{\@date\\[2cm]
    \abstractsmuggle
}%
\makeatother

%\includeonly{pgfplots.reference}


\begin{document}

\def\plotcoords{%
\addplot coordinates {
(5,8.312e-02)    (17,2.547e-02)   (49,7.407e-03)
(129,2.102e-03)  (321,5.874e-04)  (769,1.623e-04)
(1793,4.442e-05) (4097,1.207e-05) (9217,3.261e-06)
};

\addplot coordinates{
(7,8.472e-02)    (31,3.044e-02)    (111,1.022e-02)
(351,3.303e-03)  (1023,1.039e-03)  (2815,3.196e-04)
(7423,9.658e-05) (18943,2.873e-05) (47103,8.437e-06)
};

\addplot coordinates{
(9,7.881e-02)     (49,3.243e-02)    (209,1.232e-02)
(769,4.454e-03)   (2561,1.551e-03)  (7937,5.236e-04)
(23297,1.723e-04) (65537,5.545e-05) (178177,1.751e-05)
};

\addplot coordinates{
(11,6.887e-02)    (71,3.177e-02)     (351,1.341e-02)
(1471,5.334e-03)  (5503,2.027e-03)   (18943,7.415e-04)
(61183,2.628e-04) (187903,9.063e-05) (553983,3.053e-05)
};

\addplot coordinates{
(13,5.755e-02)     (97,2.925e-02)     (545,1.351e-02)
(2561,5.842e-03)   (10625,2.397e-03)  (40193,9.414e-04)
(141569,3.564e-04) (471041,1.308e-04)
(1496065,4.670e-05)
};
}%


\bookmaketitle
\tableofcontents
\include{pgfplots.title_abstract_intro}
\include{pgfplots.preliminaries}
\include{pgfplots.intro}
\include{pgfplots.reference}
\include{pgfplots.libs}
\include{pgfplots.resources}
\include{pgfplots.importexport}
\include{pgfplots.basic.reference}

\printindex

\bibliographystyle{abbrv} %gerapali} %gerabbrv} %gerunsrt.bst} %gerabbrv}% gerplain}
\nocite{pgfplotstable}
\nocite{programmingnotes}
\bibliography{pgfplots}
\end{document}

% -----------------------------------------------------------------------------
% For Stefan Pinnow as reminder on what to look for when editing the manual
% -----------------------------------------------------------------------------
% There should be no line breaks in the following environments
% - |...|
% - \declareandlabel{...}
% - \verbpdfref{...}
% If "MakeTikzPictures" isn't running through check one of the externalized
% LOG files what is the cause of that.
% -----------------------------------------------------------------------------


%%%%%%%%%%%%%%%%%%%%%%%%%%%%%%%%%%%%%%%%%%%%%%%%%%%%%%%%%%%%%%%%%%%%%%%%%%%%%
%
% Package pgfplots.sty documentation.
%
% Copyright 2007/2008 by Christian Feuersaenger.
%
% This program is free software: you can redistribute it and/or modify
% it under the terms of the GNU General Public License as published by
% the Free Software Foundation, either version 3 of the License, or
% (at your option) any later version.
%
% This program is distributed in the hope that it will be useful,
% but WITHOUT ANY WARRANTY; without even the implied warranty of
% MERCHANTABILITY or FITNESS FOR A PARTICULAR PURPOSE.  See the
% GNU General Public License for more details.
%
% You should have received a copy of the GNU General Public License
% along with this program.  If not, see <http://www.gnu.org/licenses/>.
%
%
%%%%%%%%%%%%%%%%%%%%%%%%%%%%%%%%%%%%%%%%%%%%%%%%%%%%%%%%%%%%%%%%%%%%%%%%%%%%%
% SEE pgfplots-macros.tex as well!
%\pdfminorversion=5 % to allow compression
%\pdfobjcompresslevel=2
\documentclass[a4paper,openany]{book}

\let\bookmaketitle=\maketitle

% -----------------------------------
% this here is from ltxdoc:
\usepackage{doc}[=v2]
\AtBeginDocument{\MakeShortVerb{\|}}
%\setlength{\textwidth}{355pt}
%\addtolength\marginparwidth{30pt}
%\addtolength\oddsidemargin{20pt}
%\addtolength\evensidemargin{20pt}
\def\cmd#1{\cs{\expandafter\cmd@to@cs\string#1}}
\def\cmd@to@cs#1#2{\char\number`#2\relax}
\DeclareRobustCommand\cs[1]{\texttt{\char`\\#1}}
\providecommand\marg[1]{%
  {\ttfamily\char`\{}\meta{#1}{\ttfamily\char`\}}}
\providecommand\oarg[1]{%
  {\ttfamily[}\meta{#1}{\ttfamily]}}
\providecommand\parg[1]{%
  {\ttfamily(}\meta{#1}{\ttfamily)}}
\raggedbottom

% -----------------------------------
\input{pgfplots.preamble.tex}

\makeatletter
% I want a two-column index just as in pgfmanual styles. This here
% was the best way to get one:
\def\index@prologue{\section*{Index}\addcontentsline{toc}{chapter}{Index}
}
\makeatother

%\RequirePackage[german,english,francais]{babel}

\def\matlabcolormaptext{This colormap is similar to one shipped with Matlab$^\text{\textregistered}$ under a similar name.}

\IfFileExists{tikzlibraryspy.code.tex}{%
\usetikzlibrary{spy}
}{%
    \message{ERROR: tikz SPY library NOT available. The manual will only compile partially.^^J}%
}%

\usepackage{xparse}% for colorbrewer manual

\usetikzlibrary{
    decorations.markings,
    decorations.footprints,
    shapes.arrows,
    matrix,
    positioning,
}

\usepgfplotslibrary{
    fillbetween,
    ternary,
    smithchart,
    patchplots,
    polar,
    colormaps,
    colorbrewer,
    colortol,           % docu not ready yet
}
\pgfqkeys{/codeexample}{%
    every codeexample/.append style={
        /pgfplots/every ternary axis/.append style={
            /pgfplots/legend style={fill=graphicbackground},
        }
    },
    tabsize=4,
}

\pgfplotsmanualenableexternalizationofexpensive

%\usetikzlibrary{external}
%\tikzexternalize[prefix=figures/]{pgfplots}

\title{%
    Manual for Package \PGFPlots{}\\
    {\small 2D/3D Plots in \LaTeX{}, Version \pgfplotsversion}\\
    {\small\url{https://github.com/pgf-tikz/pgfplots}}
    %\\{\small Attention: you are using an unstable development version.}
}

\makeatletter
\long\def\abstractsmuggle{%
    \centering
    \textbf{Abstract}\\[0.5cm]

    \begin{minipage}{12cm}
        \PGFPlots{} draws high-quality function plots in normal or logarithmic
        scaling with a user-friendly interface directly in \TeX{}. The user
        supplies axis labels, legend entries and the plot coordinates for one or
        more plots and \PGFPlots{} applies axis scaling, computes any logarithms
        and axis ticks and draws the plots. It supports line plots, scatter
        plots, piecewise constant plots, bar plots, area plots, mesh and surface
        plots, patch plots, contour plots, quiver plots, histogram plots, box
        plots, polar axes, ternary diagrams, smith charts and some more. It is
        based on Till Tantau's package \PGF{}/\Tikz{}.
    \end{minipage}

}%

\expandafter\date\expandafter{\@date\\[2cm]
    \abstractsmuggle
}%
\makeatother

%\includeonly{pgfplots.reference}


\begin{document}

\def\plotcoords{%
\addplot coordinates {
(5,8.312e-02)    (17,2.547e-02)   (49,7.407e-03)
(129,2.102e-03)  (321,5.874e-04)  (769,1.623e-04)
(1793,4.442e-05) (4097,1.207e-05) (9217,3.261e-06)
};

\addplot coordinates{
(7,8.472e-02)    (31,3.044e-02)    (111,1.022e-02)
(351,3.303e-03)  (1023,1.039e-03)  (2815,3.196e-04)
(7423,9.658e-05) (18943,2.873e-05) (47103,8.437e-06)
};

\addplot coordinates{
(9,7.881e-02)     (49,3.243e-02)    (209,1.232e-02)
(769,4.454e-03)   (2561,1.551e-03)  (7937,5.236e-04)
(23297,1.723e-04) (65537,5.545e-05) (178177,1.751e-05)
};

\addplot coordinates{
(11,6.887e-02)    (71,3.177e-02)     (351,1.341e-02)
(1471,5.334e-03)  (5503,2.027e-03)   (18943,7.415e-04)
(61183,2.628e-04) (187903,9.063e-05) (553983,3.053e-05)
};

\addplot coordinates{
(13,5.755e-02)     (97,2.925e-02)     (545,1.351e-02)
(2561,5.842e-03)   (10625,2.397e-03)  (40193,9.414e-04)
(141569,3.564e-04) (471041,1.308e-04)
(1496065,4.670e-05)
};
}%


\bookmaketitle
\tableofcontents
\include{pgfplots.title_abstract_intro}
\include{pgfplots.preliminaries}
\include{pgfplots.intro}
\include{pgfplots.reference}
\include{pgfplots.libs}
\include{pgfplots.resources}
\include{pgfplots.importexport}
\include{pgfplots.basic.reference}

\printindex

\bibliographystyle{abbrv} %gerapali} %gerabbrv} %gerunsrt.bst} %gerabbrv}% gerplain}
\nocite{pgfplotstable}
\nocite{programmingnotes}
\bibliography{pgfplots}
\end{document}

% -----------------------------------------------------------------------------
% For Stefan Pinnow as reminder on what to look for when editing the manual
% -----------------------------------------------------------------------------
% There should be no line breaks in the following environments
% - |...|
% - \declareandlabel{...}
% - \verbpdfref{...}
% If "MakeTikzPictures" isn't running through check one of the externalized
% LOG files what is the cause of that.
% -----------------------------------------------------------------------------


%%%%%%%%%%%%%%%%%%%%%%%%%%%%%%%%%%%%%%%%%%%%%%%%%%%%%%%%%%%%%%%%%%%%%%%%%%%%%
%
% Package pgfplots.sty documentation.
%
% Copyright 2007/2008 by Christian Feuersaenger.
%
% This program is free software: you can redistribute it and/or modify
% it under the terms of the GNU General Public License as published by
% the Free Software Foundation, either version 3 of the License, or
% (at your option) any later version.
%
% This program is distributed in the hope that it will be useful,
% but WITHOUT ANY WARRANTY; without even the implied warranty of
% MERCHANTABILITY or FITNESS FOR A PARTICULAR PURPOSE.  See the
% GNU General Public License for more details.
%
% You should have received a copy of the GNU General Public License
% along with this program.  If not, see <http://www.gnu.org/licenses/>.
%
%
%%%%%%%%%%%%%%%%%%%%%%%%%%%%%%%%%%%%%%%%%%%%%%%%%%%%%%%%%%%%%%%%%%%%%%%%%%%%%
% SEE pgfplots-macros.tex as well!
%\pdfminorversion=5 % to allow compression
%\pdfobjcompresslevel=2
\documentclass[a4paper,openany]{book}

\let\bookmaketitle=\maketitle

% -----------------------------------
% this here is from ltxdoc:
\usepackage{doc}[=v2]
\AtBeginDocument{\MakeShortVerb{\|}}
%\setlength{\textwidth}{355pt}
%\addtolength\marginparwidth{30pt}
%\addtolength\oddsidemargin{20pt}
%\addtolength\evensidemargin{20pt}
\def\cmd#1{\cs{\expandafter\cmd@to@cs\string#1}}
\def\cmd@to@cs#1#2{\char\number`#2\relax}
\DeclareRobustCommand\cs[1]{\texttt{\char`\\#1}}
\providecommand\marg[1]{%
  {\ttfamily\char`\{}\meta{#1}{\ttfamily\char`\}}}
\providecommand\oarg[1]{%
  {\ttfamily[}\meta{#1}{\ttfamily]}}
\providecommand\parg[1]{%
  {\ttfamily(}\meta{#1}{\ttfamily)}}
\raggedbottom

% -----------------------------------
\input{pgfplots.preamble.tex}

\makeatletter
% I want a two-column index just as in pgfmanual styles. This here
% was the best way to get one:
\def\index@prologue{\section*{Index}\addcontentsline{toc}{chapter}{Index}
}
\makeatother

%\RequirePackage[german,english,francais]{babel}

\def\matlabcolormaptext{This colormap is similar to one shipped with Matlab$^\text{\textregistered}$ under a similar name.}

\IfFileExists{tikzlibraryspy.code.tex}{%
\usetikzlibrary{spy}
}{%
    \message{ERROR: tikz SPY library NOT available. The manual will only compile partially.^^J}%
}%

\usepackage{xparse}% for colorbrewer manual

\usetikzlibrary{
    decorations.markings,
    decorations.footprints,
    shapes.arrows,
    matrix,
    positioning,
}

\usepgfplotslibrary{
    fillbetween,
    ternary,
    smithchart,
    patchplots,
    polar,
    colormaps,
    colorbrewer,
    colortol,           % docu not ready yet
}
\pgfqkeys{/codeexample}{%
    every codeexample/.append style={
        /pgfplots/every ternary axis/.append style={
            /pgfplots/legend style={fill=graphicbackground},
        }
    },
    tabsize=4,
}

\pgfplotsmanualenableexternalizationofexpensive

%\usetikzlibrary{external}
%\tikzexternalize[prefix=figures/]{pgfplots}

\title{%
    Manual for Package \PGFPlots{}\\
    {\small 2D/3D Plots in \LaTeX{}, Version \pgfplotsversion}\\
    {\small\url{https://github.com/pgf-tikz/pgfplots}}
    %\\{\small Attention: you are using an unstable development version.}
}

\makeatletter
\long\def\abstractsmuggle{%
    \centering
    \textbf{Abstract}\\[0.5cm]

    \begin{minipage}{12cm}
        \PGFPlots{} draws high-quality function plots in normal or logarithmic
        scaling with a user-friendly interface directly in \TeX{}. The user
        supplies axis labels, legend entries and the plot coordinates for one or
        more plots and \PGFPlots{} applies axis scaling, computes any logarithms
        and axis ticks and draws the plots. It supports line plots, scatter
        plots, piecewise constant plots, bar plots, area plots, mesh and surface
        plots, patch plots, contour plots, quiver plots, histogram plots, box
        plots, polar axes, ternary diagrams, smith charts and some more. It is
        based on Till Tantau's package \PGF{}/\Tikz{}.
    \end{minipage}

}%

\expandafter\date\expandafter{\@date\\[2cm]
    \abstractsmuggle
}%
\makeatother

%\includeonly{pgfplots.reference}


\begin{document}

\def\plotcoords{%
\addplot coordinates {
(5,8.312e-02)    (17,2.547e-02)   (49,7.407e-03)
(129,2.102e-03)  (321,5.874e-04)  (769,1.623e-04)
(1793,4.442e-05) (4097,1.207e-05) (9217,3.261e-06)
};

\addplot coordinates{
(7,8.472e-02)    (31,3.044e-02)    (111,1.022e-02)
(351,3.303e-03)  (1023,1.039e-03)  (2815,3.196e-04)
(7423,9.658e-05) (18943,2.873e-05) (47103,8.437e-06)
};

\addplot coordinates{
(9,7.881e-02)     (49,3.243e-02)    (209,1.232e-02)
(769,4.454e-03)   (2561,1.551e-03)  (7937,5.236e-04)
(23297,1.723e-04) (65537,5.545e-05) (178177,1.751e-05)
};

\addplot coordinates{
(11,6.887e-02)    (71,3.177e-02)     (351,1.341e-02)
(1471,5.334e-03)  (5503,2.027e-03)   (18943,7.415e-04)
(61183,2.628e-04) (187903,9.063e-05) (553983,3.053e-05)
};

\addplot coordinates{
(13,5.755e-02)     (97,2.925e-02)     (545,1.351e-02)
(2561,5.842e-03)   (10625,2.397e-03)  (40193,9.414e-04)
(141569,3.564e-04) (471041,1.308e-04)
(1496065,4.670e-05)
};
}%


\bookmaketitle
\tableofcontents
\include{pgfplots.title_abstract_intro}
\include{pgfplots.preliminaries}
\include{pgfplots.intro}
\include{pgfplots.reference}
\include{pgfplots.libs}
\include{pgfplots.resources}
\include{pgfplots.importexport}
\include{pgfplots.basic.reference}

\printindex

\bibliographystyle{abbrv} %gerapali} %gerabbrv} %gerunsrt.bst} %gerabbrv}% gerplain}
\nocite{pgfplotstable}
\nocite{programmingnotes}
\bibliography{pgfplots}
\end{document}

% -----------------------------------------------------------------------------
% For Stefan Pinnow as reminder on what to look for when editing the manual
% -----------------------------------------------------------------------------
% There should be no line breaks in the following environments
% - |...|
% - \declareandlabel{...}
% - \verbpdfref{...}
% If "MakeTikzPictures" isn't running through check one of the externalized
% LOG files what is the cause of that.
% -----------------------------------------------------------------------------


%%%%%%%%%%%%%%%%%%%%%%%%%%%%%%%%%%%%%%%%%%%%%%%%%%%%%%%%%%%%%%%%%%%%%%%%%%%%%
%
% Package pgfplots.sty documentation.
%
% Copyright 2007/2008 by Christian Feuersaenger.
%
% This program is free software: you can redistribute it and/or modify
% it under the terms of the GNU General Public License as published by
% the Free Software Foundation, either version 3 of the License, or
% (at your option) any later version.
%
% This program is distributed in the hope that it will be useful,
% but WITHOUT ANY WARRANTY; without even the implied warranty of
% MERCHANTABILITY or FITNESS FOR A PARTICULAR PURPOSE.  See the
% GNU General Public License for more details.
%
% You should have received a copy of the GNU General Public License
% along with this program.  If not, see <http://www.gnu.org/licenses/>.
%
%
%%%%%%%%%%%%%%%%%%%%%%%%%%%%%%%%%%%%%%%%%%%%%%%%%%%%%%%%%%%%%%%%%%%%%%%%%%%%%
% SEE pgfplots-macros.tex as well!
%\pdfminorversion=5 % to allow compression
%\pdfobjcompresslevel=2
\documentclass[a4paper,openany]{book}

\let\bookmaketitle=\maketitle

% -----------------------------------
% this here is from ltxdoc:
\usepackage{doc}[=v2]
\AtBeginDocument{\MakeShortVerb{\|}}
%\setlength{\textwidth}{355pt}
%\addtolength\marginparwidth{30pt}
%\addtolength\oddsidemargin{20pt}
%\addtolength\evensidemargin{20pt}
\def\cmd#1{\cs{\expandafter\cmd@to@cs\string#1}}
\def\cmd@to@cs#1#2{\char\number`#2\relax}
\DeclareRobustCommand\cs[1]{\texttt{\char`\\#1}}
\providecommand\marg[1]{%
  {\ttfamily\char`\{}\meta{#1}{\ttfamily\char`\}}}
\providecommand\oarg[1]{%
  {\ttfamily[}\meta{#1}{\ttfamily]}}
\providecommand\parg[1]{%
  {\ttfamily(}\meta{#1}{\ttfamily)}}
\raggedbottom

% -----------------------------------
\input{pgfplots.preamble.tex}

\makeatletter
% I want a two-column index just as in pgfmanual styles. This here
% was the best way to get one:
\def\index@prologue{\section*{Index}\addcontentsline{toc}{chapter}{Index}
}
\makeatother

%\RequirePackage[german,english,francais]{babel}

\def\matlabcolormaptext{This colormap is similar to one shipped with Matlab$^\text{\textregistered}$ under a similar name.}

\IfFileExists{tikzlibraryspy.code.tex}{%
\usetikzlibrary{spy}
}{%
    \message{ERROR: tikz SPY library NOT available. The manual will only compile partially.^^J}%
}%

\usepackage{xparse}% for colorbrewer manual

\usetikzlibrary{
    decorations.markings,
    decorations.footprints,
    shapes.arrows,
    matrix,
    positioning,
}

\usepgfplotslibrary{
    fillbetween,
    ternary,
    smithchart,
    patchplots,
    polar,
    colormaps,
    colorbrewer,
    colortol,           % docu not ready yet
}
\pgfqkeys{/codeexample}{%
    every codeexample/.append style={
        /pgfplots/every ternary axis/.append style={
            /pgfplots/legend style={fill=graphicbackground},
        }
    },
    tabsize=4,
}

\pgfplotsmanualenableexternalizationofexpensive

%\usetikzlibrary{external}
%\tikzexternalize[prefix=figures/]{pgfplots}

\title{%
    Manual for Package \PGFPlots{}\\
    {\small 2D/3D Plots in \LaTeX{}, Version \pgfplotsversion}\\
    {\small\url{https://github.com/pgf-tikz/pgfplots}}
    %\\{\small Attention: you are using an unstable development version.}
}

\makeatletter
\long\def\abstractsmuggle{%
    \centering
    \textbf{Abstract}\\[0.5cm]

    \begin{minipage}{12cm}
        \PGFPlots{} draws high-quality function plots in normal or logarithmic
        scaling with a user-friendly interface directly in \TeX{}. The user
        supplies axis labels, legend entries and the plot coordinates for one or
        more plots and \PGFPlots{} applies axis scaling, computes any logarithms
        and axis ticks and draws the plots. It supports line plots, scatter
        plots, piecewise constant plots, bar plots, area plots, mesh and surface
        plots, patch plots, contour plots, quiver plots, histogram plots, box
        plots, polar axes, ternary diagrams, smith charts and some more. It is
        based on Till Tantau's package \PGF{}/\Tikz{}.
    \end{minipage}

}%

\expandafter\date\expandafter{\@date\\[2cm]
    \abstractsmuggle
}%
\makeatother

%\includeonly{pgfplots.reference}


\begin{document}

\def\plotcoords{%
\addplot coordinates {
(5,8.312e-02)    (17,2.547e-02)   (49,7.407e-03)
(129,2.102e-03)  (321,5.874e-04)  (769,1.623e-04)
(1793,4.442e-05) (4097,1.207e-05) (9217,3.261e-06)
};

\addplot coordinates{
(7,8.472e-02)    (31,3.044e-02)    (111,1.022e-02)
(351,3.303e-03)  (1023,1.039e-03)  (2815,3.196e-04)
(7423,9.658e-05) (18943,2.873e-05) (47103,8.437e-06)
};

\addplot coordinates{
(9,7.881e-02)     (49,3.243e-02)    (209,1.232e-02)
(769,4.454e-03)   (2561,1.551e-03)  (7937,5.236e-04)
(23297,1.723e-04) (65537,5.545e-05) (178177,1.751e-05)
};

\addplot coordinates{
(11,6.887e-02)    (71,3.177e-02)     (351,1.341e-02)
(1471,5.334e-03)  (5503,2.027e-03)   (18943,7.415e-04)
(61183,2.628e-04) (187903,9.063e-05) (553983,3.053e-05)
};

\addplot coordinates{
(13,5.755e-02)     (97,2.925e-02)     (545,1.351e-02)
(2561,5.842e-03)   (10625,2.397e-03)  (40193,9.414e-04)
(141569,3.564e-04) (471041,1.308e-04)
(1496065,4.670e-05)
};
}%


\bookmaketitle
\tableofcontents
\include{pgfplots.title_abstract_intro}
\include{pgfplots.preliminaries}
\include{pgfplots.intro}
\include{pgfplots.reference}
\include{pgfplots.libs}
\include{pgfplots.resources}
\include{pgfplots.importexport}
\include{pgfplots.basic.reference}

\printindex

\bibliographystyle{abbrv} %gerapali} %gerabbrv} %gerunsrt.bst} %gerabbrv}% gerplain}
\nocite{pgfplotstable}
\nocite{programmingnotes}
\bibliography{pgfplots}
\end{document}

% -----------------------------------------------------------------------------
% For Stefan Pinnow as reminder on what to look for when editing the manual
% -----------------------------------------------------------------------------
% There should be no line breaks in the following environments
% - |...|
% - \declareandlabel{...}
% - \verbpdfref{...}
% If "MakeTikzPictures" isn't running through check one of the externalized
% LOG files what is the cause of that.
% -----------------------------------------------------------------------------


%%%%%%%%%%%%%%%%%%%%%%%%%%%%%%%%%%%%%%%%%%%%%%%%%%%%%%%%%%%%%%%%%%%%%%%%%%%%%
%
% Package pgfplots.sty documentation.
%
% Copyright 2007/2008 by Christian Feuersaenger.
%
% This program is free software: you can redistribute it and/or modify
% it under the terms of the GNU General Public License as published by
% the Free Software Foundation, either version 3 of the License, or
% (at your option) any later version.
%
% This program is distributed in the hope that it will be useful,
% but WITHOUT ANY WARRANTY; without even the implied warranty of
% MERCHANTABILITY or FITNESS FOR A PARTICULAR PURPOSE.  See the
% GNU General Public License for more details.
%
% You should have received a copy of the GNU General Public License
% along with this program.  If not, see <http://www.gnu.org/licenses/>.
%
%
%%%%%%%%%%%%%%%%%%%%%%%%%%%%%%%%%%%%%%%%%%%%%%%%%%%%%%%%%%%%%%%%%%%%%%%%%%%%%
% SEE pgfplots-macros.tex as well!
%\pdfminorversion=5 % to allow compression
%\pdfobjcompresslevel=2
\documentclass[a4paper,openany]{book}

\let\bookmaketitle=\maketitle

% -----------------------------------
% this here is from ltxdoc:
\usepackage{doc}[=v2]
\AtBeginDocument{\MakeShortVerb{\|}}
%\setlength{\textwidth}{355pt}
%\addtolength\marginparwidth{30pt}
%\addtolength\oddsidemargin{20pt}
%\addtolength\evensidemargin{20pt}
\def\cmd#1{\cs{\expandafter\cmd@to@cs\string#1}}
\def\cmd@to@cs#1#2{\char\number`#2\relax}
\DeclareRobustCommand\cs[1]{\texttt{\char`\\#1}}
\providecommand\marg[1]{%
  {\ttfamily\char`\{}\meta{#1}{\ttfamily\char`\}}}
\providecommand\oarg[1]{%
  {\ttfamily[}\meta{#1}{\ttfamily]}}
\providecommand\parg[1]{%
  {\ttfamily(}\meta{#1}{\ttfamily)}}
\raggedbottom

% -----------------------------------
\input{pgfplots.preamble.tex}

\makeatletter
% I want a two-column index just as in pgfmanual styles. This here
% was the best way to get one:
\def\index@prologue{\section*{Index}\addcontentsline{toc}{chapter}{Index}
}
\makeatother

%\RequirePackage[german,english,francais]{babel}

\def\matlabcolormaptext{This colormap is similar to one shipped with Matlab$^\text{\textregistered}$ under a similar name.}

\IfFileExists{tikzlibraryspy.code.tex}{%
\usetikzlibrary{spy}
}{%
    \message{ERROR: tikz SPY library NOT available. The manual will only compile partially.^^J}%
}%

\usepackage{xparse}% for colorbrewer manual

\usetikzlibrary{
    decorations.markings,
    decorations.footprints,
    shapes.arrows,
    matrix,
    positioning,
}

\usepgfplotslibrary{
    fillbetween,
    ternary,
    smithchart,
    patchplots,
    polar,
    colormaps,
    colorbrewer,
    colortol,           % docu not ready yet
}
\pgfqkeys{/codeexample}{%
    every codeexample/.append style={
        /pgfplots/every ternary axis/.append style={
            /pgfplots/legend style={fill=graphicbackground},
        }
    },
    tabsize=4,
}

\pgfplotsmanualenableexternalizationofexpensive

%\usetikzlibrary{external}
%\tikzexternalize[prefix=figures/]{pgfplots}

\title{%
    Manual for Package \PGFPlots{}\\
    {\small 2D/3D Plots in \LaTeX{}, Version \pgfplotsversion}\\
    {\small\url{https://github.com/pgf-tikz/pgfplots}}
    %\\{\small Attention: you are using an unstable development version.}
}

\makeatletter
\long\def\abstractsmuggle{%
    \centering
    \textbf{Abstract}\\[0.5cm]

    \begin{minipage}{12cm}
        \PGFPlots{} draws high-quality function plots in normal or logarithmic
        scaling with a user-friendly interface directly in \TeX{}. The user
        supplies axis labels, legend entries and the plot coordinates for one or
        more plots and \PGFPlots{} applies axis scaling, computes any logarithms
        and axis ticks and draws the plots. It supports line plots, scatter
        plots, piecewise constant plots, bar plots, area plots, mesh and surface
        plots, patch plots, contour plots, quiver plots, histogram plots, box
        plots, polar axes, ternary diagrams, smith charts and some more. It is
        based on Till Tantau's package \PGF{}/\Tikz{}.
    \end{minipage}

}%

\expandafter\date\expandafter{\@date\\[2cm]
    \abstractsmuggle
}%
\makeatother

%\includeonly{pgfplots.reference}


\begin{document}

\def\plotcoords{%
\addplot coordinates {
(5,8.312e-02)    (17,2.547e-02)   (49,7.407e-03)
(129,2.102e-03)  (321,5.874e-04)  (769,1.623e-04)
(1793,4.442e-05) (4097,1.207e-05) (9217,3.261e-06)
};

\addplot coordinates{
(7,8.472e-02)    (31,3.044e-02)    (111,1.022e-02)
(351,3.303e-03)  (1023,1.039e-03)  (2815,3.196e-04)
(7423,9.658e-05) (18943,2.873e-05) (47103,8.437e-06)
};

\addplot coordinates{
(9,7.881e-02)     (49,3.243e-02)    (209,1.232e-02)
(769,4.454e-03)   (2561,1.551e-03)  (7937,5.236e-04)
(23297,1.723e-04) (65537,5.545e-05) (178177,1.751e-05)
};

\addplot coordinates{
(11,6.887e-02)    (71,3.177e-02)     (351,1.341e-02)
(1471,5.334e-03)  (5503,2.027e-03)   (18943,7.415e-04)
(61183,2.628e-04) (187903,9.063e-05) (553983,3.053e-05)
};

\addplot coordinates{
(13,5.755e-02)     (97,2.925e-02)     (545,1.351e-02)
(2561,5.842e-03)   (10625,2.397e-03)  (40193,9.414e-04)
(141569,3.564e-04) (471041,1.308e-04)
(1496065,4.670e-05)
};
}%


\bookmaketitle
\tableofcontents
\include{pgfplots.title_abstract_intro}
\include{pgfplots.preliminaries}
\include{pgfplots.intro}
\include{pgfplots.reference}
\include{pgfplots.libs}
\include{pgfplots.resources}
\include{pgfplots.importexport}
\include{pgfplots.basic.reference}

\printindex

\bibliographystyle{abbrv} %gerapali} %gerabbrv} %gerunsrt.bst} %gerabbrv}% gerplain}
\nocite{pgfplotstable}
\nocite{programmingnotes}
\bibliography{pgfplots}
\end{document}

% -----------------------------------------------------------------------------
% For Stefan Pinnow as reminder on what to look for when editing the manual
% -----------------------------------------------------------------------------
% There should be no line breaks in the following environments
% - |...|
% - \declareandlabel{...}
% - \verbpdfref{...}
% If "MakeTikzPictures" isn't running through check one of the externalized
% LOG files what is the cause of that.
% -----------------------------------------------------------------------------


%%%%%%%%%%%%%%%%%%%%%%%%%%%%%%%%%%%%%%%%%%%%%%%%%%%%%%%%%%%%%%%%%%%%%%%%%%%%%
%
% Package pgfplots.sty documentation.
%
% Copyright 2007/2008 by Christian Feuersaenger.
%
% This program is free software: you can redistribute it and/or modify
% it under the terms of the GNU General Public License as published by
% the Free Software Foundation, either version 3 of the License, or
% (at your option) any later version.
%
% This program is distributed in the hope that it will be useful,
% but WITHOUT ANY WARRANTY; without even the implied warranty of
% MERCHANTABILITY or FITNESS FOR A PARTICULAR PURPOSE.  See the
% GNU General Public License for more details.
%
% You should have received a copy of the GNU General Public License
% along with this program.  If not, see <http://www.gnu.org/licenses/>.
%
%
%%%%%%%%%%%%%%%%%%%%%%%%%%%%%%%%%%%%%%%%%%%%%%%%%%%%%%%%%%%%%%%%%%%%%%%%%%%%%
% SEE pgfplots-macros.tex as well!
%\pdfminorversion=5 % to allow compression
%\pdfobjcompresslevel=2
\documentclass[a4paper,openany]{book}

\let\bookmaketitle=\maketitle

% -----------------------------------
% this here is from ltxdoc:
\usepackage{doc}[=v2]
\AtBeginDocument{\MakeShortVerb{\|}}
%\setlength{\textwidth}{355pt}
%\addtolength\marginparwidth{30pt}
%\addtolength\oddsidemargin{20pt}
%\addtolength\evensidemargin{20pt}
\def\cmd#1{\cs{\expandafter\cmd@to@cs\string#1}}
\def\cmd@to@cs#1#2{\char\number`#2\relax}
\DeclareRobustCommand\cs[1]{\texttt{\char`\\#1}}
\providecommand\marg[1]{%
  {\ttfamily\char`\{}\meta{#1}{\ttfamily\char`\}}}
\providecommand\oarg[1]{%
  {\ttfamily[}\meta{#1}{\ttfamily]}}
\providecommand\parg[1]{%
  {\ttfamily(}\meta{#1}{\ttfamily)}}
\raggedbottom

% -----------------------------------
\input{pgfplots.preamble.tex}

\makeatletter
% I want a two-column index just as in pgfmanual styles. This here
% was the best way to get one:
\def\index@prologue{\section*{Index}\addcontentsline{toc}{chapter}{Index}
}
\makeatother

%\RequirePackage[german,english,francais]{babel}

\def\matlabcolormaptext{This colormap is similar to one shipped with Matlab$^\text{\textregistered}$ under a similar name.}

\IfFileExists{tikzlibraryspy.code.tex}{%
\usetikzlibrary{spy}
}{%
    \message{ERROR: tikz SPY library NOT available. The manual will only compile partially.^^J}%
}%

\usepackage{xparse}% for colorbrewer manual

\usetikzlibrary{
    decorations.markings,
    decorations.footprints,
    shapes.arrows,
    matrix,
    positioning,
}

\usepgfplotslibrary{
    fillbetween,
    ternary,
    smithchart,
    patchplots,
    polar,
    colormaps,
    colorbrewer,
    colortol,           % docu not ready yet
}
\pgfqkeys{/codeexample}{%
    every codeexample/.append style={
        /pgfplots/every ternary axis/.append style={
            /pgfplots/legend style={fill=graphicbackground},
        }
    },
    tabsize=4,
}

\pgfplotsmanualenableexternalizationofexpensive

%\usetikzlibrary{external}
%\tikzexternalize[prefix=figures/]{pgfplots}

\title{%
    Manual for Package \PGFPlots{}\\
    {\small 2D/3D Plots in \LaTeX{}, Version \pgfplotsversion}\\
    {\small\url{https://github.com/pgf-tikz/pgfplots}}
    %\\{\small Attention: you are using an unstable development version.}
}

\makeatletter
\long\def\abstractsmuggle{%
    \centering
    \textbf{Abstract}\\[0.5cm]

    \begin{minipage}{12cm}
        \PGFPlots{} draws high-quality function plots in normal or logarithmic
        scaling with a user-friendly interface directly in \TeX{}. The user
        supplies axis labels, legend entries and the plot coordinates for one or
        more plots and \PGFPlots{} applies axis scaling, computes any logarithms
        and axis ticks and draws the plots. It supports line plots, scatter
        plots, piecewise constant plots, bar plots, area plots, mesh and surface
        plots, patch plots, contour plots, quiver plots, histogram plots, box
        plots, polar axes, ternary diagrams, smith charts and some more. It is
        based on Till Tantau's package \PGF{}/\Tikz{}.
    \end{minipage}

}%

\expandafter\date\expandafter{\@date\\[2cm]
    \abstractsmuggle
}%
\makeatother

%\includeonly{pgfplots.reference}


\begin{document}

\def\plotcoords{%
\addplot coordinates {
(5,8.312e-02)    (17,2.547e-02)   (49,7.407e-03)
(129,2.102e-03)  (321,5.874e-04)  (769,1.623e-04)
(1793,4.442e-05) (4097,1.207e-05) (9217,3.261e-06)
};

\addplot coordinates{
(7,8.472e-02)    (31,3.044e-02)    (111,1.022e-02)
(351,3.303e-03)  (1023,1.039e-03)  (2815,3.196e-04)
(7423,9.658e-05) (18943,2.873e-05) (47103,8.437e-06)
};

\addplot coordinates{
(9,7.881e-02)     (49,3.243e-02)    (209,1.232e-02)
(769,4.454e-03)   (2561,1.551e-03)  (7937,5.236e-04)
(23297,1.723e-04) (65537,5.545e-05) (178177,1.751e-05)
};

\addplot coordinates{
(11,6.887e-02)    (71,3.177e-02)     (351,1.341e-02)
(1471,5.334e-03)  (5503,2.027e-03)   (18943,7.415e-04)
(61183,2.628e-04) (187903,9.063e-05) (553983,3.053e-05)
};

\addplot coordinates{
(13,5.755e-02)     (97,2.925e-02)     (545,1.351e-02)
(2561,5.842e-03)   (10625,2.397e-03)  (40193,9.414e-04)
(141569,3.564e-04) (471041,1.308e-04)
(1496065,4.670e-05)
};
}%


\bookmaketitle
\tableofcontents
\include{pgfplots.title_abstract_intro}
\include{pgfplots.preliminaries}
\include{pgfplots.intro}
\include{pgfplots.reference}
\include{pgfplots.libs}
\include{pgfplots.resources}
\include{pgfplots.importexport}
\include{pgfplots.basic.reference}

\printindex

\bibliographystyle{abbrv} %gerapali} %gerabbrv} %gerunsrt.bst} %gerabbrv}% gerplain}
\nocite{pgfplotstable}
\nocite{programmingnotes}
\bibliography{pgfplots}
\end{document}

% -----------------------------------------------------------------------------
% For Stefan Pinnow as reminder on what to look for when editing the manual
% -----------------------------------------------------------------------------
% There should be no line breaks in the following environments
% - |...|
% - \declareandlabel{...}
% - \verbpdfref{...}
% If "MakeTikzPictures" isn't running through check one of the externalized
% LOG files what is the cause of that.
% -----------------------------------------------------------------------------


%%%%%%%%%%%%%%%%%%%%%%%%%%%%%%%%%%%%%%%%%%%%%%%%%%%%%%%%%%%%%%%%%%%%%%%%%%%%%
%
% Package pgfplots.sty documentation.
%
% Copyright 2007/2008 by Christian Feuersaenger.
%
% This program is free software: you can redistribute it and/or modify
% it under the terms of the GNU General Public License as published by
% the Free Software Foundation, either version 3 of the License, or
% (at your option) any later version.
%
% This program is distributed in the hope that it will be useful,
% but WITHOUT ANY WARRANTY; without even the implied warranty of
% MERCHANTABILITY or FITNESS FOR A PARTICULAR PURPOSE.  See the
% GNU General Public License for more details.
%
% You should have received a copy of the GNU General Public License
% along with this program.  If not, see <http://www.gnu.org/licenses/>.
%
%
%%%%%%%%%%%%%%%%%%%%%%%%%%%%%%%%%%%%%%%%%%%%%%%%%%%%%%%%%%%%%%%%%%%%%%%%%%%%%
% SEE pgfplots-macros.tex as well!
%\pdfminorversion=5 % to allow compression
%\pdfobjcompresslevel=2
\documentclass[a4paper,openany]{book}

\let\bookmaketitle=\maketitle

% -----------------------------------
% this here is from ltxdoc:
\usepackage{doc}[=v2]
\AtBeginDocument{\MakeShortVerb{\|}}
%\setlength{\textwidth}{355pt}
%\addtolength\marginparwidth{30pt}
%\addtolength\oddsidemargin{20pt}
%\addtolength\evensidemargin{20pt}
\def\cmd#1{\cs{\expandafter\cmd@to@cs\string#1}}
\def\cmd@to@cs#1#2{\char\number`#2\relax}
\DeclareRobustCommand\cs[1]{\texttt{\char`\\#1}}
\providecommand\marg[1]{%
  {\ttfamily\char`\{}\meta{#1}{\ttfamily\char`\}}}
\providecommand\oarg[1]{%
  {\ttfamily[}\meta{#1}{\ttfamily]}}
\providecommand\parg[1]{%
  {\ttfamily(}\meta{#1}{\ttfamily)}}
\raggedbottom

% -----------------------------------
\input{pgfplots.preamble.tex}

\makeatletter
% I want a two-column index just as in pgfmanual styles. This here
% was the best way to get one:
\def\index@prologue{\section*{Index}\addcontentsline{toc}{chapter}{Index}
}
\makeatother

%\RequirePackage[german,english,francais]{babel}

\def\matlabcolormaptext{This colormap is similar to one shipped with Matlab$^\text{\textregistered}$ under a similar name.}

\IfFileExists{tikzlibraryspy.code.tex}{%
\usetikzlibrary{spy}
}{%
    \message{ERROR: tikz SPY library NOT available. The manual will only compile partially.^^J}%
}%

\usepackage{xparse}% for colorbrewer manual

\usetikzlibrary{
    decorations.markings,
    decorations.footprints,
    shapes.arrows,
    matrix,
    positioning,
}

\usepgfplotslibrary{
    fillbetween,
    ternary,
    smithchart,
    patchplots,
    polar,
    colormaps,
    colorbrewer,
    colortol,           % docu not ready yet
}
\pgfqkeys{/codeexample}{%
    every codeexample/.append style={
        /pgfplots/every ternary axis/.append style={
            /pgfplots/legend style={fill=graphicbackground},
        }
    },
    tabsize=4,
}

\pgfplotsmanualenableexternalizationofexpensive

%\usetikzlibrary{external}
%\tikzexternalize[prefix=figures/]{pgfplots}

\title{%
    Manual for Package \PGFPlots{}\\
    {\small 2D/3D Plots in \LaTeX{}, Version \pgfplotsversion}\\
    {\small\url{https://github.com/pgf-tikz/pgfplots}}
    %\\{\small Attention: you are using an unstable development version.}
}

\makeatletter
\long\def\abstractsmuggle{%
    \centering
    \textbf{Abstract}\\[0.5cm]

    \begin{minipage}{12cm}
        \PGFPlots{} draws high-quality function plots in normal or logarithmic
        scaling with a user-friendly interface directly in \TeX{}. The user
        supplies axis labels, legend entries and the plot coordinates for one or
        more plots and \PGFPlots{} applies axis scaling, computes any logarithms
        and axis ticks and draws the plots. It supports line plots, scatter
        plots, piecewise constant plots, bar plots, area plots, mesh and surface
        plots, patch plots, contour plots, quiver plots, histogram plots, box
        plots, polar axes, ternary diagrams, smith charts and some more. It is
        based on Till Tantau's package \PGF{}/\Tikz{}.
    \end{minipage}

}%

\expandafter\date\expandafter{\@date\\[2cm]
    \abstractsmuggle
}%
\makeatother

%\includeonly{pgfplots.reference}


\begin{document}

\def\plotcoords{%
\addplot coordinates {
(5,8.312e-02)    (17,2.547e-02)   (49,7.407e-03)
(129,2.102e-03)  (321,5.874e-04)  (769,1.623e-04)
(1793,4.442e-05) (4097,1.207e-05) (9217,3.261e-06)
};

\addplot coordinates{
(7,8.472e-02)    (31,3.044e-02)    (111,1.022e-02)
(351,3.303e-03)  (1023,1.039e-03)  (2815,3.196e-04)
(7423,9.658e-05) (18943,2.873e-05) (47103,8.437e-06)
};

\addplot coordinates{
(9,7.881e-02)     (49,3.243e-02)    (209,1.232e-02)
(769,4.454e-03)   (2561,1.551e-03)  (7937,5.236e-04)
(23297,1.723e-04) (65537,5.545e-05) (178177,1.751e-05)
};

\addplot coordinates{
(11,6.887e-02)    (71,3.177e-02)     (351,1.341e-02)
(1471,5.334e-03)  (5503,2.027e-03)   (18943,7.415e-04)
(61183,2.628e-04) (187903,9.063e-05) (553983,3.053e-05)
};

\addplot coordinates{
(13,5.755e-02)     (97,2.925e-02)     (545,1.351e-02)
(2561,5.842e-03)   (10625,2.397e-03)  (40193,9.414e-04)
(141569,3.564e-04) (471041,1.308e-04)
(1496065,4.670e-05)
};
}%


\bookmaketitle
\tableofcontents
\include{pgfplots.title_abstract_intro}
\include{pgfplots.preliminaries}
\include{pgfplots.intro}
\include{pgfplots.reference}
\include{pgfplots.libs}
\include{pgfplots.resources}
\include{pgfplots.importexport}
\include{pgfplots.basic.reference}

\printindex

\bibliographystyle{abbrv} %gerapali} %gerabbrv} %gerunsrt.bst} %gerabbrv}% gerplain}
\nocite{pgfplotstable}
\nocite{programmingnotes}
\bibliography{pgfplots}
\end{document}

% -----------------------------------------------------------------------------
% For Stefan Pinnow as reminder on what to look for when editing the manual
% -----------------------------------------------------------------------------
% There should be no line breaks in the following environments
% - |...|
% - \declareandlabel{...}
% - \verbpdfref{...}
% If "MakeTikzPictures" isn't running through check one of the externalized
% LOG files what is the cause of that.
% -----------------------------------------------------------------------------

\include{pgfplots.basic.reference}

\printindex

\bibliographystyle{abbrv} %gerapali} %gerabbrv} %gerunsrt.bst} %gerabbrv}% gerplain}
\nocite{pgfplotstable}
\nocite{programmingnotes}
\bibliography{pgfplots}
\end{document}

% -----------------------------------------------------------------------------
% For Stefan Pinnow as reminder on what to look for when editing the manual
% -----------------------------------------------------------------------------
% There should be no line breaks in the following environments
% - |...|
% - \declareandlabel{...}
% - \verbpdfref{...}
% If "MakeTikzPictures" isn't running through check one of the externalized
% LOG files what is the cause of that.
% -----------------------------------------------------------------------------


%%%%%%%%%%%%%%%%%%%%%%%%%%%%%%%%%%%%%%%%%%%%%%%%%%%%%%%%%%%%%%%%%%%%%%%%%%%%%
%
% Package pgfplots.sty documentation.
%
% Copyright 2007/2008 by Christian Feuersaenger.
%
% This program is free software: you can redistribute it and/or modify
% it under the terms of the GNU General Public License as published by
% the Free Software Foundation, either version 3 of the License, or
% (at your option) any later version.
%
% This program is distributed in the hope that it will be useful,
% but WITHOUT ANY WARRANTY; without even the implied warranty of
% MERCHANTABILITY or FITNESS FOR A PARTICULAR PURPOSE.  See the
% GNU General Public License for more details.
%
% You should have received a copy of the GNU General Public License
% along with this program.  If not, see <http://www.gnu.org/licenses/>.
%
%
%%%%%%%%%%%%%%%%%%%%%%%%%%%%%%%%%%%%%%%%%%%%%%%%%%%%%%%%%%%%%%%%%%%%%%%%%%%%%
% SEE pgfplots-macros.tex as well!
%\pdfminorversion=5 % to allow compression
%\pdfobjcompresslevel=2
\documentclass[a4paper,openany]{book}

\let\bookmaketitle=\maketitle

% -----------------------------------
% this here is from ltxdoc:
\usepackage{doc}[=v2]
\AtBeginDocument{\MakeShortVerb{\|}}
%\setlength{\textwidth}{355pt}
%\addtolength\marginparwidth{30pt}
%\addtolength\oddsidemargin{20pt}
%\addtolength\evensidemargin{20pt}
\def\cmd#1{\cs{\expandafter\cmd@to@cs\string#1}}
\def\cmd@to@cs#1#2{\char\number`#2\relax}
\DeclareRobustCommand\cs[1]{\texttt{\char`\\#1}}
\providecommand\marg[1]{%
  {\ttfamily\char`\{}\meta{#1}{\ttfamily\char`\}}}
\providecommand\oarg[1]{%
  {\ttfamily[}\meta{#1}{\ttfamily]}}
\providecommand\parg[1]{%
  {\ttfamily(}\meta{#1}{\ttfamily)}}
\raggedbottom

% -----------------------------------
\input{pgfplots.preamble.tex}

\makeatletter
% I want a two-column index just as in pgfmanual styles. This here
% was the best way to get one:
\def\index@prologue{\section*{Index}\addcontentsline{toc}{chapter}{Index}
}
\makeatother

%\RequirePackage[german,english,francais]{babel}

\def\matlabcolormaptext{This colormap is similar to one shipped with Matlab$^\text{\textregistered}$ under a similar name.}

\IfFileExists{tikzlibraryspy.code.tex}{%
\usetikzlibrary{spy}
}{%
    \message{ERROR: tikz SPY library NOT available. The manual will only compile partially.^^J}%
}%

\usepackage{xparse}% for colorbrewer manual

\usetikzlibrary{
    decorations.markings,
    decorations.footprints,
    shapes.arrows,
    matrix,
    positioning,
}

\usepgfplotslibrary{
    fillbetween,
    ternary,
    smithchart,
    patchplots,
    polar,
    colormaps,
    colorbrewer,
    colortol,           % docu not ready yet
}
\pgfqkeys{/codeexample}{%
    every codeexample/.append style={
        /pgfplots/every ternary axis/.append style={
            /pgfplots/legend style={fill=graphicbackground},
        }
    },
    tabsize=4,
}

\pgfplotsmanualenableexternalizationofexpensive

%\usetikzlibrary{external}
%\tikzexternalize[prefix=figures/]{pgfplots}

\title{%
    Manual for Package \PGFPlots{}\\
    {\small 2D/3D Plots in \LaTeX{}, Version \pgfplotsversion}\\
    {\small\url{https://github.com/pgf-tikz/pgfplots}}
    %\\{\small Attention: you are using an unstable development version.}
}

\makeatletter
\long\def\abstractsmuggle{%
    \centering
    \textbf{Abstract}\\[0.5cm]

    \begin{minipage}{12cm}
        \PGFPlots{} draws high-quality function plots in normal or logarithmic
        scaling with a user-friendly interface directly in \TeX{}. The user
        supplies axis labels, legend entries and the plot coordinates for one or
        more plots and \PGFPlots{} applies axis scaling, computes any logarithms
        and axis ticks and draws the plots. It supports line plots, scatter
        plots, piecewise constant plots, bar plots, area plots, mesh and surface
        plots, patch plots, contour plots, quiver plots, histogram plots, box
        plots, polar axes, ternary diagrams, smith charts and some more. It is
        based on Till Tantau's package \PGF{}/\Tikz{}.
    \end{minipage}

}%

\expandafter\date\expandafter{\@date\\[2cm]
    \abstractsmuggle
}%
\makeatother

%\includeonly{pgfplots.reference}


\begin{document}

\def\plotcoords{%
\addplot coordinates {
(5,8.312e-02)    (17,2.547e-02)   (49,7.407e-03)
(129,2.102e-03)  (321,5.874e-04)  (769,1.623e-04)
(1793,4.442e-05) (4097,1.207e-05) (9217,3.261e-06)
};

\addplot coordinates{
(7,8.472e-02)    (31,3.044e-02)    (111,1.022e-02)
(351,3.303e-03)  (1023,1.039e-03)  (2815,3.196e-04)
(7423,9.658e-05) (18943,2.873e-05) (47103,8.437e-06)
};

\addplot coordinates{
(9,7.881e-02)     (49,3.243e-02)    (209,1.232e-02)
(769,4.454e-03)   (2561,1.551e-03)  (7937,5.236e-04)
(23297,1.723e-04) (65537,5.545e-05) (178177,1.751e-05)
};

\addplot coordinates{
(11,6.887e-02)    (71,3.177e-02)     (351,1.341e-02)
(1471,5.334e-03)  (5503,2.027e-03)   (18943,7.415e-04)
(61183,2.628e-04) (187903,9.063e-05) (553983,3.053e-05)
};

\addplot coordinates{
(13,5.755e-02)     (97,2.925e-02)     (545,1.351e-02)
(2561,5.842e-03)   (10625,2.397e-03)  (40193,9.414e-04)
(141569,3.564e-04) (471041,1.308e-04)
(1496065,4.670e-05)
};
}%


\bookmaketitle
\tableofcontents

%%%%%%%%%%%%%%%%%%%%%%%%%%%%%%%%%%%%%%%%%%%%%%%%%%%%%%%%%%%%%%%%%%%%%%%%%%%%%
%
% Package pgfplots.sty documentation.
%
% Copyright 2007/2008 by Christian Feuersaenger.
%
% This program is free software: you can redistribute it and/or modify
% it under the terms of the GNU General Public License as published by
% the Free Software Foundation, either version 3 of the License, or
% (at your option) any later version.
%
% This program is distributed in the hope that it will be useful,
% but WITHOUT ANY WARRANTY; without even the implied warranty of
% MERCHANTABILITY or FITNESS FOR A PARTICULAR PURPOSE.  See the
% GNU General Public License for more details.
%
% You should have received a copy of the GNU General Public License
% along with this program.  If not, see <http://www.gnu.org/licenses/>.
%
%
%%%%%%%%%%%%%%%%%%%%%%%%%%%%%%%%%%%%%%%%%%%%%%%%%%%%%%%%%%%%%%%%%%%%%%%%%%%%%
% SEE pgfplots-macros.tex as well!
%\pdfminorversion=5 % to allow compression
%\pdfobjcompresslevel=2
\documentclass[a4paper,openany]{book}

\let\bookmaketitle=\maketitle

% -----------------------------------
% this here is from ltxdoc:
\usepackage{doc}[=v2]
\AtBeginDocument{\MakeShortVerb{\|}}
%\setlength{\textwidth}{355pt}
%\addtolength\marginparwidth{30pt}
%\addtolength\oddsidemargin{20pt}
%\addtolength\evensidemargin{20pt}
\def\cmd#1{\cs{\expandafter\cmd@to@cs\string#1}}
\def\cmd@to@cs#1#2{\char\number`#2\relax}
\DeclareRobustCommand\cs[1]{\texttt{\char`\\#1}}
\providecommand\marg[1]{%
  {\ttfamily\char`\{}\meta{#1}{\ttfamily\char`\}}}
\providecommand\oarg[1]{%
  {\ttfamily[}\meta{#1}{\ttfamily]}}
\providecommand\parg[1]{%
  {\ttfamily(}\meta{#1}{\ttfamily)}}
\raggedbottom

% -----------------------------------
\input{pgfplots.preamble.tex}

\makeatletter
% I want a two-column index just as in pgfmanual styles. This here
% was the best way to get one:
\def\index@prologue{\section*{Index}\addcontentsline{toc}{chapter}{Index}
}
\makeatother

%\RequirePackage[german,english,francais]{babel}

\def\matlabcolormaptext{This colormap is similar to one shipped with Matlab$^\text{\textregistered}$ under a similar name.}

\IfFileExists{tikzlibraryspy.code.tex}{%
\usetikzlibrary{spy}
}{%
    \message{ERROR: tikz SPY library NOT available. The manual will only compile partially.^^J}%
}%

\usepackage{xparse}% for colorbrewer manual

\usetikzlibrary{
    decorations.markings,
    decorations.footprints,
    shapes.arrows,
    matrix,
    positioning,
}

\usepgfplotslibrary{
    fillbetween,
    ternary,
    smithchart,
    patchplots,
    polar,
    colormaps,
    colorbrewer,
    colortol,           % docu not ready yet
}
\pgfqkeys{/codeexample}{%
    every codeexample/.append style={
        /pgfplots/every ternary axis/.append style={
            /pgfplots/legend style={fill=graphicbackground},
        }
    },
    tabsize=4,
}

\pgfplotsmanualenableexternalizationofexpensive

%\usetikzlibrary{external}
%\tikzexternalize[prefix=figures/]{pgfplots}

\title{%
    Manual for Package \PGFPlots{}\\
    {\small 2D/3D Plots in \LaTeX{}, Version \pgfplotsversion}\\
    {\small\url{https://github.com/pgf-tikz/pgfplots}}
    %\\{\small Attention: you are using an unstable development version.}
}

\makeatletter
\long\def\abstractsmuggle{%
    \centering
    \textbf{Abstract}\\[0.5cm]

    \begin{minipage}{12cm}
        \PGFPlots{} draws high-quality function plots in normal or logarithmic
        scaling with a user-friendly interface directly in \TeX{}. The user
        supplies axis labels, legend entries and the plot coordinates for one or
        more plots and \PGFPlots{} applies axis scaling, computes any logarithms
        and axis ticks and draws the plots. It supports line plots, scatter
        plots, piecewise constant plots, bar plots, area plots, mesh and surface
        plots, patch plots, contour plots, quiver plots, histogram plots, box
        plots, polar axes, ternary diagrams, smith charts and some more. It is
        based on Till Tantau's package \PGF{}/\Tikz{}.
    \end{minipage}

}%

\expandafter\date\expandafter{\@date\\[2cm]
    \abstractsmuggle
}%
\makeatother

%\includeonly{pgfplots.reference}


\begin{document}

\def\plotcoords{%
\addplot coordinates {
(5,8.312e-02)    (17,2.547e-02)   (49,7.407e-03)
(129,2.102e-03)  (321,5.874e-04)  (769,1.623e-04)
(1793,4.442e-05) (4097,1.207e-05) (9217,3.261e-06)
};

\addplot coordinates{
(7,8.472e-02)    (31,3.044e-02)    (111,1.022e-02)
(351,3.303e-03)  (1023,1.039e-03)  (2815,3.196e-04)
(7423,9.658e-05) (18943,2.873e-05) (47103,8.437e-06)
};

\addplot coordinates{
(9,7.881e-02)     (49,3.243e-02)    (209,1.232e-02)
(769,4.454e-03)   (2561,1.551e-03)  (7937,5.236e-04)
(23297,1.723e-04) (65537,5.545e-05) (178177,1.751e-05)
};

\addplot coordinates{
(11,6.887e-02)    (71,3.177e-02)     (351,1.341e-02)
(1471,5.334e-03)  (5503,2.027e-03)   (18943,7.415e-04)
(61183,2.628e-04) (187903,9.063e-05) (553983,3.053e-05)
};

\addplot coordinates{
(13,5.755e-02)     (97,2.925e-02)     (545,1.351e-02)
(2561,5.842e-03)   (10625,2.397e-03)  (40193,9.414e-04)
(141569,3.564e-04) (471041,1.308e-04)
(1496065,4.670e-05)
};
}%


\bookmaketitle
\tableofcontents
\include{pgfplots.title_abstract_intro}
\include{pgfplots.preliminaries}
\include{pgfplots.intro}
\include{pgfplots.reference}
\include{pgfplots.libs}
\include{pgfplots.resources}
\include{pgfplots.importexport}
\include{pgfplots.basic.reference}

\printindex

\bibliographystyle{abbrv} %gerapali} %gerabbrv} %gerunsrt.bst} %gerabbrv}% gerplain}
\nocite{pgfplotstable}
\nocite{programmingnotes}
\bibliography{pgfplots}
\end{document}

% -----------------------------------------------------------------------------
% For Stefan Pinnow as reminder on what to look for when editing the manual
% -----------------------------------------------------------------------------
% There should be no line breaks in the following environments
% - |...|
% - \declareandlabel{...}
% - \verbpdfref{...}
% If "MakeTikzPictures" isn't running through check one of the externalized
% LOG files what is the cause of that.
% -----------------------------------------------------------------------------


%%%%%%%%%%%%%%%%%%%%%%%%%%%%%%%%%%%%%%%%%%%%%%%%%%%%%%%%%%%%%%%%%%%%%%%%%%%%%
%
% Package pgfplots.sty documentation.
%
% Copyright 2007/2008 by Christian Feuersaenger.
%
% This program is free software: you can redistribute it and/or modify
% it under the terms of the GNU General Public License as published by
% the Free Software Foundation, either version 3 of the License, or
% (at your option) any later version.
%
% This program is distributed in the hope that it will be useful,
% but WITHOUT ANY WARRANTY; without even the implied warranty of
% MERCHANTABILITY or FITNESS FOR A PARTICULAR PURPOSE.  See the
% GNU General Public License for more details.
%
% You should have received a copy of the GNU General Public License
% along with this program.  If not, see <http://www.gnu.org/licenses/>.
%
%
%%%%%%%%%%%%%%%%%%%%%%%%%%%%%%%%%%%%%%%%%%%%%%%%%%%%%%%%%%%%%%%%%%%%%%%%%%%%%
% SEE pgfplots-macros.tex as well!
%\pdfminorversion=5 % to allow compression
%\pdfobjcompresslevel=2
\documentclass[a4paper,openany]{book}

\let\bookmaketitle=\maketitle

% -----------------------------------
% this here is from ltxdoc:
\usepackage{doc}[=v2]
\AtBeginDocument{\MakeShortVerb{\|}}
%\setlength{\textwidth}{355pt}
%\addtolength\marginparwidth{30pt}
%\addtolength\oddsidemargin{20pt}
%\addtolength\evensidemargin{20pt}
\def\cmd#1{\cs{\expandafter\cmd@to@cs\string#1}}
\def\cmd@to@cs#1#2{\char\number`#2\relax}
\DeclareRobustCommand\cs[1]{\texttt{\char`\\#1}}
\providecommand\marg[1]{%
  {\ttfamily\char`\{}\meta{#1}{\ttfamily\char`\}}}
\providecommand\oarg[1]{%
  {\ttfamily[}\meta{#1}{\ttfamily]}}
\providecommand\parg[1]{%
  {\ttfamily(}\meta{#1}{\ttfamily)}}
\raggedbottom

% -----------------------------------
\input{pgfplots.preamble.tex}

\makeatletter
% I want a two-column index just as in pgfmanual styles. This here
% was the best way to get one:
\def\index@prologue{\section*{Index}\addcontentsline{toc}{chapter}{Index}
}
\makeatother

%\RequirePackage[german,english,francais]{babel}

\def\matlabcolormaptext{This colormap is similar to one shipped with Matlab$^\text{\textregistered}$ under a similar name.}

\IfFileExists{tikzlibraryspy.code.tex}{%
\usetikzlibrary{spy}
}{%
    \message{ERROR: tikz SPY library NOT available. The manual will only compile partially.^^J}%
}%

\usepackage{xparse}% for colorbrewer manual

\usetikzlibrary{
    decorations.markings,
    decorations.footprints,
    shapes.arrows,
    matrix,
    positioning,
}

\usepgfplotslibrary{
    fillbetween,
    ternary,
    smithchart,
    patchplots,
    polar,
    colormaps,
    colorbrewer,
    colortol,           % docu not ready yet
}
\pgfqkeys{/codeexample}{%
    every codeexample/.append style={
        /pgfplots/every ternary axis/.append style={
            /pgfplots/legend style={fill=graphicbackground},
        }
    },
    tabsize=4,
}

\pgfplotsmanualenableexternalizationofexpensive

%\usetikzlibrary{external}
%\tikzexternalize[prefix=figures/]{pgfplots}

\title{%
    Manual for Package \PGFPlots{}\\
    {\small 2D/3D Plots in \LaTeX{}, Version \pgfplotsversion}\\
    {\small\url{https://github.com/pgf-tikz/pgfplots}}
    %\\{\small Attention: you are using an unstable development version.}
}

\makeatletter
\long\def\abstractsmuggle{%
    \centering
    \textbf{Abstract}\\[0.5cm]

    \begin{minipage}{12cm}
        \PGFPlots{} draws high-quality function plots in normal or logarithmic
        scaling with a user-friendly interface directly in \TeX{}. The user
        supplies axis labels, legend entries and the plot coordinates for one or
        more plots and \PGFPlots{} applies axis scaling, computes any logarithms
        and axis ticks and draws the plots. It supports line plots, scatter
        plots, piecewise constant plots, bar plots, area plots, mesh and surface
        plots, patch plots, contour plots, quiver plots, histogram plots, box
        plots, polar axes, ternary diagrams, smith charts and some more. It is
        based on Till Tantau's package \PGF{}/\Tikz{}.
    \end{minipage}

}%

\expandafter\date\expandafter{\@date\\[2cm]
    \abstractsmuggle
}%
\makeatother

%\includeonly{pgfplots.reference}


\begin{document}

\def\plotcoords{%
\addplot coordinates {
(5,8.312e-02)    (17,2.547e-02)   (49,7.407e-03)
(129,2.102e-03)  (321,5.874e-04)  (769,1.623e-04)
(1793,4.442e-05) (4097,1.207e-05) (9217,3.261e-06)
};

\addplot coordinates{
(7,8.472e-02)    (31,3.044e-02)    (111,1.022e-02)
(351,3.303e-03)  (1023,1.039e-03)  (2815,3.196e-04)
(7423,9.658e-05) (18943,2.873e-05) (47103,8.437e-06)
};

\addplot coordinates{
(9,7.881e-02)     (49,3.243e-02)    (209,1.232e-02)
(769,4.454e-03)   (2561,1.551e-03)  (7937,5.236e-04)
(23297,1.723e-04) (65537,5.545e-05) (178177,1.751e-05)
};

\addplot coordinates{
(11,6.887e-02)    (71,3.177e-02)     (351,1.341e-02)
(1471,5.334e-03)  (5503,2.027e-03)   (18943,7.415e-04)
(61183,2.628e-04) (187903,9.063e-05) (553983,3.053e-05)
};

\addplot coordinates{
(13,5.755e-02)     (97,2.925e-02)     (545,1.351e-02)
(2561,5.842e-03)   (10625,2.397e-03)  (40193,9.414e-04)
(141569,3.564e-04) (471041,1.308e-04)
(1496065,4.670e-05)
};
}%


\bookmaketitle
\tableofcontents
\include{pgfplots.title_abstract_intro}
\include{pgfplots.preliminaries}
\include{pgfplots.intro}
\include{pgfplots.reference}
\include{pgfplots.libs}
\include{pgfplots.resources}
\include{pgfplots.importexport}
\include{pgfplots.basic.reference}

\printindex

\bibliographystyle{abbrv} %gerapali} %gerabbrv} %gerunsrt.bst} %gerabbrv}% gerplain}
\nocite{pgfplotstable}
\nocite{programmingnotes}
\bibliography{pgfplots}
\end{document}

% -----------------------------------------------------------------------------
% For Stefan Pinnow as reminder on what to look for when editing the manual
% -----------------------------------------------------------------------------
% There should be no line breaks in the following environments
% - |...|
% - \declareandlabel{...}
% - \verbpdfref{...}
% If "MakeTikzPictures" isn't running through check one of the externalized
% LOG files what is the cause of that.
% -----------------------------------------------------------------------------


%%%%%%%%%%%%%%%%%%%%%%%%%%%%%%%%%%%%%%%%%%%%%%%%%%%%%%%%%%%%%%%%%%%%%%%%%%%%%
%
% Package pgfplots.sty documentation.
%
% Copyright 2007/2008 by Christian Feuersaenger.
%
% This program is free software: you can redistribute it and/or modify
% it under the terms of the GNU General Public License as published by
% the Free Software Foundation, either version 3 of the License, or
% (at your option) any later version.
%
% This program is distributed in the hope that it will be useful,
% but WITHOUT ANY WARRANTY; without even the implied warranty of
% MERCHANTABILITY or FITNESS FOR A PARTICULAR PURPOSE.  See the
% GNU General Public License for more details.
%
% You should have received a copy of the GNU General Public License
% along with this program.  If not, see <http://www.gnu.org/licenses/>.
%
%
%%%%%%%%%%%%%%%%%%%%%%%%%%%%%%%%%%%%%%%%%%%%%%%%%%%%%%%%%%%%%%%%%%%%%%%%%%%%%
% SEE pgfplots-macros.tex as well!
%\pdfminorversion=5 % to allow compression
%\pdfobjcompresslevel=2
\documentclass[a4paper,openany]{book}

\let\bookmaketitle=\maketitle

% -----------------------------------
% this here is from ltxdoc:
\usepackage{doc}[=v2]
\AtBeginDocument{\MakeShortVerb{\|}}
%\setlength{\textwidth}{355pt}
%\addtolength\marginparwidth{30pt}
%\addtolength\oddsidemargin{20pt}
%\addtolength\evensidemargin{20pt}
\def\cmd#1{\cs{\expandafter\cmd@to@cs\string#1}}
\def\cmd@to@cs#1#2{\char\number`#2\relax}
\DeclareRobustCommand\cs[1]{\texttt{\char`\\#1}}
\providecommand\marg[1]{%
  {\ttfamily\char`\{}\meta{#1}{\ttfamily\char`\}}}
\providecommand\oarg[1]{%
  {\ttfamily[}\meta{#1}{\ttfamily]}}
\providecommand\parg[1]{%
  {\ttfamily(}\meta{#1}{\ttfamily)}}
\raggedbottom

% -----------------------------------
\input{pgfplots.preamble.tex}

\makeatletter
% I want a two-column index just as in pgfmanual styles. This here
% was the best way to get one:
\def\index@prologue{\section*{Index}\addcontentsline{toc}{chapter}{Index}
}
\makeatother

%\RequirePackage[german,english,francais]{babel}

\def\matlabcolormaptext{This colormap is similar to one shipped with Matlab$^\text{\textregistered}$ under a similar name.}

\IfFileExists{tikzlibraryspy.code.tex}{%
\usetikzlibrary{spy}
}{%
    \message{ERROR: tikz SPY library NOT available. The manual will only compile partially.^^J}%
}%

\usepackage{xparse}% for colorbrewer manual

\usetikzlibrary{
    decorations.markings,
    decorations.footprints,
    shapes.arrows,
    matrix,
    positioning,
}

\usepgfplotslibrary{
    fillbetween,
    ternary,
    smithchart,
    patchplots,
    polar,
    colormaps,
    colorbrewer,
    colortol,           % docu not ready yet
}
\pgfqkeys{/codeexample}{%
    every codeexample/.append style={
        /pgfplots/every ternary axis/.append style={
            /pgfplots/legend style={fill=graphicbackground},
        }
    },
    tabsize=4,
}

\pgfplotsmanualenableexternalizationofexpensive

%\usetikzlibrary{external}
%\tikzexternalize[prefix=figures/]{pgfplots}

\title{%
    Manual for Package \PGFPlots{}\\
    {\small 2D/3D Plots in \LaTeX{}, Version \pgfplotsversion}\\
    {\small\url{https://github.com/pgf-tikz/pgfplots}}
    %\\{\small Attention: you are using an unstable development version.}
}

\makeatletter
\long\def\abstractsmuggle{%
    \centering
    \textbf{Abstract}\\[0.5cm]

    \begin{minipage}{12cm}
        \PGFPlots{} draws high-quality function plots in normal or logarithmic
        scaling with a user-friendly interface directly in \TeX{}. The user
        supplies axis labels, legend entries and the plot coordinates for one or
        more plots and \PGFPlots{} applies axis scaling, computes any logarithms
        and axis ticks and draws the plots. It supports line plots, scatter
        plots, piecewise constant plots, bar plots, area plots, mesh and surface
        plots, patch plots, contour plots, quiver plots, histogram plots, box
        plots, polar axes, ternary diagrams, smith charts and some more. It is
        based on Till Tantau's package \PGF{}/\Tikz{}.
    \end{minipage}

}%

\expandafter\date\expandafter{\@date\\[2cm]
    \abstractsmuggle
}%
\makeatother

%\includeonly{pgfplots.reference}


\begin{document}

\def\plotcoords{%
\addplot coordinates {
(5,8.312e-02)    (17,2.547e-02)   (49,7.407e-03)
(129,2.102e-03)  (321,5.874e-04)  (769,1.623e-04)
(1793,4.442e-05) (4097,1.207e-05) (9217,3.261e-06)
};

\addplot coordinates{
(7,8.472e-02)    (31,3.044e-02)    (111,1.022e-02)
(351,3.303e-03)  (1023,1.039e-03)  (2815,3.196e-04)
(7423,9.658e-05) (18943,2.873e-05) (47103,8.437e-06)
};

\addplot coordinates{
(9,7.881e-02)     (49,3.243e-02)    (209,1.232e-02)
(769,4.454e-03)   (2561,1.551e-03)  (7937,5.236e-04)
(23297,1.723e-04) (65537,5.545e-05) (178177,1.751e-05)
};

\addplot coordinates{
(11,6.887e-02)    (71,3.177e-02)     (351,1.341e-02)
(1471,5.334e-03)  (5503,2.027e-03)   (18943,7.415e-04)
(61183,2.628e-04) (187903,9.063e-05) (553983,3.053e-05)
};

\addplot coordinates{
(13,5.755e-02)     (97,2.925e-02)     (545,1.351e-02)
(2561,5.842e-03)   (10625,2.397e-03)  (40193,9.414e-04)
(141569,3.564e-04) (471041,1.308e-04)
(1496065,4.670e-05)
};
}%


\bookmaketitle
\tableofcontents
\include{pgfplots.title_abstract_intro}
\include{pgfplots.preliminaries}
\include{pgfplots.intro}
\include{pgfplots.reference}
\include{pgfplots.libs}
\include{pgfplots.resources}
\include{pgfplots.importexport}
\include{pgfplots.basic.reference}

\printindex

\bibliographystyle{abbrv} %gerapali} %gerabbrv} %gerunsrt.bst} %gerabbrv}% gerplain}
\nocite{pgfplotstable}
\nocite{programmingnotes}
\bibliography{pgfplots}
\end{document}

% -----------------------------------------------------------------------------
% For Stefan Pinnow as reminder on what to look for when editing the manual
% -----------------------------------------------------------------------------
% There should be no line breaks in the following environments
% - |...|
% - \declareandlabel{...}
% - \verbpdfref{...}
% If "MakeTikzPictures" isn't running through check one of the externalized
% LOG files what is the cause of that.
% -----------------------------------------------------------------------------


%%%%%%%%%%%%%%%%%%%%%%%%%%%%%%%%%%%%%%%%%%%%%%%%%%%%%%%%%%%%%%%%%%%%%%%%%%%%%
%
% Package pgfplots.sty documentation.
%
% Copyright 2007/2008 by Christian Feuersaenger.
%
% This program is free software: you can redistribute it and/or modify
% it under the terms of the GNU General Public License as published by
% the Free Software Foundation, either version 3 of the License, or
% (at your option) any later version.
%
% This program is distributed in the hope that it will be useful,
% but WITHOUT ANY WARRANTY; without even the implied warranty of
% MERCHANTABILITY or FITNESS FOR A PARTICULAR PURPOSE.  See the
% GNU General Public License for more details.
%
% You should have received a copy of the GNU General Public License
% along with this program.  If not, see <http://www.gnu.org/licenses/>.
%
%
%%%%%%%%%%%%%%%%%%%%%%%%%%%%%%%%%%%%%%%%%%%%%%%%%%%%%%%%%%%%%%%%%%%%%%%%%%%%%
% SEE pgfplots-macros.tex as well!
%\pdfminorversion=5 % to allow compression
%\pdfobjcompresslevel=2
\documentclass[a4paper,openany]{book}

\let\bookmaketitle=\maketitle

% -----------------------------------
% this here is from ltxdoc:
\usepackage{doc}[=v2]
\AtBeginDocument{\MakeShortVerb{\|}}
%\setlength{\textwidth}{355pt}
%\addtolength\marginparwidth{30pt}
%\addtolength\oddsidemargin{20pt}
%\addtolength\evensidemargin{20pt}
\def\cmd#1{\cs{\expandafter\cmd@to@cs\string#1}}
\def\cmd@to@cs#1#2{\char\number`#2\relax}
\DeclareRobustCommand\cs[1]{\texttt{\char`\\#1}}
\providecommand\marg[1]{%
  {\ttfamily\char`\{}\meta{#1}{\ttfamily\char`\}}}
\providecommand\oarg[1]{%
  {\ttfamily[}\meta{#1}{\ttfamily]}}
\providecommand\parg[1]{%
  {\ttfamily(}\meta{#1}{\ttfamily)}}
\raggedbottom

% -----------------------------------
\input{pgfplots.preamble.tex}

\makeatletter
% I want a two-column index just as in pgfmanual styles. This here
% was the best way to get one:
\def\index@prologue{\section*{Index}\addcontentsline{toc}{chapter}{Index}
}
\makeatother

%\RequirePackage[german,english,francais]{babel}

\def\matlabcolormaptext{This colormap is similar to one shipped with Matlab$^\text{\textregistered}$ under a similar name.}

\IfFileExists{tikzlibraryspy.code.tex}{%
\usetikzlibrary{spy}
}{%
    \message{ERROR: tikz SPY library NOT available. The manual will only compile partially.^^J}%
}%

\usepackage{xparse}% for colorbrewer manual

\usetikzlibrary{
    decorations.markings,
    decorations.footprints,
    shapes.arrows,
    matrix,
    positioning,
}

\usepgfplotslibrary{
    fillbetween,
    ternary,
    smithchart,
    patchplots,
    polar,
    colormaps,
    colorbrewer,
    colortol,           % docu not ready yet
}
\pgfqkeys{/codeexample}{%
    every codeexample/.append style={
        /pgfplots/every ternary axis/.append style={
            /pgfplots/legend style={fill=graphicbackground},
        }
    },
    tabsize=4,
}

\pgfplotsmanualenableexternalizationofexpensive

%\usetikzlibrary{external}
%\tikzexternalize[prefix=figures/]{pgfplots}

\title{%
    Manual for Package \PGFPlots{}\\
    {\small 2D/3D Plots in \LaTeX{}, Version \pgfplotsversion}\\
    {\small\url{https://github.com/pgf-tikz/pgfplots}}
    %\\{\small Attention: you are using an unstable development version.}
}

\makeatletter
\long\def\abstractsmuggle{%
    \centering
    \textbf{Abstract}\\[0.5cm]

    \begin{minipage}{12cm}
        \PGFPlots{} draws high-quality function plots in normal or logarithmic
        scaling with a user-friendly interface directly in \TeX{}. The user
        supplies axis labels, legend entries and the plot coordinates for one or
        more plots and \PGFPlots{} applies axis scaling, computes any logarithms
        and axis ticks and draws the plots. It supports line plots, scatter
        plots, piecewise constant plots, bar plots, area plots, mesh and surface
        plots, patch plots, contour plots, quiver plots, histogram plots, box
        plots, polar axes, ternary diagrams, smith charts and some more. It is
        based on Till Tantau's package \PGF{}/\Tikz{}.
    \end{minipage}

}%

\expandafter\date\expandafter{\@date\\[2cm]
    \abstractsmuggle
}%
\makeatother

%\includeonly{pgfplots.reference}


\begin{document}

\def\plotcoords{%
\addplot coordinates {
(5,8.312e-02)    (17,2.547e-02)   (49,7.407e-03)
(129,2.102e-03)  (321,5.874e-04)  (769,1.623e-04)
(1793,4.442e-05) (4097,1.207e-05) (9217,3.261e-06)
};

\addplot coordinates{
(7,8.472e-02)    (31,3.044e-02)    (111,1.022e-02)
(351,3.303e-03)  (1023,1.039e-03)  (2815,3.196e-04)
(7423,9.658e-05) (18943,2.873e-05) (47103,8.437e-06)
};

\addplot coordinates{
(9,7.881e-02)     (49,3.243e-02)    (209,1.232e-02)
(769,4.454e-03)   (2561,1.551e-03)  (7937,5.236e-04)
(23297,1.723e-04) (65537,5.545e-05) (178177,1.751e-05)
};

\addplot coordinates{
(11,6.887e-02)    (71,3.177e-02)     (351,1.341e-02)
(1471,5.334e-03)  (5503,2.027e-03)   (18943,7.415e-04)
(61183,2.628e-04) (187903,9.063e-05) (553983,3.053e-05)
};

\addplot coordinates{
(13,5.755e-02)     (97,2.925e-02)     (545,1.351e-02)
(2561,5.842e-03)   (10625,2.397e-03)  (40193,9.414e-04)
(141569,3.564e-04) (471041,1.308e-04)
(1496065,4.670e-05)
};
}%


\bookmaketitle
\tableofcontents
\include{pgfplots.title_abstract_intro}
\include{pgfplots.preliminaries}
\include{pgfplots.intro}
\include{pgfplots.reference}
\include{pgfplots.libs}
\include{pgfplots.resources}
\include{pgfplots.importexport}
\include{pgfplots.basic.reference}

\printindex

\bibliographystyle{abbrv} %gerapali} %gerabbrv} %gerunsrt.bst} %gerabbrv}% gerplain}
\nocite{pgfplotstable}
\nocite{programmingnotes}
\bibliography{pgfplots}
\end{document}

% -----------------------------------------------------------------------------
% For Stefan Pinnow as reminder on what to look for when editing the manual
% -----------------------------------------------------------------------------
% There should be no line breaks in the following environments
% - |...|
% - \declareandlabel{...}
% - \verbpdfref{...}
% If "MakeTikzPictures" isn't running through check one of the externalized
% LOG files what is the cause of that.
% -----------------------------------------------------------------------------


%%%%%%%%%%%%%%%%%%%%%%%%%%%%%%%%%%%%%%%%%%%%%%%%%%%%%%%%%%%%%%%%%%%%%%%%%%%%%
%
% Package pgfplots.sty documentation.
%
% Copyright 2007/2008 by Christian Feuersaenger.
%
% This program is free software: you can redistribute it and/or modify
% it under the terms of the GNU General Public License as published by
% the Free Software Foundation, either version 3 of the License, or
% (at your option) any later version.
%
% This program is distributed in the hope that it will be useful,
% but WITHOUT ANY WARRANTY; without even the implied warranty of
% MERCHANTABILITY or FITNESS FOR A PARTICULAR PURPOSE.  See the
% GNU General Public License for more details.
%
% You should have received a copy of the GNU General Public License
% along with this program.  If not, see <http://www.gnu.org/licenses/>.
%
%
%%%%%%%%%%%%%%%%%%%%%%%%%%%%%%%%%%%%%%%%%%%%%%%%%%%%%%%%%%%%%%%%%%%%%%%%%%%%%
% SEE pgfplots-macros.tex as well!
%\pdfminorversion=5 % to allow compression
%\pdfobjcompresslevel=2
\documentclass[a4paper,openany]{book}

\let\bookmaketitle=\maketitle

% -----------------------------------
% this here is from ltxdoc:
\usepackage{doc}[=v2]
\AtBeginDocument{\MakeShortVerb{\|}}
%\setlength{\textwidth}{355pt}
%\addtolength\marginparwidth{30pt}
%\addtolength\oddsidemargin{20pt}
%\addtolength\evensidemargin{20pt}
\def\cmd#1{\cs{\expandafter\cmd@to@cs\string#1}}
\def\cmd@to@cs#1#2{\char\number`#2\relax}
\DeclareRobustCommand\cs[1]{\texttt{\char`\\#1}}
\providecommand\marg[1]{%
  {\ttfamily\char`\{}\meta{#1}{\ttfamily\char`\}}}
\providecommand\oarg[1]{%
  {\ttfamily[}\meta{#1}{\ttfamily]}}
\providecommand\parg[1]{%
  {\ttfamily(}\meta{#1}{\ttfamily)}}
\raggedbottom

% -----------------------------------
\input{pgfplots.preamble.tex}

\makeatletter
% I want a two-column index just as in pgfmanual styles. This here
% was the best way to get one:
\def\index@prologue{\section*{Index}\addcontentsline{toc}{chapter}{Index}
}
\makeatother

%\RequirePackage[german,english,francais]{babel}

\def\matlabcolormaptext{This colormap is similar to one shipped with Matlab$^\text{\textregistered}$ under a similar name.}

\IfFileExists{tikzlibraryspy.code.tex}{%
\usetikzlibrary{spy}
}{%
    \message{ERROR: tikz SPY library NOT available. The manual will only compile partially.^^J}%
}%

\usepackage{xparse}% for colorbrewer manual

\usetikzlibrary{
    decorations.markings,
    decorations.footprints,
    shapes.arrows,
    matrix,
    positioning,
}

\usepgfplotslibrary{
    fillbetween,
    ternary,
    smithchart,
    patchplots,
    polar,
    colormaps,
    colorbrewer,
    colortol,           % docu not ready yet
}
\pgfqkeys{/codeexample}{%
    every codeexample/.append style={
        /pgfplots/every ternary axis/.append style={
            /pgfplots/legend style={fill=graphicbackground},
        }
    },
    tabsize=4,
}

\pgfplotsmanualenableexternalizationofexpensive

%\usetikzlibrary{external}
%\tikzexternalize[prefix=figures/]{pgfplots}

\title{%
    Manual for Package \PGFPlots{}\\
    {\small 2D/3D Plots in \LaTeX{}, Version \pgfplotsversion}\\
    {\small\url{https://github.com/pgf-tikz/pgfplots}}
    %\\{\small Attention: you are using an unstable development version.}
}

\makeatletter
\long\def\abstractsmuggle{%
    \centering
    \textbf{Abstract}\\[0.5cm]

    \begin{minipage}{12cm}
        \PGFPlots{} draws high-quality function plots in normal or logarithmic
        scaling with a user-friendly interface directly in \TeX{}. The user
        supplies axis labels, legend entries and the plot coordinates for one or
        more plots and \PGFPlots{} applies axis scaling, computes any logarithms
        and axis ticks and draws the plots. It supports line plots, scatter
        plots, piecewise constant plots, bar plots, area plots, mesh and surface
        plots, patch plots, contour plots, quiver plots, histogram plots, box
        plots, polar axes, ternary diagrams, smith charts and some more. It is
        based on Till Tantau's package \PGF{}/\Tikz{}.
    \end{minipage}

}%

\expandafter\date\expandafter{\@date\\[2cm]
    \abstractsmuggle
}%
\makeatother

%\includeonly{pgfplots.reference}


\begin{document}

\def\plotcoords{%
\addplot coordinates {
(5,8.312e-02)    (17,2.547e-02)   (49,7.407e-03)
(129,2.102e-03)  (321,5.874e-04)  (769,1.623e-04)
(1793,4.442e-05) (4097,1.207e-05) (9217,3.261e-06)
};

\addplot coordinates{
(7,8.472e-02)    (31,3.044e-02)    (111,1.022e-02)
(351,3.303e-03)  (1023,1.039e-03)  (2815,3.196e-04)
(7423,9.658e-05) (18943,2.873e-05) (47103,8.437e-06)
};

\addplot coordinates{
(9,7.881e-02)     (49,3.243e-02)    (209,1.232e-02)
(769,4.454e-03)   (2561,1.551e-03)  (7937,5.236e-04)
(23297,1.723e-04) (65537,5.545e-05) (178177,1.751e-05)
};

\addplot coordinates{
(11,6.887e-02)    (71,3.177e-02)     (351,1.341e-02)
(1471,5.334e-03)  (5503,2.027e-03)   (18943,7.415e-04)
(61183,2.628e-04) (187903,9.063e-05) (553983,3.053e-05)
};

\addplot coordinates{
(13,5.755e-02)     (97,2.925e-02)     (545,1.351e-02)
(2561,5.842e-03)   (10625,2.397e-03)  (40193,9.414e-04)
(141569,3.564e-04) (471041,1.308e-04)
(1496065,4.670e-05)
};
}%


\bookmaketitle
\tableofcontents
\include{pgfplots.title_abstract_intro}
\include{pgfplots.preliminaries}
\include{pgfplots.intro}
\include{pgfplots.reference}
\include{pgfplots.libs}
\include{pgfplots.resources}
\include{pgfplots.importexport}
\include{pgfplots.basic.reference}

\printindex

\bibliographystyle{abbrv} %gerapali} %gerabbrv} %gerunsrt.bst} %gerabbrv}% gerplain}
\nocite{pgfplotstable}
\nocite{programmingnotes}
\bibliography{pgfplots}
\end{document}

% -----------------------------------------------------------------------------
% For Stefan Pinnow as reminder on what to look for when editing the manual
% -----------------------------------------------------------------------------
% There should be no line breaks in the following environments
% - |...|
% - \declareandlabel{...}
% - \verbpdfref{...}
% If "MakeTikzPictures" isn't running through check one of the externalized
% LOG files what is the cause of that.
% -----------------------------------------------------------------------------


%%%%%%%%%%%%%%%%%%%%%%%%%%%%%%%%%%%%%%%%%%%%%%%%%%%%%%%%%%%%%%%%%%%%%%%%%%%%%
%
% Package pgfplots.sty documentation.
%
% Copyright 2007/2008 by Christian Feuersaenger.
%
% This program is free software: you can redistribute it and/or modify
% it under the terms of the GNU General Public License as published by
% the Free Software Foundation, either version 3 of the License, or
% (at your option) any later version.
%
% This program is distributed in the hope that it will be useful,
% but WITHOUT ANY WARRANTY; without even the implied warranty of
% MERCHANTABILITY or FITNESS FOR A PARTICULAR PURPOSE.  See the
% GNU General Public License for more details.
%
% You should have received a copy of the GNU General Public License
% along with this program.  If not, see <http://www.gnu.org/licenses/>.
%
%
%%%%%%%%%%%%%%%%%%%%%%%%%%%%%%%%%%%%%%%%%%%%%%%%%%%%%%%%%%%%%%%%%%%%%%%%%%%%%
% SEE pgfplots-macros.tex as well!
%\pdfminorversion=5 % to allow compression
%\pdfobjcompresslevel=2
\documentclass[a4paper,openany]{book}

\let\bookmaketitle=\maketitle

% -----------------------------------
% this here is from ltxdoc:
\usepackage{doc}[=v2]
\AtBeginDocument{\MakeShortVerb{\|}}
%\setlength{\textwidth}{355pt}
%\addtolength\marginparwidth{30pt}
%\addtolength\oddsidemargin{20pt}
%\addtolength\evensidemargin{20pt}
\def\cmd#1{\cs{\expandafter\cmd@to@cs\string#1}}
\def\cmd@to@cs#1#2{\char\number`#2\relax}
\DeclareRobustCommand\cs[1]{\texttt{\char`\\#1}}
\providecommand\marg[1]{%
  {\ttfamily\char`\{}\meta{#1}{\ttfamily\char`\}}}
\providecommand\oarg[1]{%
  {\ttfamily[}\meta{#1}{\ttfamily]}}
\providecommand\parg[1]{%
  {\ttfamily(}\meta{#1}{\ttfamily)}}
\raggedbottom

% -----------------------------------
\input{pgfplots.preamble.tex}

\makeatletter
% I want a two-column index just as in pgfmanual styles. This here
% was the best way to get one:
\def\index@prologue{\section*{Index}\addcontentsline{toc}{chapter}{Index}
}
\makeatother

%\RequirePackage[german,english,francais]{babel}

\def\matlabcolormaptext{This colormap is similar to one shipped with Matlab$^\text{\textregistered}$ under a similar name.}

\IfFileExists{tikzlibraryspy.code.tex}{%
\usetikzlibrary{spy}
}{%
    \message{ERROR: tikz SPY library NOT available. The manual will only compile partially.^^J}%
}%

\usepackage{xparse}% for colorbrewer manual

\usetikzlibrary{
    decorations.markings,
    decorations.footprints,
    shapes.arrows,
    matrix,
    positioning,
}

\usepgfplotslibrary{
    fillbetween,
    ternary,
    smithchart,
    patchplots,
    polar,
    colormaps,
    colorbrewer,
    colortol,           % docu not ready yet
}
\pgfqkeys{/codeexample}{%
    every codeexample/.append style={
        /pgfplots/every ternary axis/.append style={
            /pgfplots/legend style={fill=graphicbackground},
        }
    },
    tabsize=4,
}

\pgfplotsmanualenableexternalizationofexpensive

%\usetikzlibrary{external}
%\tikzexternalize[prefix=figures/]{pgfplots}

\title{%
    Manual for Package \PGFPlots{}\\
    {\small 2D/3D Plots in \LaTeX{}, Version \pgfplotsversion}\\
    {\small\url{https://github.com/pgf-tikz/pgfplots}}
    %\\{\small Attention: you are using an unstable development version.}
}

\makeatletter
\long\def\abstractsmuggle{%
    \centering
    \textbf{Abstract}\\[0.5cm]

    \begin{minipage}{12cm}
        \PGFPlots{} draws high-quality function plots in normal or logarithmic
        scaling with a user-friendly interface directly in \TeX{}. The user
        supplies axis labels, legend entries and the plot coordinates for one or
        more plots and \PGFPlots{} applies axis scaling, computes any logarithms
        and axis ticks and draws the plots. It supports line plots, scatter
        plots, piecewise constant plots, bar plots, area plots, mesh and surface
        plots, patch plots, contour plots, quiver plots, histogram plots, box
        plots, polar axes, ternary diagrams, smith charts and some more. It is
        based on Till Tantau's package \PGF{}/\Tikz{}.
    \end{minipage}

}%

\expandafter\date\expandafter{\@date\\[2cm]
    \abstractsmuggle
}%
\makeatother

%\includeonly{pgfplots.reference}


\begin{document}

\def\plotcoords{%
\addplot coordinates {
(5,8.312e-02)    (17,2.547e-02)   (49,7.407e-03)
(129,2.102e-03)  (321,5.874e-04)  (769,1.623e-04)
(1793,4.442e-05) (4097,1.207e-05) (9217,3.261e-06)
};

\addplot coordinates{
(7,8.472e-02)    (31,3.044e-02)    (111,1.022e-02)
(351,3.303e-03)  (1023,1.039e-03)  (2815,3.196e-04)
(7423,9.658e-05) (18943,2.873e-05) (47103,8.437e-06)
};

\addplot coordinates{
(9,7.881e-02)     (49,3.243e-02)    (209,1.232e-02)
(769,4.454e-03)   (2561,1.551e-03)  (7937,5.236e-04)
(23297,1.723e-04) (65537,5.545e-05) (178177,1.751e-05)
};

\addplot coordinates{
(11,6.887e-02)    (71,3.177e-02)     (351,1.341e-02)
(1471,5.334e-03)  (5503,2.027e-03)   (18943,7.415e-04)
(61183,2.628e-04) (187903,9.063e-05) (553983,3.053e-05)
};

\addplot coordinates{
(13,5.755e-02)     (97,2.925e-02)     (545,1.351e-02)
(2561,5.842e-03)   (10625,2.397e-03)  (40193,9.414e-04)
(141569,3.564e-04) (471041,1.308e-04)
(1496065,4.670e-05)
};
}%


\bookmaketitle
\tableofcontents
\include{pgfplots.title_abstract_intro}
\include{pgfplots.preliminaries}
\include{pgfplots.intro}
\include{pgfplots.reference}
\include{pgfplots.libs}
\include{pgfplots.resources}
\include{pgfplots.importexport}
\include{pgfplots.basic.reference}

\printindex

\bibliographystyle{abbrv} %gerapali} %gerabbrv} %gerunsrt.bst} %gerabbrv}% gerplain}
\nocite{pgfplotstable}
\nocite{programmingnotes}
\bibliography{pgfplots}
\end{document}

% -----------------------------------------------------------------------------
% For Stefan Pinnow as reminder on what to look for when editing the manual
% -----------------------------------------------------------------------------
% There should be no line breaks in the following environments
% - |...|
% - \declareandlabel{...}
% - \verbpdfref{...}
% If "MakeTikzPictures" isn't running through check one of the externalized
% LOG files what is the cause of that.
% -----------------------------------------------------------------------------


%%%%%%%%%%%%%%%%%%%%%%%%%%%%%%%%%%%%%%%%%%%%%%%%%%%%%%%%%%%%%%%%%%%%%%%%%%%%%
%
% Package pgfplots.sty documentation.
%
% Copyright 2007/2008 by Christian Feuersaenger.
%
% This program is free software: you can redistribute it and/or modify
% it under the terms of the GNU General Public License as published by
% the Free Software Foundation, either version 3 of the License, or
% (at your option) any later version.
%
% This program is distributed in the hope that it will be useful,
% but WITHOUT ANY WARRANTY; without even the implied warranty of
% MERCHANTABILITY or FITNESS FOR A PARTICULAR PURPOSE.  See the
% GNU General Public License for more details.
%
% You should have received a copy of the GNU General Public License
% along with this program.  If not, see <http://www.gnu.org/licenses/>.
%
%
%%%%%%%%%%%%%%%%%%%%%%%%%%%%%%%%%%%%%%%%%%%%%%%%%%%%%%%%%%%%%%%%%%%%%%%%%%%%%
% SEE pgfplots-macros.tex as well!
%\pdfminorversion=5 % to allow compression
%\pdfobjcompresslevel=2
\documentclass[a4paper,openany]{book}

\let\bookmaketitle=\maketitle

% -----------------------------------
% this here is from ltxdoc:
\usepackage{doc}[=v2]
\AtBeginDocument{\MakeShortVerb{\|}}
%\setlength{\textwidth}{355pt}
%\addtolength\marginparwidth{30pt}
%\addtolength\oddsidemargin{20pt}
%\addtolength\evensidemargin{20pt}
\def\cmd#1{\cs{\expandafter\cmd@to@cs\string#1}}
\def\cmd@to@cs#1#2{\char\number`#2\relax}
\DeclareRobustCommand\cs[1]{\texttt{\char`\\#1}}
\providecommand\marg[1]{%
  {\ttfamily\char`\{}\meta{#1}{\ttfamily\char`\}}}
\providecommand\oarg[1]{%
  {\ttfamily[}\meta{#1}{\ttfamily]}}
\providecommand\parg[1]{%
  {\ttfamily(}\meta{#1}{\ttfamily)}}
\raggedbottom

% -----------------------------------
\input{pgfplots.preamble.tex}

\makeatletter
% I want a two-column index just as in pgfmanual styles. This here
% was the best way to get one:
\def\index@prologue{\section*{Index}\addcontentsline{toc}{chapter}{Index}
}
\makeatother

%\RequirePackage[german,english,francais]{babel}

\def\matlabcolormaptext{This colormap is similar to one shipped with Matlab$^\text{\textregistered}$ under a similar name.}

\IfFileExists{tikzlibraryspy.code.tex}{%
\usetikzlibrary{spy}
}{%
    \message{ERROR: tikz SPY library NOT available. The manual will only compile partially.^^J}%
}%

\usepackage{xparse}% for colorbrewer manual

\usetikzlibrary{
    decorations.markings,
    decorations.footprints,
    shapes.arrows,
    matrix,
    positioning,
}

\usepgfplotslibrary{
    fillbetween,
    ternary,
    smithchart,
    patchplots,
    polar,
    colormaps,
    colorbrewer,
    colortol,           % docu not ready yet
}
\pgfqkeys{/codeexample}{%
    every codeexample/.append style={
        /pgfplots/every ternary axis/.append style={
            /pgfplots/legend style={fill=graphicbackground},
        }
    },
    tabsize=4,
}

\pgfplotsmanualenableexternalizationofexpensive

%\usetikzlibrary{external}
%\tikzexternalize[prefix=figures/]{pgfplots}

\title{%
    Manual for Package \PGFPlots{}\\
    {\small 2D/3D Plots in \LaTeX{}, Version \pgfplotsversion}\\
    {\small\url{https://github.com/pgf-tikz/pgfplots}}
    %\\{\small Attention: you are using an unstable development version.}
}

\makeatletter
\long\def\abstractsmuggle{%
    \centering
    \textbf{Abstract}\\[0.5cm]

    \begin{minipage}{12cm}
        \PGFPlots{} draws high-quality function plots in normal or logarithmic
        scaling with a user-friendly interface directly in \TeX{}. The user
        supplies axis labels, legend entries and the plot coordinates for one or
        more plots and \PGFPlots{} applies axis scaling, computes any logarithms
        and axis ticks and draws the plots. It supports line plots, scatter
        plots, piecewise constant plots, bar plots, area plots, mesh and surface
        plots, patch plots, contour plots, quiver plots, histogram plots, box
        plots, polar axes, ternary diagrams, smith charts and some more. It is
        based on Till Tantau's package \PGF{}/\Tikz{}.
    \end{minipage}

}%

\expandafter\date\expandafter{\@date\\[2cm]
    \abstractsmuggle
}%
\makeatother

%\includeonly{pgfplots.reference}


\begin{document}

\def\plotcoords{%
\addplot coordinates {
(5,8.312e-02)    (17,2.547e-02)   (49,7.407e-03)
(129,2.102e-03)  (321,5.874e-04)  (769,1.623e-04)
(1793,4.442e-05) (4097,1.207e-05) (9217,3.261e-06)
};

\addplot coordinates{
(7,8.472e-02)    (31,3.044e-02)    (111,1.022e-02)
(351,3.303e-03)  (1023,1.039e-03)  (2815,3.196e-04)
(7423,9.658e-05) (18943,2.873e-05) (47103,8.437e-06)
};

\addplot coordinates{
(9,7.881e-02)     (49,3.243e-02)    (209,1.232e-02)
(769,4.454e-03)   (2561,1.551e-03)  (7937,5.236e-04)
(23297,1.723e-04) (65537,5.545e-05) (178177,1.751e-05)
};

\addplot coordinates{
(11,6.887e-02)    (71,3.177e-02)     (351,1.341e-02)
(1471,5.334e-03)  (5503,2.027e-03)   (18943,7.415e-04)
(61183,2.628e-04) (187903,9.063e-05) (553983,3.053e-05)
};

\addplot coordinates{
(13,5.755e-02)     (97,2.925e-02)     (545,1.351e-02)
(2561,5.842e-03)   (10625,2.397e-03)  (40193,9.414e-04)
(141569,3.564e-04) (471041,1.308e-04)
(1496065,4.670e-05)
};
}%


\bookmaketitle
\tableofcontents
\include{pgfplots.title_abstract_intro}
\include{pgfplots.preliminaries}
\include{pgfplots.intro}
\include{pgfplots.reference}
\include{pgfplots.libs}
\include{pgfplots.resources}
\include{pgfplots.importexport}
\include{pgfplots.basic.reference}

\printindex

\bibliographystyle{abbrv} %gerapali} %gerabbrv} %gerunsrt.bst} %gerabbrv}% gerplain}
\nocite{pgfplotstable}
\nocite{programmingnotes}
\bibliography{pgfplots}
\end{document}

% -----------------------------------------------------------------------------
% For Stefan Pinnow as reminder on what to look for when editing the manual
% -----------------------------------------------------------------------------
% There should be no line breaks in the following environments
% - |...|
% - \declareandlabel{...}
% - \verbpdfref{...}
% If "MakeTikzPictures" isn't running through check one of the externalized
% LOG files what is the cause of that.
% -----------------------------------------------------------------------------

\include{pgfplots.basic.reference}

\printindex

\bibliographystyle{abbrv} %gerapali} %gerabbrv} %gerunsrt.bst} %gerabbrv}% gerplain}
\nocite{pgfplotstable}
\nocite{programmingnotes}
\bibliography{pgfplots}
\end{document}

% -----------------------------------------------------------------------------
% For Stefan Pinnow as reminder on what to look for when editing the manual
% -----------------------------------------------------------------------------
% There should be no line breaks in the following environments
% - |...|
% - \declareandlabel{...}
% - \verbpdfref{...}
% If "MakeTikzPictures" isn't running through check one of the externalized
% LOG files what is the cause of that.
% -----------------------------------------------------------------------------


%%%%%%%%%%%%%%%%%%%%%%%%%%%%%%%%%%%%%%%%%%%%%%%%%%%%%%%%%%%%%%%%%%%%%%%%%%%%%
%
% Package pgfplots.sty documentation.
%
% Copyright 2007/2008 by Christian Feuersaenger.
%
% This program is free software: you can redistribute it and/or modify
% it under the terms of the GNU General Public License as published by
% the Free Software Foundation, either version 3 of the License, or
% (at your option) any later version.
%
% This program is distributed in the hope that it will be useful,
% but WITHOUT ANY WARRANTY; without even the implied warranty of
% MERCHANTABILITY or FITNESS FOR A PARTICULAR PURPOSE.  See the
% GNU General Public License for more details.
%
% You should have received a copy of the GNU General Public License
% along with this program.  If not, see <http://www.gnu.org/licenses/>.
%
%
%%%%%%%%%%%%%%%%%%%%%%%%%%%%%%%%%%%%%%%%%%%%%%%%%%%%%%%%%%%%%%%%%%%%%%%%%%%%%
% SEE pgfplots-macros.tex as well!
%\pdfminorversion=5 % to allow compression
%\pdfobjcompresslevel=2
\documentclass[a4paper,openany]{book}

\let\bookmaketitle=\maketitle

% -----------------------------------
% this here is from ltxdoc:
\usepackage{doc}[=v2]
\AtBeginDocument{\MakeShortVerb{\|}}
%\setlength{\textwidth}{355pt}
%\addtolength\marginparwidth{30pt}
%\addtolength\oddsidemargin{20pt}
%\addtolength\evensidemargin{20pt}
\def\cmd#1{\cs{\expandafter\cmd@to@cs\string#1}}
\def\cmd@to@cs#1#2{\char\number`#2\relax}
\DeclareRobustCommand\cs[1]{\texttt{\char`\\#1}}
\providecommand\marg[1]{%
  {\ttfamily\char`\{}\meta{#1}{\ttfamily\char`\}}}
\providecommand\oarg[1]{%
  {\ttfamily[}\meta{#1}{\ttfamily]}}
\providecommand\parg[1]{%
  {\ttfamily(}\meta{#1}{\ttfamily)}}
\raggedbottom

% -----------------------------------
\input{pgfplots.preamble.tex}

\makeatletter
% I want a two-column index just as in pgfmanual styles. This here
% was the best way to get one:
\def\index@prologue{\section*{Index}\addcontentsline{toc}{chapter}{Index}
}
\makeatother

%\RequirePackage[german,english,francais]{babel}

\def\matlabcolormaptext{This colormap is similar to one shipped with Matlab$^\text{\textregistered}$ under a similar name.}

\IfFileExists{tikzlibraryspy.code.tex}{%
\usetikzlibrary{spy}
}{%
    \message{ERROR: tikz SPY library NOT available. The manual will only compile partially.^^J}%
}%

\usepackage{xparse}% for colorbrewer manual

\usetikzlibrary{
    decorations.markings,
    decorations.footprints,
    shapes.arrows,
    matrix,
    positioning,
}

\usepgfplotslibrary{
    fillbetween,
    ternary,
    smithchart,
    patchplots,
    polar,
    colormaps,
    colorbrewer,
    colortol,           % docu not ready yet
}
\pgfqkeys{/codeexample}{%
    every codeexample/.append style={
        /pgfplots/every ternary axis/.append style={
            /pgfplots/legend style={fill=graphicbackground},
        }
    },
    tabsize=4,
}

\pgfplotsmanualenableexternalizationofexpensive

%\usetikzlibrary{external}
%\tikzexternalize[prefix=figures/]{pgfplots}

\title{%
    Manual for Package \PGFPlots{}\\
    {\small 2D/3D Plots in \LaTeX{}, Version \pgfplotsversion}\\
    {\small\url{https://github.com/pgf-tikz/pgfplots}}
    %\\{\small Attention: you are using an unstable development version.}
}

\makeatletter
\long\def\abstractsmuggle{%
    \centering
    \textbf{Abstract}\\[0.5cm]

    \begin{minipage}{12cm}
        \PGFPlots{} draws high-quality function plots in normal or logarithmic
        scaling with a user-friendly interface directly in \TeX{}. The user
        supplies axis labels, legend entries and the plot coordinates for one or
        more plots and \PGFPlots{} applies axis scaling, computes any logarithms
        and axis ticks and draws the plots. It supports line plots, scatter
        plots, piecewise constant plots, bar plots, area plots, mesh and surface
        plots, patch plots, contour plots, quiver plots, histogram plots, box
        plots, polar axes, ternary diagrams, smith charts and some more. It is
        based on Till Tantau's package \PGF{}/\Tikz{}.
    \end{minipage}

}%

\expandafter\date\expandafter{\@date\\[2cm]
    \abstractsmuggle
}%
\makeatother

%\includeonly{pgfplots.reference}


\begin{document}

\def\plotcoords{%
\addplot coordinates {
(5,8.312e-02)    (17,2.547e-02)   (49,7.407e-03)
(129,2.102e-03)  (321,5.874e-04)  (769,1.623e-04)
(1793,4.442e-05) (4097,1.207e-05) (9217,3.261e-06)
};

\addplot coordinates{
(7,8.472e-02)    (31,3.044e-02)    (111,1.022e-02)
(351,3.303e-03)  (1023,1.039e-03)  (2815,3.196e-04)
(7423,9.658e-05) (18943,2.873e-05) (47103,8.437e-06)
};

\addplot coordinates{
(9,7.881e-02)     (49,3.243e-02)    (209,1.232e-02)
(769,4.454e-03)   (2561,1.551e-03)  (7937,5.236e-04)
(23297,1.723e-04) (65537,5.545e-05) (178177,1.751e-05)
};

\addplot coordinates{
(11,6.887e-02)    (71,3.177e-02)     (351,1.341e-02)
(1471,5.334e-03)  (5503,2.027e-03)   (18943,7.415e-04)
(61183,2.628e-04) (187903,9.063e-05) (553983,3.053e-05)
};

\addplot coordinates{
(13,5.755e-02)     (97,2.925e-02)     (545,1.351e-02)
(2561,5.842e-03)   (10625,2.397e-03)  (40193,9.414e-04)
(141569,3.564e-04) (471041,1.308e-04)
(1496065,4.670e-05)
};
}%


\bookmaketitle
\tableofcontents

%%%%%%%%%%%%%%%%%%%%%%%%%%%%%%%%%%%%%%%%%%%%%%%%%%%%%%%%%%%%%%%%%%%%%%%%%%%%%
%
% Package pgfplots.sty documentation.
%
% Copyright 2007/2008 by Christian Feuersaenger.
%
% This program is free software: you can redistribute it and/or modify
% it under the terms of the GNU General Public License as published by
% the Free Software Foundation, either version 3 of the License, or
% (at your option) any later version.
%
% This program is distributed in the hope that it will be useful,
% but WITHOUT ANY WARRANTY; without even the implied warranty of
% MERCHANTABILITY or FITNESS FOR A PARTICULAR PURPOSE.  See the
% GNU General Public License for more details.
%
% You should have received a copy of the GNU General Public License
% along with this program.  If not, see <http://www.gnu.org/licenses/>.
%
%
%%%%%%%%%%%%%%%%%%%%%%%%%%%%%%%%%%%%%%%%%%%%%%%%%%%%%%%%%%%%%%%%%%%%%%%%%%%%%
% SEE pgfplots-macros.tex as well!
%\pdfminorversion=5 % to allow compression
%\pdfobjcompresslevel=2
\documentclass[a4paper,openany]{book}

\let\bookmaketitle=\maketitle

% -----------------------------------
% this here is from ltxdoc:
\usepackage{doc}[=v2]
\AtBeginDocument{\MakeShortVerb{\|}}
%\setlength{\textwidth}{355pt}
%\addtolength\marginparwidth{30pt}
%\addtolength\oddsidemargin{20pt}
%\addtolength\evensidemargin{20pt}
\def\cmd#1{\cs{\expandafter\cmd@to@cs\string#1}}
\def\cmd@to@cs#1#2{\char\number`#2\relax}
\DeclareRobustCommand\cs[1]{\texttt{\char`\\#1}}
\providecommand\marg[1]{%
  {\ttfamily\char`\{}\meta{#1}{\ttfamily\char`\}}}
\providecommand\oarg[1]{%
  {\ttfamily[}\meta{#1}{\ttfamily]}}
\providecommand\parg[1]{%
  {\ttfamily(}\meta{#1}{\ttfamily)}}
\raggedbottom

% -----------------------------------
\input{pgfplots.preamble.tex}

\makeatletter
% I want a two-column index just as in pgfmanual styles. This here
% was the best way to get one:
\def\index@prologue{\section*{Index}\addcontentsline{toc}{chapter}{Index}
}
\makeatother

%\RequirePackage[german,english,francais]{babel}

\def\matlabcolormaptext{This colormap is similar to one shipped with Matlab$^\text{\textregistered}$ under a similar name.}

\IfFileExists{tikzlibraryspy.code.tex}{%
\usetikzlibrary{spy}
}{%
    \message{ERROR: tikz SPY library NOT available. The manual will only compile partially.^^J}%
}%

\usepackage{xparse}% for colorbrewer manual

\usetikzlibrary{
    decorations.markings,
    decorations.footprints,
    shapes.arrows,
    matrix,
    positioning,
}

\usepgfplotslibrary{
    fillbetween,
    ternary,
    smithchart,
    patchplots,
    polar,
    colormaps,
    colorbrewer,
    colortol,           % docu not ready yet
}
\pgfqkeys{/codeexample}{%
    every codeexample/.append style={
        /pgfplots/every ternary axis/.append style={
            /pgfplots/legend style={fill=graphicbackground},
        }
    },
    tabsize=4,
}

\pgfplotsmanualenableexternalizationofexpensive

%\usetikzlibrary{external}
%\tikzexternalize[prefix=figures/]{pgfplots}

\title{%
    Manual for Package \PGFPlots{}\\
    {\small 2D/3D Plots in \LaTeX{}, Version \pgfplotsversion}\\
    {\small\url{https://github.com/pgf-tikz/pgfplots}}
    %\\{\small Attention: you are using an unstable development version.}
}

\makeatletter
\long\def\abstractsmuggle{%
    \centering
    \textbf{Abstract}\\[0.5cm]

    \begin{minipage}{12cm}
        \PGFPlots{} draws high-quality function plots in normal or logarithmic
        scaling with a user-friendly interface directly in \TeX{}. The user
        supplies axis labels, legend entries and the plot coordinates for one or
        more plots and \PGFPlots{} applies axis scaling, computes any logarithms
        and axis ticks and draws the plots. It supports line plots, scatter
        plots, piecewise constant plots, bar plots, area plots, mesh and surface
        plots, patch plots, contour plots, quiver plots, histogram plots, box
        plots, polar axes, ternary diagrams, smith charts and some more. It is
        based on Till Tantau's package \PGF{}/\Tikz{}.
    \end{minipage}

}%

\expandafter\date\expandafter{\@date\\[2cm]
    \abstractsmuggle
}%
\makeatother

%\includeonly{pgfplots.reference}


\begin{document}

\def\plotcoords{%
\addplot coordinates {
(5,8.312e-02)    (17,2.547e-02)   (49,7.407e-03)
(129,2.102e-03)  (321,5.874e-04)  (769,1.623e-04)
(1793,4.442e-05) (4097,1.207e-05) (9217,3.261e-06)
};

\addplot coordinates{
(7,8.472e-02)    (31,3.044e-02)    (111,1.022e-02)
(351,3.303e-03)  (1023,1.039e-03)  (2815,3.196e-04)
(7423,9.658e-05) (18943,2.873e-05) (47103,8.437e-06)
};

\addplot coordinates{
(9,7.881e-02)     (49,3.243e-02)    (209,1.232e-02)
(769,4.454e-03)   (2561,1.551e-03)  (7937,5.236e-04)
(23297,1.723e-04) (65537,5.545e-05) (178177,1.751e-05)
};

\addplot coordinates{
(11,6.887e-02)    (71,3.177e-02)     (351,1.341e-02)
(1471,5.334e-03)  (5503,2.027e-03)   (18943,7.415e-04)
(61183,2.628e-04) (187903,9.063e-05) (553983,3.053e-05)
};

\addplot coordinates{
(13,5.755e-02)     (97,2.925e-02)     (545,1.351e-02)
(2561,5.842e-03)   (10625,2.397e-03)  (40193,9.414e-04)
(141569,3.564e-04) (471041,1.308e-04)
(1496065,4.670e-05)
};
}%


\bookmaketitle
\tableofcontents
\include{pgfplots.title_abstract_intro}
\include{pgfplots.preliminaries}
\include{pgfplots.intro}
\include{pgfplots.reference}
\include{pgfplots.libs}
\include{pgfplots.resources}
\include{pgfplots.importexport}
\include{pgfplots.basic.reference}

\printindex

\bibliographystyle{abbrv} %gerapali} %gerabbrv} %gerunsrt.bst} %gerabbrv}% gerplain}
\nocite{pgfplotstable}
\nocite{programmingnotes}
\bibliography{pgfplots}
\end{document}

% -----------------------------------------------------------------------------
% For Stefan Pinnow as reminder on what to look for when editing the manual
% -----------------------------------------------------------------------------
% There should be no line breaks in the following environments
% - |...|
% - \declareandlabel{...}
% - \verbpdfref{...}
% If "MakeTikzPictures" isn't running through check one of the externalized
% LOG files what is the cause of that.
% -----------------------------------------------------------------------------


%%%%%%%%%%%%%%%%%%%%%%%%%%%%%%%%%%%%%%%%%%%%%%%%%%%%%%%%%%%%%%%%%%%%%%%%%%%%%
%
% Package pgfplots.sty documentation.
%
% Copyright 2007/2008 by Christian Feuersaenger.
%
% This program is free software: you can redistribute it and/or modify
% it under the terms of the GNU General Public License as published by
% the Free Software Foundation, either version 3 of the License, or
% (at your option) any later version.
%
% This program is distributed in the hope that it will be useful,
% but WITHOUT ANY WARRANTY; without even the implied warranty of
% MERCHANTABILITY or FITNESS FOR A PARTICULAR PURPOSE.  See the
% GNU General Public License for more details.
%
% You should have received a copy of the GNU General Public License
% along with this program.  If not, see <http://www.gnu.org/licenses/>.
%
%
%%%%%%%%%%%%%%%%%%%%%%%%%%%%%%%%%%%%%%%%%%%%%%%%%%%%%%%%%%%%%%%%%%%%%%%%%%%%%
% SEE pgfplots-macros.tex as well!
%\pdfminorversion=5 % to allow compression
%\pdfobjcompresslevel=2
\documentclass[a4paper,openany]{book}

\let\bookmaketitle=\maketitle

% -----------------------------------
% this here is from ltxdoc:
\usepackage{doc}[=v2]
\AtBeginDocument{\MakeShortVerb{\|}}
%\setlength{\textwidth}{355pt}
%\addtolength\marginparwidth{30pt}
%\addtolength\oddsidemargin{20pt}
%\addtolength\evensidemargin{20pt}
\def\cmd#1{\cs{\expandafter\cmd@to@cs\string#1}}
\def\cmd@to@cs#1#2{\char\number`#2\relax}
\DeclareRobustCommand\cs[1]{\texttt{\char`\\#1}}
\providecommand\marg[1]{%
  {\ttfamily\char`\{}\meta{#1}{\ttfamily\char`\}}}
\providecommand\oarg[1]{%
  {\ttfamily[}\meta{#1}{\ttfamily]}}
\providecommand\parg[1]{%
  {\ttfamily(}\meta{#1}{\ttfamily)}}
\raggedbottom

% -----------------------------------
\input{pgfplots.preamble.tex}

\makeatletter
% I want a two-column index just as in pgfmanual styles. This here
% was the best way to get one:
\def\index@prologue{\section*{Index}\addcontentsline{toc}{chapter}{Index}
}
\makeatother

%\RequirePackage[german,english,francais]{babel}

\def\matlabcolormaptext{This colormap is similar to one shipped with Matlab$^\text{\textregistered}$ under a similar name.}

\IfFileExists{tikzlibraryspy.code.tex}{%
\usetikzlibrary{spy}
}{%
    \message{ERROR: tikz SPY library NOT available. The manual will only compile partially.^^J}%
}%

\usepackage{xparse}% for colorbrewer manual

\usetikzlibrary{
    decorations.markings,
    decorations.footprints,
    shapes.arrows,
    matrix,
    positioning,
}

\usepgfplotslibrary{
    fillbetween,
    ternary,
    smithchart,
    patchplots,
    polar,
    colormaps,
    colorbrewer,
    colortol,           % docu not ready yet
}
\pgfqkeys{/codeexample}{%
    every codeexample/.append style={
        /pgfplots/every ternary axis/.append style={
            /pgfplots/legend style={fill=graphicbackground},
        }
    },
    tabsize=4,
}

\pgfplotsmanualenableexternalizationofexpensive

%\usetikzlibrary{external}
%\tikzexternalize[prefix=figures/]{pgfplots}

\title{%
    Manual for Package \PGFPlots{}\\
    {\small 2D/3D Plots in \LaTeX{}, Version \pgfplotsversion}\\
    {\small\url{https://github.com/pgf-tikz/pgfplots}}
    %\\{\small Attention: you are using an unstable development version.}
}

\makeatletter
\long\def\abstractsmuggle{%
    \centering
    \textbf{Abstract}\\[0.5cm]

    \begin{minipage}{12cm}
        \PGFPlots{} draws high-quality function plots in normal or logarithmic
        scaling with a user-friendly interface directly in \TeX{}. The user
        supplies axis labels, legend entries and the plot coordinates for one or
        more plots and \PGFPlots{} applies axis scaling, computes any logarithms
        and axis ticks and draws the plots. It supports line plots, scatter
        plots, piecewise constant plots, bar plots, area plots, mesh and surface
        plots, patch plots, contour plots, quiver plots, histogram plots, box
        plots, polar axes, ternary diagrams, smith charts and some more. It is
        based on Till Tantau's package \PGF{}/\Tikz{}.
    \end{minipage}

}%

\expandafter\date\expandafter{\@date\\[2cm]
    \abstractsmuggle
}%
\makeatother

%\includeonly{pgfplots.reference}


\begin{document}

\def\plotcoords{%
\addplot coordinates {
(5,8.312e-02)    (17,2.547e-02)   (49,7.407e-03)
(129,2.102e-03)  (321,5.874e-04)  (769,1.623e-04)
(1793,4.442e-05) (4097,1.207e-05) (9217,3.261e-06)
};

\addplot coordinates{
(7,8.472e-02)    (31,3.044e-02)    (111,1.022e-02)
(351,3.303e-03)  (1023,1.039e-03)  (2815,3.196e-04)
(7423,9.658e-05) (18943,2.873e-05) (47103,8.437e-06)
};

\addplot coordinates{
(9,7.881e-02)     (49,3.243e-02)    (209,1.232e-02)
(769,4.454e-03)   (2561,1.551e-03)  (7937,5.236e-04)
(23297,1.723e-04) (65537,5.545e-05) (178177,1.751e-05)
};

\addplot coordinates{
(11,6.887e-02)    (71,3.177e-02)     (351,1.341e-02)
(1471,5.334e-03)  (5503,2.027e-03)   (18943,7.415e-04)
(61183,2.628e-04) (187903,9.063e-05) (553983,3.053e-05)
};

\addplot coordinates{
(13,5.755e-02)     (97,2.925e-02)     (545,1.351e-02)
(2561,5.842e-03)   (10625,2.397e-03)  (40193,9.414e-04)
(141569,3.564e-04) (471041,1.308e-04)
(1496065,4.670e-05)
};
}%


\bookmaketitle
\tableofcontents
\include{pgfplots.title_abstract_intro}
\include{pgfplots.preliminaries}
\include{pgfplots.intro}
\include{pgfplots.reference}
\include{pgfplots.libs}
\include{pgfplots.resources}
\include{pgfplots.importexport}
\include{pgfplots.basic.reference}

\printindex

\bibliographystyle{abbrv} %gerapali} %gerabbrv} %gerunsrt.bst} %gerabbrv}% gerplain}
\nocite{pgfplotstable}
\nocite{programmingnotes}
\bibliography{pgfplots}
\end{document}

% -----------------------------------------------------------------------------
% For Stefan Pinnow as reminder on what to look for when editing the manual
% -----------------------------------------------------------------------------
% There should be no line breaks in the following environments
% - |...|
% - \declareandlabel{...}
% - \verbpdfref{...}
% If "MakeTikzPictures" isn't running through check one of the externalized
% LOG files what is the cause of that.
% -----------------------------------------------------------------------------


%%%%%%%%%%%%%%%%%%%%%%%%%%%%%%%%%%%%%%%%%%%%%%%%%%%%%%%%%%%%%%%%%%%%%%%%%%%%%
%
% Package pgfplots.sty documentation.
%
% Copyright 2007/2008 by Christian Feuersaenger.
%
% This program is free software: you can redistribute it and/or modify
% it under the terms of the GNU General Public License as published by
% the Free Software Foundation, either version 3 of the License, or
% (at your option) any later version.
%
% This program is distributed in the hope that it will be useful,
% but WITHOUT ANY WARRANTY; without even the implied warranty of
% MERCHANTABILITY or FITNESS FOR A PARTICULAR PURPOSE.  See the
% GNU General Public License for more details.
%
% You should have received a copy of the GNU General Public License
% along with this program.  If not, see <http://www.gnu.org/licenses/>.
%
%
%%%%%%%%%%%%%%%%%%%%%%%%%%%%%%%%%%%%%%%%%%%%%%%%%%%%%%%%%%%%%%%%%%%%%%%%%%%%%
% SEE pgfplots-macros.tex as well!
%\pdfminorversion=5 % to allow compression
%\pdfobjcompresslevel=2
\documentclass[a4paper,openany]{book}

\let\bookmaketitle=\maketitle

% -----------------------------------
% this here is from ltxdoc:
\usepackage{doc}[=v2]
\AtBeginDocument{\MakeShortVerb{\|}}
%\setlength{\textwidth}{355pt}
%\addtolength\marginparwidth{30pt}
%\addtolength\oddsidemargin{20pt}
%\addtolength\evensidemargin{20pt}
\def\cmd#1{\cs{\expandafter\cmd@to@cs\string#1}}
\def\cmd@to@cs#1#2{\char\number`#2\relax}
\DeclareRobustCommand\cs[1]{\texttt{\char`\\#1}}
\providecommand\marg[1]{%
  {\ttfamily\char`\{}\meta{#1}{\ttfamily\char`\}}}
\providecommand\oarg[1]{%
  {\ttfamily[}\meta{#1}{\ttfamily]}}
\providecommand\parg[1]{%
  {\ttfamily(}\meta{#1}{\ttfamily)}}
\raggedbottom

% -----------------------------------
\input{pgfplots.preamble.tex}

\makeatletter
% I want a two-column index just as in pgfmanual styles. This here
% was the best way to get one:
\def\index@prologue{\section*{Index}\addcontentsline{toc}{chapter}{Index}
}
\makeatother

%\RequirePackage[german,english,francais]{babel}

\def\matlabcolormaptext{This colormap is similar to one shipped with Matlab$^\text{\textregistered}$ under a similar name.}

\IfFileExists{tikzlibraryspy.code.tex}{%
\usetikzlibrary{spy}
}{%
    \message{ERROR: tikz SPY library NOT available. The manual will only compile partially.^^J}%
}%

\usepackage{xparse}% for colorbrewer manual

\usetikzlibrary{
    decorations.markings,
    decorations.footprints,
    shapes.arrows,
    matrix,
    positioning,
}

\usepgfplotslibrary{
    fillbetween,
    ternary,
    smithchart,
    patchplots,
    polar,
    colormaps,
    colorbrewer,
    colortol,           % docu not ready yet
}
\pgfqkeys{/codeexample}{%
    every codeexample/.append style={
        /pgfplots/every ternary axis/.append style={
            /pgfplots/legend style={fill=graphicbackground},
        }
    },
    tabsize=4,
}

\pgfplotsmanualenableexternalizationofexpensive

%\usetikzlibrary{external}
%\tikzexternalize[prefix=figures/]{pgfplots}

\title{%
    Manual for Package \PGFPlots{}\\
    {\small 2D/3D Plots in \LaTeX{}, Version \pgfplotsversion}\\
    {\small\url{https://github.com/pgf-tikz/pgfplots}}
    %\\{\small Attention: you are using an unstable development version.}
}

\makeatletter
\long\def\abstractsmuggle{%
    \centering
    \textbf{Abstract}\\[0.5cm]

    \begin{minipage}{12cm}
        \PGFPlots{} draws high-quality function plots in normal or logarithmic
        scaling with a user-friendly interface directly in \TeX{}. The user
        supplies axis labels, legend entries and the plot coordinates for one or
        more plots and \PGFPlots{} applies axis scaling, computes any logarithms
        and axis ticks and draws the plots. It supports line plots, scatter
        plots, piecewise constant plots, bar plots, area plots, mesh and surface
        plots, patch plots, contour plots, quiver plots, histogram plots, box
        plots, polar axes, ternary diagrams, smith charts and some more. It is
        based on Till Tantau's package \PGF{}/\Tikz{}.
    \end{minipage}

}%

\expandafter\date\expandafter{\@date\\[2cm]
    \abstractsmuggle
}%
\makeatother

%\includeonly{pgfplots.reference}


\begin{document}

\def\plotcoords{%
\addplot coordinates {
(5,8.312e-02)    (17,2.547e-02)   (49,7.407e-03)
(129,2.102e-03)  (321,5.874e-04)  (769,1.623e-04)
(1793,4.442e-05) (4097,1.207e-05) (9217,3.261e-06)
};

\addplot coordinates{
(7,8.472e-02)    (31,3.044e-02)    (111,1.022e-02)
(351,3.303e-03)  (1023,1.039e-03)  (2815,3.196e-04)
(7423,9.658e-05) (18943,2.873e-05) (47103,8.437e-06)
};

\addplot coordinates{
(9,7.881e-02)     (49,3.243e-02)    (209,1.232e-02)
(769,4.454e-03)   (2561,1.551e-03)  (7937,5.236e-04)
(23297,1.723e-04) (65537,5.545e-05) (178177,1.751e-05)
};

\addplot coordinates{
(11,6.887e-02)    (71,3.177e-02)     (351,1.341e-02)
(1471,5.334e-03)  (5503,2.027e-03)   (18943,7.415e-04)
(61183,2.628e-04) (187903,9.063e-05) (553983,3.053e-05)
};

\addplot coordinates{
(13,5.755e-02)     (97,2.925e-02)     (545,1.351e-02)
(2561,5.842e-03)   (10625,2.397e-03)  (40193,9.414e-04)
(141569,3.564e-04) (471041,1.308e-04)
(1496065,4.670e-05)
};
}%


\bookmaketitle
\tableofcontents
\include{pgfplots.title_abstract_intro}
\include{pgfplots.preliminaries}
\include{pgfplots.intro}
\include{pgfplots.reference}
\include{pgfplots.libs}
\include{pgfplots.resources}
\include{pgfplots.importexport}
\include{pgfplots.basic.reference}

\printindex

\bibliographystyle{abbrv} %gerapali} %gerabbrv} %gerunsrt.bst} %gerabbrv}% gerplain}
\nocite{pgfplotstable}
\nocite{programmingnotes}
\bibliography{pgfplots}
\end{document}

% -----------------------------------------------------------------------------
% For Stefan Pinnow as reminder on what to look for when editing the manual
% -----------------------------------------------------------------------------
% There should be no line breaks in the following environments
% - |...|
% - \declareandlabel{...}
% - \verbpdfref{...}
% If "MakeTikzPictures" isn't running through check one of the externalized
% LOG files what is the cause of that.
% -----------------------------------------------------------------------------


%%%%%%%%%%%%%%%%%%%%%%%%%%%%%%%%%%%%%%%%%%%%%%%%%%%%%%%%%%%%%%%%%%%%%%%%%%%%%
%
% Package pgfplots.sty documentation.
%
% Copyright 2007/2008 by Christian Feuersaenger.
%
% This program is free software: you can redistribute it and/or modify
% it under the terms of the GNU General Public License as published by
% the Free Software Foundation, either version 3 of the License, or
% (at your option) any later version.
%
% This program is distributed in the hope that it will be useful,
% but WITHOUT ANY WARRANTY; without even the implied warranty of
% MERCHANTABILITY or FITNESS FOR A PARTICULAR PURPOSE.  See the
% GNU General Public License for more details.
%
% You should have received a copy of the GNU General Public License
% along with this program.  If not, see <http://www.gnu.org/licenses/>.
%
%
%%%%%%%%%%%%%%%%%%%%%%%%%%%%%%%%%%%%%%%%%%%%%%%%%%%%%%%%%%%%%%%%%%%%%%%%%%%%%
% SEE pgfplots-macros.tex as well!
%\pdfminorversion=5 % to allow compression
%\pdfobjcompresslevel=2
\documentclass[a4paper,openany]{book}

\let\bookmaketitle=\maketitle

% -----------------------------------
% this here is from ltxdoc:
\usepackage{doc}[=v2]
\AtBeginDocument{\MakeShortVerb{\|}}
%\setlength{\textwidth}{355pt}
%\addtolength\marginparwidth{30pt}
%\addtolength\oddsidemargin{20pt}
%\addtolength\evensidemargin{20pt}
\def\cmd#1{\cs{\expandafter\cmd@to@cs\string#1}}
\def\cmd@to@cs#1#2{\char\number`#2\relax}
\DeclareRobustCommand\cs[1]{\texttt{\char`\\#1}}
\providecommand\marg[1]{%
  {\ttfamily\char`\{}\meta{#1}{\ttfamily\char`\}}}
\providecommand\oarg[1]{%
  {\ttfamily[}\meta{#1}{\ttfamily]}}
\providecommand\parg[1]{%
  {\ttfamily(}\meta{#1}{\ttfamily)}}
\raggedbottom

% -----------------------------------
\input{pgfplots.preamble.tex}

\makeatletter
% I want a two-column index just as in pgfmanual styles. This here
% was the best way to get one:
\def\index@prologue{\section*{Index}\addcontentsline{toc}{chapter}{Index}
}
\makeatother

%\RequirePackage[german,english,francais]{babel}

\def\matlabcolormaptext{This colormap is similar to one shipped with Matlab$^\text{\textregistered}$ under a similar name.}

\IfFileExists{tikzlibraryspy.code.tex}{%
\usetikzlibrary{spy}
}{%
    \message{ERROR: tikz SPY library NOT available. The manual will only compile partially.^^J}%
}%

\usepackage{xparse}% for colorbrewer manual

\usetikzlibrary{
    decorations.markings,
    decorations.footprints,
    shapes.arrows,
    matrix,
    positioning,
}

\usepgfplotslibrary{
    fillbetween,
    ternary,
    smithchart,
    patchplots,
    polar,
    colormaps,
    colorbrewer,
    colortol,           % docu not ready yet
}
\pgfqkeys{/codeexample}{%
    every codeexample/.append style={
        /pgfplots/every ternary axis/.append style={
            /pgfplots/legend style={fill=graphicbackground},
        }
    },
    tabsize=4,
}

\pgfplotsmanualenableexternalizationofexpensive

%\usetikzlibrary{external}
%\tikzexternalize[prefix=figures/]{pgfplots}

\title{%
    Manual for Package \PGFPlots{}\\
    {\small 2D/3D Plots in \LaTeX{}, Version \pgfplotsversion}\\
    {\small\url{https://github.com/pgf-tikz/pgfplots}}
    %\\{\small Attention: you are using an unstable development version.}
}

\makeatletter
\long\def\abstractsmuggle{%
    \centering
    \textbf{Abstract}\\[0.5cm]

    \begin{minipage}{12cm}
        \PGFPlots{} draws high-quality function plots in normal or logarithmic
        scaling with a user-friendly interface directly in \TeX{}. The user
        supplies axis labels, legend entries and the plot coordinates for one or
        more plots and \PGFPlots{} applies axis scaling, computes any logarithms
        and axis ticks and draws the plots. It supports line plots, scatter
        plots, piecewise constant plots, bar plots, area plots, mesh and surface
        plots, patch plots, contour plots, quiver plots, histogram plots, box
        plots, polar axes, ternary diagrams, smith charts and some more. It is
        based on Till Tantau's package \PGF{}/\Tikz{}.
    \end{minipage}

}%

\expandafter\date\expandafter{\@date\\[2cm]
    \abstractsmuggle
}%
\makeatother

%\includeonly{pgfplots.reference}


\begin{document}

\def\plotcoords{%
\addplot coordinates {
(5,8.312e-02)    (17,2.547e-02)   (49,7.407e-03)
(129,2.102e-03)  (321,5.874e-04)  (769,1.623e-04)
(1793,4.442e-05) (4097,1.207e-05) (9217,3.261e-06)
};

\addplot coordinates{
(7,8.472e-02)    (31,3.044e-02)    (111,1.022e-02)
(351,3.303e-03)  (1023,1.039e-03)  (2815,3.196e-04)
(7423,9.658e-05) (18943,2.873e-05) (47103,8.437e-06)
};

\addplot coordinates{
(9,7.881e-02)     (49,3.243e-02)    (209,1.232e-02)
(769,4.454e-03)   (2561,1.551e-03)  (7937,5.236e-04)
(23297,1.723e-04) (65537,5.545e-05) (178177,1.751e-05)
};

\addplot coordinates{
(11,6.887e-02)    (71,3.177e-02)     (351,1.341e-02)
(1471,5.334e-03)  (5503,2.027e-03)   (18943,7.415e-04)
(61183,2.628e-04) (187903,9.063e-05) (553983,3.053e-05)
};

\addplot coordinates{
(13,5.755e-02)     (97,2.925e-02)     (545,1.351e-02)
(2561,5.842e-03)   (10625,2.397e-03)  (40193,9.414e-04)
(141569,3.564e-04) (471041,1.308e-04)
(1496065,4.670e-05)
};
}%


\bookmaketitle
\tableofcontents
\include{pgfplots.title_abstract_intro}
\include{pgfplots.preliminaries}
\include{pgfplots.intro}
\include{pgfplots.reference}
\include{pgfplots.libs}
\include{pgfplots.resources}
\include{pgfplots.importexport}
\include{pgfplots.basic.reference}

\printindex

\bibliographystyle{abbrv} %gerapali} %gerabbrv} %gerunsrt.bst} %gerabbrv}% gerplain}
\nocite{pgfplotstable}
\nocite{programmingnotes}
\bibliography{pgfplots}
\end{document}

% -----------------------------------------------------------------------------
% For Stefan Pinnow as reminder on what to look for when editing the manual
% -----------------------------------------------------------------------------
% There should be no line breaks in the following environments
% - |...|
% - \declareandlabel{...}
% - \verbpdfref{...}
% If "MakeTikzPictures" isn't running through check one of the externalized
% LOG files what is the cause of that.
% -----------------------------------------------------------------------------


%%%%%%%%%%%%%%%%%%%%%%%%%%%%%%%%%%%%%%%%%%%%%%%%%%%%%%%%%%%%%%%%%%%%%%%%%%%%%
%
% Package pgfplots.sty documentation.
%
% Copyright 2007/2008 by Christian Feuersaenger.
%
% This program is free software: you can redistribute it and/or modify
% it under the terms of the GNU General Public License as published by
% the Free Software Foundation, either version 3 of the License, or
% (at your option) any later version.
%
% This program is distributed in the hope that it will be useful,
% but WITHOUT ANY WARRANTY; without even the implied warranty of
% MERCHANTABILITY or FITNESS FOR A PARTICULAR PURPOSE.  See the
% GNU General Public License for more details.
%
% You should have received a copy of the GNU General Public License
% along with this program.  If not, see <http://www.gnu.org/licenses/>.
%
%
%%%%%%%%%%%%%%%%%%%%%%%%%%%%%%%%%%%%%%%%%%%%%%%%%%%%%%%%%%%%%%%%%%%%%%%%%%%%%
% SEE pgfplots-macros.tex as well!
%\pdfminorversion=5 % to allow compression
%\pdfobjcompresslevel=2
\documentclass[a4paper,openany]{book}

\let\bookmaketitle=\maketitle

% -----------------------------------
% this here is from ltxdoc:
\usepackage{doc}[=v2]
\AtBeginDocument{\MakeShortVerb{\|}}
%\setlength{\textwidth}{355pt}
%\addtolength\marginparwidth{30pt}
%\addtolength\oddsidemargin{20pt}
%\addtolength\evensidemargin{20pt}
\def\cmd#1{\cs{\expandafter\cmd@to@cs\string#1}}
\def\cmd@to@cs#1#2{\char\number`#2\relax}
\DeclareRobustCommand\cs[1]{\texttt{\char`\\#1}}
\providecommand\marg[1]{%
  {\ttfamily\char`\{}\meta{#1}{\ttfamily\char`\}}}
\providecommand\oarg[1]{%
  {\ttfamily[}\meta{#1}{\ttfamily]}}
\providecommand\parg[1]{%
  {\ttfamily(}\meta{#1}{\ttfamily)}}
\raggedbottom

% -----------------------------------
\input{pgfplots.preamble.tex}

\makeatletter
% I want a two-column index just as in pgfmanual styles. This here
% was the best way to get one:
\def\index@prologue{\section*{Index}\addcontentsline{toc}{chapter}{Index}
}
\makeatother

%\RequirePackage[german,english,francais]{babel}

\def\matlabcolormaptext{This colormap is similar to one shipped with Matlab$^\text{\textregistered}$ under a similar name.}

\IfFileExists{tikzlibraryspy.code.tex}{%
\usetikzlibrary{spy}
}{%
    \message{ERROR: tikz SPY library NOT available. The manual will only compile partially.^^J}%
}%

\usepackage{xparse}% for colorbrewer manual

\usetikzlibrary{
    decorations.markings,
    decorations.footprints,
    shapes.arrows,
    matrix,
    positioning,
}

\usepgfplotslibrary{
    fillbetween,
    ternary,
    smithchart,
    patchplots,
    polar,
    colormaps,
    colorbrewer,
    colortol,           % docu not ready yet
}
\pgfqkeys{/codeexample}{%
    every codeexample/.append style={
        /pgfplots/every ternary axis/.append style={
            /pgfplots/legend style={fill=graphicbackground},
        }
    },
    tabsize=4,
}

\pgfplotsmanualenableexternalizationofexpensive

%\usetikzlibrary{external}
%\tikzexternalize[prefix=figures/]{pgfplots}

\title{%
    Manual for Package \PGFPlots{}\\
    {\small 2D/3D Plots in \LaTeX{}, Version \pgfplotsversion}\\
    {\small\url{https://github.com/pgf-tikz/pgfplots}}
    %\\{\small Attention: you are using an unstable development version.}
}

\makeatletter
\long\def\abstractsmuggle{%
    \centering
    \textbf{Abstract}\\[0.5cm]

    \begin{minipage}{12cm}
        \PGFPlots{} draws high-quality function plots in normal or logarithmic
        scaling with a user-friendly interface directly in \TeX{}. The user
        supplies axis labels, legend entries and the plot coordinates for one or
        more plots and \PGFPlots{} applies axis scaling, computes any logarithms
        and axis ticks and draws the plots. It supports line plots, scatter
        plots, piecewise constant plots, bar plots, area plots, mesh and surface
        plots, patch plots, contour plots, quiver plots, histogram plots, box
        plots, polar axes, ternary diagrams, smith charts and some more. It is
        based on Till Tantau's package \PGF{}/\Tikz{}.
    \end{minipage}

}%

\expandafter\date\expandafter{\@date\\[2cm]
    \abstractsmuggle
}%
\makeatother

%\includeonly{pgfplots.reference}


\begin{document}

\def\plotcoords{%
\addplot coordinates {
(5,8.312e-02)    (17,2.547e-02)   (49,7.407e-03)
(129,2.102e-03)  (321,5.874e-04)  (769,1.623e-04)
(1793,4.442e-05) (4097,1.207e-05) (9217,3.261e-06)
};

\addplot coordinates{
(7,8.472e-02)    (31,3.044e-02)    (111,1.022e-02)
(351,3.303e-03)  (1023,1.039e-03)  (2815,3.196e-04)
(7423,9.658e-05) (18943,2.873e-05) (47103,8.437e-06)
};

\addplot coordinates{
(9,7.881e-02)     (49,3.243e-02)    (209,1.232e-02)
(769,4.454e-03)   (2561,1.551e-03)  (7937,5.236e-04)
(23297,1.723e-04) (65537,5.545e-05) (178177,1.751e-05)
};

\addplot coordinates{
(11,6.887e-02)    (71,3.177e-02)     (351,1.341e-02)
(1471,5.334e-03)  (5503,2.027e-03)   (18943,7.415e-04)
(61183,2.628e-04) (187903,9.063e-05) (553983,3.053e-05)
};

\addplot coordinates{
(13,5.755e-02)     (97,2.925e-02)     (545,1.351e-02)
(2561,5.842e-03)   (10625,2.397e-03)  (40193,9.414e-04)
(141569,3.564e-04) (471041,1.308e-04)
(1496065,4.670e-05)
};
}%


\bookmaketitle
\tableofcontents
\include{pgfplots.title_abstract_intro}
\include{pgfplots.preliminaries}
\include{pgfplots.intro}
\include{pgfplots.reference}
\include{pgfplots.libs}
\include{pgfplots.resources}
\include{pgfplots.importexport}
\include{pgfplots.basic.reference}

\printindex

\bibliographystyle{abbrv} %gerapali} %gerabbrv} %gerunsrt.bst} %gerabbrv}% gerplain}
\nocite{pgfplotstable}
\nocite{programmingnotes}
\bibliography{pgfplots}
\end{document}

% -----------------------------------------------------------------------------
% For Stefan Pinnow as reminder on what to look for when editing the manual
% -----------------------------------------------------------------------------
% There should be no line breaks in the following environments
% - |...|
% - \declareandlabel{...}
% - \verbpdfref{...}
% If "MakeTikzPictures" isn't running through check one of the externalized
% LOG files what is the cause of that.
% -----------------------------------------------------------------------------


%%%%%%%%%%%%%%%%%%%%%%%%%%%%%%%%%%%%%%%%%%%%%%%%%%%%%%%%%%%%%%%%%%%%%%%%%%%%%
%
% Package pgfplots.sty documentation.
%
% Copyright 2007/2008 by Christian Feuersaenger.
%
% This program is free software: you can redistribute it and/or modify
% it under the terms of the GNU General Public License as published by
% the Free Software Foundation, either version 3 of the License, or
% (at your option) any later version.
%
% This program is distributed in the hope that it will be useful,
% but WITHOUT ANY WARRANTY; without even the implied warranty of
% MERCHANTABILITY or FITNESS FOR A PARTICULAR PURPOSE.  See the
% GNU General Public License for more details.
%
% You should have received a copy of the GNU General Public License
% along with this program.  If not, see <http://www.gnu.org/licenses/>.
%
%
%%%%%%%%%%%%%%%%%%%%%%%%%%%%%%%%%%%%%%%%%%%%%%%%%%%%%%%%%%%%%%%%%%%%%%%%%%%%%
% SEE pgfplots-macros.tex as well!
%\pdfminorversion=5 % to allow compression
%\pdfobjcompresslevel=2
\documentclass[a4paper,openany]{book}

\let\bookmaketitle=\maketitle

% -----------------------------------
% this here is from ltxdoc:
\usepackage{doc}[=v2]
\AtBeginDocument{\MakeShortVerb{\|}}
%\setlength{\textwidth}{355pt}
%\addtolength\marginparwidth{30pt}
%\addtolength\oddsidemargin{20pt}
%\addtolength\evensidemargin{20pt}
\def\cmd#1{\cs{\expandafter\cmd@to@cs\string#1}}
\def\cmd@to@cs#1#2{\char\number`#2\relax}
\DeclareRobustCommand\cs[1]{\texttt{\char`\\#1}}
\providecommand\marg[1]{%
  {\ttfamily\char`\{}\meta{#1}{\ttfamily\char`\}}}
\providecommand\oarg[1]{%
  {\ttfamily[}\meta{#1}{\ttfamily]}}
\providecommand\parg[1]{%
  {\ttfamily(}\meta{#1}{\ttfamily)}}
\raggedbottom

% -----------------------------------
\input{pgfplots.preamble.tex}

\makeatletter
% I want a two-column index just as in pgfmanual styles. This here
% was the best way to get one:
\def\index@prologue{\section*{Index}\addcontentsline{toc}{chapter}{Index}
}
\makeatother

%\RequirePackage[german,english,francais]{babel}

\def\matlabcolormaptext{This colormap is similar to one shipped with Matlab$^\text{\textregistered}$ under a similar name.}

\IfFileExists{tikzlibraryspy.code.tex}{%
\usetikzlibrary{spy}
}{%
    \message{ERROR: tikz SPY library NOT available. The manual will only compile partially.^^J}%
}%

\usepackage{xparse}% for colorbrewer manual

\usetikzlibrary{
    decorations.markings,
    decorations.footprints,
    shapes.arrows,
    matrix,
    positioning,
}

\usepgfplotslibrary{
    fillbetween,
    ternary,
    smithchart,
    patchplots,
    polar,
    colormaps,
    colorbrewer,
    colortol,           % docu not ready yet
}
\pgfqkeys{/codeexample}{%
    every codeexample/.append style={
        /pgfplots/every ternary axis/.append style={
            /pgfplots/legend style={fill=graphicbackground},
        }
    },
    tabsize=4,
}

\pgfplotsmanualenableexternalizationofexpensive

%\usetikzlibrary{external}
%\tikzexternalize[prefix=figures/]{pgfplots}

\title{%
    Manual for Package \PGFPlots{}\\
    {\small 2D/3D Plots in \LaTeX{}, Version \pgfplotsversion}\\
    {\small\url{https://github.com/pgf-tikz/pgfplots}}
    %\\{\small Attention: you are using an unstable development version.}
}

\makeatletter
\long\def\abstractsmuggle{%
    \centering
    \textbf{Abstract}\\[0.5cm]

    \begin{minipage}{12cm}
        \PGFPlots{} draws high-quality function plots in normal or logarithmic
        scaling with a user-friendly interface directly in \TeX{}. The user
        supplies axis labels, legend entries and the plot coordinates for one or
        more plots and \PGFPlots{} applies axis scaling, computes any logarithms
        and axis ticks and draws the plots. It supports line plots, scatter
        plots, piecewise constant plots, bar plots, area plots, mesh and surface
        plots, patch plots, contour plots, quiver plots, histogram plots, box
        plots, polar axes, ternary diagrams, smith charts and some more. It is
        based on Till Tantau's package \PGF{}/\Tikz{}.
    \end{minipage}

}%

\expandafter\date\expandafter{\@date\\[2cm]
    \abstractsmuggle
}%
\makeatother

%\includeonly{pgfplots.reference}


\begin{document}

\def\plotcoords{%
\addplot coordinates {
(5,8.312e-02)    (17,2.547e-02)   (49,7.407e-03)
(129,2.102e-03)  (321,5.874e-04)  (769,1.623e-04)
(1793,4.442e-05) (4097,1.207e-05) (9217,3.261e-06)
};

\addplot coordinates{
(7,8.472e-02)    (31,3.044e-02)    (111,1.022e-02)
(351,3.303e-03)  (1023,1.039e-03)  (2815,3.196e-04)
(7423,9.658e-05) (18943,2.873e-05) (47103,8.437e-06)
};

\addplot coordinates{
(9,7.881e-02)     (49,3.243e-02)    (209,1.232e-02)
(769,4.454e-03)   (2561,1.551e-03)  (7937,5.236e-04)
(23297,1.723e-04) (65537,5.545e-05) (178177,1.751e-05)
};

\addplot coordinates{
(11,6.887e-02)    (71,3.177e-02)     (351,1.341e-02)
(1471,5.334e-03)  (5503,2.027e-03)   (18943,7.415e-04)
(61183,2.628e-04) (187903,9.063e-05) (553983,3.053e-05)
};

\addplot coordinates{
(13,5.755e-02)     (97,2.925e-02)     (545,1.351e-02)
(2561,5.842e-03)   (10625,2.397e-03)  (40193,9.414e-04)
(141569,3.564e-04) (471041,1.308e-04)
(1496065,4.670e-05)
};
}%


\bookmaketitle
\tableofcontents
\include{pgfplots.title_abstract_intro}
\include{pgfplots.preliminaries}
\include{pgfplots.intro}
\include{pgfplots.reference}
\include{pgfplots.libs}
\include{pgfplots.resources}
\include{pgfplots.importexport}
\include{pgfplots.basic.reference}

\printindex

\bibliographystyle{abbrv} %gerapali} %gerabbrv} %gerunsrt.bst} %gerabbrv}% gerplain}
\nocite{pgfplotstable}
\nocite{programmingnotes}
\bibliography{pgfplots}
\end{document}

% -----------------------------------------------------------------------------
% For Stefan Pinnow as reminder on what to look for when editing the manual
% -----------------------------------------------------------------------------
% There should be no line breaks in the following environments
% - |...|
% - \declareandlabel{...}
% - \verbpdfref{...}
% If "MakeTikzPictures" isn't running through check one of the externalized
% LOG files what is the cause of that.
% -----------------------------------------------------------------------------


%%%%%%%%%%%%%%%%%%%%%%%%%%%%%%%%%%%%%%%%%%%%%%%%%%%%%%%%%%%%%%%%%%%%%%%%%%%%%
%
% Package pgfplots.sty documentation.
%
% Copyright 2007/2008 by Christian Feuersaenger.
%
% This program is free software: you can redistribute it and/or modify
% it under the terms of the GNU General Public License as published by
% the Free Software Foundation, either version 3 of the License, or
% (at your option) any later version.
%
% This program is distributed in the hope that it will be useful,
% but WITHOUT ANY WARRANTY; without even the implied warranty of
% MERCHANTABILITY or FITNESS FOR A PARTICULAR PURPOSE.  See the
% GNU General Public License for more details.
%
% You should have received a copy of the GNU General Public License
% along with this program.  If not, see <http://www.gnu.org/licenses/>.
%
%
%%%%%%%%%%%%%%%%%%%%%%%%%%%%%%%%%%%%%%%%%%%%%%%%%%%%%%%%%%%%%%%%%%%%%%%%%%%%%
% SEE pgfplots-macros.tex as well!
%\pdfminorversion=5 % to allow compression
%\pdfobjcompresslevel=2
\documentclass[a4paper,openany]{book}

\let\bookmaketitle=\maketitle

% -----------------------------------
% this here is from ltxdoc:
\usepackage{doc}[=v2]
\AtBeginDocument{\MakeShortVerb{\|}}
%\setlength{\textwidth}{355pt}
%\addtolength\marginparwidth{30pt}
%\addtolength\oddsidemargin{20pt}
%\addtolength\evensidemargin{20pt}
\def\cmd#1{\cs{\expandafter\cmd@to@cs\string#1}}
\def\cmd@to@cs#1#2{\char\number`#2\relax}
\DeclareRobustCommand\cs[1]{\texttt{\char`\\#1}}
\providecommand\marg[1]{%
  {\ttfamily\char`\{}\meta{#1}{\ttfamily\char`\}}}
\providecommand\oarg[1]{%
  {\ttfamily[}\meta{#1}{\ttfamily]}}
\providecommand\parg[1]{%
  {\ttfamily(}\meta{#1}{\ttfamily)}}
\raggedbottom

% -----------------------------------
\input{pgfplots.preamble.tex}

\makeatletter
% I want a two-column index just as in pgfmanual styles. This here
% was the best way to get one:
\def\index@prologue{\section*{Index}\addcontentsline{toc}{chapter}{Index}
}
\makeatother

%\RequirePackage[german,english,francais]{babel}

\def\matlabcolormaptext{This colormap is similar to one shipped with Matlab$^\text{\textregistered}$ under a similar name.}

\IfFileExists{tikzlibraryspy.code.tex}{%
\usetikzlibrary{spy}
}{%
    \message{ERROR: tikz SPY library NOT available. The manual will only compile partially.^^J}%
}%

\usepackage{xparse}% for colorbrewer manual

\usetikzlibrary{
    decorations.markings,
    decorations.footprints,
    shapes.arrows,
    matrix,
    positioning,
}

\usepgfplotslibrary{
    fillbetween,
    ternary,
    smithchart,
    patchplots,
    polar,
    colormaps,
    colorbrewer,
    colortol,           % docu not ready yet
}
\pgfqkeys{/codeexample}{%
    every codeexample/.append style={
        /pgfplots/every ternary axis/.append style={
            /pgfplots/legend style={fill=graphicbackground},
        }
    },
    tabsize=4,
}

\pgfplotsmanualenableexternalizationofexpensive

%\usetikzlibrary{external}
%\tikzexternalize[prefix=figures/]{pgfplots}

\title{%
    Manual for Package \PGFPlots{}\\
    {\small 2D/3D Plots in \LaTeX{}, Version \pgfplotsversion}\\
    {\small\url{https://github.com/pgf-tikz/pgfplots}}
    %\\{\small Attention: you are using an unstable development version.}
}

\makeatletter
\long\def\abstractsmuggle{%
    \centering
    \textbf{Abstract}\\[0.5cm]

    \begin{minipage}{12cm}
        \PGFPlots{} draws high-quality function plots in normal or logarithmic
        scaling with a user-friendly interface directly in \TeX{}. The user
        supplies axis labels, legend entries and the plot coordinates for one or
        more plots and \PGFPlots{} applies axis scaling, computes any logarithms
        and axis ticks and draws the plots. It supports line plots, scatter
        plots, piecewise constant plots, bar plots, area plots, mesh and surface
        plots, patch plots, contour plots, quiver plots, histogram plots, box
        plots, polar axes, ternary diagrams, smith charts and some more. It is
        based on Till Tantau's package \PGF{}/\Tikz{}.
    \end{minipage}

}%

\expandafter\date\expandafter{\@date\\[2cm]
    \abstractsmuggle
}%
\makeatother

%\includeonly{pgfplots.reference}


\begin{document}

\def\plotcoords{%
\addplot coordinates {
(5,8.312e-02)    (17,2.547e-02)   (49,7.407e-03)
(129,2.102e-03)  (321,5.874e-04)  (769,1.623e-04)
(1793,4.442e-05) (4097,1.207e-05) (9217,3.261e-06)
};

\addplot coordinates{
(7,8.472e-02)    (31,3.044e-02)    (111,1.022e-02)
(351,3.303e-03)  (1023,1.039e-03)  (2815,3.196e-04)
(7423,9.658e-05) (18943,2.873e-05) (47103,8.437e-06)
};

\addplot coordinates{
(9,7.881e-02)     (49,3.243e-02)    (209,1.232e-02)
(769,4.454e-03)   (2561,1.551e-03)  (7937,5.236e-04)
(23297,1.723e-04) (65537,5.545e-05) (178177,1.751e-05)
};

\addplot coordinates{
(11,6.887e-02)    (71,3.177e-02)     (351,1.341e-02)
(1471,5.334e-03)  (5503,2.027e-03)   (18943,7.415e-04)
(61183,2.628e-04) (187903,9.063e-05) (553983,3.053e-05)
};

\addplot coordinates{
(13,5.755e-02)     (97,2.925e-02)     (545,1.351e-02)
(2561,5.842e-03)   (10625,2.397e-03)  (40193,9.414e-04)
(141569,3.564e-04) (471041,1.308e-04)
(1496065,4.670e-05)
};
}%


\bookmaketitle
\tableofcontents
\include{pgfplots.title_abstract_intro}
\include{pgfplots.preliminaries}
\include{pgfplots.intro}
\include{pgfplots.reference}
\include{pgfplots.libs}
\include{pgfplots.resources}
\include{pgfplots.importexport}
\include{pgfplots.basic.reference}

\printindex

\bibliographystyle{abbrv} %gerapali} %gerabbrv} %gerunsrt.bst} %gerabbrv}% gerplain}
\nocite{pgfplotstable}
\nocite{programmingnotes}
\bibliography{pgfplots}
\end{document}

% -----------------------------------------------------------------------------
% For Stefan Pinnow as reminder on what to look for when editing the manual
% -----------------------------------------------------------------------------
% There should be no line breaks in the following environments
% - |...|
% - \declareandlabel{...}
% - \verbpdfref{...}
% If "MakeTikzPictures" isn't running through check one of the externalized
% LOG files what is the cause of that.
% -----------------------------------------------------------------------------

\include{pgfplots.basic.reference}

\printindex

\bibliographystyle{abbrv} %gerapali} %gerabbrv} %gerunsrt.bst} %gerabbrv}% gerplain}
\nocite{pgfplotstable}
\nocite{programmingnotes}
\bibliography{pgfplots}
\end{document}

% -----------------------------------------------------------------------------
% For Stefan Pinnow as reminder on what to look for when editing the manual
% -----------------------------------------------------------------------------
% There should be no line breaks in the following environments
% - |...|
% - \declareandlabel{...}
% - \verbpdfref{...}
% If "MakeTikzPictures" isn't running through check one of the externalized
% LOG files what is the cause of that.
% -----------------------------------------------------------------------------


%%%%%%%%%%%%%%%%%%%%%%%%%%%%%%%%%%%%%%%%%%%%%%%%%%%%%%%%%%%%%%%%%%%%%%%%%%%%%
%
% Package pgfplots.sty documentation.
%
% Copyright 2007/2008 by Christian Feuersaenger.
%
% This program is free software: you can redistribute it and/or modify
% it under the terms of the GNU General Public License as published by
% the Free Software Foundation, either version 3 of the License, or
% (at your option) any later version.
%
% This program is distributed in the hope that it will be useful,
% but WITHOUT ANY WARRANTY; without even the implied warranty of
% MERCHANTABILITY or FITNESS FOR A PARTICULAR PURPOSE.  See the
% GNU General Public License for more details.
%
% You should have received a copy of the GNU General Public License
% along with this program.  If not, see <http://www.gnu.org/licenses/>.
%
%
%%%%%%%%%%%%%%%%%%%%%%%%%%%%%%%%%%%%%%%%%%%%%%%%%%%%%%%%%%%%%%%%%%%%%%%%%%%%%
% SEE pgfplots-macros.tex as well!
%\pdfminorversion=5 % to allow compression
%\pdfobjcompresslevel=2
\documentclass[a4paper,openany]{book}

\let\bookmaketitle=\maketitle

% -----------------------------------
% this here is from ltxdoc:
\usepackage{doc}[=v2]
\AtBeginDocument{\MakeShortVerb{\|}}
%\setlength{\textwidth}{355pt}
%\addtolength\marginparwidth{30pt}
%\addtolength\oddsidemargin{20pt}
%\addtolength\evensidemargin{20pt}
\def\cmd#1{\cs{\expandafter\cmd@to@cs\string#1}}
\def\cmd@to@cs#1#2{\char\number`#2\relax}
\DeclareRobustCommand\cs[1]{\texttt{\char`\\#1}}
\providecommand\marg[1]{%
  {\ttfamily\char`\{}\meta{#1}{\ttfamily\char`\}}}
\providecommand\oarg[1]{%
  {\ttfamily[}\meta{#1}{\ttfamily]}}
\providecommand\parg[1]{%
  {\ttfamily(}\meta{#1}{\ttfamily)}}
\raggedbottom

% -----------------------------------
\input{pgfplots.preamble.tex}

\makeatletter
% I want a two-column index just as in pgfmanual styles. This here
% was the best way to get one:
\def\index@prologue{\section*{Index}\addcontentsline{toc}{chapter}{Index}
}
\makeatother

%\RequirePackage[german,english,francais]{babel}

\def\matlabcolormaptext{This colormap is similar to one shipped with Matlab$^\text{\textregistered}$ under a similar name.}

\IfFileExists{tikzlibraryspy.code.tex}{%
\usetikzlibrary{spy}
}{%
    \message{ERROR: tikz SPY library NOT available. The manual will only compile partially.^^J}%
}%

\usepackage{xparse}% for colorbrewer manual

\usetikzlibrary{
    decorations.markings,
    decorations.footprints,
    shapes.arrows,
    matrix,
    positioning,
}

\usepgfplotslibrary{
    fillbetween,
    ternary,
    smithchart,
    patchplots,
    polar,
    colormaps,
    colorbrewer,
    colortol,           % docu not ready yet
}
\pgfqkeys{/codeexample}{%
    every codeexample/.append style={
        /pgfplots/every ternary axis/.append style={
            /pgfplots/legend style={fill=graphicbackground},
        }
    },
    tabsize=4,
}

\pgfplotsmanualenableexternalizationofexpensive

%\usetikzlibrary{external}
%\tikzexternalize[prefix=figures/]{pgfplots}

\title{%
    Manual for Package \PGFPlots{}\\
    {\small 2D/3D Plots in \LaTeX{}, Version \pgfplotsversion}\\
    {\small\url{https://github.com/pgf-tikz/pgfplots}}
    %\\{\small Attention: you are using an unstable development version.}
}

\makeatletter
\long\def\abstractsmuggle{%
    \centering
    \textbf{Abstract}\\[0.5cm]

    \begin{minipage}{12cm}
        \PGFPlots{} draws high-quality function plots in normal or logarithmic
        scaling with a user-friendly interface directly in \TeX{}. The user
        supplies axis labels, legend entries and the plot coordinates for one or
        more plots and \PGFPlots{} applies axis scaling, computes any logarithms
        and axis ticks and draws the plots. It supports line plots, scatter
        plots, piecewise constant plots, bar plots, area plots, mesh and surface
        plots, patch plots, contour plots, quiver plots, histogram plots, box
        plots, polar axes, ternary diagrams, smith charts and some more. It is
        based on Till Tantau's package \PGF{}/\Tikz{}.
    \end{minipage}

}%

\expandafter\date\expandafter{\@date\\[2cm]
    \abstractsmuggle
}%
\makeatother

%\includeonly{pgfplots.reference}


\begin{document}

\def\plotcoords{%
\addplot coordinates {
(5,8.312e-02)    (17,2.547e-02)   (49,7.407e-03)
(129,2.102e-03)  (321,5.874e-04)  (769,1.623e-04)
(1793,4.442e-05) (4097,1.207e-05) (9217,3.261e-06)
};

\addplot coordinates{
(7,8.472e-02)    (31,3.044e-02)    (111,1.022e-02)
(351,3.303e-03)  (1023,1.039e-03)  (2815,3.196e-04)
(7423,9.658e-05) (18943,2.873e-05) (47103,8.437e-06)
};

\addplot coordinates{
(9,7.881e-02)     (49,3.243e-02)    (209,1.232e-02)
(769,4.454e-03)   (2561,1.551e-03)  (7937,5.236e-04)
(23297,1.723e-04) (65537,5.545e-05) (178177,1.751e-05)
};

\addplot coordinates{
(11,6.887e-02)    (71,3.177e-02)     (351,1.341e-02)
(1471,5.334e-03)  (5503,2.027e-03)   (18943,7.415e-04)
(61183,2.628e-04) (187903,9.063e-05) (553983,3.053e-05)
};

\addplot coordinates{
(13,5.755e-02)     (97,2.925e-02)     (545,1.351e-02)
(2561,5.842e-03)   (10625,2.397e-03)  (40193,9.414e-04)
(141569,3.564e-04) (471041,1.308e-04)
(1496065,4.670e-05)
};
}%


\bookmaketitle
\tableofcontents

%%%%%%%%%%%%%%%%%%%%%%%%%%%%%%%%%%%%%%%%%%%%%%%%%%%%%%%%%%%%%%%%%%%%%%%%%%%%%
%
% Package pgfplots.sty documentation.
%
% Copyright 2007/2008 by Christian Feuersaenger.
%
% This program is free software: you can redistribute it and/or modify
% it under the terms of the GNU General Public License as published by
% the Free Software Foundation, either version 3 of the License, or
% (at your option) any later version.
%
% This program is distributed in the hope that it will be useful,
% but WITHOUT ANY WARRANTY; without even the implied warranty of
% MERCHANTABILITY or FITNESS FOR A PARTICULAR PURPOSE.  See the
% GNU General Public License for more details.
%
% You should have received a copy of the GNU General Public License
% along with this program.  If not, see <http://www.gnu.org/licenses/>.
%
%
%%%%%%%%%%%%%%%%%%%%%%%%%%%%%%%%%%%%%%%%%%%%%%%%%%%%%%%%%%%%%%%%%%%%%%%%%%%%%
% SEE pgfplots-macros.tex as well!
%\pdfminorversion=5 % to allow compression
%\pdfobjcompresslevel=2
\documentclass[a4paper,openany]{book}

\let\bookmaketitle=\maketitle

% -----------------------------------
% this here is from ltxdoc:
\usepackage{doc}[=v2]
\AtBeginDocument{\MakeShortVerb{\|}}
%\setlength{\textwidth}{355pt}
%\addtolength\marginparwidth{30pt}
%\addtolength\oddsidemargin{20pt}
%\addtolength\evensidemargin{20pt}
\def\cmd#1{\cs{\expandafter\cmd@to@cs\string#1}}
\def\cmd@to@cs#1#2{\char\number`#2\relax}
\DeclareRobustCommand\cs[1]{\texttt{\char`\\#1}}
\providecommand\marg[1]{%
  {\ttfamily\char`\{}\meta{#1}{\ttfamily\char`\}}}
\providecommand\oarg[1]{%
  {\ttfamily[}\meta{#1}{\ttfamily]}}
\providecommand\parg[1]{%
  {\ttfamily(}\meta{#1}{\ttfamily)}}
\raggedbottom

% -----------------------------------
\input{pgfplots.preamble.tex}

\makeatletter
% I want a two-column index just as in pgfmanual styles. This here
% was the best way to get one:
\def\index@prologue{\section*{Index}\addcontentsline{toc}{chapter}{Index}
}
\makeatother

%\RequirePackage[german,english,francais]{babel}

\def\matlabcolormaptext{This colormap is similar to one shipped with Matlab$^\text{\textregistered}$ under a similar name.}

\IfFileExists{tikzlibraryspy.code.tex}{%
\usetikzlibrary{spy}
}{%
    \message{ERROR: tikz SPY library NOT available. The manual will only compile partially.^^J}%
}%

\usepackage{xparse}% for colorbrewer manual

\usetikzlibrary{
    decorations.markings,
    decorations.footprints,
    shapes.arrows,
    matrix,
    positioning,
}

\usepgfplotslibrary{
    fillbetween,
    ternary,
    smithchart,
    patchplots,
    polar,
    colormaps,
    colorbrewer,
    colortol,           % docu not ready yet
}
\pgfqkeys{/codeexample}{%
    every codeexample/.append style={
        /pgfplots/every ternary axis/.append style={
            /pgfplots/legend style={fill=graphicbackground},
        }
    },
    tabsize=4,
}

\pgfplotsmanualenableexternalizationofexpensive

%\usetikzlibrary{external}
%\tikzexternalize[prefix=figures/]{pgfplots}

\title{%
    Manual for Package \PGFPlots{}\\
    {\small 2D/3D Plots in \LaTeX{}, Version \pgfplotsversion}\\
    {\small\url{https://github.com/pgf-tikz/pgfplots}}
    %\\{\small Attention: you are using an unstable development version.}
}

\makeatletter
\long\def\abstractsmuggle{%
    \centering
    \textbf{Abstract}\\[0.5cm]

    \begin{minipage}{12cm}
        \PGFPlots{} draws high-quality function plots in normal or logarithmic
        scaling with a user-friendly interface directly in \TeX{}. The user
        supplies axis labels, legend entries and the plot coordinates for one or
        more plots and \PGFPlots{} applies axis scaling, computes any logarithms
        and axis ticks and draws the plots. It supports line plots, scatter
        plots, piecewise constant plots, bar plots, area plots, mesh and surface
        plots, patch plots, contour plots, quiver plots, histogram plots, box
        plots, polar axes, ternary diagrams, smith charts and some more. It is
        based on Till Tantau's package \PGF{}/\Tikz{}.
    \end{minipage}

}%

\expandafter\date\expandafter{\@date\\[2cm]
    \abstractsmuggle
}%
\makeatother

%\includeonly{pgfplots.reference}


\begin{document}

\def\plotcoords{%
\addplot coordinates {
(5,8.312e-02)    (17,2.547e-02)   (49,7.407e-03)
(129,2.102e-03)  (321,5.874e-04)  (769,1.623e-04)
(1793,4.442e-05) (4097,1.207e-05) (9217,3.261e-06)
};

\addplot coordinates{
(7,8.472e-02)    (31,3.044e-02)    (111,1.022e-02)
(351,3.303e-03)  (1023,1.039e-03)  (2815,3.196e-04)
(7423,9.658e-05) (18943,2.873e-05) (47103,8.437e-06)
};

\addplot coordinates{
(9,7.881e-02)     (49,3.243e-02)    (209,1.232e-02)
(769,4.454e-03)   (2561,1.551e-03)  (7937,5.236e-04)
(23297,1.723e-04) (65537,5.545e-05) (178177,1.751e-05)
};

\addplot coordinates{
(11,6.887e-02)    (71,3.177e-02)     (351,1.341e-02)
(1471,5.334e-03)  (5503,2.027e-03)   (18943,7.415e-04)
(61183,2.628e-04) (187903,9.063e-05) (553983,3.053e-05)
};

\addplot coordinates{
(13,5.755e-02)     (97,2.925e-02)     (545,1.351e-02)
(2561,5.842e-03)   (10625,2.397e-03)  (40193,9.414e-04)
(141569,3.564e-04) (471041,1.308e-04)
(1496065,4.670e-05)
};
}%


\bookmaketitle
\tableofcontents
\include{pgfplots.title_abstract_intro}
\include{pgfplots.preliminaries}
\include{pgfplots.intro}
\include{pgfplots.reference}
\include{pgfplots.libs}
\include{pgfplots.resources}
\include{pgfplots.importexport}
\include{pgfplots.basic.reference}

\printindex

\bibliographystyle{abbrv} %gerapali} %gerabbrv} %gerunsrt.bst} %gerabbrv}% gerplain}
\nocite{pgfplotstable}
\nocite{programmingnotes}
\bibliography{pgfplots}
\end{document}

% -----------------------------------------------------------------------------
% For Stefan Pinnow as reminder on what to look for when editing the manual
% -----------------------------------------------------------------------------
% There should be no line breaks in the following environments
% - |...|
% - \declareandlabel{...}
% - \verbpdfref{...}
% If "MakeTikzPictures" isn't running through check one of the externalized
% LOG files what is the cause of that.
% -----------------------------------------------------------------------------


%%%%%%%%%%%%%%%%%%%%%%%%%%%%%%%%%%%%%%%%%%%%%%%%%%%%%%%%%%%%%%%%%%%%%%%%%%%%%
%
% Package pgfplots.sty documentation.
%
% Copyright 2007/2008 by Christian Feuersaenger.
%
% This program is free software: you can redistribute it and/or modify
% it under the terms of the GNU General Public License as published by
% the Free Software Foundation, either version 3 of the License, or
% (at your option) any later version.
%
% This program is distributed in the hope that it will be useful,
% but WITHOUT ANY WARRANTY; without even the implied warranty of
% MERCHANTABILITY or FITNESS FOR A PARTICULAR PURPOSE.  See the
% GNU General Public License for more details.
%
% You should have received a copy of the GNU General Public License
% along with this program.  If not, see <http://www.gnu.org/licenses/>.
%
%
%%%%%%%%%%%%%%%%%%%%%%%%%%%%%%%%%%%%%%%%%%%%%%%%%%%%%%%%%%%%%%%%%%%%%%%%%%%%%
% SEE pgfplots-macros.tex as well!
%\pdfminorversion=5 % to allow compression
%\pdfobjcompresslevel=2
\documentclass[a4paper,openany]{book}

\let\bookmaketitle=\maketitle

% -----------------------------------
% this here is from ltxdoc:
\usepackage{doc}[=v2]
\AtBeginDocument{\MakeShortVerb{\|}}
%\setlength{\textwidth}{355pt}
%\addtolength\marginparwidth{30pt}
%\addtolength\oddsidemargin{20pt}
%\addtolength\evensidemargin{20pt}
\def\cmd#1{\cs{\expandafter\cmd@to@cs\string#1}}
\def\cmd@to@cs#1#2{\char\number`#2\relax}
\DeclareRobustCommand\cs[1]{\texttt{\char`\\#1}}
\providecommand\marg[1]{%
  {\ttfamily\char`\{}\meta{#1}{\ttfamily\char`\}}}
\providecommand\oarg[1]{%
  {\ttfamily[}\meta{#1}{\ttfamily]}}
\providecommand\parg[1]{%
  {\ttfamily(}\meta{#1}{\ttfamily)}}
\raggedbottom

% -----------------------------------
\input{pgfplots.preamble.tex}

\makeatletter
% I want a two-column index just as in pgfmanual styles. This here
% was the best way to get one:
\def\index@prologue{\section*{Index}\addcontentsline{toc}{chapter}{Index}
}
\makeatother

%\RequirePackage[german,english,francais]{babel}

\def\matlabcolormaptext{This colormap is similar to one shipped with Matlab$^\text{\textregistered}$ under a similar name.}

\IfFileExists{tikzlibraryspy.code.tex}{%
\usetikzlibrary{spy}
}{%
    \message{ERROR: tikz SPY library NOT available. The manual will only compile partially.^^J}%
}%

\usepackage{xparse}% for colorbrewer manual

\usetikzlibrary{
    decorations.markings,
    decorations.footprints,
    shapes.arrows,
    matrix,
    positioning,
}

\usepgfplotslibrary{
    fillbetween,
    ternary,
    smithchart,
    patchplots,
    polar,
    colormaps,
    colorbrewer,
    colortol,           % docu not ready yet
}
\pgfqkeys{/codeexample}{%
    every codeexample/.append style={
        /pgfplots/every ternary axis/.append style={
            /pgfplots/legend style={fill=graphicbackground},
        }
    },
    tabsize=4,
}

\pgfplotsmanualenableexternalizationofexpensive

%\usetikzlibrary{external}
%\tikzexternalize[prefix=figures/]{pgfplots}

\title{%
    Manual for Package \PGFPlots{}\\
    {\small 2D/3D Plots in \LaTeX{}, Version \pgfplotsversion}\\
    {\small\url{https://github.com/pgf-tikz/pgfplots}}
    %\\{\small Attention: you are using an unstable development version.}
}

\makeatletter
\long\def\abstractsmuggle{%
    \centering
    \textbf{Abstract}\\[0.5cm]

    \begin{minipage}{12cm}
        \PGFPlots{} draws high-quality function plots in normal or logarithmic
        scaling with a user-friendly interface directly in \TeX{}. The user
        supplies axis labels, legend entries and the plot coordinates for one or
        more plots and \PGFPlots{} applies axis scaling, computes any logarithms
        and axis ticks and draws the plots. It supports line plots, scatter
        plots, piecewise constant plots, bar plots, area plots, mesh and surface
        plots, patch plots, contour plots, quiver plots, histogram plots, box
        plots, polar axes, ternary diagrams, smith charts and some more. It is
        based on Till Tantau's package \PGF{}/\Tikz{}.
    \end{minipage}

}%

\expandafter\date\expandafter{\@date\\[2cm]
    \abstractsmuggle
}%
\makeatother

%\includeonly{pgfplots.reference}


\begin{document}

\def\plotcoords{%
\addplot coordinates {
(5,8.312e-02)    (17,2.547e-02)   (49,7.407e-03)
(129,2.102e-03)  (321,5.874e-04)  (769,1.623e-04)
(1793,4.442e-05) (4097,1.207e-05) (9217,3.261e-06)
};

\addplot coordinates{
(7,8.472e-02)    (31,3.044e-02)    (111,1.022e-02)
(351,3.303e-03)  (1023,1.039e-03)  (2815,3.196e-04)
(7423,9.658e-05) (18943,2.873e-05) (47103,8.437e-06)
};

\addplot coordinates{
(9,7.881e-02)     (49,3.243e-02)    (209,1.232e-02)
(769,4.454e-03)   (2561,1.551e-03)  (7937,5.236e-04)
(23297,1.723e-04) (65537,5.545e-05) (178177,1.751e-05)
};

\addplot coordinates{
(11,6.887e-02)    (71,3.177e-02)     (351,1.341e-02)
(1471,5.334e-03)  (5503,2.027e-03)   (18943,7.415e-04)
(61183,2.628e-04) (187903,9.063e-05) (553983,3.053e-05)
};

\addplot coordinates{
(13,5.755e-02)     (97,2.925e-02)     (545,1.351e-02)
(2561,5.842e-03)   (10625,2.397e-03)  (40193,9.414e-04)
(141569,3.564e-04) (471041,1.308e-04)
(1496065,4.670e-05)
};
}%


\bookmaketitle
\tableofcontents
\include{pgfplots.title_abstract_intro}
\include{pgfplots.preliminaries}
\include{pgfplots.intro}
\include{pgfplots.reference}
\include{pgfplots.libs}
\include{pgfplots.resources}
\include{pgfplots.importexport}
\include{pgfplots.basic.reference}

\printindex

\bibliographystyle{abbrv} %gerapali} %gerabbrv} %gerunsrt.bst} %gerabbrv}% gerplain}
\nocite{pgfplotstable}
\nocite{programmingnotes}
\bibliography{pgfplots}
\end{document}

% -----------------------------------------------------------------------------
% For Stefan Pinnow as reminder on what to look for when editing the manual
% -----------------------------------------------------------------------------
% There should be no line breaks in the following environments
% - |...|
% - \declareandlabel{...}
% - \verbpdfref{...}
% If "MakeTikzPictures" isn't running through check one of the externalized
% LOG files what is the cause of that.
% -----------------------------------------------------------------------------


%%%%%%%%%%%%%%%%%%%%%%%%%%%%%%%%%%%%%%%%%%%%%%%%%%%%%%%%%%%%%%%%%%%%%%%%%%%%%
%
% Package pgfplots.sty documentation.
%
% Copyright 2007/2008 by Christian Feuersaenger.
%
% This program is free software: you can redistribute it and/or modify
% it under the terms of the GNU General Public License as published by
% the Free Software Foundation, either version 3 of the License, or
% (at your option) any later version.
%
% This program is distributed in the hope that it will be useful,
% but WITHOUT ANY WARRANTY; without even the implied warranty of
% MERCHANTABILITY or FITNESS FOR A PARTICULAR PURPOSE.  See the
% GNU General Public License for more details.
%
% You should have received a copy of the GNU General Public License
% along with this program.  If not, see <http://www.gnu.org/licenses/>.
%
%
%%%%%%%%%%%%%%%%%%%%%%%%%%%%%%%%%%%%%%%%%%%%%%%%%%%%%%%%%%%%%%%%%%%%%%%%%%%%%
% SEE pgfplots-macros.tex as well!
%\pdfminorversion=5 % to allow compression
%\pdfobjcompresslevel=2
\documentclass[a4paper,openany]{book}

\let\bookmaketitle=\maketitle

% -----------------------------------
% this here is from ltxdoc:
\usepackage{doc}[=v2]
\AtBeginDocument{\MakeShortVerb{\|}}
%\setlength{\textwidth}{355pt}
%\addtolength\marginparwidth{30pt}
%\addtolength\oddsidemargin{20pt}
%\addtolength\evensidemargin{20pt}
\def\cmd#1{\cs{\expandafter\cmd@to@cs\string#1}}
\def\cmd@to@cs#1#2{\char\number`#2\relax}
\DeclareRobustCommand\cs[1]{\texttt{\char`\\#1}}
\providecommand\marg[1]{%
  {\ttfamily\char`\{}\meta{#1}{\ttfamily\char`\}}}
\providecommand\oarg[1]{%
  {\ttfamily[}\meta{#1}{\ttfamily]}}
\providecommand\parg[1]{%
  {\ttfamily(}\meta{#1}{\ttfamily)}}
\raggedbottom

% -----------------------------------
\input{pgfplots.preamble.tex}

\makeatletter
% I want a two-column index just as in pgfmanual styles. This here
% was the best way to get one:
\def\index@prologue{\section*{Index}\addcontentsline{toc}{chapter}{Index}
}
\makeatother

%\RequirePackage[german,english,francais]{babel}

\def\matlabcolormaptext{This colormap is similar to one shipped with Matlab$^\text{\textregistered}$ under a similar name.}

\IfFileExists{tikzlibraryspy.code.tex}{%
\usetikzlibrary{spy}
}{%
    \message{ERROR: tikz SPY library NOT available. The manual will only compile partially.^^J}%
}%

\usepackage{xparse}% for colorbrewer manual

\usetikzlibrary{
    decorations.markings,
    decorations.footprints,
    shapes.arrows,
    matrix,
    positioning,
}

\usepgfplotslibrary{
    fillbetween,
    ternary,
    smithchart,
    patchplots,
    polar,
    colormaps,
    colorbrewer,
    colortol,           % docu not ready yet
}
\pgfqkeys{/codeexample}{%
    every codeexample/.append style={
        /pgfplots/every ternary axis/.append style={
            /pgfplots/legend style={fill=graphicbackground},
        }
    },
    tabsize=4,
}

\pgfplotsmanualenableexternalizationofexpensive

%\usetikzlibrary{external}
%\tikzexternalize[prefix=figures/]{pgfplots}

\title{%
    Manual for Package \PGFPlots{}\\
    {\small 2D/3D Plots in \LaTeX{}, Version \pgfplotsversion}\\
    {\small\url{https://github.com/pgf-tikz/pgfplots}}
    %\\{\small Attention: you are using an unstable development version.}
}

\makeatletter
\long\def\abstractsmuggle{%
    \centering
    \textbf{Abstract}\\[0.5cm]

    \begin{minipage}{12cm}
        \PGFPlots{} draws high-quality function plots in normal or logarithmic
        scaling with a user-friendly interface directly in \TeX{}. The user
        supplies axis labels, legend entries and the plot coordinates for one or
        more plots and \PGFPlots{} applies axis scaling, computes any logarithms
        and axis ticks and draws the plots. It supports line plots, scatter
        plots, piecewise constant plots, bar plots, area plots, mesh and surface
        plots, patch plots, contour plots, quiver plots, histogram plots, box
        plots, polar axes, ternary diagrams, smith charts and some more. It is
        based on Till Tantau's package \PGF{}/\Tikz{}.
    \end{minipage}

}%

\expandafter\date\expandafter{\@date\\[2cm]
    \abstractsmuggle
}%
\makeatother

%\includeonly{pgfplots.reference}


\begin{document}

\def\plotcoords{%
\addplot coordinates {
(5,8.312e-02)    (17,2.547e-02)   (49,7.407e-03)
(129,2.102e-03)  (321,5.874e-04)  (769,1.623e-04)
(1793,4.442e-05) (4097,1.207e-05) (9217,3.261e-06)
};

\addplot coordinates{
(7,8.472e-02)    (31,3.044e-02)    (111,1.022e-02)
(351,3.303e-03)  (1023,1.039e-03)  (2815,3.196e-04)
(7423,9.658e-05) (18943,2.873e-05) (47103,8.437e-06)
};

\addplot coordinates{
(9,7.881e-02)     (49,3.243e-02)    (209,1.232e-02)
(769,4.454e-03)   (2561,1.551e-03)  (7937,5.236e-04)
(23297,1.723e-04) (65537,5.545e-05) (178177,1.751e-05)
};

\addplot coordinates{
(11,6.887e-02)    (71,3.177e-02)     (351,1.341e-02)
(1471,5.334e-03)  (5503,2.027e-03)   (18943,7.415e-04)
(61183,2.628e-04) (187903,9.063e-05) (553983,3.053e-05)
};

\addplot coordinates{
(13,5.755e-02)     (97,2.925e-02)     (545,1.351e-02)
(2561,5.842e-03)   (10625,2.397e-03)  (40193,9.414e-04)
(141569,3.564e-04) (471041,1.308e-04)
(1496065,4.670e-05)
};
}%


\bookmaketitle
\tableofcontents
\include{pgfplots.title_abstract_intro}
\include{pgfplots.preliminaries}
\include{pgfplots.intro}
\include{pgfplots.reference}
\include{pgfplots.libs}
\include{pgfplots.resources}
\include{pgfplots.importexport}
\include{pgfplots.basic.reference}

\printindex

\bibliographystyle{abbrv} %gerapali} %gerabbrv} %gerunsrt.bst} %gerabbrv}% gerplain}
\nocite{pgfplotstable}
\nocite{programmingnotes}
\bibliography{pgfplots}
\end{document}

% -----------------------------------------------------------------------------
% For Stefan Pinnow as reminder on what to look for when editing the manual
% -----------------------------------------------------------------------------
% There should be no line breaks in the following environments
% - |...|
% - \declareandlabel{...}
% - \verbpdfref{...}
% If "MakeTikzPictures" isn't running through check one of the externalized
% LOG files what is the cause of that.
% -----------------------------------------------------------------------------


%%%%%%%%%%%%%%%%%%%%%%%%%%%%%%%%%%%%%%%%%%%%%%%%%%%%%%%%%%%%%%%%%%%%%%%%%%%%%
%
% Package pgfplots.sty documentation.
%
% Copyright 2007/2008 by Christian Feuersaenger.
%
% This program is free software: you can redistribute it and/or modify
% it under the terms of the GNU General Public License as published by
% the Free Software Foundation, either version 3 of the License, or
% (at your option) any later version.
%
% This program is distributed in the hope that it will be useful,
% but WITHOUT ANY WARRANTY; without even the implied warranty of
% MERCHANTABILITY or FITNESS FOR A PARTICULAR PURPOSE.  See the
% GNU General Public License for more details.
%
% You should have received a copy of the GNU General Public License
% along with this program.  If not, see <http://www.gnu.org/licenses/>.
%
%
%%%%%%%%%%%%%%%%%%%%%%%%%%%%%%%%%%%%%%%%%%%%%%%%%%%%%%%%%%%%%%%%%%%%%%%%%%%%%
% SEE pgfplots-macros.tex as well!
%\pdfminorversion=5 % to allow compression
%\pdfobjcompresslevel=2
\documentclass[a4paper,openany]{book}

\let\bookmaketitle=\maketitle

% -----------------------------------
% this here is from ltxdoc:
\usepackage{doc}[=v2]
\AtBeginDocument{\MakeShortVerb{\|}}
%\setlength{\textwidth}{355pt}
%\addtolength\marginparwidth{30pt}
%\addtolength\oddsidemargin{20pt}
%\addtolength\evensidemargin{20pt}
\def\cmd#1{\cs{\expandafter\cmd@to@cs\string#1}}
\def\cmd@to@cs#1#2{\char\number`#2\relax}
\DeclareRobustCommand\cs[1]{\texttt{\char`\\#1}}
\providecommand\marg[1]{%
  {\ttfamily\char`\{}\meta{#1}{\ttfamily\char`\}}}
\providecommand\oarg[1]{%
  {\ttfamily[}\meta{#1}{\ttfamily]}}
\providecommand\parg[1]{%
  {\ttfamily(}\meta{#1}{\ttfamily)}}
\raggedbottom

% -----------------------------------
\input{pgfplots.preamble.tex}

\makeatletter
% I want a two-column index just as in pgfmanual styles. This here
% was the best way to get one:
\def\index@prologue{\section*{Index}\addcontentsline{toc}{chapter}{Index}
}
\makeatother

%\RequirePackage[german,english,francais]{babel}

\def\matlabcolormaptext{This colormap is similar to one shipped with Matlab$^\text{\textregistered}$ under a similar name.}

\IfFileExists{tikzlibraryspy.code.tex}{%
\usetikzlibrary{spy}
}{%
    \message{ERROR: tikz SPY library NOT available. The manual will only compile partially.^^J}%
}%

\usepackage{xparse}% for colorbrewer manual

\usetikzlibrary{
    decorations.markings,
    decorations.footprints,
    shapes.arrows,
    matrix,
    positioning,
}

\usepgfplotslibrary{
    fillbetween,
    ternary,
    smithchart,
    patchplots,
    polar,
    colormaps,
    colorbrewer,
    colortol,           % docu not ready yet
}
\pgfqkeys{/codeexample}{%
    every codeexample/.append style={
        /pgfplots/every ternary axis/.append style={
            /pgfplots/legend style={fill=graphicbackground},
        }
    },
    tabsize=4,
}

\pgfplotsmanualenableexternalizationofexpensive

%\usetikzlibrary{external}
%\tikzexternalize[prefix=figures/]{pgfplots}

\title{%
    Manual for Package \PGFPlots{}\\
    {\small 2D/3D Plots in \LaTeX{}, Version \pgfplotsversion}\\
    {\small\url{https://github.com/pgf-tikz/pgfplots}}
    %\\{\small Attention: you are using an unstable development version.}
}

\makeatletter
\long\def\abstractsmuggle{%
    \centering
    \textbf{Abstract}\\[0.5cm]

    \begin{minipage}{12cm}
        \PGFPlots{} draws high-quality function plots in normal or logarithmic
        scaling with a user-friendly interface directly in \TeX{}. The user
        supplies axis labels, legend entries and the plot coordinates for one or
        more plots and \PGFPlots{} applies axis scaling, computes any logarithms
        and axis ticks and draws the plots. It supports line plots, scatter
        plots, piecewise constant plots, bar plots, area plots, mesh and surface
        plots, patch plots, contour plots, quiver plots, histogram plots, box
        plots, polar axes, ternary diagrams, smith charts and some more. It is
        based on Till Tantau's package \PGF{}/\Tikz{}.
    \end{minipage}

}%

\expandafter\date\expandafter{\@date\\[2cm]
    \abstractsmuggle
}%
\makeatother

%\includeonly{pgfplots.reference}


\begin{document}

\def\plotcoords{%
\addplot coordinates {
(5,8.312e-02)    (17,2.547e-02)   (49,7.407e-03)
(129,2.102e-03)  (321,5.874e-04)  (769,1.623e-04)
(1793,4.442e-05) (4097,1.207e-05) (9217,3.261e-06)
};

\addplot coordinates{
(7,8.472e-02)    (31,3.044e-02)    (111,1.022e-02)
(351,3.303e-03)  (1023,1.039e-03)  (2815,3.196e-04)
(7423,9.658e-05) (18943,2.873e-05) (47103,8.437e-06)
};

\addplot coordinates{
(9,7.881e-02)     (49,3.243e-02)    (209,1.232e-02)
(769,4.454e-03)   (2561,1.551e-03)  (7937,5.236e-04)
(23297,1.723e-04) (65537,5.545e-05) (178177,1.751e-05)
};

\addplot coordinates{
(11,6.887e-02)    (71,3.177e-02)     (351,1.341e-02)
(1471,5.334e-03)  (5503,2.027e-03)   (18943,7.415e-04)
(61183,2.628e-04) (187903,9.063e-05) (553983,3.053e-05)
};

\addplot coordinates{
(13,5.755e-02)     (97,2.925e-02)     (545,1.351e-02)
(2561,5.842e-03)   (10625,2.397e-03)  (40193,9.414e-04)
(141569,3.564e-04) (471041,1.308e-04)
(1496065,4.670e-05)
};
}%


\bookmaketitle
\tableofcontents
\include{pgfplots.title_abstract_intro}
\include{pgfplots.preliminaries}
\include{pgfplots.intro}
\include{pgfplots.reference}
\include{pgfplots.libs}
\include{pgfplots.resources}
\include{pgfplots.importexport}
\include{pgfplots.basic.reference}

\printindex

\bibliographystyle{abbrv} %gerapali} %gerabbrv} %gerunsrt.bst} %gerabbrv}% gerplain}
\nocite{pgfplotstable}
\nocite{programmingnotes}
\bibliography{pgfplots}
\end{document}

% -----------------------------------------------------------------------------
% For Stefan Pinnow as reminder on what to look for when editing the manual
% -----------------------------------------------------------------------------
% There should be no line breaks in the following environments
% - |...|
% - \declareandlabel{...}
% - \verbpdfref{...}
% If "MakeTikzPictures" isn't running through check one of the externalized
% LOG files what is the cause of that.
% -----------------------------------------------------------------------------


%%%%%%%%%%%%%%%%%%%%%%%%%%%%%%%%%%%%%%%%%%%%%%%%%%%%%%%%%%%%%%%%%%%%%%%%%%%%%
%
% Package pgfplots.sty documentation.
%
% Copyright 2007/2008 by Christian Feuersaenger.
%
% This program is free software: you can redistribute it and/or modify
% it under the terms of the GNU General Public License as published by
% the Free Software Foundation, either version 3 of the License, or
% (at your option) any later version.
%
% This program is distributed in the hope that it will be useful,
% but WITHOUT ANY WARRANTY; without even the implied warranty of
% MERCHANTABILITY or FITNESS FOR A PARTICULAR PURPOSE.  See the
% GNU General Public License for more details.
%
% You should have received a copy of the GNU General Public License
% along with this program.  If not, see <http://www.gnu.org/licenses/>.
%
%
%%%%%%%%%%%%%%%%%%%%%%%%%%%%%%%%%%%%%%%%%%%%%%%%%%%%%%%%%%%%%%%%%%%%%%%%%%%%%
% SEE pgfplots-macros.tex as well!
%\pdfminorversion=5 % to allow compression
%\pdfobjcompresslevel=2
\documentclass[a4paper,openany]{book}

\let\bookmaketitle=\maketitle

% -----------------------------------
% this here is from ltxdoc:
\usepackage{doc}[=v2]
\AtBeginDocument{\MakeShortVerb{\|}}
%\setlength{\textwidth}{355pt}
%\addtolength\marginparwidth{30pt}
%\addtolength\oddsidemargin{20pt}
%\addtolength\evensidemargin{20pt}
\def\cmd#1{\cs{\expandafter\cmd@to@cs\string#1}}
\def\cmd@to@cs#1#2{\char\number`#2\relax}
\DeclareRobustCommand\cs[1]{\texttt{\char`\\#1}}
\providecommand\marg[1]{%
  {\ttfamily\char`\{}\meta{#1}{\ttfamily\char`\}}}
\providecommand\oarg[1]{%
  {\ttfamily[}\meta{#1}{\ttfamily]}}
\providecommand\parg[1]{%
  {\ttfamily(}\meta{#1}{\ttfamily)}}
\raggedbottom

% -----------------------------------
\input{pgfplots.preamble.tex}

\makeatletter
% I want a two-column index just as in pgfmanual styles. This here
% was the best way to get one:
\def\index@prologue{\section*{Index}\addcontentsline{toc}{chapter}{Index}
}
\makeatother

%\RequirePackage[german,english,francais]{babel}

\def\matlabcolormaptext{This colormap is similar to one shipped with Matlab$^\text{\textregistered}$ under a similar name.}

\IfFileExists{tikzlibraryspy.code.tex}{%
\usetikzlibrary{spy}
}{%
    \message{ERROR: tikz SPY library NOT available. The manual will only compile partially.^^J}%
}%

\usepackage{xparse}% for colorbrewer manual

\usetikzlibrary{
    decorations.markings,
    decorations.footprints,
    shapes.arrows,
    matrix,
    positioning,
}

\usepgfplotslibrary{
    fillbetween,
    ternary,
    smithchart,
    patchplots,
    polar,
    colormaps,
    colorbrewer,
    colortol,           % docu not ready yet
}
\pgfqkeys{/codeexample}{%
    every codeexample/.append style={
        /pgfplots/every ternary axis/.append style={
            /pgfplots/legend style={fill=graphicbackground},
        }
    },
    tabsize=4,
}

\pgfplotsmanualenableexternalizationofexpensive

%\usetikzlibrary{external}
%\tikzexternalize[prefix=figures/]{pgfplots}

\title{%
    Manual for Package \PGFPlots{}\\
    {\small 2D/3D Plots in \LaTeX{}, Version \pgfplotsversion}\\
    {\small\url{https://github.com/pgf-tikz/pgfplots}}
    %\\{\small Attention: you are using an unstable development version.}
}

\makeatletter
\long\def\abstractsmuggle{%
    \centering
    \textbf{Abstract}\\[0.5cm]

    \begin{minipage}{12cm}
        \PGFPlots{} draws high-quality function plots in normal or logarithmic
        scaling with a user-friendly interface directly in \TeX{}. The user
        supplies axis labels, legend entries and the plot coordinates for one or
        more plots and \PGFPlots{} applies axis scaling, computes any logarithms
        and axis ticks and draws the plots. It supports line plots, scatter
        plots, piecewise constant plots, bar plots, area plots, mesh and surface
        plots, patch plots, contour plots, quiver plots, histogram plots, box
        plots, polar axes, ternary diagrams, smith charts and some more. It is
        based on Till Tantau's package \PGF{}/\Tikz{}.
    \end{minipage}

}%

\expandafter\date\expandafter{\@date\\[2cm]
    \abstractsmuggle
}%
\makeatother

%\includeonly{pgfplots.reference}


\begin{document}

\def\plotcoords{%
\addplot coordinates {
(5,8.312e-02)    (17,2.547e-02)   (49,7.407e-03)
(129,2.102e-03)  (321,5.874e-04)  (769,1.623e-04)
(1793,4.442e-05) (4097,1.207e-05) (9217,3.261e-06)
};

\addplot coordinates{
(7,8.472e-02)    (31,3.044e-02)    (111,1.022e-02)
(351,3.303e-03)  (1023,1.039e-03)  (2815,3.196e-04)
(7423,9.658e-05) (18943,2.873e-05) (47103,8.437e-06)
};

\addplot coordinates{
(9,7.881e-02)     (49,3.243e-02)    (209,1.232e-02)
(769,4.454e-03)   (2561,1.551e-03)  (7937,5.236e-04)
(23297,1.723e-04) (65537,5.545e-05) (178177,1.751e-05)
};

\addplot coordinates{
(11,6.887e-02)    (71,3.177e-02)     (351,1.341e-02)
(1471,5.334e-03)  (5503,2.027e-03)   (18943,7.415e-04)
(61183,2.628e-04) (187903,9.063e-05) (553983,3.053e-05)
};

\addplot coordinates{
(13,5.755e-02)     (97,2.925e-02)     (545,1.351e-02)
(2561,5.842e-03)   (10625,2.397e-03)  (40193,9.414e-04)
(141569,3.564e-04) (471041,1.308e-04)
(1496065,4.670e-05)
};
}%


\bookmaketitle
\tableofcontents
\include{pgfplots.title_abstract_intro}
\include{pgfplots.preliminaries}
\include{pgfplots.intro}
\include{pgfplots.reference}
\include{pgfplots.libs}
\include{pgfplots.resources}
\include{pgfplots.importexport}
\include{pgfplots.basic.reference}

\printindex

\bibliographystyle{abbrv} %gerapali} %gerabbrv} %gerunsrt.bst} %gerabbrv}% gerplain}
\nocite{pgfplotstable}
\nocite{programmingnotes}
\bibliography{pgfplots}
\end{document}

% -----------------------------------------------------------------------------
% For Stefan Pinnow as reminder on what to look for when editing the manual
% -----------------------------------------------------------------------------
% There should be no line breaks in the following environments
% - |...|
% - \declareandlabel{...}
% - \verbpdfref{...}
% If "MakeTikzPictures" isn't running through check one of the externalized
% LOG files what is the cause of that.
% -----------------------------------------------------------------------------


%%%%%%%%%%%%%%%%%%%%%%%%%%%%%%%%%%%%%%%%%%%%%%%%%%%%%%%%%%%%%%%%%%%%%%%%%%%%%
%
% Package pgfplots.sty documentation.
%
% Copyright 2007/2008 by Christian Feuersaenger.
%
% This program is free software: you can redistribute it and/or modify
% it under the terms of the GNU General Public License as published by
% the Free Software Foundation, either version 3 of the License, or
% (at your option) any later version.
%
% This program is distributed in the hope that it will be useful,
% but WITHOUT ANY WARRANTY; without even the implied warranty of
% MERCHANTABILITY or FITNESS FOR A PARTICULAR PURPOSE.  See the
% GNU General Public License for more details.
%
% You should have received a copy of the GNU General Public License
% along with this program.  If not, see <http://www.gnu.org/licenses/>.
%
%
%%%%%%%%%%%%%%%%%%%%%%%%%%%%%%%%%%%%%%%%%%%%%%%%%%%%%%%%%%%%%%%%%%%%%%%%%%%%%
% SEE pgfplots-macros.tex as well!
%\pdfminorversion=5 % to allow compression
%\pdfobjcompresslevel=2
\documentclass[a4paper,openany]{book}

\let\bookmaketitle=\maketitle

% -----------------------------------
% this here is from ltxdoc:
\usepackage{doc}[=v2]
\AtBeginDocument{\MakeShortVerb{\|}}
%\setlength{\textwidth}{355pt}
%\addtolength\marginparwidth{30pt}
%\addtolength\oddsidemargin{20pt}
%\addtolength\evensidemargin{20pt}
\def\cmd#1{\cs{\expandafter\cmd@to@cs\string#1}}
\def\cmd@to@cs#1#2{\char\number`#2\relax}
\DeclareRobustCommand\cs[1]{\texttt{\char`\\#1}}
\providecommand\marg[1]{%
  {\ttfamily\char`\{}\meta{#1}{\ttfamily\char`\}}}
\providecommand\oarg[1]{%
  {\ttfamily[}\meta{#1}{\ttfamily]}}
\providecommand\parg[1]{%
  {\ttfamily(}\meta{#1}{\ttfamily)}}
\raggedbottom

% -----------------------------------
\input{pgfplots.preamble.tex}

\makeatletter
% I want a two-column index just as in pgfmanual styles. This here
% was the best way to get one:
\def\index@prologue{\section*{Index}\addcontentsline{toc}{chapter}{Index}
}
\makeatother

%\RequirePackage[german,english,francais]{babel}

\def\matlabcolormaptext{This colormap is similar to one shipped with Matlab$^\text{\textregistered}$ under a similar name.}

\IfFileExists{tikzlibraryspy.code.tex}{%
\usetikzlibrary{spy}
}{%
    \message{ERROR: tikz SPY library NOT available. The manual will only compile partially.^^J}%
}%

\usepackage{xparse}% for colorbrewer manual

\usetikzlibrary{
    decorations.markings,
    decorations.footprints,
    shapes.arrows,
    matrix,
    positioning,
}

\usepgfplotslibrary{
    fillbetween,
    ternary,
    smithchart,
    patchplots,
    polar,
    colormaps,
    colorbrewer,
    colortol,           % docu not ready yet
}
\pgfqkeys{/codeexample}{%
    every codeexample/.append style={
        /pgfplots/every ternary axis/.append style={
            /pgfplots/legend style={fill=graphicbackground},
        }
    },
    tabsize=4,
}

\pgfplotsmanualenableexternalizationofexpensive

%\usetikzlibrary{external}
%\tikzexternalize[prefix=figures/]{pgfplots}

\title{%
    Manual for Package \PGFPlots{}\\
    {\small 2D/3D Plots in \LaTeX{}, Version \pgfplotsversion}\\
    {\small\url{https://github.com/pgf-tikz/pgfplots}}
    %\\{\small Attention: you are using an unstable development version.}
}

\makeatletter
\long\def\abstractsmuggle{%
    \centering
    \textbf{Abstract}\\[0.5cm]

    \begin{minipage}{12cm}
        \PGFPlots{} draws high-quality function plots in normal or logarithmic
        scaling with a user-friendly interface directly in \TeX{}. The user
        supplies axis labels, legend entries and the plot coordinates for one or
        more plots and \PGFPlots{} applies axis scaling, computes any logarithms
        and axis ticks and draws the plots. It supports line plots, scatter
        plots, piecewise constant plots, bar plots, area plots, mesh and surface
        plots, patch plots, contour plots, quiver plots, histogram plots, box
        plots, polar axes, ternary diagrams, smith charts and some more. It is
        based on Till Tantau's package \PGF{}/\Tikz{}.
    \end{minipage}

}%

\expandafter\date\expandafter{\@date\\[2cm]
    \abstractsmuggle
}%
\makeatother

%\includeonly{pgfplots.reference}


\begin{document}

\def\plotcoords{%
\addplot coordinates {
(5,8.312e-02)    (17,2.547e-02)   (49,7.407e-03)
(129,2.102e-03)  (321,5.874e-04)  (769,1.623e-04)
(1793,4.442e-05) (4097,1.207e-05) (9217,3.261e-06)
};

\addplot coordinates{
(7,8.472e-02)    (31,3.044e-02)    (111,1.022e-02)
(351,3.303e-03)  (1023,1.039e-03)  (2815,3.196e-04)
(7423,9.658e-05) (18943,2.873e-05) (47103,8.437e-06)
};

\addplot coordinates{
(9,7.881e-02)     (49,3.243e-02)    (209,1.232e-02)
(769,4.454e-03)   (2561,1.551e-03)  (7937,5.236e-04)
(23297,1.723e-04) (65537,5.545e-05) (178177,1.751e-05)
};

\addplot coordinates{
(11,6.887e-02)    (71,3.177e-02)     (351,1.341e-02)
(1471,5.334e-03)  (5503,2.027e-03)   (18943,7.415e-04)
(61183,2.628e-04) (187903,9.063e-05) (553983,3.053e-05)
};

\addplot coordinates{
(13,5.755e-02)     (97,2.925e-02)     (545,1.351e-02)
(2561,5.842e-03)   (10625,2.397e-03)  (40193,9.414e-04)
(141569,3.564e-04) (471041,1.308e-04)
(1496065,4.670e-05)
};
}%


\bookmaketitle
\tableofcontents
\include{pgfplots.title_abstract_intro}
\include{pgfplots.preliminaries}
\include{pgfplots.intro}
\include{pgfplots.reference}
\include{pgfplots.libs}
\include{pgfplots.resources}
\include{pgfplots.importexport}
\include{pgfplots.basic.reference}

\printindex

\bibliographystyle{abbrv} %gerapali} %gerabbrv} %gerunsrt.bst} %gerabbrv}% gerplain}
\nocite{pgfplotstable}
\nocite{programmingnotes}
\bibliography{pgfplots}
\end{document}

% -----------------------------------------------------------------------------
% For Stefan Pinnow as reminder on what to look for when editing the manual
% -----------------------------------------------------------------------------
% There should be no line breaks in the following environments
% - |...|
% - \declareandlabel{...}
% - \verbpdfref{...}
% If "MakeTikzPictures" isn't running through check one of the externalized
% LOG files what is the cause of that.
% -----------------------------------------------------------------------------


%%%%%%%%%%%%%%%%%%%%%%%%%%%%%%%%%%%%%%%%%%%%%%%%%%%%%%%%%%%%%%%%%%%%%%%%%%%%%
%
% Package pgfplots.sty documentation.
%
% Copyright 2007/2008 by Christian Feuersaenger.
%
% This program is free software: you can redistribute it and/or modify
% it under the terms of the GNU General Public License as published by
% the Free Software Foundation, either version 3 of the License, or
% (at your option) any later version.
%
% This program is distributed in the hope that it will be useful,
% but WITHOUT ANY WARRANTY; without even the implied warranty of
% MERCHANTABILITY or FITNESS FOR A PARTICULAR PURPOSE.  See the
% GNU General Public License for more details.
%
% You should have received a copy of the GNU General Public License
% along with this program.  If not, see <http://www.gnu.org/licenses/>.
%
%
%%%%%%%%%%%%%%%%%%%%%%%%%%%%%%%%%%%%%%%%%%%%%%%%%%%%%%%%%%%%%%%%%%%%%%%%%%%%%
% SEE pgfplots-macros.tex as well!
%\pdfminorversion=5 % to allow compression
%\pdfobjcompresslevel=2
\documentclass[a4paper,openany]{book}

\let\bookmaketitle=\maketitle

% -----------------------------------
% this here is from ltxdoc:
\usepackage{doc}[=v2]
\AtBeginDocument{\MakeShortVerb{\|}}
%\setlength{\textwidth}{355pt}
%\addtolength\marginparwidth{30pt}
%\addtolength\oddsidemargin{20pt}
%\addtolength\evensidemargin{20pt}
\def\cmd#1{\cs{\expandafter\cmd@to@cs\string#1}}
\def\cmd@to@cs#1#2{\char\number`#2\relax}
\DeclareRobustCommand\cs[1]{\texttt{\char`\\#1}}
\providecommand\marg[1]{%
  {\ttfamily\char`\{}\meta{#1}{\ttfamily\char`\}}}
\providecommand\oarg[1]{%
  {\ttfamily[}\meta{#1}{\ttfamily]}}
\providecommand\parg[1]{%
  {\ttfamily(}\meta{#1}{\ttfamily)}}
\raggedbottom

% -----------------------------------
\input{pgfplots.preamble.tex}

\makeatletter
% I want a two-column index just as in pgfmanual styles. This here
% was the best way to get one:
\def\index@prologue{\section*{Index}\addcontentsline{toc}{chapter}{Index}
}
\makeatother

%\RequirePackage[german,english,francais]{babel}

\def\matlabcolormaptext{This colormap is similar to one shipped with Matlab$^\text{\textregistered}$ under a similar name.}

\IfFileExists{tikzlibraryspy.code.tex}{%
\usetikzlibrary{spy}
}{%
    \message{ERROR: tikz SPY library NOT available. The manual will only compile partially.^^J}%
}%

\usepackage{xparse}% for colorbrewer manual

\usetikzlibrary{
    decorations.markings,
    decorations.footprints,
    shapes.arrows,
    matrix,
    positioning,
}

\usepgfplotslibrary{
    fillbetween,
    ternary,
    smithchart,
    patchplots,
    polar,
    colormaps,
    colorbrewer,
    colortol,           % docu not ready yet
}
\pgfqkeys{/codeexample}{%
    every codeexample/.append style={
        /pgfplots/every ternary axis/.append style={
            /pgfplots/legend style={fill=graphicbackground},
        }
    },
    tabsize=4,
}

\pgfplotsmanualenableexternalizationofexpensive

%\usetikzlibrary{external}
%\tikzexternalize[prefix=figures/]{pgfplots}

\title{%
    Manual for Package \PGFPlots{}\\
    {\small 2D/3D Plots in \LaTeX{}, Version \pgfplotsversion}\\
    {\small\url{https://github.com/pgf-tikz/pgfplots}}
    %\\{\small Attention: you are using an unstable development version.}
}

\makeatletter
\long\def\abstractsmuggle{%
    \centering
    \textbf{Abstract}\\[0.5cm]

    \begin{minipage}{12cm}
        \PGFPlots{} draws high-quality function plots in normal or logarithmic
        scaling with a user-friendly interface directly in \TeX{}. The user
        supplies axis labels, legend entries and the plot coordinates for one or
        more plots and \PGFPlots{} applies axis scaling, computes any logarithms
        and axis ticks and draws the plots. It supports line plots, scatter
        plots, piecewise constant plots, bar plots, area plots, mesh and surface
        plots, patch plots, contour plots, quiver plots, histogram plots, box
        plots, polar axes, ternary diagrams, smith charts and some more. It is
        based on Till Tantau's package \PGF{}/\Tikz{}.
    \end{minipage}

}%

\expandafter\date\expandafter{\@date\\[2cm]
    \abstractsmuggle
}%
\makeatother

%\includeonly{pgfplots.reference}


\begin{document}

\def\plotcoords{%
\addplot coordinates {
(5,8.312e-02)    (17,2.547e-02)   (49,7.407e-03)
(129,2.102e-03)  (321,5.874e-04)  (769,1.623e-04)
(1793,4.442e-05) (4097,1.207e-05) (9217,3.261e-06)
};

\addplot coordinates{
(7,8.472e-02)    (31,3.044e-02)    (111,1.022e-02)
(351,3.303e-03)  (1023,1.039e-03)  (2815,3.196e-04)
(7423,9.658e-05) (18943,2.873e-05) (47103,8.437e-06)
};

\addplot coordinates{
(9,7.881e-02)     (49,3.243e-02)    (209,1.232e-02)
(769,4.454e-03)   (2561,1.551e-03)  (7937,5.236e-04)
(23297,1.723e-04) (65537,5.545e-05) (178177,1.751e-05)
};

\addplot coordinates{
(11,6.887e-02)    (71,3.177e-02)     (351,1.341e-02)
(1471,5.334e-03)  (5503,2.027e-03)   (18943,7.415e-04)
(61183,2.628e-04) (187903,9.063e-05) (553983,3.053e-05)
};

\addplot coordinates{
(13,5.755e-02)     (97,2.925e-02)     (545,1.351e-02)
(2561,5.842e-03)   (10625,2.397e-03)  (40193,9.414e-04)
(141569,3.564e-04) (471041,1.308e-04)
(1496065,4.670e-05)
};
}%


\bookmaketitle
\tableofcontents
\include{pgfplots.title_abstract_intro}
\include{pgfplots.preliminaries}
\include{pgfplots.intro}
\include{pgfplots.reference}
\include{pgfplots.libs}
\include{pgfplots.resources}
\include{pgfplots.importexport}
\include{pgfplots.basic.reference}

\printindex

\bibliographystyle{abbrv} %gerapali} %gerabbrv} %gerunsrt.bst} %gerabbrv}% gerplain}
\nocite{pgfplotstable}
\nocite{programmingnotes}
\bibliography{pgfplots}
\end{document}

% -----------------------------------------------------------------------------
% For Stefan Pinnow as reminder on what to look for when editing the manual
% -----------------------------------------------------------------------------
% There should be no line breaks in the following environments
% - |...|
% - \declareandlabel{...}
% - \verbpdfref{...}
% If "MakeTikzPictures" isn't running through check one of the externalized
% LOG files what is the cause of that.
% -----------------------------------------------------------------------------

\include{pgfplots.basic.reference}

\printindex

\bibliographystyle{abbrv} %gerapali} %gerabbrv} %gerunsrt.bst} %gerabbrv}% gerplain}
\nocite{pgfplotstable}
\nocite{programmingnotes}
\bibliography{pgfplots}
\end{document}

% -----------------------------------------------------------------------------
% For Stefan Pinnow as reminder on what to look for when editing the manual
% -----------------------------------------------------------------------------
% There should be no line breaks in the following environments
% - |...|
% - \declareandlabel{...}
% - \verbpdfref{...}
% If "MakeTikzPictures" isn't running through check one of the externalized
% LOG files what is the cause of that.
% -----------------------------------------------------------------------------


%%%%%%%%%%%%%%%%%%%%%%%%%%%%%%%%%%%%%%%%%%%%%%%%%%%%%%%%%%%%%%%%%%%%%%%%%%%%%
%
% Package pgfplots.sty documentation.
%
% Copyright 2007/2008 by Christian Feuersaenger.
%
% This program is free software: you can redistribute it and/or modify
% it under the terms of the GNU General Public License as published by
% the Free Software Foundation, either version 3 of the License, or
% (at your option) any later version.
%
% This program is distributed in the hope that it will be useful,
% but WITHOUT ANY WARRANTY; without even the implied warranty of
% MERCHANTABILITY or FITNESS FOR A PARTICULAR PURPOSE.  See the
% GNU General Public License for more details.
%
% You should have received a copy of the GNU General Public License
% along with this program.  If not, see <http://www.gnu.org/licenses/>.
%
%
%%%%%%%%%%%%%%%%%%%%%%%%%%%%%%%%%%%%%%%%%%%%%%%%%%%%%%%%%%%%%%%%%%%%%%%%%%%%%
% SEE pgfplots-macros.tex as well!
%\pdfminorversion=5 % to allow compression
%\pdfobjcompresslevel=2
\documentclass[a4paper,openany]{book}

\let\bookmaketitle=\maketitle

% -----------------------------------
% this here is from ltxdoc:
\usepackage{doc}[=v2]
\AtBeginDocument{\MakeShortVerb{\|}}
%\setlength{\textwidth}{355pt}
%\addtolength\marginparwidth{30pt}
%\addtolength\oddsidemargin{20pt}
%\addtolength\evensidemargin{20pt}
\def\cmd#1{\cs{\expandafter\cmd@to@cs\string#1}}
\def\cmd@to@cs#1#2{\char\number`#2\relax}
\DeclareRobustCommand\cs[1]{\texttt{\char`\\#1}}
\providecommand\marg[1]{%
  {\ttfamily\char`\{}\meta{#1}{\ttfamily\char`\}}}
\providecommand\oarg[1]{%
  {\ttfamily[}\meta{#1}{\ttfamily]}}
\providecommand\parg[1]{%
  {\ttfamily(}\meta{#1}{\ttfamily)}}
\raggedbottom

% -----------------------------------
\input{pgfplots.preamble.tex}

\makeatletter
% I want a two-column index just as in pgfmanual styles. This here
% was the best way to get one:
\def\index@prologue{\section*{Index}\addcontentsline{toc}{chapter}{Index}
}
\makeatother

%\RequirePackage[german,english,francais]{babel}

\def\matlabcolormaptext{This colormap is similar to one shipped with Matlab$^\text{\textregistered}$ under a similar name.}

\IfFileExists{tikzlibraryspy.code.tex}{%
\usetikzlibrary{spy}
}{%
    \message{ERROR: tikz SPY library NOT available. The manual will only compile partially.^^J}%
}%

\usepackage{xparse}% for colorbrewer manual

\usetikzlibrary{
    decorations.markings,
    decorations.footprints,
    shapes.arrows,
    matrix,
    positioning,
}

\usepgfplotslibrary{
    fillbetween,
    ternary,
    smithchart,
    patchplots,
    polar,
    colormaps,
    colorbrewer,
    colortol,           % docu not ready yet
}
\pgfqkeys{/codeexample}{%
    every codeexample/.append style={
        /pgfplots/every ternary axis/.append style={
            /pgfplots/legend style={fill=graphicbackground},
        }
    },
    tabsize=4,
}

\pgfplotsmanualenableexternalizationofexpensive

%\usetikzlibrary{external}
%\tikzexternalize[prefix=figures/]{pgfplots}

\title{%
    Manual for Package \PGFPlots{}\\
    {\small 2D/3D Plots in \LaTeX{}, Version \pgfplotsversion}\\
    {\small\url{https://github.com/pgf-tikz/pgfplots}}
    %\\{\small Attention: you are using an unstable development version.}
}

\makeatletter
\long\def\abstractsmuggle{%
    \centering
    \textbf{Abstract}\\[0.5cm]

    \begin{minipage}{12cm}
        \PGFPlots{} draws high-quality function plots in normal or logarithmic
        scaling with a user-friendly interface directly in \TeX{}. The user
        supplies axis labels, legend entries and the plot coordinates for one or
        more plots and \PGFPlots{} applies axis scaling, computes any logarithms
        and axis ticks and draws the plots. It supports line plots, scatter
        plots, piecewise constant plots, bar plots, area plots, mesh and surface
        plots, patch plots, contour plots, quiver plots, histogram plots, box
        plots, polar axes, ternary diagrams, smith charts and some more. It is
        based on Till Tantau's package \PGF{}/\Tikz{}.
    \end{minipage}

}%

\expandafter\date\expandafter{\@date\\[2cm]
    \abstractsmuggle
}%
\makeatother

%\includeonly{pgfplots.reference}


\begin{document}

\def\plotcoords{%
\addplot coordinates {
(5,8.312e-02)    (17,2.547e-02)   (49,7.407e-03)
(129,2.102e-03)  (321,5.874e-04)  (769,1.623e-04)
(1793,4.442e-05) (4097,1.207e-05) (9217,3.261e-06)
};

\addplot coordinates{
(7,8.472e-02)    (31,3.044e-02)    (111,1.022e-02)
(351,3.303e-03)  (1023,1.039e-03)  (2815,3.196e-04)
(7423,9.658e-05) (18943,2.873e-05) (47103,8.437e-06)
};

\addplot coordinates{
(9,7.881e-02)     (49,3.243e-02)    (209,1.232e-02)
(769,4.454e-03)   (2561,1.551e-03)  (7937,5.236e-04)
(23297,1.723e-04) (65537,5.545e-05) (178177,1.751e-05)
};

\addplot coordinates{
(11,6.887e-02)    (71,3.177e-02)     (351,1.341e-02)
(1471,5.334e-03)  (5503,2.027e-03)   (18943,7.415e-04)
(61183,2.628e-04) (187903,9.063e-05) (553983,3.053e-05)
};

\addplot coordinates{
(13,5.755e-02)     (97,2.925e-02)     (545,1.351e-02)
(2561,5.842e-03)   (10625,2.397e-03)  (40193,9.414e-04)
(141569,3.564e-04) (471041,1.308e-04)
(1496065,4.670e-05)
};
}%


\bookmaketitle
\tableofcontents

%%%%%%%%%%%%%%%%%%%%%%%%%%%%%%%%%%%%%%%%%%%%%%%%%%%%%%%%%%%%%%%%%%%%%%%%%%%%%
%
% Package pgfplots.sty documentation.
%
% Copyright 2007/2008 by Christian Feuersaenger.
%
% This program is free software: you can redistribute it and/or modify
% it under the terms of the GNU General Public License as published by
% the Free Software Foundation, either version 3 of the License, or
% (at your option) any later version.
%
% This program is distributed in the hope that it will be useful,
% but WITHOUT ANY WARRANTY; without even the implied warranty of
% MERCHANTABILITY or FITNESS FOR A PARTICULAR PURPOSE.  See the
% GNU General Public License for more details.
%
% You should have received a copy of the GNU General Public License
% along with this program.  If not, see <http://www.gnu.org/licenses/>.
%
%
%%%%%%%%%%%%%%%%%%%%%%%%%%%%%%%%%%%%%%%%%%%%%%%%%%%%%%%%%%%%%%%%%%%%%%%%%%%%%
% SEE pgfplots-macros.tex as well!
%\pdfminorversion=5 % to allow compression
%\pdfobjcompresslevel=2
\documentclass[a4paper,openany]{book}

\let\bookmaketitle=\maketitle

% -----------------------------------
% this here is from ltxdoc:
\usepackage{doc}[=v2]
\AtBeginDocument{\MakeShortVerb{\|}}
%\setlength{\textwidth}{355pt}
%\addtolength\marginparwidth{30pt}
%\addtolength\oddsidemargin{20pt}
%\addtolength\evensidemargin{20pt}
\def\cmd#1{\cs{\expandafter\cmd@to@cs\string#1}}
\def\cmd@to@cs#1#2{\char\number`#2\relax}
\DeclareRobustCommand\cs[1]{\texttt{\char`\\#1}}
\providecommand\marg[1]{%
  {\ttfamily\char`\{}\meta{#1}{\ttfamily\char`\}}}
\providecommand\oarg[1]{%
  {\ttfamily[}\meta{#1}{\ttfamily]}}
\providecommand\parg[1]{%
  {\ttfamily(}\meta{#1}{\ttfamily)}}
\raggedbottom

% -----------------------------------
\input{pgfplots.preamble.tex}

\makeatletter
% I want a two-column index just as in pgfmanual styles. This here
% was the best way to get one:
\def\index@prologue{\section*{Index}\addcontentsline{toc}{chapter}{Index}
}
\makeatother

%\RequirePackage[german,english,francais]{babel}

\def\matlabcolormaptext{This colormap is similar to one shipped with Matlab$^\text{\textregistered}$ under a similar name.}

\IfFileExists{tikzlibraryspy.code.tex}{%
\usetikzlibrary{spy}
}{%
    \message{ERROR: tikz SPY library NOT available. The manual will only compile partially.^^J}%
}%

\usepackage{xparse}% for colorbrewer manual

\usetikzlibrary{
    decorations.markings,
    decorations.footprints,
    shapes.arrows,
    matrix,
    positioning,
}

\usepgfplotslibrary{
    fillbetween,
    ternary,
    smithchart,
    patchplots,
    polar,
    colormaps,
    colorbrewer,
    colortol,           % docu not ready yet
}
\pgfqkeys{/codeexample}{%
    every codeexample/.append style={
        /pgfplots/every ternary axis/.append style={
            /pgfplots/legend style={fill=graphicbackground},
        }
    },
    tabsize=4,
}

\pgfplotsmanualenableexternalizationofexpensive

%\usetikzlibrary{external}
%\tikzexternalize[prefix=figures/]{pgfplots}

\title{%
    Manual for Package \PGFPlots{}\\
    {\small 2D/3D Plots in \LaTeX{}, Version \pgfplotsversion}\\
    {\small\url{https://github.com/pgf-tikz/pgfplots}}
    %\\{\small Attention: you are using an unstable development version.}
}

\makeatletter
\long\def\abstractsmuggle{%
    \centering
    \textbf{Abstract}\\[0.5cm]

    \begin{minipage}{12cm}
        \PGFPlots{} draws high-quality function plots in normal or logarithmic
        scaling with a user-friendly interface directly in \TeX{}. The user
        supplies axis labels, legend entries and the plot coordinates for one or
        more plots and \PGFPlots{} applies axis scaling, computes any logarithms
        and axis ticks and draws the plots. It supports line plots, scatter
        plots, piecewise constant plots, bar plots, area plots, mesh and surface
        plots, patch plots, contour plots, quiver plots, histogram plots, box
        plots, polar axes, ternary diagrams, smith charts and some more. It is
        based on Till Tantau's package \PGF{}/\Tikz{}.
    \end{minipage}

}%

\expandafter\date\expandafter{\@date\\[2cm]
    \abstractsmuggle
}%
\makeatother

%\includeonly{pgfplots.reference}


\begin{document}

\def\plotcoords{%
\addplot coordinates {
(5,8.312e-02)    (17,2.547e-02)   (49,7.407e-03)
(129,2.102e-03)  (321,5.874e-04)  (769,1.623e-04)
(1793,4.442e-05) (4097,1.207e-05) (9217,3.261e-06)
};

\addplot coordinates{
(7,8.472e-02)    (31,3.044e-02)    (111,1.022e-02)
(351,3.303e-03)  (1023,1.039e-03)  (2815,3.196e-04)
(7423,9.658e-05) (18943,2.873e-05) (47103,8.437e-06)
};

\addplot coordinates{
(9,7.881e-02)     (49,3.243e-02)    (209,1.232e-02)
(769,4.454e-03)   (2561,1.551e-03)  (7937,5.236e-04)
(23297,1.723e-04) (65537,5.545e-05) (178177,1.751e-05)
};

\addplot coordinates{
(11,6.887e-02)    (71,3.177e-02)     (351,1.341e-02)
(1471,5.334e-03)  (5503,2.027e-03)   (18943,7.415e-04)
(61183,2.628e-04) (187903,9.063e-05) (553983,3.053e-05)
};

\addplot coordinates{
(13,5.755e-02)     (97,2.925e-02)     (545,1.351e-02)
(2561,5.842e-03)   (10625,2.397e-03)  (40193,9.414e-04)
(141569,3.564e-04) (471041,1.308e-04)
(1496065,4.670e-05)
};
}%


\bookmaketitle
\tableofcontents
\include{pgfplots.title_abstract_intro}
\include{pgfplots.preliminaries}
\include{pgfplots.intro}
\include{pgfplots.reference}
\include{pgfplots.libs}
\include{pgfplots.resources}
\include{pgfplots.importexport}
\include{pgfplots.basic.reference}

\printindex

\bibliographystyle{abbrv} %gerapali} %gerabbrv} %gerunsrt.bst} %gerabbrv}% gerplain}
\nocite{pgfplotstable}
\nocite{programmingnotes}
\bibliography{pgfplots}
\end{document}

% -----------------------------------------------------------------------------
% For Stefan Pinnow as reminder on what to look for when editing the manual
% -----------------------------------------------------------------------------
% There should be no line breaks in the following environments
% - |...|
% - \declareandlabel{...}
% - \verbpdfref{...}
% If "MakeTikzPictures" isn't running through check one of the externalized
% LOG files what is the cause of that.
% -----------------------------------------------------------------------------


%%%%%%%%%%%%%%%%%%%%%%%%%%%%%%%%%%%%%%%%%%%%%%%%%%%%%%%%%%%%%%%%%%%%%%%%%%%%%
%
% Package pgfplots.sty documentation.
%
% Copyright 2007/2008 by Christian Feuersaenger.
%
% This program is free software: you can redistribute it and/or modify
% it under the terms of the GNU General Public License as published by
% the Free Software Foundation, either version 3 of the License, or
% (at your option) any later version.
%
% This program is distributed in the hope that it will be useful,
% but WITHOUT ANY WARRANTY; without even the implied warranty of
% MERCHANTABILITY or FITNESS FOR A PARTICULAR PURPOSE.  See the
% GNU General Public License for more details.
%
% You should have received a copy of the GNU General Public License
% along with this program.  If not, see <http://www.gnu.org/licenses/>.
%
%
%%%%%%%%%%%%%%%%%%%%%%%%%%%%%%%%%%%%%%%%%%%%%%%%%%%%%%%%%%%%%%%%%%%%%%%%%%%%%
% SEE pgfplots-macros.tex as well!
%\pdfminorversion=5 % to allow compression
%\pdfobjcompresslevel=2
\documentclass[a4paper,openany]{book}

\let\bookmaketitle=\maketitle

% -----------------------------------
% this here is from ltxdoc:
\usepackage{doc}[=v2]
\AtBeginDocument{\MakeShortVerb{\|}}
%\setlength{\textwidth}{355pt}
%\addtolength\marginparwidth{30pt}
%\addtolength\oddsidemargin{20pt}
%\addtolength\evensidemargin{20pt}
\def\cmd#1{\cs{\expandafter\cmd@to@cs\string#1}}
\def\cmd@to@cs#1#2{\char\number`#2\relax}
\DeclareRobustCommand\cs[1]{\texttt{\char`\\#1}}
\providecommand\marg[1]{%
  {\ttfamily\char`\{}\meta{#1}{\ttfamily\char`\}}}
\providecommand\oarg[1]{%
  {\ttfamily[}\meta{#1}{\ttfamily]}}
\providecommand\parg[1]{%
  {\ttfamily(}\meta{#1}{\ttfamily)}}
\raggedbottom

% -----------------------------------
\input{pgfplots.preamble.tex}

\makeatletter
% I want a two-column index just as in pgfmanual styles. This here
% was the best way to get one:
\def\index@prologue{\section*{Index}\addcontentsline{toc}{chapter}{Index}
}
\makeatother

%\RequirePackage[german,english,francais]{babel}

\def\matlabcolormaptext{This colormap is similar to one shipped with Matlab$^\text{\textregistered}$ under a similar name.}

\IfFileExists{tikzlibraryspy.code.tex}{%
\usetikzlibrary{spy}
}{%
    \message{ERROR: tikz SPY library NOT available. The manual will only compile partially.^^J}%
}%

\usepackage{xparse}% for colorbrewer manual

\usetikzlibrary{
    decorations.markings,
    decorations.footprints,
    shapes.arrows,
    matrix,
    positioning,
}

\usepgfplotslibrary{
    fillbetween,
    ternary,
    smithchart,
    patchplots,
    polar,
    colormaps,
    colorbrewer,
    colortol,           % docu not ready yet
}
\pgfqkeys{/codeexample}{%
    every codeexample/.append style={
        /pgfplots/every ternary axis/.append style={
            /pgfplots/legend style={fill=graphicbackground},
        }
    },
    tabsize=4,
}

\pgfplotsmanualenableexternalizationofexpensive

%\usetikzlibrary{external}
%\tikzexternalize[prefix=figures/]{pgfplots}

\title{%
    Manual for Package \PGFPlots{}\\
    {\small 2D/3D Plots in \LaTeX{}, Version \pgfplotsversion}\\
    {\small\url{https://github.com/pgf-tikz/pgfplots}}
    %\\{\small Attention: you are using an unstable development version.}
}

\makeatletter
\long\def\abstractsmuggle{%
    \centering
    \textbf{Abstract}\\[0.5cm]

    \begin{minipage}{12cm}
        \PGFPlots{} draws high-quality function plots in normal or logarithmic
        scaling with a user-friendly interface directly in \TeX{}. The user
        supplies axis labels, legend entries and the plot coordinates for one or
        more plots and \PGFPlots{} applies axis scaling, computes any logarithms
        and axis ticks and draws the plots. It supports line plots, scatter
        plots, piecewise constant plots, bar plots, area plots, mesh and surface
        plots, patch plots, contour plots, quiver plots, histogram plots, box
        plots, polar axes, ternary diagrams, smith charts and some more. It is
        based on Till Tantau's package \PGF{}/\Tikz{}.
    \end{minipage}

}%

\expandafter\date\expandafter{\@date\\[2cm]
    \abstractsmuggle
}%
\makeatother

%\includeonly{pgfplots.reference}


\begin{document}

\def\plotcoords{%
\addplot coordinates {
(5,8.312e-02)    (17,2.547e-02)   (49,7.407e-03)
(129,2.102e-03)  (321,5.874e-04)  (769,1.623e-04)
(1793,4.442e-05) (4097,1.207e-05) (9217,3.261e-06)
};

\addplot coordinates{
(7,8.472e-02)    (31,3.044e-02)    (111,1.022e-02)
(351,3.303e-03)  (1023,1.039e-03)  (2815,3.196e-04)
(7423,9.658e-05) (18943,2.873e-05) (47103,8.437e-06)
};

\addplot coordinates{
(9,7.881e-02)     (49,3.243e-02)    (209,1.232e-02)
(769,4.454e-03)   (2561,1.551e-03)  (7937,5.236e-04)
(23297,1.723e-04) (65537,5.545e-05) (178177,1.751e-05)
};

\addplot coordinates{
(11,6.887e-02)    (71,3.177e-02)     (351,1.341e-02)
(1471,5.334e-03)  (5503,2.027e-03)   (18943,7.415e-04)
(61183,2.628e-04) (187903,9.063e-05) (553983,3.053e-05)
};

\addplot coordinates{
(13,5.755e-02)     (97,2.925e-02)     (545,1.351e-02)
(2561,5.842e-03)   (10625,2.397e-03)  (40193,9.414e-04)
(141569,3.564e-04) (471041,1.308e-04)
(1496065,4.670e-05)
};
}%


\bookmaketitle
\tableofcontents
\include{pgfplots.title_abstract_intro}
\include{pgfplots.preliminaries}
\include{pgfplots.intro}
\include{pgfplots.reference}
\include{pgfplots.libs}
\include{pgfplots.resources}
\include{pgfplots.importexport}
\include{pgfplots.basic.reference}

\printindex

\bibliographystyle{abbrv} %gerapali} %gerabbrv} %gerunsrt.bst} %gerabbrv}% gerplain}
\nocite{pgfplotstable}
\nocite{programmingnotes}
\bibliography{pgfplots}
\end{document}

% -----------------------------------------------------------------------------
% For Stefan Pinnow as reminder on what to look for when editing the manual
% -----------------------------------------------------------------------------
% There should be no line breaks in the following environments
% - |...|
% - \declareandlabel{...}
% - \verbpdfref{...}
% If "MakeTikzPictures" isn't running through check one of the externalized
% LOG files what is the cause of that.
% -----------------------------------------------------------------------------


%%%%%%%%%%%%%%%%%%%%%%%%%%%%%%%%%%%%%%%%%%%%%%%%%%%%%%%%%%%%%%%%%%%%%%%%%%%%%
%
% Package pgfplots.sty documentation.
%
% Copyright 2007/2008 by Christian Feuersaenger.
%
% This program is free software: you can redistribute it and/or modify
% it under the terms of the GNU General Public License as published by
% the Free Software Foundation, either version 3 of the License, or
% (at your option) any later version.
%
% This program is distributed in the hope that it will be useful,
% but WITHOUT ANY WARRANTY; without even the implied warranty of
% MERCHANTABILITY or FITNESS FOR A PARTICULAR PURPOSE.  See the
% GNU General Public License for more details.
%
% You should have received a copy of the GNU General Public License
% along with this program.  If not, see <http://www.gnu.org/licenses/>.
%
%
%%%%%%%%%%%%%%%%%%%%%%%%%%%%%%%%%%%%%%%%%%%%%%%%%%%%%%%%%%%%%%%%%%%%%%%%%%%%%
% SEE pgfplots-macros.tex as well!
%\pdfminorversion=5 % to allow compression
%\pdfobjcompresslevel=2
\documentclass[a4paper,openany]{book}

\let\bookmaketitle=\maketitle

% -----------------------------------
% this here is from ltxdoc:
\usepackage{doc}[=v2]
\AtBeginDocument{\MakeShortVerb{\|}}
%\setlength{\textwidth}{355pt}
%\addtolength\marginparwidth{30pt}
%\addtolength\oddsidemargin{20pt}
%\addtolength\evensidemargin{20pt}
\def\cmd#1{\cs{\expandafter\cmd@to@cs\string#1}}
\def\cmd@to@cs#1#2{\char\number`#2\relax}
\DeclareRobustCommand\cs[1]{\texttt{\char`\\#1}}
\providecommand\marg[1]{%
  {\ttfamily\char`\{}\meta{#1}{\ttfamily\char`\}}}
\providecommand\oarg[1]{%
  {\ttfamily[}\meta{#1}{\ttfamily]}}
\providecommand\parg[1]{%
  {\ttfamily(}\meta{#1}{\ttfamily)}}
\raggedbottom

% -----------------------------------
\input{pgfplots.preamble.tex}

\makeatletter
% I want a two-column index just as in pgfmanual styles. This here
% was the best way to get one:
\def\index@prologue{\section*{Index}\addcontentsline{toc}{chapter}{Index}
}
\makeatother

%\RequirePackage[german,english,francais]{babel}

\def\matlabcolormaptext{This colormap is similar to one shipped with Matlab$^\text{\textregistered}$ under a similar name.}

\IfFileExists{tikzlibraryspy.code.tex}{%
\usetikzlibrary{spy}
}{%
    \message{ERROR: tikz SPY library NOT available. The manual will only compile partially.^^J}%
}%

\usepackage{xparse}% for colorbrewer manual

\usetikzlibrary{
    decorations.markings,
    decorations.footprints,
    shapes.arrows,
    matrix,
    positioning,
}

\usepgfplotslibrary{
    fillbetween,
    ternary,
    smithchart,
    patchplots,
    polar,
    colormaps,
    colorbrewer,
    colortol,           % docu not ready yet
}
\pgfqkeys{/codeexample}{%
    every codeexample/.append style={
        /pgfplots/every ternary axis/.append style={
            /pgfplots/legend style={fill=graphicbackground},
        }
    },
    tabsize=4,
}

\pgfplotsmanualenableexternalizationofexpensive

%\usetikzlibrary{external}
%\tikzexternalize[prefix=figures/]{pgfplots}

\title{%
    Manual for Package \PGFPlots{}\\
    {\small 2D/3D Plots in \LaTeX{}, Version \pgfplotsversion}\\
    {\small\url{https://github.com/pgf-tikz/pgfplots}}
    %\\{\small Attention: you are using an unstable development version.}
}

\makeatletter
\long\def\abstractsmuggle{%
    \centering
    \textbf{Abstract}\\[0.5cm]

    \begin{minipage}{12cm}
        \PGFPlots{} draws high-quality function plots in normal or logarithmic
        scaling with a user-friendly interface directly in \TeX{}. The user
        supplies axis labels, legend entries and the plot coordinates for one or
        more plots and \PGFPlots{} applies axis scaling, computes any logarithms
        and axis ticks and draws the plots. It supports line plots, scatter
        plots, piecewise constant plots, bar plots, area plots, mesh and surface
        plots, patch plots, contour plots, quiver plots, histogram plots, box
        plots, polar axes, ternary diagrams, smith charts and some more. It is
        based on Till Tantau's package \PGF{}/\Tikz{}.
    \end{minipage}

}%

\expandafter\date\expandafter{\@date\\[2cm]
    \abstractsmuggle
}%
\makeatother

%\includeonly{pgfplots.reference}


\begin{document}

\def\plotcoords{%
\addplot coordinates {
(5,8.312e-02)    (17,2.547e-02)   (49,7.407e-03)
(129,2.102e-03)  (321,5.874e-04)  (769,1.623e-04)
(1793,4.442e-05) (4097,1.207e-05) (9217,3.261e-06)
};

\addplot coordinates{
(7,8.472e-02)    (31,3.044e-02)    (111,1.022e-02)
(351,3.303e-03)  (1023,1.039e-03)  (2815,3.196e-04)
(7423,9.658e-05) (18943,2.873e-05) (47103,8.437e-06)
};

\addplot coordinates{
(9,7.881e-02)     (49,3.243e-02)    (209,1.232e-02)
(769,4.454e-03)   (2561,1.551e-03)  (7937,5.236e-04)
(23297,1.723e-04) (65537,5.545e-05) (178177,1.751e-05)
};

\addplot coordinates{
(11,6.887e-02)    (71,3.177e-02)     (351,1.341e-02)
(1471,5.334e-03)  (5503,2.027e-03)   (18943,7.415e-04)
(61183,2.628e-04) (187903,9.063e-05) (553983,3.053e-05)
};

\addplot coordinates{
(13,5.755e-02)     (97,2.925e-02)     (545,1.351e-02)
(2561,5.842e-03)   (10625,2.397e-03)  (40193,9.414e-04)
(141569,3.564e-04) (471041,1.308e-04)
(1496065,4.670e-05)
};
}%


\bookmaketitle
\tableofcontents
\include{pgfplots.title_abstract_intro}
\include{pgfplots.preliminaries}
\include{pgfplots.intro}
\include{pgfplots.reference}
\include{pgfplots.libs}
\include{pgfplots.resources}
\include{pgfplots.importexport}
\include{pgfplots.basic.reference}

\printindex

\bibliographystyle{abbrv} %gerapali} %gerabbrv} %gerunsrt.bst} %gerabbrv}% gerplain}
\nocite{pgfplotstable}
\nocite{programmingnotes}
\bibliography{pgfplots}
\end{document}

% -----------------------------------------------------------------------------
% For Stefan Pinnow as reminder on what to look for when editing the manual
% -----------------------------------------------------------------------------
% There should be no line breaks in the following environments
% - |...|
% - \declareandlabel{...}
% - \verbpdfref{...}
% If "MakeTikzPictures" isn't running through check one of the externalized
% LOG files what is the cause of that.
% -----------------------------------------------------------------------------


%%%%%%%%%%%%%%%%%%%%%%%%%%%%%%%%%%%%%%%%%%%%%%%%%%%%%%%%%%%%%%%%%%%%%%%%%%%%%
%
% Package pgfplots.sty documentation.
%
% Copyright 2007/2008 by Christian Feuersaenger.
%
% This program is free software: you can redistribute it and/or modify
% it under the terms of the GNU General Public License as published by
% the Free Software Foundation, either version 3 of the License, or
% (at your option) any later version.
%
% This program is distributed in the hope that it will be useful,
% but WITHOUT ANY WARRANTY; without even the implied warranty of
% MERCHANTABILITY or FITNESS FOR A PARTICULAR PURPOSE.  See the
% GNU General Public License for more details.
%
% You should have received a copy of the GNU General Public License
% along with this program.  If not, see <http://www.gnu.org/licenses/>.
%
%
%%%%%%%%%%%%%%%%%%%%%%%%%%%%%%%%%%%%%%%%%%%%%%%%%%%%%%%%%%%%%%%%%%%%%%%%%%%%%
% SEE pgfplots-macros.tex as well!
%\pdfminorversion=5 % to allow compression
%\pdfobjcompresslevel=2
\documentclass[a4paper,openany]{book}

\let\bookmaketitle=\maketitle

% -----------------------------------
% this here is from ltxdoc:
\usepackage{doc}[=v2]
\AtBeginDocument{\MakeShortVerb{\|}}
%\setlength{\textwidth}{355pt}
%\addtolength\marginparwidth{30pt}
%\addtolength\oddsidemargin{20pt}
%\addtolength\evensidemargin{20pt}
\def\cmd#1{\cs{\expandafter\cmd@to@cs\string#1}}
\def\cmd@to@cs#1#2{\char\number`#2\relax}
\DeclareRobustCommand\cs[1]{\texttt{\char`\\#1}}
\providecommand\marg[1]{%
  {\ttfamily\char`\{}\meta{#1}{\ttfamily\char`\}}}
\providecommand\oarg[1]{%
  {\ttfamily[}\meta{#1}{\ttfamily]}}
\providecommand\parg[1]{%
  {\ttfamily(}\meta{#1}{\ttfamily)}}
\raggedbottom

% -----------------------------------
\input{pgfplots.preamble.tex}

\makeatletter
% I want a two-column index just as in pgfmanual styles. This here
% was the best way to get one:
\def\index@prologue{\section*{Index}\addcontentsline{toc}{chapter}{Index}
}
\makeatother

%\RequirePackage[german,english,francais]{babel}

\def\matlabcolormaptext{This colormap is similar to one shipped with Matlab$^\text{\textregistered}$ under a similar name.}

\IfFileExists{tikzlibraryspy.code.tex}{%
\usetikzlibrary{spy}
}{%
    \message{ERROR: tikz SPY library NOT available. The manual will only compile partially.^^J}%
}%

\usepackage{xparse}% for colorbrewer manual

\usetikzlibrary{
    decorations.markings,
    decorations.footprints,
    shapes.arrows,
    matrix,
    positioning,
}

\usepgfplotslibrary{
    fillbetween,
    ternary,
    smithchart,
    patchplots,
    polar,
    colormaps,
    colorbrewer,
    colortol,           % docu not ready yet
}
\pgfqkeys{/codeexample}{%
    every codeexample/.append style={
        /pgfplots/every ternary axis/.append style={
            /pgfplots/legend style={fill=graphicbackground},
        }
    },
    tabsize=4,
}

\pgfplotsmanualenableexternalizationofexpensive

%\usetikzlibrary{external}
%\tikzexternalize[prefix=figures/]{pgfplots}

\title{%
    Manual for Package \PGFPlots{}\\
    {\small 2D/3D Plots in \LaTeX{}, Version \pgfplotsversion}\\
    {\small\url{https://github.com/pgf-tikz/pgfplots}}
    %\\{\small Attention: you are using an unstable development version.}
}

\makeatletter
\long\def\abstractsmuggle{%
    \centering
    \textbf{Abstract}\\[0.5cm]

    \begin{minipage}{12cm}
        \PGFPlots{} draws high-quality function plots in normal or logarithmic
        scaling with a user-friendly interface directly in \TeX{}. The user
        supplies axis labels, legend entries and the plot coordinates for one or
        more plots and \PGFPlots{} applies axis scaling, computes any logarithms
        and axis ticks and draws the plots. It supports line plots, scatter
        plots, piecewise constant plots, bar plots, area plots, mesh and surface
        plots, patch plots, contour plots, quiver plots, histogram plots, box
        plots, polar axes, ternary diagrams, smith charts and some more. It is
        based on Till Tantau's package \PGF{}/\Tikz{}.
    \end{minipage}

}%

\expandafter\date\expandafter{\@date\\[2cm]
    \abstractsmuggle
}%
\makeatother

%\includeonly{pgfplots.reference}


\begin{document}

\def\plotcoords{%
\addplot coordinates {
(5,8.312e-02)    (17,2.547e-02)   (49,7.407e-03)
(129,2.102e-03)  (321,5.874e-04)  (769,1.623e-04)
(1793,4.442e-05) (4097,1.207e-05) (9217,3.261e-06)
};

\addplot coordinates{
(7,8.472e-02)    (31,3.044e-02)    (111,1.022e-02)
(351,3.303e-03)  (1023,1.039e-03)  (2815,3.196e-04)
(7423,9.658e-05) (18943,2.873e-05) (47103,8.437e-06)
};

\addplot coordinates{
(9,7.881e-02)     (49,3.243e-02)    (209,1.232e-02)
(769,4.454e-03)   (2561,1.551e-03)  (7937,5.236e-04)
(23297,1.723e-04) (65537,5.545e-05) (178177,1.751e-05)
};

\addplot coordinates{
(11,6.887e-02)    (71,3.177e-02)     (351,1.341e-02)
(1471,5.334e-03)  (5503,2.027e-03)   (18943,7.415e-04)
(61183,2.628e-04) (187903,9.063e-05) (553983,3.053e-05)
};

\addplot coordinates{
(13,5.755e-02)     (97,2.925e-02)     (545,1.351e-02)
(2561,5.842e-03)   (10625,2.397e-03)  (40193,9.414e-04)
(141569,3.564e-04) (471041,1.308e-04)
(1496065,4.670e-05)
};
}%


\bookmaketitle
\tableofcontents
\include{pgfplots.title_abstract_intro}
\include{pgfplots.preliminaries}
\include{pgfplots.intro}
\include{pgfplots.reference}
\include{pgfplots.libs}
\include{pgfplots.resources}
\include{pgfplots.importexport}
\include{pgfplots.basic.reference}

\printindex

\bibliographystyle{abbrv} %gerapali} %gerabbrv} %gerunsrt.bst} %gerabbrv}% gerplain}
\nocite{pgfplotstable}
\nocite{programmingnotes}
\bibliography{pgfplots}
\end{document}

% -----------------------------------------------------------------------------
% For Stefan Pinnow as reminder on what to look for when editing the manual
% -----------------------------------------------------------------------------
% There should be no line breaks in the following environments
% - |...|
% - \declareandlabel{...}
% - \verbpdfref{...}
% If "MakeTikzPictures" isn't running through check one of the externalized
% LOG files what is the cause of that.
% -----------------------------------------------------------------------------


%%%%%%%%%%%%%%%%%%%%%%%%%%%%%%%%%%%%%%%%%%%%%%%%%%%%%%%%%%%%%%%%%%%%%%%%%%%%%
%
% Package pgfplots.sty documentation.
%
% Copyright 2007/2008 by Christian Feuersaenger.
%
% This program is free software: you can redistribute it and/or modify
% it under the terms of the GNU General Public License as published by
% the Free Software Foundation, either version 3 of the License, or
% (at your option) any later version.
%
% This program is distributed in the hope that it will be useful,
% but WITHOUT ANY WARRANTY; without even the implied warranty of
% MERCHANTABILITY or FITNESS FOR A PARTICULAR PURPOSE.  See the
% GNU General Public License for more details.
%
% You should have received a copy of the GNU General Public License
% along with this program.  If not, see <http://www.gnu.org/licenses/>.
%
%
%%%%%%%%%%%%%%%%%%%%%%%%%%%%%%%%%%%%%%%%%%%%%%%%%%%%%%%%%%%%%%%%%%%%%%%%%%%%%
% SEE pgfplots-macros.tex as well!
%\pdfminorversion=5 % to allow compression
%\pdfobjcompresslevel=2
\documentclass[a4paper,openany]{book}

\let\bookmaketitle=\maketitle

% -----------------------------------
% this here is from ltxdoc:
\usepackage{doc}[=v2]
\AtBeginDocument{\MakeShortVerb{\|}}
%\setlength{\textwidth}{355pt}
%\addtolength\marginparwidth{30pt}
%\addtolength\oddsidemargin{20pt}
%\addtolength\evensidemargin{20pt}
\def\cmd#1{\cs{\expandafter\cmd@to@cs\string#1}}
\def\cmd@to@cs#1#2{\char\number`#2\relax}
\DeclareRobustCommand\cs[1]{\texttt{\char`\\#1}}
\providecommand\marg[1]{%
  {\ttfamily\char`\{}\meta{#1}{\ttfamily\char`\}}}
\providecommand\oarg[1]{%
  {\ttfamily[}\meta{#1}{\ttfamily]}}
\providecommand\parg[1]{%
  {\ttfamily(}\meta{#1}{\ttfamily)}}
\raggedbottom

% -----------------------------------
\input{pgfplots.preamble.tex}

\makeatletter
% I want a two-column index just as in pgfmanual styles. This here
% was the best way to get one:
\def\index@prologue{\section*{Index}\addcontentsline{toc}{chapter}{Index}
}
\makeatother

%\RequirePackage[german,english,francais]{babel}

\def\matlabcolormaptext{This colormap is similar to one shipped with Matlab$^\text{\textregistered}$ under a similar name.}

\IfFileExists{tikzlibraryspy.code.tex}{%
\usetikzlibrary{spy}
}{%
    \message{ERROR: tikz SPY library NOT available. The manual will only compile partially.^^J}%
}%

\usepackage{xparse}% for colorbrewer manual

\usetikzlibrary{
    decorations.markings,
    decorations.footprints,
    shapes.arrows,
    matrix,
    positioning,
}

\usepgfplotslibrary{
    fillbetween,
    ternary,
    smithchart,
    patchplots,
    polar,
    colormaps,
    colorbrewer,
    colortol,           % docu not ready yet
}
\pgfqkeys{/codeexample}{%
    every codeexample/.append style={
        /pgfplots/every ternary axis/.append style={
            /pgfplots/legend style={fill=graphicbackground},
        }
    },
    tabsize=4,
}

\pgfplotsmanualenableexternalizationofexpensive

%\usetikzlibrary{external}
%\tikzexternalize[prefix=figures/]{pgfplots}

\title{%
    Manual for Package \PGFPlots{}\\
    {\small 2D/3D Plots in \LaTeX{}, Version \pgfplotsversion}\\
    {\small\url{https://github.com/pgf-tikz/pgfplots}}
    %\\{\small Attention: you are using an unstable development version.}
}

\makeatletter
\long\def\abstractsmuggle{%
    \centering
    \textbf{Abstract}\\[0.5cm]

    \begin{minipage}{12cm}
        \PGFPlots{} draws high-quality function plots in normal or logarithmic
        scaling with a user-friendly interface directly in \TeX{}. The user
        supplies axis labels, legend entries and the plot coordinates for one or
        more plots and \PGFPlots{} applies axis scaling, computes any logarithms
        and axis ticks and draws the plots. It supports line plots, scatter
        plots, piecewise constant plots, bar plots, area plots, mesh and surface
        plots, patch plots, contour plots, quiver plots, histogram plots, box
        plots, polar axes, ternary diagrams, smith charts and some more. It is
        based on Till Tantau's package \PGF{}/\Tikz{}.
    \end{minipage}

}%

\expandafter\date\expandafter{\@date\\[2cm]
    \abstractsmuggle
}%
\makeatother

%\includeonly{pgfplots.reference}


\begin{document}

\def\plotcoords{%
\addplot coordinates {
(5,8.312e-02)    (17,2.547e-02)   (49,7.407e-03)
(129,2.102e-03)  (321,5.874e-04)  (769,1.623e-04)
(1793,4.442e-05) (4097,1.207e-05) (9217,3.261e-06)
};

\addplot coordinates{
(7,8.472e-02)    (31,3.044e-02)    (111,1.022e-02)
(351,3.303e-03)  (1023,1.039e-03)  (2815,3.196e-04)
(7423,9.658e-05) (18943,2.873e-05) (47103,8.437e-06)
};

\addplot coordinates{
(9,7.881e-02)     (49,3.243e-02)    (209,1.232e-02)
(769,4.454e-03)   (2561,1.551e-03)  (7937,5.236e-04)
(23297,1.723e-04) (65537,5.545e-05) (178177,1.751e-05)
};

\addplot coordinates{
(11,6.887e-02)    (71,3.177e-02)     (351,1.341e-02)
(1471,5.334e-03)  (5503,2.027e-03)   (18943,7.415e-04)
(61183,2.628e-04) (187903,9.063e-05) (553983,3.053e-05)
};

\addplot coordinates{
(13,5.755e-02)     (97,2.925e-02)     (545,1.351e-02)
(2561,5.842e-03)   (10625,2.397e-03)  (40193,9.414e-04)
(141569,3.564e-04) (471041,1.308e-04)
(1496065,4.670e-05)
};
}%


\bookmaketitle
\tableofcontents
\include{pgfplots.title_abstract_intro}
\include{pgfplots.preliminaries}
\include{pgfplots.intro}
\include{pgfplots.reference}
\include{pgfplots.libs}
\include{pgfplots.resources}
\include{pgfplots.importexport}
\include{pgfplots.basic.reference}

\printindex

\bibliographystyle{abbrv} %gerapali} %gerabbrv} %gerunsrt.bst} %gerabbrv}% gerplain}
\nocite{pgfplotstable}
\nocite{programmingnotes}
\bibliography{pgfplots}
\end{document}

% -----------------------------------------------------------------------------
% For Stefan Pinnow as reminder on what to look for when editing the manual
% -----------------------------------------------------------------------------
% There should be no line breaks in the following environments
% - |...|
% - \declareandlabel{...}
% - \verbpdfref{...}
% If "MakeTikzPictures" isn't running through check one of the externalized
% LOG files what is the cause of that.
% -----------------------------------------------------------------------------


%%%%%%%%%%%%%%%%%%%%%%%%%%%%%%%%%%%%%%%%%%%%%%%%%%%%%%%%%%%%%%%%%%%%%%%%%%%%%
%
% Package pgfplots.sty documentation.
%
% Copyright 2007/2008 by Christian Feuersaenger.
%
% This program is free software: you can redistribute it and/or modify
% it under the terms of the GNU General Public License as published by
% the Free Software Foundation, either version 3 of the License, or
% (at your option) any later version.
%
% This program is distributed in the hope that it will be useful,
% but WITHOUT ANY WARRANTY; without even the implied warranty of
% MERCHANTABILITY or FITNESS FOR A PARTICULAR PURPOSE.  See the
% GNU General Public License for more details.
%
% You should have received a copy of the GNU General Public License
% along with this program.  If not, see <http://www.gnu.org/licenses/>.
%
%
%%%%%%%%%%%%%%%%%%%%%%%%%%%%%%%%%%%%%%%%%%%%%%%%%%%%%%%%%%%%%%%%%%%%%%%%%%%%%
% SEE pgfplots-macros.tex as well!
%\pdfminorversion=5 % to allow compression
%\pdfobjcompresslevel=2
\documentclass[a4paper,openany]{book}

\let\bookmaketitle=\maketitle

% -----------------------------------
% this here is from ltxdoc:
\usepackage{doc}[=v2]
\AtBeginDocument{\MakeShortVerb{\|}}
%\setlength{\textwidth}{355pt}
%\addtolength\marginparwidth{30pt}
%\addtolength\oddsidemargin{20pt}
%\addtolength\evensidemargin{20pt}
\def\cmd#1{\cs{\expandafter\cmd@to@cs\string#1}}
\def\cmd@to@cs#1#2{\char\number`#2\relax}
\DeclareRobustCommand\cs[1]{\texttt{\char`\\#1}}
\providecommand\marg[1]{%
  {\ttfamily\char`\{}\meta{#1}{\ttfamily\char`\}}}
\providecommand\oarg[1]{%
  {\ttfamily[}\meta{#1}{\ttfamily]}}
\providecommand\parg[1]{%
  {\ttfamily(}\meta{#1}{\ttfamily)}}
\raggedbottom

% -----------------------------------
\input{pgfplots.preamble.tex}

\makeatletter
% I want a two-column index just as in pgfmanual styles. This here
% was the best way to get one:
\def\index@prologue{\section*{Index}\addcontentsline{toc}{chapter}{Index}
}
\makeatother

%\RequirePackage[german,english,francais]{babel}

\def\matlabcolormaptext{This colormap is similar to one shipped with Matlab$^\text{\textregistered}$ under a similar name.}

\IfFileExists{tikzlibraryspy.code.tex}{%
\usetikzlibrary{spy}
}{%
    \message{ERROR: tikz SPY library NOT available. The manual will only compile partially.^^J}%
}%

\usepackage{xparse}% for colorbrewer manual

\usetikzlibrary{
    decorations.markings,
    decorations.footprints,
    shapes.arrows,
    matrix,
    positioning,
}

\usepgfplotslibrary{
    fillbetween,
    ternary,
    smithchart,
    patchplots,
    polar,
    colormaps,
    colorbrewer,
    colortol,           % docu not ready yet
}
\pgfqkeys{/codeexample}{%
    every codeexample/.append style={
        /pgfplots/every ternary axis/.append style={
            /pgfplots/legend style={fill=graphicbackground},
        }
    },
    tabsize=4,
}

\pgfplotsmanualenableexternalizationofexpensive

%\usetikzlibrary{external}
%\tikzexternalize[prefix=figures/]{pgfplots}

\title{%
    Manual for Package \PGFPlots{}\\
    {\small 2D/3D Plots in \LaTeX{}, Version \pgfplotsversion}\\
    {\small\url{https://github.com/pgf-tikz/pgfplots}}
    %\\{\small Attention: you are using an unstable development version.}
}

\makeatletter
\long\def\abstractsmuggle{%
    \centering
    \textbf{Abstract}\\[0.5cm]

    \begin{minipage}{12cm}
        \PGFPlots{} draws high-quality function plots in normal or logarithmic
        scaling with a user-friendly interface directly in \TeX{}. The user
        supplies axis labels, legend entries and the plot coordinates for one or
        more plots and \PGFPlots{} applies axis scaling, computes any logarithms
        and axis ticks and draws the plots. It supports line plots, scatter
        plots, piecewise constant plots, bar plots, area plots, mesh and surface
        plots, patch plots, contour plots, quiver plots, histogram plots, box
        plots, polar axes, ternary diagrams, smith charts and some more. It is
        based on Till Tantau's package \PGF{}/\Tikz{}.
    \end{minipage}

}%

\expandafter\date\expandafter{\@date\\[2cm]
    \abstractsmuggle
}%
\makeatother

%\includeonly{pgfplots.reference}


\begin{document}

\def\plotcoords{%
\addplot coordinates {
(5,8.312e-02)    (17,2.547e-02)   (49,7.407e-03)
(129,2.102e-03)  (321,5.874e-04)  (769,1.623e-04)
(1793,4.442e-05) (4097,1.207e-05) (9217,3.261e-06)
};

\addplot coordinates{
(7,8.472e-02)    (31,3.044e-02)    (111,1.022e-02)
(351,3.303e-03)  (1023,1.039e-03)  (2815,3.196e-04)
(7423,9.658e-05) (18943,2.873e-05) (47103,8.437e-06)
};

\addplot coordinates{
(9,7.881e-02)     (49,3.243e-02)    (209,1.232e-02)
(769,4.454e-03)   (2561,1.551e-03)  (7937,5.236e-04)
(23297,1.723e-04) (65537,5.545e-05) (178177,1.751e-05)
};

\addplot coordinates{
(11,6.887e-02)    (71,3.177e-02)     (351,1.341e-02)
(1471,5.334e-03)  (5503,2.027e-03)   (18943,7.415e-04)
(61183,2.628e-04) (187903,9.063e-05) (553983,3.053e-05)
};

\addplot coordinates{
(13,5.755e-02)     (97,2.925e-02)     (545,1.351e-02)
(2561,5.842e-03)   (10625,2.397e-03)  (40193,9.414e-04)
(141569,3.564e-04) (471041,1.308e-04)
(1496065,4.670e-05)
};
}%


\bookmaketitle
\tableofcontents
\include{pgfplots.title_abstract_intro}
\include{pgfplots.preliminaries}
\include{pgfplots.intro}
\include{pgfplots.reference}
\include{pgfplots.libs}
\include{pgfplots.resources}
\include{pgfplots.importexport}
\include{pgfplots.basic.reference}

\printindex

\bibliographystyle{abbrv} %gerapali} %gerabbrv} %gerunsrt.bst} %gerabbrv}% gerplain}
\nocite{pgfplotstable}
\nocite{programmingnotes}
\bibliography{pgfplots}
\end{document}

% -----------------------------------------------------------------------------
% For Stefan Pinnow as reminder on what to look for when editing the manual
% -----------------------------------------------------------------------------
% There should be no line breaks in the following environments
% - |...|
% - \declareandlabel{...}
% - \verbpdfref{...}
% If "MakeTikzPictures" isn't running through check one of the externalized
% LOG files what is the cause of that.
% -----------------------------------------------------------------------------


%%%%%%%%%%%%%%%%%%%%%%%%%%%%%%%%%%%%%%%%%%%%%%%%%%%%%%%%%%%%%%%%%%%%%%%%%%%%%
%
% Package pgfplots.sty documentation.
%
% Copyright 2007/2008 by Christian Feuersaenger.
%
% This program is free software: you can redistribute it and/or modify
% it under the terms of the GNU General Public License as published by
% the Free Software Foundation, either version 3 of the License, or
% (at your option) any later version.
%
% This program is distributed in the hope that it will be useful,
% but WITHOUT ANY WARRANTY; without even the implied warranty of
% MERCHANTABILITY or FITNESS FOR A PARTICULAR PURPOSE.  See the
% GNU General Public License for more details.
%
% You should have received a copy of the GNU General Public License
% along with this program.  If not, see <http://www.gnu.org/licenses/>.
%
%
%%%%%%%%%%%%%%%%%%%%%%%%%%%%%%%%%%%%%%%%%%%%%%%%%%%%%%%%%%%%%%%%%%%%%%%%%%%%%
% SEE pgfplots-macros.tex as well!
%\pdfminorversion=5 % to allow compression
%\pdfobjcompresslevel=2
\documentclass[a4paper,openany]{book}

\let\bookmaketitle=\maketitle

% -----------------------------------
% this here is from ltxdoc:
\usepackage{doc}[=v2]
\AtBeginDocument{\MakeShortVerb{\|}}
%\setlength{\textwidth}{355pt}
%\addtolength\marginparwidth{30pt}
%\addtolength\oddsidemargin{20pt}
%\addtolength\evensidemargin{20pt}
\def\cmd#1{\cs{\expandafter\cmd@to@cs\string#1}}
\def\cmd@to@cs#1#2{\char\number`#2\relax}
\DeclareRobustCommand\cs[1]{\texttt{\char`\\#1}}
\providecommand\marg[1]{%
  {\ttfamily\char`\{}\meta{#1}{\ttfamily\char`\}}}
\providecommand\oarg[1]{%
  {\ttfamily[}\meta{#1}{\ttfamily]}}
\providecommand\parg[1]{%
  {\ttfamily(}\meta{#1}{\ttfamily)}}
\raggedbottom

% -----------------------------------
\input{pgfplots.preamble.tex}

\makeatletter
% I want a two-column index just as in pgfmanual styles. This here
% was the best way to get one:
\def\index@prologue{\section*{Index}\addcontentsline{toc}{chapter}{Index}
}
\makeatother

%\RequirePackage[german,english,francais]{babel}

\def\matlabcolormaptext{This colormap is similar to one shipped with Matlab$^\text{\textregistered}$ under a similar name.}

\IfFileExists{tikzlibraryspy.code.tex}{%
\usetikzlibrary{spy}
}{%
    \message{ERROR: tikz SPY library NOT available. The manual will only compile partially.^^J}%
}%

\usepackage{xparse}% for colorbrewer manual

\usetikzlibrary{
    decorations.markings,
    decorations.footprints,
    shapes.arrows,
    matrix,
    positioning,
}

\usepgfplotslibrary{
    fillbetween,
    ternary,
    smithchart,
    patchplots,
    polar,
    colormaps,
    colorbrewer,
    colortol,           % docu not ready yet
}
\pgfqkeys{/codeexample}{%
    every codeexample/.append style={
        /pgfplots/every ternary axis/.append style={
            /pgfplots/legend style={fill=graphicbackground},
        }
    },
    tabsize=4,
}

\pgfplotsmanualenableexternalizationofexpensive

%\usetikzlibrary{external}
%\tikzexternalize[prefix=figures/]{pgfplots}

\title{%
    Manual for Package \PGFPlots{}\\
    {\small 2D/3D Plots in \LaTeX{}, Version \pgfplotsversion}\\
    {\small\url{https://github.com/pgf-tikz/pgfplots}}
    %\\{\small Attention: you are using an unstable development version.}
}

\makeatletter
\long\def\abstractsmuggle{%
    \centering
    \textbf{Abstract}\\[0.5cm]

    \begin{minipage}{12cm}
        \PGFPlots{} draws high-quality function plots in normal or logarithmic
        scaling with a user-friendly interface directly in \TeX{}. The user
        supplies axis labels, legend entries and the plot coordinates for one or
        more plots and \PGFPlots{} applies axis scaling, computes any logarithms
        and axis ticks and draws the plots. It supports line plots, scatter
        plots, piecewise constant plots, bar plots, area plots, mesh and surface
        plots, patch plots, contour plots, quiver plots, histogram plots, box
        plots, polar axes, ternary diagrams, smith charts and some more. It is
        based on Till Tantau's package \PGF{}/\Tikz{}.
    \end{minipage}

}%

\expandafter\date\expandafter{\@date\\[2cm]
    \abstractsmuggle
}%
\makeatother

%\includeonly{pgfplots.reference}


\begin{document}

\def\plotcoords{%
\addplot coordinates {
(5,8.312e-02)    (17,2.547e-02)   (49,7.407e-03)
(129,2.102e-03)  (321,5.874e-04)  (769,1.623e-04)
(1793,4.442e-05) (4097,1.207e-05) (9217,3.261e-06)
};

\addplot coordinates{
(7,8.472e-02)    (31,3.044e-02)    (111,1.022e-02)
(351,3.303e-03)  (1023,1.039e-03)  (2815,3.196e-04)
(7423,9.658e-05) (18943,2.873e-05) (47103,8.437e-06)
};

\addplot coordinates{
(9,7.881e-02)     (49,3.243e-02)    (209,1.232e-02)
(769,4.454e-03)   (2561,1.551e-03)  (7937,5.236e-04)
(23297,1.723e-04) (65537,5.545e-05) (178177,1.751e-05)
};

\addplot coordinates{
(11,6.887e-02)    (71,3.177e-02)     (351,1.341e-02)
(1471,5.334e-03)  (5503,2.027e-03)   (18943,7.415e-04)
(61183,2.628e-04) (187903,9.063e-05) (553983,3.053e-05)
};

\addplot coordinates{
(13,5.755e-02)     (97,2.925e-02)     (545,1.351e-02)
(2561,5.842e-03)   (10625,2.397e-03)  (40193,9.414e-04)
(141569,3.564e-04) (471041,1.308e-04)
(1496065,4.670e-05)
};
}%


\bookmaketitle
\tableofcontents
\include{pgfplots.title_abstract_intro}
\include{pgfplots.preliminaries}
\include{pgfplots.intro}
\include{pgfplots.reference}
\include{pgfplots.libs}
\include{pgfplots.resources}
\include{pgfplots.importexport}
\include{pgfplots.basic.reference}

\printindex

\bibliographystyle{abbrv} %gerapali} %gerabbrv} %gerunsrt.bst} %gerabbrv}% gerplain}
\nocite{pgfplotstable}
\nocite{programmingnotes}
\bibliography{pgfplots}
\end{document}

% -----------------------------------------------------------------------------
% For Stefan Pinnow as reminder on what to look for when editing the manual
% -----------------------------------------------------------------------------
% There should be no line breaks in the following environments
% - |...|
% - \declareandlabel{...}
% - \verbpdfref{...}
% If "MakeTikzPictures" isn't running through check one of the externalized
% LOG files what is the cause of that.
% -----------------------------------------------------------------------------

\include{pgfplots.basic.reference}

\printindex

\bibliographystyle{abbrv} %gerapali} %gerabbrv} %gerunsrt.bst} %gerabbrv}% gerplain}
\nocite{pgfplotstable}
\nocite{programmingnotes}
\bibliography{pgfplots}
\end{document}

% -----------------------------------------------------------------------------
% For Stefan Pinnow as reminder on what to look for when editing the manual
% -----------------------------------------------------------------------------
% There should be no line breaks in the following environments
% - |...|
% - \declareandlabel{...}
% - \verbpdfref{...}
% If "MakeTikzPictures" isn't running through check one of the externalized
% LOG files what is the cause of that.
% -----------------------------------------------------------------------------


%%%%%%%%%%%%%%%%%%%%%%%%%%%%%%%%%%%%%%%%%%%%%%%%%%%%%%%%%%%%%%%%%%%%%%%%%%%%%
%
% Package pgfplots.sty documentation.
%
% Copyright 2007/2008 by Christian Feuersaenger.
%
% This program is free software: you can redistribute it and/or modify
% it under the terms of the GNU General Public License as published by
% the Free Software Foundation, either version 3 of the License, or
% (at your option) any later version.
%
% This program is distributed in the hope that it will be useful,
% but WITHOUT ANY WARRANTY; without even the implied warranty of
% MERCHANTABILITY or FITNESS FOR A PARTICULAR PURPOSE.  See the
% GNU General Public License for more details.
%
% You should have received a copy of the GNU General Public License
% along with this program.  If not, see <http://www.gnu.org/licenses/>.
%
%
%%%%%%%%%%%%%%%%%%%%%%%%%%%%%%%%%%%%%%%%%%%%%%%%%%%%%%%%%%%%%%%%%%%%%%%%%%%%%
% SEE pgfplots-macros.tex as well!
%\pdfminorversion=5 % to allow compression
%\pdfobjcompresslevel=2
\documentclass[a4paper,openany]{book}

\let\bookmaketitle=\maketitle

% -----------------------------------
% this here is from ltxdoc:
\usepackage{doc}[=v2]
\AtBeginDocument{\MakeShortVerb{\|}}
%\setlength{\textwidth}{355pt}
%\addtolength\marginparwidth{30pt}
%\addtolength\oddsidemargin{20pt}
%\addtolength\evensidemargin{20pt}
\def\cmd#1{\cs{\expandafter\cmd@to@cs\string#1}}
\def\cmd@to@cs#1#2{\char\number`#2\relax}
\DeclareRobustCommand\cs[1]{\texttt{\char`\\#1}}
\providecommand\marg[1]{%
  {\ttfamily\char`\{}\meta{#1}{\ttfamily\char`\}}}
\providecommand\oarg[1]{%
  {\ttfamily[}\meta{#1}{\ttfamily]}}
\providecommand\parg[1]{%
  {\ttfamily(}\meta{#1}{\ttfamily)}}
\raggedbottom

% -----------------------------------
\input{pgfplots.preamble.tex}

\makeatletter
% I want a two-column index just as in pgfmanual styles. This here
% was the best way to get one:
\def\index@prologue{\section*{Index}\addcontentsline{toc}{chapter}{Index}
}
\makeatother

%\RequirePackage[german,english,francais]{babel}

\def\matlabcolormaptext{This colormap is similar to one shipped with Matlab$^\text{\textregistered}$ under a similar name.}

\IfFileExists{tikzlibraryspy.code.tex}{%
\usetikzlibrary{spy}
}{%
    \message{ERROR: tikz SPY library NOT available. The manual will only compile partially.^^J}%
}%

\usepackage{xparse}% for colorbrewer manual

\usetikzlibrary{
    decorations.markings,
    decorations.footprints,
    shapes.arrows,
    matrix,
    positioning,
}

\usepgfplotslibrary{
    fillbetween,
    ternary,
    smithchart,
    patchplots,
    polar,
    colormaps,
    colorbrewer,
    colortol,           % docu not ready yet
}
\pgfqkeys{/codeexample}{%
    every codeexample/.append style={
        /pgfplots/every ternary axis/.append style={
            /pgfplots/legend style={fill=graphicbackground},
        }
    },
    tabsize=4,
}

\pgfplotsmanualenableexternalizationofexpensive

%\usetikzlibrary{external}
%\tikzexternalize[prefix=figures/]{pgfplots}

\title{%
    Manual for Package \PGFPlots{}\\
    {\small 2D/3D Plots in \LaTeX{}, Version \pgfplotsversion}\\
    {\small\url{https://github.com/pgf-tikz/pgfplots}}
    %\\{\small Attention: you are using an unstable development version.}
}

\makeatletter
\long\def\abstractsmuggle{%
    \centering
    \textbf{Abstract}\\[0.5cm]

    \begin{minipage}{12cm}
        \PGFPlots{} draws high-quality function plots in normal or logarithmic
        scaling with a user-friendly interface directly in \TeX{}. The user
        supplies axis labels, legend entries and the plot coordinates for one or
        more plots and \PGFPlots{} applies axis scaling, computes any logarithms
        and axis ticks and draws the plots. It supports line plots, scatter
        plots, piecewise constant plots, bar plots, area plots, mesh and surface
        plots, patch plots, contour plots, quiver plots, histogram plots, box
        plots, polar axes, ternary diagrams, smith charts and some more. It is
        based on Till Tantau's package \PGF{}/\Tikz{}.
    \end{minipage}

}%

\expandafter\date\expandafter{\@date\\[2cm]
    \abstractsmuggle
}%
\makeatother

%\includeonly{pgfplots.reference}


\begin{document}

\def\plotcoords{%
\addplot coordinates {
(5,8.312e-02)    (17,2.547e-02)   (49,7.407e-03)
(129,2.102e-03)  (321,5.874e-04)  (769,1.623e-04)
(1793,4.442e-05) (4097,1.207e-05) (9217,3.261e-06)
};

\addplot coordinates{
(7,8.472e-02)    (31,3.044e-02)    (111,1.022e-02)
(351,3.303e-03)  (1023,1.039e-03)  (2815,3.196e-04)
(7423,9.658e-05) (18943,2.873e-05) (47103,8.437e-06)
};

\addplot coordinates{
(9,7.881e-02)     (49,3.243e-02)    (209,1.232e-02)
(769,4.454e-03)   (2561,1.551e-03)  (7937,5.236e-04)
(23297,1.723e-04) (65537,5.545e-05) (178177,1.751e-05)
};

\addplot coordinates{
(11,6.887e-02)    (71,3.177e-02)     (351,1.341e-02)
(1471,5.334e-03)  (5503,2.027e-03)   (18943,7.415e-04)
(61183,2.628e-04) (187903,9.063e-05) (553983,3.053e-05)
};

\addplot coordinates{
(13,5.755e-02)     (97,2.925e-02)     (545,1.351e-02)
(2561,5.842e-03)   (10625,2.397e-03)  (40193,9.414e-04)
(141569,3.564e-04) (471041,1.308e-04)
(1496065,4.670e-05)
};
}%


\bookmaketitle
\tableofcontents

%%%%%%%%%%%%%%%%%%%%%%%%%%%%%%%%%%%%%%%%%%%%%%%%%%%%%%%%%%%%%%%%%%%%%%%%%%%%%
%
% Package pgfplots.sty documentation.
%
% Copyright 2007/2008 by Christian Feuersaenger.
%
% This program is free software: you can redistribute it and/or modify
% it under the terms of the GNU General Public License as published by
% the Free Software Foundation, either version 3 of the License, or
% (at your option) any later version.
%
% This program is distributed in the hope that it will be useful,
% but WITHOUT ANY WARRANTY; without even the implied warranty of
% MERCHANTABILITY or FITNESS FOR A PARTICULAR PURPOSE.  See the
% GNU General Public License for more details.
%
% You should have received a copy of the GNU General Public License
% along with this program.  If not, see <http://www.gnu.org/licenses/>.
%
%
%%%%%%%%%%%%%%%%%%%%%%%%%%%%%%%%%%%%%%%%%%%%%%%%%%%%%%%%%%%%%%%%%%%%%%%%%%%%%
% SEE pgfplots-macros.tex as well!
%\pdfminorversion=5 % to allow compression
%\pdfobjcompresslevel=2
\documentclass[a4paper,openany]{book}

\let\bookmaketitle=\maketitle

% -----------------------------------
% this here is from ltxdoc:
\usepackage{doc}[=v2]
\AtBeginDocument{\MakeShortVerb{\|}}
%\setlength{\textwidth}{355pt}
%\addtolength\marginparwidth{30pt}
%\addtolength\oddsidemargin{20pt}
%\addtolength\evensidemargin{20pt}
\def\cmd#1{\cs{\expandafter\cmd@to@cs\string#1}}
\def\cmd@to@cs#1#2{\char\number`#2\relax}
\DeclareRobustCommand\cs[1]{\texttt{\char`\\#1}}
\providecommand\marg[1]{%
  {\ttfamily\char`\{}\meta{#1}{\ttfamily\char`\}}}
\providecommand\oarg[1]{%
  {\ttfamily[}\meta{#1}{\ttfamily]}}
\providecommand\parg[1]{%
  {\ttfamily(}\meta{#1}{\ttfamily)}}
\raggedbottom

% -----------------------------------
\input{pgfplots.preamble.tex}

\makeatletter
% I want a two-column index just as in pgfmanual styles. This here
% was the best way to get one:
\def\index@prologue{\section*{Index}\addcontentsline{toc}{chapter}{Index}
}
\makeatother

%\RequirePackage[german,english,francais]{babel}

\def\matlabcolormaptext{This colormap is similar to one shipped with Matlab$^\text{\textregistered}$ under a similar name.}

\IfFileExists{tikzlibraryspy.code.tex}{%
\usetikzlibrary{spy}
}{%
    \message{ERROR: tikz SPY library NOT available. The manual will only compile partially.^^J}%
}%

\usepackage{xparse}% for colorbrewer manual

\usetikzlibrary{
    decorations.markings,
    decorations.footprints,
    shapes.arrows,
    matrix,
    positioning,
}

\usepgfplotslibrary{
    fillbetween,
    ternary,
    smithchart,
    patchplots,
    polar,
    colormaps,
    colorbrewer,
    colortol,           % docu not ready yet
}
\pgfqkeys{/codeexample}{%
    every codeexample/.append style={
        /pgfplots/every ternary axis/.append style={
            /pgfplots/legend style={fill=graphicbackground},
        }
    },
    tabsize=4,
}

\pgfplotsmanualenableexternalizationofexpensive

%\usetikzlibrary{external}
%\tikzexternalize[prefix=figures/]{pgfplots}

\title{%
    Manual for Package \PGFPlots{}\\
    {\small 2D/3D Plots in \LaTeX{}, Version \pgfplotsversion}\\
    {\small\url{https://github.com/pgf-tikz/pgfplots}}
    %\\{\small Attention: you are using an unstable development version.}
}

\makeatletter
\long\def\abstractsmuggle{%
    \centering
    \textbf{Abstract}\\[0.5cm]

    \begin{minipage}{12cm}
        \PGFPlots{} draws high-quality function plots in normal or logarithmic
        scaling with a user-friendly interface directly in \TeX{}. The user
        supplies axis labels, legend entries and the plot coordinates for one or
        more plots and \PGFPlots{} applies axis scaling, computes any logarithms
        and axis ticks and draws the plots. It supports line plots, scatter
        plots, piecewise constant plots, bar plots, area plots, mesh and surface
        plots, patch plots, contour plots, quiver plots, histogram plots, box
        plots, polar axes, ternary diagrams, smith charts and some more. It is
        based on Till Tantau's package \PGF{}/\Tikz{}.
    \end{minipage}

}%

\expandafter\date\expandafter{\@date\\[2cm]
    \abstractsmuggle
}%
\makeatother

%\includeonly{pgfplots.reference}


\begin{document}

\def\plotcoords{%
\addplot coordinates {
(5,8.312e-02)    (17,2.547e-02)   (49,7.407e-03)
(129,2.102e-03)  (321,5.874e-04)  (769,1.623e-04)
(1793,4.442e-05) (4097,1.207e-05) (9217,3.261e-06)
};

\addplot coordinates{
(7,8.472e-02)    (31,3.044e-02)    (111,1.022e-02)
(351,3.303e-03)  (1023,1.039e-03)  (2815,3.196e-04)
(7423,9.658e-05) (18943,2.873e-05) (47103,8.437e-06)
};

\addplot coordinates{
(9,7.881e-02)     (49,3.243e-02)    (209,1.232e-02)
(769,4.454e-03)   (2561,1.551e-03)  (7937,5.236e-04)
(23297,1.723e-04) (65537,5.545e-05) (178177,1.751e-05)
};

\addplot coordinates{
(11,6.887e-02)    (71,3.177e-02)     (351,1.341e-02)
(1471,5.334e-03)  (5503,2.027e-03)   (18943,7.415e-04)
(61183,2.628e-04) (187903,9.063e-05) (553983,3.053e-05)
};

\addplot coordinates{
(13,5.755e-02)     (97,2.925e-02)     (545,1.351e-02)
(2561,5.842e-03)   (10625,2.397e-03)  (40193,9.414e-04)
(141569,3.564e-04) (471041,1.308e-04)
(1496065,4.670e-05)
};
}%


\bookmaketitle
\tableofcontents
\include{pgfplots.title_abstract_intro}
\include{pgfplots.preliminaries}
\include{pgfplots.intro}
\include{pgfplots.reference}
\include{pgfplots.libs}
\include{pgfplots.resources}
\include{pgfplots.importexport}
\include{pgfplots.basic.reference}

\printindex

\bibliographystyle{abbrv} %gerapali} %gerabbrv} %gerunsrt.bst} %gerabbrv}% gerplain}
\nocite{pgfplotstable}
\nocite{programmingnotes}
\bibliography{pgfplots}
\end{document}

% -----------------------------------------------------------------------------
% For Stefan Pinnow as reminder on what to look for when editing the manual
% -----------------------------------------------------------------------------
% There should be no line breaks in the following environments
% - |...|
% - \declareandlabel{...}
% - \verbpdfref{...}
% If "MakeTikzPictures" isn't running through check one of the externalized
% LOG files what is the cause of that.
% -----------------------------------------------------------------------------


%%%%%%%%%%%%%%%%%%%%%%%%%%%%%%%%%%%%%%%%%%%%%%%%%%%%%%%%%%%%%%%%%%%%%%%%%%%%%
%
% Package pgfplots.sty documentation.
%
% Copyright 2007/2008 by Christian Feuersaenger.
%
% This program is free software: you can redistribute it and/or modify
% it under the terms of the GNU General Public License as published by
% the Free Software Foundation, either version 3 of the License, or
% (at your option) any later version.
%
% This program is distributed in the hope that it will be useful,
% but WITHOUT ANY WARRANTY; without even the implied warranty of
% MERCHANTABILITY or FITNESS FOR A PARTICULAR PURPOSE.  See the
% GNU General Public License for more details.
%
% You should have received a copy of the GNU General Public License
% along with this program.  If not, see <http://www.gnu.org/licenses/>.
%
%
%%%%%%%%%%%%%%%%%%%%%%%%%%%%%%%%%%%%%%%%%%%%%%%%%%%%%%%%%%%%%%%%%%%%%%%%%%%%%
% SEE pgfplots-macros.tex as well!
%\pdfminorversion=5 % to allow compression
%\pdfobjcompresslevel=2
\documentclass[a4paper,openany]{book}

\let\bookmaketitle=\maketitle

% -----------------------------------
% this here is from ltxdoc:
\usepackage{doc}[=v2]
\AtBeginDocument{\MakeShortVerb{\|}}
%\setlength{\textwidth}{355pt}
%\addtolength\marginparwidth{30pt}
%\addtolength\oddsidemargin{20pt}
%\addtolength\evensidemargin{20pt}
\def\cmd#1{\cs{\expandafter\cmd@to@cs\string#1}}
\def\cmd@to@cs#1#2{\char\number`#2\relax}
\DeclareRobustCommand\cs[1]{\texttt{\char`\\#1}}
\providecommand\marg[1]{%
  {\ttfamily\char`\{}\meta{#1}{\ttfamily\char`\}}}
\providecommand\oarg[1]{%
  {\ttfamily[}\meta{#1}{\ttfamily]}}
\providecommand\parg[1]{%
  {\ttfamily(}\meta{#1}{\ttfamily)}}
\raggedbottom

% -----------------------------------
\input{pgfplots.preamble.tex}

\makeatletter
% I want a two-column index just as in pgfmanual styles. This here
% was the best way to get one:
\def\index@prologue{\section*{Index}\addcontentsline{toc}{chapter}{Index}
}
\makeatother

%\RequirePackage[german,english,francais]{babel}

\def\matlabcolormaptext{This colormap is similar to one shipped with Matlab$^\text{\textregistered}$ under a similar name.}

\IfFileExists{tikzlibraryspy.code.tex}{%
\usetikzlibrary{spy}
}{%
    \message{ERROR: tikz SPY library NOT available. The manual will only compile partially.^^J}%
}%

\usepackage{xparse}% for colorbrewer manual

\usetikzlibrary{
    decorations.markings,
    decorations.footprints,
    shapes.arrows,
    matrix,
    positioning,
}

\usepgfplotslibrary{
    fillbetween,
    ternary,
    smithchart,
    patchplots,
    polar,
    colormaps,
    colorbrewer,
    colortol,           % docu not ready yet
}
\pgfqkeys{/codeexample}{%
    every codeexample/.append style={
        /pgfplots/every ternary axis/.append style={
            /pgfplots/legend style={fill=graphicbackground},
        }
    },
    tabsize=4,
}

\pgfplotsmanualenableexternalizationofexpensive

%\usetikzlibrary{external}
%\tikzexternalize[prefix=figures/]{pgfplots}

\title{%
    Manual for Package \PGFPlots{}\\
    {\small 2D/3D Plots in \LaTeX{}, Version \pgfplotsversion}\\
    {\small\url{https://github.com/pgf-tikz/pgfplots}}
    %\\{\small Attention: you are using an unstable development version.}
}

\makeatletter
\long\def\abstractsmuggle{%
    \centering
    \textbf{Abstract}\\[0.5cm]

    \begin{minipage}{12cm}
        \PGFPlots{} draws high-quality function plots in normal or logarithmic
        scaling with a user-friendly interface directly in \TeX{}. The user
        supplies axis labels, legend entries and the plot coordinates for one or
        more plots and \PGFPlots{} applies axis scaling, computes any logarithms
        and axis ticks and draws the plots. It supports line plots, scatter
        plots, piecewise constant plots, bar plots, area plots, mesh and surface
        plots, patch plots, contour plots, quiver plots, histogram plots, box
        plots, polar axes, ternary diagrams, smith charts and some more. It is
        based on Till Tantau's package \PGF{}/\Tikz{}.
    \end{minipage}

}%

\expandafter\date\expandafter{\@date\\[2cm]
    \abstractsmuggle
}%
\makeatother

%\includeonly{pgfplots.reference}


\begin{document}

\def\plotcoords{%
\addplot coordinates {
(5,8.312e-02)    (17,2.547e-02)   (49,7.407e-03)
(129,2.102e-03)  (321,5.874e-04)  (769,1.623e-04)
(1793,4.442e-05) (4097,1.207e-05) (9217,3.261e-06)
};

\addplot coordinates{
(7,8.472e-02)    (31,3.044e-02)    (111,1.022e-02)
(351,3.303e-03)  (1023,1.039e-03)  (2815,3.196e-04)
(7423,9.658e-05) (18943,2.873e-05) (47103,8.437e-06)
};

\addplot coordinates{
(9,7.881e-02)     (49,3.243e-02)    (209,1.232e-02)
(769,4.454e-03)   (2561,1.551e-03)  (7937,5.236e-04)
(23297,1.723e-04) (65537,5.545e-05) (178177,1.751e-05)
};

\addplot coordinates{
(11,6.887e-02)    (71,3.177e-02)     (351,1.341e-02)
(1471,5.334e-03)  (5503,2.027e-03)   (18943,7.415e-04)
(61183,2.628e-04) (187903,9.063e-05) (553983,3.053e-05)
};

\addplot coordinates{
(13,5.755e-02)     (97,2.925e-02)     (545,1.351e-02)
(2561,5.842e-03)   (10625,2.397e-03)  (40193,9.414e-04)
(141569,3.564e-04) (471041,1.308e-04)
(1496065,4.670e-05)
};
}%


\bookmaketitle
\tableofcontents
\include{pgfplots.title_abstract_intro}
\include{pgfplots.preliminaries}
\include{pgfplots.intro}
\include{pgfplots.reference}
\include{pgfplots.libs}
\include{pgfplots.resources}
\include{pgfplots.importexport}
\include{pgfplots.basic.reference}

\printindex

\bibliographystyle{abbrv} %gerapali} %gerabbrv} %gerunsrt.bst} %gerabbrv}% gerplain}
\nocite{pgfplotstable}
\nocite{programmingnotes}
\bibliography{pgfplots}
\end{document}

% -----------------------------------------------------------------------------
% For Stefan Pinnow as reminder on what to look for when editing the manual
% -----------------------------------------------------------------------------
% There should be no line breaks in the following environments
% - |...|
% - \declareandlabel{...}
% - \verbpdfref{...}
% If "MakeTikzPictures" isn't running through check one of the externalized
% LOG files what is the cause of that.
% -----------------------------------------------------------------------------


%%%%%%%%%%%%%%%%%%%%%%%%%%%%%%%%%%%%%%%%%%%%%%%%%%%%%%%%%%%%%%%%%%%%%%%%%%%%%
%
% Package pgfplots.sty documentation.
%
% Copyright 2007/2008 by Christian Feuersaenger.
%
% This program is free software: you can redistribute it and/or modify
% it under the terms of the GNU General Public License as published by
% the Free Software Foundation, either version 3 of the License, or
% (at your option) any later version.
%
% This program is distributed in the hope that it will be useful,
% but WITHOUT ANY WARRANTY; without even the implied warranty of
% MERCHANTABILITY or FITNESS FOR A PARTICULAR PURPOSE.  See the
% GNU General Public License for more details.
%
% You should have received a copy of the GNU General Public License
% along with this program.  If not, see <http://www.gnu.org/licenses/>.
%
%
%%%%%%%%%%%%%%%%%%%%%%%%%%%%%%%%%%%%%%%%%%%%%%%%%%%%%%%%%%%%%%%%%%%%%%%%%%%%%
% SEE pgfplots-macros.tex as well!
%\pdfminorversion=5 % to allow compression
%\pdfobjcompresslevel=2
\documentclass[a4paper,openany]{book}

\let\bookmaketitle=\maketitle

% -----------------------------------
% this here is from ltxdoc:
\usepackage{doc}[=v2]
\AtBeginDocument{\MakeShortVerb{\|}}
%\setlength{\textwidth}{355pt}
%\addtolength\marginparwidth{30pt}
%\addtolength\oddsidemargin{20pt}
%\addtolength\evensidemargin{20pt}
\def\cmd#1{\cs{\expandafter\cmd@to@cs\string#1}}
\def\cmd@to@cs#1#2{\char\number`#2\relax}
\DeclareRobustCommand\cs[1]{\texttt{\char`\\#1}}
\providecommand\marg[1]{%
  {\ttfamily\char`\{}\meta{#1}{\ttfamily\char`\}}}
\providecommand\oarg[1]{%
  {\ttfamily[}\meta{#1}{\ttfamily]}}
\providecommand\parg[1]{%
  {\ttfamily(}\meta{#1}{\ttfamily)}}
\raggedbottom

% -----------------------------------
\input{pgfplots.preamble.tex}

\makeatletter
% I want a two-column index just as in pgfmanual styles. This here
% was the best way to get one:
\def\index@prologue{\section*{Index}\addcontentsline{toc}{chapter}{Index}
}
\makeatother

%\RequirePackage[german,english,francais]{babel}

\def\matlabcolormaptext{This colormap is similar to one shipped with Matlab$^\text{\textregistered}$ under a similar name.}

\IfFileExists{tikzlibraryspy.code.tex}{%
\usetikzlibrary{spy}
}{%
    \message{ERROR: tikz SPY library NOT available. The manual will only compile partially.^^J}%
}%

\usepackage{xparse}% for colorbrewer manual

\usetikzlibrary{
    decorations.markings,
    decorations.footprints,
    shapes.arrows,
    matrix,
    positioning,
}

\usepgfplotslibrary{
    fillbetween,
    ternary,
    smithchart,
    patchplots,
    polar,
    colormaps,
    colorbrewer,
    colortol,           % docu not ready yet
}
\pgfqkeys{/codeexample}{%
    every codeexample/.append style={
        /pgfplots/every ternary axis/.append style={
            /pgfplots/legend style={fill=graphicbackground},
        }
    },
    tabsize=4,
}

\pgfplotsmanualenableexternalizationofexpensive

%\usetikzlibrary{external}
%\tikzexternalize[prefix=figures/]{pgfplots}

\title{%
    Manual for Package \PGFPlots{}\\
    {\small 2D/3D Plots in \LaTeX{}, Version \pgfplotsversion}\\
    {\small\url{https://github.com/pgf-tikz/pgfplots}}
    %\\{\small Attention: you are using an unstable development version.}
}

\makeatletter
\long\def\abstractsmuggle{%
    \centering
    \textbf{Abstract}\\[0.5cm]

    \begin{minipage}{12cm}
        \PGFPlots{} draws high-quality function plots in normal or logarithmic
        scaling with a user-friendly interface directly in \TeX{}. The user
        supplies axis labels, legend entries and the plot coordinates for one or
        more plots and \PGFPlots{} applies axis scaling, computes any logarithms
        and axis ticks and draws the plots. It supports line plots, scatter
        plots, piecewise constant plots, bar plots, area plots, mesh and surface
        plots, patch plots, contour plots, quiver plots, histogram plots, box
        plots, polar axes, ternary diagrams, smith charts and some more. It is
        based on Till Tantau's package \PGF{}/\Tikz{}.
    \end{minipage}

}%

\expandafter\date\expandafter{\@date\\[2cm]
    \abstractsmuggle
}%
\makeatother

%\includeonly{pgfplots.reference}


\begin{document}

\def\plotcoords{%
\addplot coordinates {
(5,8.312e-02)    (17,2.547e-02)   (49,7.407e-03)
(129,2.102e-03)  (321,5.874e-04)  (769,1.623e-04)
(1793,4.442e-05) (4097,1.207e-05) (9217,3.261e-06)
};

\addplot coordinates{
(7,8.472e-02)    (31,3.044e-02)    (111,1.022e-02)
(351,3.303e-03)  (1023,1.039e-03)  (2815,3.196e-04)
(7423,9.658e-05) (18943,2.873e-05) (47103,8.437e-06)
};

\addplot coordinates{
(9,7.881e-02)     (49,3.243e-02)    (209,1.232e-02)
(769,4.454e-03)   (2561,1.551e-03)  (7937,5.236e-04)
(23297,1.723e-04) (65537,5.545e-05) (178177,1.751e-05)
};

\addplot coordinates{
(11,6.887e-02)    (71,3.177e-02)     (351,1.341e-02)
(1471,5.334e-03)  (5503,2.027e-03)   (18943,7.415e-04)
(61183,2.628e-04) (187903,9.063e-05) (553983,3.053e-05)
};

\addplot coordinates{
(13,5.755e-02)     (97,2.925e-02)     (545,1.351e-02)
(2561,5.842e-03)   (10625,2.397e-03)  (40193,9.414e-04)
(141569,3.564e-04) (471041,1.308e-04)
(1496065,4.670e-05)
};
}%


\bookmaketitle
\tableofcontents
\include{pgfplots.title_abstract_intro}
\include{pgfplots.preliminaries}
\include{pgfplots.intro}
\include{pgfplots.reference}
\include{pgfplots.libs}
\include{pgfplots.resources}
\include{pgfplots.importexport}
\include{pgfplots.basic.reference}

\printindex

\bibliographystyle{abbrv} %gerapali} %gerabbrv} %gerunsrt.bst} %gerabbrv}% gerplain}
\nocite{pgfplotstable}
\nocite{programmingnotes}
\bibliography{pgfplots}
\end{document}

% -----------------------------------------------------------------------------
% For Stefan Pinnow as reminder on what to look for when editing the manual
% -----------------------------------------------------------------------------
% There should be no line breaks in the following environments
% - |...|
% - \declareandlabel{...}
% - \verbpdfref{...}
% If "MakeTikzPictures" isn't running through check one of the externalized
% LOG files what is the cause of that.
% -----------------------------------------------------------------------------


%%%%%%%%%%%%%%%%%%%%%%%%%%%%%%%%%%%%%%%%%%%%%%%%%%%%%%%%%%%%%%%%%%%%%%%%%%%%%
%
% Package pgfplots.sty documentation.
%
% Copyright 2007/2008 by Christian Feuersaenger.
%
% This program is free software: you can redistribute it and/or modify
% it under the terms of the GNU General Public License as published by
% the Free Software Foundation, either version 3 of the License, or
% (at your option) any later version.
%
% This program is distributed in the hope that it will be useful,
% but WITHOUT ANY WARRANTY; without even the implied warranty of
% MERCHANTABILITY or FITNESS FOR A PARTICULAR PURPOSE.  See the
% GNU General Public License for more details.
%
% You should have received a copy of the GNU General Public License
% along with this program.  If not, see <http://www.gnu.org/licenses/>.
%
%
%%%%%%%%%%%%%%%%%%%%%%%%%%%%%%%%%%%%%%%%%%%%%%%%%%%%%%%%%%%%%%%%%%%%%%%%%%%%%
% SEE pgfplots-macros.tex as well!
%\pdfminorversion=5 % to allow compression
%\pdfobjcompresslevel=2
\documentclass[a4paper,openany]{book}

\let\bookmaketitle=\maketitle

% -----------------------------------
% this here is from ltxdoc:
\usepackage{doc}[=v2]
\AtBeginDocument{\MakeShortVerb{\|}}
%\setlength{\textwidth}{355pt}
%\addtolength\marginparwidth{30pt}
%\addtolength\oddsidemargin{20pt}
%\addtolength\evensidemargin{20pt}
\def\cmd#1{\cs{\expandafter\cmd@to@cs\string#1}}
\def\cmd@to@cs#1#2{\char\number`#2\relax}
\DeclareRobustCommand\cs[1]{\texttt{\char`\\#1}}
\providecommand\marg[1]{%
  {\ttfamily\char`\{}\meta{#1}{\ttfamily\char`\}}}
\providecommand\oarg[1]{%
  {\ttfamily[}\meta{#1}{\ttfamily]}}
\providecommand\parg[1]{%
  {\ttfamily(}\meta{#1}{\ttfamily)}}
\raggedbottom

% -----------------------------------
\input{pgfplots.preamble.tex}

\makeatletter
% I want a two-column index just as in pgfmanual styles. This here
% was the best way to get one:
\def\index@prologue{\section*{Index}\addcontentsline{toc}{chapter}{Index}
}
\makeatother

%\RequirePackage[german,english,francais]{babel}

\def\matlabcolormaptext{This colormap is similar to one shipped with Matlab$^\text{\textregistered}$ under a similar name.}

\IfFileExists{tikzlibraryspy.code.tex}{%
\usetikzlibrary{spy}
}{%
    \message{ERROR: tikz SPY library NOT available. The manual will only compile partially.^^J}%
}%

\usepackage{xparse}% for colorbrewer manual

\usetikzlibrary{
    decorations.markings,
    decorations.footprints,
    shapes.arrows,
    matrix,
    positioning,
}

\usepgfplotslibrary{
    fillbetween,
    ternary,
    smithchart,
    patchplots,
    polar,
    colormaps,
    colorbrewer,
    colortol,           % docu not ready yet
}
\pgfqkeys{/codeexample}{%
    every codeexample/.append style={
        /pgfplots/every ternary axis/.append style={
            /pgfplots/legend style={fill=graphicbackground},
        }
    },
    tabsize=4,
}

\pgfplotsmanualenableexternalizationofexpensive

%\usetikzlibrary{external}
%\tikzexternalize[prefix=figures/]{pgfplots}

\title{%
    Manual for Package \PGFPlots{}\\
    {\small 2D/3D Plots in \LaTeX{}, Version \pgfplotsversion}\\
    {\small\url{https://github.com/pgf-tikz/pgfplots}}
    %\\{\small Attention: you are using an unstable development version.}
}

\makeatletter
\long\def\abstractsmuggle{%
    \centering
    \textbf{Abstract}\\[0.5cm]

    \begin{minipage}{12cm}
        \PGFPlots{} draws high-quality function plots in normal or logarithmic
        scaling with a user-friendly interface directly in \TeX{}. The user
        supplies axis labels, legend entries and the plot coordinates for one or
        more plots and \PGFPlots{} applies axis scaling, computes any logarithms
        and axis ticks and draws the plots. It supports line plots, scatter
        plots, piecewise constant plots, bar plots, area plots, mesh and surface
        plots, patch plots, contour plots, quiver plots, histogram plots, box
        plots, polar axes, ternary diagrams, smith charts and some more. It is
        based on Till Tantau's package \PGF{}/\Tikz{}.
    \end{minipage}

}%

\expandafter\date\expandafter{\@date\\[2cm]
    \abstractsmuggle
}%
\makeatother

%\includeonly{pgfplots.reference}


\begin{document}

\def\plotcoords{%
\addplot coordinates {
(5,8.312e-02)    (17,2.547e-02)   (49,7.407e-03)
(129,2.102e-03)  (321,5.874e-04)  (769,1.623e-04)
(1793,4.442e-05) (4097,1.207e-05) (9217,3.261e-06)
};

\addplot coordinates{
(7,8.472e-02)    (31,3.044e-02)    (111,1.022e-02)
(351,3.303e-03)  (1023,1.039e-03)  (2815,3.196e-04)
(7423,9.658e-05) (18943,2.873e-05) (47103,8.437e-06)
};

\addplot coordinates{
(9,7.881e-02)     (49,3.243e-02)    (209,1.232e-02)
(769,4.454e-03)   (2561,1.551e-03)  (7937,5.236e-04)
(23297,1.723e-04) (65537,5.545e-05) (178177,1.751e-05)
};

\addplot coordinates{
(11,6.887e-02)    (71,3.177e-02)     (351,1.341e-02)
(1471,5.334e-03)  (5503,2.027e-03)   (18943,7.415e-04)
(61183,2.628e-04) (187903,9.063e-05) (553983,3.053e-05)
};

\addplot coordinates{
(13,5.755e-02)     (97,2.925e-02)     (545,1.351e-02)
(2561,5.842e-03)   (10625,2.397e-03)  (40193,9.414e-04)
(141569,3.564e-04) (471041,1.308e-04)
(1496065,4.670e-05)
};
}%


\bookmaketitle
\tableofcontents
\include{pgfplots.title_abstract_intro}
\include{pgfplots.preliminaries}
\include{pgfplots.intro}
\include{pgfplots.reference}
\include{pgfplots.libs}
\include{pgfplots.resources}
\include{pgfplots.importexport}
\include{pgfplots.basic.reference}

\printindex

\bibliographystyle{abbrv} %gerapali} %gerabbrv} %gerunsrt.bst} %gerabbrv}% gerplain}
\nocite{pgfplotstable}
\nocite{programmingnotes}
\bibliography{pgfplots}
\end{document}

% -----------------------------------------------------------------------------
% For Stefan Pinnow as reminder on what to look for when editing the manual
% -----------------------------------------------------------------------------
% There should be no line breaks in the following environments
% - |...|
% - \declareandlabel{...}
% - \verbpdfref{...}
% If "MakeTikzPictures" isn't running through check one of the externalized
% LOG files what is the cause of that.
% -----------------------------------------------------------------------------


%%%%%%%%%%%%%%%%%%%%%%%%%%%%%%%%%%%%%%%%%%%%%%%%%%%%%%%%%%%%%%%%%%%%%%%%%%%%%
%
% Package pgfplots.sty documentation.
%
% Copyright 2007/2008 by Christian Feuersaenger.
%
% This program is free software: you can redistribute it and/or modify
% it under the terms of the GNU General Public License as published by
% the Free Software Foundation, either version 3 of the License, or
% (at your option) any later version.
%
% This program is distributed in the hope that it will be useful,
% but WITHOUT ANY WARRANTY; without even the implied warranty of
% MERCHANTABILITY or FITNESS FOR A PARTICULAR PURPOSE.  See the
% GNU General Public License for more details.
%
% You should have received a copy of the GNU General Public License
% along with this program.  If not, see <http://www.gnu.org/licenses/>.
%
%
%%%%%%%%%%%%%%%%%%%%%%%%%%%%%%%%%%%%%%%%%%%%%%%%%%%%%%%%%%%%%%%%%%%%%%%%%%%%%
% SEE pgfplots-macros.tex as well!
%\pdfminorversion=5 % to allow compression
%\pdfobjcompresslevel=2
\documentclass[a4paper,openany]{book}

\let\bookmaketitle=\maketitle

% -----------------------------------
% this here is from ltxdoc:
\usepackage{doc}[=v2]
\AtBeginDocument{\MakeShortVerb{\|}}
%\setlength{\textwidth}{355pt}
%\addtolength\marginparwidth{30pt}
%\addtolength\oddsidemargin{20pt}
%\addtolength\evensidemargin{20pt}
\def\cmd#1{\cs{\expandafter\cmd@to@cs\string#1}}
\def\cmd@to@cs#1#2{\char\number`#2\relax}
\DeclareRobustCommand\cs[1]{\texttt{\char`\\#1}}
\providecommand\marg[1]{%
  {\ttfamily\char`\{}\meta{#1}{\ttfamily\char`\}}}
\providecommand\oarg[1]{%
  {\ttfamily[}\meta{#1}{\ttfamily]}}
\providecommand\parg[1]{%
  {\ttfamily(}\meta{#1}{\ttfamily)}}
\raggedbottom

% -----------------------------------
\input{pgfplots.preamble.tex}

\makeatletter
% I want a two-column index just as in pgfmanual styles. This here
% was the best way to get one:
\def\index@prologue{\section*{Index}\addcontentsline{toc}{chapter}{Index}
}
\makeatother

%\RequirePackage[german,english,francais]{babel}

\def\matlabcolormaptext{This colormap is similar to one shipped with Matlab$^\text{\textregistered}$ under a similar name.}

\IfFileExists{tikzlibraryspy.code.tex}{%
\usetikzlibrary{spy}
}{%
    \message{ERROR: tikz SPY library NOT available. The manual will only compile partially.^^J}%
}%

\usepackage{xparse}% for colorbrewer manual

\usetikzlibrary{
    decorations.markings,
    decorations.footprints,
    shapes.arrows,
    matrix,
    positioning,
}

\usepgfplotslibrary{
    fillbetween,
    ternary,
    smithchart,
    patchplots,
    polar,
    colormaps,
    colorbrewer,
    colortol,           % docu not ready yet
}
\pgfqkeys{/codeexample}{%
    every codeexample/.append style={
        /pgfplots/every ternary axis/.append style={
            /pgfplots/legend style={fill=graphicbackground},
        }
    },
    tabsize=4,
}

\pgfplotsmanualenableexternalizationofexpensive

%\usetikzlibrary{external}
%\tikzexternalize[prefix=figures/]{pgfplots}

\title{%
    Manual for Package \PGFPlots{}\\
    {\small 2D/3D Plots in \LaTeX{}, Version \pgfplotsversion}\\
    {\small\url{https://github.com/pgf-tikz/pgfplots}}
    %\\{\small Attention: you are using an unstable development version.}
}

\makeatletter
\long\def\abstractsmuggle{%
    \centering
    \textbf{Abstract}\\[0.5cm]

    \begin{minipage}{12cm}
        \PGFPlots{} draws high-quality function plots in normal or logarithmic
        scaling with a user-friendly interface directly in \TeX{}. The user
        supplies axis labels, legend entries and the plot coordinates for one or
        more plots and \PGFPlots{} applies axis scaling, computes any logarithms
        and axis ticks and draws the plots. It supports line plots, scatter
        plots, piecewise constant plots, bar plots, area plots, mesh and surface
        plots, patch plots, contour plots, quiver plots, histogram plots, box
        plots, polar axes, ternary diagrams, smith charts and some more. It is
        based on Till Tantau's package \PGF{}/\Tikz{}.
    \end{minipage}

}%

\expandafter\date\expandafter{\@date\\[2cm]
    \abstractsmuggle
}%
\makeatother

%\includeonly{pgfplots.reference}


\begin{document}

\def\plotcoords{%
\addplot coordinates {
(5,8.312e-02)    (17,2.547e-02)   (49,7.407e-03)
(129,2.102e-03)  (321,5.874e-04)  (769,1.623e-04)
(1793,4.442e-05) (4097,1.207e-05) (9217,3.261e-06)
};

\addplot coordinates{
(7,8.472e-02)    (31,3.044e-02)    (111,1.022e-02)
(351,3.303e-03)  (1023,1.039e-03)  (2815,3.196e-04)
(7423,9.658e-05) (18943,2.873e-05) (47103,8.437e-06)
};

\addplot coordinates{
(9,7.881e-02)     (49,3.243e-02)    (209,1.232e-02)
(769,4.454e-03)   (2561,1.551e-03)  (7937,5.236e-04)
(23297,1.723e-04) (65537,5.545e-05) (178177,1.751e-05)
};

\addplot coordinates{
(11,6.887e-02)    (71,3.177e-02)     (351,1.341e-02)
(1471,5.334e-03)  (5503,2.027e-03)   (18943,7.415e-04)
(61183,2.628e-04) (187903,9.063e-05) (553983,3.053e-05)
};

\addplot coordinates{
(13,5.755e-02)     (97,2.925e-02)     (545,1.351e-02)
(2561,5.842e-03)   (10625,2.397e-03)  (40193,9.414e-04)
(141569,3.564e-04) (471041,1.308e-04)
(1496065,4.670e-05)
};
}%


\bookmaketitle
\tableofcontents
\include{pgfplots.title_abstract_intro}
\include{pgfplots.preliminaries}
\include{pgfplots.intro}
\include{pgfplots.reference}
\include{pgfplots.libs}
\include{pgfplots.resources}
\include{pgfplots.importexport}
\include{pgfplots.basic.reference}

\printindex

\bibliographystyle{abbrv} %gerapali} %gerabbrv} %gerunsrt.bst} %gerabbrv}% gerplain}
\nocite{pgfplotstable}
\nocite{programmingnotes}
\bibliography{pgfplots}
\end{document}

% -----------------------------------------------------------------------------
% For Stefan Pinnow as reminder on what to look for when editing the manual
% -----------------------------------------------------------------------------
% There should be no line breaks in the following environments
% - |...|
% - \declareandlabel{...}
% - \verbpdfref{...}
% If "MakeTikzPictures" isn't running through check one of the externalized
% LOG files what is the cause of that.
% -----------------------------------------------------------------------------


%%%%%%%%%%%%%%%%%%%%%%%%%%%%%%%%%%%%%%%%%%%%%%%%%%%%%%%%%%%%%%%%%%%%%%%%%%%%%
%
% Package pgfplots.sty documentation.
%
% Copyright 2007/2008 by Christian Feuersaenger.
%
% This program is free software: you can redistribute it and/or modify
% it under the terms of the GNU General Public License as published by
% the Free Software Foundation, either version 3 of the License, or
% (at your option) any later version.
%
% This program is distributed in the hope that it will be useful,
% but WITHOUT ANY WARRANTY; without even the implied warranty of
% MERCHANTABILITY or FITNESS FOR A PARTICULAR PURPOSE.  See the
% GNU General Public License for more details.
%
% You should have received a copy of the GNU General Public License
% along with this program.  If not, see <http://www.gnu.org/licenses/>.
%
%
%%%%%%%%%%%%%%%%%%%%%%%%%%%%%%%%%%%%%%%%%%%%%%%%%%%%%%%%%%%%%%%%%%%%%%%%%%%%%
% SEE pgfplots-macros.tex as well!
%\pdfminorversion=5 % to allow compression
%\pdfobjcompresslevel=2
\documentclass[a4paper,openany]{book}

\let\bookmaketitle=\maketitle

% -----------------------------------
% this here is from ltxdoc:
\usepackage{doc}[=v2]
\AtBeginDocument{\MakeShortVerb{\|}}
%\setlength{\textwidth}{355pt}
%\addtolength\marginparwidth{30pt}
%\addtolength\oddsidemargin{20pt}
%\addtolength\evensidemargin{20pt}
\def\cmd#1{\cs{\expandafter\cmd@to@cs\string#1}}
\def\cmd@to@cs#1#2{\char\number`#2\relax}
\DeclareRobustCommand\cs[1]{\texttt{\char`\\#1}}
\providecommand\marg[1]{%
  {\ttfamily\char`\{}\meta{#1}{\ttfamily\char`\}}}
\providecommand\oarg[1]{%
  {\ttfamily[}\meta{#1}{\ttfamily]}}
\providecommand\parg[1]{%
  {\ttfamily(}\meta{#1}{\ttfamily)}}
\raggedbottom

% -----------------------------------
\input{pgfplots.preamble.tex}

\makeatletter
% I want a two-column index just as in pgfmanual styles. This here
% was the best way to get one:
\def\index@prologue{\section*{Index}\addcontentsline{toc}{chapter}{Index}
}
\makeatother

%\RequirePackage[german,english,francais]{babel}

\def\matlabcolormaptext{This colormap is similar to one shipped with Matlab$^\text{\textregistered}$ under a similar name.}

\IfFileExists{tikzlibraryspy.code.tex}{%
\usetikzlibrary{spy}
}{%
    \message{ERROR: tikz SPY library NOT available. The manual will only compile partially.^^J}%
}%

\usepackage{xparse}% for colorbrewer manual

\usetikzlibrary{
    decorations.markings,
    decorations.footprints,
    shapes.arrows,
    matrix,
    positioning,
}

\usepgfplotslibrary{
    fillbetween,
    ternary,
    smithchart,
    patchplots,
    polar,
    colormaps,
    colorbrewer,
    colortol,           % docu not ready yet
}
\pgfqkeys{/codeexample}{%
    every codeexample/.append style={
        /pgfplots/every ternary axis/.append style={
            /pgfplots/legend style={fill=graphicbackground},
        }
    },
    tabsize=4,
}

\pgfplotsmanualenableexternalizationofexpensive

%\usetikzlibrary{external}
%\tikzexternalize[prefix=figures/]{pgfplots}

\title{%
    Manual for Package \PGFPlots{}\\
    {\small 2D/3D Plots in \LaTeX{}, Version \pgfplotsversion}\\
    {\small\url{https://github.com/pgf-tikz/pgfplots}}
    %\\{\small Attention: you are using an unstable development version.}
}

\makeatletter
\long\def\abstractsmuggle{%
    \centering
    \textbf{Abstract}\\[0.5cm]

    \begin{minipage}{12cm}
        \PGFPlots{} draws high-quality function plots in normal or logarithmic
        scaling with a user-friendly interface directly in \TeX{}. The user
        supplies axis labels, legend entries and the plot coordinates for one or
        more plots and \PGFPlots{} applies axis scaling, computes any logarithms
        and axis ticks and draws the plots. It supports line plots, scatter
        plots, piecewise constant plots, bar plots, area plots, mesh and surface
        plots, patch plots, contour plots, quiver plots, histogram plots, box
        plots, polar axes, ternary diagrams, smith charts and some more. It is
        based on Till Tantau's package \PGF{}/\Tikz{}.
    \end{minipage}

}%

\expandafter\date\expandafter{\@date\\[2cm]
    \abstractsmuggle
}%
\makeatother

%\includeonly{pgfplots.reference}


\begin{document}

\def\plotcoords{%
\addplot coordinates {
(5,8.312e-02)    (17,2.547e-02)   (49,7.407e-03)
(129,2.102e-03)  (321,5.874e-04)  (769,1.623e-04)
(1793,4.442e-05) (4097,1.207e-05) (9217,3.261e-06)
};

\addplot coordinates{
(7,8.472e-02)    (31,3.044e-02)    (111,1.022e-02)
(351,3.303e-03)  (1023,1.039e-03)  (2815,3.196e-04)
(7423,9.658e-05) (18943,2.873e-05) (47103,8.437e-06)
};

\addplot coordinates{
(9,7.881e-02)     (49,3.243e-02)    (209,1.232e-02)
(769,4.454e-03)   (2561,1.551e-03)  (7937,5.236e-04)
(23297,1.723e-04) (65537,5.545e-05) (178177,1.751e-05)
};

\addplot coordinates{
(11,6.887e-02)    (71,3.177e-02)     (351,1.341e-02)
(1471,5.334e-03)  (5503,2.027e-03)   (18943,7.415e-04)
(61183,2.628e-04) (187903,9.063e-05) (553983,3.053e-05)
};

\addplot coordinates{
(13,5.755e-02)     (97,2.925e-02)     (545,1.351e-02)
(2561,5.842e-03)   (10625,2.397e-03)  (40193,9.414e-04)
(141569,3.564e-04) (471041,1.308e-04)
(1496065,4.670e-05)
};
}%


\bookmaketitle
\tableofcontents
\include{pgfplots.title_abstract_intro}
\include{pgfplots.preliminaries}
\include{pgfplots.intro}
\include{pgfplots.reference}
\include{pgfplots.libs}
\include{pgfplots.resources}
\include{pgfplots.importexport}
\include{pgfplots.basic.reference}

\printindex

\bibliographystyle{abbrv} %gerapali} %gerabbrv} %gerunsrt.bst} %gerabbrv}% gerplain}
\nocite{pgfplotstable}
\nocite{programmingnotes}
\bibliography{pgfplots}
\end{document}

% -----------------------------------------------------------------------------
% For Stefan Pinnow as reminder on what to look for when editing the manual
% -----------------------------------------------------------------------------
% There should be no line breaks in the following environments
% - |...|
% - \declareandlabel{...}
% - \verbpdfref{...}
% If "MakeTikzPictures" isn't running through check one of the externalized
% LOG files what is the cause of that.
% -----------------------------------------------------------------------------


%%%%%%%%%%%%%%%%%%%%%%%%%%%%%%%%%%%%%%%%%%%%%%%%%%%%%%%%%%%%%%%%%%%%%%%%%%%%%
%
% Package pgfplots.sty documentation.
%
% Copyright 2007/2008 by Christian Feuersaenger.
%
% This program is free software: you can redistribute it and/or modify
% it under the terms of the GNU General Public License as published by
% the Free Software Foundation, either version 3 of the License, or
% (at your option) any later version.
%
% This program is distributed in the hope that it will be useful,
% but WITHOUT ANY WARRANTY; without even the implied warranty of
% MERCHANTABILITY or FITNESS FOR A PARTICULAR PURPOSE.  See the
% GNU General Public License for more details.
%
% You should have received a copy of the GNU General Public License
% along with this program.  If not, see <http://www.gnu.org/licenses/>.
%
%
%%%%%%%%%%%%%%%%%%%%%%%%%%%%%%%%%%%%%%%%%%%%%%%%%%%%%%%%%%%%%%%%%%%%%%%%%%%%%
% SEE pgfplots-macros.tex as well!
%\pdfminorversion=5 % to allow compression
%\pdfobjcompresslevel=2
\documentclass[a4paper,openany]{book}

\let\bookmaketitle=\maketitle

% -----------------------------------
% this here is from ltxdoc:
\usepackage{doc}[=v2]
\AtBeginDocument{\MakeShortVerb{\|}}
%\setlength{\textwidth}{355pt}
%\addtolength\marginparwidth{30pt}
%\addtolength\oddsidemargin{20pt}
%\addtolength\evensidemargin{20pt}
\def\cmd#1{\cs{\expandafter\cmd@to@cs\string#1}}
\def\cmd@to@cs#1#2{\char\number`#2\relax}
\DeclareRobustCommand\cs[1]{\texttt{\char`\\#1}}
\providecommand\marg[1]{%
  {\ttfamily\char`\{}\meta{#1}{\ttfamily\char`\}}}
\providecommand\oarg[1]{%
  {\ttfamily[}\meta{#1}{\ttfamily]}}
\providecommand\parg[1]{%
  {\ttfamily(}\meta{#1}{\ttfamily)}}
\raggedbottom

% -----------------------------------
\input{pgfplots.preamble.tex}

\makeatletter
% I want a two-column index just as in pgfmanual styles. This here
% was the best way to get one:
\def\index@prologue{\section*{Index}\addcontentsline{toc}{chapter}{Index}
}
\makeatother

%\RequirePackage[german,english,francais]{babel}

\def\matlabcolormaptext{This colormap is similar to one shipped with Matlab$^\text{\textregistered}$ under a similar name.}

\IfFileExists{tikzlibraryspy.code.tex}{%
\usetikzlibrary{spy}
}{%
    \message{ERROR: tikz SPY library NOT available. The manual will only compile partially.^^J}%
}%

\usepackage{xparse}% for colorbrewer manual

\usetikzlibrary{
    decorations.markings,
    decorations.footprints,
    shapes.arrows,
    matrix,
    positioning,
}

\usepgfplotslibrary{
    fillbetween,
    ternary,
    smithchart,
    patchplots,
    polar,
    colormaps,
    colorbrewer,
    colortol,           % docu not ready yet
}
\pgfqkeys{/codeexample}{%
    every codeexample/.append style={
        /pgfplots/every ternary axis/.append style={
            /pgfplots/legend style={fill=graphicbackground},
        }
    },
    tabsize=4,
}

\pgfplotsmanualenableexternalizationofexpensive

%\usetikzlibrary{external}
%\tikzexternalize[prefix=figures/]{pgfplots}

\title{%
    Manual for Package \PGFPlots{}\\
    {\small 2D/3D Plots in \LaTeX{}, Version \pgfplotsversion}\\
    {\small\url{https://github.com/pgf-tikz/pgfplots}}
    %\\{\small Attention: you are using an unstable development version.}
}

\makeatletter
\long\def\abstractsmuggle{%
    \centering
    \textbf{Abstract}\\[0.5cm]

    \begin{minipage}{12cm}
        \PGFPlots{} draws high-quality function plots in normal or logarithmic
        scaling with a user-friendly interface directly in \TeX{}. The user
        supplies axis labels, legend entries and the plot coordinates for one or
        more plots and \PGFPlots{} applies axis scaling, computes any logarithms
        and axis ticks and draws the plots. It supports line plots, scatter
        plots, piecewise constant plots, bar plots, area plots, mesh and surface
        plots, patch plots, contour plots, quiver plots, histogram plots, box
        plots, polar axes, ternary diagrams, smith charts and some more. It is
        based on Till Tantau's package \PGF{}/\Tikz{}.
    \end{minipage}

}%

\expandafter\date\expandafter{\@date\\[2cm]
    \abstractsmuggle
}%
\makeatother

%\includeonly{pgfplots.reference}


\begin{document}

\def\plotcoords{%
\addplot coordinates {
(5,8.312e-02)    (17,2.547e-02)   (49,7.407e-03)
(129,2.102e-03)  (321,5.874e-04)  (769,1.623e-04)
(1793,4.442e-05) (4097,1.207e-05) (9217,3.261e-06)
};

\addplot coordinates{
(7,8.472e-02)    (31,3.044e-02)    (111,1.022e-02)
(351,3.303e-03)  (1023,1.039e-03)  (2815,3.196e-04)
(7423,9.658e-05) (18943,2.873e-05) (47103,8.437e-06)
};

\addplot coordinates{
(9,7.881e-02)     (49,3.243e-02)    (209,1.232e-02)
(769,4.454e-03)   (2561,1.551e-03)  (7937,5.236e-04)
(23297,1.723e-04) (65537,5.545e-05) (178177,1.751e-05)
};

\addplot coordinates{
(11,6.887e-02)    (71,3.177e-02)     (351,1.341e-02)
(1471,5.334e-03)  (5503,2.027e-03)   (18943,7.415e-04)
(61183,2.628e-04) (187903,9.063e-05) (553983,3.053e-05)
};

\addplot coordinates{
(13,5.755e-02)     (97,2.925e-02)     (545,1.351e-02)
(2561,5.842e-03)   (10625,2.397e-03)  (40193,9.414e-04)
(141569,3.564e-04) (471041,1.308e-04)
(1496065,4.670e-05)
};
}%


\bookmaketitle
\tableofcontents
\include{pgfplots.title_abstract_intro}
\include{pgfplots.preliminaries}
\include{pgfplots.intro}
\include{pgfplots.reference}
\include{pgfplots.libs}
\include{pgfplots.resources}
\include{pgfplots.importexport}
\include{pgfplots.basic.reference}

\printindex

\bibliographystyle{abbrv} %gerapali} %gerabbrv} %gerunsrt.bst} %gerabbrv}% gerplain}
\nocite{pgfplotstable}
\nocite{programmingnotes}
\bibliography{pgfplots}
\end{document}

% -----------------------------------------------------------------------------
% For Stefan Pinnow as reminder on what to look for when editing the manual
% -----------------------------------------------------------------------------
% There should be no line breaks in the following environments
% - |...|
% - \declareandlabel{...}
% - \verbpdfref{...}
% If "MakeTikzPictures" isn't running through check one of the externalized
% LOG files what is the cause of that.
% -----------------------------------------------------------------------------

\include{pgfplots.basic.reference}

\printindex

\bibliographystyle{abbrv} %gerapali} %gerabbrv} %gerunsrt.bst} %gerabbrv}% gerplain}
\nocite{pgfplotstable}
\nocite{programmingnotes}
\bibliography{pgfplots}
\end{document}

% -----------------------------------------------------------------------------
% For Stefan Pinnow as reminder on what to look for when editing the manual
% -----------------------------------------------------------------------------
% There should be no line breaks in the following environments
% - |...|
% - \declareandlabel{...}
% - \verbpdfref{...}
% If "MakeTikzPictures" isn't running through check one of the externalized
% LOG files what is the cause of that.
% -----------------------------------------------------------------------------

\include{pgfplots.basic.reference}

\printindex

\bibliographystyle{abbrv} %gerapali} %gerabbrv} %gerunsrt.bst} %gerabbrv}% gerplain}
\nocite{pgfplotstable}
\nocite{programmingnotes}
\bibliography{pgfplots}
\end{document}

% -----------------------------------------------------------------------------
% For Stefan Pinnow as reminder on what to look for when editing the manual
% -----------------------------------------------------------------------------
% There should be no line breaks in the following environments
% - |...|
% - \declareandlabel{...}
% - \verbpdfref{...}
% If "MakeTikzPictures" isn't running through check one of the externalized
% LOG files what is the cause of that.
% -----------------------------------------------------------------------------


%%%%%%%%%%%%%%%%%%%%%%%%%%%%%%%%%%%%%%%%%%%%%%%%%%%%%%%%%%%%%%%%%%%%%%%%%%%%%
%
% Package pgfplots.sty documentation.
%
% Copyright 2007/2008 by Christian Feuersaenger.
%
% This program is free software: you can redistribute it and/or modify
% it under the terms of the GNU General Public License as published by
% the Free Software Foundation, either version 3 of the License, or
% (at your option) any later version.
%
% This program is distributed in the hope that it will be useful,
% but WITHOUT ANY WARRANTY; without even the implied warranty of
% MERCHANTABILITY or FITNESS FOR A PARTICULAR PURPOSE.  See the
% GNU General Public License for more details.
%
% You should have received a copy of the GNU General Public License
% along with this program.  If not, see <http://www.gnu.org/licenses/>.
%
%
%%%%%%%%%%%%%%%%%%%%%%%%%%%%%%%%%%%%%%%%%%%%%%%%%%%%%%%%%%%%%%%%%%%%%%%%%%%%%
% SEE pgfplots-macros.tex as well!
%\pdfminorversion=5 % to allow compression
%\pdfobjcompresslevel=2
\documentclass[a4paper,openany]{book}

\let\bookmaketitle=\maketitle

% -----------------------------------
% this here is from ltxdoc:
\usepackage{doc}[=v2]
\AtBeginDocument{\MakeShortVerb{\|}}
%\setlength{\textwidth}{355pt}
%\addtolength\marginparwidth{30pt}
%\addtolength\oddsidemargin{20pt}
%\addtolength\evensidemargin{20pt}
\def\cmd#1{\cs{\expandafter\cmd@to@cs\string#1}}
\def\cmd@to@cs#1#2{\char\number`#2\relax}
\DeclareRobustCommand\cs[1]{\texttt{\char`\\#1}}
\providecommand\marg[1]{%
  {\ttfamily\char`\{}\meta{#1}{\ttfamily\char`\}}}
\providecommand\oarg[1]{%
  {\ttfamily[}\meta{#1}{\ttfamily]}}
\providecommand\parg[1]{%
  {\ttfamily(}\meta{#1}{\ttfamily)}}
\raggedbottom

% -----------------------------------
\input{pgfplots.preamble.tex}

\makeatletter
% I want a two-column index just as in pgfmanual styles. This here
% was the best way to get one:
\def\index@prologue{\section*{Index}\addcontentsline{toc}{chapter}{Index}
}
\makeatother

%\RequirePackage[german,english,francais]{babel}

\def\matlabcolormaptext{This colormap is similar to one shipped with Matlab$^\text{\textregistered}$ under a similar name.}

\IfFileExists{tikzlibraryspy.code.tex}{%
\usetikzlibrary{spy}
}{%
    \message{ERROR: tikz SPY library NOT available. The manual will only compile partially.^^J}%
}%

\usepackage{xparse}% for colorbrewer manual

\usetikzlibrary{
    decorations.markings,
    decorations.footprints,
    shapes.arrows,
    matrix,
    positioning,
}

\usepgfplotslibrary{
    fillbetween,
    ternary,
    smithchart,
    patchplots,
    polar,
    colormaps,
    colorbrewer,
    colortol,           % docu not ready yet
}
\pgfqkeys{/codeexample}{%
    every codeexample/.append style={
        /pgfplots/every ternary axis/.append style={
            /pgfplots/legend style={fill=graphicbackground},
        }
    },
    tabsize=4,
}

\pgfplotsmanualenableexternalizationofexpensive

%\usetikzlibrary{external}
%\tikzexternalize[prefix=figures/]{pgfplots}

\title{%
    Manual for Package \PGFPlots{}\\
    {\small 2D/3D Plots in \LaTeX{}, Version \pgfplotsversion}\\
    {\small\url{https://github.com/pgf-tikz/pgfplots}}
    %\\{\small Attention: you are using an unstable development version.}
}

\makeatletter
\long\def\abstractsmuggle{%
    \centering
    \textbf{Abstract}\\[0.5cm]

    \begin{minipage}{12cm}
        \PGFPlots{} draws high-quality function plots in normal or logarithmic
        scaling with a user-friendly interface directly in \TeX{}. The user
        supplies axis labels, legend entries and the plot coordinates for one or
        more plots and \PGFPlots{} applies axis scaling, computes any logarithms
        and axis ticks and draws the plots. It supports line plots, scatter
        plots, piecewise constant plots, bar plots, area plots, mesh and surface
        plots, patch plots, contour plots, quiver plots, histogram plots, box
        plots, polar axes, ternary diagrams, smith charts and some more. It is
        based on Till Tantau's package \PGF{}/\Tikz{}.
    \end{minipage}

}%

\expandafter\date\expandafter{\@date\\[2cm]
    \abstractsmuggle
}%
\makeatother

%\includeonly{pgfplots.reference}


\begin{document}

\def\plotcoords{%
\addplot coordinates {
(5,8.312e-02)    (17,2.547e-02)   (49,7.407e-03)
(129,2.102e-03)  (321,5.874e-04)  (769,1.623e-04)
(1793,4.442e-05) (4097,1.207e-05) (9217,3.261e-06)
};

\addplot coordinates{
(7,8.472e-02)    (31,3.044e-02)    (111,1.022e-02)
(351,3.303e-03)  (1023,1.039e-03)  (2815,3.196e-04)
(7423,9.658e-05) (18943,2.873e-05) (47103,8.437e-06)
};

\addplot coordinates{
(9,7.881e-02)     (49,3.243e-02)    (209,1.232e-02)
(769,4.454e-03)   (2561,1.551e-03)  (7937,5.236e-04)
(23297,1.723e-04) (65537,5.545e-05) (178177,1.751e-05)
};

\addplot coordinates{
(11,6.887e-02)    (71,3.177e-02)     (351,1.341e-02)
(1471,5.334e-03)  (5503,2.027e-03)   (18943,7.415e-04)
(61183,2.628e-04) (187903,9.063e-05) (553983,3.053e-05)
};

\addplot coordinates{
(13,5.755e-02)     (97,2.925e-02)     (545,1.351e-02)
(2561,5.842e-03)   (10625,2.397e-03)  (40193,9.414e-04)
(141569,3.564e-04) (471041,1.308e-04)
(1496065,4.670e-05)
};
}%


\bookmaketitle
\tableofcontents

%%%%%%%%%%%%%%%%%%%%%%%%%%%%%%%%%%%%%%%%%%%%%%%%%%%%%%%%%%%%%%%%%%%%%%%%%%%%%
%
% Package pgfplots.sty documentation.
%
% Copyright 2007/2008 by Christian Feuersaenger.
%
% This program is free software: you can redistribute it and/or modify
% it under the terms of the GNU General Public License as published by
% the Free Software Foundation, either version 3 of the License, or
% (at your option) any later version.
%
% This program is distributed in the hope that it will be useful,
% but WITHOUT ANY WARRANTY; without even the implied warranty of
% MERCHANTABILITY or FITNESS FOR A PARTICULAR PURPOSE.  See the
% GNU General Public License for more details.
%
% You should have received a copy of the GNU General Public License
% along with this program.  If not, see <http://www.gnu.org/licenses/>.
%
%
%%%%%%%%%%%%%%%%%%%%%%%%%%%%%%%%%%%%%%%%%%%%%%%%%%%%%%%%%%%%%%%%%%%%%%%%%%%%%
% SEE pgfplots-macros.tex as well!
%\pdfminorversion=5 % to allow compression
%\pdfobjcompresslevel=2
\documentclass[a4paper,openany]{book}

\let\bookmaketitle=\maketitle

% -----------------------------------
% this here is from ltxdoc:
\usepackage{doc}[=v2]
\AtBeginDocument{\MakeShortVerb{\|}}
%\setlength{\textwidth}{355pt}
%\addtolength\marginparwidth{30pt}
%\addtolength\oddsidemargin{20pt}
%\addtolength\evensidemargin{20pt}
\def\cmd#1{\cs{\expandafter\cmd@to@cs\string#1}}
\def\cmd@to@cs#1#2{\char\number`#2\relax}
\DeclareRobustCommand\cs[1]{\texttt{\char`\\#1}}
\providecommand\marg[1]{%
  {\ttfamily\char`\{}\meta{#1}{\ttfamily\char`\}}}
\providecommand\oarg[1]{%
  {\ttfamily[}\meta{#1}{\ttfamily]}}
\providecommand\parg[1]{%
  {\ttfamily(}\meta{#1}{\ttfamily)}}
\raggedbottom

% -----------------------------------
\input{pgfplots.preamble.tex}

\makeatletter
% I want a two-column index just as in pgfmanual styles. This here
% was the best way to get one:
\def\index@prologue{\section*{Index}\addcontentsline{toc}{chapter}{Index}
}
\makeatother

%\RequirePackage[german,english,francais]{babel}

\def\matlabcolormaptext{This colormap is similar to one shipped with Matlab$^\text{\textregistered}$ under a similar name.}

\IfFileExists{tikzlibraryspy.code.tex}{%
\usetikzlibrary{spy}
}{%
    \message{ERROR: tikz SPY library NOT available. The manual will only compile partially.^^J}%
}%

\usepackage{xparse}% for colorbrewer manual

\usetikzlibrary{
    decorations.markings,
    decorations.footprints,
    shapes.arrows,
    matrix,
    positioning,
}

\usepgfplotslibrary{
    fillbetween,
    ternary,
    smithchart,
    patchplots,
    polar,
    colormaps,
    colorbrewer,
    colortol,           % docu not ready yet
}
\pgfqkeys{/codeexample}{%
    every codeexample/.append style={
        /pgfplots/every ternary axis/.append style={
            /pgfplots/legend style={fill=graphicbackground},
        }
    },
    tabsize=4,
}

\pgfplotsmanualenableexternalizationofexpensive

%\usetikzlibrary{external}
%\tikzexternalize[prefix=figures/]{pgfplots}

\title{%
    Manual for Package \PGFPlots{}\\
    {\small 2D/3D Plots in \LaTeX{}, Version \pgfplotsversion}\\
    {\small\url{https://github.com/pgf-tikz/pgfplots}}
    %\\{\small Attention: you are using an unstable development version.}
}

\makeatletter
\long\def\abstractsmuggle{%
    \centering
    \textbf{Abstract}\\[0.5cm]

    \begin{minipage}{12cm}
        \PGFPlots{} draws high-quality function plots in normal or logarithmic
        scaling with a user-friendly interface directly in \TeX{}. The user
        supplies axis labels, legend entries and the plot coordinates for one or
        more plots and \PGFPlots{} applies axis scaling, computes any logarithms
        and axis ticks and draws the plots. It supports line plots, scatter
        plots, piecewise constant plots, bar plots, area plots, mesh and surface
        plots, patch plots, contour plots, quiver plots, histogram plots, box
        plots, polar axes, ternary diagrams, smith charts and some more. It is
        based on Till Tantau's package \PGF{}/\Tikz{}.
    \end{minipage}

}%

\expandafter\date\expandafter{\@date\\[2cm]
    \abstractsmuggle
}%
\makeatother

%\includeonly{pgfplots.reference}


\begin{document}

\def\plotcoords{%
\addplot coordinates {
(5,8.312e-02)    (17,2.547e-02)   (49,7.407e-03)
(129,2.102e-03)  (321,5.874e-04)  (769,1.623e-04)
(1793,4.442e-05) (4097,1.207e-05) (9217,3.261e-06)
};

\addplot coordinates{
(7,8.472e-02)    (31,3.044e-02)    (111,1.022e-02)
(351,3.303e-03)  (1023,1.039e-03)  (2815,3.196e-04)
(7423,9.658e-05) (18943,2.873e-05) (47103,8.437e-06)
};

\addplot coordinates{
(9,7.881e-02)     (49,3.243e-02)    (209,1.232e-02)
(769,4.454e-03)   (2561,1.551e-03)  (7937,5.236e-04)
(23297,1.723e-04) (65537,5.545e-05) (178177,1.751e-05)
};

\addplot coordinates{
(11,6.887e-02)    (71,3.177e-02)     (351,1.341e-02)
(1471,5.334e-03)  (5503,2.027e-03)   (18943,7.415e-04)
(61183,2.628e-04) (187903,9.063e-05) (553983,3.053e-05)
};

\addplot coordinates{
(13,5.755e-02)     (97,2.925e-02)     (545,1.351e-02)
(2561,5.842e-03)   (10625,2.397e-03)  (40193,9.414e-04)
(141569,3.564e-04) (471041,1.308e-04)
(1496065,4.670e-05)
};
}%


\bookmaketitle
\tableofcontents

%%%%%%%%%%%%%%%%%%%%%%%%%%%%%%%%%%%%%%%%%%%%%%%%%%%%%%%%%%%%%%%%%%%%%%%%%%%%%
%
% Package pgfplots.sty documentation.
%
% Copyright 2007/2008 by Christian Feuersaenger.
%
% This program is free software: you can redistribute it and/or modify
% it under the terms of the GNU General Public License as published by
% the Free Software Foundation, either version 3 of the License, or
% (at your option) any later version.
%
% This program is distributed in the hope that it will be useful,
% but WITHOUT ANY WARRANTY; without even the implied warranty of
% MERCHANTABILITY or FITNESS FOR A PARTICULAR PURPOSE.  See the
% GNU General Public License for more details.
%
% You should have received a copy of the GNU General Public License
% along with this program.  If not, see <http://www.gnu.org/licenses/>.
%
%
%%%%%%%%%%%%%%%%%%%%%%%%%%%%%%%%%%%%%%%%%%%%%%%%%%%%%%%%%%%%%%%%%%%%%%%%%%%%%
% SEE pgfplots-macros.tex as well!
%\pdfminorversion=5 % to allow compression
%\pdfobjcompresslevel=2
\documentclass[a4paper,openany]{book}

\let\bookmaketitle=\maketitle

% -----------------------------------
% this here is from ltxdoc:
\usepackage{doc}[=v2]
\AtBeginDocument{\MakeShortVerb{\|}}
%\setlength{\textwidth}{355pt}
%\addtolength\marginparwidth{30pt}
%\addtolength\oddsidemargin{20pt}
%\addtolength\evensidemargin{20pt}
\def\cmd#1{\cs{\expandafter\cmd@to@cs\string#1}}
\def\cmd@to@cs#1#2{\char\number`#2\relax}
\DeclareRobustCommand\cs[1]{\texttt{\char`\\#1}}
\providecommand\marg[1]{%
  {\ttfamily\char`\{}\meta{#1}{\ttfamily\char`\}}}
\providecommand\oarg[1]{%
  {\ttfamily[}\meta{#1}{\ttfamily]}}
\providecommand\parg[1]{%
  {\ttfamily(}\meta{#1}{\ttfamily)}}
\raggedbottom

% -----------------------------------
\input{pgfplots.preamble.tex}

\makeatletter
% I want a two-column index just as in pgfmanual styles. This here
% was the best way to get one:
\def\index@prologue{\section*{Index}\addcontentsline{toc}{chapter}{Index}
}
\makeatother

%\RequirePackage[german,english,francais]{babel}

\def\matlabcolormaptext{This colormap is similar to one shipped with Matlab$^\text{\textregistered}$ under a similar name.}

\IfFileExists{tikzlibraryspy.code.tex}{%
\usetikzlibrary{spy}
}{%
    \message{ERROR: tikz SPY library NOT available. The manual will only compile partially.^^J}%
}%

\usepackage{xparse}% for colorbrewer manual

\usetikzlibrary{
    decorations.markings,
    decorations.footprints,
    shapes.arrows,
    matrix,
    positioning,
}

\usepgfplotslibrary{
    fillbetween,
    ternary,
    smithchart,
    patchplots,
    polar,
    colormaps,
    colorbrewer,
    colortol,           % docu not ready yet
}
\pgfqkeys{/codeexample}{%
    every codeexample/.append style={
        /pgfplots/every ternary axis/.append style={
            /pgfplots/legend style={fill=graphicbackground},
        }
    },
    tabsize=4,
}

\pgfplotsmanualenableexternalizationofexpensive

%\usetikzlibrary{external}
%\tikzexternalize[prefix=figures/]{pgfplots}

\title{%
    Manual for Package \PGFPlots{}\\
    {\small 2D/3D Plots in \LaTeX{}, Version \pgfplotsversion}\\
    {\small\url{https://github.com/pgf-tikz/pgfplots}}
    %\\{\small Attention: you are using an unstable development version.}
}

\makeatletter
\long\def\abstractsmuggle{%
    \centering
    \textbf{Abstract}\\[0.5cm]

    \begin{minipage}{12cm}
        \PGFPlots{} draws high-quality function plots in normal or logarithmic
        scaling with a user-friendly interface directly in \TeX{}. The user
        supplies axis labels, legend entries and the plot coordinates for one or
        more plots and \PGFPlots{} applies axis scaling, computes any logarithms
        and axis ticks and draws the plots. It supports line plots, scatter
        plots, piecewise constant plots, bar plots, area plots, mesh and surface
        plots, patch plots, contour plots, quiver plots, histogram plots, box
        plots, polar axes, ternary diagrams, smith charts and some more. It is
        based on Till Tantau's package \PGF{}/\Tikz{}.
    \end{minipage}

}%

\expandafter\date\expandafter{\@date\\[2cm]
    \abstractsmuggle
}%
\makeatother

%\includeonly{pgfplots.reference}


\begin{document}

\def\plotcoords{%
\addplot coordinates {
(5,8.312e-02)    (17,2.547e-02)   (49,7.407e-03)
(129,2.102e-03)  (321,5.874e-04)  (769,1.623e-04)
(1793,4.442e-05) (4097,1.207e-05) (9217,3.261e-06)
};

\addplot coordinates{
(7,8.472e-02)    (31,3.044e-02)    (111,1.022e-02)
(351,3.303e-03)  (1023,1.039e-03)  (2815,3.196e-04)
(7423,9.658e-05) (18943,2.873e-05) (47103,8.437e-06)
};

\addplot coordinates{
(9,7.881e-02)     (49,3.243e-02)    (209,1.232e-02)
(769,4.454e-03)   (2561,1.551e-03)  (7937,5.236e-04)
(23297,1.723e-04) (65537,5.545e-05) (178177,1.751e-05)
};

\addplot coordinates{
(11,6.887e-02)    (71,3.177e-02)     (351,1.341e-02)
(1471,5.334e-03)  (5503,2.027e-03)   (18943,7.415e-04)
(61183,2.628e-04) (187903,9.063e-05) (553983,3.053e-05)
};

\addplot coordinates{
(13,5.755e-02)     (97,2.925e-02)     (545,1.351e-02)
(2561,5.842e-03)   (10625,2.397e-03)  (40193,9.414e-04)
(141569,3.564e-04) (471041,1.308e-04)
(1496065,4.670e-05)
};
}%


\bookmaketitle
\tableofcontents
\include{pgfplots.title_abstract_intro}
\include{pgfplots.preliminaries}
\include{pgfplots.intro}
\include{pgfplots.reference}
\include{pgfplots.libs}
\include{pgfplots.resources}
\include{pgfplots.importexport}
\include{pgfplots.basic.reference}

\printindex

\bibliographystyle{abbrv} %gerapali} %gerabbrv} %gerunsrt.bst} %gerabbrv}% gerplain}
\nocite{pgfplotstable}
\nocite{programmingnotes}
\bibliography{pgfplots}
\end{document}

% -----------------------------------------------------------------------------
% For Stefan Pinnow as reminder on what to look for when editing the manual
% -----------------------------------------------------------------------------
% There should be no line breaks in the following environments
% - |...|
% - \declareandlabel{...}
% - \verbpdfref{...}
% If "MakeTikzPictures" isn't running through check one of the externalized
% LOG files what is the cause of that.
% -----------------------------------------------------------------------------


%%%%%%%%%%%%%%%%%%%%%%%%%%%%%%%%%%%%%%%%%%%%%%%%%%%%%%%%%%%%%%%%%%%%%%%%%%%%%
%
% Package pgfplots.sty documentation.
%
% Copyright 2007/2008 by Christian Feuersaenger.
%
% This program is free software: you can redistribute it and/or modify
% it under the terms of the GNU General Public License as published by
% the Free Software Foundation, either version 3 of the License, or
% (at your option) any later version.
%
% This program is distributed in the hope that it will be useful,
% but WITHOUT ANY WARRANTY; without even the implied warranty of
% MERCHANTABILITY or FITNESS FOR A PARTICULAR PURPOSE.  See the
% GNU General Public License for more details.
%
% You should have received a copy of the GNU General Public License
% along with this program.  If not, see <http://www.gnu.org/licenses/>.
%
%
%%%%%%%%%%%%%%%%%%%%%%%%%%%%%%%%%%%%%%%%%%%%%%%%%%%%%%%%%%%%%%%%%%%%%%%%%%%%%
% SEE pgfplots-macros.tex as well!
%\pdfminorversion=5 % to allow compression
%\pdfobjcompresslevel=2
\documentclass[a4paper,openany]{book}

\let\bookmaketitle=\maketitle

% -----------------------------------
% this here is from ltxdoc:
\usepackage{doc}[=v2]
\AtBeginDocument{\MakeShortVerb{\|}}
%\setlength{\textwidth}{355pt}
%\addtolength\marginparwidth{30pt}
%\addtolength\oddsidemargin{20pt}
%\addtolength\evensidemargin{20pt}
\def\cmd#1{\cs{\expandafter\cmd@to@cs\string#1}}
\def\cmd@to@cs#1#2{\char\number`#2\relax}
\DeclareRobustCommand\cs[1]{\texttt{\char`\\#1}}
\providecommand\marg[1]{%
  {\ttfamily\char`\{}\meta{#1}{\ttfamily\char`\}}}
\providecommand\oarg[1]{%
  {\ttfamily[}\meta{#1}{\ttfamily]}}
\providecommand\parg[1]{%
  {\ttfamily(}\meta{#1}{\ttfamily)}}
\raggedbottom

% -----------------------------------
\input{pgfplots.preamble.tex}

\makeatletter
% I want a two-column index just as in pgfmanual styles. This here
% was the best way to get one:
\def\index@prologue{\section*{Index}\addcontentsline{toc}{chapter}{Index}
}
\makeatother

%\RequirePackage[german,english,francais]{babel}

\def\matlabcolormaptext{This colormap is similar to one shipped with Matlab$^\text{\textregistered}$ under a similar name.}

\IfFileExists{tikzlibraryspy.code.tex}{%
\usetikzlibrary{spy}
}{%
    \message{ERROR: tikz SPY library NOT available. The manual will only compile partially.^^J}%
}%

\usepackage{xparse}% for colorbrewer manual

\usetikzlibrary{
    decorations.markings,
    decorations.footprints,
    shapes.arrows,
    matrix,
    positioning,
}

\usepgfplotslibrary{
    fillbetween,
    ternary,
    smithchart,
    patchplots,
    polar,
    colormaps,
    colorbrewer,
    colortol,           % docu not ready yet
}
\pgfqkeys{/codeexample}{%
    every codeexample/.append style={
        /pgfplots/every ternary axis/.append style={
            /pgfplots/legend style={fill=graphicbackground},
        }
    },
    tabsize=4,
}

\pgfplotsmanualenableexternalizationofexpensive

%\usetikzlibrary{external}
%\tikzexternalize[prefix=figures/]{pgfplots}

\title{%
    Manual for Package \PGFPlots{}\\
    {\small 2D/3D Plots in \LaTeX{}, Version \pgfplotsversion}\\
    {\small\url{https://github.com/pgf-tikz/pgfplots}}
    %\\{\small Attention: you are using an unstable development version.}
}

\makeatletter
\long\def\abstractsmuggle{%
    \centering
    \textbf{Abstract}\\[0.5cm]

    \begin{minipage}{12cm}
        \PGFPlots{} draws high-quality function plots in normal or logarithmic
        scaling with a user-friendly interface directly in \TeX{}. The user
        supplies axis labels, legend entries and the plot coordinates for one or
        more plots and \PGFPlots{} applies axis scaling, computes any logarithms
        and axis ticks and draws the plots. It supports line plots, scatter
        plots, piecewise constant plots, bar plots, area plots, mesh and surface
        plots, patch plots, contour plots, quiver plots, histogram plots, box
        plots, polar axes, ternary diagrams, smith charts and some more. It is
        based on Till Tantau's package \PGF{}/\Tikz{}.
    \end{minipage}

}%

\expandafter\date\expandafter{\@date\\[2cm]
    \abstractsmuggle
}%
\makeatother

%\includeonly{pgfplots.reference}


\begin{document}

\def\plotcoords{%
\addplot coordinates {
(5,8.312e-02)    (17,2.547e-02)   (49,7.407e-03)
(129,2.102e-03)  (321,5.874e-04)  (769,1.623e-04)
(1793,4.442e-05) (4097,1.207e-05) (9217,3.261e-06)
};

\addplot coordinates{
(7,8.472e-02)    (31,3.044e-02)    (111,1.022e-02)
(351,3.303e-03)  (1023,1.039e-03)  (2815,3.196e-04)
(7423,9.658e-05) (18943,2.873e-05) (47103,8.437e-06)
};

\addplot coordinates{
(9,7.881e-02)     (49,3.243e-02)    (209,1.232e-02)
(769,4.454e-03)   (2561,1.551e-03)  (7937,5.236e-04)
(23297,1.723e-04) (65537,5.545e-05) (178177,1.751e-05)
};

\addplot coordinates{
(11,6.887e-02)    (71,3.177e-02)     (351,1.341e-02)
(1471,5.334e-03)  (5503,2.027e-03)   (18943,7.415e-04)
(61183,2.628e-04) (187903,9.063e-05) (553983,3.053e-05)
};

\addplot coordinates{
(13,5.755e-02)     (97,2.925e-02)     (545,1.351e-02)
(2561,5.842e-03)   (10625,2.397e-03)  (40193,9.414e-04)
(141569,3.564e-04) (471041,1.308e-04)
(1496065,4.670e-05)
};
}%


\bookmaketitle
\tableofcontents
\include{pgfplots.title_abstract_intro}
\include{pgfplots.preliminaries}
\include{pgfplots.intro}
\include{pgfplots.reference}
\include{pgfplots.libs}
\include{pgfplots.resources}
\include{pgfplots.importexport}
\include{pgfplots.basic.reference}

\printindex

\bibliographystyle{abbrv} %gerapali} %gerabbrv} %gerunsrt.bst} %gerabbrv}% gerplain}
\nocite{pgfplotstable}
\nocite{programmingnotes}
\bibliography{pgfplots}
\end{document}

% -----------------------------------------------------------------------------
% For Stefan Pinnow as reminder on what to look for when editing the manual
% -----------------------------------------------------------------------------
% There should be no line breaks in the following environments
% - |...|
% - \declareandlabel{...}
% - \verbpdfref{...}
% If "MakeTikzPictures" isn't running through check one of the externalized
% LOG files what is the cause of that.
% -----------------------------------------------------------------------------


%%%%%%%%%%%%%%%%%%%%%%%%%%%%%%%%%%%%%%%%%%%%%%%%%%%%%%%%%%%%%%%%%%%%%%%%%%%%%
%
% Package pgfplots.sty documentation.
%
% Copyright 2007/2008 by Christian Feuersaenger.
%
% This program is free software: you can redistribute it and/or modify
% it under the terms of the GNU General Public License as published by
% the Free Software Foundation, either version 3 of the License, or
% (at your option) any later version.
%
% This program is distributed in the hope that it will be useful,
% but WITHOUT ANY WARRANTY; without even the implied warranty of
% MERCHANTABILITY or FITNESS FOR A PARTICULAR PURPOSE.  See the
% GNU General Public License for more details.
%
% You should have received a copy of the GNU General Public License
% along with this program.  If not, see <http://www.gnu.org/licenses/>.
%
%
%%%%%%%%%%%%%%%%%%%%%%%%%%%%%%%%%%%%%%%%%%%%%%%%%%%%%%%%%%%%%%%%%%%%%%%%%%%%%
% SEE pgfplots-macros.tex as well!
%\pdfminorversion=5 % to allow compression
%\pdfobjcompresslevel=2
\documentclass[a4paper,openany]{book}

\let\bookmaketitle=\maketitle

% -----------------------------------
% this here is from ltxdoc:
\usepackage{doc}[=v2]
\AtBeginDocument{\MakeShortVerb{\|}}
%\setlength{\textwidth}{355pt}
%\addtolength\marginparwidth{30pt}
%\addtolength\oddsidemargin{20pt}
%\addtolength\evensidemargin{20pt}
\def\cmd#1{\cs{\expandafter\cmd@to@cs\string#1}}
\def\cmd@to@cs#1#2{\char\number`#2\relax}
\DeclareRobustCommand\cs[1]{\texttt{\char`\\#1}}
\providecommand\marg[1]{%
  {\ttfamily\char`\{}\meta{#1}{\ttfamily\char`\}}}
\providecommand\oarg[1]{%
  {\ttfamily[}\meta{#1}{\ttfamily]}}
\providecommand\parg[1]{%
  {\ttfamily(}\meta{#1}{\ttfamily)}}
\raggedbottom

% -----------------------------------
\input{pgfplots.preamble.tex}

\makeatletter
% I want a two-column index just as in pgfmanual styles. This here
% was the best way to get one:
\def\index@prologue{\section*{Index}\addcontentsline{toc}{chapter}{Index}
}
\makeatother

%\RequirePackage[german,english,francais]{babel}

\def\matlabcolormaptext{This colormap is similar to one shipped with Matlab$^\text{\textregistered}$ under a similar name.}

\IfFileExists{tikzlibraryspy.code.tex}{%
\usetikzlibrary{spy}
}{%
    \message{ERROR: tikz SPY library NOT available. The manual will only compile partially.^^J}%
}%

\usepackage{xparse}% for colorbrewer manual

\usetikzlibrary{
    decorations.markings,
    decorations.footprints,
    shapes.arrows,
    matrix,
    positioning,
}

\usepgfplotslibrary{
    fillbetween,
    ternary,
    smithchart,
    patchplots,
    polar,
    colormaps,
    colorbrewer,
    colortol,           % docu not ready yet
}
\pgfqkeys{/codeexample}{%
    every codeexample/.append style={
        /pgfplots/every ternary axis/.append style={
            /pgfplots/legend style={fill=graphicbackground},
        }
    },
    tabsize=4,
}

\pgfplotsmanualenableexternalizationofexpensive

%\usetikzlibrary{external}
%\tikzexternalize[prefix=figures/]{pgfplots}

\title{%
    Manual for Package \PGFPlots{}\\
    {\small 2D/3D Plots in \LaTeX{}, Version \pgfplotsversion}\\
    {\small\url{https://github.com/pgf-tikz/pgfplots}}
    %\\{\small Attention: you are using an unstable development version.}
}

\makeatletter
\long\def\abstractsmuggle{%
    \centering
    \textbf{Abstract}\\[0.5cm]

    \begin{minipage}{12cm}
        \PGFPlots{} draws high-quality function plots in normal or logarithmic
        scaling with a user-friendly interface directly in \TeX{}. The user
        supplies axis labels, legend entries and the plot coordinates for one or
        more plots and \PGFPlots{} applies axis scaling, computes any logarithms
        and axis ticks and draws the plots. It supports line plots, scatter
        plots, piecewise constant plots, bar plots, area plots, mesh and surface
        plots, patch plots, contour plots, quiver plots, histogram plots, box
        plots, polar axes, ternary diagrams, smith charts and some more. It is
        based on Till Tantau's package \PGF{}/\Tikz{}.
    \end{minipage}

}%

\expandafter\date\expandafter{\@date\\[2cm]
    \abstractsmuggle
}%
\makeatother

%\includeonly{pgfplots.reference}


\begin{document}

\def\plotcoords{%
\addplot coordinates {
(5,8.312e-02)    (17,2.547e-02)   (49,7.407e-03)
(129,2.102e-03)  (321,5.874e-04)  (769,1.623e-04)
(1793,4.442e-05) (4097,1.207e-05) (9217,3.261e-06)
};

\addplot coordinates{
(7,8.472e-02)    (31,3.044e-02)    (111,1.022e-02)
(351,3.303e-03)  (1023,1.039e-03)  (2815,3.196e-04)
(7423,9.658e-05) (18943,2.873e-05) (47103,8.437e-06)
};

\addplot coordinates{
(9,7.881e-02)     (49,3.243e-02)    (209,1.232e-02)
(769,4.454e-03)   (2561,1.551e-03)  (7937,5.236e-04)
(23297,1.723e-04) (65537,5.545e-05) (178177,1.751e-05)
};

\addplot coordinates{
(11,6.887e-02)    (71,3.177e-02)     (351,1.341e-02)
(1471,5.334e-03)  (5503,2.027e-03)   (18943,7.415e-04)
(61183,2.628e-04) (187903,9.063e-05) (553983,3.053e-05)
};

\addplot coordinates{
(13,5.755e-02)     (97,2.925e-02)     (545,1.351e-02)
(2561,5.842e-03)   (10625,2.397e-03)  (40193,9.414e-04)
(141569,3.564e-04) (471041,1.308e-04)
(1496065,4.670e-05)
};
}%


\bookmaketitle
\tableofcontents
\include{pgfplots.title_abstract_intro}
\include{pgfplots.preliminaries}
\include{pgfplots.intro}
\include{pgfplots.reference}
\include{pgfplots.libs}
\include{pgfplots.resources}
\include{pgfplots.importexport}
\include{pgfplots.basic.reference}

\printindex

\bibliographystyle{abbrv} %gerapali} %gerabbrv} %gerunsrt.bst} %gerabbrv}% gerplain}
\nocite{pgfplotstable}
\nocite{programmingnotes}
\bibliography{pgfplots}
\end{document}

% -----------------------------------------------------------------------------
% For Stefan Pinnow as reminder on what to look for when editing the manual
% -----------------------------------------------------------------------------
% There should be no line breaks in the following environments
% - |...|
% - \declareandlabel{...}
% - \verbpdfref{...}
% If "MakeTikzPictures" isn't running through check one of the externalized
% LOG files what is the cause of that.
% -----------------------------------------------------------------------------


%%%%%%%%%%%%%%%%%%%%%%%%%%%%%%%%%%%%%%%%%%%%%%%%%%%%%%%%%%%%%%%%%%%%%%%%%%%%%
%
% Package pgfplots.sty documentation.
%
% Copyright 2007/2008 by Christian Feuersaenger.
%
% This program is free software: you can redistribute it and/or modify
% it under the terms of the GNU General Public License as published by
% the Free Software Foundation, either version 3 of the License, or
% (at your option) any later version.
%
% This program is distributed in the hope that it will be useful,
% but WITHOUT ANY WARRANTY; without even the implied warranty of
% MERCHANTABILITY or FITNESS FOR A PARTICULAR PURPOSE.  See the
% GNU General Public License for more details.
%
% You should have received a copy of the GNU General Public License
% along with this program.  If not, see <http://www.gnu.org/licenses/>.
%
%
%%%%%%%%%%%%%%%%%%%%%%%%%%%%%%%%%%%%%%%%%%%%%%%%%%%%%%%%%%%%%%%%%%%%%%%%%%%%%
% SEE pgfplots-macros.tex as well!
%\pdfminorversion=5 % to allow compression
%\pdfobjcompresslevel=2
\documentclass[a4paper,openany]{book}

\let\bookmaketitle=\maketitle

% -----------------------------------
% this here is from ltxdoc:
\usepackage{doc}[=v2]
\AtBeginDocument{\MakeShortVerb{\|}}
%\setlength{\textwidth}{355pt}
%\addtolength\marginparwidth{30pt}
%\addtolength\oddsidemargin{20pt}
%\addtolength\evensidemargin{20pt}
\def\cmd#1{\cs{\expandafter\cmd@to@cs\string#1}}
\def\cmd@to@cs#1#2{\char\number`#2\relax}
\DeclareRobustCommand\cs[1]{\texttt{\char`\\#1}}
\providecommand\marg[1]{%
  {\ttfamily\char`\{}\meta{#1}{\ttfamily\char`\}}}
\providecommand\oarg[1]{%
  {\ttfamily[}\meta{#1}{\ttfamily]}}
\providecommand\parg[1]{%
  {\ttfamily(}\meta{#1}{\ttfamily)}}
\raggedbottom

% -----------------------------------
\input{pgfplots.preamble.tex}

\makeatletter
% I want a two-column index just as in pgfmanual styles. This here
% was the best way to get one:
\def\index@prologue{\section*{Index}\addcontentsline{toc}{chapter}{Index}
}
\makeatother

%\RequirePackage[german,english,francais]{babel}

\def\matlabcolormaptext{This colormap is similar to one shipped with Matlab$^\text{\textregistered}$ under a similar name.}

\IfFileExists{tikzlibraryspy.code.tex}{%
\usetikzlibrary{spy}
}{%
    \message{ERROR: tikz SPY library NOT available. The manual will only compile partially.^^J}%
}%

\usepackage{xparse}% for colorbrewer manual

\usetikzlibrary{
    decorations.markings,
    decorations.footprints,
    shapes.arrows,
    matrix,
    positioning,
}

\usepgfplotslibrary{
    fillbetween,
    ternary,
    smithchart,
    patchplots,
    polar,
    colormaps,
    colorbrewer,
    colortol,           % docu not ready yet
}
\pgfqkeys{/codeexample}{%
    every codeexample/.append style={
        /pgfplots/every ternary axis/.append style={
            /pgfplots/legend style={fill=graphicbackground},
        }
    },
    tabsize=4,
}

\pgfplotsmanualenableexternalizationofexpensive

%\usetikzlibrary{external}
%\tikzexternalize[prefix=figures/]{pgfplots}

\title{%
    Manual for Package \PGFPlots{}\\
    {\small 2D/3D Plots in \LaTeX{}, Version \pgfplotsversion}\\
    {\small\url{https://github.com/pgf-tikz/pgfplots}}
    %\\{\small Attention: you are using an unstable development version.}
}

\makeatletter
\long\def\abstractsmuggle{%
    \centering
    \textbf{Abstract}\\[0.5cm]

    \begin{minipage}{12cm}
        \PGFPlots{} draws high-quality function plots in normal or logarithmic
        scaling with a user-friendly interface directly in \TeX{}. The user
        supplies axis labels, legend entries and the plot coordinates for one or
        more plots and \PGFPlots{} applies axis scaling, computes any logarithms
        and axis ticks and draws the plots. It supports line plots, scatter
        plots, piecewise constant plots, bar plots, area plots, mesh and surface
        plots, patch plots, contour plots, quiver plots, histogram plots, box
        plots, polar axes, ternary diagrams, smith charts and some more. It is
        based on Till Tantau's package \PGF{}/\Tikz{}.
    \end{minipage}

}%

\expandafter\date\expandafter{\@date\\[2cm]
    \abstractsmuggle
}%
\makeatother

%\includeonly{pgfplots.reference}


\begin{document}

\def\plotcoords{%
\addplot coordinates {
(5,8.312e-02)    (17,2.547e-02)   (49,7.407e-03)
(129,2.102e-03)  (321,5.874e-04)  (769,1.623e-04)
(1793,4.442e-05) (4097,1.207e-05) (9217,3.261e-06)
};

\addplot coordinates{
(7,8.472e-02)    (31,3.044e-02)    (111,1.022e-02)
(351,3.303e-03)  (1023,1.039e-03)  (2815,3.196e-04)
(7423,9.658e-05) (18943,2.873e-05) (47103,8.437e-06)
};

\addplot coordinates{
(9,7.881e-02)     (49,3.243e-02)    (209,1.232e-02)
(769,4.454e-03)   (2561,1.551e-03)  (7937,5.236e-04)
(23297,1.723e-04) (65537,5.545e-05) (178177,1.751e-05)
};

\addplot coordinates{
(11,6.887e-02)    (71,3.177e-02)     (351,1.341e-02)
(1471,5.334e-03)  (5503,2.027e-03)   (18943,7.415e-04)
(61183,2.628e-04) (187903,9.063e-05) (553983,3.053e-05)
};

\addplot coordinates{
(13,5.755e-02)     (97,2.925e-02)     (545,1.351e-02)
(2561,5.842e-03)   (10625,2.397e-03)  (40193,9.414e-04)
(141569,3.564e-04) (471041,1.308e-04)
(1496065,4.670e-05)
};
}%


\bookmaketitle
\tableofcontents
\include{pgfplots.title_abstract_intro}
\include{pgfplots.preliminaries}
\include{pgfplots.intro}
\include{pgfplots.reference}
\include{pgfplots.libs}
\include{pgfplots.resources}
\include{pgfplots.importexport}
\include{pgfplots.basic.reference}

\printindex

\bibliographystyle{abbrv} %gerapali} %gerabbrv} %gerunsrt.bst} %gerabbrv}% gerplain}
\nocite{pgfplotstable}
\nocite{programmingnotes}
\bibliography{pgfplots}
\end{document}

% -----------------------------------------------------------------------------
% For Stefan Pinnow as reminder on what to look for when editing the manual
% -----------------------------------------------------------------------------
% There should be no line breaks in the following environments
% - |...|
% - \declareandlabel{...}
% - \verbpdfref{...}
% If "MakeTikzPictures" isn't running through check one of the externalized
% LOG files what is the cause of that.
% -----------------------------------------------------------------------------


%%%%%%%%%%%%%%%%%%%%%%%%%%%%%%%%%%%%%%%%%%%%%%%%%%%%%%%%%%%%%%%%%%%%%%%%%%%%%
%
% Package pgfplots.sty documentation.
%
% Copyright 2007/2008 by Christian Feuersaenger.
%
% This program is free software: you can redistribute it and/or modify
% it under the terms of the GNU General Public License as published by
% the Free Software Foundation, either version 3 of the License, or
% (at your option) any later version.
%
% This program is distributed in the hope that it will be useful,
% but WITHOUT ANY WARRANTY; without even the implied warranty of
% MERCHANTABILITY or FITNESS FOR A PARTICULAR PURPOSE.  See the
% GNU General Public License for more details.
%
% You should have received a copy of the GNU General Public License
% along with this program.  If not, see <http://www.gnu.org/licenses/>.
%
%
%%%%%%%%%%%%%%%%%%%%%%%%%%%%%%%%%%%%%%%%%%%%%%%%%%%%%%%%%%%%%%%%%%%%%%%%%%%%%
% SEE pgfplots-macros.tex as well!
%\pdfminorversion=5 % to allow compression
%\pdfobjcompresslevel=2
\documentclass[a4paper,openany]{book}

\let\bookmaketitle=\maketitle

% -----------------------------------
% this here is from ltxdoc:
\usepackage{doc}[=v2]
\AtBeginDocument{\MakeShortVerb{\|}}
%\setlength{\textwidth}{355pt}
%\addtolength\marginparwidth{30pt}
%\addtolength\oddsidemargin{20pt}
%\addtolength\evensidemargin{20pt}
\def\cmd#1{\cs{\expandafter\cmd@to@cs\string#1}}
\def\cmd@to@cs#1#2{\char\number`#2\relax}
\DeclareRobustCommand\cs[1]{\texttt{\char`\\#1}}
\providecommand\marg[1]{%
  {\ttfamily\char`\{}\meta{#1}{\ttfamily\char`\}}}
\providecommand\oarg[1]{%
  {\ttfamily[}\meta{#1}{\ttfamily]}}
\providecommand\parg[1]{%
  {\ttfamily(}\meta{#1}{\ttfamily)}}
\raggedbottom

% -----------------------------------
\input{pgfplots.preamble.tex}

\makeatletter
% I want a two-column index just as in pgfmanual styles. This here
% was the best way to get one:
\def\index@prologue{\section*{Index}\addcontentsline{toc}{chapter}{Index}
}
\makeatother

%\RequirePackage[german,english,francais]{babel}

\def\matlabcolormaptext{This colormap is similar to one shipped with Matlab$^\text{\textregistered}$ under a similar name.}

\IfFileExists{tikzlibraryspy.code.tex}{%
\usetikzlibrary{spy}
}{%
    \message{ERROR: tikz SPY library NOT available. The manual will only compile partially.^^J}%
}%

\usepackage{xparse}% for colorbrewer manual

\usetikzlibrary{
    decorations.markings,
    decorations.footprints,
    shapes.arrows,
    matrix,
    positioning,
}

\usepgfplotslibrary{
    fillbetween,
    ternary,
    smithchart,
    patchplots,
    polar,
    colormaps,
    colorbrewer,
    colortol,           % docu not ready yet
}
\pgfqkeys{/codeexample}{%
    every codeexample/.append style={
        /pgfplots/every ternary axis/.append style={
            /pgfplots/legend style={fill=graphicbackground},
        }
    },
    tabsize=4,
}

\pgfplotsmanualenableexternalizationofexpensive

%\usetikzlibrary{external}
%\tikzexternalize[prefix=figures/]{pgfplots}

\title{%
    Manual for Package \PGFPlots{}\\
    {\small 2D/3D Plots in \LaTeX{}, Version \pgfplotsversion}\\
    {\small\url{https://github.com/pgf-tikz/pgfplots}}
    %\\{\small Attention: you are using an unstable development version.}
}

\makeatletter
\long\def\abstractsmuggle{%
    \centering
    \textbf{Abstract}\\[0.5cm]

    \begin{minipage}{12cm}
        \PGFPlots{} draws high-quality function plots in normal or logarithmic
        scaling with a user-friendly interface directly in \TeX{}. The user
        supplies axis labels, legend entries and the plot coordinates for one or
        more plots and \PGFPlots{} applies axis scaling, computes any logarithms
        and axis ticks and draws the plots. It supports line plots, scatter
        plots, piecewise constant plots, bar plots, area plots, mesh and surface
        plots, patch plots, contour plots, quiver plots, histogram plots, box
        plots, polar axes, ternary diagrams, smith charts and some more. It is
        based on Till Tantau's package \PGF{}/\Tikz{}.
    \end{minipage}

}%

\expandafter\date\expandafter{\@date\\[2cm]
    \abstractsmuggle
}%
\makeatother

%\includeonly{pgfplots.reference}


\begin{document}

\def\plotcoords{%
\addplot coordinates {
(5,8.312e-02)    (17,2.547e-02)   (49,7.407e-03)
(129,2.102e-03)  (321,5.874e-04)  (769,1.623e-04)
(1793,4.442e-05) (4097,1.207e-05) (9217,3.261e-06)
};

\addplot coordinates{
(7,8.472e-02)    (31,3.044e-02)    (111,1.022e-02)
(351,3.303e-03)  (1023,1.039e-03)  (2815,3.196e-04)
(7423,9.658e-05) (18943,2.873e-05) (47103,8.437e-06)
};

\addplot coordinates{
(9,7.881e-02)     (49,3.243e-02)    (209,1.232e-02)
(769,4.454e-03)   (2561,1.551e-03)  (7937,5.236e-04)
(23297,1.723e-04) (65537,5.545e-05) (178177,1.751e-05)
};

\addplot coordinates{
(11,6.887e-02)    (71,3.177e-02)     (351,1.341e-02)
(1471,5.334e-03)  (5503,2.027e-03)   (18943,7.415e-04)
(61183,2.628e-04) (187903,9.063e-05) (553983,3.053e-05)
};

\addplot coordinates{
(13,5.755e-02)     (97,2.925e-02)     (545,1.351e-02)
(2561,5.842e-03)   (10625,2.397e-03)  (40193,9.414e-04)
(141569,3.564e-04) (471041,1.308e-04)
(1496065,4.670e-05)
};
}%


\bookmaketitle
\tableofcontents
\include{pgfplots.title_abstract_intro}
\include{pgfplots.preliminaries}
\include{pgfplots.intro}
\include{pgfplots.reference}
\include{pgfplots.libs}
\include{pgfplots.resources}
\include{pgfplots.importexport}
\include{pgfplots.basic.reference}

\printindex

\bibliographystyle{abbrv} %gerapali} %gerabbrv} %gerunsrt.bst} %gerabbrv}% gerplain}
\nocite{pgfplotstable}
\nocite{programmingnotes}
\bibliography{pgfplots}
\end{document}

% -----------------------------------------------------------------------------
% For Stefan Pinnow as reminder on what to look for when editing the manual
% -----------------------------------------------------------------------------
% There should be no line breaks in the following environments
% - |...|
% - \declareandlabel{...}
% - \verbpdfref{...}
% If "MakeTikzPictures" isn't running through check one of the externalized
% LOG files what is the cause of that.
% -----------------------------------------------------------------------------


%%%%%%%%%%%%%%%%%%%%%%%%%%%%%%%%%%%%%%%%%%%%%%%%%%%%%%%%%%%%%%%%%%%%%%%%%%%%%
%
% Package pgfplots.sty documentation.
%
% Copyright 2007/2008 by Christian Feuersaenger.
%
% This program is free software: you can redistribute it and/or modify
% it under the terms of the GNU General Public License as published by
% the Free Software Foundation, either version 3 of the License, or
% (at your option) any later version.
%
% This program is distributed in the hope that it will be useful,
% but WITHOUT ANY WARRANTY; without even the implied warranty of
% MERCHANTABILITY or FITNESS FOR A PARTICULAR PURPOSE.  See the
% GNU General Public License for more details.
%
% You should have received a copy of the GNU General Public License
% along with this program.  If not, see <http://www.gnu.org/licenses/>.
%
%
%%%%%%%%%%%%%%%%%%%%%%%%%%%%%%%%%%%%%%%%%%%%%%%%%%%%%%%%%%%%%%%%%%%%%%%%%%%%%
% SEE pgfplots-macros.tex as well!
%\pdfminorversion=5 % to allow compression
%\pdfobjcompresslevel=2
\documentclass[a4paper,openany]{book}

\let\bookmaketitle=\maketitle

% -----------------------------------
% this here is from ltxdoc:
\usepackage{doc}[=v2]
\AtBeginDocument{\MakeShortVerb{\|}}
%\setlength{\textwidth}{355pt}
%\addtolength\marginparwidth{30pt}
%\addtolength\oddsidemargin{20pt}
%\addtolength\evensidemargin{20pt}
\def\cmd#1{\cs{\expandafter\cmd@to@cs\string#1}}
\def\cmd@to@cs#1#2{\char\number`#2\relax}
\DeclareRobustCommand\cs[1]{\texttt{\char`\\#1}}
\providecommand\marg[1]{%
  {\ttfamily\char`\{}\meta{#1}{\ttfamily\char`\}}}
\providecommand\oarg[1]{%
  {\ttfamily[}\meta{#1}{\ttfamily]}}
\providecommand\parg[1]{%
  {\ttfamily(}\meta{#1}{\ttfamily)}}
\raggedbottom

% -----------------------------------
\input{pgfplots.preamble.tex}

\makeatletter
% I want a two-column index just as in pgfmanual styles. This here
% was the best way to get one:
\def\index@prologue{\section*{Index}\addcontentsline{toc}{chapter}{Index}
}
\makeatother

%\RequirePackage[german,english,francais]{babel}

\def\matlabcolormaptext{This colormap is similar to one shipped with Matlab$^\text{\textregistered}$ under a similar name.}

\IfFileExists{tikzlibraryspy.code.tex}{%
\usetikzlibrary{spy}
}{%
    \message{ERROR: tikz SPY library NOT available. The manual will only compile partially.^^J}%
}%

\usepackage{xparse}% for colorbrewer manual

\usetikzlibrary{
    decorations.markings,
    decorations.footprints,
    shapes.arrows,
    matrix,
    positioning,
}

\usepgfplotslibrary{
    fillbetween,
    ternary,
    smithchart,
    patchplots,
    polar,
    colormaps,
    colorbrewer,
    colortol,           % docu not ready yet
}
\pgfqkeys{/codeexample}{%
    every codeexample/.append style={
        /pgfplots/every ternary axis/.append style={
            /pgfplots/legend style={fill=graphicbackground},
        }
    },
    tabsize=4,
}

\pgfplotsmanualenableexternalizationofexpensive

%\usetikzlibrary{external}
%\tikzexternalize[prefix=figures/]{pgfplots}

\title{%
    Manual for Package \PGFPlots{}\\
    {\small 2D/3D Plots in \LaTeX{}, Version \pgfplotsversion}\\
    {\small\url{https://github.com/pgf-tikz/pgfplots}}
    %\\{\small Attention: you are using an unstable development version.}
}

\makeatletter
\long\def\abstractsmuggle{%
    \centering
    \textbf{Abstract}\\[0.5cm]

    \begin{minipage}{12cm}
        \PGFPlots{} draws high-quality function plots in normal or logarithmic
        scaling with a user-friendly interface directly in \TeX{}. The user
        supplies axis labels, legend entries and the plot coordinates for one or
        more plots and \PGFPlots{} applies axis scaling, computes any logarithms
        and axis ticks and draws the plots. It supports line plots, scatter
        plots, piecewise constant plots, bar plots, area plots, mesh and surface
        plots, patch plots, contour plots, quiver plots, histogram plots, box
        plots, polar axes, ternary diagrams, smith charts and some more. It is
        based on Till Tantau's package \PGF{}/\Tikz{}.
    \end{minipage}

}%

\expandafter\date\expandafter{\@date\\[2cm]
    \abstractsmuggle
}%
\makeatother

%\includeonly{pgfplots.reference}


\begin{document}

\def\plotcoords{%
\addplot coordinates {
(5,8.312e-02)    (17,2.547e-02)   (49,7.407e-03)
(129,2.102e-03)  (321,5.874e-04)  (769,1.623e-04)
(1793,4.442e-05) (4097,1.207e-05) (9217,3.261e-06)
};

\addplot coordinates{
(7,8.472e-02)    (31,3.044e-02)    (111,1.022e-02)
(351,3.303e-03)  (1023,1.039e-03)  (2815,3.196e-04)
(7423,9.658e-05) (18943,2.873e-05) (47103,8.437e-06)
};

\addplot coordinates{
(9,7.881e-02)     (49,3.243e-02)    (209,1.232e-02)
(769,4.454e-03)   (2561,1.551e-03)  (7937,5.236e-04)
(23297,1.723e-04) (65537,5.545e-05) (178177,1.751e-05)
};

\addplot coordinates{
(11,6.887e-02)    (71,3.177e-02)     (351,1.341e-02)
(1471,5.334e-03)  (5503,2.027e-03)   (18943,7.415e-04)
(61183,2.628e-04) (187903,9.063e-05) (553983,3.053e-05)
};

\addplot coordinates{
(13,5.755e-02)     (97,2.925e-02)     (545,1.351e-02)
(2561,5.842e-03)   (10625,2.397e-03)  (40193,9.414e-04)
(141569,3.564e-04) (471041,1.308e-04)
(1496065,4.670e-05)
};
}%


\bookmaketitle
\tableofcontents
\include{pgfplots.title_abstract_intro}
\include{pgfplots.preliminaries}
\include{pgfplots.intro}
\include{pgfplots.reference}
\include{pgfplots.libs}
\include{pgfplots.resources}
\include{pgfplots.importexport}
\include{pgfplots.basic.reference}

\printindex

\bibliographystyle{abbrv} %gerapali} %gerabbrv} %gerunsrt.bst} %gerabbrv}% gerplain}
\nocite{pgfplotstable}
\nocite{programmingnotes}
\bibliography{pgfplots}
\end{document}

% -----------------------------------------------------------------------------
% For Stefan Pinnow as reminder on what to look for when editing the manual
% -----------------------------------------------------------------------------
% There should be no line breaks in the following environments
% - |...|
% - \declareandlabel{...}
% - \verbpdfref{...}
% If "MakeTikzPictures" isn't running through check one of the externalized
% LOG files what is the cause of that.
% -----------------------------------------------------------------------------


%%%%%%%%%%%%%%%%%%%%%%%%%%%%%%%%%%%%%%%%%%%%%%%%%%%%%%%%%%%%%%%%%%%%%%%%%%%%%
%
% Package pgfplots.sty documentation.
%
% Copyright 2007/2008 by Christian Feuersaenger.
%
% This program is free software: you can redistribute it and/or modify
% it under the terms of the GNU General Public License as published by
% the Free Software Foundation, either version 3 of the License, or
% (at your option) any later version.
%
% This program is distributed in the hope that it will be useful,
% but WITHOUT ANY WARRANTY; without even the implied warranty of
% MERCHANTABILITY or FITNESS FOR A PARTICULAR PURPOSE.  See the
% GNU General Public License for more details.
%
% You should have received a copy of the GNU General Public License
% along with this program.  If not, see <http://www.gnu.org/licenses/>.
%
%
%%%%%%%%%%%%%%%%%%%%%%%%%%%%%%%%%%%%%%%%%%%%%%%%%%%%%%%%%%%%%%%%%%%%%%%%%%%%%
% SEE pgfplots-macros.tex as well!
%\pdfminorversion=5 % to allow compression
%\pdfobjcompresslevel=2
\documentclass[a4paper,openany]{book}

\let\bookmaketitle=\maketitle

% -----------------------------------
% this here is from ltxdoc:
\usepackage{doc}[=v2]
\AtBeginDocument{\MakeShortVerb{\|}}
%\setlength{\textwidth}{355pt}
%\addtolength\marginparwidth{30pt}
%\addtolength\oddsidemargin{20pt}
%\addtolength\evensidemargin{20pt}
\def\cmd#1{\cs{\expandafter\cmd@to@cs\string#1}}
\def\cmd@to@cs#1#2{\char\number`#2\relax}
\DeclareRobustCommand\cs[1]{\texttt{\char`\\#1}}
\providecommand\marg[1]{%
  {\ttfamily\char`\{}\meta{#1}{\ttfamily\char`\}}}
\providecommand\oarg[1]{%
  {\ttfamily[}\meta{#1}{\ttfamily]}}
\providecommand\parg[1]{%
  {\ttfamily(}\meta{#1}{\ttfamily)}}
\raggedbottom

% -----------------------------------
\input{pgfplots.preamble.tex}

\makeatletter
% I want a two-column index just as in pgfmanual styles. This here
% was the best way to get one:
\def\index@prologue{\section*{Index}\addcontentsline{toc}{chapter}{Index}
}
\makeatother

%\RequirePackage[german,english,francais]{babel}

\def\matlabcolormaptext{This colormap is similar to one shipped with Matlab$^\text{\textregistered}$ under a similar name.}

\IfFileExists{tikzlibraryspy.code.tex}{%
\usetikzlibrary{spy}
}{%
    \message{ERROR: tikz SPY library NOT available. The manual will only compile partially.^^J}%
}%

\usepackage{xparse}% for colorbrewer manual

\usetikzlibrary{
    decorations.markings,
    decorations.footprints,
    shapes.arrows,
    matrix,
    positioning,
}

\usepgfplotslibrary{
    fillbetween,
    ternary,
    smithchart,
    patchplots,
    polar,
    colormaps,
    colorbrewer,
    colortol,           % docu not ready yet
}
\pgfqkeys{/codeexample}{%
    every codeexample/.append style={
        /pgfplots/every ternary axis/.append style={
            /pgfplots/legend style={fill=graphicbackground},
        }
    },
    tabsize=4,
}

\pgfplotsmanualenableexternalizationofexpensive

%\usetikzlibrary{external}
%\tikzexternalize[prefix=figures/]{pgfplots}

\title{%
    Manual for Package \PGFPlots{}\\
    {\small 2D/3D Plots in \LaTeX{}, Version \pgfplotsversion}\\
    {\small\url{https://github.com/pgf-tikz/pgfplots}}
    %\\{\small Attention: you are using an unstable development version.}
}

\makeatletter
\long\def\abstractsmuggle{%
    \centering
    \textbf{Abstract}\\[0.5cm]

    \begin{minipage}{12cm}
        \PGFPlots{} draws high-quality function plots in normal or logarithmic
        scaling with a user-friendly interface directly in \TeX{}. The user
        supplies axis labels, legend entries and the plot coordinates for one or
        more plots and \PGFPlots{} applies axis scaling, computes any logarithms
        and axis ticks and draws the plots. It supports line plots, scatter
        plots, piecewise constant plots, bar plots, area plots, mesh and surface
        plots, patch plots, contour plots, quiver plots, histogram plots, box
        plots, polar axes, ternary diagrams, smith charts and some more. It is
        based on Till Tantau's package \PGF{}/\Tikz{}.
    \end{minipage}

}%

\expandafter\date\expandafter{\@date\\[2cm]
    \abstractsmuggle
}%
\makeatother

%\includeonly{pgfplots.reference}


\begin{document}

\def\plotcoords{%
\addplot coordinates {
(5,8.312e-02)    (17,2.547e-02)   (49,7.407e-03)
(129,2.102e-03)  (321,5.874e-04)  (769,1.623e-04)
(1793,4.442e-05) (4097,1.207e-05) (9217,3.261e-06)
};

\addplot coordinates{
(7,8.472e-02)    (31,3.044e-02)    (111,1.022e-02)
(351,3.303e-03)  (1023,1.039e-03)  (2815,3.196e-04)
(7423,9.658e-05) (18943,2.873e-05) (47103,8.437e-06)
};

\addplot coordinates{
(9,7.881e-02)     (49,3.243e-02)    (209,1.232e-02)
(769,4.454e-03)   (2561,1.551e-03)  (7937,5.236e-04)
(23297,1.723e-04) (65537,5.545e-05) (178177,1.751e-05)
};

\addplot coordinates{
(11,6.887e-02)    (71,3.177e-02)     (351,1.341e-02)
(1471,5.334e-03)  (5503,2.027e-03)   (18943,7.415e-04)
(61183,2.628e-04) (187903,9.063e-05) (553983,3.053e-05)
};

\addplot coordinates{
(13,5.755e-02)     (97,2.925e-02)     (545,1.351e-02)
(2561,5.842e-03)   (10625,2.397e-03)  (40193,9.414e-04)
(141569,3.564e-04) (471041,1.308e-04)
(1496065,4.670e-05)
};
}%


\bookmaketitle
\tableofcontents
\include{pgfplots.title_abstract_intro}
\include{pgfplots.preliminaries}
\include{pgfplots.intro}
\include{pgfplots.reference}
\include{pgfplots.libs}
\include{pgfplots.resources}
\include{pgfplots.importexport}
\include{pgfplots.basic.reference}

\printindex

\bibliographystyle{abbrv} %gerapali} %gerabbrv} %gerunsrt.bst} %gerabbrv}% gerplain}
\nocite{pgfplotstable}
\nocite{programmingnotes}
\bibliography{pgfplots}
\end{document}

% -----------------------------------------------------------------------------
% For Stefan Pinnow as reminder on what to look for when editing the manual
% -----------------------------------------------------------------------------
% There should be no line breaks in the following environments
% - |...|
% - \declareandlabel{...}
% - \verbpdfref{...}
% If "MakeTikzPictures" isn't running through check one of the externalized
% LOG files what is the cause of that.
% -----------------------------------------------------------------------------

\include{pgfplots.basic.reference}

\printindex

\bibliographystyle{abbrv} %gerapali} %gerabbrv} %gerunsrt.bst} %gerabbrv}% gerplain}
\nocite{pgfplotstable}
\nocite{programmingnotes}
\bibliography{pgfplots}
\end{document}

% -----------------------------------------------------------------------------
% For Stefan Pinnow as reminder on what to look for when editing the manual
% -----------------------------------------------------------------------------
% There should be no line breaks in the following environments
% - |...|
% - \declareandlabel{...}
% - \verbpdfref{...}
% If "MakeTikzPictures" isn't running through check one of the externalized
% LOG files what is the cause of that.
% -----------------------------------------------------------------------------


%%%%%%%%%%%%%%%%%%%%%%%%%%%%%%%%%%%%%%%%%%%%%%%%%%%%%%%%%%%%%%%%%%%%%%%%%%%%%
%
% Package pgfplots.sty documentation.
%
% Copyright 2007/2008 by Christian Feuersaenger.
%
% This program is free software: you can redistribute it and/or modify
% it under the terms of the GNU General Public License as published by
% the Free Software Foundation, either version 3 of the License, or
% (at your option) any later version.
%
% This program is distributed in the hope that it will be useful,
% but WITHOUT ANY WARRANTY; without even the implied warranty of
% MERCHANTABILITY or FITNESS FOR A PARTICULAR PURPOSE.  See the
% GNU General Public License for more details.
%
% You should have received a copy of the GNU General Public License
% along with this program.  If not, see <http://www.gnu.org/licenses/>.
%
%
%%%%%%%%%%%%%%%%%%%%%%%%%%%%%%%%%%%%%%%%%%%%%%%%%%%%%%%%%%%%%%%%%%%%%%%%%%%%%
% SEE pgfplots-macros.tex as well!
%\pdfminorversion=5 % to allow compression
%\pdfobjcompresslevel=2
\documentclass[a4paper,openany]{book}

\let\bookmaketitle=\maketitle

% -----------------------------------
% this here is from ltxdoc:
\usepackage{doc}[=v2]
\AtBeginDocument{\MakeShortVerb{\|}}
%\setlength{\textwidth}{355pt}
%\addtolength\marginparwidth{30pt}
%\addtolength\oddsidemargin{20pt}
%\addtolength\evensidemargin{20pt}
\def\cmd#1{\cs{\expandafter\cmd@to@cs\string#1}}
\def\cmd@to@cs#1#2{\char\number`#2\relax}
\DeclareRobustCommand\cs[1]{\texttt{\char`\\#1}}
\providecommand\marg[1]{%
  {\ttfamily\char`\{}\meta{#1}{\ttfamily\char`\}}}
\providecommand\oarg[1]{%
  {\ttfamily[}\meta{#1}{\ttfamily]}}
\providecommand\parg[1]{%
  {\ttfamily(}\meta{#1}{\ttfamily)}}
\raggedbottom

% -----------------------------------
\input{pgfplots.preamble.tex}

\makeatletter
% I want a two-column index just as in pgfmanual styles. This here
% was the best way to get one:
\def\index@prologue{\section*{Index}\addcontentsline{toc}{chapter}{Index}
}
\makeatother

%\RequirePackage[german,english,francais]{babel}

\def\matlabcolormaptext{This colormap is similar to one shipped with Matlab$^\text{\textregistered}$ under a similar name.}

\IfFileExists{tikzlibraryspy.code.tex}{%
\usetikzlibrary{spy}
}{%
    \message{ERROR: tikz SPY library NOT available. The manual will only compile partially.^^J}%
}%

\usepackage{xparse}% for colorbrewer manual

\usetikzlibrary{
    decorations.markings,
    decorations.footprints,
    shapes.arrows,
    matrix,
    positioning,
}

\usepgfplotslibrary{
    fillbetween,
    ternary,
    smithchart,
    patchplots,
    polar,
    colormaps,
    colorbrewer,
    colortol,           % docu not ready yet
}
\pgfqkeys{/codeexample}{%
    every codeexample/.append style={
        /pgfplots/every ternary axis/.append style={
            /pgfplots/legend style={fill=graphicbackground},
        }
    },
    tabsize=4,
}

\pgfplotsmanualenableexternalizationofexpensive

%\usetikzlibrary{external}
%\tikzexternalize[prefix=figures/]{pgfplots}

\title{%
    Manual for Package \PGFPlots{}\\
    {\small 2D/3D Plots in \LaTeX{}, Version \pgfplotsversion}\\
    {\small\url{https://github.com/pgf-tikz/pgfplots}}
    %\\{\small Attention: you are using an unstable development version.}
}

\makeatletter
\long\def\abstractsmuggle{%
    \centering
    \textbf{Abstract}\\[0.5cm]

    \begin{minipage}{12cm}
        \PGFPlots{} draws high-quality function plots in normal or logarithmic
        scaling with a user-friendly interface directly in \TeX{}. The user
        supplies axis labels, legend entries and the plot coordinates for one or
        more plots and \PGFPlots{} applies axis scaling, computes any logarithms
        and axis ticks and draws the plots. It supports line plots, scatter
        plots, piecewise constant plots, bar plots, area plots, mesh and surface
        plots, patch plots, contour plots, quiver plots, histogram plots, box
        plots, polar axes, ternary diagrams, smith charts and some more. It is
        based on Till Tantau's package \PGF{}/\Tikz{}.
    \end{minipage}

}%

\expandafter\date\expandafter{\@date\\[2cm]
    \abstractsmuggle
}%
\makeatother

%\includeonly{pgfplots.reference}


\begin{document}

\def\plotcoords{%
\addplot coordinates {
(5,8.312e-02)    (17,2.547e-02)   (49,7.407e-03)
(129,2.102e-03)  (321,5.874e-04)  (769,1.623e-04)
(1793,4.442e-05) (4097,1.207e-05) (9217,3.261e-06)
};

\addplot coordinates{
(7,8.472e-02)    (31,3.044e-02)    (111,1.022e-02)
(351,3.303e-03)  (1023,1.039e-03)  (2815,3.196e-04)
(7423,9.658e-05) (18943,2.873e-05) (47103,8.437e-06)
};

\addplot coordinates{
(9,7.881e-02)     (49,3.243e-02)    (209,1.232e-02)
(769,4.454e-03)   (2561,1.551e-03)  (7937,5.236e-04)
(23297,1.723e-04) (65537,5.545e-05) (178177,1.751e-05)
};

\addplot coordinates{
(11,6.887e-02)    (71,3.177e-02)     (351,1.341e-02)
(1471,5.334e-03)  (5503,2.027e-03)   (18943,7.415e-04)
(61183,2.628e-04) (187903,9.063e-05) (553983,3.053e-05)
};

\addplot coordinates{
(13,5.755e-02)     (97,2.925e-02)     (545,1.351e-02)
(2561,5.842e-03)   (10625,2.397e-03)  (40193,9.414e-04)
(141569,3.564e-04) (471041,1.308e-04)
(1496065,4.670e-05)
};
}%


\bookmaketitle
\tableofcontents

%%%%%%%%%%%%%%%%%%%%%%%%%%%%%%%%%%%%%%%%%%%%%%%%%%%%%%%%%%%%%%%%%%%%%%%%%%%%%
%
% Package pgfplots.sty documentation.
%
% Copyright 2007/2008 by Christian Feuersaenger.
%
% This program is free software: you can redistribute it and/or modify
% it under the terms of the GNU General Public License as published by
% the Free Software Foundation, either version 3 of the License, or
% (at your option) any later version.
%
% This program is distributed in the hope that it will be useful,
% but WITHOUT ANY WARRANTY; without even the implied warranty of
% MERCHANTABILITY or FITNESS FOR A PARTICULAR PURPOSE.  See the
% GNU General Public License for more details.
%
% You should have received a copy of the GNU General Public License
% along with this program.  If not, see <http://www.gnu.org/licenses/>.
%
%
%%%%%%%%%%%%%%%%%%%%%%%%%%%%%%%%%%%%%%%%%%%%%%%%%%%%%%%%%%%%%%%%%%%%%%%%%%%%%
% SEE pgfplots-macros.tex as well!
%\pdfminorversion=5 % to allow compression
%\pdfobjcompresslevel=2
\documentclass[a4paper,openany]{book}

\let\bookmaketitle=\maketitle

% -----------------------------------
% this here is from ltxdoc:
\usepackage{doc}[=v2]
\AtBeginDocument{\MakeShortVerb{\|}}
%\setlength{\textwidth}{355pt}
%\addtolength\marginparwidth{30pt}
%\addtolength\oddsidemargin{20pt}
%\addtolength\evensidemargin{20pt}
\def\cmd#1{\cs{\expandafter\cmd@to@cs\string#1}}
\def\cmd@to@cs#1#2{\char\number`#2\relax}
\DeclareRobustCommand\cs[1]{\texttt{\char`\\#1}}
\providecommand\marg[1]{%
  {\ttfamily\char`\{}\meta{#1}{\ttfamily\char`\}}}
\providecommand\oarg[1]{%
  {\ttfamily[}\meta{#1}{\ttfamily]}}
\providecommand\parg[1]{%
  {\ttfamily(}\meta{#1}{\ttfamily)}}
\raggedbottom

% -----------------------------------
\input{pgfplots.preamble.tex}

\makeatletter
% I want a two-column index just as in pgfmanual styles. This here
% was the best way to get one:
\def\index@prologue{\section*{Index}\addcontentsline{toc}{chapter}{Index}
}
\makeatother

%\RequirePackage[german,english,francais]{babel}

\def\matlabcolormaptext{This colormap is similar to one shipped with Matlab$^\text{\textregistered}$ under a similar name.}

\IfFileExists{tikzlibraryspy.code.tex}{%
\usetikzlibrary{spy}
}{%
    \message{ERROR: tikz SPY library NOT available. The manual will only compile partially.^^J}%
}%

\usepackage{xparse}% for colorbrewer manual

\usetikzlibrary{
    decorations.markings,
    decorations.footprints,
    shapes.arrows,
    matrix,
    positioning,
}

\usepgfplotslibrary{
    fillbetween,
    ternary,
    smithchart,
    patchplots,
    polar,
    colormaps,
    colorbrewer,
    colortol,           % docu not ready yet
}
\pgfqkeys{/codeexample}{%
    every codeexample/.append style={
        /pgfplots/every ternary axis/.append style={
            /pgfplots/legend style={fill=graphicbackground},
        }
    },
    tabsize=4,
}

\pgfplotsmanualenableexternalizationofexpensive

%\usetikzlibrary{external}
%\tikzexternalize[prefix=figures/]{pgfplots}

\title{%
    Manual for Package \PGFPlots{}\\
    {\small 2D/3D Plots in \LaTeX{}, Version \pgfplotsversion}\\
    {\small\url{https://github.com/pgf-tikz/pgfplots}}
    %\\{\small Attention: you are using an unstable development version.}
}

\makeatletter
\long\def\abstractsmuggle{%
    \centering
    \textbf{Abstract}\\[0.5cm]

    \begin{minipage}{12cm}
        \PGFPlots{} draws high-quality function plots in normal or logarithmic
        scaling with a user-friendly interface directly in \TeX{}. The user
        supplies axis labels, legend entries and the plot coordinates for one or
        more plots and \PGFPlots{} applies axis scaling, computes any logarithms
        and axis ticks and draws the plots. It supports line plots, scatter
        plots, piecewise constant plots, bar plots, area plots, mesh and surface
        plots, patch plots, contour plots, quiver plots, histogram plots, box
        plots, polar axes, ternary diagrams, smith charts and some more. It is
        based on Till Tantau's package \PGF{}/\Tikz{}.
    \end{minipage}

}%

\expandafter\date\expandafter{\@date\\[2cm]
    \abstractsmuggle
}%
\makeatother

%\includeonly{pgfplots.reference}


\begin{document}

\def\plotcoords{%
\addplot coordinates {
(5,8.312e-02)    (17,2.547e-02)   (49,7.407e-03)
(129,2.102e-03)  (321,5.874e-04)  (769,1.623e-04)
(1793,4.442e-05) (4097,1.207e-05) (9217,3.261e-06)
};

\addplot coordinates{
(7,8.472e-02)    (31,3.044e-02)    (111,1.022e-02)
(351,3.303e-03)  (1023,1.039e-03)  (2815,3.196e-04)
(7423,9.658e-05) (18943,2.873e-05) (47103,8.437e-06)
};

\addplot coordinates{
(9,7.881e-02)     (49,3.243e-02)    (209,1.232e-02)
(769,4.454e-03)   (2561,1.551e-03)  (7937,5.236e-04)
(23297,1.723e-04) (65537,5.545e-05) (178177,1.751e-05)
};

\addplot coordinates{
(11,6.887e-02)    (71,3.177e-02)     (351,1.341e-02)
(1471,5.334e-03)  (5503,2.027e-03)   (18943,7.415e-04)
(61183,2.628e-04) (187903,9.063e-05) (553983,3.053e-05)
};

\addplot coordinates{
(13,5.755e-02)     (97,2.925e-02)     (545,1.351e-02)
(2561,5.842e-03)   (10625,2.397e-03)  (40193,9.414e-04)
(141569,3.564e-04) (471041,1.308e-04)
(1496065,4.670e-05)
};
}%


\bookmaketitle
\tableofcontents
\include{pgfplots.title_abstract_intro}
\include{pgfplots.preliminaries}
\include{pgfplots.intro}
\include{pgfplots.reference}
\include{pgfplots.libs}
\include{pgfplots.resources}
\include{pgfplots.importexport}
\include{pgfplots.basic.reference}

\printindex

\bibliographystyle{abbrv} %gerapali} %gerabbrv} %gerunsrt.bst} %gerabbrv}% gerplain}
\nocite{pgfplotstable}
\nocite{programmingnotes}
\bibliography{pgfplots}
\end{document}

% -----------------------------------------------------------------------------
% For Stefan Pinnow as reminder on what to look for when editing the manual
% -----------------------------------------------------------------------------
% There should be no line breaks in the following environments
% - |...|
% - \declareandlabel{...}
% - \verbpdfref{...}
% If "MakeTikzPictures" isn't running through check one of the externalized
% LOG files what is the cause of that.
% -----------------------------------------------------------------------------


%%%%%%%%%%%%%%%%%%%%%%%%%%%%%%%%%%%%%%%%%%%%%%%%%%%%%%%%%%%%%%%%%%%%%%%%%%%%%
%
% Package pgfplots.sty documentation.
%
% Copyright 2007/2008 by Christian Feuersaenger.
%
% This program is free software: you can redistribute it and/or modify
% it under the terms of the GNU General Public License as published by
% the Free Software Foundation, either version 3 of the License, or
% (at your option) any later version.
%
% This program is distributed in the hope that it will be useful,
% but WITHOUT ANY WARRANTY; without even the implied warranty of
% MERCHANTABILITY or FITNESS FOR A PARTICULAR PURPOSE.  See the
% GNU General Public License for more details.
%
% You should have received a copy of the GNU General Public License
% along with this program.  If not, see <http://www.gnu.org/licenses/>.
%
%
%%%%%%%%%%%%%%%%%%%%%%%%%%%%%%%%%%%%%%%%%%%%%%%%%%%%%%%%%%%%%%%%%%%%%%%%%%%%%
% SEE pgfplots-macros.tex as well!
%\pdfminorversion=5 % to allow compression
%\pdfobjcompresslevel=2
\documentclass[a4paper,openany]{book}

\let\bookmaketitle=\maketitle

% -----------------------------------
% this here is from ltxdoc:
\usepackage{doc}[=v2]
\AtBeginDocument{\MakeShortVerb{\|}}
%\setlength{\textwidth}{355pt}
%\addtolength\marginparwidth{30pt}
%\addtolength\oddsidemargin{20pt}
%\addtolength\evensidemargin{20pt}
\def\cmd#1{\cs{\expandafter\cmd@to@cs\string#1}}
\def\cmd@to@cs#1#2{\char\number`#2\relax}
\DeclareRobustCommand\cs[1]{\texttt{\char`\\#1}}
\providecommand\marg[1]{%
  {\ttfamily\char`\{}\meta{#1}{\ttfamily\char`\}}}
\providecommand\oarg[1]{%
  {\ttfamily[}\meta{#1}{\ttfamily]}}
\providecommand\parg[1]{%
  {\ttfamily(}\meta{#1}{\ttfamily)}}
\raggedbottom

% -----------------------------------
\input{pgfplots.preamble.tex}

\makeatletter
% I want a two-column index just as in pgfmanual styles. This here
% was the best way to get one:
\def\index@prologue{\section*{Index}\addcontentsline{toc}{chapter}{Index}
}
\makeatother

%\RequirePackage[german,english,francais]{babel}

\def\matlabcolormaptext{This colormap is similar to one shipped with Matlab$^\text{\textregistered}$ under a similar name.}

\IfFileExists{tikzlibraryspy.code.tex}{%
\usetikzlibrary{spy}
}{%
    \message{ERROR: tikz SPY library NOT available. The manual will only compile partially.^^J}%
}%

\usepackage{xparse}% for colorbrewer manual

\usetikzlibrary{
    decorations.markings,
    decorations.footprints,
    shapes.arrows,
    matrix,
    positioning,
}

\usepgfplotslibrary{
    fillbetween,
    ternary,
    smithchart,
    patchplots,
    polar,
    colormaps,
    colorbrewer,
    colortol,           % docu not ready yet
}
\pgfqkeys{/codeexample}{%
    every codeexample/.append style={
        /pgfplots/every ternary axis/.append style={
            /pgfplots/legend style={fill=graphicbackground},
        }
    },
    tabsize=4,
}

\pgfplotsmanualenableexternalizationofexpensive

%\usetikzlibrary{external}
%\tikzexternalize[prefix=figures/]{pgfplots}

\title{%
    Manual for Package \PGFPlots{}\\
    {\small 2D/3D Plots in \LaTeX{}, Version \pgfplotsversion}\\
    {\small\url{https://github.com/pgf-tikz/pgfplots}}
    %\\{\small Attention: you are using an unstable development version.}
}

\makeatletter
\long\def\abstractsmuggle{%
    \centering
    \textbf{Abstract}\\[0.5cm]

    \begin{minipage}{12cm}
        \PGFPlots{} draws high-quality function plots in normal or logarithmic
        scaling with a user-friendly interface directly in \TeX{}. The user
        supplies axis labels, legend entries and the plot coordinates for one or
        more plots and \PGFPlots{} applies axis scaling, computes any logarithms
        and axis ticks and draws the plots. It supports line plots, scatter
        plots, piecewise constant plots, bar plots, area plots, mesh and surface
        plots, patch plots, contour plots, quiver plots, histogram plots, box
        plots, polar axes, ternary diagrams, smith charts and some more. It is
        based on Till Tantau's package \PGF{}/\Tikz{}.
    \end{minipage}

}%

\expandafter\date\expandafter{\@date\\[2cm]
    \abstractsmuggle
}%
\makeatother

%\includeonly{pgfplots.reference}


\begin{document}

\def\plotcoords{%
\addplot coordinates {
(5,8.312e-02)    (17,2.547e-02)   (49,7.407e-03)
(129,2.102e-03)  (321,5.874e-04)  (769,1.623e-04)
(1793,4.442e-05) (4097,1.207e-05) (9217,3.261e-06)
};

\addplot coordinates{
(7,8.472e-02)    (31,3.044e-02)    (111,1.022e-02)
(351,3.303e-03)  (1023,1.039e-03)  (2815,3.196e-04)
(7423,9.658e-05) (18943,2.873e-05) (47103,8.437e-06)
};

\addplot coordinates{
(9,7.881e-02)     (49,3.243e-02)    (209,1.232e-02)
(769,4.454e-03)   (2561,1.551e-03)  (7937,5.236e-04)
(23297,1.723e-04) (65537,5.545e-05) (178177,1.751e-05)
};

\addplot coordinates{
(11,6.887e-02)    (71,3.177e-02)     (351,1.341e-02)
(1471,5.334e-03)  (5503,2.027e-03)   (18943,7.415e-04)
(61183,2.628e-04) (187903,9.063e-05) (553983,3.053e-05)
};

\addplot coordinates{
(13,5.755e-02)     (97,2.925e-02)     (545,1.351e-02)
(2561,5.842e-03)   (10625,2.397e-03)  (40193,9.414e-04)
(141569,3.564e-04) (471041,1.308e-04)
(1496065,4.670e-05)
};
}%


\bookmaketitle
\tableofcontents
\include{pgfplots.title_abstract_intro}
\include{pgfplots.preliminaries}
\include{pgfplots.intro}
\include{pgfplots.reference}
\include{pgfplots.libs}
\include{pgfplots.resources}
\include{pgfplots.importexport}
\include{pgfplots.basic.reference}

\printindex

\bibliographystyle{abbrv} %gerapali} %gerabbrv} %gerunsrt.bst} %gerabbrv}% gerplain}
\nocite{pgfplotstable}
\nocite{programmingnotes}
\bibliography{pgfplots}
\end{document}

% -----------------------------------------------------------------------------
% For Stefan Pinnow as reminder on what to look for when editing the manual
% -----------------------------------------------------------------------------
% There should be no line breaks in the following environments
% - |...|
% - \declareandlabel{...}
% - \verbpdfref{...}
% If "MakeTikzPictures" isn't running through check one of the externalized
% LOG files what is the cause of that.
% -----------------------------------------------------------------------------


%%%%%%%%%%%%%%%%%%%%%%%%%%%%%%%%%%%%%%%%%%%%%%%%%%%%%%%%%%%%%%%%%%%%%%%%%%%%%
%
% Package pgfplots.sty documentation.
%
% Copyright 2007/2008 by Christian Feuersaenger.
%
% This program is free software: you can redistribute it and/or modify
% it under the terms of the GNU General Public License as published by
% the Free Software Foundation, either version 3 of the License, or
% (at your option) any later version.
%
% This program is distributed in the hope that it will be useful,
% but WITHOUT ANY WARRANTY; without even the implied warranty of
% MERCHANTABILITY or FITNESS FOR A PARTICULAR PURPOSE.  See the
% GNU General Public License for more details.
%
% You should have received a copy of the GNU General Public License
% along with this program.  If not, see <http://www.gnu.org/licenses/>.
%
%
%%%%%%%%%%%%%%%%%%%%%%%%%%%%%%%%%%%%%%%%%%%%%%%%%%%%%%%%%%%%%%%%%%%%%%%%%%%%%
% SEE pgfplots-macros.tex as well!
%\pdfminorversion=5 % to allow compression
%\pdfobjcompresslevel=2
\documentclass[a4paper,openany]{book}

\let\bookmaketitle=\maketitle

% -----------------------------------
% this here is from ltxdoc:
\usepackage{doc}[=v2]
\AtBeginDocument{\MakeShortVerb{\|}}
%\setlength{\textwidth}{355pt}
%\addtolength\marginparwidth{30pt}
%\addtolength\oddsidemargin{20pt}
%\addtolength\evensidemargin{20pt}
\def\cmd#1{\cs{\expandafter\cmd@to@cs\string#1}}
\def\cmd@to@cs#1#2{\char\number`#2\relax}
\DeclareRobustCommand\cs[1]{\texttt{\char`\\#1}}
\providecommand\marg[1]{%
  {\ttfamily\char`\{}\meta{#1}{\ttfamily\char`\}}}
\providecommand\oarg[1]{%
  {\ttfamily[}\meta{#1}{\ttfamily]}}
\providecommand\parg[1]{%
  {\ttfamily(}\meta{#1}{\ttfamily)}}
\raggedbottom

% -----------------------------------
\input{pgfplots.preamble.tex}

\makeatletter
% I want a two-column index just as in pgfmanual styles. This here
% was the best way to get one:
\def\index@prologue{\section*{Index}\addcontentsline{toc}{chapter}{Index}
}
\makeatother

%\RequirePackage[german,english,francais]{babel}

\def\matlabcolormaptext{This colormap is similar to one shipped with Matlab$^\text{\textregistered}$ under a similar name.}

\IfFileExists{tikzlibraryspy.code.tex}{%
\usetikzlibrary{spy}
}{%
    \message{ERROR: tikz SPY library NOT available. The manual will only compile partially.^^J}%
}%

\usepackage{xparse}% for colorbrewer manual

\usetikzlibrary{
    decorations.markings,
    decorations.footprints,
    shapes.arrows,
    matrix,
    positioning,
}

\usepgfplotslibrary{
    fillbetween,
    ternary,
    smithchart,
    patchplots,
    polar,
    colormaps,
    colorbrewer,
    colortol,           % docu not ready yet
}
\pgfqkeys{/codeexample}{%
    every codeexample/.append style={
        /pgfplots/every ternary axis/.append style={
            /pgfplots/legend style={fill=graphicbackground},
        }
    },
    tabsize=4,
}

\pgfplotsmanualenableexternalizationofexpensive

%\usetikzlibrary{external}
%\tikzexternalize[prefix=figures/]{pgfplots}

\title{%
    Manual for Package \PGFPlots{}\\
    {\small 2D/3D Plots in \LaTeX{}, Version \pgfplotsversion}\\
    {\small\url{https://github.com/pgf-tikz/pgfplots}}
    %\\{\small Attention: you are using an unstable development version.}
}

\makeatletter
\long\def\abstractsmuggle{%
    \centering
    \textbf{Abstract}\\[0.5cm]

    \begin{minipage}{12cm}
        \PGFPlots{} draws high-quality function plots in normal or logarithmic
        scaling with a user-friendly interface directly in \TeX{}. The user
        supplies axis labels, legend entries and the plot coordinates for one or
        more plots and \PGFPlots{} applies axis scaling, computes any logarithms
        and axis ticks and draws the plots. It supports line plots, scatter
        plots, piecewise constant plots, bar plots, area plots, mesh and surface
        plots, patch plots, contour plots, quiver plots, histogram plots, box
        plots, polar axes, ternary diagrams, smith charts and some more. It is
        based on Till Tantau's package \PGF{}/\Tikz{}.
    \end{minipage}

}%

\expandafter\date\expandafter{\@date\\[2cm]
    \abstractsmuggle
}%
\makeatother

%\includeonly{pgfplots.reference}


\begin{document}

\def\plotcoords{%
\addplot coordinates {
(5,8.312e-02)    (17,2.547e-02)   (49,7.407e-03)
(129,2.102e-03)  (321,5.874e-04)  (769,1.623e-04)
(1793,4.442e-05) (4097,1.207e-05) (9217,3.261e-06)
};

\addplot coordinates{
(7,8.472e-02)    (31,3.044e-02)    (111,1.022e-02)
(351,3.303e-03)  (1023,1.039e-03)  (2815,3.196e-04)
(7423,9.658e-05) (18943,2.873e-05) (47103,8.437e-06)
};

\addplot coordinates{
(9,7.881e-02)     (49,3.243e-02)    (209,1.232e-02)
(769,4.454e-03)   (2561,1.551e-03)  (7937,5.236e-04)
(23297,1.723e-04) (65537,5.545e-05) (178177,1.751e-05)
};

\addplot coordinates{
(11,6.887e-02)    (71,3.177e-02)     (351,1.341e-02)
(1471,5.334e-03)  (5503,2.027e-03)   (18943,7.415e-04)
(61183,2.628e-04) (187903,9.063e-05) (553983,3.053e-05)
};

\addplot coordinates{
(13,5.755e-02)     (97,2.925e-02)     (545,1.351e-02)
(2561,5.842e-03)   (10625,2.397e-03)  (40193,9.414e-04)
(141569,3.564e-04) (471041,1.308e-04)
(1496065,4.670e-05)
};
}%


\bookmaketitle
\tableofcontents
\include{pgfplots.title_abstract_intro}
\include{pgfplots.preliminaries}
\include{pgfplots.intro}
\include{pgfplots.reference}
\include{pgfplots.libs}
\include{pgfplots.resources}
\include{pgfplots.importexport}
\include{pgfplots.basic.reference}

\printindex

\bibliographystyle{abbrv} %gerapali} %gerabbrv} %gerunsrt.bst} %gerabbrv}% gerplain}
\nocite{pgfplotstable}
\nocite{programmingnotes}
\bibliography{pgfplots}
\end{document}

% -----------------------------------------------------------------------------
% For Stefan Pinnow as reminder on what to look for when editing the manual
% -----------------------------------------------------------------------------
% There should be no line breaks in the following environments
% - |...|
% - \declareandlabel{...}
% - \verbpdfref{...}
% If "MakeTikzPictures" isn't running through check one of the externalized
% LOG files what is the cause of that.
% -----------------------------------------------------------------------------


%%%%%%%%%%%%%%%%%%%%%%%%%%%%%%%%%%%%%%%%%%%%%%%%%%%%%%%%%%%%%%%%%%%%%%%%%%%%%
%
% Package pgfplots.sty documentation.
%
% Copyright 2007/2008 by Christian Feuersaenger.
%
% This program is free software: you can redistribute it and/or modify
% it under the terms of the GNU General Public License as published by
% the Free Software Foundation, either version 3 of the License, or
% (at your option) any later version.
%
% This program is distributed in the hope that it will be useful,
% but WITHOUT ANY WARRANTY; without even the implied warranty of
% MERCHANTABILITY or FITNESS FOR A PARTICULAR PURPOSE.  See the
% GNU General Public License for more details.
%
% You should have received a copy of the GNU General Public License
% along with this program.  If not, see <http://www.gnu.org/licenses/>.
%
%
%%%%%%%%%%%%%%%%%%%%%%%%%%%%%%%%%%%%%%%%%%%%%%%%%%%%%%%%%%%%%%%%%%%%%%%%%%%%%
% SEE pgfplots-macros.tex as well!
%\pdfminorversion=5 % to allow compression
%\pdfobjcompresslevel=2
\documentclass[a4paper,openany]{book}

\let\bookmaketitle=\maketitle

% -----------------------------------
% this here is from ltxdoc:
\usepackage{doc}[=v2]
\AtBeginDocument{\MakeShortVerb{\|}}
%\setlength{\textwidth}{355pt}
%\addtolength\marginparwidth{30pt}
%\addtolength\oddsidemargin{20pt}
%\addtolength\evensidemargin{20pt}
\def\cmd#1{\cs{\expandafter\cmd@to@cs\string#1}}
\def\cmd@to@cs#1#2{\char\number`#2\relax}
\DeclareRobustCommand\cs[1]{\texttt{\char`\\#1}}
\providecommand\marg[1]{%
  {\ttfamily\char`\{}\meta{#1}{\ttfamily\char`\}}}
\providecommand\oarg[1]{%
  {\ttfamily[}\meta{#1}{\ttfamily]}}
\providecommand\parg[1]{%
  {\ttfamily(}\meta{#1}{\ttfamily)}}
\raggedbottom

% -----------------------------------
\input{pgfplots.preamble.tex}

\makeatletter
% I want a two-column index just as in pgfmanual styles. This here
% was the best way to get one:
\def\index@prologue{\section*{Index}\addcontentsline{toc}{chapter}{Index}
}
\makeatother

%\RequirePackage[german,english,francais]{babel}

\def\matlabcolormaptext{This colormap is similar to one shipped with Matlab$^\text{\textregistered}$ under a similar name.}

\IfFileExists{tikzlibraryspy.code.tex}{%
\usetikzlibrary{spy}
}{%
    \message{ERROR: tikz SPY library NOT available. The manual will only compile partially.^^J}%
}%

\usepackage{xparse}% for colorbrewer manual

\usetikzlibrary{
    decorations.markings,
    decorations.footprints,
    shapes.arrows,
    matrix,
    positioning,
}

\usepgfplotslibrary{
    fillbetween,
    ternary,
    smithchart,
    patchplots,
    polar,
    colormaps,
    colorbrewer,
    colortol,           % docu not ready yet
}
\pgfqkeys{/codeexample}{%
    every codeexample/.append style={
        /pgfplots/every ternary axis/.append style={
            /pgfplots/legend style={fill=graphicbackground},
        }
    },
    tabsize=4,
}

\pgfplotsmanualenableexternalizationofexpensive

%\usetikzlibrary{external}
%\tikzexternalize[prefix=figures/]{pgfplots}

\title{%
    Manual for Package \PGFPlots{}\\
    {\small 2D/3D Plots in \LaTeX{}, Version \pgfplotsversion}\\
    {\small\url{https://github.com/pgf-tikz/pgfplots}}
    %\\{\small Attention: you are using an unstable development version.}
}

\makeatletter
\long\def\abstractsmuggle{%
    \centering
    \textbf{Abstract}\\[0.5cm]

    \begin{minipage}{12cm}
        \PGFPlots{} draws high-quality function plots in normal or logarithmic
        scaling with a user-friendly interface directly in \TeX{}. The user
        supplies axis labels, legend entries and the plot coordinates for one or
        more plots and \PGFPlots{} applies axis scaling, computes any logarithms
        and axis ticks and draws the plots. It supports line plots, scatter
        plots, piecewise constant plots, bar plots, area plots, mesh and surface
        plots, patch plots, contour plots, quiver plots, histogram plots, box
        plots, polar axes, ternary diagrams, smith charts and some more. It is
        based on Till Tantau's package \PGF{}/\Tikz{}.
    \end{minipage}

}%

\expandafter\date\expandafter{\@date\\[2cm]
    \abstractsmuggle
}%
\makeatother

%\includeonly{pgfplots.reference}


\begin{document}

\def\plotcoords{%
\addplot coordinates {
(5,8.312e-02)    (17,2.547e-02)   (49,7.407e-03)
(129,2.102e-03)  (321,5.874e-04)  (769,1.623e-04)
(1793,4.442e-05) (4097,1.207e-05) (9217,3.261e-06)
};

\addplot coordinates{
(7,8.472e-02)    (31,3.044e-02)    (111,1.022e-02)
(351,3.303e-03)  (1023,1.039e-03)  (2815,3.196e-04)
(7423,9.658e-05) (18943,2.873e-05) (47103,8.437e-06)
};

\addplot coordinates{
(9,7.881e-02)     (49,3.243e-02)    (209,1.232e-02)
(769,4.454e-03)   (2561,1.551e-03)  (7937,5.236e-04)
(23297,1.723e-04) (65537,5.545e-05) (178177,1.751e-05)
};

\addplot coordinates{
(11,6.887e-02)    (71,3.177e-02)     (351,1.341e-02)
(1471,5.334e-03)  (5503,2.027e-03)   (18943,7.415e-04)
(61183,2.628e-04) (187903,9.063e-05) (553983,3.053e-05)
};

\addplot coordinates{
(13,5.755e-02)     (97,2.925e-02)     (545,1.351e-02)
(2561,5.842e-03)   (10625,2.397e-03)  (40193,9.414e-04)
(141569,3.564e-04) (471041,1.308e-04)
(1496065,4.670e-05)
};
}%


\bookmaketitle
\tableofcontents
\include{pgfplots.title_abstract_intro}
\include{pgfplots.preliminaries}
\include{pgfplots.intro}
\include{pgfplots.reference}
\include{pgfplots.libs}
\include{pgfplots.resources}
\include{pgfplots.importexport}
\include{pgfplots.basic.reference}

\printindex

\bibliographystyle{abbrv} %gerapali} %gerabbrv} %gerunsrt.bst} %gerabbrv}% gerplain}
\nocite{pgfplotstable}
\nocite{programmingnotes}
\bibliography{pgfplots}
\end{document}

% -----------------------------------------------------------------------------
% For Stefan Pinnow as reminder on what to look for when editing the manual
% -----------------------------------------------------------------------------
% There should be no line breaks in the following environments
% - |...|
% - \declareandlabel{...}
% - \verbpdfref{...}
% If "MakeTikzPictures" isn't running through check one of the externalized
% LOG files what is the cause of that.
% -----------------------------------------------------------------------------


%%%%%%%%%%%%%%%%%%%%%%%%%%%%%%%%%%%%%%%%%%%%%%%%%%%%%%%%%%%%%%%%%%%%%%%%%%%%%
%
% Package pgfplots.sty documentation.
%
% Copyright 2007/2008 by Christian Feuersaenger.
%
% This program is free software: you can redistribute it and/or modify
% it under the terms of the GNU General Public License as published by
% the Free Software Foundation, either version 3 of the License, or
% (at your option) any later version.
%
% This program is distributed in the hope that it will be useful,
% but WITHOUT ANY WARRANTY; without even the implied warranty of
% MERCHANTABILITY or FITNESS FOR A PARTICULAR PURPOSE.  See the
% GNU General Public License for more details.
%
% You should have received a copy of the GNU General Public License
% along with this program.  If not, see <http://www.gnu.org/licenses/>.
%
%
%%%%%%%%%%%%%%%%%%%%%%%%%%%%%%%%%%%%%%%%%%%%%%%%%%%%%%%%%%%%%%%%%%%%%%%%%%%%%
% SEE pgfplots-macros.tex as well!
%\pdfminorversion=5 % to allow compression
%\pdfobjcompresslevel=2
\documentclass[a4paper,openany]{book}

\let\bookmaketitle=\maketitle

% -----------------------------------
% this here is from ltxdoc:
\usepackage{doc}[=v2]
\AtBeginDocument{\MakeShortVerb{\|}}
%\setlength{\textwidth}{355pt}
%\addtolength\marginparwidth{30pt}
%\addtolength\oddsidemargin{20pt}
%\addtolength\evensidemargin{20pt}
\def\cmd#1{\cs{\expandafter\cmd@to@cs\string#1}}
\def\cmd@to@cs#1#2{\char\number`#2\relax}
\DeclareRobustCommand\cs[1]{\texttt{\char`\\#1}}
\providecommand\marg[1]{%
  {\ttfamily\char`\{}\meta{#1}{\ttfamily\char`\}}}
\providecommand\oarg[1]{%
  {\ttfamily[}\meta{#1}{\ttfamily]}}
\providecommand\parg[1]{%
  {\ttfamily(}\meta{#1}{\ttfamily)}}
\raggedbottom

% -----------------------------------
\input{pgfplots.preamble.tex}

\makeatletter
% I want a two-column index just as in pgfmanual styles. This here
% was the best way to get one:
\def\index@prologue{\section*{Index}\addcontentsline{toc}{chapter}{Index}
}
\makeatother

%\RequirePackage[german,english,francais]{babel}

\def\matlabcolormaptext{This colormap is similar to one shipped with Matlab$^\text{\textregistered}$ under a similar name.}

\IfFileExists{tikzlibraryspy.code.tex}{%
\usetikzlibrary{spy}
}{%
    \message{ERROR: tikz SPY library NOT available. The manual will only compile partially.^^J}%
}%

\usepackage{xparse}% for colorbrewer manual

\usetikzlibrary{
    decorations.markings,
    decorations.footprints,
    shapes.arrows,
    matrix,
    positioning,
}

\usepgfplotslibrary{
    fillbetween,
    ternary,
    smithchart,
    patchplots,
    polar,
    colormaps,
    colorbrewer,
    colortol,           % docu not ready yet
}
\pgfqkeys{/codeexample}{%
    every codeexample/.append style={
        /pgfplots/every ternary axis/.append style={
            /pgfplots/legend style={fill=graphicbackground},
        }
    },
    tabsize=4,
}

\pgfplotsmanualenableexternalizationofexpensive

%\usetikzlibrary{external}
%\tikzexternalize[prefix=figures/]{pgfplots}

\title{%
    Manual for Package \PGFPlots{}\\
    {\small 2D/3D Plots in \LaTeX{}, Version \pgfplotsversion}\\
    {\small\url{https://github.com/pgf-tikz/pgfplots}}
    %\\{\small Attention: you are using an unstable development version.}
}

\makeatletter
\long\def\abstractsmuggle{%
    \centering
    \textbf{Abstract}\\[0.5cm]

    \begin{minipage}{12cm}
        \PGFPlots{} draws high-quality function plots in normal or logarithmic
        scaling with a user-friendly interface directly in \TeX{}. The user
        supplies axis labels, legend entries and the plot coordinates for one or
        more plots and \PGFPlots{} applies axis scaling, computes any logarithms
        and axis ticks and draws the plots. It supports line plots, scatter
        plots, piecewise constant plots, bar plots, area plots, mesh and surface
        plots, patch plots, contour plots, quiver plots, histogram plots, box
        plots, polar axes, ternary diagrams, smith charts and some more. It is
        based on Till Tantau's package \PGF{}/\Tikz{}.
    \end{minipage}

}%

\expandafter\date\expandafter{\@date\\[2cm]
    \abstractsmuggle
}%
\makeatother

%\includeonly{pgfplots.reference}


\begin{document}

\def\plotcoords{%
\addplot coordinates {
(5,8.312e-02)    (17,2.547e-02)   (49,7.407e-03)
(129,2.102e-03)  (321,5.874e-04)  (769,1.623e-04)
(1793,4.442e-05) (4097,1.207e-05) (9217,3.261e-06)
};

\addplot coordinates{
(7,8.472e-02)    (31,3.044e-02)    (111,1.022e-02)
(351,3.303e-03)  (1023,1.039e-03)  (2815,3.196e-04)
(7423,9.658e-05) (18943,2.873e-05) (47103,8.437e-06)
};

\addplot coordinates{
(9,7.881e-02)     (49,3.243e-02)    (209,1.232e-02)
(769,4.454e-03)   (2561,1.551e-03)  (7937,5.236e-04)
(23297,1.723e-04) (65537,5.545e-05) (178177,1.751e-05)
};

\addplot coordinates{
(11,6.887e-02)    (71,3.177e-02)     (351,1.341e-02)
(1471,5.334e-03)  (5503,2.027e-03)   (18943,7.415e-04)
(61183,2.628e-04) (187903,9.063e-05) (553983,3.053e-05)
};

\addplot coordinates{
(13,5.755e-02)     (97,2.925e-02)     (545,1.351e-02)
(2561,5.842e-03)   (10625,2.397e-03)  (40193,9.414e-04)
(141569,3.564e-04) (471041,1.308e-04)
(1496065,4.670e-05)
};
}%


\bookmaketitle
\tableofcontents
\include{pgfplots.title_abstract_intro}
\include{pgfplots.preliminaries}
\include{pgfplots.intro}
\include{pgfplots.reference}
\include{pgfplots.libs}
\include{pgfplots.resources}
\include{pgfplots.importexport}
\include{pgfplots.basic.reference}

\printindex

\bibliographystyle{abbrv} %gerapali} %gerabbrv} %gerunsrt.bst} %gerabbrv}% gerplain}
\nocite{pgfplotstable}
\nocite{programmingnotes}
\bibliography{pgfplots}
\end{document}

% -----------------------------------------------------------------------------
% For Stefan Pinnow as reminder on what to look for when editing the manual
% -----------------------------------------------------------------------------
% There should be no line breaks in the following environments
% - |...|
% - \declareandlabel{...}
% - \verbpdfref{...}
% If "MakeTikzPictures" isn't running through check one of the externalized
% LOG files what is the cause of that.
% -----------------------------------------------------------------------------


%%%%%%%%%%%%%%%%%%%%%%%%%%%%%%%%%%%%%%%%%%%%%%%%%%%%%%%%%%%%%%%%%%%%%%%%%%%%%
%
% Package pgfplots.sty documentation.
%
% Copyright 2007/2008 by Christian Feuersaenger.
%
% This program is free software: you can redistribute it and/or modify
% it under the terms of the GNU General Public License as published by
% the Free Software Foundation, either version 3 of the License, or
% (at your option) any later version.
%
% This program is distributed in the hope that it will be useful,
% but WITHOUT ANY WARRANTY; without even the implied warranty of
% MERCHANTABILITY or FITNESS FOR A PARTICULAR PURPOSE.  See the
% GNU General Public License for more details.
%
% You should have received a copy of the GNU General Public License
% along with this program.  If not, see <http://www.gnu.org/licenses/>.
%
%
%%%%%%%%%%%%%%%%%%%%%%%%%%%%%%%%%%%%%%%%%%%%%%%%%%%%%%%%%%%%%%%%%%%%%%%%%%%%%
% SEE pgfplots-macros.tex as well!
%\pdfminorversion=5 % to allow compression
%\pdfobjcompresslevel=2
\documentclass[a4paper,openany]{book}

\let\bookmaketitle=\maketitle

% -----------------------------------
% this here is from ltxdoc:
\usepackage{doc}[=v2]
\AtBeginDocument{\MakeShortVerb{\|}}
%\setlength{\textwidth}{355pt}
%\addtolength\marginparwidth{30pt}
%\addtolength\oddsidemargin{20pt}
%\addtolength\evensidemargin{20pt}
\def\cmd#1{\cs{\expandafter\cmd@to@cs\string#1}}
\def\cmd@to@cs#1#2{\char\number`#2\relax}
\DeclareRobustCommand\cs[1]{\texttt{\char`\\#1}}
\providecommand\marg[1]{%
  {\ttfamily\char`\{}\meta{#1}{\ttfamily\char`\}}}
\providecommand\oarg[1]{%
  {\ttfamily[}\meta{#1}{\ttfamily]}}
\providecommand\parg[1]{%
  {\ttfamily(}\meta{#1}{\ttfamily)}}
\raggedbottom

% -----------------------------------
\input{pgfplots.preamble.tex}

\makeatletter
% I want a two-column index just as in pgfmanual styles. This here
% was the best way to get one:
\def\index@prologue{\section*{Index}\addcontentsline{toc}{chapter}{Index}
}
\makeatother

%\RequirePackage[german,english,francais]{babel}

\def\matlabcolormaptext{This colormap is similar to one shipped with Matlab$^\text{\textregistered}$ under a similar name.}

\IfFileExists{tikzlibraryspy.code.tex}{%
\usetikzlibrary{spy}
}{%
    \message{ERROR: tikz SPY library NOT available. The manual will only compile partially.^^J}%
}%

\usepackage{xparse}% for colorbrewer manual

\usetikzlibrary{
    decorations.markings,
    decorations.footprints,
    shapes.arrows,
    matrix,
    positioning,
}

\usepgfplotslibrary{
    fillbetween,
    ternary,
    smithchart,
    patchplots,
    polar,
    colormaps,
    colorbrewer,
    colortol,           % docu not ready yet
}
\pgfqkeys{/codeexample}{%
    every codeexample/.append style={
        /pgfplots/every ternary axis/.append style={
            /pgfplots/legend style={fill=graphicbackground},
        }
    },
    tabsize=4,
}

\pgfplotsmanualenableexternalizationofexpensive

%\usetikzlibrary{external}
%\tikzexternalize[prefix=figures/]{pgfplots}

\title{%
    Manual for Package \PGFPlots{}\\
    {\small 2D/3D Plots in \LaTeX{}, Version \pgfplotsversion}\\
    {\small\url{https://github.com/pgf-tikz/pgfplots}}
    %\\{\small Attention: you are using an unstable development version.}
}

\makeatletter
\long\def\abstractsmuggle{%
    \centering
    \textbf{Abstract}\\[0.5cm]

    \begin{minipage}{12cm}
        \PGFPlots{} draws high-quality function plots in normal or logarithmic
        scaling with a user-friendly interface directly in \TeX{}. The user
        supplies axis labels, legend entries and the plot coordinates for one or
        more plots and \PGFPlots{} applies axis scaling, computes any logarithms
        and axis ticks and draws the plots. It supports line plots, scatter
        plots, piecewise constant plots, bar plots, area plots, mesh and surface
        plots, patch plots, contour plots, quiver plots, histogram plots, box
        plots, polar axes, ternary diagrams, smith charts and some more. It is
        based on Till Tantau's package \PGF{}/\Tikz{}.
    \end{minipage}

}%

\expandafter\date\expandafter{\@date\\[2cm]
    \abstractsmuggle
}%
\makeatother

%\includeonly{pgfplots.reference}


\begin{document}

\def\plotcoords{%
\addplot coordinates {
(5,8.312e-02)    (17,2.547e-02)   (49,7.407e-03)
(129,2.102e-03)  (321,5.874e-04)  (769,1.623e-04)
(1793,4.442e-05) (4097,1.207e-05) (9217,3.261e-06)
};

\addplot coordinates{
(7,8.472e-02)    (31,3.044e-02)    (111,1.022e-02)
(351,3.303e-03)  (1023,1.039e-03)  (2815,3.196e-04)
(7423,9.658e-05) (18943,2.873e-05) (47103,8.437e-06)
};

\addplot coordinates{
(9,7.881e-02)     (49,3.243e-02)    (209,1.232e-02)
(769,4.454e-03)   (2561,1.551e-03)  (7937,5.236e-04)
(23297,1.723e-04) (65537,5.545e-05) (178177,1.751e-05)
};

\addplot coordinates{
(11,6.887e-02)    (71,3.177e-02)     (351,1.341e-02)
(1471,5.334e-03)  (5503,2.027e-03)   (18943,7.415e-04)
(61183,2.628e-04) (187903,9.063e-05) (553983,3.053e-05)
};

\addplot coordinates{
(13,5.755e-02)     (97,2.925e-02)     (545,1.351e-02)
(2561,5.842e-03)   (10625,2.397e-03)  (40193,9.414e-04)
(141569,3.564e-04) (471041,1.308e-04)
(1496065,4.670e-05)
};
}%


\bookmaketitle
\tableofcontents
\include{pgfplots.title_abstract_intro}
\include{pgfplots.preliminaries}
\include{pgfplots.intro}
\include{pgfplots.reference}
\include{pgfplots.libs}
\include{pgfplots.resources}
\include{pgfplots.importexport}
\include{pgfplots.basic.reference}

\printindex

\bibliographystyle{abbrv} %gerapali} %gerabbrv} %gerunsrt.bst} %gerabbrv}% gerplain}
\nocite{pgfplotstable}
\nocite{programmingnotes}
\bibliography{pgfplots}
\end{document}

% -----------------------------------------------------------------------------
% For Stefan Pinnow as reminder on what to look for when editing the manual
% -----------------------------------------------------------------------------
% There should be no line breaks in the following environments
% - |...|
% - \declareandlabel{...}
% - \verbpdfref{...}
% If "MakeTikzPictures" isn't running through check one of the externalized
% LOG files what is the cause of that.
% -----------------------------------------------------------------------------


%%%%%%%%%%%%%%%%%%%%%%%%%%%%%%%%%%%%%%%%%%%%%%%%%%%%%%%%%%%%%%%%%%%%%%%%%%%%%
%
% Package pgfplots.sty documentation.
%
% Copyright 2007/2008 by Christian Feuersaenger.
%
% This program is free software: you can redistribute it and/or modify
% it under the terms of the GNU General Public License as published by
% the Free Software Foundation, either version 3 of the License, or
% (at your option) any later version.
%
% This program is distributed in the hope that it will be useful,
% but WITHOUT ANY WARRANTY; without even the implied warranty of
% MERCHANTABILITY or FITNESS FOR A PARTICULAR PURPOSE.  See the
% GNU General Public License for more details.
%
% You should have received a copy of the GNU General Public License
% along with this program.  If not, see <http://www.gnu.org/licenses/>.
%
%
%%%%%%%%%%%%%%%%%%%%%%%%%%%%%%%%%%%%%%%%%%%%%%%%%%%%%%%%%%%%%%%%%%%%%%%%%%%%%
% SEE pgfplots-macros.tex as well!
%\pdfminorversion=5 % to allow compression
%\pdfobjcompresslevel=2
\documentclass[a4paper,openany]{book}

\let\bookmaketitle=\maketitle

% -----------------------------------
% this here is from ltxdoc:
\usepackage{doc}[=v2]
\AtBeginDocument{\MakeShortVerb{\|}}
%\setlength{\textwidth}{355pt}
%\addtolength\marginparwidth{30pt}
%\addtolength\oddsidemargin{20pt}
%\addtolength\evensidemargin{20pt}
\def\cmd#1{\cs{\expandafter\cmd@to@cs\string#1}}
\def\cmd@to@cs#1#2{\char\number`#2\relax}
\DeclareRobustCommand\cs[1]{\texttt{\char`\\#1}}
\providecommand\marg[1]{%
  {\ttfamily\char`\{}\meta{#1}{\ttfamily\char`\}}}
\providecommand\oarg[1]{%
  {\ttfamily[}\meta{#1}{\ttfamily]}}
\providecommand\parg[1]{%
  {\ttfamily(}\meta{#1}{\ttfamily)}}
\raggedbottom

% -----------------------------------
\input{pgfplots.preamble.tex}

\makeatletter
% I want a two-column index just as in pgfmanual styles. This here
% was the best way to get one:
\def\index@prologue{\section*{Index}\addcontentsline{toc}{chapter}{Index}
}
\makeatother

%\RequirePackage[german,english,francais]{babel}

\def\matlabcolormaptext{This colormap is similar to one shipped with Matlab$^\text{\textregistered}$ under a similar name.}

\IfFileExists{tikzlibraryspy.code.tex}{%
\usetikzlibrary{spy}
}{%
    \message{ERROR: tikz SPY library NOT available. The manual will only compile partially.^^J}%
}%

\usepackage{xparse}% for colorbrewer manual

\usetikzlibrary{
    decorations.markings,
    decorations.footprints,
    shapes.arrows,
    matrix,
    positioning,
}

\usepgfplotslibrary{
    fillbetween,
    ternary,
    smithchart,
    patchplots,
    polar,
    colormaps,
    colorbrewer,
    colortol,           % docu not ready yet
}
\pgfqkeys{/codeexample}{%
    every codeexample/.append style={
        /pgfplots/every ternary axis/.append style={
            /pgfplots/legend style={fill=graphicbackground},
        }
    },
    tabsize=4,
}

\pgfplotsmanualenableexternalizationofexpensive

%\usetikzlibrary{external}
%\tikzexternalize[prefix=figures/]{pgfplots}

\title{%
    Manual for Package \PGFPlots{}\\
    {\small 2D/3D Plots in \LaTeX{}, Version \pgfplotsversion}\\
    {\small\url{https://github.com/pgf-tikz/pgfplots}}
    %\\{\small Attention: you are using an unstable development version.}
}

\makeatletter
\long\def\abstractsmuggle{%
    \centering
    \textbf{Abstract}\\[0.5cm]

    \begin{minipage}{12cm}
        \PGFPlots{} draws high-quality function plots in normal or logarithmic
        scaling with a user-friendly interface directly in \TeX{}. The user
        supplies axis labels, legend entries and the plot coordinates for one or
        more plots and \PGFPlots{} applies axis scaling, computes any logarithms
        and axis ticks and draws the plots. It supports line plots, scatter
        plots, piecewise constant plots, bar plots, area plots, mesh and surface
        plots, patch plots, contour plots, quiver plots, histogram plots, box
        plots, polar axes, ternary diagrams, smith charts and some more. It is
        based on Till Tantau's package \PGF{}/\Tikz{}.
    \end{minipage}

}%

\expandafter\date\expandafter{\@date\\[2cm]
    \abstractsmuggle
}%
\makeatother

%\includeonly{pgfplots.reference}


\begin{document}

\def\plotcoords{%
\addplot coordinates {
(5,8.312e-02)    (17,2.547e-02)   (49,7.407e-03)
(129,2.102e-03)  (321,5.874e-04)  (769,1.623e-04)
(1793,4.442e-05) (4097,1.207e-05) (9217,3.261e-06)
};

\addplot coordinates{
(7,8.472e-02)    (31,3.044e-02)    (111,1.022e-02)
(351,3.303e-03)  (1023,1.039e-03)  (2815,3.196e-04)
(7423,9.658e-05) (18943,2.873e-05) (47103,8.437e-06)
};

\addplot coordinates{
(9,7.881e-02)     (49,3.243e-02)    (209,1.232e-02)
(769,4.454e-03)   (2561,1.551e-03)  (7937,5.236e-04)
(23297,1.723e-04) (65537,5.545e-05) (178177,1.751e-05)
};

\addplot coordinates{
(11,6.887e-02)    (71,3.177e-02)     (351,1.341e-02)
(1471,5.334e-03)  (5503,2.027e-03)   (18943,7.415e-04)
(61183,2.628e-04) (187903,9.063e-05) (553983,3.053e-05)
};

\addplot coordinates{
(13,5.755e-02)     (97,2.925e-02)     (545,1.351e-02)
(2561,5.842e-03)   (10625,2.397e-03)  (40193,9.414e-04)
(141569,3.564e-04) (471041,1.308e-04)
(1496065,4.670e-05)
};
}%


\bookmaketitle
\tableofcontents
\include{pgfplots.title_abstract_intro}
\include{pgfplots.preliminaries}
\include{pgfplots.intro}
\include{pgfplots.reference}
\include{pgfplots.libs}
\include{pgfplots.resources}
\include{pgfplots.importexport}
\include{pgfplots.basic.reference}

\printindex

\bibliographystyle{abbrv} %gerapali} %gerabbrv} %gerunsrt.bst} %gerabbrv}% gerplain}
\nocite{pgfplotstable}
\nocite{programmingnotes}
\bibliography{pgfplots}
\end{document}

% -----------------------------------------------------------------------------
% For Stefan Pinnow as reminder on what to look for when editing the manual
% -----------------------------------------------------------------------------
% There should be no line breaks in the following environments
% - |...|
% - \declareandlabel{...}
% - \verbpdfref{...}
% If "MakeTikzPictures" isn't running through check one of the externalized
% LOG files what is the cause of that.
% -----------------------------------------------------------------------------

\include{pgfplots.basic.reference}

\printindex

\bibliographystyle{abbrv} %gerapali} %gerabbrv} %gerunsrt.bst} %gerabbrv}% gerplain}
\nocite{pgfplotstable}
\nocite{programmingnotes}
\bibliography{pgfplots}
\end{document}

% -----------------------------------------------------------------------------
% For Stefan Pinnow as reminder on what to look for when editing the manual
% -----------------------------------------------------------------------------
% There should be no line breaks in the following environments
% - |...|
% - \declareandlabel{...}
% - \verbpdfref{...}
% If "MakeTikzPictures" isn't running through check one of the externalized
% LOG files what is the cause of that.
% -----------------------------------------------------------------------------


%%%%%%%%%%%%%%%%%%%%%%%%%%%%%%%%%%%%%%%%%%%%%%%%%%%%%%%%%%%%%%%%%%%%%%%%%%%%%
%
% Package pgfplots.sty documentation.
%
% Copyright 2007/2008 by Christian Feuersaenger.
%
% This program is free software: you can redistribute it and/or modify
% it under the terms of the GNU General Public License as published by
% the Free Software Foundation, either version 3 of the License, or
% (at your option) any later version.
%
% This program is distributed in the hope that it will be useful,
% but WITHOUT ANY WARRANTY; without even the implied warranty of
% MERCHANTABILITY or FITNESS FOR A PARTICULAR PURPOSE.  See the
% GNU General Public License for more details.
%
% You should have received a copy of the GNU General Public License
% along with this program.  If not, see <http://www.gnu.org/licenses/>.
%
%
%%%%%%%%%%%%%%%%%%%%%%%%%%%%%%%%%%%%%%%%%%%%%%%%%%%%%%%%%%%%%%%%%%%%%%%%%%%%%
% SEE pgfplots-macros.tex as well!
%\pdfminorversion=5 % to allow compression
%\pdfobjcompresslevel=2
\documentclass[a4paper,openany]{book}

\let\bookmaketitle=\maketitle

% -----------------------------------
% this here is from ltxdoc:
\usepackage{doc}[=v2]
\AtBeginDocument{\MakeShortVerb{\|}}
%\setlength{\textwidth}{355pt}
%\addtolength\marginparwidth{30pt}
%\addtolength\oddsidemargin{20pt}
%\addtolength\evensidemargin{20pt}
\def\cmd#1{\cs{\expandafter\cmd@to@cs\string#1}}
\def\cmd@to@cs#1#2{\char\number`#2\relax}
\DeclareRobustCommand\cs[1]{\texttt{\char`\\#1}}
\providecommand\marg[1]{%
  {\ttfamily\char`\{}\meta{#1}{\ttfamily\char`\}}}
\providecommand\oarg[1]{%
  {\ttfamily[}\meta{#1}{\ttfamily]}}
\providecommand\parg[1]{%
  {\ttfamily(}\meta{#1}{\ttfamily)}}
\raggedbottom

% -----------------------------------
\input{pgfplots.preamble.tex}

\makeatletter
% I want a two-column index just as in pgfmanual styles. This here
% was the best way to get one:
\def\index@prologue{\section*{Index}\addcontentsline{toc}{chapter}{Index}
}
\makeatother

%\RequirePackage[german,english,francais]{babel}

\def\matlabcolormaptext{This colormap is similar to one shipped with Matlab$^\text{\textregistered}$ under a similar name.}

\IfFileExists{tikzlibraryspy.code.tex}{%
\usetikzlibrary{spy}
}{%
    \message{ERROR: tikz SPY library NOT available. The manual will only compile partially.^^J}%
}%

\usepackage{xparse}% for colorbrewer manual

\usetikzlibrary{
    decorations.markings,
    decorations.footprints,
    shapes.arrows,
    matrix,
    positioning,
}

\usepgfplotslibrary{
    fillbetween,
    ternary,
    smithchart,
    patchplots,
    polar,
    colormaps,
    colorbrewer,
    colortol,           % docu not ready yet
}
\pgfqkeys{/codeexample}{%
    every codeexample/.append style={
        /pgfplots/every ternary axis/.append style={
            /pgfplots/legend style={fill=graphicbackground},
        }
    },
    tabsize=4,
}

\pgfplotsmanualenableexternalizationofexpensive

%\usetikzlibrary{external}
%\tikzexternalize[prefix=figures/]{pgfplots}

\title{%
    Manual for Package \PGFPlots{}\\
    {\small 2D/3D Plots in \LaTeX{}, Version \pgfplotsversion}\\
    {\small\url{https://github.com/pgf-tikz/pgfplots}}
    %\\{\small Attention: you are using an unstable development version.}
}

\makeatletter
\long\def\abstractsmuggle{%
    \centering
    \textbf{Abstract}\\[0.5cm]

    \begin{minipage}{12cm}
        \PGFPlots{} draws high-quality function plots in normal or logarithmic
        scaling with a user-friendly interface directly in \TeX{}. The user
        supplies axis labels, legend entries and the plot coordinates for one or
        more plots and \PGFPlots{} applies axis scaling, computes any logarithms
        and axis ticks and draws the plots. It supports line plots, scatter
        plots, piecewise constant plots, bar plots, area plots, mesh and surface
        plots, patch plots, contour plots, quiver plots, histogram plots, box
        plots, polar axes, ternary diagrams, smith charts and some more. It is
        based on Till Tantau's package \PGF{}/\Tikz{}.
    \end{minipage}

}%

\expandafter\date\expandafter{\@date\\[2cm]
    \abstractsmuggle
}%
\makeatother

%\includeonly{pgfplots.reference}


\begin{document}

\def\plotcoords{%
\addplot coordinates {
(5,8.312e-02)    (17,2.547e-02)   (49,7.407e-03)
(129,2.102e-03)  (321,5.874e-04)  (769,1.623e-04)
(1793,4.442e-05) (4097,1.207e-05) (9217,3.261e-06)
};

\addplot coordinates{
(7,8.472e-02)    (31,3.044e-02)    (111,1.022e-02)
(351,3.303e-03)  (1023,1.039e-03)  (2815,3.196e-04)
(7423,9.658e-05) (18943,2.873e-05) (47103,8.437e-06)
};

\addplot coordinates{
(9,7.881e-02)     (49,3.243e-02)    (209,1.232e-02)
(769,4.454e-03)   (2561,1.551e-03)  (7937,5.236e-04)
(23297,1.723e-04) (65537,5.545e-05) (178177,1.751e-05)
};

\addplot coordinates{
(11,6.887e-02)    (71,3.177e-02)     (351,1.341e-02)
(1471,5.334e-03)  (5503,2.027e-03)   (18943,7.415e-04)
(61183,2.628e-04) (187903,9.063e-05) (553983,3.053e-05)
};

\addplot coordinates{
(13,5.755e-02)     (97,2.925e-02)     (545,1.351e-02)
(2561,5.842e-03)   (10625,2.397e-03)  (40193,9.414e-04)
(141569,3.564e-04) (471041,1.308e-04)
(1496065,4.670e-05)
};
}%


\bookmaketitle
\tableofcontents

%%%%%%%%%%%%%%%%%%%%%%%%%%%%%%%%%%%%%%%%%%%%%%%%%%%%%%%%%%%%%%%%%%%%%%%%%%%%%
%
% Package pgfplots.sty documentation.
%
% Copyright 2007/2008 by Christian Feuersaenger.
%
% This program is free software: you can redistribute it and/or modify
% it under the terms of the GNU General Public License as published by
% the Free Software Foundation, either version 3 of the License, or
% (at your option) any later version.
%
% This program is distributed in the hope that it will be useful,
% but WITHOUT ANY WARRANTY; without even the implied warranty of
% MERCHANTABILITY or FITNESS FOR A PARTICULAR PURPOSE.  See the
% GNU General Public License for more details.
%
% You should have received a copy of the GNU General Public License
% along with this program.  If not, see <http://www.gnu.org/licenses/>.
%
%
%%%%%%%%%%%%%%%%%%%%%%%%%%%%%%%%%%%%%%%%%%%%%%%%%%%%%%%%%%%%%%%%%%%%%%%%%%%%%
% SEE pgfplots-macros.tex as well!
%\pdfminorversion=5 % to allow compression
%\pdfobjcompresslevel=2
\documentclass[a4paper,openany]{book}

\let\bookmaketitle=\maketitle

% -----------------------------------
% this here is from ltxdoc:
\usepackage{doc}[=v2]
\AtBeginDocument{\MakeShortVerb{\|}}
%\setlength{\textwidth}{355pt}
%\addtolength\marginparwidth{30pt}
%\addtolength\oddsidemargin{20pt}
%\addtolength\evensidemargin{20pt}
\def\cmd#1{\cs{\expandafter\cmd@to@cs\string#1}}
\def\cmd@to@cs#1#2{\char\number`#2\relax}
\DeclareRobustCommand\cs[1]{\texttt{\char`\\#1}}
\providecommand\marg[1]{%
  {\ttfamily\char`\{}\meta{#1}{\ttfamily\char`\}}}
\providecommand\oarg[1]{%
  {\ttfamily[}\meta{#1}{\ttfamily]}}
\providecommand\parg[1]{%
  {\ttfamily(}\meta{#1}{\ttfamily)}}
\raggedbottom

% -----------------------------------
\input{pgfplots.preamble.tex}

\makeatletter
% I want a two-column index just as in pgfmanual styles. This here
% was the best way to get one:
\def\index@prologue{\section*{Index}\addcontentsline{toc}{chapter}{Index}
}
\makeatother

%\RequirePackage[german,english,francais]{babel}

\def\matlabcolormaptext{This colormap is similar to one shipped with Matlab$^\text{\textregistered}$ under a similar name.}

\IfFileExists{tikzlibraryspy.code.tex}{%
\usetikzlibrary{spy}
}{%
    \message{ERROR: tikz SPY library NOT available. The manual will only compile partially.^^J}%
}%

\usepackage{xparse}% for colorbrewer manual

\usetikzlibrary{
    decorations.markings,
    decorations.footprints,
    shapes.arrows,
    matrix,
    positioning,
}

\usepgfplotslibrary{
    fillbetween,
    ternary,
    smithchart,
    patchplots,
    polar,
    colormaps,
    colorbrewer,
    colortol,           % docu not ready yet
}
\pgfqkeys{/codeexample}{%
    every codeexample/.append style={
        /pgfplots/every ternary axis/.append style={
            /pgfplots/legend style={fill=graphicbackground},
        }
    },
    tabsize=4,
}

\pgfplotsmanualenableexternalizationofexpensive

%\usetikzlibrary{external}
%\tikzexternalize[prefix=figures/]{pgfplots}

\title{%
    Manual for Package \PGFPlots{}\\
    {\small 2D/3D Plots in \LaTeX{}, Version \pgfplotsversion}\\
    {\small\url{https://github.com/pgf-tikz/pgfplots}}
    %\\{\small Attention: you are using an unstable development version.}
}

\makeatletter
\long\def\abstractsmuggle{%
    \centering
    \textbf{Abstract}\\[0.5cm]

    \begin{minipage}{12cm}
        \PGFPlots{} draws high-quality function plots in normal or logarithmic
        scaling with a user-friendly interface directly in \TeX{}. The user
        supplies axis labels, legend entries and the plot coordinates for one or
        more plots and \PGFPlots{} applies axis scaling, computes any logarithms
        and axis ticks and draws the plots. It supports line plots, scatter
        plots, piecewise constant plots, bar plots, area plots, mesh and surface
        plots, patch plots, contour plots, quiver plots, histogram plots, box
        plots, polar axes, ternary diagrams, smith charts and some more. It is
        based on Till Tantau's package \PGF{}/\Tikz{}.
    \end{minipage}

}%

\expandafter\date\expandafter{\@date\\[2cm]
    \abstractsmuggle
}%
\makeatother

%\includeonly{pgfplots.reference}


\begin{document}

\def\plotcoords{%
\addplot coordinates {
(5,8.312e-02)    (17,2.547e-02)   (49,7.407e-03)
(129,2.102e-03)  (321,5.874e-04)  (769,1.623e-04)
(1793,4.442e-05) (4097,1.207e-05) (9217,3.261e-06)
};

\addplot coordinates{
(7,8.472e-02)    (31,3.044e-02)    (111,1.022e-02)
(351,3.303e-03)  (1023,1.039e-03)  (2815,3.196e-04)
(7423,9.658e-05) (18943,2.873e-05) (47103,8.437e-06)
};

\addplot coordinates{
(9,7.881e-02)     (49,3.243e-02)    (209,1.232e-02)
(769,4.454e-03)   (2561,1.551e-03)  (7937,5.236e-04)
(23297,1.723e-04) (65537,5.545e-05) (178177,1.751e-05)
};

\addplot coordinates{
(11,6.887e-02)    (71,3.177e-02)     (351,1.341e-02)
(1471,5.334e-03)  (5503,2.027e-03)   (18943,7.415e-04)
(61183,2.628e-04) (187903,9.063e-05) (553983,3.053e-05)
};

\addplot coordinates{
(13,5.755e-02)     (97,2.925e-02)     (545,1.351e-02)
(2561,5.842e-03)   (10625,2.397e-03)  (40193,9.414e-04)
(141569,3.564e-04) (471041,1.308e-04)
(1496065,4.670e-05)
};
}%


\bookmaketitle
\tableofcontents
\include{pgfplots.title_abstract_intro}
\include{pgfplots.preliminaries}
\include{pgfplots.intro}
\include{pgfplots.reference}
\include{pgfplots.libs}
\include{pgfplots.resources}
\include{pgfplots.importexport}
\include{pgfplots.basic.reference}

\printindex

\bibliographystyle{abbrv} %gerapali} %gerabbrv} %gerunsrt.bst} %gerabbrv}% gerplain}
\nocite{pgfplotstable}
\nocite{programmingnotes}
\bibliography{pgfplots}
\end{document}

% -----------------------------------------------------------------------------
% For Stefan Pinnow as reminder on what to look for when editing the manual
% -----------------------------------------------------------------------------
% There should be no line breaks in the following environments
% - |...|
% - \declareandlabel{...}
% - \verbpdfref{...}
% If "MakeTikzPictures" isn't running through check one of the externalized
% LOG files what is the cause of that.
% -----------------------------------------------------------------------------


%%%%%%%%%%%%%%%%%%%%%%%%%%%%%%%%%%%%%%%%%%%%%%%%%%%%%%%%%%%%%%%%%%%%%%%%%%%%%
%
% Package pgfplots.sty documentation.
%
% Copyright 2007/2008 by Christian Feuersaenger.
%
% This program is free software: you can redistribute it and/or modify
% it under the terms of the GNU General Public License as published by
% the Free Software Foundation, either version 3 of the License, or
% (at your option) any later version.
%
% This program is distributed in the hope that it will be useful,
% but WITHOUT ANY WARRANTY; without even the implied warranty of
% MERCHANTABILITY or FITNESS FOR A PARTICULAR PURPOSE.  See the
% GNU General Public License for more details.
%
% You should have received a copy of the GNU General Public License
% along with this program.  If not, see <http://www.gnu.org/licenses/>.
%
%
%%%%%%%%%%%%%%%%%%%%%%%%%%%%%%%%%%%%%%%%%%%%%%%%%%%%%%%%%%%%%%%%%%%%%%%%%%%%%
% SEE pgfplots-macros.tex as well!
%\pdfminorversion=5 % to allow compression
%\pdfobjcompresslevel=2
\documentclass[a4paper,openany]{book}

\let\bookmaketitle=\maketitle

% -----------------------------------
% this here is from ltxdoc:
\usepackage{doc}[=v2]
\AtBeginDocument{\MakeShortVerb{\|}}
%\setlength{\textwidth}{355pt}
%\addtolength\marginparwidth{30pt}
%\addtolength\oddsidemargin{20pt}
%\addtolength\evensidemargin{20pt}
\def\cmd#1{\cs{\expandafter\cmd@to@cs\string#1}}
\def\cmd@to@cs#1#2{\char\number`#2\relax}
\DeclareRobustCommand\cs[1]{\texttt{\char`\\#1}}
\providecommand\marg[1]{%
  {\ttfamily\char`\{}\meta{#1}{\ttfamily\char`\}}}
\providecommand\oarg[1]{%
  {\ttfamily[}\meta{#1}{\ttfamily]}}
\providecommand\parg[1]{%
  {\ttfamily(}\meta{#1}{\ttfamily)}}
\raggedbottom

% -----------------------------------
\input{pgfplots.preamble.tex}

\makeatletter
% I want a two-column index just as in pgfmanual styles. This here
% was the best way to get one:
\def\index@prologue{\section*{Index}\addcontentsline{toc}{chapter}{Index}
}
\makeatother

%\RequirePackage[german,english,francais]{babel}

\def\matlabcolormaptext{This colormap is similar to one shipped with Matlab$^\text{\textregistered}$ under a similar name.}

\IfFileExists{tikzlibraryspy.code.tex}{%
\usetikzlibrary{spy}
}{%
    \message{ERROR: tikz SPY library NOT available. The manual will only compile partially.^^J}%
}%

\usepackage{xparse}% for colorbrewer manual

\usetikzlibrary{
    decorations.markings,
    decorations.footprints,
    shapes.arrows,
    matrix,
    positioning,
}

\usepgfplotslibrary{
    fillbetween,
    ternary,
    smithchart,
    patchplots,
    polar,
    colormaps,
    colorbrewer,
    colortol,           % docu not ready yet
}
\pgfqkeys{/codeexample}{%
    every codeexample/.append style={
        /pgfplots/every ternary axis/.append style={
            /pgfplots/legend style={fill=graphicbackground},
        }
    },
    tabsize=4,
}

\pgfplotsmanualenableexternalizationofexpensive

%\usetikzlibrary{external}
%\tikzexternalize[prefix=figures/]{pgfplots}

\title{%
    Manual for Package \PGFPlots{}\\
    {\small 2D/3D Plots in \LaTeX{}, Version \pgfplotsversion}\\
    {\small\url{https://github.com/pgf-tikz/pgfplots}}
    %\\{\small Attention: you are using an unstable development version.}
}

\makeatletter
\long\def\abstractsmuggle{%
    \centering
    \textbf{Abstract}\\[0.5cm]

    \begin{minipage}{12cm}
        \PGFPlots{} draws high-quality function plots in normal or logarithmic
        scaling with a user-friendly interface directly in \TeX{}. The user
        supplies axis labels, legend entries and the plot coordinates for one or
        more plots and \PGFPlots{} applies axis scaling, computes any logarithms
        and axis ticks and draws the plots. It supports line plots, scatter
        plots, piecewise constant plots, bar plots, area plots, mesh and surface
        plots, patch plots, contour plots, quiver plots, histogram plots, box
        plots, polar axes, ternary diagrams, smith charts and some more. It is
        based on Till Tantau's package \PGF{}/\Tikz{}.
    \end{minipage}

}%

\expandafter\date\expandafter{\@date\\[2cm]
    \abstractsmuggle
}%
\makeatother

%\includeonly{pgfplots.reference}


\begin{document}

\def\plotcoords{%
\addplot coordinates {
(5,8.312e-02)    (17,2.547e-02)   (49,7.407e-03)
(129,2.102e-03)  (321,5.874e-04)  (769,1.623e-04)
(1793,4.442e-05) (4097,1.207e-05) (9217,3.261e-06)
};

\addplot coordinates{
(7,8.472e-02)    (31,3.044e-02)    (111,1.022e-02)
(351,3.303e-03)  (1023,1.039e-03)  (2815,3.196e-04)
(7423,9.658e-05) (18943,2.873e-05) (47103,8.437e-06)
};

\addplot coordinates{
(9,7.881e-02)     (49,3.243e-02)    (209,1.232e-02)
(769,4.454e-03)   (2561,1.551e-03)  (7937,5.236e-04)
(23297,1.723e-04) (65537,5.545e-05) (178177,1.751e-05)
};

\addplot coordinates{
(11,6.887e-02)    (71,3.177e-02)     (351,1.341e-02)
(1471,5.334e-03)  (5503,2.027e-03)   (18943,7.415e-04)
(61183,2.628e-04) (187903,9.063e-05) (553983,3.053e-05)
};

\addplot coordinates{
(13,5.755e-02)     (97,2.925e-02)     (545,1.351e-02)
(2561,5.842e-03)   (10625,2.397e-03)  (40193,9.414e-04)
(141569,3.564e-04) (471041,1.308e-04)
(1496065,4.670e-05)
};
}%


\bookmaketitle
\tableofcontents
\include{pgfplots.title_abstract_intro}
\include{pgfplots.preliminaries}
\include{pgfplots.intro}
\include{pgfplots.reference}
\include{pgfplots.libs}
\include{pgfplots.resources}
\include{pgfplots.importexport}
\include{pgfplots.basic.reference}

\printindex

\bibliographystyle{abbrv} %gerapali} %gerabbrv} %gerunsrt.bst} %gerabbrv}% gerplain}
\nocite{pgfplotstable}
\nocite{programmingnotes}
\bibliography{pgfplots}
\end{document}

% -----------------------------------------------------------------------------
% For Stefan Pinnow as reminder on what to look for when editing the manual
% -----------------------------------------------------------------------------
% There should be no line breaks in the following environments
% - |...|
% - \declareandlabel{...}
% - \verbpdfref{...}
% If "MakeTikzPictures" isn't running through check one of the externalized
% LOG files what is the cause of that.
% -----------------------------------------------------------------------------


%%%%%%%%%%%%%%%%%%%%%%%%%%%%%%%%%%%%%%%%%%%%%%%%%%%%%%%%%%%%%%%%%%%%%%%%%%%%%
%
% Package pgfplots.sty documentation.
%
% Copyright 2007/2008 by Christian Feuersaenger.
%
% This program is free software: you can redistribute it and/or modify
% it under the terms of the GNU General Public License as published by
% the Free Software Foundation, either version 3 of the License, or
% (at your option) any later version.
%
% This program is distributed in the hope that it will be useful,
% but WITHOUT ANY WARRANTY; without even the implied warranty of
% MERCHANTABILITY or FITNESS FOR A PARTICULAR PURPOSE.  See the
% GNU General Public License for more details.
%
% You should have received a copy of the GNU General Public License
% along with this program.  If not, see <http://www.gnu.org/licenses/>.
%
%
%%%%%%%%%%%%%%%%%%%%%%%%%%%%%%%%%%%%%%%%%%%%%%%%%%%%%%%%%%%%%%%%%%%%%%%%%%%%%
% SEE pgfplots-macros.tex as well!
%\pdfminorversion=5 % to allow compression
%\pdfobjcompresslevel=2
\documentclass[a4paper,openany]{book}

\let\bookmaketitle=\maketitle

% -----------------------------------
% this here is from ltxdoc:
\usepackage{doc}[=v2]
\AtBeginDocument{\MakeShortVerb{\|}}
%\setlength{\textwidth}{355pt}
%\addtolength\marginparwidth{30pt}
%\addtolength\oddsidemargin{20pt}
%\addtolength\evensidemargin{20pt}
\def\cmd#1{\cs{\expandafter\cmd@to@cs\string#1}}
\def\cmd@to@cs#1#2{\char\number`#2\relax}
\DeclareRobustCommand\cs[1]{\texttt{\char`\\#1}}
\providecommand\marg[1]{%
  {\ttfamily\char`\{}\meta{#1}{\ttfamily\char`\}}}
\providecommand\oarg[1]{%
  {\ttfamily[}\meta{#1}{\ttfamily]}}
\providecommand\parg[1]{%
  {\ttfamily(}\meta{#1}{\ttfamily)}}
\raggedbottom

% -----------------------------------
\input{pgfplots.preamble.tex}

\makeatletter
% I want a two-column index just as in pgfmanual styles. This here
% was the best way to get one:
\def\index@prologue{\section*{Index}\addcontentsline{toc}{chapter}{Index}
}
\makeatother

%\RequirePackage[german,english,francais]{babel}

\def\matlabcolormaptext{This colormap is similar to one shipped with Matlab$^\text{\textregistered}$ under a similar name.}

\IfFileExists{tikzlibraryspy.code.tex}{%
\usetikzlibrary{spy}
}{%
    \message{ERROR: tikz SPY library NOT available. The manual will only compile partially.^^J}%
}%

\usepackage{xparse}% for colorbrewer manual

\usetikzlibrary{
    decorations.markings,
    decorations.footprints,
    shapes.arrows,
    matrix,
    positioning,
}

\usepgfplotslibrary{
    fillbetween,
    ternary,
    smithchart,
    patchplots,
    polar,
    colormaps,
    colorbrewer,
    colortol,           % docu not ready yet
}
\pgfqkeys{/codeexample}{%
    every codeexample/.append style={
        /pgfplots/every ternary axis/.append style={
            /pgfplots/legend style={fill=graphicbackground},
        }
    },
    tabsize=4,
}

\pgfplotsmanualenableexternalizationofexpensive

%\usetikzlibrary{external}
%\tikzexternalize[prefix=figures/]{pgfplots}

\title{%
    Manual for Package \PGFPlots{}\\
    {\small 2D/3D Plots in \LaTeX{}, Version \pgfplotsversion}\\
    {\small\url{https://github.com/pgf-tikz/pgfplots}}
    %\\{\small Attention: you are using an unstable development version.}
}

\makeatletter
\long\def\abstractsmuggle{%
    \centering
    \textbf{Abstract}\\[0.5cm]

    \begin{minipage}{12cm}
        \PGFPlots{} draws high-quality function plots in normal or logarithmic
        scaling with a user-friendly interface directly in \TeX{}. The user
        supplies axis labels, legend entries and the plot coordinates for one or
        more plots and \PGFPlots{} applies axis scaling, computes any logarithms
        and axis ticks and draws the plots. It supports line plots, scatter
        plots, piecewise constant plots, bar plots, area plots, mesh and surface
        plots, patch plots, contour plots, quiver plots, histogram plots, box
        plots, polar axes, ternary diagrams, smith charts and some more. It is
        based on Till Tantau's package \PGF{}/\Tikz{}.
    \end{minipage}

}%

\expandafter\date\expandafter{\@date\\[2cm]
    \abstractsmuggle
}%
\makeatother

%\includeonly{pgfplots.reference}


\begin{document}

\def\plotcoords{%
\addplot coordinates {
(5,8.312e-02)    (17,2.547e-02)   (49,7.407e-03)
(129,2.102e-03)  (321,5.874e-04)  (769,1.623e-04)
(1793,4.442e-05) (4097,1.207e-05) (9217,3.261e-06)
};

\addplot coordinates{
(7,8.472e-02)    (31,3.044e-02)    (111,1.022e-02)
(351,3.303e-03)  (1023,1.039e-03)  (2815,3.196e-04)
(7423,9.658e-05) (18943,2.873e-05) (47103,8.437e-06)
};

\addplot coordinates{
(9,7.881e-02)     (49,3.243e-02)    (209,1.232e-02)
(769,4.454e-03)   (2561,1.551e-03)  (7937,5.236e-04)
(23297,1.723e-04) (65537,5.545e-05) (178177,1.751e-05)
};

\addplot coordinates{
(11,6.887e-02)    (71,3.177e-02)     (351,1.341e-02)
(1471,5.334e-03)  (5503,2.027e-03)   (18943,7.415e-04)
(61183,2.628e-04) (187903,9.063e-05) (553983,3.053e-05)
};

\addplot coordinates{
(13,5.755e-02)     (97,2.925e-02)     (545,1.351e-02)
(2561,5.842e-03)   (10625,2.397e-03)  (40193,9.414e-04)
(141569,3.564e-04) (471041,1.308e-04)
(1496065,4.670e-05)
};
}%


\bookmaketitle
\tableofcontents
\include{pgfplots.title_abstract_intro}
\include{pgfplots.preliminaries}
\include{pgfplots.intro}
\include{pgfplots.reference}
\include{pgfplots.libs}
\include{pgfplots.resources}
\include{pgfplots.importexport}
\include{pgfplots.basic.reference}

\printindex

\bibliographystyle{abbrv} %gerapali} %gerabbrv} %gerunsrt.bst} %gerabbrv}% gerplain}
\nocite{pgfplotstable}
\nocite{programmingnotes}
\bibliography{pgfplots}
\end{document}

% -----------------------------------------------------------------------------
% For Stefan Pinnow as reminder on what to look for when editing the manual
% -----------------------------------------------------------------------------
% There should be no line breaks in the following environments
% - |...|
% - \declareandlabel{...}
% - \verbpdfref{...}
% If "MakeTikzPictures" isn't running through check one of the externalized
% LOG files what is the cause of that.
% -----------------------------------------------------------------------------


%%%%%%%%%%%%%%%%%%%%%%%%%%%%%%%%%%%%%%%%%%%%%%%%%%%%%%%%%%%%%%%%%%%%%%%%%%%%%
%
% Package pgfplots.sty documentation.
%
% Copyright 2007/2008 by Christian Feuersaenger.
%
% This program is free software: you can redistribute it and/or modify
% it under the terms of the GNU General Public License as published by
% the Free Software Foundation, either version 3 of the License, or
% (at your option) any later version.
%
% This program is distributed in the hope that it will be useful,
% but WITHOUT ANY WARRANTY; without even the implied warranty of
% MERCHANTABILITY or FITNESS FOR A PARTICULAR PURPOSE.  See the
% GNU General Public License for more details.
%
% You should have received a copy of the GNU General Public License
% along with this program.  If not, see <http://www.gnu.org/licenses/>.
%
%
%%%%%%%%%%%%%%%%%%%%%%%%%%%%%%%%%%%%%%%%%%%%%%%%%%%%%%%%%%%%%%%%%%%%%%%%%%%%%
% SEE pgfplots-macros.tex as well!
%\pdfminorversion=5 % to allow compression
%\pdfobjcompresslevel=2
\documentclass[a4paper,openany]{book}

\let\bookmaketitle=\maketitle

% -----------------------------------
% this here is from ltxdoc:
\usepackage{doc}[=v2]
\AtBeginDocument{\MakeShortVerb{\|}}
%\setlength{\textwidth}{355pt}
%\addtolength\marginparwidth{30pt}
%\addtolength\oddsidemargin{20pt}
%\addtolength\evensidemargin{20pt}
\def\cmd#1{\cs{\expandafter\cmd@to@cs\string#1}}
\def\cmd@to@cs#1#2{\char\number`#2\relax}
\DeclareRobustCommand\cs[1]{\texttt{\char`\\#1}}
\providecommand\marg[1]{%
  {\ttfamily\char`\{}\meta{#1}{\ttfamily\char`\}}}
\providecommand\oarg[1]{%
  {\ttfamily[}\meta{#1}{\ttfamily]}}
\providecommand\parg[1]{%
  {\ttfamily(}\meta{#1}{\ttfamily)}}
\raggedbottom

% -----------------------------------
\input{pgfplots.preamble.tex}

\makeatletter
% I want a two-column index just as in pgfmanual styles. This here
% was the best way to get one:
\def\index@prologue{\section*{Index}\addcontentsline{toc}{chapter}{Index}
}
\makeatother

%\RequirePackage[german,english,francais]{babel}

\def\matlabcolormaptext{This colormap is similar to one shipped with Matlab$^\text{\textregistered}$ under a similar name.}

\IfFileExists{tikzlibraryspy.code.tex}{%
\usetikzlibrary{spy}
}{%
    \message{ERROR: tikz SPY library NOT available. The manual will only compile partially.^^J}%
}%

\usepackage{xparse}% for colorbrewer manual

\usetikzlibrary{
    decorations.markings,
    decorations.footprints,
    shapes.arrows,
    matrix,
    positioning,
}

\usepgfplotslibrary{
    fillbetween,
    ternary,
    smithchart,
    patchplots,
    polar,
    colormaps,
    colorbrewer,
    colortol,           % docu not ready yet
}
\pgfqkeys{/codeexample}{%
    every codeexample/.append style={
        /pgfplots/every ternary axis/.append style={
            /pgfplots/legend style={fill=graphicbackground},
        }
    },
    tabsize=4,
}

\pgfplotsmanualenableexternalizationofexpensive

%\usetikzlibrary{external}
%\tikzexternalize[prefix=figures/]{pgfplots}

\title{%
    Manual for Package \PGFPlots{}\\
    {\small 2D/3D Plots in \LaTeX{}, Version \pgfplotsversion}\\
    {\small\url{https://github.com/pgf-tikz/pgfplots}}
    %\\{\small Attention: you are using an unstable development version.}
}

\makeatletter
\long\def\abstractsmuggle{%
    \centering
    \textbf{Abstract}\\[0.5cm]

    \begin{minipage}{12cm}
        \PGFPlots{} draws high-quality function plots in normal or logarithmic
        scaling with a user-friendly interface directly in \TeX{}. The user
        supplies axis labels, legend entries and the plot coordinates for one or
        more plots and \PGFPlots{} applies axis scaling, computes any logarithms
        and axis ticks and draws the plots. It supports line plots, scatter
        plots, piecewise constant plots, bar plots, area plots, mesh and surface
        plots, patch plots, contour plots, quiver plots, histogram plots, box
        plots, polar axes, ternary diagrams, smith charts and some more. It is
        based on Till Tantau's package \PGF{}/\Tikz{}.
    \end{minipage}

}%

\expandafter\date\expandafter{\@date\\[2cm]
    \abstractsmuggle
}%
\makeatother

%\includeonly{pgfplots.reference}


\begin{document}

\def\plotcoords{%
\addplot coordinates {
(5,8.312e-02)    (17,2.547e-02)   (49,7.407e-03)
(129,2.102e-03)  (321,5.874e-04)  (769,1.623e-04)
(1793,4.442e-05) (4097,1.207e-05) (9217,3.261e-06)
};

\addplot coordinates{
(7,8.472e-02)    (31,3.044e-02)    (111,1.022e-02)
(351,3.303e-03)  (1023,1.039e-03)  (2815,3.196e-04)
(7423,9.658e-05) (18943,2.873e-05) (47103,8.437e-06)
};

\addplot coordinates{
(9,7.881e-02)     (49,3.243e-02)    (209,1.232e-02)
(769,4.454e-03)   (2561,1.551e-03)  (7937,5.236e-04)
(23297,1.723e-04) (65537,5.545e-05) (178177,1.751e-05)
};

\addplot coordinates{
(11,6.887e-02)    (71,3.177e-02)     (351,1.341e-02)
(1471,5.334e-03)  (5503,2.027e-03)   (18943,7.415e-04)
(61183,2.628e-04) (187903,9.063e-05) (553983,3.053e-05)
};

\addplot coordinates{
(13,5.755e-02)     (97,2.925e-02)     (545,1.351e-02)
(2561,5.842e-03)   (10625,2.397e-03)  (40193,9.414e-04)
(141569,3.564e-04) (471041,1.308e-04)
(1496065,4.670e-05)
};
}%


\bookmaketitle
\tableofcontents
\include{pgfplots.title_abstract_intro}
\include{pgfplots.preliminaries}
\include{pgfplots.intro}
\include{pgfplots.reference}
\include{pgfplots.libs}
\include{pgfplots.resources}
\include{pgfplots.importexport}
\include{pgfplots.basic.reference}

\printindex

\bibliographystyle{abbrv} %gerapali} %gerabbrv} %gerunsrt.bst} %gerabbrv}% gerplain}
\nocite{pgfplotstable}
\nocite{programmingnotes}
\bibliography{pgfplots}
\end{document}

% -----------------------------------------------------------------------------
% For Stefan Pinnow as reminder on what to look for when editing the manual
% -----------------------------------------------------------------------------
% There should be no line breaks in the following environments
% - |...|
% - \declareandlabel{...}
% - \verbpdfref{...}
% If "MakeTikzPictures" isn't running through check one of the externalized
% LOG files what is the cause of that.
% -----------------------------------------------------------------------------


%%%%%%%%%%%%%%%%%%%%%%%%%%%%%%%%%%%%%%%%%%%%%%%%%%%%%%%%%%%%%%%%%%%%%%%%%%%%%
%
% Package pgfplots.sty documentation.
%
% Copyright 2007/2008 by Christian Feuersaenger.
%
% This program is free software: you can redistribute it and/or modify
% it under the terms of the GNU General Public License as published by
% the Free Software Foundation, either version 3 of the License, or
% (at your option) any later version.
%
% This program is distributed in the hope that it will be useful,
% but WITHOUT ANY WARRANTY; without even the implied warranty of
% MERCHANTABILITY or FITNESS FOR A PARTICULAR PURPOSE.  See the
% GNU General Public License for more details.
%
% You should have received a copy of the GNU General Public License
% along with this program.  If not, see <http://www.gnu.org/licenses/>.
%
%
%%%%%%%%%%%%%%%%%%%%%%%%%%%%%%%%%%%%%%%%%%%%%%%%%%%%%%%%%%%%%%%%%%%%%%%%%%%%%
% SEE pgfplots-macros.tex as well!
%\pdfminorversion=5 % to allow compression
%\pdfobjcompresslevel=2
\documentclass[a4paper,openany]{book}

\let\bookmaketitle=\maketitle

% -----------------------------------
% this here is from ltxdoc:
\usepackage{doc}[=v2]
\AtBeginDocument{\MakeShortVerb{\|}}
%\setlength{\textwidth}{355pt}
%\addtolength\marginparwidth{30pt}
%\addtolength\oddsidemargin{20pt}
%\addtolength\evensidemargin{20pt}
\def\cmd#1{\cs{\expandafter\cmd@to@cs\string#1}}
\def\cmd@to@cs#1#2{\char\number`#2\relax}
\DeclareRobustCommand\cs[1]{\texttt{\char`\\#1}}
\providecommand\marg[1]{%
  {\ttfamily\char`\{}\meta{#1}{\ttfamily\char`\}}}
\providecommand\oarg[1]{%
  {\ttfamily[}\meta{#1}{\ttfamily]}}
\providecommand\parg[1]{%
  {\ttfamily(}\meta{#1}{\ttfamily)}}
\raggedbottom

% -----------------------------------
\input{pgfplots.preamble.tex}

\makeatletter
% I want a two-column index just as in pgfmanual styles. This here
% was the best way to get one:
\def\index@prologue{\section*{Index}\addcontentsline{toc}{chapter}{Index}
}
\makeatother

%\RequirePackage[german,english,francais]{babel}

\def\matlabcolormaptext{This colormap is similar to one shipped with Matlab$^\text{\textregistered}$ under a similar name.}

\IfFileExists{tikzlibraryspy.code.tex}{%
\usetikzlibrary{spy}
}{%
    \message{ERROR: tikz SPY library NOT available. The manual will only compile partially.^^J}%
}%

\usepackage{xparse}% for colorbrewer manual

\usetikzlibrary{
    decorations.markings,
    decorations.footprints,
    shapes.arrows,
    matrix,
    positioning,
}

\usepgfplotslibrary{
    fillbetween,
    ternary,
    smithchart,
    patchplots,
    polar,
    colormaps,
    colorbrewer,
    colortol,           % docu not ready yet
}
\pgfqkeys{/codeexample}{%
    every codeexample/.append style={
        /pgfplots/every ternary axis/.append style={
            /pgfplots/legend style={fill=graphicbackground},
        }
    },
    tabsize=4,
}

\pgfplotsmanualenableexternalizationofexpensive

%\usetikzlibrary{external}
%\tikzexternalize[prefix=figures/]{pgfplots}

\title{%
    Manual for Package \PGFPlots{}\\
    {\small 2D/3D Plots in \LaTeX{}, Version \pgfplotsversion}\\
    {\small\url{https://github.com/pgf-tikz/pgfplots}}
    %\\{\small Attention: you are using an unstable development version.}
}

\makeatletter
\long\def\abstractsmuggle{%
    \centering
    \textbf{Abstract}\\[0.5cm]

    \begin{minipage}{12cm}
        \PGFPlots{} draws high-quality function plots in normal or logarithmic
        scaling with a user-friendly interface directly in \TeX{}. The user
        supplies axis labels, legend entries and the plot coordinates for one or
        more plots and \PGFPlots{} applies axis scaling, computes any logarithms
        and axis ticks and draws the plots. It supports line plots, scatter
        plots, piecewise constant plots, bar plots, area plots, mesh and surface
        plots, patch plots, contour plots, quiver plots, histogram plots, box
        plots, polar axes, ternary diagrams, smith charts and some more. It is
        based on Till Tantau's package \PGF{}/\Tikz{}.
    \end{minipage}

}%

\expandafter\date\expandafter{\@date\\[2cm]
    \abstractsmuggle
}%
\makeatother

%\includeonly{pgfplots.reference}


\begin{document}

\def\plotcoords{%
\addplot coordinates {
(5,8.312e-02)    (17,2.547e-02)   (49,7.407e-03)
(129,2.102e-03)  (321,5.874e-04)  (769,1.623e-04)
(1793,4.442e-05) (4097,1.207e-05) (9217,3.261e-06)
};

\addplot coordinates{
(7,8.472e-02)    (31,3.044e-02)    (111,1.022e-02)
(351,3.303e-03)  (1023,1.039e-03)  (2815,3.196e-04)
(7423,9.658e-05) (18943,2.873e-05) (47103,8.437e-06)
};

\addplot coordinates{
(9,7.881e-02)     (49,3.243e-02)    (209,1.232e-02)
(769,4.454e-03)   (2561,1.551e-03)  (7937,5.236e-04)
(23297,1.723e-04) (65537,5.545e-05) (178177,1.751e-05)
};

\addplot coordinates{
(11,6.887e-02)    (71,3.177e-02)     (351,1.341e-02)
(1471,5.334e-03)  (5503,2.027e-03)   (18943,7.415e-04)
(61183,2.628e-04) (187903,9.063e-05) (553983,3.053e-05)
};

\addplot coordinates{
(13,5.755e-02)     (97,2.925e-02)     (545,1.351e-02)
(2561,5.842e-03)   (10625,2.397e-03)  (40193,9.414e-04)
(141569,3.564e-04) (471041,1.308e-04)
(1496065,4.670e-05)
};
}%


\bookmaketitle
\tableofcontents
\include{pgfplots.title_abstract_intro}
\include{pgfplots.preliminaries}
\include{pgfplots.intro}
\include{pgfplots.reference}
\include{pgfplots.libs}
\include{pgfplots.resources}
\include{pgfplots.importexport}
\include{pgfplots.basic.reference}

\printindex

\bibliographystyle{abbrv} %gerapali} %gerabbrv} %gerunsrt.bst} %gerabbrv}% gerplain}
\nocite{pgfplotstable}
\nocite{programmingnotes}
\bibliography{pgfplots}
\end{document}

% -----------------------------------------------------------------------------
% For Stefan Pinnow as reminder on what to look for when editing the manual
% -----------------------------------------------------------------------------
% There should be no line breaks in the following environments
% - |...|
% - \declareandlabel{...}
% - \verbpdfref{...}
% If "MakeTikzPictures" isn't running through check one of the externalized
% LOG files what is the cause of that.
% -----------------------------------------------------------------------------


%%%%%%%%%%%%%%%%%%%%%%%%%%%%%%%%%%%%%%%%%%%%%%%%%%%%%%%%%%%%%%%%%%%%%%%%%%%%%
%
% Package pgfplots.sty documentation.
%
% Copyright 2007/2008 by Christian Feuersaenger.
%
% This program is free software: you can redistribute it and/or modify
% it under the terms of the GNU General Public License as published by
% the Free Software Foundation, either version 3 of the License, or
% (at your option) any later version.
%
% This program is distributed in the hope that it will be useful,
% but WITHOUT ANY WARRANTY; without even the implied warranty of
% MERCHANTABILITY or FITNESS FOR A PARTICULAR PURPOSE.  See the
% GNU General Public License for more details.
%
% You should have received a copy of the GNU General Public License
% along with this program.  If not, see <http://www.gnu.org/licenses/>.
%
%
%%%%%%%%%%%%%%%%%%%%%%%%%%%%%%%%%%%%%%%%%%%%%%%%%%%%%%%%%%%%%%%%%%%%%%%%%%%%%
% SEE pgfplots-macros.tex as well!
%\pdfminorversion=5 % to allow compression
%\pdfobjcompresslevel=2
\documentclass[a4paper,openany]{book}

\let\bookmaketitle=\maketitle

% -----------------------------------
% this here is from ltxdoc:
\usepackage{doc}[=v2]
\AtBeginDocument{\MakeShortVerb{\|}}
%\setlength{\textwidth}{355pt}
%\addtolength\marginparwidth{30pt}
%\addtolength\oddsidemargin{20pt}
%\addtolength\evensidemargin{20pt}
\def\cmd#1{\cs{\expandafter\cmd@to@cs\string#1}}
\def\cmd@to@cs#1#2{\char\number`#2\relax}
\DeclareRobustCommand\cs[1]{\texttt{\char`\\#1}}
\providecommand\marg[1]{%
  {\ttfamily\char`\{}\meta{#1}{\ttfamily\char`\}}}
\providecommand\oarg[1]{%
  {\ttfamily[}\meta{#1}{\ttfamily]}}
\providecommand\parg[1]{%
  {\ttfamily(}\meta{#1}{\ttfamily)}}
\raggedbottom

% -----------------------------------
\input{pgfplots.preamble.tex}

\makeatletter
% I want a two-column index just as in pgfmanual styles. This here
% was the best way to get one:
\def\index@prologue{\section*{Index}\addcontentsline{toc}{chapter}{Index}
}
\makeatother

%\RequirePackage[german,english,francais]{babel}

\def\matlabcolormaptext{This colormap is similar to one shipped with Matlab$^\text{\textregistered}$ under a similar name.}

\IfFileExists{tikzlibraryspy.code.tex}{%
\usetikzlibrary{spy}
}{%
    \message{ERROR: tikz SPY library NOT available. The manual will only compile partially.^^J}%
}%

\usepackage{xparse}% for colorbrewer manual

\usetikzlibrary{
    decorations.markings,
    decorations.footprints,
    shapes.arrows,
    matrix,
    positioning,
}

\usepgfplotslibrary{
    fillbetween,
    ternary,
    smithchart,
    patchplots,
    polar,
    colormaps,
    colorbrewer,
    colortol,           % docu not ready yet
}
\pgfqkeys{/codeexample}{%
    every codeexample/.append style={
        /pgfplots/every ternary axis/.append style={
            /pgfplots/legend style={fill=graphicbackground},
        }
    },
    tabsize=4,
}

\pgfplotsmanualenableexternalizationofexpensive

%\usetikzlibrary{external}
%\tikzexternalize[prefix=figures/]{pgfplots}

\title{%
    Manual for Package \PGFPlots{}\\
    {\small 2D/3D Plots in \LaTeX{}, Version \pgfplotsversion}\\
    {\small\url{https://github.com/pgf-tikz/pgfplots}}
    %\\{\small Attention: you are using an unstable development version.}
}

\makeatletter
\long\def\abstractsmuggle{%
    \centering
    \textbf{Abstract}\\[0.5cm]

    \begin{minipage}{12cm}
        \PGFPlots{} draws high-quality function plots in normal or logarithmic
        scaling with a user-friendly interface directly in \TeX{}. The user
        supplies axis labels, legend entries and the plot coordinates for one or
        more plots and \PGFPlots{} applies axis scaling, computes any logarithms
        and axis ticks and draws the plots. It supports line plots, scatter
        plots, piecewise constant plots, bar plots, area plots, mesh and surface
        plots, patch plots, contour plots, quiver plots, histogram plots, box
        plots, polar axes, ternary diagrams, smith charts and some more. It is
        based on Till Tantau's package \PGF{}/\Tikz{}.
    \end{minipage}

}%

\expandafter\date\expandafter{\@date\\[2cm]
    \abstractsmuggle
}%
\makeatother

%\includeonly{pgfplots.reference}


\begin{document}

\def\plotcoords{%
\addplot coordinates {
(5,8.312e-02)    (17,2.547e-02)   (49,7.407e-03)
(129,2.102e-03)  (321,5.874e-04)  (769,1.623e-04)
(1793,4.442e-05) (4097,1.207e-05) (9217,3.261e-06)
};

\addplot coordinates{
(7,8.472e-02)    (31,3.044e-02)    (111,1.022e-02)
(351,3.303e-03)  (1023,1.039e-03)  (2815,3.196e-04)
(7423,9.658e-05) (18943,2.873e-05) (47103,8.437e-06)
};

\addplot coordinates{
(9,7.881e-02)     (49,3.243e-02)    (209,1.232e-02)
(769,4.454e-03)   (2561,1.551e-03)  (7937,5.236e-04)
(23297,1.723e-04) (65537,5.545e-05) (178177,1.751e-05)
};

\addplot coordinates{
(11,6.887e-02)    (71,3.177e-02)     (351,1.341e-02)
(1471,5.334e-03)  (5503,2.027e-03)   (18943,7.415e-04)
(61183,2.628e-04) (187903,9.063e-05) (553983,3.053e-05)
};

\addplot coordinates{
(13,5.755e-02)     (97,2.925e-02)     (545,1.351e-02)
(2561,5.842e-03)   (10625,2.397e-03)  (40193,9.414e-04)
(141569,3.564e-04) (471041,1.308e-04)
(1496065,4.670e-05)
};
}%


\bookmaketitle
\tableofcontents
\include{pgfplots.title_abstract_intro}
\include{pgfplots.preliminaries}
\include{pgfplots.intro}
\include{pgfplots.reference}
\include{pgfplots.libs}
\include{pgfplots.resources}
\include{pgfplots.importexport}
\include{pgfplots.basic.reference}

\printindex

\bibliographystyle{abbrv} %gerapali} %gerabbrv} %gerunsrt.bst} %gerabbrv}% gerplain}
\nocite{pgfplotstable}
\nocite{programmingnotes}
\bibliography{pgfplots}
\end{document}

% -----------------------------------------------------------------------------
% For Stefan Pinnow as reminder on what to look for when editing the manual
% -----------------------------------------------------------------------------
% There should be no line breaks in the following environments
% - |...|
% - \declareandlabel{...}
% - \verbpdfref{...}
% If "MakeTikzPictures" isn't running through check one of the externalized
% LOG files what is the cause of that.
% -----------------------------------------------------------------------------


%%%%%%%%%%%%%%%%%%%%%%%%%%%%%%%%%%%%%%%%%%%%%%%%%%%%%%%%%%%%%%%%%%%%%%%%%%%%%
%
% Package pgfplots.sty documentation.
%
% Copyright 2007/2008 by Christian Feuersaenger.
%
% This program is free software: you can redistribute it and/or modify
% it under the terms of the GNU General Public License as published by
% the Free Software Foundation, either version 3 of the License, or
% (at your option) any later version.
%
% This program is distributed in the hope that it will be useful,
% but WITHOUT ANY WARRANTY; without even the implied warranty of
% MERCHANTABILITY or FITNESS FOR A PARTICULAR PURPOSE.  See the
% GNU General Public License for more details.
%
% You should have received a copy of the GNU General Public License
% along with this program.  If not, see <http://www.gnu.org/licenses/>.
%
%
%%%%%%%%%%%%%%%%%%%%%%%%%%%%%%%%%%%%%%%%%%%%%%%%%%%%%%%%%%%%%%%%%%%%%%%%%%%%%
% SEE pgfplots-macros.tex as well!
%\pdfminorversion=5 % to allow compression
%\pdfobjcompresslevel=2
\documentclass[a4paper,openany]{book}

\let\bookmaketitle=\maketitle

% -----------------------------------
% this here is from ltxdoc:
\usepackage{doc}[=v2]
\AtBeginDocument{\MakeShortVerb{\|}}
%\setlength{\textwidth}{355pt}
%\addtolength\marginparwidth{30pt}
%\addtolength\oddsidemargin{20pt}
%\addtolength\evensidemargin{20pt}
\def\cmd#1{\cs{\expandafter\cmd@to@cs\string#1}}
\def\cmd@to@cs#1#2{\char\number`#2\relax}
\DeclareRobustCommand\cs[1]{\texttt{\char`\\#1}}
\providecommand\marg[1]{%
  {\ttfamily\char`\{}\meta{#1}{\ttfamily\char`\}}}
\providecommand\oarg[1]{%
  {\ttfamily[}\meta{#1}{\ttfamily]}}
\providecommand\parg[1]{%
  {\ttfamily(}\meta{#1}{\ttfamily)}}
\raggedbottom

% -----------------------------------
\input{pgfplots.preamble.tex}

\makeatletter
% I want a two-column index just as in pgfmanual styles. This here
% was the best way to get one:
\def\index@prologue{\section*{Index}\addcontentsline{toc}{chapter}{Index}
}
\makeatother

%\RequirePackage[german,english,francais]{babel}

\def\matlabcolormaptext{This colormap is similar to one shipped with Matlab$^\text{\textregistered}$ under a similar name.}

\IfFileExists{tikzlibraryspy.code.tex}{%
\usetikzlibrary{spy}
}{%
    \message{ERROR: tikz SPY library NOT available. The manual will only compile partially.^^J}%
}%

\usepackage{xparse}% for colorbrewer manual

\usetikzlibrary{
    decorations.markings,
    decorations.footprints,
    shapes.arrows,
    matrix,
    positioning,
}

\usepgfplotslibrary{
    fillbetween,
    ternary,
    smithchart,
    patchplots,
    polar,
    colormaps,
    colorbrewer,
    colortol,           % docu not ready yet
}
\pgfqkeys{/codeexample}{%
    every codeexample/.append style={
        /pgfplots/every ternary axis/.append style={
            /pgfplots/legend style={fill=graphicbackground},
        }
    },
    tabsize=4,
}

\pgfplotsmanualenableexternalizationofexpensive

%\usetikzlibrary{external}
%\tikzexternalize[prefix=figures/]{pgfplots}

\title{%
    Manual for Package \PGFPlots{}\\
    {\small 2D/3D Plots in \LaTeX{}, Version \pgfplotsversion}\\
    {\small\url{https://github.com/pgf-tikz/pgfplots}}
    %\\{\small Attention: you are using an unstable development version.}
}

\makeatletter
\long\def\abstractsmuggle{%
    \centering
    \textbf{Abstract}\\[0.5cm]

    \begin{minipage}{12cm}
        \PGFPlots{} draws high-quality function plots in normal or logarithmic
        scaling with a user-friendly interface directly in \TeX{}. The user
        supplies axis labels, legend entries and the plot coordinates for one or
        more plots and \PGFPlots{} applies axis scaling, computes any logarithms
        and axis ticks and draws the plots. It supports line plots, scatter
        plots, piecewise constant plots, bar plots, area plots, mesh and surface
        plots, patch plots, contour plots, quiver plots, histogram plots, box
        plots, polar axes, ternary diagrams, smith charts and some more. It is
        based on Till Tantau's package \PGF{}/\Tikz{}.
    \end{minipage}

}%

\expandafter\date\expandafter{\@date\\[2cm]
    \abstractsmuggle
}%
\makeatother

%\includeonly{pgfplots.reference}


\begin{document}

\def\plotcoords{%
\addplot coordinates {
(5,8.312e-02)    (17,2.547e-02)   (49,7.407e-03)
(129,2.102e-03)  (321,5.874e-04)  (769,1.623e-04)
(1793,4.442e-05) (4097,1.207e-05) (9217,3.261e-06)
};

\addplot coordinates{
(7,8.472e-02)    (31,3.044e-02)    (111,1.022e-02)
(351,3.303e-03)  (1023,1.039e-03)  (2815,3.196e-04)
(7423,9.658e-05) (18943,2.873e-05) (47103,8.437e-06)
};

\addplot coordinates{
(9,7.881e-02)     (49,3.243e-02)    (209,1.232e-02)
(769,4.454e-03)   (2561,1.551e-03)  (7937,5.236e-04)
(23297,1.723e-04) (65537,5.545e-05) (178177,1.751e-05)
};

\addplot coordinates{
(11,6.887e-02)    (71,3.177e-02)     (351,1.341e-02)
(1471,5.334e-03)  (5503,2.027e-03)   (18943,7.415e-04)
(61183,2.628e-04) (187903,9.063e-05) (553983,3.053e-05)
};

\addplot coordinates{
(13,5.755e-02)     (97,2.925e-02)     (545,1.351e-02)
(2561,5.842e-03)   (10625,2.397e-03)  (40193,9.414e-04)
(141569,3.564e-04) (471041,1.308e-04)
(1496065,4.670e-05)
};
}%


\bookmaketitle
\tableofcontents
\include{pgfplots.title_abstract_intro}
\include{pgfplots.preliminaries}
\include{pgfplots.intro}
\include{pgfplots.reference}
\include{pgfplots.libs}
\include{pgfplots.resources}
\include{pgfplots.importexport}
\include{pgfplots.basic.reference}

\printindex

\bibliographystyle{abbrv} %gerapali} %gerabbrv} %gerunsrt.bst} %gerabbrv}% gerplain}
\nocite{pgfplotstable}
\nocite{programmingnotes}
\bibliography{pgfplots}
\end{document}

% -----------------------------------------------------------------------------
% For Stefan Pinnow as reminder on what to look for when editing the manual
% -----------------------------------------------------------------------------
% There should be no line breaks in the following environments
% - |...|
% - \declareandlabel{...}
% - \verbpdfref{...}
% If "MakeTikzPictures" isn't running through check one of the externalized
% LOG files what is the cause of that.
% -----------------------------------------------------------------------------

\include{pgfplots.basic.reference}

\printindex

\bibliographystyle{abbrv} %gerapali} %gerabbrv} %gerunsrt.bst} %gerabbrv}% gerplain}
\nocite{pgfplotstable}
\nocite{programmingnotes}
\bibliography{pgfplots}
\end{document}

% -----------------------------------------------------------------------------
% For Stefan Pinnow as reminder on what to look for when editing the manual
% -----------------------------------------------------------------------------
% There should be no line breaks in the following environments
% - |...|
% - \declareandlabel{...}
% - \verbpdfref{...}
% If "MakeTikzPictures" isn't running through check one of the externalized
% LOG files what is the cause of that.
% -----------------------------------------------------------------------------


%%%%%%%%%%%%%%%%%%%%%%%%%%%%%%%%%%%%%%%%%%%%%%%%%%%%%%%%%%%%%%%%%%%%%%%%%%%%%
%
% Package pgfplots.sty documentation.
%
% Copyright 2007/2008 by Christian Feuersaenger.
%
% This program is free software: you can redistribute it and/or modify
% it under the terms of the GNU General Public License as published by
% the Free Software Foundation, either version 3 of the License, or
% (at your option) any later version.
%
% This program is distributed in the hope that it will be useful,
% but WITHOUT ANY WARRANTY; without even the implied warranty of
% MERCHANTABILITY or FITNESS FOR A PARTICULAR PURPOSE.  See the
% GNU General Public License for more details.
%
% You should have received a copy of the GNU General Public License
% along with this program.  If not, see <http://www.gnu.org/licenses/>.
%
%
%%%%%%%%%%%%%%%%%%%%%%%%%%%%%%%%%%%%%%%%%%%%%%%%%%%%%%%%%%%%%%%%%%%%%%%%%%%%%
% SEE pgfplots-macros.tex as well!
%\pdfminorversion=5 % to allow compression
%\pdfobjcompresslevel=2
\documentclass[a4paper,openany]{book}

\let\bookmaketitle=\maketitle

% -----------------------------------
% this here is from ltxdoc:
\usepackage{doc}[=v2]
\AtBeginDocument{\MakeShortVerb{\|}}
%\setlength{\textwidth}{355pt}
%\addtolength\marginparwidth{30pt}
%\addtolength\oddsidemargin{20pt}
%\addtolength\evensidemargin{20pt}
\def\cmd#1{\cs{\expandafter\cmd@to@cs\string#1}}
\def\cmd@to@cs#1#2{\char\number`#2\relax}
\DeclareRobustCommand\cs[1]{\texttt{\char`\\#1}}
\providecommand\marg[1]{%
  {\ttfamily\char`\{}\meta{#1}{\ttfamily\char`\}}}
\providecommand\oarg[1]{%
  {\ttfamily[}\meta{#1}{\ttfamily]}}
\providecommand\parg[1]{%
  {\ttfamily(}\meta{#1}{\ttfamily)}}
\raggedbottom

% -----------------------------------
\input{pgfplots.preamble.tex}

\makeatletter
% I want a two-column index just as in pgfmanual styles. This here
% was the best way to get one:
\def\index@prologue{\section*{Index}\addcontentsline{toc}{chapter}{Index}
}
\makeatother

%\RequirePackage[german,english,francais]{babel}

\def\matlabcolormaptext{This colormap is similar to one shipped with Matlab$^\text{\textregistered}$ under a similar name.}

\IfFileExists{tikzlibraryspy.code.tex}{%
\usetikzlibrary{spy}
}{%
    \message{ERROR: tikz SPY library NOT available. The manual will only compile partially.^^J}%
}%

\usepackage{xparse}% for colorbrewer manual

\usetikzlibrary{
    decorations.markings,
    decorations.footprints,
    shapes.arrows,
    matrix,
    positioning,
}

\usepgfplotslibrary{
    fillbetween,
    ternary,
    smithchart,
    patchplots,
    polar,
    colormaps,
    colorbrewer,
    colortol,           % docu not ready yet
}
\pgfqkeys{/codeexample}{%
    every codeexample/.append style={
        /pgfplots/every ternary axis/.append style={
            /pgfplots/legend style={fill=graphicbackground},
        }
    },
    tabsize=4,
}

\pgfplotsmanualenableexternalizationofexpensive

%\usetikzlibrary{external}
%\tikzexternalize[prefix=figures/]{pgfplots}

\title{%
    Manual for Package \PGFPlots{}\\
    {\small 2D/3D Plots in \LaTeX{}, Version \pgfplotsversion}\\
    {\small\url{https://github.com/pgf-tikz/pgfplots}}
    %\\{\small Attention: you are using an unstable development version.}
}

\makeatletter
\long\def\abstractsmuggle{%
    \centering
    \textbf{Abstract}\\[0.5cm]

    \begin{minipage}{12cm}
        \PGFPlots{} draws high-quality function plots in normal or logarithmic
        scaling with a user-friendly interface directly in \TeX{}. The user
        supplies axis labels, legend entries and the plot coordinates for one or
        more plots and \PGFPlots{} applies axis scaling, computes any logarithms
        and axis ticks and draws the plots. It supports line plots, scatter
        plots, piecewise constant plots, bar plots, area plots, mesh and surface
        plots, patch plots, contour plots, quiver plots, histogram plots, box
        plots, polar axes, ternary diagrams, smith charts and some more. It is
        based on Till Tantau's package \PGF{}/\Tikz{}.
    \end{minipage}

}%

\expandafter\date\expandafter{\@date\\[2cm]
    \abstractsmuggle
}%
\makeatother

%\includeonly{pgfplots.reference}


\begin{document}

\def\plotcoords{%
\addplot coordinates {
(5,8.312e-02)    (17,2.547e-02)   (49,7.407e-03)
(129,2.102e-03)  (321,5.874e-04)  (769,1.623e-04)
(1793,4.442e-05) (4097,1.207e-05) (9217,3.261e-06)
};

\addplot coordinates{
(7,8.472e-02)    (31,3.044e-02)    (111,1.022e-02)
(351,3.303e-03)  (1023,1.039e-03)  (2815,3.196e-04)
(7423,9.658e-05) (18943,2.873e-05) (47103,8.437e-06)
};

\addplot coordinates{
(9,7.881e-02)     (49,3.243e-02)    (209,1.232e-02)
(769,4.454e-03)   (2561,1.551e-03)  (7937,5.236e-04)
(23297,1.723e-04) (65537,5.545e-05) (178177,1.751e-05)
};

\addplot coordinates{
(11,6.887e-02)    (71,3.177e-02)     (351,1.341e-02)
(1471,5.334e-03)  (5503,2.027e-03)   (18943,7.415e-04)
(61183,2.628e-04) (187903,9.063e-05) (553983,3.053e-05)
};

\addplot coordinates{
(13,5.755e-02)     (97,2.925e-02)     (545,1.351e-02)
(2561,5.842e-03)   (10625,2.397e-03)  (40193,9.414e-04)
(141569,3.564e-04) (471041,1.308e-04)
(1496065,4.670e-05)
};
}%


\bookmaketitle
\tableofcontents

%%%%%%%%%%%%%%%%%%%%%%%%%%%%%%%%%%%%%%%%%%%%%%%%%%%%%%%%%%%%%%%%%%%%%%%%%%%%%
%
% Package pgfplots.sty documentation.
%
% Copyright 2007/2008 by Christian Feuersaenger.
%
% This program is free software: you can redistribute it and/or modify
% it under the terms of the GNU General Public License as published by
% the Free Software Foundation, either version 3 of the License, or
% (at your option) any later version.
%
% This program is distributed in the hope that it will be useful,
% but WITHOUT ANY WARRANTY; without even the implied warranty of
% MERCHANTABILITY or FITNESS FOR A PARTICULAR PURPOSE.  See the
% GNU General Public License for more details.
%
% You should have received a copy of the GNU General Public License
% along with this program.  If not, see <http://www.gnu.org/licenses/>.
%
%
%%%%%%%%%%%%%%%%%%%%%%%%%%%%%%%%%%%%%%%%%%%%%%%%%%%%%%%%%%%%%%%%%%%%%%%%%%%%%
% SEE pgfplots-macros.tex as well!
%\pdfminorversion=5 % to allow compression
%\pdfobjcompresslevel=2
\documentclass[a4paper,openany]{book}

\let\bookmaketitle=\maketitle

% -----------------------------------
% this here is from ltxdoc:
\usepackage{doc}[=v2]
\AtBeginDocument{\MakeShortVerb{\|}}
%\setlength{\textwidth}{355pt}
%\addtolength\marginparwidth{30pt}
%\addtolength\oddsidemargin{20pt}
%\addtolength\evensidemargin{20pt}
\def\cmd#1{\cs{\expandafter\cmd@to@cs\string#1}}
\def\cmd@to@cs#1#2{\char\number`#2\relax}
\DeclareRobustCommand\cs[1]{\texttt{\char`\\#1}}
\providecommand\marg[1]{%
  {\ttfamily\char`\{}\meta{#1}{\ttfamily\char`\}}}
\providecommand\oarg[1]{%
  {\ttfamily[}\meta{#1}{\ttfamily]}}
\providecommand\parg[1]{%
  {\ttfamily(}\meta{#1}{\ttfamily)}}
\raggedbottom

% -----------------------------------
\input{pgfplots.preamble.tex}

\makeatletter
% I want a two-column index just as in pgfmanual styles. This here
% was the best way to get one:
\def\index@prologue{\section*{Index}\addcontentsline{toc}{chapter}{Index}
}
\makeatother

%\RequirePackage[german,english,francais]{babel}

\def\matlabcolormaptext{This colormap is similar to one shipped with Matlab$^\text{\textregistered}$ under a similar name.}

\IfFileExists{tikzlibraryspy.code.tex}{%
\usetikzlibrary{spy}
}{%
    \message{ERROR: tikz SPY library NOT available. The manual will only compile partially.^^J}%
}%

\usepackage{xparse}% for colorbrewer manual

\usetikzlibrary{
    decorations.markings,
    decorations.footprints,
    shapes.arrows,
    matrix,
    positioning,
}

\usepgfplotslibrary{
    fillbetween,
    ternary,
    smithchart,
    patchplots,
    polar,
    colormaps,
    colorbrewer,
    colortol,           % docu not ready yet
}
\pgfqkeys{/codeexample}{%
    every codeexample/.append style={
        /pgfplots/every ternary axis/.append style={
            /pgfplots/legend style={fill=graphicbackground},
        }
    },
    tabsize=4,
}

\pgfplotsmanualenableexternalizationofexpensive

%\usetikzlibrary{external}
%\tikzexternalize[prefix=figures/]{pgfplots}

\title{%
    Manual for Package \PGFPlots{}\\
    {\small 2D/3D Plots in \LaTeX{}, Version \pgfplotsversion}\\
    {\small\url{https://github.com/pgf-tikz/pgfplots}}
    %\\{\small Attention: you are using an unstable development version.}
}

\makeatletter
\long\def\abstractsmuggle{%
    \centering
    \textbf{Abstract}\\[0.5cm]

    \begin{minipage}{12cm}
        \PGFPlots{} draws high-quality function plots in normal or logarithmic
        scaling with a user-friendly interface directly in \TeX{}. The user
        supplies axis labels, legend entries and the plot coordinates for one or
        more plots and \PGFPlots{} applies axis scaling, computes any logarithms
        and axis ticks and draws the plots. It supports line plots, scatter
        plots, piecewise constant plots, bar plots, area plots, mesh and surface
        plots, patch plots, contour plots, quiver plots, histogram plots, box
        plots, polar axes, ternary diagrams, smith charts and some more. It is
        based on Till Tantau's package \PGF{}/\Tikz{}.
    \end{minipage}

}%

\expandafter\date\expandafter{\@date\\[2cm]
    \abstractsmuggle
}%
\makeatother

%\includeonly{pgfplots.reference}


\begin{document}

\def\plotcoords{%
\addplot coordinates {
(5,8.312e-02)    (17,2.547e-02)   (49,7.407e-03)
(129,2.102e-03)  (321,5.874e-04)  (769,1.623e-04)
(1793,4.442e-05) (4097,1.207e-05) (9217,3.261e-06)
};

\addplot coordinates{
(7,8.472e-02)    (31,3.044e-02)    (111,1.022e-02)
(351,3.303e-03)  (1023,1.039e-03)  (2815,3.196e-04)
(7423,9.658e-05) (18943,2.873e-05) (47103,8.437e-06)
};

\addplot coordinates{
(9,7.881e-02)     (49,3.243e-02)    (209,1.232e-02)
(769,4.454e-03)   (2561,1.551e-03)  (7937,5.236e-04)
(23297,1.723e-04) (65537,5.545e-05) (178177,1.751e-05)
};

\addplot coordinates{
(11,6.887e-02)    (71,3.177e-02)     (351,1.341e-02)
(1471,5.334e-03)  (5503,2.027e-03)   (18943,7.415e-04)
(61183,2.628e-04) (187903,9.063e-05) (553983,3.053e-05)
};

\addplot coordinates{
(13,5.755e-02)     (97,2.925e-02)     (545,1.351e-02)
(2561,5.842e-03)   (10625,2.397e-03)  (40193,9.414e-04)
(141569,3.564e-04) (471041,1.308e-04)
(1496065,4.670e-05)
};
}%


\bookmaketitle
\tableofcontents
\include{pgfplots.title_abstract_intro}
\include{pgfplots.preliminaries}
\include{pgfplots.intro}
\include{pgfplots.reference}
\include{pgfplots.libs}
\include{pgfplots.resources}
\include{pgfplots.importexport}
\include{pgfplots.basic.reference}

\printindex

\bibliographystyle{abbrv} %gerapali} %gerabbrv} %gerunsrt.bst} %gerabbrv}% gerplain}
\nocite{pgfplotstable}
\nocite{programmingnotes}
\bibliography{pgfplots}
\end{document}

% -----------------------------------------------------------------------------
% For Stefan Pinnow as reminder on what to look for when editing the manual
% -----------------------------------------------------------------------------
% There should be no line breaks in the following environments
% - |...|
% - \declareandlabel{...}
% - \verbpdfref{...}
% If "MakeTikzPictures" isn't running through check one of the externalized
% LOG files what is the cause of that.
% -----------------------------------------------------------------------------


%%%%%%%%%%%%%%%%%%%%%%%%%%%%%%%%%%%%%%%%%%%%%%%%%%%%%%%%%%%%%%%%%%%%%%%%%%%%%
%
% Package pgfplots.sty documentation.
%
% Copyright 2007/2008 by Christian Feuersaenger.
%
% This program is free software: you can redistribute it and/or modify
% it under the terms of the GNU General Public License as published by
% the Free Software Foundation, either version 3 of the License, or
% (at your option) any later version.
%
% This program is distributed in the hope that it will be useful,
% but WITHOUT ANY WARRANTY; without even the implied warranty of
% MERCHANTABILITY or FITNESS FOR A PARTICULAR PURPOSE.  See the
% GNU General Public License for more details.
%
% You should have received a copy of the GNU General Public License
% along with this program.  If not, see <http://www.gnu.org/licenses/>.
%
%
%%%%%%%%%%%%%%%%%%%%%%%%%%%%%%%%%%%%%%%%%%%%%%%%%%%%%%%%%%%%%%%%%%%%%%%%%%%%%
% SEE pgfplots-macros.tex as well!
%\pdfminorversion=5 % to allow compression
%\pdfobjcompresslevel=2
\documentclass[a4paper,openany]{book}

\let\bookmaketitle=\maketitle

% -----------------------------------
% this here is from ltxdoc:
\usepackage{doc}[=v2]
\AtBeginDocument{\MakeShortVerb{\|}}
%\setlength{\textwidth}{355pt}
%\addtolength\marginparwidth{30pt}
%\addtolength\oddsidemargin{20pt}
%\addtolength\evensidemargin{20pt}
\def\cmd#1{\cs{\expandafter\cmd@to@cs\string#1}}
\def\cmd@to@cs#1#2{\char\number`#2\relax}
\DeclareRobustCommand\cs[1]{\texttt{\char`\\#1}}
\providecommand\marg[1]{%
  {\ttfamily\char`\{}\meta{#1}{\ttfamily\char`\}}}
\providecommand\oarg[1]{%
  {\ttfamily[}\meta{#1}{\ttfamily]}}
\providecommand\parg[1]{%
  {\ttfamily(}\meta{#1}{\ttfamily)}}
\raggedbottom

% -----------------------------------
\input{pgfplots.preamble.tex}

\makeatletter
% I want a two-column index just as in pgfmanual styles. This here
% was the best way to get one:
\def\index@prologue{\section*{Index}\addcontentsline{toc}{chapter}{Index}
}
\makeatother

%\RequirePackage[german,english,francais]{babel}

\def\matlabcolormaptext{This colormap is similar to one shipped with Matlab$^\text{\textregistered}$ under a similar name.}

\IfFileExists{tikzlibraryspy.code.tex}{%
\usetikzlibrary{spy}
}{%
    \message{ERROR: tikz SPY library NOT available. The manual will only compile partially.^^J}%
}%

\usepackage{xparse}% for colorbrewer manual

\usetikzlibrary{
    decorations.markings,
    decorations.footprints,
    shapes.arrows,
    matrix,
    positioning,
}

\usepgfplotslibrary{
    fillbetween,
    ternary,
    smithchart,
    patchplots,
    polar,
    colormaps,
    colorbrewer,
    colortol,           % docu not ready yet
}
\pgfqkeys{/codeexample}{%
    every codeexample/.append style={
        /pgfplots/every ternary axis/.append style={
            /pgfplots/legend style={fill=graphicbackground},
        }
    },
    tabsize=4,
}

\pgfplotsmanualenableexternalizationofexpensive

%\usetikzlibrary{external}
%\tikzexternalize[prefix=figures/]{pgfplots}

\title{%
    Manual for Package \PGFPlots{}\\
    {\small 2D/3D Plots in \LaTeX{}, Version \pgfplotsversion}\\
    {\small\url{https://github.com/pgf-tikz/pgfplots}}
    %\\{\small Attention: you are using an unstable development version.}
}

\makeatletter
\long\def\abstractsmuggle{%
    \centering
    \textbf{Abstract}\\[0.5cm]

    \begin{minipage}{12cm}
        \PGFPlots{} draws high-quality function plots in normal or logarithmic
        scaling with a user-friendly interface directly in \TeX{}. The user
        supplies axis labels, legend entries and the plot coordinates for one or
        more plots and \PGFPlots{} applies axis scaling, computes any logarithms
        and axis ticks and draws the plots. It supports line plots, scatter
        plots, piecewise constant plots, bar plots, area plots, mesh and surface
        plots, patch plots, contour plots, quiver plots, histogram plots, box
        plots, polar axes, ternary diagrams, smith charts and some more. It is
        based on Till Tantau's package \PGF{}/\Tikz{}.
    \end{minipage}

}%

\expandafter\date\expandafter{\@date\\[2cm]
    \abstractsmuggle
}%
\makeatother

%\includeonly{pgfplots.reference}


\begin{document}

\def\plotcoords{%
\addplot coordinates {
(5,8.312e-02)    (17,2.547e-02)   (49,7.407e-03)
(129,2.102e-03)  (321,5.874e-04)  (769,1.623e-04)
(1793,4.442e-05) (4097,1.207e-05) (9217,3.261e-06)
};

\addplot coordinates{
(7,8.472e-02)    (31,3.044e-02)    (111,1.022e-02)
(351,3.303e-03)  (1023,1.039e-03)  (2815,3.196e-04)
(7423,9.658e-05) (18943,2.873e-05) (47103,8.437e-06)
};

\addplot coordinates{
(9,7.881e-02)     (49,3.243e-02)    (209,1.232e-02)
(769,4.454e-03)   (2561,1.551e-03)  (7937,5.236e-04)
(23297,1.723e-04) (65537,5.545e-05) (178177,1.751e-05)
};

\addplot coordinates{
(11,6.887e-02)    (71,3.177e-02)     (351,1.341e-02)
(1471,5.334e-03)  (5503,2.027e-03)   (18943,7.415e-04)
(61183,2.628e-04) (187903,9.063e-05) (553983,3.053e-05)
};

\addplot coordinates{
(13,5.755e-02)     (97,2.925e-02)     (545,1.351e-02)
(2561,5.842e-03)   (10625,2.397e-03)  (40193,9.414e-04)
(141569,3.564e-04) (471041,1.308e-04)
(1496065,4.670e-05)
};
}%


\bookmaketitle
\tableofcontents
\include{pgfplots.title_abstract_intro}
\include{pgfplots.preliminaries}
\include{pgfplots.intro}
\include{pgfplots.reference}
\include{pgfplots.libs}
\include{pgfplots.resources}
\include{pgfplots.importexport}
\include{pgfplots.basic.reference}

\printindex

\bibliographystyle{abbrv} %gerapali} %gerabbrv} %gerunsrt.bst} %gerabbrv}% gerplain}
\nocite{pgfplotstable}
\nocite{programmingnotes}
\bibliography{pgfplots}
\end{document}

% -----------------------------------------------------------------------------
% For Stefan Pinnow as reminder on what to look for when editing the manual
% -----------------------------------------------------------------------------
% There should be no line breaks in the following environments
% - |...|
% - \declareandlabel{...}
% - \verbpdfref{...}
% If "MakeTikzPictures" isn't running through check one of the externalized
% LOG files what is the cause of that.
% -----------------------------------------------------------------------------


%%%%%%%%%%%%%%%%%%%%%%%%%%%%%%%%%%%%%%%%%%%%%%%%%%%%%%%%%%%%%%%%%%%%%%%%%%%%%
%
% Package pgfplots.sty documentation.
%
% Copyright 2007/2008 by Christian Feuersaenger.
%
% This program is free software: you can redistribute it and/or modify
% it under the terms of the GNU General Public License as published by
% the Free Software Foundation, either version 3 of the License, or
% (at your option) any later version.
%
% This program is distributed in the hope that it will be useful,
% but WITHOUT ANY WARRANTY; without even the implied warranty of
% MERCHANTABILITY or FITNESS FOR A PARTICULAR PURPOSE.  See the
% GNU General Public License for more details.
%
% You should have received a copy of the GNU General Public License
% along with this program.  If not, see <http://www.gnu.org/licenses/>.
%
%
%%%%%%%%%%%%%%%%%%%%%%%%%%%%%%%%%%%%%%%%%%%%%%%%%%%%%%%%%%%%%%%%%%%%%%%%%%%%%
% SEE pgfplots-macros.tex as well!
%\pdfminorversion=5 % to allow compression
%\pdfobjcompresslevel=2
\documentclass[a4paper,openany]{book}

\let\bookmaketitle=\maketitle

% -----------------------------------
% this here is from ltxdoc:
\usepackage{doc}[=v2]
\AtBeginDocument{\MakeShortVerb{\|}}
%\setlength{\textwidth}{355pt}
%\addtolength\marginparwidth{30pt}
%\addtolength\oddsidemargin{20pt}
%\addtolength\evensidemargin{20pt}
\def\cmd#1{\cs{\expandafter\cmd@to@cs\string#1}}
\def\cmd@to@cs#1#2{\char\number`#2\relax}
\DeclareRobustCommand\cs[1]{\texttt{\char`\\#1}}
\providecommand\marg[1]{%
  {\ttfamily\char`\{}\meta{#1}{\ttfamily\char`\}}}
\providecommand\oarg[1]{%
  {\ttfamily[}\meta{#1}{\ttfamily]}}
\providecommand\parg[1]{%
  {\ttfamily(}\meta{#1}{\ttfamily)}}
\raggedbottom

% -----------------------------------
\input{pgfplots.preamble.tex}

\makeatletter
% I want a two-column index just as in pgfmanual styles. This here
% was the best way to get one:
\def\index@prologue{\section*{Index}\addcontentsline{toc}{chapter}{Index}
}
\makeatother

%\RequirePackage[german,english,francais]{babel}

\def\matlabcolormaptext{This colormap is similar to one shipped with Matlab$^\text{\textregistered}$ under a similar name.}

\IfFileExists{tikzlibraryspy.code.tex}{%
\usetikzlibrary{spy}
}{%
    \message{ERROR: tikz SPY library NOT available. The manual will only compile partially.^^J}%
}%

\usepackage{xparse}% for colorbrewer manual

\usetikzlibrary{
    decorations.markings,
    decorations.footprints,
    shapes.arrows,
    matrix,
    positioning,
}

\usepgfplotslibrary{
    fillbetween,
    ternary,
    smithchart,
    patchplots,
    polar,
    colormaps,
    colorbrewer,
    colortol,           % docu not ready yet
}
\pgfqkeys{/codeexample}{%
    every codeexample/.append style={
        /pgfplots/every ternary axis/.append style={
            /pgfplots/legend style={fill=graphicbackground},
        }
    },
    tabsize=4,
}

\pgfplotsmanualenableexternalizationofexpensive

%\usetikzlibrary{external}
%\tikzexternalize[prefix=figures/]{pgfplots}

\title{%
    Manual for Package \PGFPlots{}\\
    {\small 2D/3D Plots in \LaTeX{}, Version \pgfplotsversion}\\
    {\small\url{https://github.com/pgf-tikz/pgfplots}}
    %\\{\small Attention: you are using an unstable development version.}
}

\makeatletter
\long\def\abstractsmuggle{%
    \centering
    \textbf{Abstract}\\[0.5cm]

    \begin{minipage}{12cm}
        \PGFPlots{} draws high-quality function plots in normal or logarithmic
        scaling with a user-friendly interface directly in \TeX{}. The user
        supplies axis labels, legend entries and the plot coordinates for one or
        more plots and \PGFPlots{} applies axis scaling, computes any logarithms
        and axis ticks and draws the plots. It supports line plots, scatter
        plots, piecewise constant plots, bar plots, area plots, mesh and surface
        plots, patch plots, contour plots, quiver plots, histogram plots, box
        plots, polar axes, ternary diagrams, smith charts and some more. It is
        based on Till Tantau's package \PGF{}/\Tikz{}.
    \end{minipage}

}%

\expandafter\date\expandafter{\@date\\[2cm]
    \abstractsmuggle
}%
\makeatother

%\includeonly{pgfplots.reference}


\begin{document}

\def\plotcoords{%
\addplot coordinates {
(5,8.312e-02)    (17,2.547e-02)   (49,7.407e-03)
(129,2.102e-03)  (321,5.874e-04)  (769,1.623e-04)
(1793,4.442e-05) (4097,1.207e-05) (9217,3.261e-06)
};

\addplot coordinates{
(7,8.472e-02)    (31,3.044e-02)    (111,1.022e-02)
(351,3.303e-03)  (1023,1.039e-03)  (2815,3.196e-04)
(7423,9.658e-05) (18943,2.873e-05) (47103,8.437e-06)
};

\addplot coordinates{
(9,7.881e-02)     (49,3.243e-02)    (209,1.232e-02)
(769,4.454e-03)   (2561,1.551e-03)  (7937,5.236e-04)
(23297,1.723e-04) (65537,5.545e-05) (178177,1.751e-05)
};

\addplot coordinates{
(11,6.887e-02)    (71,3.177e-02)     (351,1.341e-02)
(1471,5.334e-03)  (5503,2.027e-03)   (18943,7.415e-04)
(61183,2.628e-04) (187903,9.063e-05) (553983,3.053e-05)
};

\addplot coordinates{
(13,5.755e-02)     (97,2.925e-02)     (545,1.351e-02)
(2561,5.842e-03)   (10625,2.397e-03)  (40193,9.414e-04)
(141569,3.564e-04) (471041,1.308e-04)
(1496065,4.670e-05)
};
}%


\bookmaketitle
\tableofcontents
\include{pgfplots.title_abstract_intro}
\include{pgfplots.preliminaries}
\include{pgfplots.intro}
\include{pgfplots.reference}
\include{pgfplots.libs}
\include{pgfplots.resources}
\include{pgfplots.importexport}
\include{pgfplots.basic.reference}

\printindex

\bibliographystyle{abbrv} %gerapali} %gerabbrv} %gerunsrt.bst} %gerabbrv}% gerplain}
\nocite{pgfplotstable}
\nocite{programmingnotes}
\bibliography{pgfplots}
\end{document}

% -----------------------------------------------------------------------------
% For Stefan Pinnow as reminder on what to look for when editing the manual
% -----------------------------------------------------------------------------
% There should be no line breaks in the following environments
% - |...|
% - \declareandlabel{...}
% - \verbpdfref{...}
% If "MakeTikzPictures" isn't running through check one of the externalized
% LOG files what is the cause of that.
% -----------------------------------------------------------------------------


%%%%%%%%%%%%%%%%%%%%%%%%%%%%%%%%%%%%%%%%%%%%%%%%%%%%%%%%%%%%%%%%%%%%%%%%%%%%%
%
% Package pgfplots.sty documentation.
%
% Copyright 2007/2008 by Christian Feuersaenger.
%
% This program is free software: you can redistribute it and/or modify
% it under the terms of the GNU General Public License as published by
% the Free Software Foundation, either version 3 of the License, or
% (at your option) any later version.
%
% This program is distributed in the hope that it will be useful,
% but WITHOUT ANY WARRANTY; without even the implied warranty of
% MERCHANTABILITY or FITNESS FOR A PARTICULAR PURPOSE.  See the
% GNU General Public License for more details.
%
% You should have received a copy of the GNU General Public License
% along with this program.  If not, see <http://www.gnu.org/licenses/>.
%
%
%%%%%%%%%%%%%%%%%%%%%%%%%%%%%%%%%%%%%%%%%%%%%%%%%%%%%%%%%%%%%%%%%%%%%%%%%%%%%
% SEE pgfplots-macros.tex as well!
%\pdfminorversion=5 % to allow compression
%\pdfobjcompresslevel=2
\documentclass[a4paper,openany]{book}

\let\bookmaketitle=\maketitle

% -----------------------------------
% this here is from ltxdoc:
\usepackage{doc}[=v2]
\AtBeginDocument{\MakeShortVerb{\|}}
%\setlength{\textwidth}{355pt}
%\addtolength\marginparwidth{30pt}
%\addtolength\oddsidemargin{20pt}
%\addtolength\evensidemargin{20pt}
\def\cmd#1{\cs{\expandafter\cmd@to@cs\string#1}}
\def\cmd@to@cs#1#2{\char\number`#2\relax}
\DeclareRobustCommand\cs[1]{\texttt{\char`\\#1}}
\providecommand\marg[1]{%
  {\ttfamily\char`\{}\meta{#1}{\ttfamily\char`\}}}
\providecommand\oarg[1]{%
  {\ttfamily[}\meta{#1}{\ttfamily]}}
\providecommand\parg[1]{%
  {\ttfamily(}\meta{#1}{\ttfamily)}}
\raggedbottom

% -----------------------------------
\input{pgfplots.preamble.tex}

\makeatletter
% I want a two-column index just as in pgfmanual styles. This here
% was the best way to get one:
\def\index@prologue{\section*{Index}\addcontentsline{toc}{chapter}{Index}
}
\makeatother

%\RequirePackage[german,english,francais]{babel}

\def\matlabcolormaptext{This colormap is similar to one shipped with Matlab$^\text{\textregistered}$ under a similar name.}

\IfFileExists{tikzlibraryspy.code.tex}{%
\usetikzlibrary{spy}
}{%
    \message{ERROR: tikz SPY library NOT available. The manual will only compile partially.^^J}%
}%

\usepackage{xparse}% for colorbrewer manual

\usetikzlibrary{
    decorations.markings,
    decorations.footprints,
    shapes.arrows,
    matrix,
    positioning,
}

\usepgfplotslibrary{
    fillbetween,
    ternary,
    smithchart,
    patchplots,
    polar,
    colormaps,
    colorbrewer,
    colortol,           % docu not ready yet
}
\pgfqkeys{/codeexample}{%
    every codeexample/.append style={
        /pgfplots/every ternary axis/.append style={
            /pgfplots/legend style={fill=graphicbackground},
        }
    },
    tabsize=4,
}

\pgfplotsmanualenableexternalizationofexpensive

%\usetikzlibrary{external}
%\tikzexternalize[prefix=figures/]{pgfplots}

\title{%
    Manual for Package \PGFPlots{}\\
    {\small 2D/3D Plots in \LaTeX{}, Version \pgfplotsversion}\\
    {\small\url{https://github.com/pgf-tikz/pgfplots}}
    %\\{\small Attention: you are using an unstable development version.}
}

\makeatletter
\long\def\abstractsmuggle{%
    \centering
    \textbf{Abstract}\\[0.5cm]

    \begin{minipage}{12cm}
        \PGFPlots{} draws high-quality function plots in normal or logarithmic
        scaling with a user-friendly interface directly in \TeX{}. The user
        supplies axis labels, legend entries and the plot coordinates for one or
        more plots and \PGFPlots{} applies axis scaling, computes any logarithms
        and axis ticks and draws the plots. It supports line plots, scatter
        plots, piecewise constant plots, bar plots, area plots, mesh and surface
        plots, patch plots, contour plots, quiver plots, histogram plots, box
        plots, polar axes, ternary diagrams, smith charts and some more. It is
        based on Till Tantau's package \PGF{}/\Tikz{}.
    \end{minipage}

}%

\expandafter\date\expandafter{\@date\\[2cm]
    \abstractsmuggle
}%
\makeatother

%\includeonly{pgfplots.reference}


\begin{document}

\def\plotcoords{%
\addplot coordinates {
(5,8.312e-02)    (17,2.547e-02)   (49,7.407e-03)
(129,2.102e-03)  (321,5.874e-04)  (769,1.623e-04)
(1793,4.442e-05) (4097,1.207e-05) (9217,3.261e-06)
};

\addplot coordinates{
(7,8.472e-02)    (31,3.044e-02)    (111,1.022e-02)
(351,3.303e-03)  (1023,1.039e-03)  (2815,3.196e-04)
(7423,9.658e-05) (18943,2.873e-05) (47103,8.437e-06)
};

\addplot coordinates{
(9,7.881e-02)     (49,3.243e-02)    (209,1.232e-02)
(769,4.454e-03)   (2561,1.551e-03)  (7937,5.236e-04)
(23297,1.723e-04) (65537,5.545e-05) (178177,1.751e-05)
};

\addplot coordinates{
(11,6.887e-02)    (71,3.177e-02)     (351,1.341e-02)
(1471,5.334e-03)  (5503,2.027e-03)   (18943,7.415e-04)
(61183,2.628e-04) (187903,9.063e-05) (553983,3.053e-05)
};

\addplot coordinates{
(13,5.755e-02)     (97,2.925e-02)     (545,1.351e-02)
(2561,5.842e-03)   (10625,2.397e-03)  (40193,9.414e-04)
(141569,3.564e-04) (471041,1.308e-04)
(1496065,4.670e-05)
};
}%


\bookmaketitle
\tableofcontents
\include{pgfplots.title_abstract_intro}
\include{pgfplots.preliminaries}
\include{pgfplots.intro}
\include{pgfplots.reference}
\include{pgfplots.libs}
\include{pgfplots.resources}
\include{pgfplots.importexport}
\include{pgfplots.basic.reference}

\printindex

\bibliographystyle{abbrv} %gerapali} %gerabbrv} %gerunsrt.bst} %gerabbrv}% gerplain}
\nocite{pgfplotstable}
\nocite{programmingnotes}
\bibliography{pgfplots}
\end{document}

% -----------------------------------------------------------------------------
% For Stefan Pinnow as reminder on what to look for when editing the manual
% -----------------------------------------------------------------------------
% There should be no line breaks in the following environments
% - |...|
% - \declareandlabel{...}
% - \verbpdfref{...}
% If "MakeTikzPictures" isn't running through check one of the externalized
% LOG files what is the cause of that.
% -----------------------------------------------------------------------------


%%%%%%%%%%%%%%%%%%%%%%%%%%%%%%%%%%%%%%%%%%%%%%%%%%%%%%%%%%%%%%%%%%%%%%%%%%%%%
%
% Package pgfplots.sty documentation.
%
% Copyright 2007/2008 by Christian Feuersaenger.
%
% This program is free software: you can redistribute it and/or modify
% it under the terms of the GNU General Public License as published by
% the Free Software Foundation, either version 3 of the License, or
% (at your option) any later version.
%
% This program is distributed in the hope that it will be useful,
% but WITHOUT ANY WARRANTY; without even the implied warranty of
% MERCHANTABILITY or FITNESS FOR A PARTICULAR PURPOSE.  See the
% GNU General Public License for more details.
%
% You should have received a copy of the GNU General Public License
% along with this program.  If not, see <http://www.gnu.org/licenses/>.
%
%
%%%%%%%%%%%%%%%%%%%%%%%%%%%%%%%%%%%%%%%%%%%%%%%%%%%%%%%%%%%%%%%%%%%%%%%%%%%%%
% SEE pgfplots-macros.tex as well!
%\pdfminorversion=5 % to allow compression
%\pdfobjcompresslevel=2
\documentclass[a4paper,openany]{book}

\let\bookmaketitle=\maketitle

% -----------------------------------
% this here is from ltxdoc:
\usepackage{doc}[=v2]
\AtBeginDocument{\MakeShortVerb{\|}}
%\setlength{\textwidth}{355pt}
%\addtolength\marginparwidth{30pt}
%\addtolength\oddsidemargin{20pt}
%\addtolength\evensidemargin{20pt}
\def\cmd#1{\cs{\expandafter\cmd@to@cs\string#1}}
\def\cmd@to@cs#1#2{\char\number`#2\relax}
\DeclareRobustCommand\cs[1]{\texttt{\char`\\#1}}
\providecommand\marg[1]{%
  {\ttfamily\char`\{}\meta{#1}{\ttfamily\char`\}}}
\providecommand\oarg[1]{%
  {\ttfamily[}\meta{#1}{\ttfamily]}}
\providecommand\parg[1]{%
  {\ttfamily(}\meta{#1}{\ttfamily)}}
\raggedbottom

% -----------------------------------
\input{pgfplots.preamble.tex}

\makeatletter
% I want a two-column index just as in pgfmanual styles. This here
% was the best way to get one:
\def\index@prologue{\section*{Index}\addcontentsline{toc}{chapter}{Index}
}
\makeatother

%\RequirePackage[german,english,francais]{babel}

\def\matlabcolormaptext{This colormap is similar to one shipped with Matlab$^\text{\textregistered}$ under a similar name.}

\IfFileExists{tikzlibraryspy.code.tex}{%
\usetikzlibrary{spy}
}{%
    \message{ERROR: tikz SPY library NOT available. The manual will only compile partially.^^J}%
}%

\usepackage{xparse}% for colorbrewer manual

\usetikzlibrary{
    decorations.markings,
    decorations.footprints,
    shapes.arrows,
    matrix,
    positioning,
}

\usepgfplotslibrary{
    fillbetween,
    ternary,
    smithchart,
    patchplots,
    polar,
    colormaps,
    colorbrewer,
    colortol,           % docu not ready yet
}
\pgfqkeys{/codeexample}{%
    every codeexample/.append style={
        /pgfplots/every ternary axis/.append style={
            /pgfplots/legend style={fill=graphicbackground},
        }
    },
    tabsize=4,
}

\pgfplotsmanualenableexternalizationofexpensive

%\usetikzlibrary{external}
%\tikzexternalize[prefix=figures/]{pgfplots}

\title{%
    Manual for Package \PGFPlots{}\\
    {\small 2D/3D Plots in \LaTeX{}, Version \pgfplotsversion}\\
    {\small\url{https://github.com/pgf-tikz/pgfplots}}
    %\\{\small Attention: you are using an unstable development version.}
}

\makeatletter
\long\def\abstractsmuggle{%
    \centering
    \textbf{Abstract}\\[0.5cm]

    \begin{minipage}{12cm}
        \PGFPlots{} draws high-quality function plots in normal or logarithmic
        scaling with a user-friendly interface directly in \TeX{}. The user
        supplies axis labels, legend entries and the plot coordinates for one or
        more plots and \PGFPlots{} applies axis scaling, computes any logarithms
        and axis ticks and draws the plots. It supports line plots, scatter
        plots, piecewise constant plots, bar plots, area plots, mesh and surface
        plots, patch plots, contour plots, quiver plots, histogram plots, box
        plots, polar axes, ternary diagrams, smith charts and some more. It is
        based on Till Tantau's package \PGF{}/\Tikz{}.
    \end{minipage}

}%

\expandafter\date\expandafter{\@date\\[2cm]
    \abstractsmuggle
}%
\makeatother

%\includeonly{pgfplots.reference}


\begin{document}

\def\plotcoords{%
\addplot coordinates {
(5,8.312e-02)    (17,2.547e-02)   (49,7.407e-03)
(129,2.102e-03)  (321,5.874e-04)  (769,1.623e-04)
(1793,4.442e-05) (4097,1.207e-05) (9217,3.261e-06)
};

\addplot coordinates{
(7,8.472e-02)    (31,3.044e-02)    (111,1.022e-02)
(351,3.303e-03)  (1023,1.039e-03)  (2815,3.196e-04)
(7423,9.658e-05) (18943,2.873e-05) (47103,8.437e-06)
};

\addplot coordinates{
(9,7.881e-02)     (49,3.243e-02)    (209,1.232e-02)
(769,4.454e-03)   (2561,1.551e-03)  (7937,5.236e-04)
(23297,1.723e-04) (65537,5.545e-05) (178177,1.751e-05)
};

\addplot coordinates{
(11,6.887e-02)    (71,3.177e-02)     (351,1.341e-02)
(1471,5.334e-03)  (5503,2.027e-03)   (18943,7.415e-04)
(61183,2.628e-04) (187903,9.063e-05) (553983,3.053e-05)
};

\addplot coordinates{
(13,5.755e-02)     (97,2.925e-02)     (545,1.351e-02)
(2561,5.842e-03)   (10625,2.397e-03)  (40193,9.414e-04)
(141569,3.564e-04) (471041,1.308e-04)
(1496065,4.670e-05)
};
}%


\bookmaketitle
\tableofcontents
\include{pgfplots.title_abstract_intro}
\include{pgfplots.preliminaries}
\include{pgfplots.intro}
\include{pgfplots.reference}
\include{pgfplots.libs}
\include{pgfplots.resources}
\include{pgfplots.importexport}
\include{pgfplots.basic.reference}

\printindex

\bibliographystyle{abbrv} %gerapali} %gerabbrv} %gerunsrt.bst} %gerabbrv}% gerplain}
\nocite{pgfplotstable}
\nocite{programmingnotes}
\bibliography{pgfplots}
\end{document}

% -----------------------------------------------------------------------------
% For Stefan Pinnow as reminder on what to look for when editing the manual
% -----------------------------------------------------------------------------
% There should be no line breaks in the following environments
% - |...|
% - \declareandlabel{...}
% - \verbpdfref{...}
% If "MakeTikzPictures" isn't running through check one of the externalized
% LOG files what is the cause of that.
% -----------------------------------------------------------------------------


%%%%%%%%%%%%%%%%%%%%%%%%%%%%%%%%%%%%%%%%%%%%%%%%%%%%%%%%%%%%%%%%%%%%%%%%%%%%%
%
% Package pgfplots.sty documentation.
%
% Copyright 2007/2008 by Christian Feuersaenger.
%
% This program is free software: you can redistribute it and/or modify
% it under the terms of the GNU General Public License as published by
% the Free Software Foundation, either version 3 of the License, or
% (at your option) any later version.
%
% This program is distributed in the hope that it will be useful,
% but WITHOUT ANY WARRANTY; without even the implied warranty of
% MERCHANTABILITY or FITNESS FOR A PARTICULAR PURPOSE.  See the
% GNU General Public License for more details.
%
% You should have received a copy of the GNU General Public License
% along with this program.  If not, see <http://www.gnu.org/licenses/>.
%
%
%%%%%%%%%%%%%%%%%%%%%%%%%%%%%%%%%%%%%%%%%%%%%%%%%%%%%%%%%%%%%%%%%%%%%%%%%%%%%
% SEE pgfplots-macros.tex as well!
%\pdfminorversion=5 % to allow compression
%\pdfobjcompresslevel=2
\documentclass[a4paper,openany]{book}

\let\bookmaketitle=\maketitle

% -----------------------------------
% this here is from ltxdoc:
\usepackage{doc}[=v2]
\AtBeginDocument{\MakeShortVerb{\|}}
%\setlength{\textwidth}{355pt}
%\addtolength\marginparwidth{30pt}
%\addtolength\oddsidemargin{20pt}
%\addtolength\evensidemargin{20pt}
\def\cmd#1{\cs{\expandafter\cmd@to@cs\string#1}}
\def\cmd@to@cs#1#2{\char\number`#2\relax}
\DeclareRobustCommand\cs[1]{\texttt{\char`\\#1}}
\providecommand\marg[1]{%
  {\ttfamily\char`\{}\meta{#1}{\ttfamily\char`\}}}
\providecommand\oarg[1]{%
  {\ttfamily[}\meta{#1}{\ttfamily]}}
\providecommand\parg[1]{%
  {\ttfamily(}\meta{#1}{\ttfamily)}}
\raggedbottom

% -----------------------------------
\input{pgfplots.preamble.tex}

\makeatletter
% I want a two-column index just as in pgfmanual styles. This here
% was the best way to get one:
\def\index@prologue{\section*{Index}\addcontentsline{toc}{chapter}{Index}
}
\makeatother

%\RequirePackage[german,english,francais]{babel}

\def\matlabcolormaptext{This colormap is similar to one shipped with Matlab$^\text{\textregistered}$ under a similar name.}

\IfFileExists{tikzlibraryspy.code.tex}{%
\usetikzlibrary{spy}
}{%
    \message{ERROR: tikz SPY library NOT available. The manual will only compile partially.^^J}%
}%

\usepackage{xparse}% for colorbrewer manual

\usetikzlibrary{
    decorations.markings,
    decorations.footprints,
    shapes.arrows,
    matrix,
    positioning,
}

\usepgfplotslibrary{
    fillbetween,
    ternary,
    smithchart,
    patchplots,
    polar,
    colormaps,
    colorbrewer,
    colortol,           % docu not ready yet
}
\pgfqkeys{/codeexample}{%
    every codeexample/.append style={
        /pgfplots/every ternary axis/.append style={
            /pgfplots/legend style={fill=graphicbackground},
        }
    },
    tabsize=4,
}

\pgfplotsmanualenableexternalizationofexpensive

%\usetikzlibrary{external}
%\tikzexternalize[prefix=figures/]{pgfplots}

\title{%
    Manual for Package \PGFPlots{}\\
    {\small 2D/3D Plots in \LaTeX{}, Version \pgfplotsversion}\\
    {\small\url{https://github.com/pgf-tikz/pgfplots}}
    %\\{\small Attention: you are using an unstable development version.}
}

\makeatletter
\long\def\abstractsmuggle{%
    \centering
    \textbf{Abstract}\\[0.5cm]

    \begin{minipage}{12cm}
        \PGFPlots{} draws high-quality function plots in normal or logarithmic
        scaling with a user-friendly interface directly in \TeX{}. The user
        supplies axis labels, legend entries and the plot coordinates for one or
        more plots and \PGFPlots{} applies axis scaling, computes any logarithms
        and axis ticks and draws the plots. It supports line plots, scatter
        plots, piecewise constant plots, bar plots, area plots, mesh and surface
        plots, patch plots, contour plots, quiver plots, histogram plots, box
        plots, polar axes, ternary diagrams, smith charts and some more. It is
        based on Till Tantau's package \PGF{}/\Tikz{}.
    \end{minipage}

}%

\expandafter\date\expandafter{\@date\\[2cm]
    \abstractsmuggle
}%
\makeatother

%\includeonly{pgfplots.reference}


\begin{document}

\def\plotcoords{%
\addplot coordinates {
(5,8.312e-02)    (17,2.547e-02)   (49,7.407e-03)
(129,2.102e-03)  (321,5.874e-04)  (769,1.623e-04)
(1793,4.442e-05) (4097,1.207e-05) (9217,3.261e-06)
};

\addplot coordinates{
(7,8.472e-02)    (31,3.044e-02)    (111,1.022e-02)
(351,3.303e-03)  (1023,1.039e-03)  (2815,3.196e-04)
(7423,9.658e-05) (18943,2.873e-05) (47103,8.437e-06)
};

\addplot coordinates{
(9,7.881e-02)     (49,3.243e-02)    (209,1.232e-02)
(769,4.454e-03)   (2561,1.551e-03)  (7937,5.236e-04)
(23297,1.723e-04) (65537,5.545e-05) (178177,1.751e-05)
};

\addplot coordinates{
(11,6.887e-02)    (71,3.177e-02)     (351,1.341e-02)
(1471,5.334e-03)  (5503,2.027e-03)   (18943,7.415e-04)
(61183,2.628e-04) (187903,9.063e-05) (553983,3.053e-05)
};

\addplot coordinates{
(13,5.755e-02)     (97,2.925e-02)     (545,1.351e-02)
(2561,5.842e-03)   (10625,2.397e-03)  (40193,9.414e-04)
(141569,3.564e-04) (471041,1.308e-04)
(1496065,4.670e-05)
};
}%


\bookmaketitle
\tableofcontents
\include{pgfplots.title_abstract_intro}
\include{pgfplots.preliminaries}
\include{pgfplots.intro}
\include{pgfplots.reference}
\include{pgfplots.libs}
\include{pgfplots.resources}
\include{pgfplots.importexport}
\include{pgfplots.basic.reference}

\printindex

\bibliographystyle{abbrv} %gerapali} %gerabbrv} %gerunsrt.bst} %gerabbrv}% gerplain}
\nocite{pgfplotstable}
\nocite{programmingnotes}
\bibliography{pgfplots}
\end{document}

% -----------------------------------------------------------------------------
% For Stefan Pinnow as reminder on what to look for when editing the manual
% -----------------------------------------------------------------------------
% There should be no line breaks in the following environments
% - |...|
% - \declareandlabel{...}
% - \verbpdfref{...}
% If "MakeTikzPictures" isn't running through check one of the externalized
% LOG files what is the cause of that.
% -----------------------------------------------------------------------------


%%%%%%%%%%%%%%%%%%%%%%%%%%%%%%%%%%%%%%%%%%%%%%%%%%%%%%%%%%%%%%%%%%%%%%%%%%%%%
%
% Package pgfplots.sty documentation.
%
% Copyright 2007/2008 by Christian Feuersaenger.
%
% This program is free software: you can redistribute it and/or modify
% it under the terms of the GNU General Public License as published by
% the Free Software Foundation, either version 3 of the License, or
% (at your option) any later version.
%
% This program is distributed in the hope that it will be useful,
% but WITHOUT ANY WARRANTY; without even the implied warranty of
% MERCHANTABILITY or FITNESS FOR A PARTICULAR PURPOSE.  See the
% GNU General Public License for more details.
%
% You should have received a copy of the GNU General Public License
% along with this program.  If not, see <http://www.gnu.org/licenses/>.
%
%
%%%%%%%%%%%%%%%%%%%%%%%%%%%%%%%%%%%%%%%%%%%%%%%%%%%%%%%%%%%%%%%%%%%%%%%%%%%%%
% SEE pgfplots-macros.tex as well!
%\pdfminorversion=5 % to allow compression
%\pdfobjcompresslevel=2
\documentclass[a4paper,openany]{book}

\let\bookmaketitle=\maketitle

% -----------------------------------
% this here is from ltxdoc:
\usepackage{doc}[=v2]
\AtBeginDocument{\MakeShortVerb{\|}}
%\setlength{\textwidth}{355pt}
%\addtolength\marginparwidth{30pt}
%\addtolength\oddsidemargin{20pt}
%\addtolength\evensidemargin{20pt}
\def\cmd#1{\cs{\expandafter\cmd@to@cs\string#1}}
\def\cmd@to@cs#1#2{\char\number`#2\relax}
\DeclareRobustCommand\cs[1]{\texttt{\char`\\#1}}
\providecommand\marg[1]{%
  {\ttfamily\char`\{}\meta{#1}{\ttfamily\char`\}}}
\providecommand\oarg[1]{%
  {\ttfamily[}\meta{#1}{\ttfamily]}}
\providecommand\parg[1]{%
  {\ttfamily(}\meta{#1}{\ttfamily)}}
\raggedbottom

% -----------------------------------
\input{pgfplots.preamble.tex}

\makeatletter
% I want a two-column index just as in pgfmanual styles. This here
% was the best way to get one:
\def\index@prologue{\section*{Index}\addcontentsline{toc}{chapter}{Index}
}
\makeatother

%\RequirePackage[german,english,francais]{babel}

\def\matlabcolormaptext{This colormap is similar to one shipped with Matlab$^\text{\textregistered}$ under a similar name.}

\IfFileExists{tikzlibraryspy.code.tex}{%
\usetikzlibrary{spy}
}{%
    \message{ERROR: tikz SPY library NOT available. The manual will only compile partially.^^J}%
}%

\usepackage{xparse}% for colorbrewer manual

\usetikzlibrary{
    decorations.markings,
    decorations.footprints,
    shapes.arrows,
    matrix,
    positioning,
}

\usepgfplotslibrary{
    fillbetween,
    ternary,
    smithchart,
    patchplots,
    polar,
    colormaps,
    colorbrewer,
    colortol,           % docu not ready yet
}
\pgfqkeys{/codeexample}{%
    every codeexample/.append style={
        /pgfplots/every ternary axis/.append style={
            /pgfplots/legend style={fill=graphicbackground},
        }
    },
    tabsize=4,
}

\pgfplotsmanualenableexternalizationofexpensive

%\usetikzlibrary{external}
%\tikzexternalize[prefix=figures/]{pgfplots}

\title{%
    Manual for Package \PGFPlots{}\\
    {\small 2D/3D Plots in \LaTeX{}, Version \pgfplotsversion}\\
    {\small\url{https://github.com/pgf-tikz/pgfplots}}
    %\\{\small Attention: you are using an unstable development version.}
}

\makeatletter
\long\def\abstractsmuggle{%
    \centering
    \textbf{Abstract}\\[0.5cm]

    \begin{minipage}{12cm}
        \PGFPlots{} draws high-quality function plots in normal or logarithmic
        scaling with a user-friendly interface directly in \TeX{}. The user
        supplies axis labels, legend entries and the plot coordinates for one or
        more plots and \PGFPlots{} applies axis scaling, computes any logarithms
        and axis ticks and draws the plots. It supports line plots, scatter
        plots, piecewise constant plots, bar plots, area plots, mesh and surface
        plots, patch plots, contour plots, quiver plots, histogram plots, box
        plots, polar axes, ternary diagrams, smith charts and some more. It is
        based on Till Tantau's package \PGF{}/\Tikz{}.
    \end{minipage}

}%

\expandafter\date\expandafter{\@date\\[2cm]
    \abstractsmuggle
}%
\makeatother

%\includeonly{pgfplots.reference}


\begin{document}

\def\plotcoords{%
\addplot coordinates {
(5,8.312e-02)    (17,2.547e-02)   (49,7.407e-03)
(129,2.102e-03)  (321,5.874e-04)  (769,1.623e-04)
(1793,4.442e-05) (4097,1.207e-05) (9217,3.261e-06)
};

\addplot coordinates{
(7,8.472e-02)    (31,3.044e-02)    (111,1.022e-02)
(351,3.303e-03)  (1023,1.039e-03)  (2815,3.196e-04)
(7423,9.658e-05) (18943,2.873e-05) (47103,8.437e-06)
};

\addplot coordinates{
(9,7.881e-02)     (49,3.243e-02)    (209,1.232e-02)
(769,4.454e-03)   (2561,1.551e-03)  (7937,5.236e-04)
(23297,1.723e-04) (65537,5.545e-05) (178177,1.751e-05)
};

\addplot coordinates{
(11,6.887e-02)    (71,3.177e-02)     (351,1.341e-02)
(1471,5.334e-03)  (5503,2.027e-03)   (18943,7.415e-04)
(61183,2.628e-04) (187903,9.063e-05) (553983,3.053e-05)
};

\addplot coordinates{
(13,5.755e-02)     (97,2.925e-02)     (545,1.351e-02)
(2561,5.842e-03)   (10625,2.397e-03)  (40193,9.414e-04)
(141569,3.564e-04) (471041,1.308e-04)
(1496065,4.670e-05)
};
}%


\bookmaketitle
\tableofcontents
\include{pgfplots.title_abstract_intro}
\include{pgfplots.preliminaries}
\include{pgfplots.intro}
\include{pgfplots.reference}
\include{pgfplots.libs}
\include{pgfplots.resources}
\include{pgfplots.importexport}
\include{pgfplots.basic.reference}

\printindex

\bibliographystyle{abbrv} %gerapali} %gerabbrv} %gerunsrt.bst} %gerabbrv}% gerplain}
\nocite{pgfplotstable}
\nocite{programmingnotes}
\bibliography{pgfplots}
\end{document}

% -----------------------------------------------------------------------------
% For Stefan Pinnow as reminder on what to look for when editing the manual
% -----------------------------------------------------------------------------
% There should be no line breaks in the following environments
% - |...|
% - \declareandlabel{...}
% - \verbpdfref{...}
% If "MakeTikzPictures" isn't running through check one of the externalized
% LOG files what is the cause of that.
% -----------------------------------------------------------------------------

\include{pgfplots.basic.reference}

\printindex

\bibliographystyle{abbrv} %gerapali} %gerabbrv} %gerunsrt.bst} %gerabbrv}% gerplain}
\nocite{pgfplotstable}
\nocite{programmingnotes}
\bibliography{pgfplots}
\end{document}

% -----------------------------------------------------------------------------
% For Stefan Pinnow as reminder on what to look for when editing the manual
% -----------------------------------------------------------------------------
% There should be no line breaks in the following environments
% - |...|
% - \declareandlabel{...}
% - \verbpdfref{...}
% If "MakeTikzPictures" isn't running through check one of the externalized
% LOG files what is the cause of that.
% -----------------------------------------------------------------------------


%%%%%%%%%%%%%%%%%%%%%%%%%%%%%%%%%%%%%%%%%%%%%%%%%%%%%%%%%%%%%%%%%%%%%%%%%%%%%
%
% Package pgfplots.sty documentation.
%
% Copyright 2007/2008 by Christian Feuersaenger.
%
% This program is free software: you can redistribute it and/or modify
% it under the terms of the GNU General Public License as published by
% the Free Software Foundation, either version 3 of the License, or
% (at your option) any later version.
%
% This program is distributed in the hope that it will be useful,
% but WITHOUT ANY WARRANTY; without even the implied warranty of
% MERCHANTABILITY or FITNESS FOR A PARTICULAR PURPOSE.  See the
% GNU General Public License for more details.
%
% You should have received a copy of the GNU General Public License
% along with this program.  If not, see <http://www.gnu.org/licenses/>.
%
%
%%%%%%%%%%%%%%%%%%%%%%%%%%%%%%%%%%%%%%%%%%%%%%%%%%%%%%%%%%%%%%%%%%%%%%%%%%%%%
% SEE pgfplots-macros.tex as well!
%\pdfminorversion=5 % to allow compression
%\pdfobjcompresslevel=2
\documentclass[a4paper,openany]{book}

\let\bookmaketitle=\maketitle

% -----------------------------------
% this here is from ltxdoc:
\usepackage{doc}[=v2]
\AtBeginDocument{\MakeShortVerb{\|}}
%\setlength{\textwidth}{355pt}
%\addtolength\marginparwidth{30pt}
%\addtolength\oddsidemargin{20pt}
%\addtolength\evensidemargin{20pt}
\def\cmd#1{\cs{\expandafter\cmd@to@cs\string#1}}
\def\cmd@to@cs#1#2{\char\number`#2\relax}
\DeclareRobustCommand\cs[1]{\texttt{\char`\\#1}}
\providecommand\marg[1]{%
  {\ttfamily\char`\{}\meta{#1}{\ttfamily\char`\}}}
\providecommand\oarg[1]{%
  {\ttfamily[}\meta{#1}{\ttfamily]}}
\providecommand\parg[1]{%
  {\ttfamily(}\meta{#1}{\ttfamily)}}
\raggedbottom

% -----------------------------------
\input{pgfplots.preamble.tex}

\makeatletter
% I want a two-column index just as in pgfmanual styles. This here
% was the best way to get one:
\def\index@prologue{\section*{Index}\addcontentsline{toc}{chapter}{Index}
}
\makeatother

%\RequirePackage[german,english,francais]{babel}

\def\matlabcolormaptext{This colormap is similar to one shipped with Matlab$^\text{\textregistered}$ under a similar name.}

\IfFileExists{tikzlibraryspy.code.tex}{%
\usetikzlibrary{spy}
}{%
    \message{ERROR: tikz SPY library NOT available. The manual will only compile partially.^^J}%
}%

\usepackage{xparse}% for colorbrewer manual

\usetikzlibrary{
    decorations.markings,
    decorations.footprints,
    shapes.arrows,
    matrix,
    positioning,
}

\usepgfplotslibrary{
    fillbetween,
    ternary,
    smithchart,
    patchplots,
    polar,
    colormaps,
    colorbrewer,
    colortol,           % docu not ready yet
}
\pgfqkeys{/codeexample}{%
    every codeexample/.append style={
        /pgfplots/every ternary axis/.append style={
            /pgfplots/legend style={fill=graphicbackground},
        }
    },
    tabsize=4,
}

\pgfplotsmanualenableexternalizationofexpensive

%\usetikzlibrary{external}
%\tikzexternalize[prefix=figures/]{pgfplots}

\title{%
    Manual for Package \PGFPlots{}\\
    {\small 2D/3D Plots in \LaTeX{}, Version \pgfplotsversion}\\
    {\small\url{https://github.com/pgf-tikz/pgfplots}}
    %\\{\small Attention: you are using an unstable development version.}
}

\makeatletter
\long\def\abstractsmuggle{%
    \centering
    \textbf{Abstract}\\[0.5cm]

    \begin{minipage}{12cm}
        \PGFPlots{} draws high-quality function plots in normal or logarithmic
        scaling with a user-friendly interface directly in \TeX{}. The user
        supplies axis labels, legend entries and the plot coordinates for one or
        more plots and \PGFPlots{} applies axis scaling, computes any logarithms
        and axis ticks and draws the plots. It supports line plots, scatter
        plots, piecewise constant plots, bar plots, area plots, mesh and surface
        plots, patch plots, contour plots, quiver plots, histogram plots, box
        plots, polar axes, ternary diagrams, smith charts and some more. It is
        based on Till Tantau's package \PGF{}/\Tikz{}.
    \end{minipage}

}%

\expandafter\date\expandafter{\@date\\[2cm]
    \abstractsmuggle
}%
\makeatother

%\includeonly{pgfplots.reference}


\begin{document}

\def\plotcoords{%
\addplot coordinates {
(5,8.312e-02)    (17,2.547e-02)   (49,7.407e-03)
(129,2.102e-03)  (321,5.874e-04)  (769,1.623e-04)
(1793,4.442e-05) (4097,1.207e-05) (9217,3.261e-06)
};

\addplot coordinates{
(7,8.472e-02)    (31,3.044e-02)    (111,1.022e-02)
(351,3.303e-03)  (1023,1.039e-03)  (2815,3.196e-04)
(7423,9.658e-05) (18943,2.873e-05) (47103,8.437e-06)
};

\addplot coordinates{
(9,7.881e-02)     (49,3.243e-02)    (209,1.232e-02)
(769,4.454e-03)   (2561,1.551e-03)  (7937,5.236e-04)
(23297,1.723e-04) (65537,5.545e-05) (178177,1.751e-05)
};

\addplot coordinates{
(11,6.887e-02)    (71,3.177e-02)     (351,1.341e-02)
(1471,5.334e-03)  (5503,2.027e-03)   (18943,7.415e-04)
(61183,2.628e-04) (187903,9.063e-05) (553983,3.053e-05)
};

\addplot coordinates{
(13,5.755e-02)     (97,2.925e-02)     (545,1.351e-02)
(2561,5.842e-03)   (10625,2.397e-03)  (40193,9.414e-04)
(141569,3.564e-04) (471041,1.308e-04)
(1496065,4.670e-05)
};
}%


\bookmaketitle
\tableofcontents

%%%%%%%%%%%%%%%%%%%%%%%%%%%%%%%%%%%%%%%%%%%%%%%%%%%%%%%%%%%%%%%%%%%%%%%%%%%%%
%
% Package pgfplots.sty documentation.
%
% Copyright 2007/2008 by Christian Feuersaenger.
%
% This program is free software: you can redistribute it and/or modify
% it under the terms of the GNU General Public License as published by
% the Free Software Foundation, either version 3 of the License, or
% (at your option) any later version.
%
% This program is distributed in the hope that it will be useful,
% but WITHOUT ANY WARRANTY; without even the implied warranty of
% MERCHANTABILITY or FITNESS FOR A PARTICULAR PURPOSE.  See the
% GNU General Public License for more details.
%
% You should have received a copy of the GNU General Public License
% along with this program.  If not, see <http://www.gnu.org/licenses/>.
%
%
%%%%%%%%%%%%%%%%%%%%%%%%%%%%%%%%%%%%%%%%%%%%%%%%%%%%%%%%%%%%%%%%%%%%%%%%%%%%%
% SEE pgfplots-macros.tex as well!
%\pdfminorversion=5 % to allow compression
%\pdfobjcompresslevel=2
\documentclass[a4paper,openany]{book}

\let\bookmaketitle=\maketitle

% -----------------------------------
% this here is from ltxdoc:
\usepackage{doc}[=v2]
\AtBeginDocument{\MakeShortVerb{\|}}
%\setlength{\textwidth}{355pt}
%\addtolength\marginparwidth{30pt}
%\addtolength\oddsidemargin{20pt}
%\addtolength\evensidemargin{20pt}
\def\cmd#1{\cs{\expandafter\cmd@to@cs\string#1}}
\def\cmd@to@cs#1#2{\char\number`#2\relax}
\DeclareRobustCommand\cs[1]{\texttt{\char`\\#1}}
\providecommand\marg[1]{%
  {\ttfamily\char`\{}\meta{#1}{\ttfamily\char`\}}}
\providecommand\oarg[1]{%
  {\ttfamily[}\meta{#1}{\ttfamily]}}
\providecommand\parg[1]{%
  {\ttfamily(}\meta{#1}{\ttfamily)}}
\raggedbottom

% -----------------------------------
\input{pgfplots.preamble.tex}

\makeatletter
% I want a two-column index just as in pgfmanual styles. This here
% was the best way to get one:
\def\index@prologue{\section*{Index}\addcontentsline{toc}{chapter}{Index}
}
\makeatother

%\RequirePackage[german,english,francais]{babel}

\def\matlabcolormaptext{This colormap is similar to one shipped with Matlab$^\text{\textregistered}$ under a similar name.}

\IfFileExists{tikzlibraryspy.code.tex}{%
\usetikzlibrary{spy}
}{%
    \message{ERROR: tikz SPY library NOT available. The manual will only compile partially.^^J}%
}%

\usepackage{xparse}% for colorbrewer manual

\usetikzlibrary{
    decorations.markings,
    decorations.footprints,
    shapes.arrows,
    matrix,
    positioning,
}

\usepgfplotslibrary{
    fillbetween,
    ternary,
    smithchart,
    patchplots,
    polar,
    colormaps,
    colorbrewer,
    colortol,           % docu not ready yet
}
\pgfqkeys{/codeexample}{%
    every codeexample/.append style={
        /pgfplots/every ternary axis/.append style={
            /pgfplots/legend style={fill=graphicbackground},
        }
    },
    tabsize=4,
}

\pgfplotsmanualenableexternalizationofexpensive

%\usetikzlibrary{external}
%\tikzexternalize[prefix=figures/]{pgfplots}

\title{%
    Manual for Package \PGFPlots{}\\
    {\small 2D/3D Plots in \LaTeX{}, Version \pgfplotsversion}\\
    {\small\url{https://github.com/pgf-tikz/pgfplots}}
    %\\{\small Attention: you are using an unstable development version.}
}

\makeatletter
\long\def\abstractsmuggle{%
    \centering
    \textbf{Abstract}\\[0.5cm]

    \begin{minipage}{12cm}
        \PGFPlots{} draws high-quality function plots in normal or logarithmic
        scaling with a user-friendly interface directly in \TeX{}. The user
        supplies axis labels, legend entries and the plot coordinates for one or
        more plots and \PGFPlots{} applies axis scaling, computes any logarithms
        and axis ticks and draws the plots. It supports line plots, scatter
        plots, piecewise constant plots, bar plots, area plots, mesh and surface
        plots, patch plots, contour plots, quiver plots, histogram plots, box
        plots, polar axes, ternary diagrams, smith charts and some more. It is
        based on Till Tantau's package \PGF{}/\Tikz{}.
    \end{minipage}

}%

\expandafter\date\expandafter{\@date\\[2cm]
    \abstractsmuggle
}%
\makeatother

%\includeonly{pgfplots.reference}


\begin{document}

\def\plotcoords{%
\addplot coordinates {
(5,8.312e-02)    (17,2.547e-02)   (49,7.407e-03)
(129,2.102e-03)  (321,5.874e-04)  (769,1.623e-04)
(1793,4.442e-05) (4097,1.207e-05) (9217,3.261e-06)
};

\addplot coordinates{
(7,8.472e-02)    (31,3.044e-02)    (111,1.022e-02)
(351,3.303e-03)  (1023,1.039e-03)  (2815,3.196e-04)
(7423,9.658e-05) (18943,2.873e-05) (47103,8.437e-06)
};

\addplot coordinates{
(9,7.881e-02)     (49,3.243e-02)    (209,1.232e-02)
(769,4.454e-03)   (2561,1.551e-03)  (7937,5.236e-04)
(23297,1.723e-04) (65537,5.545e-05) (178177,1.751e-05)
};

\addplot coordinates{
(11,6.887e-02)    (71,3.177e-02)     (351,1.341e-02)
(1471,5.334e-03)  (5503,2.027e-03)   (18943,7.415e-04)
(61183,2.628e-04) (187903,9.063e-05) (553983,3.053e-05)
};

\addplot coordinates{
(13,5.755e-02)     (97,2.925e-02)     (545,1.351e-02)
(2561,5.842e-03)   (10625,2.397e-03)  (40193,9.414e-04)
(141569,3.564e-04) (471041,1.308e-04)
(1496065,4.670e-05)
};
}%


\bookmaketitle
\tableofcontents
\include{pgfplots.title_abstract_intro}
\include{pgfplots.preliminaries}
\include{pgfplots.intro}
\include{pgfplots.reference}
\include{pgfplots.libs}
\include{pgfplots.resources}
\include{pgfplots.importexport}
\include{pgfplots.basic.reference}

\printindex

\bibliographystyle{abbrv} %gerapali} %gerabbrv} %gerunsrt.bst} %gerabbrv}% gerplain}
\nocite{pgfplotstable}
\nocite{programmingnotes}
\bibliography{pgfplots}
\end{document}

% -----------------------------------------------------------------------------
% For Stefan Pinnow as reminder on what to look for when editing the manual
% -----------------------------------------------------------------------------
% There should be no line breaks in the following environments
% - |...|
% - \declareandlabel{...}
% - \verbpdfref{...}
% If "MakeTikzPictures" isn't running through check one of the externalized
% LOG files what is the cause of that.
% -----------------------------------------------------------------------------


%%%%%%%%%%%%%%%%%%%%%%%%%%%%%%%%%%%%%%%%%%%%%%%%%%%%%%%%%%%%%%%%%%%%%%%%%%%%%
%
% Package pgfplots.sty documentation.
%
% Copyright 2007/2008 by Christian Feuersaenger.
%
% This program is free software: you can redistribute it and/or modify
% it under the terms of the GNU General Public License as published by
% the Free Software Foundation, either version 3 of the License, or
% (at your option) any later version.
%
% This program is distributed in the hope that it will be useful,
% but WITHOUT ANY WARRANTY; without even the implied warranty of
% MERCHANTABILITY or FITNESS FOR A PARTICULAR PURPOSE.  See the
% GNU General Public License for more details.
%
% You should have received a copy of the GNU General Public License
% along with this program.  If not, see <http://www.gnu.org/licenses/>.
%
%
%%%%%%%%%%%%%%%%%%%%%%%%%%%%%%%%%%%%%%%%%%%%%%%%%%%%%%%%%%%%%%%%%%%%%%%%%%%%%
% SEE pgfplots-macros.tex as well!
%\pdfminorversion=5 % to allow compression
%\pdfobjcompresslevel=2
\documentclass[a4paper,openany]{book}

\let\bookmaketitle=\maketitle

% -----------------------------------
% this here is from ltxdoc:
\usepackage{doc}[=v2]
\AtBeginDocument{\MakeShortVerb{\|}}
%\setlength{\textwidth}{355pt}
%\addtolength\marginparwidth{30pt}
%\addtolength\oddsidemargin{20pt}
%\addtolength\evensidemargin{20pt}
\def\cmd#1{\cs{\expandafter\cmd@to@cs\string#1}}
\def\cmd@to@cs#1#2{\char\number`#2\relax}
\DeclareRobustCommand\cs[1]{\texttt{\char`\\#1}}
\providecommand\marg[1]{%
  {\ttfamily\char`\{}\meta{#1}{\ttfamily\char`\}}}
\providecommand\oarg[1]{%
  {\ttfamily[}\meta{#1}{\ttfamily]}}
\providecommand\parg[1]{%
  {\ttfamily(}\meta{#1}{\ttfamily)}}
\raggedbottom

% -----------------------------------
\input{pgfplots.preamble.tex}

\makeatletter
% I want a two-column index just as in pgfmanual styles. This here
% was the best way to get one:
\def\index@prologue{\section*{Index}\addcontentsline{toc}{chapter}{Index}
}
\makeatother

%\RequirePackage[german,english,francais]{babel}

\def\matlabcolormaptext{This colormap is similar to one shipped with Matlab$^\text{\textregistered}$ under a similar name.}

\IfFileExists{tikzlibraryspy.code.tex}{%
\usetikzlibrary{spy}
}{%
    \message{ERROR: tikz SPY library NOT available. The manual will only compile partially.^^J}%
}%

\usepackage{xparse}% for colorbrewer manual

\usetikzlibrary{
    decorations.markings,
    decorations.footprints,
    shapes.arrows,
    matrix,
    positioning,
}

\usepgfplotslibrary{
    fillbetween,
    ternary,
    smithchart,
    patchplots,
    polar,
    colormaps,
    colorbrewer,
    colortol,           % docu not ready yet
}
\pgfqkeys{/codeexample}{%
    every codeexample/.append style={
        /pgfplots/every ternary axis/.append style={
            /pgfplots/legend style={fill=graphicbackground},
        }
    },
    tabsize=4,
}

\pgfplotsmanualenableexternalizationofexpensive

%\usetikzlibrary{external}
%\tikzexternalize[prefix=figures/]{pgfplots}

\title{%
    Manual for Package \PGFPlots{}\\
    {\small 2D/3D Plots in \LaTeX{}, Version \pgfplotsversion}\\
    {\small\url{https://github.com/pgf-tikz/pgfplots}}
    %\\{\small Attention: you are using an unstable development version.}
}

\makeatletter
\long\def\abstractsmuggle{%
    \centering
    \textbf{Abstract}\\[0.5cm]

    \begin{minipage}{12cm}
        \PGFPlots{} draws high-quality function plots in normal or logarithmic
        scaling with a user-friendly interface directly in \TeX{}. The user
        supplies axis labels, legend entries and the plot coordinates for one or
        more plots and \PGFPlots{} applies axis scaling, computes any logarithms
        and axis ticks and draws the plots. It supports line plots, scatter
        plots, piecewise constant plots, bar plots, area plots, mesh and surface
        plots, patch plots, contour plots, quiver plots, histogram plots, box
        plots, polar axes, ternary diagrams, smith charts and some more. It is
        based on Till Tantau's package \PGF{}/\Tikz{}.
    \end{minipage}

}%

\expandafter\date\expandafter{\@date\\[2cm]
    \abstractsmuggle
}%
\makeatother

%\includeonly{pgfplots.reference}


\begin{document}

\def\plotcoords{%
\addplot coordinates {
(5,8.312e-02)    (17,2.547e-02)   (49,7.407e-03)
(129,2.102e-03)  (321,5.874e-04)  (769,1.623e-04)
(1793,4.442e-05) (4097,1.207e-05) (9217,3.261e-06)
};

\addplot coordinates{
(7,8.472e-02)    (31,3.044e-02)    (111,1.022e-02)
(351,3.303e-03)  (1023,1.039e-03)  (2815,3.196e-04)
(7423,9.658e-05) (18943,2.873e-05) (47103,8.437e-06)
};

\addplot coordinates{
(9,7.881e-02)     (49,3.243e-02)    (209,1.232e-02)
(769,4.454e-03)   (2561,1.551e-03)  (7937,5.236e-04)
(23297,1.723e-04) (65537,5.545e-05) (178177,1.751e-05)
};

\addplot coordinates{
(11,6.887e-02)    (71,3.177e-02)     (351,1.341e-02)
(1471,5.334e-03)  (5503,2.027e-03)   (18943,7.415e-04)
(61183,2.628e-04) (187903,9.063e-05) (553983,3.053e-05)
};

\addplot coordinates{
(13,5.755e-02)     (97,2.925e-02)     (545,1.351e-02)
(2561,5.842e-03)   (10625,2.397e-03)  (40193,9.414e-04)
(141569,3.564e-04) (471041,1.308e-04)
(1496065,4.670e-05)
};
}%


\bookmaketitle
\tableofcontents
\include{pgfplots.title_abstract_intro}
\include{pgfplots.preliminaries}
\include{pgfplots.intro}
\include{pgfplots.reference}
\include{pgfplots.libs}
\include{pgfplots.resources}
\include{pgfplots.importexport}
\include{pgfplots.basic.reference}

\printindex

\bibliographystyle{abbrv} %gerapali} %gerabbrv} %gerunsrt.bst} %gerabbrv}% gerplain}
\nocite{pgfplotstable}
\nocite{programmingnotes}
\bibliography{pgfplots}
\end{document}

% -----------------------------------------------------------------------------
% For Stefan Pinnow as reminder on what to look for when editing the manual
% -----------------------------------------------------------------------------
% There should be no line breaks in the following environments
% - |...|
% - \declareandlabel{...}
% - \verbpdfref{...}
% If "MakeTikzPictures" isn't running through check one of the externalized
% LOG files what is the cause of that.
% -----------------------------------------------------------------------------


%%%%%%%%%%%%%%%%%%%%%%%%%%%%%%%%%%%%%%%%%%%%%%%%%%%%%%%%%%%%%%%%%%%%%%%%%%%%%
%
% Package pgfplots.sty documentation.
%
% Copyright 2007/2008 by Christian Feuersaenger.
%
% This program is free software: you can redistribute it and/or modify
% it under the terms of the GNU General Public License as published by
% the Free Software Foundation, either version 3 of the License, or
% (at your option) any later version.
%
% This program is distributed in the hope that it will be useful,
% but WITHOUT ANY WARRANTY; without even the implied warranty of
% MERCHANTABILITY or FITNESS FOR A PARTICULAR PURPOSE.  See the
% GNU General Public License for more details.
%
% You should have received a copy of the GNU General Public License
% along with this program.  If not, see <http://www.gnu.org/licenses/>.
%
%
%%%%%%%%%%%%%%%%%%%%%%%%%%%%%%%%%%%%%%%%%%%%%%%%%%%%%%%%%%%%%%%%%%%%%%%%%%%%%
% SEE pgfplots-macros.tex as well!
%\pdfminorversion=5 % to allow compression
%\pdfobjcompresslevel=2
\documentclass[a4paper,openany]{book}

\let\bookmaketitle=\maketitle

% -----------------------------------
% this here is from ltxdoc:
\usepackage{doc}[=v2]
\AtBeginDocument{\MakeShortVerb{\|}}
%\setlength{\textwidth}{355pt}
%\addtolength\marginparwidth{30pt}
%\addtolength\oddsidemargin{20pt}
%\addtolength\evensidemargin{20pt}
\def\cmd#1{\cs{\expandafter\cmd@to@cs\string#1}}
\def\cmd@to@cs#1#2{\char\number`#2\relax}
\DeclareRobustCommand\cs[1]{\texttt{\char`\\#1}}
\providecommand\marg[1]{%
  {\ttfamily\char`\{}\meta{#1}{\ttfamily\char`\}}}
\providecommand\oarg[1]{%
  {\ttfamily[}\meta{#1}{\ttfamily]}}
\providecommand\parg[1]{%
  {\ttfamily(}\meta{#1}{\ttfamily)}}
\raggedbottom

% -----------------------------------
\input{pgfplots.preamble.tex}

\makeatletter
% I want a two-column index just as in pgfmanual styles. This here
% was the best way to get one:
\def\index@prologue{\section*{Index}\addcontentsline{toc}{chapter}{Index}
}
\makeatother

%\RequirePackage[german,english,francais]{babel}

\def\matlabcolormaptext{This colormap is similar to one shipped with Matlab$^\text{\textregistered}$ under a similar name.}

\IfFileExists{tikzlibraryspy.code.tex}{%
\usetikzlibrary{spy}
}{%
    \message{ERROR: tikz SPY library NOT available. The manual will only compile partially.^^J}%
}%

\usepackage{xparse}% for colorbrewer manual

\usetikzlibrary{
    decorations.markings,
    decorations.footprints,
    shapes.arrows,
    matrix,
    positioning,
}

\usepgfplotslibrary{
    fillbetween,
    ternary,
    smithchart,
    patchplots,
    polar,
    colormaps,
    colorbrewer,
    colortol,           % docu not ready yet
}
\pgfqkeys{/codeexample}{%
    every codeexample/.append style={
        /pgfplots/every ternary axis/.append style={
            /pgfplots/legend style={fill=graphicbackground},
        }
    },
    tabsize=4,
}

\pgfplotsmanualenableexternalizationofexpensive

%\usetikzlibrary{external}
%\tikzexternalize[prefix=figures/]{pgfplots}

\title{%
    Manual for Package \PGFPlots{}\\
    {\small 2D/3D Plots in \LaTeX{}, Version \pgfplotsversion}\\
    {\small\url{https://github.com/pgf-tikz/pgfplots}}
    %\\{\small Attention: you are using an unstable development version.}
}

\makeatletter
\long\def\abstractsmuggle{%
    \centering
    \textbf{Abstract}\\[0.5cm]

    \begin{minipage}{12cm}
        \PGFPlots{} draws high-quality function plots in normal or logarithmic
        scaling with a user-friendly interface directly in \TeX{}. The user
        supplies axis labels, legend entries and the plot coordinates for one or
        more plots and \PGFPlots{} applies axis scaling, computes any logarithms
        and axis ticks and draws the plots. It supports line plots, scatter
        plots, piecewise constant plots, bar plots, area plots, mesh and surface
        plots, patch plots, contour plots, quiver plots, histogram plots, box
        plots, polar axes, ternary diagrams, smith charts and some more. It is
        based on Till Tantau's package \PGF{}/\Tikz{}.
    \end{minipage}

}%

\expandafter\date\expandafter{\@date\\[2cm]
    \abstractsmuggle
}%
\makeatother

%\includeonly{pgfplots.reference}


\begin{document}

\def\plotcoords{%
\addplot coordinates {
(5,8.312e-02)    (17,2.547e-02)   (49,7.407e-03)
(129,2.102e-03)  (321,5.874e-04)  (769,1.623e-04)
(1793,4.442e-05) (4097,1.207e-05) (9217,3.261e-06)
};

\addplot coordinates{
(7,8.472e-02)    (31,3.044e-02)    (111,1.022e-02)
(351,3.303e-03)  (1023,1.039e-03)  (2815,3.196e-04)
(7423,9.658e-05) (18943,2.873e-05) (47103,8.437e-06)
};

\addplot coordinates{
(9,7.881e-02)     (49,3.243e-02)    (209,1.232e-02)
(769,4.454e-03)   (2561,1.551e-03)  (7937,5.236e-04)
(23297,1.723e-04) (65537,5.545e-05) (178177,1.751e-05)
};

\addplot coordinates{
(11,6.887e-02)    (71,3.177e-02)     (351,1.341e-02)
(1471,5.334e-03)  (5503,2.027e-03)   (18943,7.415e-04)
(61183,2.628e-04) (187903,9.063e-05) (553983,3.053e-05)
};

\addplot coordinates{
(13,5.755e-02)     (97,2.925e-02)     (545,1.351e-02)
(2561,5.842e-03)   (10625,2.397e-03)  (40193,9.414e-04)
(141569,3.564e-04) (471041,1.308e-04)
(1496065,4.670e-05)
};
}%


\bookmaketitle
\tableofcontents
\include{pgfplots.title_abstract_intro}
\include{pgfplots.preliminaries}
\include{pgfplots.intro}
\include{pgfplots.reference}
\include{pgfplots.libs}
\include{pgfplots.resources}
\include{pgfplots.importexport}
\include{pgfplots.basic.reference}

\printindex

\bibliographystyle{abbrv} %gerapali} %gerabbrv} %gerunsrt.bst} %gerabbrv}% gerplain}
\nocite{pgfplotstable}
\nocite{programmingnotes}
\bibliography{pgfplots}
\end{document}

% -----------------------------------------------------------------------------
% For Stefan Pinnow as reminder on what to look for when editing the manual
% -----------------------------------------------------------------------------
% There should be no line breaks in the following environments
% - |...|
% - \declareandlabel{...}
% - \verbpdfref{...}
% If "MakeTikzPictures" isn't running through check one of the externalized
% LOG files what is the cause of that.
% -----------------------------------------------------------------------------


%%%%%%%%%%%%%%%%%%%%%%%%%%%%%%%%%%%%%%%%%%%%%%%%%%%%%%%%%%%%%%%%%%%%%%%%%%%%%
%
% Package pgfplots.sty documentation.
%
% Copyright 2007/2008 by Christian Feuersaenger.
%
% This program is free software: you can redistribute it and/or modify
% it under the terms of the GNU General Public License as published by
% the Free Software Foundation, either version 3 of the License, or
% (at your option) any later version.
%
% This program is distributed in the hope that it will be useful,
% but WITHOUT ANY WARRANTY; without even the implied warranty of
% MERCHANTABILITY or FITNESS FOR A PARTICULAR PURPOSE.  See the
% GNU General Public License for more details.
%
% You should have received a copy of the GNU General Public License
% along with this program.  If not, see <http://www.gnu.org/licenses/>.
%
%
%%%%%%%%%%%%%%%%%%%%%%%%%%%%%%%%%%%%%%%%%%%%%%%%%%%%%%%%%%%%%%%%%%%%%%%%%%%%%
% SEE pgfplots-macros.tex as well!
%\pdfminorversion=5 % to allow compression
%\pdfobjcompresslevel=2
\documentclass[a4paper,openany]{book}

\let\bookmaketitle=\maketitle

% -----------------------------------
% this here is from ltxdoc:
\usepackage{doc}[=v2]
\AtBeginDocument{\MakeShortVerb{\|}}
%\setlength{\textwidth}{355pt}
%\addtolength\marginparwidth{30pt}
%\addtolength\oddsidemargin{20pt}
%\addtolength\evensidemargin{20pt}
\def\cmd#1{\cs{\expandafter\cmd@to@cs\string#1}}
\def\cmd@to@cs#1#2{\char\number`#2\relax}
\DeclareRobustCommand\cs[1]{\texttt{\char`\\#1}}
\providecommand\marg[1]{%
  {\ttfamily\char`\{}\meta{#1}{\ttfamily\char`\}}}
\providecommand\oarg[1]{%
  {\ttfamily[}\meta{#1}{\ttfamily]}}
\providecommand\parg[1]{%
  {\ttfamily(}\meta{#1}{\ttfamily)}}
\raggedbottom

% -----------------------------------
\input{pgfplots.preamble.tex}

\makeatletter
% I want a two-column index just as in pgfmanual styles. This here
% was the best way to get one:
\def\index@prologue{\section*{Index}\addcontentsline{toc}{chapter}{Index}
}
\makeatother

%\RequirePackage[german,english,francais]{babel}

\def\matlabcolormaptext{This colormap is similar to one shipped with Matlab$^\text{\textregistered}$ under a similar name.}

\IfFileExists{tikzlibraryspy.code.tex}{%
\usetikzlibrary{spy}
}{%
    \message{ERROR: tikz SPY library NOT available. The manual will only compile partially.^^J}%
}%

\usepackage{xparse}% for colorbrewer manual

\usetikzlibrary{
    decorations.markings,
    decorations.footprints,
    shapes.arrows,
    matrix,
    positioning,
}

\usepgfplotslibrary{
    fillbetween,
    ternary,
    smithchart,
    patchplots,
    polar,
    colormaps,
    colorbrewer,
    colortol,           % docu not ready yet
}
\pgfqkeys{/codeexample}{%
    every codeexample/.append style={
        /pgfplots/every ternary axis/.append style={
            /pgfplots/legend style={fill=graphicbackground},
        }
    },
    tabsize=4,
}

\pgfplotsmanualenableexternalizationofexpensive

%\usetikzlibrary{external}
%\tikzexternalize[prefix=figures/]{pgfplots}

\title{%
    Manual for Package \PGFPlots{}\\
    {\small 2D/3D Plots in \LaTeX{}, Version \pgfplotsversion}\\
    {\small\url{https://github.com/pgf-tikz/pgfplots}}
    %\\{\small Attention: you are using an unstable development version.}
}

\makeatletter
\long\def\abstractsmuggle{%
    \centering
    \textbf{Abstract}\\[0.5cm]

    \begin{minipage}{12cm}
        \PGFPlots{} draws high-quality function plots in normal or logarithmic
        scaling with a user-friendly interface directly in \TeX{}. The user
        supplies axis labels, legend entries and the plot coordinates for one or
        more plots and \PGFPlots{} applies axis scaling, computes any logarithms
        and axis ticks and draws the plots. It supports line plots, scatter
        plots, piecewise constant plots, bar plots, area plots, mesh and surface
        plots, patch plots, contour plots, quiver plots, histogram plots, box
        plots, polar axes, ternary diagrams, smith charts and some more. It is
        based on Till Tantau's package \PGF{}/\Tikz{}.
    \end{minipage}

}%

\expandafter\date\expandafter{\@date\\[2cm]
    \abstractsmuggle
}%
\makeatother

%\includeonly{pgfplots.reference}


\begin{document}

\def\plotcoords{%
\addplot coordinates {
(5,8.312e-02)    (17,2.547e-02)   (49,7.407e-03)
(129,2.102e-03)  (321,5.874e-04)  (769,1.623e-04)
(1793,4.442e-05) (4097,1.207e-05) (9217,3.261e-06)
};

\addplot coordinates{
(7,8.472e-02)    (31,3.044e-02)    (111,1.022e-02)
(351,3.303e-03)  (1023,1.039e-03)  (2815,3.196e-04)
(7423,9.658e-05) (18943,2.873e-05) (47103,8.437e-06)
};

\addplot coordinates{
(9,7.881e-02)     (49,3.243e-02)    (209,1.232e-02)
(769,4.454e-03)   (2561,1.551e-03)  (7937,5.236e-04)
(23297,1.723e-04) (65537,5.545e-05) (178177,1.751e-05)
};

\addplot coordinates{
(11,6.887e-02)    (71,3.177e-02)     (351,1.341e-02)
(1471,5.334e-03)  (5503,2.027e-03)   (18943,7.415e-04)
(61183,2.628e-04) (187903,9.063e-05) (553983,3.053e-05)
};

\addplot coordinates{
(13,5.755e-02)     (97,2.925e-02)     (545,1.351e-02)
(2561,5.842e-03)   (10625,2.397e-03)  (40193,9.414e-04)
(141569,3.564e-04) (471041,1.308e-04)
(1496065,4.670e-05)
};
}%


\bookmaketitle
\tableofcontents
\include{pgfplots.title_abstract_intro}
\include{pgfplots.preliminaries}
\include{pgfplots.intro}
\include{pgfplots.reference}
\include{pgfplots.libs}
\include{pgfplots.resources}
\include{pgfplots.importexport}
\include{pgfplots.basic.reference}

\printindex

\bibliographystyle{abbrv} %gerapali} %gerabbrv} %gerunsrt.bst} %gerabbrv}% gerplain}
\nocite{pgfplotstable}
\nocite{programmingnotes}
\bibliography{pgfplots}
\end{document}

% -----------------------------------------------------------------------------
% For Stefan Pinnow as reminder on what to look for when editing the manual
% -----------------------------------------------------------------------------
% There should be no line breaks in the following environments
% - |...|
% - \declareandlabel{...}
% - \verbpdfref{...}
% If "MakeTikzPictures" isn't running through check one of the externalized
% LOG files what is the cause of that.
% -----------------------------------------------------------------------------


%%%%%%%%%%%%%%%%%%%%%%%%%%%%%%%%%%%%%%%%%%%%%%%%%%%%%%%%%%%%%%%%%%%%%%%%%%%%%
%
% Package pgfplots.sty documentation.
%
% Copyright 2007/2008 by Christian Feuersaenger.
%
% This program is free software: you can redistribute it and/or modify
% it under the terms of the GNU General Public License as published by
% the Free Software Foundation, either version 3 of the License, or
% (at your option) any later version.
%
% This program is distributed in the hope that it will be useful,
% but WITHOUT ANY WARRANTY; without even the implied warranty of
% MERCHANTABILITY or FITNESS FOR A PARTICULAR PURPOSE.  See the
% GNU General Public License for more details.
%
% You should have received a copy of the GNU General Public License
% along with this program.  If not, see <http://www.gnu.org/licenses/>.
%
%
%%%%%%%%%%%%%%%%%%%%%%%%%%%%%%%%%%%%%%%%%%%%%%%%%%%%%%%%%%%%%%%%%%%%%%%%%%%%%
% SEE pgfplots-macros.tex as well!
%\pdfminorversion=5 % to allow compression
%\pdfobjcompresslevel=2
\documentclass[a4paper,openany]{book}

\let\bookmaketitle=\maketitle

% -----------------------------------
% this here is from ltxdoc:
\usepackage{doc}[=v2]
\AtBeginDocument{\MakeShortVerb{\|}}
%\setlength{\textwidth}{355pt}
%\addtolength\marginparwidth{30pt}
%\addtolength\oddsidemargin{20pt}
%\addtolength\evensidemargin{20pt}
\def\cmd#1{\cs{\expandafter\cmd@to@cs\string#1}}
\def\cmd@to@cs#1#2{\char\number`#2\relax}
\DeclareRobustCommand\cs[1]{\texttt{\char`\\#1}}
\providecommand\marg[1]{%
  {\ttfamily\char`\{}\meta{#1}{\ttfamily\char`\}}}
\providecommand\oarg[1]{%
  {\ttfamily[}\meta{#1}{\ttfamily]}}
\providecommand\parg[1]{%
  {\ttfamily(}\meta{#1}{\ttfamily)}}
\raggedbottom

% -----------------------------------
\input{pgfplots.preamble.tex}

\makeatletter
% I want a two-column index just as in pgfmanual styles. This here
% was the best way to get one:
\def\index@prologue{\section*{Index}\addcontentsline{toc}{chapter}{Index}
}
\makeatother

%\RequirePackage[german,english,francais]{babel}

\def\matlabcolormaptext{This colormap is similar to one shipped with Matlab$^\text{\textregistered}$ under a similar name.}

\IfFileExists{tikzlibraryspy.code.tex}{%
\usetikzlibrary{spy}
}{%
    \message{ERROR: tikz SPY library NOT available. The manual will only compile partially.^^J}%
}%

\usepackage{xparse}% for colorbrewer manual

\usetikzlibrary{
    decorations.markings,
    decorations.footprints,
    shapes.arrows,
    matrix,
    positioning,
}

\usepgfplotslibrary{
    fillbetween,
    ternary,
    smithchart,
    patchplots,
    polar,
    colormaps,
    colorbrewer,
    colortol,           % docu not ready yet
}
\pgfqkeys{/codeexample}{%
    every codeexample/.append style={
        /pgfplots/every ternary axis/.append style={
            /pgfplots/legend style={fill=graphicbackground},
        }
    },
    tabsize=4,
}

\pgfplotsmanualenableexternalizationofexpensive

%\usetikzlibrary{external}
%\tikzexternalize[prefix=figures/]{pgfplots}

\title{%
    Manual for Package \PGFPlots{}\\
    {\small 2D/3D Plots in \LaTeX{}, Version \pgfplotsversion}\\
    {\small\url{https://github.com/pgf-tikz/pgfplots}}
    %\\{\small Attention: you are using an unstable development version.}
}

\makeatletter
\long\def\abstractsmuggle{%
    \centering
    \textbf{Abstract}\\[0.5cm]

    \begin{minipage}{12cm}
        \PGFPlots{} draws high-quality function plots in normal or logarithmic
        scaling with a user-friendly interface directly in \TeX{}. The user
        supplies axis labels, legend entries and the plot coordinates for one or
        more plots and \PGFPlots{} applies axis scaling, computes any logarithms
        and axis ticks and draws the plots. It supports line plots, scatter
        plots, piecewise constant plots, bar plots, area plots, mesh and surface
        plots, patch plots, contour plots, quiver plots, histogram plots, box
        plots, polar axes, ternary diagrams, smith charts and some more. It is
        based on Till Tantau's package \PGF{}/\Tikz{}.
    \end{minipage}

}%

\expandafter\date\expandafter{\@date\\[2cm]
    \abstractsmuggle
}%
\makeatother

%\includeonly{pgfplots.reference}


\begin{document}

\def\plotcoords{%
\addplot coordinates {
(5,8.312e-02)    (17,2.547e-02)   (49,7.407e-03)
(129,2.102e-03)  (321,5.874e-04)  (769,1.623e-04)
(1793,4.442e-05) (4097,1.207e-05) (9217,3.261e-06)
};

\addplot coordinates{
(7,8.472e-02)    (31,3.044e-02)    (111,1.022e-02)
(351,3.303e-03)  (1023,1.039e-03)  (2815,3.196e-04)
(7423,9.658e-05) (18943,2.873e-05) (47103,8.437e-06)
};

\addplot coordinates{
(9,7.881e-02)     (49,3.243e-02)    (209,1.232e-02)
(769,4.454e-03)   (2561,1.551e-03)  (7937,5.236e-04)
(23297,1.723e-04) (65537,5.545e-05) (178177,1.751e-05)
};

\addplot coordinates{
(11,6.887e-02)    (71,3.177e-02)     (351,1.341e-02)
(1471,5.334e-03)  (5503,2.027e-03)   (18943,7.415e-04)
(61183,2.628e-04) (187903,9.063e-05) (553983,3.053e-05)
};

\addplot coordinates{
(13,5.755e-02)     (97,2.925e-02)     (545,1.351e-02)
(2561,5.842e-03)   (10625,2.397e-03)  (40193,9.414e-04)
(141569,3.564e-04) (471041,1.308e-04)
(1496065,4.670e-05)
};
}%


\bookmaketitle
\tableofcontents
\include{pgfplots.title_abstract_intro}
\include{pgfplots.preliminaries}
\include{pgfplots.intro}
\include{pgfplots.reference}
\include{pgfplots.libs}
\include{pgfplots.resources}
\include{pgfplots.importexport}
\include{pgfplots.basic.reference}

\printindex

\bibliographystyle{abbrv} %gerapali} %gerabbrv} %gerunsrt.bst} %gerabbrv}% gerplain}
\nocite{pgfplotstable}
\nocite{programmingnotes}
\bibliography{pgfplots}
\end{document}

% -----------------------------------------------------------------------------
% For Stefan Pinnow as reminder on what to look for when editing the manual
% -----------------------------------------------------------------------------
% There should be no line breaks in the following environments
% - |...|
% - \declareandlabel{...}
% - \verbpdfref{...}
% If "MakeTikzPictures" isn't running through check one of the externalized
% LOG files what is the cause of that.
% -----------------------------------------------------------------------------


%%%%%%%%%%%%%%%%%%%%%%%%%%%%%%%%%%%%%%%%%%%%%%%%%%%%%%%%%%%%%%%%%%%%%%%%%%%%%
%
% Package pgfplots.sty documentation.
%
% Copyright 2007/2008 by Christian Feuersaenger.
%
% This program is free software: you can redistribute it and/or modify
% it under the terms of the GNU General Public License as published by
% the Free Software Foundation, either version 3 of the License, or
% (at your option) any later version.
%
% This program is distributed in the hope that it will be useful,
% but WITHOUT ANY WARRANTY; without even the implied warranty of
% MERCHANTABILITY or FITNESS FOR A PARTICULAR PURPOSE.  See the
% GNU General Public License for more details.
%
% You should have received a copy of the GNU General Public License
% along with this program.  If not, see <http://www.gnu.org/licenses/>.
%
%
%%%%%%%%%%%%%%%%%%%%%%%%%%%%%%%%%%%%%%%%%%%%%%%%%%%%%%%%%%%%%%%%%%%%%%%%%%%%%
% SEE pgfplots-macros.tex as well!
%\pdfminorversion=5 % to allow compression
%\pdfobjcompresslevel=2
\documentclass[a4paper,openany]{book}

\let\bookmaketitle=\maketitle

% -----------------------------------
% this here is from ltxdoc:
\usepackage{doc}[=v2]
\AtBeginDocument{\MakeShortVerb{\|}}
%\setlength{\textwidth}{355pt}
%\addtolength\marginparwidth{30pt}
%\addtolength\oddsidemargin{20pt}
%\addtolength\evensidemargin{20pt}
\def\cmd#1{\cs{\expandafter\cmd@to@cs\string#1}}
\def\cmd@to@cs#1#2{\char\number`#2\relax}
\DeclareRobustCommand\cs[1]{\texttt{\char`\\#1}}
\providecommand\marg[1]{%
  {\ttfamily\char`\{}\meta{#1}{\ttfamily\char`\}}}
\providecommand\oarg[1]{%
  {\ttfamily[}\meta{#1}{\ttfamily]}}
\providecommand\parg[1]{%
  {\ttfamily(}\meta{#1}{\ttfamily)}}
\raggedbottom

% -----------------------------------
\input{pgfplots.preamble.tex}

\makeatletter
% I want a two-column index just as in pgfmanual styles. This here
% was the best way to get one:
\def\index@prologue{\section*{Index}\addcontentsline{toc}{chapter}{Index}
}
\makeatother

%\RequirePackage[german,english,francais]{babel}

\def\matlabcolormaptext{This colormap is similar to one shipped with Matlab$^\text{\textregistered}$ under a similar name.}

\IfFileExists{tikzlibraryspy.code.tex}{%
\usetikzlibrary{spy}
}{%
    \message{ERROR: tikz SPY library NOT available. The manual will only compile partially.^^J}%
}%

\usepackage{xparse}% for colorbrewer manual

\usetikzlibrary{
    decorations.markings,
    decorations.footprints,
    shapes.arrows,
    matrix,
    positioning,
}

\usepgfplotslibrary{
    fillbetween,
    ternary,
    smithchart,
    patchplots,
    polar,
    colormaps,
    colorbrewer,
    colortol,           % docu not ready yet
}
\pgfqkeys{/codeexample}{%
    every codeexample/.append style={
        /pgfplots/every ternary axis/.append style={
            /pgfplots/legend style={fill=graphicbackground},
        }
    },
    tabsize=4,
}

\pgfplotsmanualenableexternalizationofexpensive

%\usetikzlibrary{external}
%\tikzexternalize[prefix=figures/]{pgfplots}

\title{%
    Manual for Package \PGFPlots{}\\
    {\small 2D/3D Plots in \LaTeX{}, Version \pgfplotsversion}\\
    {\small\url{https://github.com/pgf-tikz/pgfplots}}
    %\\{\small Attention: you are using an unstable development version.}
}

\makeatletter
\long\def\abstractsmuggle{%
    \centering
    \textbf{Abstract}\\[0.5cm]

    \begin{minipage}{12cm}
        \PGFPlots{} draws high-quality function plots in normal or logarithmic
        scaling with a user-friendly interface directly in \TeX{}. The user
        supplies axis labels, legend entries and the plot coordinates for one or
        more plots and \PGFPlots{} applies axis scaling, computes any logarithms
        and axis ticks and draws the plots. It supports line plots, scatter
        plots, piecewise constant plots, bar plots, area plots, mesh and surface
        plots, patch plots, contour plots, quiver plots, histogram plots, box
        plots, polar axes, ternary diagrams, smith charts and some more. It is
        based on Till Tantau's package \PGF{}/\Tikz{}.
    \end{minipage}

}%

\expandafter\date\expandafter{\@date\\[2cm]
    \abstractsmuggle
}%
\makeatother

%\includeonly{pgfplots.reference}


\begin{document}

\def\plotcoords{%
\addplot coordinates {
(5,8.312e-02)    (17,2.547e-02)   (49,7.407e-03)
(129,2.102e-03)  (321,5.874e-04)  (769,1.623e-04)
(1793,4.442e-05) (4097,1.207e-05) (9217,3.261e-06)
};

\addplot coordinates{
(7,8.472e-02)    (31,3.044e-02)    (111,1.022e-02)
(351,3.303e-03)  (1023,1.039e-03)  (2815,3.196e-04)
(7423,9.658e-05) (18943,2.873e-05) (47103,8.437e-06)
};

\addplot coordinates{
(9,7.881e-02)     (49,3.243e-02)    (209,1.232e-02)
(769,4.454e-03)   (2561,1.551e-03)  (7937,5.236e-04)
(23297,1.723e-04) (65537,5.545e-05) (178177,1.751e-05)
};

\addplot coordinates{
(11,6.887e-02)    (71,3.177e-02)     (351,1.341e-02)
(1471,5.334e-03)  (5503,2.027e-03)   (18943,7.415e-04)
(61183,2.628e-04) (187903,9.063e-05) (553983,3.053e-05)
};

\addplot coordinates{
(13,5.755e-02)     (97,2.925e-02)     (545,1.351e-02)
(2561,5.842e-03)   (10625,2.397e-03)  (40193,9.414e-04)
(141569,3.564e-04) (471041,1.308e-04)
(1496065,4.670e-05)
};
}%


\bookmaketitle
\tableofcontents
\include{pgfplots.title_abstract_intro}
\include{pgfplots.preliminaries}
\include{pgfplots.intro}
\include{pgfplots.reference}
\include{pgfplots.libs}
\include{pgfplots.resources}
\include{pgfplots.importexport}
\include{pgfplots.basic.reference}

\printindex

\bibliographystyle{abbrv} %gerapali} %gerabbrv} %gerunsrt.bst} %gerabbrv}% gerplain}
\nocite{pgfplotstable}
\nocite{programmingnotes}
\bibliography{pgfplots}
\end{document}

% -----------------------------------------------------------------------------
% For Stefan Pinnow as reminder on what to look for when editing the manual
% -----------------------------------------------------------------------------
% There should be no line breaks in the following environments
% - |...|
% - \declareandlabel{...}
% - \verbpdfref{...}
% If "MakeTikzPictures" isn't running through check one of the externalized
% LOG files what is the cause of that.
% -----------------------------------------------------------------------------


%%%%%%%%%%%%%%%%%%%%%%%%%%%%%%%%%%%%%%%%%%%%%%%%%%%%%%%%%%%%%%%%%%%%%%%%%%%%%
%
% Package pgfplots.sty documentation.
%
% Copyright 2007/2008 by Christian Feuersaenger.
%
% This program is free software: you can redistribute it and/or modify
% it under the terms of the GNU General Public License as published by
% the Free Software Foundation, either version 3 of the License, or
% (at your option) any later version.
%
% This program is distributed in the hope that it will be useful,
% but WITHOUT ANY WARRANTY; without even the implied warranty of
% MERCHANTABILITY or FITNESS FOR A PARTICULAR PURPOSE.  See the
% GNU General Public License for more details.
%
% You should have received a copy of the GNU General Public License
% along with this program.  If not, see <http://www.gnu.org/licenses/>.
%
%
%%%%%%%%%%%%%%%%%%%%%%%%%%%%%%%%%%%%%%%%%%%%%%%%%%%%%%%%%%%%%%%%%%%%%%%%%%%%%
% SEE pgfplots-macros.tex as well!
%\pdfminorversion=5 % to allow compression
%\pdfobjcompresslevel=2
\documentclass[a4paper,openany]{book}

\let\bookmaketitle=\maketitle

% -----------------------------------
% this here is from ltxdoc:
\usepackage{doc}[=v2]
\AtBeginDocument{\MakeShortVerb{\|}}
%\setlength{\textwidth}{355pt}
%\addtolength\marginparwidth{30pt}
%\addtolength\oddsidemargin{20pt}
%\addtolength\evensidemargin{20pt}
\def\cmd#1{\cs{\expandafter\cmd@to@cs\string#1}}
\def\cmd@to@cs#1#2{\char\number`#2\relax}
\DeclareRobustCommand\cs[1]{\texttt{\char`\\#1}}
\providecommand\marg[1]{%
  {\ttfamily\char`\{}\meta{#1}{\ttfamily\char`\}}}
\providecommand\oarg[1]{%
  {\ttfamily[}\meta{#1}{\ttfamily]}}
\providecommand\parg[1]{%
  {\ttfamily(}\meta{#1}{\ttfamily)}}
\raggedbottom

% -----------------------------------
\input{pgfplots.preamble.tex}

\makeatletter
% I want a two-column index just as in pgfmanual styles. This here
% was the best way to get one:
\def\index@prologue{\section*{Index}\addcontentsline{toc}{chapter}{Index}
}
\makeatother

%\RequirePackage[german,english,francais]{babel}

\def\matlabcolormaptext{This colormap is similar to one shipped with Matlab$^\text{\textregistered}$ under a similar name.}

\IfFileExists{tikzlibraryspy.code.tex}{%
\usetikzlibrary{spy}
}{%
    \message{ERROR: tikz SPY library NOT available. The manual will only compile partially.^^J}%
}%

\usepackage{xparse}% for colorbrewer manual

\usetikzlibrary{
    decorations.markings,
    decorations.footprints,
    shapes.arrows,
    matrix,
    positioning,
}

\usepgfplotslibrary{
    fillbetween,
    ternary,
    smithchart,
    patchplots,
    polar,
    colormaps,
    colorbrewer,
    colortol,           % docu not ready yet
}
\pgfqkeys{/codeexample}{%
    every codeexample/.append style={
        /pgfplots/every ternary axis/.append style={
            /pgfplots/legend style={fill=graphicbackground},
        }
    },
    tabsize=4,
}

\pgfplotsmanualenableexternalizationofexpensive

%\usetikzlibrary{external}
%\tikzexternalize[prefix=figures/]{pgfplots}

\title{%
    Manual for Package \PGFPlots{}\\
    {\small 2D/3D Plots in \LaTeX{}, Version \pgfplotsversion}\\
    {\small\url{https://github.com/pgf-tikz/pgfplots}}
    %\\{\small Attention: you are using an unstable development version.}
}

\makeatletter
\long\def\abstractsmuggle{%
    \centering
    \textbf{Abstract}\\[0.5cm]

    \begin{minipage}{12cm}
        \PGFPlots{} draws high-quality function plots in normal or logarithmic
        scaling with a user-friendly interface directly in \TeX{}. The user
        supplies axis labels, legend entries and the plot coordinates for one or
        more plots and \PGFPlots{} applies axis scaling, computes any logarithms
        and axis ticks and draws the plots. It supports line plots, scatter
        plots, piecewise constant plots, bar plots, area plots, mesh and surface
        plots, patch plots, contour plots, quiver plots, histogram plots, box
        plots, polar axes, ternary diagrams, smith charts and some more. It is
        based on Till Tantau's package \PGF{}/\Tikz{}.
    \end{minipage}

}%

\expandafter\date\expandafter{\@date\\[2cm]
    \abstractsmuggle
}%
\makeatother

%\includeonly{pgfplots.reference}


\begin{document}

\def\plotcoords{%
\addplot coordinates {
(5,8.312e-02)    (17,2.547e-02)   (49,7.407e-03)
(129,2.102e-03)  (321,5.874e-04)  (769,1.623e-04)
(1793,4.442e-05) (4097,1.207e-05) (9217,3.261e-06)
};

\addplot coordinates{
(7,8.472e-02)    (31,3.044e-02)    (111,1.022e-02)
(351,3.303e-03)  (1023,1.039e-03)  (2815,3.196e-04)
(7423,9.658e-05) (18943,2.873e-05) (47103,8.437e-06)
};

\addplot coordinates{
(9,7.881e-02)     (49,3.243e-02)    (209,1.232e-02)
(769,4.454e-03)   (2561,1.551e-03)  (7937,5.236e-04)
(23297,1.723e-04) (65537,5.545e-05) (178177,1.751e-05)
};

\addplot coordinates{
(11,6.887e-02)    (71,3.177e-02)     (351,1.341e-02)
(1471,5.334e-03)  (5503,2.027e-03)   (18943,7.415e-04)
(61183,2.628e-04) (187903,9.063e-05) (553983,3.053e-05)
};

\addplot coordinates{
(13,5.755e-02)     (97,2.925e-02)     (545,1.351e-02)
(2561,5.842e-03)   (10625,2.397e-03)  (40193,9.414e-04)
(141569,3.564e-04) (471041,1.308e-04)
(1496065,4.670e-05)
};
}%


\bookmaketitle
\tableofcontents
\include{pgfplots.title_abstract_intro}
\include{pgfplots.preliminaries}
\include{pgfplots.intro}
\include{pgfplots.reference}
\include{pgfplots.libs}
\include{pgfplots.resources}
\include{pgfplots.importexport}
\include{pgfplots.basic.reference}

\printindex

\bibliographystyle{abbrv} %gerapali} %gerabbrv} %gerunsrt.bst} %gerabbrv}% gerplain}
\nocite{pgfplotstable}
\nocite{programmingnotes}
\bibliography{pgfplots}
\end{document}

% -----------------------------------------------------------------------------
% For Stefan Pinnow as reminder on what to look for when editing the manual
% -----------------------------------------------------------------------------
% There should be no line breaks in the following environments
% - |...|
% - \declareandlabel{...}
% - \verbpdfref{...}
% If "MakeTikzPictures" isn't running through check one of the externalized
% LOG files what is the cause of that.
% -----------------------------------------------------------------------------

\include{pgfplots.basic.reference}

\printindex

\bibliographystyle{abbrv} %gerapali} %gerabbrv} %gerunsrt.bst} %gerabbrv}% gerplain}
\nocite{pgfplotstable}
\nocite{programmingnotes}
\bibliography{pgfplots}
\end{document}

% -----------------------------------------------------------------------------
% For Stefan Pinnow as reminder on what to look for when editing the manual
% -----------------------------------------------------------------------------
% There should be no line breaks in the following environments
% - |...|
% - \declareandlabel{...}
% - \verbpdfref{...}
% If "MakeTikzPictures" isn't running through check one of the externalized
% LOG files what is the cause of that.
% -----------------------------------------------------------------------------


%%%%%%%%%%%%%%%%%%%%%%%%%%%%%%%%%%%%%%%%%%%%%%%%%%%%%%%%%%%%%%%%%%%%%%%%%%%%%
%
% Package pgfplots.sty documentation.
%
% Copyright 2007/2008 by Christian Feuersaenger.
%
% This program is free software: you can redistribute it and/or modify
% it under the terms of the GNU General Public License as published by
% the Free Software Foundation, either version 3 of the License, or
% (at your option) any later version.
%
% This program is distributed in the hope that it will be useful,
% but WITHOUT ANY WARRANTY; without even the implied warranty of
% MERCHANTABILITY or FITNESS FOR A PARTICULAR PURPOSE.  See the
% GNU General Public License for more details.
%
% You should have received a copy of the GNU General Public License
% along with this program.  If not, see <http://www.gnu.org/licenses/>.
%
%
%%%%%%%%%%%%%%%%%%%%%%%%%%%%%%%%%%%%%%%%%%%%%%%%%%%%%%%%%%%%%%%%%%%%%%%%%%%%%
% SEE pgfplots-macros.tex as well!
%\pdfminorversion=5 % to allow compression
%\pdfobjcompresslevel=2
\documentclass[a4paper,openany]{book}

\let\bookmaketitle=\maketitle

% -----------------------------------
% this here is from ltxdoc:
\usepackage{doc}[=v2]
\AtBeginDocument{\MakeShortVerb{\|}}
%\setlength{\textwidth}{355pt}
%\addtolength\marginparwidth{30pt}
%\addtolength\oddsidemargin{20pt}
%\addtolength\evensidemargin{20pt}
\def\cmd#1{\cs{\expandafter\cmd@to@cs\string#1}}
\def\cmd@to@cs#1#2{\char\number`#2\relax}
\DeclareRobustCommand\cs[1]{\texttt{\char`\\#1}}
\providecommand\marg[1]{%
  {\ttfamily\char`\{}\meta{#1}{\ttfamily\char`\}}}
\providecommand\oarg[1]{%
  {\ttfamily[}\meta{#1}{\ttfamily]}}
\providecommand\parg[1]{%
  {\ttfamily(}\meta{#1}{\ttfamily)}}
\raggedbottom

% -----------------------------------
\input{pgfplots.preamble.tex}

\makeatletter
% I want a two-column index just as in pgfmanual styles. This here
% was the best way to get one:
\def\index@prologue{\section*{Index}\addcontentsline{toc}{chapter}{Index}
}
\makeatother

%\RequirePackage[german,english,francais]{babel}

\def\matlabcolormaptext{This colormap is similar to one shipped with Matlab$^\text{\textregistered}$ under a similar name.}

\IfFileExists{tikzlibraryspy.code.tex}{%
\usetikzlibrary{spy}
}{%
    \message{ERROR: tikz SPY library NOT available. The manual will only compile partially.^^J}%
}%

\usepackage{xparse}% for colorbrewer manual

\usetikzlibrary{
    decorations.markings,
    decorations.footprints,
    shapes.arrows,
    matrix,
    positioning,
}

\usepgfplotslibrary{
    fillbetween,
    ternary,
    smithchart,
    patchplots,
    polar,
    colormaps,
    colorbrewer,
    colortol,           % docu not ready yet
}
\pgfqkeys{/codeexample}{%
    every codeexample/.append style={
        /pgfplots/every ternary axis/.append style={
            /pgfplots/legend style={fill=graphicbackground},
        }
    },
    tabsize=4,
}

\pgfplotsmanualenableexternalizationofexpensive

%\usetikzlibrary{external}
%\tikzexternalize[prefix=figures/]{pgfplots}

\title{%
    Manual for Package \PGFPlots{}\\
    {\small 2D/3D Plots in \LaTeX{}, Version \pgfplotsversion}\\
    {\small\url{https://github.com/pgf-tikz/pgfplots}}
    %\\{\small Attention: you are using an unstable development version.}
}

\makeatletter
\long\def\abstractsmuggle{%
    \centering
    \textbf{Abstract}\\[0.5cm]

    \begin{minipage}{12cm}
        \PGFPlots{} draws high-quality function plots in normal or logarithmic
        scaling with a user-friendly interface directly in \TeX{}. The user
        supplies axis labels, legend entries and the plot coordinates for one or
        more plots and \PGFPlots{} applies axis scaling, computes any logarithms
        and axis ticks and draws the plots. It supports line plots, scatter
        plots, piecewise constant plots, bar plots, area plots, mesh and surface
        plots, patch plots, contour plots, quiver plots, histogram plots, box
        plots, polar axes, ternary diagrams, smith charts and some more. It is
        based on Till Tantau's package \PGF{}/\Tikz{}.
    \end{minipage}

}%

\expandafter\date\expandafter{\@date\\[2cm]
    \abstractsmuggle
}%
\makeatother

%\includeonly{pgfplots.reference}


\begin{document}

\def\plotcoords{%
\addplot coordinates {
(5,8.312e-02)    (17,2.547e-02)   (49,7.407e-03)
(129,2.102e-03)  (321,5.874e-04)  (769,1.623e-04)
(1793,4.442e-05) (4097,1.207e-05) (9217,3.261e-06)
};

\addplot coordinates{
(7,8.472e-02)    (31,3.044e-02)    (111,1.022e-02)
(351,3.303e-03)  (1023,1.039e-03)  (2815,3.196e-04)
(7423,9.658e-05) (18943,2.873e-05) (47103,8.437e-06)
};

\addplot coordinates{
(9,7.881e-02)     (49,3.243e-02)    (209,1.232e-02)
(769,4.454e-03)   (2561,1.551e-03)  (7937,5.236e-04)
(23297,1.723e-04) (65537,5.545e-05) (178177,1.751e-05)
};

\addplot coordinates{
(11,6.887e-02)    (71,3.177e-02)     (351,1.341e-02)
(1471,5.334e-03)  (5503,2.027e-03)   (18943,7.415e-04)
(61183,2.628e-04) (187903,9.063e-05) (553983,3.053e-05)
};

\addplot coordinates{
(13,5.755e-02)     (97,2.925e-02)     (545,1.351e-02)
(2561,5.842e-03)   (10625,2.397e-03)  (40193,9.414e-04)
(141569,3.564e-04) (471041,1.308e-04)
(1496065,4.670e-05)
};
}%


\bookmaketitle
\tableofcontents

%%%%%%%%%%%%%%%%%%%%%%%%%%%%%%%%%%%%%%%%%%%%%%%%%%%%%%%%%%%%%%%%%%%%%%%%%%%%%
%
% Package pgfplots.sty documentation.
%
% Copyright 2007/2008 by Christian Feuersaenger.
%
% This program is free software: you can redistribute it and/or modify
% it under the terms of the GNU General Public License as published by
% the Free Software Foundation, either version 3 of the License, or
% (at your option) any later version.
%
% This program is distributed in the hope that it will be useful,
% but WITHOUT ANY WARRANTY; without even the implied warranty of
% MERCHANTABILITY or FITNESS FOR A PARTICULAR PURPOSE.  See the
% GNU General Public License for more details.
%
% You should have received a copy of the GNU General Public License
% along with this program.  If not, see <http://www.gnu.org/licenses/>.
%
%
%%%%%%%%%%%%%%%%%%%%%%%%%%%%%%%%%%%%%%%%%%%%%%%%%%%%%%%%%%%%%%%%%%%%%%%%%%%%%
% SEE pgfplots-macros.tex as well!
%\pdfminorversion=5 % to allow compression
%\pdfobjcompresslevel=2
\documentclass[a4paper,openany]{book}

\let\bookmaketitle=\maketitle

% -----------------------------------
% this here is from ltxdoc:
\usepackage{doc}[=v2]
\AtBeginDocument{\MakeShortVerb{\|}}
%\setlength{\textwidth}{355pt}
%\addtolength\marginparwidth{30pt}
%\addtolength\oddsidemargin{20pt}
%\addtolength\evensidemargin{20pt}
\def\cmd#1{\cs{\expandafter\cmd@to@cs\string#1}}
\def\cmd@to@cs#1#2{\char\number`#2\relax}
\DeclareRobustCommand\cs[1]{\texttt{\char`\\#1}}
\providecommand\marg[1]{%
  {\ttfamily\char`\{}\meta{#1}{\ttfamily\char`\}}}
\providecommand\oarg[1]{%
  {\ttfamily[}\meta{#1}{\ttfamily]}}
\providecommand\parg[1]{%
  {\ttfamily(}\meta{#1}{\ttfamily)}}
\raggedbottom

% -----------------------------------
\input{pgfplots.preamble.tex}

\makeatletter
% I want a two-column index just as in pgfmanual styles. This here
% was the best way to get one:
\def\index@prologue{\section*{Index}\addcontentsline{toc}{chapter}{Index}
}
\makeatother

%\RequirePackage[german,english,francais]{babel}

\def\matlabcolormaptext{This colormap is similar to one shipped with Matlab$^\text{\textregistered}$ under a similar name.}

\IfFileExists{tikzlibraryspy.code.tex}{%
\usetikzlibrary{spy}
}{%
    \message{ERROR: tikz SPY library NOT available. The manual will only compile partially.^^J}%
}%

\usepackage{xparse}% for colorbrewer manual

\usetikzlibrary{
    decorations.markings,
    decorations.footprints,
    shapes.arrows,
    matrix,
    positioning,
}

\usepgfplotslibrary{
    fillbetween,
    ternary,
    smithchart,
    patchplots,
    polar,
    colormaps,
    colorbrewer,
    colortol,           % docu not ready yet
}
\pgfqkeys{/codeexample}{%
    every codeexample/.append style={
        /pgfplots/every ternary axis/.append style={
            /pgfplots/legend style={fill=graphicbackground},
        }
    },
    tabsize=4,
}

\pgfplotsmanualenableexternalizationofexpensive

%\usetikzlibrary{external}
%\tikzexternalize[prefix=figures/]{pgfplots}

\title{%
    Manual for Package \PGFPlots{}\\
    {\small 2D/3D Plots in \LaTeX{}, Version \pgfplotsversion}\\
    {\small\url{https://github.com/pgf-tikz/pgfplots}}
    %\\{\small Attention: you are using an unstable development version.}
}

\makeatletter
\long\def\abstractsmuggle{%
    \centering
    \textbf{Abstract}\\[0.5cm]

    \begin{minipage}{12cm}
        \PGFPlots{} draws high-quality function plots in normal or logarithmic
        scaling with a user-friendly interface directly in \TeX{}. The user
        supplies axis labels, legend entries and the plot coordinates for one or
        more plots and \PGFPlots{} applies axis scaling, computes any logarithms
        and axis ticks and draws the plots. It supports line plots, scatter
        plots, piecewise constant plots, bar plots, area plots, mesh and surface
        plots, patch plots, contour plots, quiver plots, histogram plots, box
        plots, polar axes, ternary diagrams, smith charts and some more. It is
        based on Till Tantau's package \PGF{}/\Tikz{}.
    \end{minipage}

}%

\expandafter\date\expandafter{\@date\\[2cm]
    \abstractsmuggle
}%
\makeatother

%\includeonly{pgfplots.reference}


\begin{document}

\def\plotcoords{%
\addplot coordinates {
(5,8.312e-02)    (17,2.547e-02)   (49,7.407e-03)
(129,2.102e-03)  (321,5.874e-04)  (769,1.623e-04)
(1793,4.442e-05) (4097,1.207e-05) (9217,3.261e-06)
};

\addplot coordinates{
(7,8.472e-02)    (31,3.044e-02)    (111,1.022e-02)
(351,3.303e-03)  (1023,1.039e-03)  (2815,3.196e-04)
(7423,9.658e-05) (18943,2.873e-05) (47103,8.437e-06)
};

\addplot coordinates{
(9,7.881e-02)     (49,3.243e-02)    (209,1.232e-02)
(769,4.454e-03)   (2561,1.551e-03)  (7937,5.236e-04)
(23297,1.723e-04) (65537,5.545e-05) (178177,1.751e-05)
};

\addplot coordinates{
(11,6.887e-02)    (71,3.177e-02)     (351,1.341e-02)
(1471,5.334e-03)  (5503,2.027e-03)   (18943,7.415e-04)
(61183,2.628e-04) (187903,9.063e-05) (553983,3.053e-05)
};

\addplot coordinates{
(13,5.755e-02)     (97,2.925e-02)     (545,1.351e-02)
(2561,5.842e-03)   (10625,2.397e-03)  (40193,9.414e-04)
(141569,3.564e-04) (471041,1.308e-04)
(1496065,4.670e-05)
};
}%


\bookmaketitle
\tableofcontents
\include{pgfplots.title_abstract_intro}
\include{pgfplots.preliminaries}
\include{pgfplots.intro}
\include{pgfplots.reference}
\include{pgfplots.libs}
\include{pgfplots.resources}
\include{pgfplots.importexport}
\include{pgfplots.basic.reference}

\printindex

\bibliographystyle{abbrv} %gerapali} %gerabbrv} %gerunsrt.bst} %gerabbrv}% gerplain}
\nocite{pgfplotstable}
\nocite{programmingnotes}
\bibliography{pgfplots}
\end{document}

% -----------------------------------------------------------------------------
% For Stefan Pinnow as reminder on what to look for when editing the manual
% -----------------------------------------------------------------------------
% There should be no line breaks in the following environments
% - |...|
% - \declareandlabel{...}
% - \verbpdfref{...}
% If "MakeTikzPictures" isn't running through check one of the externalized
% LOG files what is the cause of that.
% -----------------------------------------------------------------------------


%%%%%%%%%%%%%%%%%%%%%%%%%%%%%%%%%%%%%%%%%%%%%%%%%%%%%%%%%%%%%%%%%%%%%%%%%%%%%
%
% Package pgfplots.sty documentation.
%
% Copyright 2007/2008 by Christian Feuersaenger.
%
% This program is free software: you can redistribute it and/or modify
% it under the terms of the GNU General Public License as published by
% the Free Software Foundation, either version 3 of the License, or
% (at your option) any later version.
%
% This program is distributed in the hope that it will be useful,
% but WITHOUT ANY WARRANTY; without even the implied warranty of
% MERCHANTABILITY or FITNESS FOR A PARTICULAR PURPOSE.  See the
% GNU General Public License for more details.
%
% You should have received a copy of the GNU General Public License
% along with this program.  If not, see <http://www.gnu.org/licenses/>.
%
%
%%%%%%%%%%%%%%%%%%%%%%%%%%%%%%%%%%%%%%%%%%%%%%%%%%%%%%%%%%%%%%%%%%%%%%%%%%%%%
% SEE pgfplots-macros.tex as well!
%\pdfminorversion=5 % to allow compression
%\pdfobjcompresslevel=2
\documentclass[a4paper,openany]{book}

\let\bookmaketitle=\maketitle

% -----------------------------------
% this here is from ltxdoc:
\usepackage{doc}[=v2]
\AtBeginDocument{\MakeShortVerb{\|}}
%\setlength{\textwidth}{355pt}
%\addtolength\marginparwidth{30pt}
%\addtolength\oddsidemargin{20pt}
%\addtolength\evensidemargin{20pt}
\def\cmd#1{\cs{\expandafter\cmd@to@cs\string#1}}
\def\cmd@to@cs#1#2{\char\number`#2\relax}
\DeclareRobustCommand\cs[1]{\texttt{\char`\\#1}}
\providecommand\marg[1]{%
  {\ttfamily\char`\{}\meta{#1}{\ttfamily\char`\}}}
\providecommand\oarg[1]{%
  {\ttfamily[}\meta{#1}{\ttfamily]}}
\providecommand\parg[1]{%
  {\ttfamily(}\meta{#1}{\ttfamily)}}
\raggedbottom

% -----------------------------------
\input{pgfplots.preamble.tex}

\makeatletter
% I want a two-column index just as in pgfmanual styles. This here
% was the best way to get one:
\def\index@prologue{\section*{Index}\addcontentsline{toc}{chapter}{Index}
}
\makeatother

%\RequirePackage[german,english,francais]{babel}

\def\matlabcolormaptext{This colormap is similar to one shipped with Matlab$^\text{\textregistered}$ under a similar name.}

\IfFileExists{tikzlibraryspy.code.tex}{%
\usetikzlibrary{spy}
}{%
    \message{ERROR: tikz SPY library NOT available. The manual will only compile partially.^^J}%
}%

\usepackage{xparse}% for colorbrewer manual

\usetikzlibrary{
    decorations.markings,
    decorations.footprints,
    shapes.arrows,
    matrix,
    positioning,
}

\usepgfplotslibrary{
    fillbetween,
    ternary,
    smithchart,
    patchplots,
    polar,
    colormaps,
    colorbrewer,
    colortol,           % docu not ready yet
}
\pgfqkeys{/codeexample}{%
    every codeexample/.append style={
        /pgfplots/every ternary axis/.append style={
            /pgfplots/legend style={fill=graphicbackground},
        }
    },
    tabsize=4,
}

\pgfplotsmanualenableexternalizationofexpensive

%\usetikzlibrary{external}
%\tikzexternalize[prefix=figures/]{pgfplots}

\title{%
    Manual for Package \PGFPlots{}\\
    {\small 2D/3D Plots in \LaTeX{}, Version \pgfplotsversion}\\
    {\small\url{https://github.com/pgf-tikz/pgfplots}}
    %\\{\small Attention: you are using an unstable development version.}
}

\makeatletter
\long\def\abstractsmuggle{%
    \centering
    \textbf{Abstract}\\[0.5cm]

    \begin{minipage}{12cm}
        \PGFPlots{} draws high-quality function plots in normal or logarithmic
        scaling with a user-friendly interface directly in \TeX{}. The user
        supplies axis labels, legend entries and the plot coordinates for one or
        more plots and \PGFPlots{} applies axis scaling, computes any logarithms
        and axis ticks and draws the plots. It supports line plots, scatter
        plots, piecewise constant plots, bar plots, area plots, mesh and surface
        plots, patch plots, contour plots, quiver plots, histogram plots, box
        plots, polar axes, ternary diagrams, smith charts and some more. It is
        based on Till Tantau's package \PGF{}/\Tikz{}.
    \end{minipage}

}%

\expandafter\date\expandafter{\@date\\[2cm]
    \abstractsmuggle
}%
\makeatother

%\includeonly{pgfplots.reference}


\begin{document}

\def\plotcoords{%
\addplot coordinates {
(5,8.312e-02)    (17,2.547e-02)   (49,7.407e-03)
(129,2.102e-03)  (321,5.874e-04)  (769,1.623e-04)
(1793,4.442e-05) (4097,1.207e-05) (9217,3.261e-06)
};

\addplot coordinates{
(7,8.472e-02)    (31,3.044e-02)    (111,1.022e-02)
(351,3.303e-03)  (1023,1.039e-03)  (2815,3.196e-04)
(7423,9.658e-05) (18943,2.873e-05) (47103,8.437e-06)
};

\addplot coordinates{
(9,7.881e-02)     (49,3.243e-02)    (209,1.232e-02)
(769,4.454e-03)   (2561,1.551e-03)  (7937,5.236e-04)
(23297,1.723e-04) (65537,5.545e-05) (178177,1.751e-05)
};

\addplot coordinates{
(11,6.887e-02)    (71,3.177e-02)     (351,1.341e-02)
(1471,5.334e-03)  (5503,2.027e-03)   (18943,7.415e-04)
(61183,2.628e-04) (187903,9.063e-05) (553983,3.053e-05)
};

\addplot coordinates{
(13,5.755e-02)     (97,2.925e-02)     (545,1.351e-02)
(2561,5.842e-03)   (10625,2.397e-03)  (40193,9.414e-04)
(141569,3.564e-04) (471041,1.308e-04)
(1496065,4.670e-05)
};
}%


\bookmaketitle
\tableofcontents
\include{pgfplots.title_abstract_intro}
\include{pgfplots.preliminaries}
\include{pgfplots.intro}
\include{pgfplots.reference}
\include{pgfplots.libs}
\include{pgfplots.resources}
\include{pgfplots.importexport}
\include{pgfplots.basic.reference}

\printindex

\bibliographystyle{abbrv} %gerapali} %gerabbrv} %gerunsrt.bst} %gerabbrv}% gerplain}
\nocite{pgfplotstable}
\nocite{programmingnotes}
\bibliography{pgfplots}
\end{document}

% -----------------------------------------------------------------------------
% For Stefan Pinnow as reminder on what to look for when editing the manual
% -----------------------------------------------------------------------------
% There should be no line breaks in the following environments
% - |...|
% - \declareandlabel{...}
% - \verbpdfref{...}
% If "MakeTikzPictures" isn't running through check one of the externalized
% LOG files what is the cause of that.
% -----------------------------------------------------------------------------


%%%%%%%%%%%%%%%%%%%%%%%%%%%%%%%%%%%%%%%%%%%%%%%%%%%%%%%%%%%%%%%%%%%%%%%%%%%%%
%
% Package pgfplots.sty documentation.
%
% Copyright 2007/2008 by Christian Feuersaenger.
%
% This program is free software: you can redistribute it and/or modify
% it under the terms of the GNU General Public License as published by
% the Free Software Foundation, either version 3 of the License, or
% (at your option) any later version.
%
% This program is distributed in the hope that it will be useful,
% but WITHOUT ANY WARRANTY; without even the implied warranty of
% MERCHANTABILITY or FITNESS FOR A PARTICULAR PURPOSE.  See the
% GNU General Public License for more details.
%
% You should have received a copy of the GNU General Public License
% along with this program.  If not, see <http://www.gnu.org/licenses/>.
%
%
%%%%%%%%%%%%%%%%%%%%%%%%%%%%%%%%%%%%%%%%%%%%%%%%%%%%%%%%%%%%%%%%%%%%%%%%%%%%%
% SEE pgfplots-macros.tex as well!
%\pdfminorversion=5 % to allow compression
%\pdfobjcompresslevel=2
\documentclass[a4paper,openany]{book}

\let\bookmaketitle=\maketitle

% -----------------------------------
% this here is from ltxdoc:
\usepackage{doc}[=v2]
\AtBeginDocument{\MakeShortVerb{\|}}
%\setlength{\textwidth}{355pt}
%\addtolength\marginparwidth{30pt}
%\addtolength\oddsidemargin{20pt}
%\addtolength\evensidemargin{20pt}
\def\cmd#1{\cs{\expandafter\cmd@to@cs\string#1}}
\def\cmd@to@cs#1#2{\char\number`#2\relax}
\DeclareRobustCommand\cs[1]{\texttt{\char`\\#1}}
\providecommand\marg[1]{%
  {\ttfamily\char`\{}\meta{#1}{\ttfamily\char`\}}}
\providecommand\oarg[1]{%
  {\ttfamily[}\meta{#1}{\ttfamily]}}
\providecommand\parg[1]{%
  {\ttfamily(}\meta{#1}{\ttfamily)}}
\raggedbottom

% -----------------------------------
\input{pgfplots.preamble.tex}

\makeatletter
% I want a two-column index just as in pgfmanual styles. This here
% was the best way to get one:
\def\index@prologue{\section*{Index}\addcontentsline{toc}{chapter}{Index}
}
\makeatother

%\RequirePackage[german,english,francais]{babel}

\def\matlabcolormaptext{This colormap is similar to one shipped with Matlab$^\text{\textregistered}$ under a similar name.}

\IfFileExists{tikzlibraryspy.code.tex}{%
\usetikzlibrary{spy}
}{%
    \message{ERROR: tikz SPY library NOT available. The manual will only compile partially.^^J}%
}%

\usepackage{xparse}% for colorbrewer manual

\usetikzlibrary{
    decorations.markings,
    decorations.footprints,
    shapes.arrows,
    matrix,
    positioning,
}

\usepgfplotslibrary{
    fillbetween,
    ternary,
    smithchart,
    patchplots,
    polar,
    colormaps,
    colorbrewer,
    colortol,           % docu not ready yet
}
\pgfqkeys{/codeexample}{%
    every codeexample/.append style={
        /pgfplots/every ternary axis/.append style={
            /pgfplots/legend style={fill=graphicbackground},
        }
    },
    tabsize=4,
}

\pgfplotsmanualenableexternalizationofexpensive

%\usetikzlibrary{external}
%\tikzexternalize[prefix=figures/]{pgfplots}

\title{%
    Manual for Package \PGFPlots{}\\
    {\small 2D/3D Plots in \LaTeX{}, Version \pgfplotsversion}\\
    {\small\url{https://github.com/pgf-tikz/pgfplots}}
    %\\{\small Attention: you are using an unstable development version.}
}

\makeatletter
\long\def\abstractsmuggle{%
    \centering
    \textbf{Abstract}\\[0.5cm]

    \begin{minipage}{12cm}
        \PGFPlots{} draws high-quality function plots in normal or logarithmic
        scaling with a user-friendly interface directly in \TeX{}. The user
        supplies axis labels, legend entries and the plot coordinates for one or
        more plots and \PGFPlots{} applies axis scaling, computes any logarithms
        and axis ticks and draws the plots. It supports line plots, scatter
        plots, piecewise constant plots, bar plots, area plots, mesh and surface
        plots, patch plots, contour plots, quiver plots, histogram plots, box
        plots, polar axes, ternary diagrams, smith charts and some more. It is
        based on Till Tantau's package \PGF{}/\Tikz{}.
    \end{minipage}

}%

\expandafter\date\expandafter{\@date\\[2cm]
    \abstractsmuggle
}%
\makeatother

%\includeonly{pgfplots.reference}


\begin{document}

\def\plotcoords{%
\addplot coordinates {
(5,8.312e-02)    (17,2.547e-02)   (49,7.407e-03)
(129,2.102e-03)  (321,5.874e-04)  (769,1.623e-04)
(1793,4.442e-05) (4097,1.207e-05) (9217,3.261e-06)
};

\addplot coordinates{
(7,8.472e-02)    (31,3.044e-02)    (111,1.022e-02)
(351,3.303e-03)  (1023,1.039e-03)  (2815,3.196e-04)
(7423,9.658e-05) (18943,2.873e-05) (47103,8.437e-06)
};

\addplot coordinates{
(9,7.881e-02)     (49,3.243e-02)    (209,1.232e-02)
(769,4.454e-03)   (2561,1.551e-03)  (7937,5.236e-04)
(23297,1.723e-04) (65537,5.545e-05) (178177,1.751e-05)
};

\addplot coordinates{
(11,6.887e-02)    (71,3.177e-02)     (351,1.341e-02)
(1471,5.334e-03)  (5503,2.027e-03)   (18943,7.415e-04)
(61183,2.628e-04) (187903,9.063e-05) (553983,3.053e-05)
};

\addplot coordinates{
(13,5.755e-02)     (97,2.925e-02)     (545,1.351e-02)
(2561,5.842e-03)   (10625,2.397e-03)  (40193,9.414e-04)
(141569,3.564e-04) (471041,1.308e-04)
(1496065,4.670e-05)
};
}%


\bookmaketitle
\tableofcontents
\include{pgfplots.title_abstract_intro}
\include{pgfplots.preliminaries}
\include{pgfplots.intro}
\include{pgfplots.reference}
\include{pgfplots.libs}
\include{pgfplots.resources}
\include{pgfplots.importexport}
\include{pgfplots.basic.reference}

\printindex

\bibliographystyle{abbrv} %gerapali} %gerabbrv} %gerunsrt.bst} %gerabbrv}% gerplain}
\nocite{pgfplotstable}
\nocite{programmingnotes}
\bibliography{pgfplots}
\end{document}

% -----------------------------------------------------------------------------
% For Stefan Pinnow as reminder on what to look for when editing the manual
% -----------------------------------------------------------------------------
% There should be no line breaks in the following environments
% - |...|
% - \declareandlabel{...}
% - \verbpdfref{...}
% If "MakeTikzPictures" isn't running through check one of the externalized
% LOG files what is the cause of that.
% -----------------------------------------------------------------------------


%%%%%%%%%%%%%%%%%%%%%%%%%%%%%%%%%%%%%%%%%%%%%%%%%%%%%%%%%%%%%%%%%%%%%%%%%%%%%
%
% Package pgfplots.sty documentation.
%
% Copyright 2007/2008 by Christian Feuersaenger.
%
% This program is free software: you can redistribute it and/or modify
% it under the terms of the GNU General Public License as published by
% the Free Software Foundation, either version 3 of the License, or
% (at your option) any later version.
%
% This program is distributed in the hope that it will be useful,
% but WITHOUT ANY WARRANTY; without even the implied warranty of
% MERCHANTABILITY or FITNESS FOR A PARTICULAR PURPOSE.  See the
% GNU General Public License for more details.
%
% You should have received a copy of the GNU General Public License
% along with this program.  If not, see <http://www.gnu.org/licenses/>.
%
%
%%%%%%%%%%%%%%%%%%%%%%%%%%%%%%%%%%%%%%%%%%%%%%%%%%%%%%%%%%%%%%%%%%%%%%%%%%%%%
% SEE pgfplots-macros.tex as well!
%\pdfminorversion=5 % to allow compression
%\pdfobjcompresslevel=2
\documentclass[a4paper,openany]{book}

\let\bookmaketitle=\maketitle

% -----------------------------------
% this here is from ltxdoc:
\usepackage{doc}[=v2]
\AtBeginDocument{\MakeShortVerb{\|}}
%\setlength{\textwidth}{355pt}
%\addtolength\marginparwidth{30pt}
%\addtolength\oddsidemargin{20pt}
%\addtolength\evensidemargin{20pt}
\def\cmd#1{\cs{\expandafter\cmd@to@cs\string#1}}
\def\cmd@to@cs#1#2{\char\number`#2\relax}
\DeclareRobustCommand\cs[1]{\texttt{\char`\\#1}}
\providecommand\marg[1]{%
  {\ttfamily\char`\{}\meta{#1}{\ttfamily\char`\}}}
\providecommand\oarg[1]{%
  {\ttfamily[}\meta{#1}{\ttfamily]}}
\providecommand\parg[1]{%
  {\ttfamily(}\meta{#1}{\ttfamily)}}
\raggedbottom

% -----------------------------------
\input{pgfplots.preamble.tex}

\makeatletter
% I want a two-column index just as in pgfmanual styles. This here
% was the best way to get one:
\def\index@prologue{\section*{Index}\addcontentsline{toc}{chapter}{Index}
}
\makeatother

%\RequirePackage[german,english,francais]{babel}

\def\matlabcolormaptext{This colormap is similar to one shipped with Matlab$^\text{\textregistered}$ under a similar name.}

\IfFileExists{tikzlibraryspy.code.tex}{%
\usetikzlibrary{spy}
}{%
    \message{ERROR: tikz SPY library NOT available. The manual will only compile partially.^^J}%
}%

\usepackage{xparse}% for colorbrewer manual

\usetikzlibrary{
    decorations.markings,
    decorations.footprints,
    shapes.arrows,
    matrix,
    positioning,
}

\usepgfplotslibrary{
    fillbetween,
    ternary,
    smithchart,
    patchplots,
    polar,
    colormaps,
    colorbrewer,
    colortol,           % docu not ready yet
}
\pgfqkeys{/codeexample}{%
    every codeexample/.append style={
        /pgfplots/every ternary axis/.append style={
            /pgfplots/legend style={fill=graphicbackground},
        }
    },
    tabsize=4,
}

\pgfplotsmanualenableexternalizationofexpensive

%\usetikzlibrary{external}
%\tikzexternalize[prefix=figures/]{pgfplots}

\title{%
    Manual for Package \PGFPlots{}\\
    {\small 2D/3D Plots in \LaTeX{}, Version \pgfplotsversion}\\
    {\small\url{https://github.com/pgf-tikz/pgfplots}}
    %\\{\small Attention: you are using an unstable development version.}
}

\makeatletter
\long\def\abstractsmuggle{%
    \centering
    \textbf{Abstract}\\[0.5cm]

    \begin{minipage}{12cm}
        \PGFPlots{} draws high-quality function plots in normal or logarithmic
        scaling with a user-friendly interface directly in \TeX{}. The user
        supplies axis labels, legend entries and the plot coordinates for one or
        more plots and \PGFPlots{} applies axis scaling, computes any logarithms
        and axis ticks and draws the plots. It supports line plots, scatter
        plots, piecewise constant plots, bar plots, area plots, mesh and surface
        plots, patch plots, contour plots, quiver plots, histogram plots, box
        plots, polar axes, ternary diagrams, smith charts and some more. It is
        based on Till Tantau's package \PGF{}/\Tikz{}.
    \end{minipage}

}%

\expandafter\date\expandafter{\@date\\[2cm]
    \abstractsmuggle
}%
\makeatother

%\includeonly{pgfplots.reference}


\begin{document}

\def\plotcoords{%
\addplot coordinates {
(5,8.312e-02)    (17,2.547e-02)   (49,7.407e-03)
(129,2.102e-03)  (321,5.874e-04)  (769,1.623e-04)
(1793,4.442e-05) (4097,1.207e-05) (9217,3.261e-06)
};

\addplot coordinates{
(7,8.472e-02)    (31,3.044e-02)    (111,1.022e-02)
(351,3.303e-03)  (1023,1.039e-03)  (2815,3.196e-04)
(7423,9.658e-05) (18943,2.873e-05) (47103,8.437e-06)
};

\addplot coordinates{
(9,7.881e-02)     (49,3.243e-02)    (209,1.232e-02)
(769,4.454e-03)   (2561,1.551e-03)  (7937,5.236e-04)
(23297,1.723e-04) (65537,5.545e-05) (178177,1.751e-05)
};

\addplot coordinates{
(11,6.887e-02)    (71,3.177e-02)     (351,1.341e-02)
(1471,5.334e-03)  (5503,2.027e-03)   (18943,7.415e-04)
(61183,2.628e-04) (187903,9.063e-05) (553983,3.053e-05)
};

\addplot coordinates{
(13,5.755e-02)     (97,2.925e-02)     (545,1.351e-02)
(2561,5.842e-03)   (10625,2.397e-03)  (40193,9.414e-04)
(141569,3.564e-04) (471041,1.308e-04)
(1496065,4.670e-05)
};
}%


\bookmaketitle
\tableofcontents
\include{pgfplots.title_abstract_intro}
\include{pgfplots.preliminaries}
\include{pgfplots.intro}
\include{pgfplots.reference}
\include{pgfplots.libs}
\include{pgfplots.resources}
\include{pgfplots.importexport}
\include{pgfplots.basic.reference}

\printindex

\bibliographystyle{abbrv} %gerapali} %gerabbrv} %gerunsrt.bst} %gerabbrv}% gerplain}
\nocite{pgfplotstable}
\nocite{programmingnotes}
\bibliography{pgfplots}
\end{document}

% -----------------------------------------------------------------------------
% For Stefan Pinnow as reminder on what to look for when editing the manual
% -----------------------------------------------------------------------------
% There should be no line breaks in the following environments
% - |...|
% - \declareandlabel{...}
% - \verbpdfref{...}
% If "MakeTikzPictures" isn't running through check one of the externalized
% LOG files what is the cause of that.
% -----------------------------------------------------------------------------


%%%%%%%%%%%%%%%%%%%%%%%%%%%%%%%%%%%%%%%%%%%%%%%%%%%%%%%%%%%%%%%%%%%%%%%%%%%%%
%
% Package pgfplots.sty documentation.
%
% Copyright 2007/2008 by Christian Feuersaenger.
%
% This program is free software: you can redistribute it and/or modify
% it under the terms of the GNU General Public License as published by
% the Free Software Foundation, either version 3 of the License, or
% (at your option) any later version.
%
% This program is distributed in the hope that it will be useful,
% but WITHOUT ANY WARRANTY; without even the implied warranty of
% MERCHANTABILITY or FITNESS FOR A PARTICULAR PURPOSE.  See the
% GNU General Public License for more details.
%
% You should have received a copy of the GNU General Public License
% along with this program.  If not, see <http://www.gnu.org/licenses/>.
%
%
%%%%%%%%%%%%%%%%%%%%%%%%%%%%%%%%%%%%%%%%%%%%%%%%%%%%%%%%%%%%%%%%%%%%%%%%%%%%%
% SEE pgfplots-macros.tex as well!
%\pdfminorversion=5 % to allow compression
%\pdfobjcompresslevel=2
\documentclass[a4paper,openany]{book}

\let\bookmaketitle=\maketitle

% -----------------------------------
% this here is from ltxdoc:
\usepackage{doc}[=v2]
\AtBeginDocument{\MakeShortVerb{\|}}
%\setlength{\textwidth}{355pt}
%\addtolength\marginparwidth{30pt}
%\addtolength\oddsidemargin{20pt}
%\addtolength\evensidemargin{20pt}
\def\cmd#1{\cs{\expandafter\cmd@to@cs\string#1}}
\def\cmd@to@cs#1#2{\char\number`#2\relax}
\DeclareRobustCommand\cs[1]{\texttt{\char`\\#1}}
\providecommand\marg[1]{%
  {\ttfamily\char`\{}\meta{#1}{\ttfamily\char`\}}}
\providecommand\oarg[1]{%
  {\ttfamily[}\meta{#1}{\ttfamily]}}
\providecommand\parg[1]{%
  {\ttfamily(}\meta{#1}{\ttfamily)}}
\raggedbottom

% -----------------------------------
\input{pgfplots.preamble.tex}

\makeatletter
% I want a two-column index just as in pgfmanual styles. This here
% was the best way to get one:
\def\index@prologue{\section*{Index}\addcontentsline{toc}{chapter}{Index}
}
\makeatother

%\RequirePackage[german,english,francais]{babel}

\def\matlabcolormaptext{This colormap is similar to one shipped with Matlab$^\text{\textregistered}$ under a similar name.}

\IfFileExists{tikzlibraryspy.code.tex}{%
\usetikzlibrary{spy}
}{%
    \message{ERROR: tikz SPY library NOT available. The manual will only compile partially.^^J}%
}%

\usepackage{xparse}% for colorbrewer manual

\usetikzlibrary{
    decorations.markings,
    decorations.footprints,
    shapes.arrows,
    matrix,
    positioning,
}

\usepgfplotslibrary{
    fillbetween,
    ternary,
    smithchart,
    patchplots,
    polar,
    colormaps,
    colorbrewer,
    colortol,           % docu not ready yet
}
\pgfqkeys{/codeexample}{%
    every codeexample/.append style={
        /pgfplots/every ternary axis/.append style={
            /pgfplots/legend style={fill=graphicbackground},
        }
    },
    tabsize=4,
}

\pgfplotsmanualenableexternalizationofexpensive

%\usetikzlibrary{external}
%\tikzexternalize[prefix=figures/]{pgfplots}

\title{%
    Manual for Package \PGFPlots{}\\
    {\small 2D/3D Plots in \LaTeX{}, Version \pgfplotsversion}\\
    {\small\url{https://github.com/pgf-tikz/pgfplots}}
    %\\{\small Attention: you are using an unstable development version.}
}

\makeatletter
\long\def\abstractsmuggle{%
    \centering
    \textbf{Abstract}\\[0.5cm]

    \begin{minipage}{12cm}
        \PGFPlots{} draws high-quality function plots in normal or logarithmic
        scaling with a user-friendly interface directly in \TeX{}. The user
        supplies axis labels, legend entries and the plot coordinates for one or
        more plots and \PGFPlots{} applies axis scaling, computes any logarithms
        and axis ticks and draws the plots. It supports line plots, scatter
        plots, piecewise constant plots, bar plots, area plots, mesh and surface
        plots, patch plots, contour plots, quiver plots, histogram plots, box
        plots, polar axes, ternary diagrams, smith charts and some more. It is
        based on Till Tantau's package \PGF{}/\Tikz{}.
    \end{minipage}

}%

\expandafter\date\expandafter{\@date\\[2cm]
    \abstractsmuggle
}%
\makeatother

%\includeonly{pgfplots.reference}


\begin{document}

\def\plotcoords{%
\addplot coordinates {
(5,8.312e-02)    (17,2.547e-02)   (49,7.407e-03)
(129,2.102e-03)  (321,5.874e-04)  (769,1.623e-04)
(1793,4.442e-05) (4097,1.207e-05) (9217,3.261e-06)
};

\addplot coordinates{
(7,8.472e-02)    (31,3.044e-02)    (111,1.022e-02)
(351,3.303e-03)  (1023,1.039e-03)  (2815,3.196e-04)
(7423,9.658e-05) (18943,2.873e-05) (47103,8.437e-06)
};

\addplot coordinates{
(9,7.881e-02)     (49,3.243e-02)    (209,1.232e-02)
(769,4.454e-03)   (2561,1.551e-03)  (7937,5.236e-04)
(23297,1.723e-04) (65537,5.545e-05) (178177,1.751e-05)
};

\addplot coordinates{
(11,6.887e-02)    (71,3.177e-02)     (351,1.341e-02)
(1471,5.334e-03)  (5503,2.027e-03)   (18943,7.415e-04)
(61183,2.628e-04) (187903,9.063e-05) (553983,3.053e-05)
};

\addplot coordinates{
(13,5.755e-02)     (97,2.925e-02)     (545,1.351e-02)
(2561,5.842e-03)   (10625,2.397e-03)  (40193,9.414e-04)
(141569,3.564e-04) (471041,1.308e-04)
(1496065,4.670e-05)
};
}%


\bookmaketitle
\tableofcontents
\include{pgfplots.title_abstract_intro}
\include{pgfplots.preliminaries}
\include{pgfplots.intro}
\include{pgfplots.reference}
\include{pgfplots.libs}
\include{pgfplots.resources}
\include{pgfplots.importexport}
\include{pgfplots.basic.reference}

\printindex

\bibliographystyle{abbrv} %gerapali} %gerabbrv} %gerunsrt.bst} %gerabbrv}% gerplain}
\nocite{pgfplotstable}
\nocite{programmingnotes}
\bibliography{pgfplots}
\end{document}

% -----------------------------------------------------------------------------
% For Stefan Pinnow as reminder on what to look for when editing the manual
% -----------------------------------------------------------------------------
% There should be no line breaks in the following environments
% - |...|
% - \declareandlabel{...}
% - \verbpdfref{...}
% If "MakeTikzPictures" isn't running through check one of the externalized
% LOG files what is the cause of that.
% -----------------------------------------------------------------------------


%%%%%%%%%%%%%%%%%%%%%%%%%%%%%%%%%%%%%%%%%%%%%%%%%%%%%%%%%%%%%%%%%%%%%%%%%%%%%
%
% Package pgfplots.sty documentation.
%
% Copyright 2007/2008 by Christian Feuersaenger.
%
% This program is free software: you can redistribute it and/or modify
% it under the terms of the GNU General Public License as published by
% the Free Software Foundation, either version 3 of the License, or
% (at your option) any later version.
%
% This program is distributed in the hope that it will be useful,
% but WITHOUT ANY WARRANTY; without even the implied warranty of
% MERCHANTABILITY or FITNESS FOR A PARTICULAR PURPOSE.  See the
% GNU General Public License for more details.
%
% You should have received a copy of the GNU General Public License
% along with this program.  If not, see <http://www.gnu.org/licenses/>.
%
%
%%%%%%%%%%%%%%%%%%%%%%%%%%%%%%%%%%%%%%%%%%%%%%%%%%%%%%%%%%%%%%%%%%%%%%%%%%%%%
% SEE pgfplots-macros.tex as well!
%\pdfminorversion=5 % to allow compression
%\pdfobjcompresslevel=2
\documentclass[a4paper,openany]{book}

\let\bookmaketitle=\maketitle

% -----------------------------------
% this here is from ltxdoc:
\usepackage{doc}[=v2]
\AtBeginDocument{\MakeShortVerb{\|}}
%\setlength{\textwidth}{355pt}
%\addtolength\marginparwidth{30pt}
%\addtolength\oddsidemargin{20pt}
%\addtolength\evensidemargin{20pt}
\def\cmd#1{\cs{\expandafter\cmd@to@cs\string#1}}
\def\cmd@to@cs#1#2{\char\number`#2\relax}
\DeclareRobustCommand\cs[1]{\texttt{\char`\\#1}}
\providecommand\marg[1]{%
  {\ttfamily\char`\{}\meta{#1}{\ttfamily\char`\}}}
\providecommand\oarg[1]{%
  {\ttfamily[}\meta{#1}{\ttfamily]}}
\providecommand\parg[1]{%
  {\ttfamily(}\meta{#1}{\ttfamily)}}
\raggedbottom

% -----------------------------------
\input{pgfplots.preamble.tex}

\makeatletter
% I want a two-column index just as in pgfmanual styles. This here
% was the best way to get one:
\def\index@prologue{\section*{Index}\addcontentsline{toc}{chapter}{Index}
}
\makeatother

%\RequirePackage[german,english,francais]{babel}

\def\matlabcolormaptext{This colormap is similar to one shipped with Matlab$^\text{\textregistered}$ under a similar name.}

\IfFileExists{tikzlibraryspy.code.tex}{%
\usetikzlibrary{spy}
}{%
    \message{ERROR: tikz SPY library NOT available. The manual will only compile partially.^^J}%
}%

\usepackage{xparse}% for colorbrewer manual

\usetikzlibrary{
    decorations.markings,
    decorations.footprints,
    shapes.arrows,
    matrix,
    positioning,
}

\usepgfplotslibrary{
    fillbetween,
    ternary,
    smithchart,
    patchplots,
    polar,
    colormaps,
    colorbrewer,
    colortol,           % docu not ready yet
}
\pgfqkeys{/codeexample}{%
    every codeexample/.append style={
        /pgfplots/every ternary axis/.append style={
            /pgfplots/legend style={fill=graphicbackground},
        }
    },
    tabsize=4,
}

\pgfplotsmanualenableexternalizationofexpensive

%\usetikzlibrary{external}
%\tikzexternalize[prefix=figures/]{pgfplots}

\title{%
    Manual for Package \PGFPlots{}\\
    {\small 2D/3D Plots in \LaTeX{}, Version \pgfplotsversion}\\
    {\small\url{https://github.com/pgf-tikz/pgfplots}}
    %\\{\small Attention: you are using an unstable development version.}
}

\makeatletter
\long\def\abstractsmuggle{%
    \centering
    \textbf{Abstract}\\[0.5cm]

    \begin{minipage}{12cm}
        \PGFPlots{} draws high-quality function plots in normal or logarithmic
        scaling with a user-friendly interface directly in \TeX{}. The user
        supplies axis labels, legend entries and the plot coordinates for one or
        more plots and \PGFPlots{} applies axis scaling, computes any logarithms
        and axis ticks and draws the plots. It supports line plots, scatter
        plots, piecewise constant plots, bar plots, area plots, mesh and surface
        plots, patch plots, contour plots, quiver plots, histogram plots, box
        plots, polar axes, ternary diagrams, smith charts and some more. It is
        based on Till Tantau's package \PGF{}/\Tikz{}.
    \end{minipage}

}%

\expandafter\date\expandafter{\@date\\[2cm]
    \abstractsmuggle
}%
\makeatother

%\includeonly{pgfplots.reference}


\begin{document}

\def\plotcoords{%
\addplot coordinates {
(5,8.312e-02)    (17,2.547e-02)   (49,7.407e-03)
(129,2.102e-03)  (321,5.874e-04)  (769,1.623e-04)
(1793,4.442e-05) (4097,1.207e-05) (9217,3.261e-06)
};

\addplot coordinates{
(7,8.472e-02)    (31,3.044e-02)    (111,1.022e-02)
(351,3.303e-03)  (1023,1.039e-03)  (2815,3.196e-04)
(7423,9.658e-05) (18943,2.873e-05) (47103,8.437e-06)
};

\addplot coordinates{
(9,7.881e-02)     (49,3.243e-02)    (209,1.232e-02)
(769,4.454e-03)   (2561,1.551e-03)  (7937,5.236e-04)
(23297,1.723e-04) (65537,5.545e-05) (178177,1.751e-05)
};

\addplot coordinates{
(11,6.887e-02)    (71,3.177e-02)     (351,1.341e-02)
(1471,5.334e-03)  (5503,2.027e-03)   (18943,7.415e-04)
(61183,2.628e-04) (187903,9.063e-05) (553983,3.053e-05)
};

\addplot coordinates{
(13,5.755e-02)     (97,2.925e-02)     (545,1.351e-02)
(2561,5.842e-03)   (10625,2.397e-03)  (40193,9.414e-04)
(141569,3.564e-04) (471041,1.308e-04)
(1496065,4.670e-05)
};
}%


\bookmaketitle
\tableofcontents
\include{pgfplots.title_abstract_intro}
\include{pgfplots.preliminaries}
\include{pgfplots.intro}
\include{pgfplots.reference}
\include{pgfplots.libs}
\include{pgfplots.resources}
\include{pgfplots.importexport}
\include{pgfplots.basic.reference}

\printindex

\bibliographystyle{abbrv} %gerapali} %gerabbrv} %gerunsrt.bst} %gerabbrv}% gerplain}
\nocite{pgfplotstable}
\nocite{programmingnotes}
\bibliography{pgfplots}
\end{document}

% -----------------------------------------------------------------------------
% For Stefan Pinnow as reminder on what to look for when editing the manual
% -----------------------------------------------------------------------------
% There should be no line breaks in the following environments
% - |...|
% - \declareandlabel{...}
% - \verbpdfref{...}
% If "MakeTikzPictures" isn't running through check one of the externalized
% LOG files what is the cause of that.
% -----------------------------------------------------------------------------


%%%%%%%%%%%%%%%%%%%%%%%%%%%%%%%%%%%%%%%%%%%%%%%%%%%%%%%%%%%%%%%%%%%%%%%%%%%%%
%
% Package pgfplots.sty documentation.
%
% Copyright 2007/2008 by Christian Feuersaenger.
%
% This program is free software: you can redistribute it and/or modify
% it under the terms of the GNU General Public License as published by
% the Free Software Foundation, either version 3 of the License, or
% (at your option) any later version.
%
% This program is distributed in the hope that it will be useful,
% but WITHOUT ANY WARRANTY; without even the implied warranty of
% MERCHANTABILITY or FITNESS FOR A PARTICULAR PURPOSE.  See the
% GNU General Public License for more details.
%
% You should have received a copy of the GNU General Public License
% along with this program.  If not, see <http://www.gnu.org/licenses/>.
%
%
%%%%%%%%%%%%%%%%%%%%%%%%%%%%%%%%%%%%%%%%%%%%%%%%%%%%%%%%%%%%%%%%%%%%%%%%%%%%%
% SEE pgfplots-macros.tex as well!
%\pdfminorversion=5 % to allow compression
%\pdfobjcompresslevel=2
\documentclass[a4paper,openany]{book}

\let\bookmaketitle=\maketitle

% -----------------------------------
% this here is from ltxdoc:
\usepackage{doc}[=v2]
\AtBeginDocument{\MakeShortVerb{\|}}
%\setlength{\textwidth}{355pt}
%\addtolength\marginparwidth{30pt}
%\addtolength\oddsidemargin{20pt}
%\addtolength\evensidemargin{20pt}
\def\cmd#1{\cs{\expandafter\cmd@to@cs\string#1}}
\def\cmd@to@cs#1#2{\char\number`#2\relax}
\DeclareRobustCommand\cs[1]{\texttt{\char`\\#1}}
\providecommand\marg[1]{%
  {\ttfamily\char`\{}\meta{#1}{\ttfamily\char`\}}}
\providecommand\oarg[1]{%
  {\ttfamily[}\meta{#1}{\ttfamily]}}
\providecommand\parg[1]{%
  {\ttfamily(}\meta{#1}{\ttfamily)}}
\raggedbottom

% -----------------------------------
\input{pgfplots.preamble.tex}

\makeatletter
% I want a two-column index just as in pgfmanual styles. This here
% was the best way to get one:
\def\index@prologue{\section*{Index}\addcontentsline{toc}{chapter}{Index}
}
\makeatother

%\RequirePackage[german,english,francais]{babel}

\def\matlabcolormaptext{This colormap is similar to one shipped with Matlab$^\text{\textregistered}$ under a similar name.}

\IfFileExists{tikzlibraryspy.code.tex}{%
\usetikzlibrary{spy}
}{%
    \message{ERROR: tikz SPY library NOT available. The manual will only compile partially.^^J}%
}%

\usepackage{xparse}% for colorbrewer manual

\usetikzlibrary{
    decorations.markings,
    decorations.footprints,
    shapes.arrows,
    matrix,
    positioning,
}

\usepgfplotslibrary{
    fillbetween,
    ternary,
    smithchart,
    patchplots,
    polar,
    colormaps,
    colorbrewer,
    colortol,           % docu not ready yet
}
\pgfqkeys{/codeexample}{%
    every codeexample/.append style={
        /pgfplots/every ternary axis/.append style={
            /pgfplots/legend style={fill=graphicbackground},
        }
    },
    tabsize=4,
}

\pgfplotsmanualenableexternalizationofexpensive

%\usetikzlibrary{external}
%\tikzexternalize[prefix=figures/]{pgfplots}

\title{%
    Manual for Package \PGFPlots{}\\
    {\small 2D/3D Plots in \LaTeX{}, Version \pgfplotsversion}\\
    {\small\url{https://github.com/pgf-tikz/pgfplots}}
    %\\{\small Attention: you are using an unstable development version.}
}

\makeatletter
\long\def\abstractsmuggle{%
    \centering
    \textbf{Abstract}\\[0.5cm]

    \begin{minipage}{12cm}
        \PGFPlots{} draws high-quality function plots in normal or logarithmic
        scaling with a user-friendly interface directly in \TeX{}. The user
        supplies axis labels, legend entries and the plot coordinates for one or
        more plots and \PGFPlots{} applies axis scaling, computes any logarithms
        and axis ticks and draws the plots. It supports line plots, scatter
        plots, piecewise constant plots, bar plots, area plots, mesh and surface
        plots, patch plots, contour plots, quiver plots, histogram plots, box
        plots, polar axes, ternary diagrams, smith charts and some more. It is
        based on Till Tantau's package \PGF{}/\Tikz{}.
    \end{minipage}

}%

\expandafter\date\expandafter{\@date\\[2cm]
    \abstractsmuggle
}%
\makeatother

%\includeonly{pgfplots.reference}


\begin{document}

\def\plotcoords{%
\addplot coordinates {
(5,8.312e-02)    (17,2.547e-02)   (49,7.407e-03)
(129,2.102e-03)  (321,5.874e-04)  (769,1.623e-04)
(1793,4.442e-05) (4097,1.207e-05) (9217,3.261e-06)
};

\addplot coordinates{
(7,8.472e-02)    (31,3.044e-02)    (111,1.022e-02)
(351,3.303e-03)  (1023,1.039e-03)  (2815,3.196e-04)
(7423,9.658e-05) (18943,2.873e-05) (47103,8.437e-06)
};

\addplot coordinates{
(9,7.881e-02)     (49,3.243e-02)    (209,1.232e-02)
(769,4.454e-03)   (2561,1.551e-03)  (7937,5.236e-04)
(23297,1.723e-04) (65537,5.545e-05) (178177,1.751e-05)
};

\addplot coordinates{
(11,6.887e-02)    (71,3.177e-02)     (351,1.341e-02)
(1471,5.334e-03)  (5503,2.027e-03)   (18943,7.415e-04)
(61183,2.628e-04) (187903,9.063e-05) (553983,3.053e-05)
};

\addplot coordinates{
(13,5.755e-02)     (97,2.925e-02)     (545,1.351e-02)
(2561,5.842e-03)   (10625,2.397e-03)  (40193,9.414e-04)
(141569,3.564e-04) (471041,1.308e-04)
(1496065,4.670e-05)
};
}%


\bookmaketitle
\tableofcontents
\include{pgfplots.title_abstract_intro}
\include{pgfplots.preliminaries}
\include{pgfplots.intro}
\include{pgfplots.reference}
\include{pgfplots.libs}
\include{pgfplots.resources}
\include{pgfplots.importexport}
\include{pgfplots.basic.reference}

\printindex

\bibliographystyle{abbrv} %gerapali} %gerabbrv} %gerunsrt.bst} %gerabbrv}% gerplain}
\nocite{pgfplotstable}
\nocite{programmingnotes}
\bibliography{pgfplots}
\end{document}

% -----------------------------------------------------------------------------
% For Stefan Pinnow as reminder on what to look for when editing the manual
% -----------------------------------------------------------------------------
% There should be no line breaks in the following environments
% - |...|
% - \declareandlabel{...}
% - \verbpdfref{...}
% If "MakeTikzPictures" isn't running through check one of the externalized
% LOG files what is the cause of that.
% -----------------------------------------------------------------------------

\include{pgfplots.basic.reference}

\printindex

\bibliographystyle{abbrv} %gerapali} %gerabbrv} %gerunsrt.bst} %gerabbrv}% gerplain}
\nocite{pgfplotstable}
\nocite{programmingnotes}
\bibliography{pgfplots}
\end{document}

% -----------------------------------------------------------------------------
% For Stefan Pinnow as reminder on what to look for when editing the manual
% -----------------------------------------------------------------------------
% There should be no line breaks in the following environments
% - |...|
% - \declareandlabel{...}
% - \verbpdfref{...}
% If "MakeTikzPictures" isn't running through check one of the externalized
% LOG files what is the cause of that.
% -----------------------------------------------------------------------------


%%%%%%%%%%%%%%%%%%%%%%%%%%%%%%%%%%%%%%%%%%%%%%%%%%%%%%%%%%%%%%%%%%%%%%%%%%%%%
%
% Package pgfplots.sty documentation.
%
% Copyright 2007/2008 by Christian Feuersaenger.
%
% This program is free software: you can redistribute it and/or modify
% it under the terms of the GNU General Public License as published by
% the Free Software Foundation, either version 3 of the License, or
% (at your option) any later version.
%
% This program is distributed in the hope that it will be useful,
% but WITHOUT ANY WARRANTY; without even the implied warranty of
% MERCHANTABILITY or FITNESS FOR A PARTICULAR PURPOSE.  See the
% GNU General Public License for more details.
%
% You should have received a copy of the GNU General Public License
% along with this program.  If not, see <http://www.gnu.org/licenses/>.
%
%
%%%%%%%%%%%%%%%%%%%%%%%%%%%%%%%%%%%%%%%%%%%%%%%%%%%%%%%%%%%%%%%%%%%%%%%%%%%%%
% SEE pgfplots-macros.tex as well!
%\pdfminorversion=5 % to allow compression
%\pdfobjcompresslevel=2
\documentclass[a4paper,openany]{book}

\let\bookmaketitle=\maketitle

% -----------------------------------
% this here is from ltxdoc:
\usepackage{doc}[=v2]
\AtBeginDocument{\MakeShortVerb{\|}}
%\setlength{\textwidth}{355pt}
%\addtolength\marginparwidth{30pt}
%\addtolength\oddsidemargin{20pt}
%\addtolength\evensidemargin{20pt}
\def\cmd#1{\cs{\expandafter\cmd@to@cs\string#1}}
\def\cmd@to@cs#1#2{\char\number`#2\relax}
\DeclareRobustCommand\cs[1]{\texttt{\char`\\#1}}
\providecommand\marg[1]{%
  {\ttfamily\char`\{}\meta{#1}{\ttfamily\char`\}}}
\providecommand\oarg[1]{%
  {\ttfamily[}\meta{#1}{\ttfamily]}}
\providecommand\parg[1]{%
  {\ttfamily(}\meta{#1}{\ttfamily)}}
\raggedbottom

% -----------------------------------
\input{pgfplots.preamble.tex}

\makeatletter
% I want a two-column index just as in pgfmanual styles. This here
% was the best way to get one:
\def\index@prologue{\section*{Index}\addcontentsline{toc}{chapter}{Index}
}
\makeatother

%\RequirePackage[german,english,francais]{babel}

\def\matlabcolormaptext{This colormap is similar to one shipped with Matlab$^\text{\textregistered}$ under a similar name.}

\IfFileExists{tikzlibraryspy.code.tex}{%
\usetikzlibrary{spy}
}{%
    \message{ERROR: tikz SPY library NOT available. The manual will only compile partially.^^J}%
}%

\usepackage{xparse}% for colorbrewer manual

\usetikzlibrary{
    decorations.markings,
    decorations.footprints,
    shapes.arrows,
    matrix,
    positioning,
}

\usepgfplotslibrary{
    fillbetween,
    ternary,
    smithchart,
    patchplots,
    polar,
    colormaps,
    colorbrewer,
    colortol,           % docu not ready yet
}
\pgfqkeys{/codeexample}{%
    every codeexample/.append style={
        /pgfplots/every ternary axis/.append style={
            /pgfplots/legend style={fill=graphicbackground},
        }
    },
    tabsize=4,
}

\pgfplotsmanualenableexternalizationofexpensive

%\usetikzlibrary{external}
%\tikzexternalize[prefix=figures/]{pgfplots}

\title{%
    Manual for Package \PGFPlots{}\\
    {\small 2D/3D Plots in \LaTeX{}, Version \pgfplotsversion}\\
    {\small\url{https://github.com/pgf-tikz/pgfplots}}
    %\\{\small Attention: you are using an unstable development version.}
}

\makeatletter
\long\def\abstractsmuggle{%
    \centering
    \textbf{Abstract}\\[0.5cm]

    \begin{minipage}{12cm}
        \PGFPlots{} draws high-quality function plots in normal or logarithmic
        scaling with a user-friendly interface directly in \TeX{}. The user
        supplies axis labels, legend entries and the plot coordinates for one or
        more plots and \PGFPlots{} applies axis scaling, computes any logarithms
        and axis ticks and draws the plots. It supports line plots, scatter
        plots, piecewise constant plots, bar plots, area plots, mesh and surface
        plots, patch plots, contour plots, quiver plots, histogram plots, box
        plots, polar axes, ternary diagrams, smith charts and some more. It is
        based on Till Tantau's package \PGF{}/\Tikz{}.
    \end{minipage}

}%

\expandafter\date\expandafter{\@date\\[2cm]
    \abstractsmuggle
}%
\makeatother

%\includeonly{pgfplots.reference}


\begin{document}

\def\plotcoords{%
\addplot coordinates {
(5,8.312e-02)    (17,2.547e-02)   (49,7.407e-03)
(129,2.102e-03)  (321,5.874e-04)  (769,1.623e-04)
(1793,4.442e-05) (4097,1.207e-05) (9217,3.261e-06)
};

\addplot coordinates{
(7,8.472e-02)    (31,3.044e-02)    (111,1.022e-02)
(351,3.303e-03)  (1023,1.039e-03)  (2815,3.196e-04)
(7423,9.658e-05) (18943,2.873e-05) (47103,8.437e-06)
};

\addplot coordinates{
(9,7.881e-02)     (49,3.243e-02)    (209,1.232e-02)
(769,4.454e-03)   (2561,1.551e-03)  (7937,5.236e-04)
(23297,1.723e-04) (65537,5.545e-05) (178177,1.751e-05)
};

\addplot coordinates{
(11,6.887e-02)    (71,3.177e-02)     (351,1.341e-02)
(1471,5.334e-03)  (5503,2.027e-03)   (18943,7.415e-04)
(61183,2.628e-04) (187903,9.063e-05) (553983,3.053e-05)
};

\addplot coordinates{
(13,5.755e-02)     (97,2.925e-02)     (545,1.351e-02)
(2561,5.842e-03)   (10625,2.397e-03)  (40193,9.414e-04)
(141569,3.564e-04) (471041,1.308e-04)
(1496065,4.670e-05)
};
}%


\bookmaketitle
\tableofcontents

%%%%%%%%%%%%%%%%%%%%%%%%%%%%%%%%%%%%%%%%%%%%%%%%%%%%%%%%%%%%%%%%%%%%%%%%%%%%%
%
% Package pgfplots.sty documentation.
%
% Copyright 2007/2008 by Christian Feuersaenger.
%
% This program is free software: you can redistribute it and/or modify
% it under the terms of the GNU General Public License as published by
% the Free Software Foundation, either version 3 of the License, or
% (at your option) any later version.
%
% This program is distributed in the hope that it will be useful,
% but WITHOUT ANY WARRANTY; without even the implied warranty of
% MERCHANTABILITY or FITNESS FOR A PARTICULAR PURPOSE.  See the
% GNU General Public License for more details.
%
% You should have received a copy of the GNU General Public License
% along with this program.  If not, see <http://www.gnu.org/licenses/>.
%
%
%%%%%%%%%%%%%%%%%%%%%%%%%%%%%%%%%%%%%%%%%%%%%%%%%%%%%%%%%%%%%%%%%%%%%%%%%%%%%
% SEE pgfplots-macros.tex as well!
%\pdfminorversion=5 % to allow compression
%\pdfobjcompresslevel=2
\documentclass[a4paper,openany]{book}

\let\bookmaketitle=\maketitle

% -----------------------------------
% this here is from ltxdoc:
\usepackage{doc}[=v2]
\AtBeginDocument{\MakeShortVerb{\|}}
%\setlength{\textwidth}{355pt}
%\addtolength\marginparwidth{30pt}
%\addtolength\oddsidemargin{20pt}
%\addtolength\evensidemargin{20pt}
\def\cmd#1{\cs{\expandafter\cmd@to@cs\string#1}}
\def\cmd@to@cs#1#2{\char\number`#2\relax}
\DeclareRobustCommand\cs[1]{\texttt{\char`\\#1}}
\providecommand\marg[1]{%
  {\ttfamily\char`\{}\meta{#1}{\ttfamily\char`\}}}
\providecommand\oarg[1]{%
  {\ttfamily[}\meta{#1}{\ttfamily]}}
\providecommand\parg[1]{%
  {\ttfamily(}\meta{#1}{\ttfamily)}}
\raggedbottom

% -----------------------------------
\input{pgfplots.preamble.tex}

\makeatletter
% I want a two-column index just as in pgfmanual styles. This here
% was the best way to get one:
\def\index@prologue{\section*{Index}\addcontentsline{toc}{chapter}{Index}
}
\makeatother

%\RequirePackage[german,english,francais]{babel}

\def\matlabcolormaptext{This colormap is similar to one shipped with Matlab$^\text{\textregistered}$ under a similar name.}

\IfFileExists{tikzlibraryspy.code.tex}{%
\usetikzlibrary{spy}
}{%
    \message{ERROR: tikz SPY library NOT available. The manual will only compile partially.^^J}%
}%

\usepackage{xparse}% for colorbrewer manual

\usetikzlibrary{
    decorations.markings,
    decorations.footprints,
    shapes.arrows,
    matrix,
    positioning,
}

\usepgfplotslibrary{
    fillbetween,
    ternary,
    smithchart,
    patchplots,
    polar,
    colormaps,
    colorbrewer,
    colortol,           % docu not ready yet
}
\pgfqkeys{/codeexample}{%
    every codeexample/.append style={
        /pgfplots/every ternary axis/.append style={
            /pgfplots/legend style={fill=graphicbackground},
        }
    },
    tabsize=4,
}

\pgfplotsmanualenableexternalizationofexpensive

%\usetikzlibrary{external}
%\tikzexternalize[prefix=figures/]{pgfplots}

\title{%
    Manual for Package \PGFPlots{}\\
    {\small 2D/3D Plots in \LaTeX{}, Version \pgfplotsversion}\\
    {\small\url{https://github.com/pgf-tikz/pgfplots}}
    %\\{\small Attention: you are using an unstable development version.}
}

\makeatletter
\long\def\abstractsmuggle{%
    \centering
    \textbf{Abstract}\\[0.5cm]

    \begin{minipage}{12cm}
        \PGFPlots{} draws high-quality function plots in normal or logarithmic
        scaling with a user-friendly interface directly in \TeX{}. The user
        supplies axis labels, legend entries and the plot coordinates for one or
        more plots and \PGFPlots{} applies axis scaling, computes any logarithms
        and axis ticks and draws the plots. It supports line plots, scatter
        plots, piecewise constant plots, bar plots, area plots, mesh and surface
        plots, patch plots, contour plots, quiver plots, histogram plots, box
        plots, polar axes, ternary diagrams, smith charts and some more. It is
        based on Till Tantau's package \PGF{}/\Tikz{}.
    \end{minipage}

}%

\expandafter\date\expandafter{\@date\\[2cm]
    \abstractsmuggle
}%
\makeatother

%\includeonly{pgfplots.reference}


\begin{document}

\def\plotcoords{%
\addplot coordinates {
(5,8.312e-02)    (17,2.547e-02)   (49,7.407e-03)
(129,2.102e-03)  (321,5.874e-04)  (769,1.623e-04)
(1793,4.442e-05) (4097,1.207e-05) (9217,3.261e-06)
};

\addplot coordinates{
(7,8.472e-02)    (31,3.044e-02)    (111,1.022e-02)
(351,3.303e-03)  (1023,1.039e-03)  (2815,3.196e-04)
(7423,9.658e-05) (18943,2.873e-05) (47103,8.437e-06)
};

\addplot coordinates{
(9,7.881e-02)     (49,3.243e-02)    (209,1.232e-02)
(769,4.454e-03)   (2561,1.551e-03)  (7937,5.236e-04)
(23297,1.723e-04) (65537,5.545e-05) (178177,1.751e-05)
};

\addplot coordinates{
(11,6.887e-02)    (71,3.177e-02)     (351,1.341e-02)
(1471,5.334e-03)  (5503,2.027e-03)   (18943,7.415e-04)
(61183,2.628e-04) (187903,9.063e-05) (553983,3.053e-05)
};

\addplot coordinates{
(13,5.755e-02)     (97,2.925e-02)     (545,1.351e-02)
(2561,5.842e-03)   (10625,2.397e-03)  (40193,9.414e-04)
(141569,3.564e-04) (471041,1.308e-04)
(1496065,4.670e-05)
};
}%


\bookmaketitle
\tableofcontents
\include{pgfplots.title_abstract_intro}
\include{pgfplots.preliminaries}
\include{pgfplots.intro}
\include{pgfplots.reference}
\include{pgfplots.libs}
\include{pgfplots.resources}
\include{pgfplots.importexport}
\include{pgfplots.basic.reference}

\printindex

\bibliographystyle{abbrv} %gerapali} %gerabbrv} %gerunsrt.bst} %gerabbrv}% gerplain}
\nocite{pgfplotstable}
\nocite{programmingnotes}
\bibliography{pgfplots}
\end{document}

% -----------------------------------------------------------------------------
% For Stefan Pinnow as reminder on what to look for when editing the manual
% -----------------------------------------------------------------------------
% There should be no line breaks in the following environments
% - |...|
% - \declareandlabel{...}
% - \verbpdfref{...}
% If "MakeTikzPictures" isn't running through check one of the externalized
% LOG files what is the cause of that.
% -----------------------------------------------------------------------------


%%%%%%%%%%%%%%%%%%%%%%%%%%%%%%%%%%%%%%%%%%%%%%%%%%%%%%%%%%%%%%%%%%%%%%%%%%%%%
%
% Package pgfplots.sty documentation.
%
% Copyright 2007/2008 by Christian Feuersaenger.
%
% This program is free software: you can redistribute it and/or modify
% it under the terms of the GNU General Public License as published by
% the Free Software Foundation, either version 3 of the License, or
% (at your option) any later version.
%
% This program is distributed in the hope that it will be useful,
% but WITHOUT ANY WARRANTY; without even the implied warranty of
% MERCHANTABILITY or FITNESS FOR A PARTICULAR PURPOSE.  See the
% GNU General Public License for more details.
%
% You should have received a copy of the GNU General Public License
% along with this program.  If not, see <http://www.gnu.org/licenses/>.
%
%
%%%%%%%%%%%%%%%%%%%%%%%%%%%%%%%%%%%%%%%%%%%%%%%%%%%%%%%%%%%%%%%%%%%%%%%%%%%%%
% SEE pgfplots-macros.tex as well!
%\pdfminorversion=5 % to allow compression
%\pdfobjcompresslevel=2
\documentclass[a4paper,openany]{book}

\let\bookmaketitle=\maketitle

% -----------------------------------
% this here is from ltxdoc:
\usepackage{doc}[=v2]
\AtBeginDocument{\MakeShortVerb{\|}}
%\setlength{\textwidth}{355pt}
%\addtolength\marginparwidth{30pt}
%\addtolength\oddsidemargin{20pt}
%\addtolength\evensidemargin{20pt}
\def\cmd#1{\cs{\expandafter\cmd@to@cs\string#1}}
\def\cmd@to@cs#1#2{\char\number`#2\relax}
\DeclareRobustCommand\cs[1]{\texttt{\char`\\#1}}
\providecommand\marg[1]{%
  {\ttfamily\char`\{}\meta{#1}{\ttfamily\char`\}}}
\providecommand\oarg[1]{%
  {\ttfamily[}\meta{#1}{\ttfamily]}}
\providecommand\parg[1]{%
  {\ttfamily(}\meta{#1}{\ttfamily)}}
\raggedbottom

% -----------------------------------
\input{pgfplots.preamble.tex}

\makeatletter
% I want a two-column index just as in pgfmanual styles. This here
% was the best way to get one:
\def\index@prologue{\section*{Index}\addcontentsline{toc}{chapter}{Index}
}
\makeatother

%\RequirePackage[german,english,francais]{babel}

\def\matlabcolormaptext{This colormap is similar to one shipped with Matlab$^\text{\textregistered}$ under a similar name.}

\IfFileExists{tikzlibraryspy.code.tex}{%
\usetikzlibrary{spy}
}{%
    \message{ERROR: tikz SPY library NOT available. The manual will only compile partially.^^J}%
}%

\usepackage{xparse}% for colorbrewer manual

\usetikzlibrary{
    decorations.markings,
    decorations.footprints,
    shapes.arrows,
    matrix,
    positioning,
}

\usepgfplotslibrary{
    fillbetween,
    ternary,
    smithchart,
    patchplots,
    polar,
    colormaps,
    colorbrewer,
    colortol,           % docu not ready yet
}
\pgfqkeys{/codeexample}{%
    every codeexample/.append style={
        /pgfplots/every ternary axis/.append style={
            /pgfplots/legend style={fill=graphicbackground},
        }
    },
    tabsize=4,
}

\pgfplotsmanualenableexternalizationofexpensive

%\usetikzlibrary{external}
%\tikzexternalize[prefix=figures/]{pgfplots}

\title{%
    Manual for Package \PGFPlots{}\\
    {\small 2D/3D Plots in \LaTeX{}, Version \pgfplotsversion}\\
    {\small\url{https://github.com/pgf-tikz/pgfplots}}
    %\\{\small Attention: you are using an unstable development version.}
}

\makeatletter
\long\def\abstractsmuggle{%
    \centering
    \textbf{Abstract}\\[0.5cm]

    \begin{minipage}{12cm}
        \PGFPlots{} draws high-quality function plots in normal or logarithmic
        scaling with a user-friendly interface directly in \TeX{}. The user
        supplies axis labels, legend entries and the plot coordinates for one or
        more plots and \PGFPlots{} applies axis scaling, computes any logarithms
        and axis ticks and draws the plots. It supports line plots, scatter
        plots, piecewise constant plots, bar plots, area plots, mesh and surface
        plots, patch plots, contour plots, quiver plots, histogram plots, box
        plots, polar axes, ternary diagrams, smith charts and some more. It is
        based on Till Tantau's package \PGF{}/\Tikz{}.
    \end{minipage}

}%

\expandafter\date\expandafter{\@date\\[2cm]
    \abstractsmuggle
}%
\makeatother

%\includeonly{pgfplots.reference}


\begin{document}

\def\plotcoords{%
\addplot coordinates {
(5,8.312e-02)    (17,2.547e-02)   (49,7.407e-03)
(129,2.102e-03)  (321,5.874e-04)  (769,1.623e-04)
(1793,4.442e-05) (4097,1.207e-05) (9217,3.261e-06)
};

\addplot coordinates{
(7,8.472e-02)    (31,3.044e-02)    (111,1.022e-02)
(351,3.303e-03)  (1023,1.039e-03)  (2815,3.196e-04)
(7423,9.658e-05) (18943,2.873e-05) (47103,8.437e-06)
};

\addplot coordinates{
(9,7.881e-02)     (49,3.243e-02)    (209,1.232e-02)
(769,4.454e-03)   (2561,1.551e-03)  (7937,5.236e-04)
(23297,1.723e-04) (65537,5.545e-05) (178177,1.751e-05)
};

\addplot coordinates{
(11,6.887e-02)    (71,3.177e-02)     (351,1.341e-02)
(1471,5.334e-03)  (5503,2.027e-03)   (18943,7.415e-04)
(61183,2.628e-04) (187903,9.063e-05) (553983,3.053e-05)
};

\addplot coordinates{
(13,5.755e-02)     (97,2.925e-02)     (545,1.351e-02)
(2561,5.842e-03)   (10625,2.397e-03)  (40193,9.414e-04)
(141569,3.564e-04) (471041,1.308e-04)
(1496065,4.670e-05)
};
}%


\bookmaketitle
\tableofcontents
\include{pgfplots.title_abstract_intro}
\include{pgfplots.preliminaries}
\include{pgfplots.intro}
\include{pgfplots.reference}
\include{pgfplots.libs}
\include{pgfplots.resources}
\include{pgfplots.importexport}
\include{pgfplots.basic.reference}

\printindex

\bibliographystyle{abbrv} %gerapali} %gerabbrv} %gerunsrt.bst} %gerabbrv}% gerplain}
\nocite{pgfplotstable}
\nocite{programmingnotes}
\bibliography{pgfplots}
\end{document}

% -----------------------------------------------------------------------------
% For Stefan Pinnow as reminder on what to look for when editing the manual
% -----------------------------------------------------------------------------
% There should be no line breaks in the following environments
% - |...|
% - \declareandlabel{...}
% - \verbpdfref{...}
% If "MakeTikzPictures" isn't running through check one of the externalized
% LOG files what is the cause of that.
% -----------------------------------------------------------------------------


%%%%%%%%%%%%%%%%%%%%%%%%%%%%%%%%%%%%%%%%%%%%%%%%%%%%%%%%%%%%%%%%%%%%%%%%%%%%%
%
% Package pgfplots.sty documentation.
%
% Copyright 2007/2008 by Christian Feuersaenger.
%
% This program is free software: you can redistribute it and/or modify
% it under the terms of the GNU General Public License as published by
% the Free Software Foundation, either version 3 of the License, or
% (at your option) any later version.
%
% This program is distributed in the hope that it will be useful,
% but WITHOUT ANY WARRANTY; without even the implied warranty of
% MERCHANTABILITY or FITNESS FOR A PARTICULAR PURPOSE.  See the
% GNU General Public License for more details.
%
% You should have received a copy of the GNU General Public License
% along with this program.  If not, see <http://www.gnu.org/licenses/>.
%
%
%%%%%%%%%%%%%%%%%%%%%%%%%%%%%%%%%%%%%%%%%%%%%%%%%%%%%%%%%%%%%%%%%%%%%%%%%%%%%
% SEE pgfplots-macros.tex as well!
%\pdfminorversion=5 % to allow compression
%\pdfobjcompresslevel=2
\documentclass[a4paper,openany]{book}

\let\bookmaketitle=\maketitle

% -----------------------------------
% this here is from ltxdoc:
\usepackage{doc}[=v2]
\AtBeginDocument{\MakeShortVerb{\|}}
%\setlength{\textwidth}{355pt}
%\addtolength\marginparwidth{30pt}
%\addtolength\oddsidemargin{20pt}
%\addtolength\evensidemargin{20pt}
\def\cmd#1{\cs{\expandafter\cmd@to@cs\string#1}}
\def\cmd@to@cs#1#2{\char\number`#2\relax}
\DeclareRobustCommand\cs[1]{\texttt{\char`\\#1}}
\providecommand\marg[1]{%
  {\ttfamily\char`\{}\meta{#1}{\ttfamily\char`\}}}
\providecommand\oarg[1]{%
  {\ttfamily[}\meta{#1}{\ttfamily]}}
\providecommand\parg[1]{%
  {\ttfamily(}\meta{#1}{\ttfamily)}}
\raggedbottom

% -----------------------------------
\input{pgfplots.preamble.tex}

\makeatletter
% I want a two-column index just as in pgfmanual styles. This here
% was the best way to get one:
\def\index@prologue{\section*{Index}\addcontentsline{toc}{chapter}{Index}
}
\makeatother

%\RequirePackage[german,english,francais]{babel}

\def\matlabcolormaptext{This colormap is similar to one shipped with Matlab$^\text{\textregistered}$ under a similar name.}

\IfFileExists{tikzlibraryspy.code.tex}{%
\usetikzlibrary{spy}
}{%
    \message{ERROR: tikz SPY library NOT available. The manual will only compile partially.^^J}%
}%

\usepackage{xparse}% for colorbrewer manual

\usetikzlibrary{
    decorations.markings,
    decorations.footprints,
    shapes.arrows,
    matrix,
    positioning,
}

\usepgfplotslibrary{
    fillbetween,
    ternary,
    smithchart,
    patchplots,
    polar,
    colormaps,
    colorbrewer,
    colortol,           % docu not ready yet
}
\pgfqkeys{/codeexample}{%
    every codeexample/.append style={
        /pgfplots/every ternary axis/.append style={
            /pgfplots/legend style={fill=graphicbackground},
        }
    },
    tabsize=4,
}

\pgfplotsmanualenableexternalizationofexpensive

%\usetikzlibrary{external}
%\tikzexternalize[prefix=figures/]{pgfplots}

\title{%
    Manual for Package \PGFPlots{}\\
    {\small 2D/3D Plots in \LaTeX{}, Version \pgfplotsversion}\\
    {\small\url{https://github.com/pgf-tikz/pgfplots}}
    %\\{\small Attention: you are using an unstable development version.}
}

\makeatletter
\long\def\abstractsmuggle{%
    \centering
    \textbf{Abstract}\\[0.5cm]

    \begin{minipage}{12cm}
        \PGFPlots{} draws high-quality function plots in normal or logarithmic
        scaling with a user-friendly interface directly in \TeX{}. The user
        supplies axis labels, legend entries and the plot coordinates for one or
        more plots and \PGFPlots{} applies axis scaling, computes any logarithms
        and axis ticks and draws the plots. It supports line plots, scatter
        plots, piecewise constant plots, bar plots, area plots, mesh and surface
        plots, patch plots, contour plots, quiver plots, histogram plots, box
        plots, polar axes, ternary diagrams, smith charts and some more. It is
        based on Till Tantau's package \PGF{}/\Tikz{}.
    \end{minipage}

}%

\expandafter\date\expandafter{\@date\\[2cm]
    \abstractsmuggle
}%
\makeatother

%\includeonly{pgfplots.reference}


\begin{document}

\def\plotcoords{%
\addplot coordinates {
(5,8.312e-02)    (17,2.547e-02)   (49,7.407e-03)
(129,2.102e-03)  (321,5.874e-04)  (769,1.623e-04)
(1793,4.442e-05) (4097,1.207e-05) (9217,3.261e-06)
};

\addplot coordinates{
(7,8.472e-02)    (31,3.044e-02)    (111,1.022e-02)
(351,3.303e-03)  (1023,1.039e-03)  (2815,3.196e-04)
(7423,9.658e-05) (18943,2.873e-05) (47103,8.437e-06)
};

\addplot coordinates{
(9,7.881e-02)     (49,3.243e-02)    (209,1.232e-02)
(769,4.454e-03)   (2561,1.551e-03)  (7937,5.236e-04)
(23297,1.723e-04) (65537,5.545e-05) (178177,1.751e-05)
};

\addplot coordinates{
(11,6.887e-02)    (71,3.177e-02)     (351,1.341e-02)
(1471,5.334e-03)  (5503,2.027e-03)   (18943,7.415e-04)
(61183,2.628e-04) (187903,9.063e-05) (553983,3.053e-05)
};

\addplot coordinates{
(13,5.755e-02)     (97,2.925e-02)     (545,1.351e-02)
(2561,5.842e-03)   (10625,2.397e-03)  (40193,9.414e-04)
(141569,3.564e-04) (471041,1.308e-04)
(1496065,4.670e-05)
};
}%


\bookmaketitle
\tableofcontents
\include{pgfplots.title_abstract_intro}
\include{pgfplots.preliminaries}
\include{pgfplots.intro}
\include{pgfplots.reference}
\include{pgfplots.libs}
\include{pgfplots.resources}
\include{pgfplots.importexport}
\include{pgfplots.basic.reference}

\printindex

\bibliographystyle{abbrv} %gerapali} %gerabbrv} %gerunsrt.bst} %gerabbrv}% gerplain}
\nocite{pgfplotstable}
\nocite{programmingnotes}
\bibliography{pgfplots}
\end{document}

% -----------------------------------------------------------------------------
% For Stefan Pinnow as reminder on what to look for when editing the manual
% -----------------------------------------------------------------------------
% There should be no line breaks in the following environments
% - |...|
% - \declareandlabel{...}
% - \verbpdfref{...}
% If "MakeTikzPictures" isn't running through check one of the externalized
% LOG files what is the cause of that.
% -----------------------------------------------------------------------------


%%%%%%%%%%%%%%%%%%%%%%%%%%%%%%%%%%%%%%%%%%%%%%%%%%%%%%%%%%%%%%%%%%%%%%%%%%%%%
%
% Package pgfplots.sty documentation.
%
% Copyright 2007/2008 by Christian Feuersaenger.
%
% This program is free software: you can redistribute it and/or modify
% it under the terms of the GNU General Public License as published by
% the Free Software Foundation, either version 3 of the License, or
% (at your option) any later version.
%
% This program is distributed in the hope that it will be useful,
% but WITHOUT ANY WARRANTY; without even the implied warranty of
% MERCHANTABILITY or FITNESS FOR A PARTICULAR PURPOSE.  See the
% GNU General Public License for more details.
%
% You should have received a copy of the GNU General Public License
% along with this program.  If not, see <http://www.gnu.org/licenses/>.
%
%
%%%%%%%%%%%%%%%%%%%%%%%%%%%%%%%%%%%%%%%%%%%%%%%%%%%%%%%%%%%%%%%%%%%%%%%%%%%%%
% SEE pgfplots-macros.tex as well!
%\pdfminorversion=5 % to allow compression
%\pdfobjcompresslevel=2
\documentclass[a4paper,openany]{book}

\let\bookmaketitle=\maketitle

% -----------------------------------
% this here is from ltxdoc:
\usepackage{doc}[=v2]
\AtBeginDocument{\MakeShortVerb{\|}}
%\setlength{\textwidth}{355pt}
%\addtolength\marginparwidth{30pt}
%\addtolength\oddsidemargin{20pt}
%\addtolength\evensidemargin{20pt}
\def\cmd#1{\cs{\expandafter\cmd@to@cs\string#1}}
\def\cmd@to@cs#1#2{\char\number`#2\relax}
\DeclareRobustCommand\cs[1]{\texttt{\char`\\#1}}
\providecommand\marg[1]{%
  {\ttfamily\char`\{}\meta{#1}{\ttfamily\char`\}}}
\providecommand\oarg[1]{%
  {\ttfamily[}\meta{#1}{\ttfamily]}}
\providecommand\parg[1]{%
  {\ttfamily(}\meta{#1}{\ttfamily)}}
\raggedbottom

% -----------------------------------
\input{pgfplots.preamble.tex}

\makeatletter
% I want a two-column index just as in pgfmanual styles. This here
% was the best way to get one:
\def\index@prologue{\section*{Index}\addcontentsline{toc}{chapter}{Index}
}
\makeatother

%\RequirePackage[german,english,francais]{babel}

\def\matlabcolormaptext{This colormap is similar to one shipped with Matlab$^\text{\textregistered}$ under a similar name.}

\IfFileExists{tikzlibraryspy.code.tex}{%
\usetikzlibrary{spy}
}{%
    \message{ERROR: tikz SPY library NOT available. The manual will only compile partially.^^J}%
}%

\usepackage{xparse}% for colorbrewer manual

\usetikzlibrary{
    decorations.markings,
    decorations.footprints,
    shapes.arrows,
    matrix,
    positioning,
}

\usepgfplotslibrary{
    fillbetween,
    ternary,
    smithchart,
    patchplots,
    polar,
    colormaps,
    colorbrewer,
    colortol,           % docu not ready yet
}
\pgfqkeys{/codeexample}{%
    every codeexample/.append style={
        /pgfplots/every ternary axis/.append style={
            /pgfplots/legend style={fill=graphicbackground},
        }
    },
    tabsize=4,
}

\pgfplotsmanualenableexternalizationofexpensive

%\usetikzlibrary{external}
%\tikzexternalize[prefix=figures/]{pgfplots}

\title{%
    Manual for Package \PGFPlots{}\\
    {\small 2D/3D Plots in \LaTeX{}, Version \pgfplotsversion}\\
    {\small\url{https://github.com/pgf-tikz/pgfplots}}
    %\\{\small Attention: you are using an unstable development version.}
}

\makeatletter
\long\def\abstractsmuggle{%
    \centering
    \textbf{Abstract}\\[0.5cm]

    \begin{minipage}{12cm}
        \PGFPlots{} draws high-quality function plots in normal or logarithmic
        scaling with a user-friendly interface directly in \TeX{}. The user
        supplies axis labels, legend entries and the plot coordinates for one or
        more plots and \PGFPlots{} applies axis scaling, computes any logarithms
        and axis ticks and draws the plots. It supports line plots, scatter
        plots, piecewise constant plots, bar plots, area plots, mesh and surface
        plots, patch plots, contour plots, quiver plots, histogram plots, box
        plots, polar axes, ternary diagrams, smith charts and some more. It is
        based on Till Tantau's package \PGF{}/\Tikz{}.
    \end{minipage}

}%

\expandafter\date\expandafter{\@date\\[2cm]
    \abstractsmuggle
}%
\makeatother

%\includeonly{pgfplots.reference}


\begin{document}

\def\plotcoords{%
\addplot coordinates {
(5,8.312e-02)    (17,2.547e-02)   (49,7.407e-03)
(129,2.102e-03)  (321,5.874e-04)  (769,1.623e-04)
(1793,4.442e-05) (4097,1.207e-05) (9217,3.261e-06)
};

\addplot coordinates{
(7,8.472e-02)    (31,3.044e-02)    (111,1.022e-02)
(351,3.303e-03)  (1023,1.039e-03)  (2815,3.196e-04)
(7423,9.658e-05) (18943,2.873e-05) (47103,8.437e-06)
};

\addplot coordinates{
(9,7.881e-02)     (49,3.243e-02)    (209,1.232e-02)
(769,4.454e-03)   (2561,1.551e-03)  (7937,5.236e-04)
(23297,1.723e-04) (65537,5.545e-05) (178177,1.751e-05)
};

\addplot coordinates{
(11,6.887e-02)    (71,3.177e-02)     (351,1.341e-02)
(1471,5.334e-03)  (5503,2.027e-03)   (18943,7.415e-04)
(61183,2.628e-04) (187903,9.063e-05) (553983,3.053e-05)
};

\addplot coordinates{
(13,5.755e-02)     (97,2.925e-02)     (545,1.351e-02)
(2561,5.842e-03)   (10625,2.397e-03)  (40193,9.414e-04)
(141569,3.564e-04) (471041,1.308e-04)
(1496065,4.670e-05)
};
}%


\bookmaketitle
\tableofcontents
\include{pgfplots.title_abstract_intro}
\include{pgfplots.preliminaries}
\include{pgfplots.intro}
\include{pgfplots.reference}
\include{pgfplots.libs}
\include{pgfplots.resources}
\include{pgfplots.importexport}
\include{pgfplots.basic.reference}

\printindex

\bibliographystyle{abbrv} %gerapali} %gerabbrv} %gerunsrt.bst} %gerabbrv}% gerplain}
\nocite{pgfplotstable}
\nocite{programmingnotes}
\bibliography{pgfplots}
\end{document}

% -----------------------------------------------------------------------------
% For Stefan Pinnow as reminder on what to look for when editing the manual
% -----------------------------------------------------------------------------
% There should be no line breaks in the following environments
% - |...|
% - \declareandlabel{...}
% - \verbpdfref{...}
% If "MakeTikzPictures" isn't running through check one of the externalized
% LOG files what is the cause of that.
% -----------------------------------------------------------------------------


%%%%%%%%%%%%%%%%%%%%%%%%%%%%%%%%%%%%%%%%%%%%%%%%%%%%%%%%%%%%%%%%%%%%%%%%%%%%%
%
% Package pgfplots.sty documentation.
%
% Copyright 2007/2008 by Christian Feuersaenger.
%
% This program is free software: you can redistribute it and/or modify
% it under the terms of the GNU General Public License as published by
% the Free Software Foundation, either version 3 of the License, or
% (at your option) any later version.
%
% This program is distributed in the hope that it will be useful,
% but WITHOUT ANY WARRANTY; without even the implied warranty of
% MERCHANTABILITY or FITNESS FOR A PARTICULAR PURPOSE.  See the
% GNU General Public License for more details.
%
% You should have received a copy of the GNU General Public License
% along with this program.  If not, see <http://www.gnu.org/licenses/>.
%
%
%%%%%%%%%%%%%%%%%%%%%%%%%%%%%%%%%%%%%%%%%%%%%%%%%%%%%%%%%%%%%%%%%%%%%%%%%%%%%
% SEE pgfplots-macros.tex as well!
%\pdfminorversion=5 % to allow compression
%\pdfobjcompresslevel=2
\documentclass[a4paper,openany]{book}

\let\bookmaketitle=\maketitle

% -----------------------------------
% this here is from ltxdoc:
\usepackage{doc}[=v2]
\AtBeginDocument{\MakeShortVerb{\|}}
%\setlength{\textwidth}{355pt}
%\addtolength\marginparwidth{30pt}
%\addtolength\oddsidemargin{20pt}
%\addtolength\evensidemargin{20pt}
\def\cmd#1{\cs{\expandafter\cmd@to@cs\string#1}}
\def\cmd@to@cs#1#2{\char\number`#2\relax}
\DeclareRobustCommand\cs[1]{\texttt{\char`\\#1}}
\providecommand\marg[1]{%
  {\ttfamily\char`\{}\meta{#1}{\ttfamily\char`\}}}
\providecommand\oarg[1]{%
  {\ttfamily[}\meta{#1}{\ttfamily]}}
\providecommand\parg[1]{%
  {\ttfamily(}\meta{#1}{\ttfamily)}}
\raggedbottom

% -----------------------------------
\input{pgfplots.preamble.tex}

\makeatletter
% I want a two-column index just as in pgfmanual styles. This here
% was the best way to get one:
\def\index@prologue{\section*{Index}\addcontentsline{toc}{chapter}{Index}
}
\makeatother

%\RequirePackage[german,english,francais]{babel}

\def\matlabcolormaptext{This colormap is similar to one shipped with Matlab$^\text{\textregistered}$ under a similar name.}

\IfFileExists{tikzlibraryspy.code.tex}{%
\usetikzlibrary{spy}
}{%
    \message{ERROR: tikz SPY library NOT available. The manual will only compile partially.^^J}%
}%

\usepackage{xparse}% for colorbrewer manual

\usetikzlibrary{
    decorations.markings,
    decorations.footprints,
    shapes.arrows,
    matrix,
    positioning,
}

\usepgfplotslibrary{
    fillbetween,
    ternary,
    smithchart,
    patchplots,
    polar,
    colormaps,
    colorbrewer,
    colortol,           % docu not ready yet
}
\pgfqkeys{/codeexample}{%
    every codeexample/.append style={
        /pgfplots/every ternary axis/.append style={
            /pgfplots/legend style={fill=graphicbackground},
        }
    },
    tabsize=4,
}

\pgfplotsmanualenableexternalizationofexpensive

%\usetikzlibrary{external}
%\tikzexternalize[prefix=figures/]{pgfplots}

\title{%
    Manual for Package \PGFPlots{}\\
    {\small 2D/3D Plots in \LaTeX{}, Version \pgfplotsversion}\\
    {\small\url{https://github.com/pgf-tikz/pgfplots}}
    %\\{\small Attention: you are using an unstable development version.}
}

\makeatletter
\long\def\abstractsmuggle{%
    \centering
    \textbf{Abstract}\\[0.5cm]

    \begin{minipage}{12cm}
        \PGFPlots{} draws high-quality function plots in normal or logarithmic
        scaling with a user-friendly interface directly in \TeX{}. The user
        supplies axis labels, legend entries and the plot coordinates for one or
        more plots and \PGFPlots{} applies axis scaling, computes any logarithms
        and axis ticks and draws the plots. It supports line plots, scatter
        plots, piecewise constant plots, bar plots, area plots, mesh and surface
        plots, patch plots, contour plots, quiver plots, histogram plots, box
        plots, polar axes, ternary diagrams, smith charts and some more. It is
        based on Till Tantau's package \PGF{}/\Tikz{}.
    \end{minipage}

}%

\expandafter\date\expandafter{\@date\\[2cm]
    \abstractsmuggle
}%
\makeatother

%\includeonly{pgfplots.reference}


\begin{document}

\def\plotcoords{%
\addplot coordinates {
(5,8.312e-02)    (17,2.547e-02)   (49,7.407e-03)
(129,2.102e-03)  (321,5.874e-04)  (769,1.623e-04)
(1793,4.442e-05) (4097,1.207e-05) (9217,3.261e-06)
};

\addplot coordinates{
(7,8.472e-02)    (31,3.044e-02)    (111,1.022e-02)
(351,3.303e-03)  (1023,1.039e-03)  (2815,3.196e-04)
(7423,9.658e-05) (18943,2.873e-05) (47103,8.437e-06)
};

\addplot coordinates{
(9,7.881e-02)     (49,3.243e-02)    (209,1.232e-02)
(769,4.454e-03)   (2561,1.551e-03)  (7937,5.236e-04)
(23297,1.723e-04) (65537,5.545e-05) (178177,1.751e-05)
};

\addplot coordinates{
(11,6.887e-02)    (71,3.177e-02)     (351,1.341e-02)
(1471,5.334e-03)  (5503,2.027e-03)   (18943,7.415e-04)
(61183,2.628e-04) (187903,9.063e-05) (553983,3.053e-05)
};

\addplot coordinates{
(13,5.755e-02)     (97,2.925e-02)     (545,1.351e-02)
(2561,5.842e-03)   (10625,2.397e-03)  (40193,9.414e-04)
(141569,3.564e-04) (471041,1.308e-04)
(1496065,4.670e-05)
};
}%


\bookmaketitle
\tableofcontents
\include{pgfplots.title_abstract_intro}
\include{pgfplots.preliminaries}
\include{pgfplots.intro}
\include{pgfplots.reference}
\include{pgfplots.libs}
\include{pgfplots.resources}
\include{pgfplots.importexport}
\include{pgfplots.basic.reference}

\printindex

\bibliographystyle{abbrv} %gerapali} %gerabbrv} %gerunsrt.bst} %gerabbrv}% gerplain}
\nocite{pgfplotstable}
\nocite{programmingnotes}
\bibliography{pgfplots}
\end{document}

% -----------------------------------------------------------------------------
% For Stefan Pinnow as reminder on what to look for when editing the manual
% -----------------------------------------------------------------------------
% There should be no line breaks in the following environments
% - |...|
% - \declareandlabel{...}
% - \verbpdfref{...}
% If "MakeTikzPictures" isn't running through check one of the externalized
% LOG files what is the cause of that.
% -----------------------------------------------------------------------------


%%%%%%%%%%%%%%%%%%%%%%%%%%%%%%%%%%%%%%%%%%%%%%%%%%%%%%%%%%%%%%%%%%%%%%%%%%%%%
%
% Package pgfplots.sty documentation.
%
% Copyright 2007/2008 by Christian Feuersaenger.
%
% This program is free software: you can redistribute it and/or modify
% it under the terms of the GNU General Public License as published by
% the Free Software Foundation, either version 3 of the License, or
% (at your option) any later version.
%
% This program is distributed in the hope that it will be useful,
% but WITHOUT ANY WARRANTY; without even the implied warranty of
% MERCHANTABILITY or FITNESS FOR A PARTICULAR PURPOSE.  See the
% GNU General Public License for more details.
%
% You should have received a copy of the GNU General Public License
% along with this program.  If not, see <http://www.gnu.org/licenses/>.
%
%
%%%%%%%%%%%%%%%%%%%%%%%%%%%%%%%%%%%%%%%%%%%%%%%%%%%%%%%%%%%%%%%%%%%%%%%%%%%%%
% SEE pgfplots-macros.tex as well!
%\pdfminorversion=5 % to allow compression
%\pdfobjcompresslevel=2
\documentclass[a4paper,openany]{book}

\let\bookmaketitle=\maketitle

% -----------------------------------
% this here is from ltxdoc:
\usepackage{doc}[=v2]
\AtBeginDocument{\MakeShortVerb{\|}}
%\setlength{\textwidth}{355pt}
%\addtolength\marginparwidth{30pt}
%\addtolength\oddsidemargin{20pt}
%\addtolength\evensidemargin{20pt}
\def\cmd#1{\cs{\expandafter\cmd@to@cs\string#1}}
\def\cmd@to@cs#1#2{\char\number`#2\relax}
\DeclareRobustCommand\cs[1]{\texttt{\char`\\#1}}
\providecommand\marg[1]{%
  {\ttfamily\char`\{}\meta{#1}{\ttfamily\char`\}}}
\providecommand\oarg[1]{%
  {\ttfamily[}\meta{#1}{\ttfamily]}}
\providecommand\parg[1]{%
  {\ttfamily(}\meta{#1}{\ttfamily)}}
\raggedbottom

% -----------------------------------
\input{pgfplots.preamble.tex}

\makeatletter
% I want a two-column index just as in pgfmanual styles. This here
% was the best way to get one:
\def\index@prologue{\section*{Index}\addcontentsline{toc}{chapter}{Index}
}
\makeatother

%\RequirePackage[german,english,francais]{babel}

\def\matlabcolormaptext{This colormap is similar to one shipped with Matlab$^\text{\textregistered}$ under a similar name.}

\IfFileExists{tikzlibraryspy.code.tex}{%
\usetikzlibrary{spy}
}{%
    \message{ERROR: tikz SPY library NOT available. The manual will only compile partially.^^J}%
}%

\usepackage{xparse}% for colorbrewer manual

\usetikzlibrary{
    decorations.markings,
    decorations.footprints,
    shapes.arrows,
    matrix,
    positioning,
}

\usepgfplotslibrary{
    fillbetween,
    ternary,
    smithchart,
    patchplots,
    polar,
    colormaps,
    colorbrewer,
    colortol,           % docu not ready yet
}
\pgfqkeys{/codeexample}{%
    every codeexample/.append style={
        /pgfplots/every ternary axis/.append style={
            /pgfplots/legend style={fill=graphicbackground},
        }
    },
    tabsize=4,
}

\pgfplotsmanualenableexternalizationofexpensive

%\usetikzlibrary{external}
%\tikzexternalize[prefix=figures/]{pgfplots}

\title{%
    Manual for Package \PGFPlots{}\\
    {\small 2D/3D Plots in \LaTeX{}, Version \pgfplotsversion}\\
    {\small\url{https://github.com/pgf-tikz/pgfplots}}
    %\\{\small Attention: you are using an unstable development version.}
}

\makeatletter
\long\def\abstractsmuggle{%
    \centering
    \textbf{Abstract}\\[0.5cm]

    \begin{minipage}{12cm}
        \PGFPlots{} draws high-quality function plots in normal or logarithmic
        scaling with a user-friendly interface directly in \TeX{}. The user
        supplies axis labels, legend entries and the plot coordinates for one or
        more plots and \PGFPlots{} applies axis scaling, computes any logarithms
        and axis ticks and draws the plots. It supports line plots, scatter
        plots, piecewise constant plots, bar plots, area plots, mesh and surface
        plots, patch plots, contour plots, quiver plots, histogram plots, box
        plots, polar axes, ternary diagrams, smith charts and some more. It is
        based on Till Tantau's package \PGF{}/\Tikz{}.
    \end{minipage}

}%

\expandafter\date\expandafter{\@date\\[2cm]
    \abstractsmuggle
}%
\makeatother

%\includeonly{pgfplots.reference}


\begin{document}

\def\plotcoords{%
\addplot coordinates {
(5,8.312e-02)    (17,2.547e-02)   (49,7.407e-03)
(129,2.102e-03)  (321,5.874e-04)  (769,1.623e-04)
(1793,4.442e-05) (4097,1.207e-05) (9217,3.261e-06)
};

\addplot coordinates{
(7,8.472e-02)    (31,3.044e-02)    (111,1.022e-02)
(351,3.303e-03)  (1023,1.039e-03)  (2815,3.196e-04)
(7423,9.658e-05) (18943,2.873e-05) (47103,8.437e-06)
};

\addplot coordinates{
(9,7.881e-02)     (49,3.243e-02)    (209,1.232e-02)
(769,4.454e-03)   (2561,1.551e-03)  (7937,5.236e-04)
(23297,1.723e-04) (65537,5.545e-05) (178177,1.751e-05)
};

\addplot coordinates{
(11,6.887e-02)    (71,3.177e-02)     (351,1.341e-02)
(1471,5.334e-03)  (5503,2.027e-03)   (18943,7.415e-04)
(61183,2.628e-04) (187903,9.063e-05) (553983,3.053e-05)
};

\addplot coordinates{
(13,5.755e-02)     (97,2.925e-02)     (545,1.351e-02)
(2561,5.842e-03)   (10625,2.397e-03)  (40193,9.414e-04)
(141569,3.564e-04) (471041,1.308e-04)
(1496065,4.670e-05)
};
}%


\bookmaketitle
\tableofcontents
\include{pgfplots.title_abstract_intro}
\include{pgfplots.preliminaries}
\include{pgfplots.intro}
\include{pgfplots.reference}
\include{pgfplots.libs}
\include{pgfplots.resources}
\include{pgfplots.importexport}
\include{pgfplots.basic.reference}

\printindex

\bibliographystyle{abbrv} %gerapali} %gerabbrv} %gerunsrt.bst} %gerabbrv}% gerplain}
\nocite{pgfplotstable}
\nocite{programmingnotes}
\bibliography{pgfplots}
\end{document}

% -----------------------------------------------------------------------------
% For Stefan Pinnow as reminder on what to look for when editing the manual
% -----------------------------------------------------------------------------
% There should be no line breaks in the following environments
% - |...|
% - \declareandlabel{...}
% - \verbpdfref{...}
% If "MakeTikzPictures" isn't running through check one of the externalized
% LOG files what is the cause of that.
% -----------------------------------------------------------------------------


%%%%%%%%%%%%%%%%%%%%%%%%%%%%%%%%%%%%%%%%%%%%%%%%%%%%%%%%%%%%%%%%%%%%%%%%%%%%%
%
% Package pgfplots.sty documentation.
%
% Copyright 2007/2008 by Christian Feuersaenger.
%
% This program is free software: you can redistribute it and/or modify
% it under the terms of the GNU General Public License as published by
% the Free Software Foundation, either version 3 of the License, or
% (at your option) any later version.
%
% This program is distributed in the hope that it will be useful,
% but WITHOUT ANY WARRANTY; without even the implied warranty of
% MERCHANTABILITY or FITNESS FOR A PARTICULAR PURPOSE.  See the
% GNU General Public License for more details.
%
% You should have received a copy of the GNU General Public License
% along with this program.  If not, see <http://www.gnu.org/licenses/>.
%
%
%%%%%%%%%%%%%%%%%%%%%%%%%%%%%%%%%%%%%%%%%%%%%%%%%%%%%%%%%%%%%%%%%%%%%%%%%%%%%
% SEE pgfplots-macros.tex as well!
%\pdfminorversion=5 % to allow compression
%\pdfobjcompresslevel=2
\documentclass[a4paper,openany]{book}

\let\bookmaketitle=\maketitle

% -----------------------------------
% this here is from ltxdoc:
\usepackage{doc}[=v2]
\AtBeginDocument{\MakeShortVerb{\|}}
%\setlength{\textwidth}{355pt}
%\addtolength\marginparwidth{30pt}
%\addtolength\oddsidemargin{20pt}
%\addtolength\evensidemargin{20pt}
\def\cmd#1{\cs{\expandafter\cmd@to@cs\string#1}}
\def\cmd@to@cs#1#2{\char\number`#2\relax}
\DeclareRobustCommand\cs[1]{\texttt{\char`\\#1}}
\providecommand\marg[1]{%
  {\ttfamily\char`\{}\meta{#1}{\ttfamily\char`\}}}
\providecommand\oarg[1]{%
  {\ttfamily[}\meta{#1}{\ttfamily]}}
\providecommand\parg[1]{%
  {\ttfamily(}\meta{#1}{\ttfamily)}}
\raggedbottom

% -----------------------------------
\input{pgfplots.preamble.tex}

\makeatletter
% I want a two-column index just as in pgfmanual styles. This here
% was the best way to get one:
\def\index@prologue{\section*{Index}\addcontentsline{toc}{chapter}{Index}
}
\makeatother

%\RequirePackage[german,english,francais]{babel}

\def\matlabcolormaptext{This colormap is similar to one shipped with Matlab$^\text{\textregistered}$ under a similar name.}

\IfFileExists{tikzlibraryspy.code.tex}{%
\usetikzlibrary{spy}
}{%
    \message{ERROR: tikz SPY library NOT available. The manual will only compile partially.^^J}%
}%

\usepackage{xparse}% for colorbrewer manual

\usetikzlibrary{
    decorations.markings,
    decorations.footprints,
    shapes.arrows,
    matrix,
    positioning,
}

\usepgfplotslibrary{
    fillbetween,
    ternary,
    smithchart,
    patchplots,
    polar,
    colormaps,
    colorbrewer,
    colortol,           % docu not ready yet
}
\pgfqkeys{/codeexample}{%
    every codeexample/.append style={
        /pgfplots/every ternary axis/.append style={
            /pgfplots/legend style={fill=graphicbackground},
        }
    },
    tabsize=4,
}

\pgfplotsmanualenableexternalizationofexpensive

%\usetikzlibrary{external}
%\tikzexternalize[prefix=figures/]{pgfplots}

\title{%
    Manual for Package \PGFPlots{}\\
    {\small 2D/3D Plots in \LaTeX{}, Version \pgfplotsversion}\\
    {\small\url{https://github.com/pgf-tikz/pgfplots}}
    %\\{\small Attention: you are using an unstable development version.}
}

\makeatletter
\long\def\abstractsmuggle{%
    \centering
    \textbf{Abstract}\\[0.5cm]

    \begin{minipage}{12cm}
        \PGFPlots{} draws high-quality function plots in normal or logarithmic
        scaling with a user-friendly interface directly in \TeX{}. The user
        supplies axis labels, legend entries and the plot coordinates for one or
        more plots and \PGFPlots{} applies axis scaling, computes any logarithms
        and axis ticks and draws the plots. It supports line plots, scatter
        plots, piecewise constant plots, bar plots, area plots, mesh and surface
        plots, patch plots, contour plots, quiver plots, histogram plots, box
        plots, polar axes, ternary diagrams, smith charts and some more. It is
        based on Till Tantau's package \PGF{}/\Tikz{}.
    \end{minipage}

}%

\expandafter\date\expandafter{\@date\\[2cm]
    \abstractsmuggle
}%
\makeatother

%\includeonly{pgfplots.reference}


\begin{document}

\def\plotcoords{%
\addplot coordinates {
(5,8.312e-02)    (17,2.547e-02)   (49,7.407e-03)
(129,2.102e-03)  (321,5.874e-04)  (769,1.623e-04)
(1793,4.442e-05) (4097,1.207e-05) (9217,3.261e-06)
};

\addplot coordinates{
(7,8.472e-02)    (31,3.044e-02)    (111,1.022e-02)
(351,3.303e-03)  (1023,1.039e-03)  (2815,3.196e-04)
(7423,9.658e-05) (18943,2.873e-05) (47103,8.437e-06)
};

\addplot coordinates{
(9,7.881e-02)     (49,3.243e-02)    (209,1.232e-02)
(769,4.454e-03)   (2561,1.551e-03)  (7937,5.236e-04)
(23297,1.723e-04) (65537,5.545e-05) (178177,1.751e-05)
};

\addplot coordinates{
(11,6.887e-02)    (71,3.177e-02)     (351,1.341e-02)
(1471,5.334e-03)  (5503,2.027e-03)   (18943,7.415e-04)
(61183,2.628e-04) (187903,9.063e-05) (553983,3.053e-05)
};

\addplot coordinates{
(13,5.755e-02)     (97,2.925e-02)     (545,1.351e-02)
(2561,5.842e-03)   (10625,2.397e-03)  (40193,9.414e-04)
(141569,3.564e-04) (471041,1.308e-04)
(1496065,4.670e-05)
};
}%


\bookmaketitle
\tableofcontents
\include{pgfplots.title_abstract_intro}
\include{pgfplots.preliminaries}
\include{pgfplots.intro}
\include{pgfplots.reference}
\include{pgfplots.libs}
\include{pgfplots.resources}
\include{pgfplots.importexport}
\include{pgfplots.basic.reference}

\printindex

\bibliographystyle{abbrv} %gerapali} %gerabbrv} %gerunsrt.bst} %gerabbrv}% gerplain}
\nocite{pgfplotstable}
\nocite{programmingnotes}
\bibliography{pgfplots}
\end{document}

% -----------------------------------------------------------------------------
% For Stefan Pinnow as reminder on what to look for when editing the manual
% -----------------------------------------------------------------------------
% There should be no line breaks in the following environments
% - |...|
% - \declareandlabel{...}
% - \verbpdfref{...}
% If "MakeTikzPictures" isn't running through check one of the externalized
% LOG files what is the cause of that.
% -----------------------------------------------------------------------------

\include{pgfplots.basic.reference}

\printindex

\bibliographystyle{abbrv} %gerapali} %gerabbrv} %gerunsrt.bst} %gerabbrv}% gerplain}
\nocite{pgfplotstable}
\nocite{programmingnotes}
\bibliography{pgfplots}
\end{document}

% -----------------------------------------------------------------------------
% For Stefan Pinnow as reminder on what to look for when editing the manual
% -----------------------------------------------------------------------------
% There should be no line breaks in the following environments
% - |...|
% - \declareandlabel{...}
% - \verbpdfref{...}
% If "MakeTikzPictures" isn't running through check one of the externalized
% LOG files what is the cause of that.
% -----------------------------------------------------------------------------

\include{pgfplots.basic.reference}

\printindex

\bibliographystyle{abbrv} %gerapali} %gerabbrv} %gerunsrt.bst} %gerabbrv}% gerplain}
\nocite{pgfplotstable}
\nocite{programmingnotes}
\bibliography{pgfplots}
\end{document}

% -----------------------------------------------------------------------------
% For Stefan Pinnow as reminder on what to look for when editing the manual
% -----------------------------------------------------------------------------
% There should be no line breaks in the following environments
% - |...|
% - \declareandlabel{...}
% - \verbpdfref{...}
% If "MakeTikzPictures" isn't running through check one of the externalized
% LOG files what is the cause of that.
% -----------------------------------------------------------------------------


%%%%%%%%%%%%%%%%%%%%%%%%%%%%%%%%%%%%%%%%%%%%%%%%%%%%%%%%%%%%%%%%%%%%%%%%%%%%%
%
% Package pgfplots.sty documentation.
%
% Copyright 2007/2008 by Christian Feuersaenger.
%
% This program is free software: you can redistribute it and/or modify
% it under the terms of the GNU General Public License as published by
% the Free Software Foundation, either version 3 of the License, or
% (at your option) any later version.
%
% This program is distributed in the hope that it will be useful,
% but WITHOUT ANY WARRANTY; without even the implied warranty of
% MERCHANTABILITY or FITNESS FOR A PARTICULAR PURPOSE.  See the
% GNU General Public License for more details.
%
% You should have received a copy of the GNU General Public License
% along with this program.  If not, see <http://www.gnu.org/licenses/>.
%
%
%%%%%%%%%%%%%%%%%%%%%%%%%%%%%%%%%%%%%%%%%%%%%%%%%%%%%%%%%%%%%%%%%%%%%%%%%%%%%
% SEE pgfplots-macros.tex as well!
%\pdfminorversion=5 % to allow compression
%\pdfobjcompresslevel=2
\documentclass[a4paper,openany]{book}

\let\bookmaketitle=\maketitle

% -----------------------------------
% this here is from ltxdoc:
\usepackage{doc}[=v2]
\AtBeginDocument{\MakeShortVerb{\|}}
%\setlength{\textwidth}{355pt}
%\addtolength\marginparwidth{30pt}
%\addtolength\oddsidemargin{20pt}
%\addtolength\evensidemargin{20pt}
\def\cmd#1{\cs{\expandafter\cmd@to@cs\string#1}}
\def\cmd@to@cs#1#2{\char\number`#2\relax}
\DeclareRobustCommand\cs[1]{\texttt{\char`\\#1}}
\providecommand\marg[1]{%
  {\ttfamily\char`\{}\meta{#1}{\ttfamily\char`\}}}
\providecommand\oarg[1]{%
  {\ttfamily[}\meta{#1}{\ttfamily]}}
\providecommand\parg[1]{%
  {\ttfamily(}\meta{#1}{\ttfamily)}}
\raggedbottom

% -----------------------------------
\input{pgfplots.preamble.tex}

\makeatletter
% I want a two-column index just as in pgfmanual styles. This here
% was the best way to get one:
\def\index@prologue{\section*{Index}\addcontentsline{toc}{chapter}{Index}
}
\makeatother

%\RequirePackage[german,english,francais]{babel}

\def\matlabcolormaptext{This colormap is similar to one shipped with Matlab$^\text{\textregistered}$ under a similar name.}

\IfFileExists{tikzlibraryspy.code.tex}{%
\usetikzlibrary{spy}
}{%
    \message{ERROR: tikz SPY library NOT available. The manual will only compile partially.^^J}%
}%

\usepackage{xparse}% for colorbrewer manual

\usetikzlibrary{
    decorations.markings,
    decorations.footprints,
    shapes.arrows,
    matrix,
    positioning,
}

\usepgfplotslibrary{
    fillbetween,
    ternary,
    smithchart,
    patchplots,
    polar,
    colormaps,
    colorbrewer,
    colortol,           % docu not ready yet
}
\pgfqkeys{/codeexample}{%
    every codeexample/.append style={
        /pgfplots/every ternary axis/.append style={
            /pgfplots/legend style={fill=graphicbackground},
        }
    },
    tabsize=4,
}

\pgfplotsmanualenableexternalizationofexpensive

%\usetikzlibrary{external}
%\tikzexternalize[prefix=figures/]{pgfplots}

\title{%
    Manual for Package \PGFPlots{}\\
    {\small 2D/3D Plots in \LaTeX{}, Version \pgfplotsversion}\\
    {\small\url{https://github.com/pgf-tikz/pgfplots}}
    %\\{\small Attention: you are using an unstable development version.}
}

\makeatletter
\long\def\abstractsmuggle{%
    \centering
    \textbf{Abstract}\\[0.5cm]

    \begin{minipage}{12cm}
        \PGFPlots{} draws high-quality function plots in normal or logarithmic
        scaling with a user-friendly interface directly in \TeX{}. The user
        supplies axis labels, legend entries and the plot coordinates for one or
        more plots and \PGFPlots{} applies axis scaling, computes any logarithms
        and axis ticks and draws the plots. It supports line plots, scatter
        plots, piecewise constant plots, bar plots, area plots, mesh and surface
        plots, patch plots, contour plots, quiver plots, histogram plots, box
        plots, polar axes, ternary diagrams, smith charts and some more. It is
        based on Till Tantau's package \PGF{}/\Tikz{}.
    \end{minipage}

}%

\expandafter\date\expandafter{\@date\\[2cm]
    \abstractsmuggle
}%
\makeatother

%\includeonly{pgfplots.reference}


\begin{document}

\def\plotcoords{%
\addplot coordinates {
(5,8.312e-02)    (17,2.547e-02)   (49,7.407e-03)
(129,2.102e-03)  (321,5.874e-04)  (769,1.623e-04)
(1793,4.442e-05) (4097,1.207e-05) (9217,3.261e-06)
};

\addplot coordinates{
(7,8.472e-02)    (31,3.044e-02)    (111,1.022e-02)
(351,3.303e-03)  (1023,1.039e-03)  (2815,3.196e-04)
(7423,9.658e-05) (18943,2.873e-05) (47103,8.437e-06)
};

\addplot coordinates{
(9,7.881e-02)     (49,3.243e-02)    (209,1.232e-02)
(769,4.454e-03)   (2561,1.551e-03)  (7937,5.236e-04)
(23297,1.723e-04) (65537,5.545e-05) (178177,1.751e-05)
};

\addplot coordinates{
(11,6.887e-02)    (71,3.177e-02)     (351,1.341e-02)
(1471,5.334e-03)  (5503,2.027e-03)   (18943,7.415e-04)
(61183,2.628e-04) (187903,9.063e-05) (553983,3.053e-05)
};

\addplot coordinates{
(13,5.755e-02)     (97,2.925e-02)     (545,1.351e-02)
(2561,5.842e-03)   (10625,2.397e-03)  (40193,9.414e-04)
(141569,3.564e-04) (471041,1.308e-04)
(1496065,4.670e-05)
};
}%


\bookmaketitle
\tableofcontents

%%%%%%%%%%%%%%%%%%%%%%%%%%%%%%%%%%%%%%%%%%%%%%%%%%%%%%%%%%%%%%%%%%%%%%%%%%%%%
%
% Package pgfplots.sty documentation.
%
% Copyright 2007/2008 by Christian Feuersaenger.
%
% This program is free software: you can redistribute it and/or modify
% it under the terms of the GNU General Public License as published by
% the Free Software Foundation, either version 3 of the License, or
% (at your option) any later version.
%
% This program is distributed in the hope that it will be useful,
% but WITHOUT ANY WARRANTY; without even the implied warranty of
% MERCHANTABILITY or FITNESS FOR A PARTICULAR PURPOSE.  See the
% GNU General Public License for more details.
%
% You should have received a copy of the GNU General Public License
% along with this program.  If not, see <http://www.gnu.org/licenses/>.
%
%
%%%%%%%%%%%%%%%%%%%%%%%%%%%%%%%%%%%%%%%%%%%%%%%%%%%%%%%%%%%%%%%%%%%%%%%%%%%%%
% SEE pgfplots-macros.tex as well!
%\pdfminorversion=5 % to allow compression
%\pdfobjcompresslevel=2
\documentclass[a4paper,openany]{book}

\let\bookmaketitle=\maketitle

% -----------------------------------
% this here is from ltxdoc:
\usepackage{doc}[=v2]
\AtBeginDocument{\MakeShortVerb{\|}}
%\setlength{\textwidth}{355pt}
%\addtolength\marginparwidth{30pt}
%\addtolength\oddsidemargin{20pt}
%\addtolength\evensidemargin{20pt}
\def\cmd#1{\cs{\expandafter\cmd@to@cs\string#1}}
\def\cmd@to@cs#1#2{\char\number`#2\relax}
\DeclareRobustCommand\cs[1]{\texttt{\char`\\#1}}
\providecommand\marg[1]{%
  {\ttfamily\char`\{}\meta{#1}{\ttfamily\char`\}}}
\providecommand\oarg[1]{%
  {\ttfamily[}\meta{#1}{\ttfamily]}}
\providecommand\parg[1]{%
  {\ttfamily(}\meta{#1}{\ttfamily)}}
\raggedbottom

% -----------------------------------
\input{pgfplots.preamble.tex}

\makeatletter
% I want a two-column index just as in pgfmanual styles. This here
% was the best way to get one:
\def\index@prologue{\section*{Index}\addcontentsline{toc}{chapter}{Index}
}
\makeatother

%\RequirePackage[german,english,francais]{babel}

\def\matlabcolormaptext{This colormap is similar to one shipped with Matlab$^\text{\textregistered}$ under a similar name.}

\IfFileExists{tikzlibraryspy.code.tex}{%
\usetikzlibrary{spy}
}{%
    \message{ERROR: tikz SPY library NOT available. The manual will only compile partially.^^J}%
}%

\usepackage{xparse}% for colorbrewer manual

\usetikzlibrary{
    decorations.markings,
    decorations.footprints,
    shapes.arrows,
    matrix,
    positioning,
}

\usepgfplotslibrary{
    fillbetween,
    ternary,
    smithchart,
    patchplots,
    polar,
    colormaps,
    colorbrewer,
    colortol,           % docu not ready yet
}
\pgfqkeys{/codeexample}{%
    every codeexample/.append style={
        /pgfplots/every ternary axis/.append style={
            /pgfplots/legend style={fill=graphicbackground},
        }
    },
    tabsize=4,
}

\pgfplotsmanualenableexternalizationofexpensive

%\usetikzlibrary{external}
%\tikzexternalize[prefix=figures/]{pgfplots}

\title{%
    Manual for Package \PGFPlots{}\\
    {\small 2D/3D Plots in \LaTeX{}, Version \pgfplotsversion}\\
    {\small\url{https://github.com/pgf-tikz/pgfplots}}
    %\\{\small Attention: you are using an unstable development version.}
}

\makeatletter
\long\def\abstractsmuggle{%
    \centering
    \textbf{Abstract}\\[0.5cm]

    \begin{minipage}{12cm}
        \PGFPlots{} draws high-quality function plots in normal or logarithmic
        scaling with a user-friendly interface directly in \TeX{}. The user
        supplies axis labels, legend entries and the plot coordinates for one or
        more plots and \PGFPlots{} applies axis scaling, computes any logarithms
        and axis ticks and draws the plots. It supports line plots, scatter
        plots, piecewise constant plots, bar plots, area plots, mesh and surface
        plots, patch plots, contour plots, quiver plots, histogram plots, box
        plots, polar axes, ternary diagrams, smith charts and some more. It is
        based on Till Tantau's package \PGF{}/\Tikz{}.
    \end{minipage}

}%

\expandafter\date\expandafter{\@date\\[2cm]
    \abstractsmuggle
}%
\makeatother

%\includeonly{pgfplots.reference}


\begin{document}

\def\plotcoords{%
\addplot coordinates {
(5,8.312e-02)    (17,2.547e-02)   (49,7.407e-03)
(129,2.102e-03)  (321,5.874e-04)  (769,1.623e-04)
(1793,4.442e-05) (4097,1.207e-05) (9217,3.261e-06)
};

\addplot coordinates{
(7,8.472e-02)    (31,3.044e-02)    (111,1.022e-02)
(351,3.303e-03)  (1023,1.039e-03)  (2815,3.196e-04)
(7423,9.658e-05) (18943,2.873e-05) (47103,8.437e-06)
};

\addplot coordinates{
(9,7.881e-02)     (49,3.243e-02)    (209,1.232e-02)
(769,4.454e-03)   (2561,1.551e-03)  (7937,5.236e-04)
(23297,1.723e-04) (65537,5.545e-05) (178177,1.751e-05)
};

\addplot coordinates{
(11,6.887e-02)    (71,3.177e-02)     (351,1.341e-02)
(1471,5.334e-03)  (5503,2.027e-03)   (18943,7.415e-04)
(61183,2.628e-04) (187903,9.063e-05) (553983,3.053e-05)
};

\addplot coordinates{
(13,5.755e-02)     (97,2.925e-02)     (545,1.351e-02)
(2561,5.842e-03)   (10625,2.397e-03)  (40193,9.414e-04)
(141569,3.564e-04) (471041,1.308e-04)
(1496065,4.670e-05)
};
}%


\bookmaketitle
\tableofcontents

%%%%%%%%%%%%%%%%%%%%%%%%%%%%%%%%%%%%%%%%%%%%%%%%%%%%%%%%%%%%%%%%%%%%%%%%%%%%%
%
% Package pgfplots.sty documentation.
%
% Copyright 2007/2008 by Christian Feuersaenger.
%
% This program is free software: you can redistribute it and/or modify
% it under the terms of the GNU General Public License as published by
% the Free Software Foundation, either version 3 of the License, or
% (at your option) any later version.
%
% This program is distributed in the hope that it will be useful,
% but WITHOUT ANY WARRANTY; without even the implied warranty of
% MERCHANTABILITY or FITNESS FOR A PARTICULAR PURPOSE.  See the
% GNU General Public License for more details.
%
% You should have received a copy of the GNU General Public License
% along with this program.  If not, see <http://www.gnu.org/licenses/>.
%
%
%%%%%%%%%%%%%%%%%%%%%%%%%%%%%%%%%%%%%%%%%%%%%%%%%%%%%%%%%%%%%%%%%%%%%%%%%%%%%
% SEE pgfplots-macros.tex as well!
%\pdfminorversion=5 % to allow compression
%\pdfobjcompresslevel=2
\documentclass[a4paper,openany]{book}

\let\bookmaketitle=\maketitle

% -----------------------------------
% this here is from ltxdoc:
\usepackage{doc}[=v2]
\AtBeginDocument{\MakeShortVerb{\|}}
%\setlength{\textwidth}{355pt}
%\addtolength\marginparwidth{30pt}
%\addtolength\oddsidemargin{20pt}
%\addtolength\evensidemargin{20pt}
\def\cmd#1{\cs{\expandafter\cmd@to@cs\string#1}}
\def\cmd@to@cs#1#2{\char\number`#2\relax}
\DeclareRobustCommand\cs[1]{\texttt{\char`\\#1}}
\providecommand\marg[1]{%
  {\ttfamily\char`\{}\meta{#1}{\ttfamily\char`\}}}
\providecommand\oarg[1]{%
  {\ttfamily[}\meta{#1}{\ttfamily]}}
\providecommand\parg[1]{%
  {\ttfamily(}\meta{#1}{\ttfamily)}}
\raggedbottom

% -----------------------------------
\input{pgfplots.preamble.tex}

\makeatletter
% I want a two-column index just as in pgfmanual styles. This here
% was the best way to get one:
\def\index@prologue{\section*{Index}\addcontentsline{toc}{chapter}{Index}
}
\makeatother

%\RequirePackage[german,english,francais]{babel}

\def\matlabcolormaptext{This colormap is similar to one shipped with Matlab$^\text{\textregistered}$ under a similar name.}

\IfFileExists{tikzlibraryspy.code.tex}{%
\usetikzlibrary{spy}
}{%
    \message{ERROR: tikz SPY library NOT available. The manual will only compile partially.^^J}%
}%

\usepackage{xparse}% for colorbrewer manual

\usetikzlibrary{
    decorations.markings,
    decorations.footprints,
    shapes.arrows,
    matrix,
    positioning,
}

\usepgfplotslibrary{
    fillbetween,
    ternary,
    smithchart,
    patchplots,
    polar,
    colormaps,
    colorbrewer,
    colortol,           % docu not ready yet
}
\pgfqkeys{/codeexample}{%
    every codeexample/.append style={
        /pgfplots/every ternary axis/.append style={
            /pgfplots/legend style={fill=graphicbackground},
        }
    },
    tabsize=4,
}

\pgfplotsmanualenableexternalizationofexpensive

%\usetikzlibrary{external}
%\tikzexternalize[prefix=figures/]{pgfplots}

\title{%
    Manual for Package \PGFPlots{}\\
    {\small 2D/3D Plots in \LaTeX{}, Version \pgfplotsversion}\\
    {\small\url{https://github.com/pgf-tikz/pgfplots}}
    %\\{\small Attention: you are using an unstable development version.}
}

\makeatletter
\long\def\abstractsmuggle{%
    \centering
    \textbf{Abstract}\\[0.5cm]

    \begin{minipage}{12cm}
        \PGFPlots{} draws high-quality function plots in normal or logarithmic
        scaling with a user-friendly interface directly in \TeX{}. The user
        supplies axis labels, legend entries and the plot coordinates for one or
        more plots and \PGFPlots{} applies axis scaling, computes any logarithms
        and axis ticks and draws the plots. It supports line plots, scatter
        plots, piecewise constant plots, bar plots, area plots, mesh and surface
        plots, patch plots, contour plots, quiver plots, histogram plots, box
        plots, polar axes, ternary diagrams, smith charts and some more. It is
        based on Till Tantau's package \PGF{}/\Tikz{}.
    \end{minipage}

}%

\expandafter\date\expandafter{\@date\\[2cm]
    \abstractsmuggle
}%
\makeatother

%\includeonly{pgfplots.reference}


\begin{document}

\def\plotcoords{%
\addplot coordinates {
(5,8.312e-02)    (17,2.547e-02)   (49,7.407e-03)
(129,2.102e-03)  (321,5.874e-04)  (769,1.623e-04)
(1793,4.442e-05) (4097,1.207e-05) (9217,3.261e-06)
};

\addplot coordinates{
(7,8.472e-02)    (31,3.044e-02)    (111,1.022e-02)
(351,3.303e-03)  (1023,1.039e-03)  (2815,3.196e-04)
(7423,9.658e-05) (18943,2.873e-05) (47103,8.437e-06)
};

\addplot coordinates{
(9,7.881e-02)     (49,3.243e-02)    (209,1.232e-02)
(769,4.454e-03)   (2561,1.551e-03)  (7937,5.236e-04)
(23297,1.723e-04) (65537,5.545e-05) (178177,1.751e-05)
};

\addplot coordinates{
(11,6.887e-02)    (71,3.177e-02)     (351,1.341e-02)
(1471,5.334e-03)  (5503,2.027e-03)   (18943,7.415e-04)
(61183,2.628e-04) (187903,9.063e-05) (553983,3.053e-05)
};

\addplot coordinates{
(13,5.755e-02)     (97,2.925e-02)     (545,1.351e-02)
(2561,5.842e-03)   (10625,2.397e-03)  (40193,9.414e-04)
(141569,3.564e-04) (471041,1.308e-04)
(1496065,4.670e-05)
};
}%


\bookmaketitle
\tableofcontents
\include{pgfplots.title_abstract_intro}
\include{pgfplots.preliminaries}
\include{pgfplots.intro}
\include{pgfplots.reference}
\include{pgfplots.libs}
\include{pgfplots.resources}
\include{pgfplots.importexport}
\include{pgfplots.basic.reference}

\printindex

\bibliographystyle{abbrv} %gerapali} %gerabbrv} %gerunsrt.bst} %gerabbrv}% gerplain}
\nocite{pgfplotstable}
\nocite{programmingnotes}
\bibliography{pgfplots}
\end{document}

% -----------------------------------------------------------------------------
% For Stefan Pinnow as reminder on what to look for when editing the manual
% -----------------------------------------------------------------------------
% There should be no line breaks in the following environments
% - |...|
% - \declareandlabel{...}
% - \verbpdfref{...}
% If "MakeTikzPictures" isn't running through check one of the externalized
% LOG files what is the cause of that.
% -----------------------------------------------------------------------------


%%%%%%%%%%%%%%%%%%%%%%%%%%%%%%%%%%%%%%%%%%%%%%%%%%%%%%%%%%%%%%%%%%%%%%%%%%%%%
%
% Package pgfplots.sty documentation.
%
% Copyright 2007/2008 by Christian Feuersaenger.
%
% This program is free software: you can redistribute it and/or modify
% it under the terms of the GNU General Public License as published by
% the Free Software Foundation, either version 3 of the License, or
% (at your option) any later version.
%
% This program is distributed in the hope that it will be useful,
% but WITHOUT ANY WARRANTY; without even the implied warranty of
% MERCHANTABILITY or FITNESS FOR A PARTICULAR PURPOSE.  See the
% GNU General Public License for more details.
%
% You should have received a copy of the GNU General Public License
% along with this program.  If not, see <http://www.gnu.org/licenses/>.
%
%
%%%%%%%%%%%%%%%%%%%%%%%%%%%%%%%%%%%%%%%%%%%%%%%%%%%%%%%%%%%%%%%%%%%%%%%%%%%%%
% SEE pgfplots-macros.tex as well!
%\pdfminorversion=5 % to allow compression
%\pdfobjcompresslevel=2
\documentclass[a4paper,openany]{book}

\let\bookmaketitle=\maketitle

% -----------------------------------
% this here is from ltxdoc:
\usepackage{doc}[=v2]
\AtBeginDocument{\MakeShortVerb{\|}}
%\setlength{\textwidth}{355pt}
%\addtolength\marginparwidth{30pt}
%\addtolength\oddsidemargin{20pt}
%\addtolength\evensidemargin{20pt}
\def\cmd#1{\cs{\expandafter\cmd@to@cs\string#1}}
\def\cmd@to@cs#1#2{\char\number`#2\relax}
\DeclareRobustCommand\cs[1]{\texttt{\char`\\#1}}
\providecommand\marg[1]{%
  {\ttfamily\char`\{}\meta{#1}{\ttfamily\char`\}}}
\providecommand\oarg[1]{%
  {\ttfamily[}\meta{#1}{\ttfamily]}}
\providecommand\parg[1]{%
  {\ttfamily(}\meta{#1}{\ttfamily)}}
\raggedbottom

% -----------------------------------
\input{pgfplots.preamble.tex}

\makeatletter
% I want a two-column index just as in pgfmanual styles. This here
% was the best way to get one:
\def\index@prologue{\section*{Index}\addcontentsline{toc}{chapter}{Index}
}
\makeatother

%\RequirePackage[german,english,francais]{babel}

\def\matlabcolormaptext{This colormap is similar to one shipped with Matlab$^\text{\textregistered}$ under a similar name.}

\IfFileExists{tikzlibraryspy.code.tex}{%
\usetikzlibrary{spy}
}{%
    \message{ERROR: tikz SPY library NOT available. The manual will only compile partially.^^J}%
}%

\usepackage{xparse}% for colorbrewer manual

\usetikzlibrary{
    decorations.markings,
    decorations.footprints,
    shapes.arrows,
    matrix,
    positioning,
}

\usepgfplotslibrary{
    fillbetween,
    ternary,
    smithchart,
    patchplots,
    polar,
    colormaps,
    colorbrewer,
    colortol,           % docu not ready yet
}
\pgfqkeys{/codeexample}{%
    every codeexample/.append style={
        /pgfplots/every ternary axis/.append style={
            /pgfplots/legend style={fill=graphicbackground},
        }
    },
    tabsize=4,
}

\pgfplotsmanualenableexternalizationofexpensive

%\usetikzlibrary{external}
%\tikzexternalize[prefix=figures/]{pgfplots}

\title{%
    Manual for Package \PGFPlots{}\\
    {\small 2D/3D Plots in \LaTeX{}, Version \pgfplotsversion}\\
    {\small\url{https://github.com/pgf-tikz/pgfplots}}
    %\\{\small Attention: you are using an unstable development version.}
}

\makeatletter
\long\def\abstractsmuggle{%
    \centering
    \textbf{Abstract}\\[0.5cm]

    \begin{minipage}{12cm}
        \PGFPlots{} draws high-quality function plots in normal or logarithmic
        scaling with a user-friendly interface directly in \TeX{}. The user
        supplies axis labels, legend entries and the plot coordinates for one or
        more plots and \PGFPlots{} applies axis scaling, computes any logarithms
        and axis ticks and draws the plots. It supports line plots, scatter
        plots, piecewise constant plots, bar plots, area plots, mesh and surface
        plots, patch plots, contour plots, quiver plots, histogram plots, box
        plots, polar axes, ternary diagrams, smith charts and some more. It is
        based on Till Tantau's package \PGF{}/\Tikz{}.
    \end{minipage}

}%

\expandafter\date\expandafter{\@date\\[2cm]
    \abstractsmuggle
}%
\makeatother

%\includeonly{pgfplots.reference}


\begin{document}

\def\plotcoords{%
\addplot coordinates {
(5,8.312e-02)    (17,2.547e-02)   (49,7.407e-03)
(129,2.102e-03)  (321,5.874e-04)  (769,1.623e-04)
(1793,4.442e-05) (4097,1.207e-05) (9217,3.261e-06)
};

\addplot coordinates{
(7,8.472e-02)    (31,3.044e-02)    (111,1.022e-02)
(351,3.303e-03)  (1023,1.039e-03)  (2815,3.196e-04)
(7423,9.658e-05) (18943,2.873e-05) (47103,8.437e-06)
};

\addplot coordinates{
(9,7.881e-02)     (49,3.243e-02)    (209,1.232e-02)
(769,4.454e-03)   (2561,1.551e-03)  (7937,5.236e-04)
(23297,1.723e-04) (65537,5.545e-05) (178177,1.751e-05)
};

\addplot coordinates{
(11,6.887e-02)    (71,3.177e-02)     (351,1.341e-02)
(1471,5.334e-03)  (5503,2.027e-03)   (18943,7.415e-04)
(61183,2.628e-04) (187903,9.063e-05) (553983,3.053e-05)
};

\addplot coordinates{
(13,5.755e-02)     (97,2.925e-02)     (545,1.351e-02)
(2561,5.842e-03)   (10625,2.397e-03)  (40193,9.414e-04)
(141569,3.564e-04) (471041,1.308e-04)
(1496065,4.670e-05)
};
}%


\bookmaketitle
\tableofcontents
\include{pgfplots.title_abstract_intro}
\include{pgfplots.preliminaries}
\include{pgfplots.intro}
\include{pgfplots.reference}
\include{pgfplots.libs}
\include{pgfplots.resources}
\include{pgfplots.importexport}
\include{pgfplots.basic.reference}

\printindex

\bibliographystyle{abbrv} %gerapali} %gerabbrv} %gerunsrt.bst} %gerabbrv}% gerplain}
\nocite{pgfplotstable}
\nocite{programmingnotes}
\bibliography{pgfplots}
\end{document}

% -----------------------------------------------------------------------------
% For Stefan Pinnow as reminder on what to look for when editing the manual
% -----------------------------------------------------------------------------
% There should be no line breaks in the following environments
% - |...|
% - \declareandlabel{...}
% - \verbpdfref{...}
% If "MakeTikzPictures" isn't running through check one of the externalized
% LOG files what is the cause of that.
% -----------------------------------------------------------------------------


%%%%%%%%%%%%%%%%%%%%%%%%%%%%%%%%%%%%%%%%%%%%%%%%%%%%%%%%%%%%%%%%%%%%%%%%%%%%%
%
% Package pgfplots.sty documentation.
%
% Copyright 2007/2008 by Christian Feuersaenger.
%
% This program is free software: you can redistribute it and/or modify
% it under the terms of the GNU General Public License as published by
% the Free Software Foundation, either version 3 of the License, or
% (at your option) any later version.
%
% This program is distributed in the hope that it will be useful,
% but WITHOUT ANY WARRANTY; without even the implied warranty of
% MERCHANTABILITY or FITNESS FOR A PARTICULAR PURPOSE.  See the
% GNU General Public License for more details.
%
% You should have received a copy of the GNU General Public License
% along with this program.  If not, see <http://www.gnu.org/licenses/>.
%
%
%%%%%%%%%%%%%%%%%%%%%%%%%%%%%%%%%%%%%%%%%%%%%%%%%%%%%%%%%%%%%%%%%%%%%%%%%%%%%
% SEE pgfplots-macros.tex as well!
%\pdfminorversion=5 % to allow compression
%\pdfobjcompresslevel=2
\documentclass[a4paper,openany]{book}

\let\bookmaketitle=\maketitle

% -----------------------------------
% this here is from ltxdoc:
\usepackage{doc}[=v2]
\AtBeginDocument{\MakeShortVerb{\|}}
%\setlength{\textwidth}{355pt}
%\addtolength\marginparwidth{30pt}
%\addtolength\oddsidemargin{20pt}
%\addtolength\evensidemargin{20pt}
\def\cmd#1{\cs{\expandafter\cmd@to@cs\string#1}}
\def\cmd@to@cs#1#2{\char\number`#2\relax}
\DeclareRobustCommand\cs[1]{\texttt{\char`\\#1}}
\providecommand\marg[1]{%
  {\ttfamily\char`\{}\meta{#1}{\ttfamily\char`\}}}
\providecommand\oarg[1]{%
  {\ttfamily[}\meta{#1}{\ttfamily]}}
\providecommand\parg[1]{%
  {\ttfamily(}\meta{#1}{\ttfamily)}}
\raggedbottom

% -----------------------------------
\input{pgfplots.preamble.tex}

\makeatletter
% I want a two-column index just as in pgfmanual styles. This here
% was the best way to get one:
\def\index@prologue{\section*{Index}\addcontentsline{toc}{chapter}{Index}
}
\makeatother

%\RequirePackage[german,english,francais]{babel}

\def\matlabcolormaptext{This colormap is similar to one shipped with Matlab$^\text{\textregistered}$ under a similar name.}

\IfFileExists{tikzlibraryspy.code.tex}{%
\usetikzlibrary{spy}
}{%
    \message{ERROR: tikz SPY library NOT available. The manual will only compile partially.^^J}%
}%

\usepackage{xparse}% for colorbrewer manual

\usetikzlibrary{
    decorations.markings,
    decorations.footprints,
    shapes.arrows,
    matrix,
    positioning,
}

\usepgfplotslibrary{
    fillbetween,
    ternary,
    smithchart,
    patchplots,
    polar,
    colormaps,
    colorbrewer,
    colortol,           % docu not ready yet
}
\pgfqkeys{/codeexample}{%
    every codeexample/.append style={
        /pgfplots/every ternary axis/.append style={
            /pgfplots/legend style={fill=graphicbackground},
        }
    },
    tabsize=4,
}

\pgfplotsmanualenableexternalizationofexpensive

%\usetikzlibrary{external}
%\tikzexternalize[prefix=figures/]{pgfplots}

\title{%
    Manual for Package \PGFPlots{}\\
    {\small 2D/3D Plots in \LaTeX{}, Version \pgfplotsversion}\\
    {\small\url{https://github.com/pgf-tikz/pgfplots}}
    %\\{\small Attention: you are using an unstable development version.}
}

\makeatletter
\long\def\abstractsmuggle{%
    \centering
    \textbf{Abstract}\\[0.5cm]

    \begin{minipage}{12cm}
        \PGFPlots{} draws high-quality function plots in normal or logarithmic
        scaling with a user-friendly interface directly in \TeX{}. The user
        supplies axis labels, legend entries and the plot coordinates for one or
        more plots and \PGFPlots{} applies axis scaling, computes any logarithms
        and axis ticks and draws the plots. It supports line plots, scatter
        plots, piecewise constant plots, bar plots, area plots, mesh and surface
        plots, patch plots, contour plots, quiver plots, histogram plots, box
        plots, polar axes, ternary diagrams, smith charts and some more. It is
        based on Till Tantau's package \PGF{}/\Tikz{}.
    \end{minipage}

}%

\expandafter\date\expandafter{\@date\\[2cm]
    \abstractsmuggle
}%
\makeatother

%\includeonly{pgfplots.reference}


\begin{document}

\def\plotcoords{%
\addplot coordinates {
(5,8.312e-02)    (17,2.547e-02)   (49,7.407e-03)
(129,2.102e-03)  (321,5.874e-04)  (769,1.623e-04)
(1793,4.442e-05) (4097,1.207e-05) (9217,3.261e-06)
};

\addplot coordinates{
(7,8.472e-02)    (31,3.044e-02)    (111,1.022e-02)
(351,3.303e-03)  (1023,1.039e-03)  (2815,3.196e-04)
(7423,9.658e-05) (18943,2.873e-05) (47103,8.437e-06)
};

\addplot coordinates{
(9,7.881e-02)     (49,3.243e-02)    (209,1.232e-02)
(769,4.454e-03)   (2561,1.551e-03)  (7937,5.236e-04)
(23297,1.723e-04) (65537,5.545e-05) (178177,1.751e-05)
};

\addplot coordinates{
(11,6.887e-02)    (71,3.177e-02)     (351,1.341e-02)
(1471,5.334e-03)  (5503,2.027e-03)   (18943,7.415e-04)
(61183,2.628e-04) (187903,9.063e-05) (553983,3.053e-05)
};

\addplot coordinates{
(13,5.755e-02)     (97,2.925e-02)     (545,1.351e-02)
(2561,5.842e-03)   (10625,2.397e-03)  (40193,9.414e-04)
(141569,3.564e-04) (471041,1.308e-04)
(1496065,4.670e-05)
};
}%


\bookmaketitle
\tableofcontents
\include{pgfplots.title_abstract_intro}
\include{pgfplots.preliminaries}
\include{pgfplots.intro}
\include{pgfplots.reference}
\include{pgfplots.libs}
\include{pgfplots.resources}
\include{pgfplots.importexport}
\include{pgfplots.basic.reference}

\printindex

\bibliographystyle{abbrv} %gerapali} %gerabbrv} %gerunsrt.bst} %gerabbrv}% gerplain}
\nocite{pgfplotstable}
\nocite{programmingnotes}
\bibliography{pgfplots}
\end{document}

% -----------------------------------------------------------------------------
% For Stefan Pinnow as reminder on what to look for when editing the manual
% -----------------------------------------------------------------------------
% There should be no line breaks in the following environments
% - |...|
% - \declareandlabel{...}
% - \verbpdfref{...}
% If "MakeTikzPictures" isn't running through check one of the externalized
% LOG files what is the cause of that.
% -----------------------------------------------------------------------------


%%%%%%%%%%%%%%%%%%%%%%%%%%%%%%%%%%%%%%%%%%%%%%%%%%%%%%%%%%%%%%%%%%%%%%%%%%%%%
%
% Package pgfplots.sty documentation.
%
% Copyright 2007/2008 by Christian Feuersaenger.
%
% This program is free software: you can redistribute it and/or modify
% it under the terms of the GNU General Public License as published by
% the Free Software Foundation, either version 3 of the License, or
% (at your option) any later version.
%
% This program is distributed in the hope that it will be useful,
% but WITHOUT ANY WARRANTY; without even the implied warranty of
% MERCHANTABILITY or FITNESS FOR A PARTICULAR PURPOSE.  See the
% GNU General Public License for more details.
%
% You should have received a copy of the GNU General Public License
% along with this program.  If not, see <http://www.gnu.org/licenses/>.
%
%
%%%%%%%%%%%%%%%%%%%%%%%%%%%%%%%%%%%%%%%%%%%%%%%%%%%%%%%%%%%%%%%%%%%%%%%%%%%%%
% SEE pgfplots-macros.tex as well!
%\pdfminorversion=5 % to allow compression
%\pdfobjcompresslevel=2
\documentclass[a4paper,openany]{book}

\let\bookmaketitle=\maketitle

% -----------------------------------
% this here is from ltxdoc:
\usepackage{doc}[=v2]
\AtBeginDocument{\MakeShortVerb{\|}}
%\setlength{\textwidth}{355pt}
%\addtolength\marginparwidth{30pt}
%\addtolength\oddsidemargin{20pt}
%\addtolength\evensidemargin{20pt}
\def\cmd#1{\cs{\expandafter\cmd@to@cs\string#1}}
\def\cmd@to@cs#1#2{\char\number`#2\relax}
\DeclareRobustCommand\cs[1]{\texttt{\char`\\#1}}
\providecommand\marg[1]{%
  {\ttfamily\char`\{}\meta{#1}{\ttfamily\char`\}}}
\providecommand\oarg[1]{%
  {\ttfamily[}\meta{#1}{\ttfamily]}}
\providecommand\parg[1]{%
  {\ttfamily(}\meta{#1}{\ttfamily)}}
\raggedbottom

% -----------------------------------
\input{pgfplots.preamble.tex}

\makeatletter
% I want a two-column index just as in pgfmanual styles. This here
% was the best way to get one:
\def\index@prologue{\section*{Index}\addcontentsline{toc}{chapter}{Index}
}
\makeatother

%\RequirePackage[german,english,francais]{babel}

\def\matlabcolormaptext{This colormap is similar to one shipped with Matlab$^\text{\textregistered}$ under a similar name.}

\IfFileExists{tikzlibraryspy.code.tex}{%
\usetikzlibrary{spy}
}{%
    \message{ERROR: tikz SPY library NOT available. The manual will only compile partially.^^J}%
}%

\usepackage{xparse}% for colorbrewer manual

\usetikzlibrary{
    decorations.markings,
    decorations.footprints,
    shapes.arrows,
    matrix,
    positioning,
}

\usepgfplotslibrary{
    fillbetween,
    ternary,
    smithchart,
    patchplots,
    polar,
    colormaps,
    colorbrewer,
    colortol,           % docu not ready yet
}
\pgfqkeys{/codeexample}{%
    every codeexample/.append style={
        /pgfplots/every ternary axis/.append style={
            /pgfplots/legend style={fill=graphicbackground},
        }
    },
    tabsize=4,
}

\pgfplotsmanualenableexternalizationofexpensive

%\usetikzlibrary{external}
%\tikzexternalize[prefix=figures/]{pgfplots}

\title{%
    Manual for Package \PGFPlots{}\\
    {\small 2D/3D Plots in \LaTeX{}, Version \pgfplotsversion}\\
    {\small\url{https://github.com/pgf-tikz/pgfplots}}
    %\\{\small Attention: you are using an unstable development version.}
}

\makeatletter
\long\def\abstractsmuggle{%
    \centering
    \textbf{Abstract}\\[0.5cm]

    \begin{minipage}{12cm}
        \PGFPlots{} draws high-quality function plots in normal or logarithmic
        scaling with a user-friendly interface directly in \TeX{}. The user
        supplies axis labels, legend entries and the plot coordinates for one or
        more plots and \PGFPlots{} applies axis scaling, computes any logarithms
        and axis ticks and draws the plots. It supports line plots, scatter
        plots, piecewise constant plots, bar plots, area plots, mesh and surface
        plots, patch plots, contour plots, quiver plots, histogram plots, box
        plots, polar axes, ternary diagrams, smith charts and some more. It is
        based on Till Tantau's package \PGF{}/\Tikz{}.
    \end{minipage}

}%

\expandafter\date\expandafter{\@date\\[2cm]
    \abstractsmuggle
}%
\makeatother

%\includeonly{pgfplots.reference}


\begin{document}

\def\plotcoords{%
\addplot coordinates {
(5,8.312e-02)    (17,2.547e-02)   (49,7.407e-03)
(129,2.102e-03)  (321,5.874e-04)  (769,1.623e-04)
(1793,4.442e-05) (4097,1.207e-05) (9217,3.261e-06)
};

\addplot coordinates{
(7,8.472e-02)    (31,3.044e-02)    (111,1.022e-02)
(351,3.303e-03)  (1023,1.039e-03)  (2815,3.196e-04)
(7423,9.658e-05) (18943,2.873e-05) (47103,8.437e-06)
};

\addplot coordinates{
(9,7.881e-02)     (49,3.243e-02)    (209,1.232e-02)
(769,4.454e-03)   (2561,1.551e-03)  (7937,5.236e-04)
(23297,1.723e-04) (65537,5.545e-05) (178177,1.751e-05)
};

\addplot coordinates{
(11,6.887e-02)    (71,3.177e-02)     (351,1.341e-02)
(1471,5.334e-03)  (5503,2.027e-03)   (18943,7.415e-04)
(61183,2.628e-04) (187903,9.063e-05) (553983,3.053e-05)
};

\addplot coordinates{
(13,5.755e-02)     (97,2.925e-02)     (545,1.351e-02)
(2561,5.842e-03)   (10625,2.397e-03)  (40193,9.414e-04)
(141569,3.564e-04) (471041,1.308e-04)
(1496065,4.670e-05)
};
}%


\bookmaketitle
\tableofcontents
\include{pgfplots.title_abstract_intro}
\include{pgfplots.preliminaries}
\include{pgfplots.intro}
\include{pgfplots.reference}
\include{pgfplots.libs}
\include{pgfplots.resources}
\include{pgfplots.importexport}
\include{pgfplots.basic.reference}

\printindex

\bibliographystyle{abbrv} %gerapali} %gerabbrv} %gerunsrt.bst} %gerabbrv}% gerplain}
\nocite{pgfplotstable}
\nocite{programmingnotes}
\bibliography{pgfplots}
\end{document}

% -----------------------------------------------------------------------------
% For Stefan Pinnow as reminder on what to look for when editing the manual
% -----------------------------------------------------------------------------
% There should be no line breaks in the following environments
% - |...|
% - \declareandlabel{...}
% - \verbpdfref{...}
% If "MakeTikzPictures" isn't running through check one of the externalized
% LOG files what is the cause of that.
% -----------------------------------------------------------------------------


%%%%%%%%%%%%%%%%%%%%%%%%%%%%%%%%%%%%%%%%%%%%%%%%%%%%%%%%%%%%%%%%%%%%%%%%%%%%%
%
% Package pgfplots.sty documentation.
%
% Copyright 2007/2008 by Christian Feuersaenger.
%
% This program is free software: you can redistribute it and/or modify
% it under the terms of the GNU General Public License as published by
% the Free Software Foundation, either version 3 of the License, or
% (at your option) any later version.
%
% This program is distributed in the hope that it will be useful,
% but WITHOUT ANY WARRANTY; without even the implied warranty of
% MERCHANTABILITY or FITNESS FOR A PARTICULAR PURPOSE.  See the
% GNU General Public License for more details.
%
% You should have received a copy of the GNU General Public License
% along with this program.  If not, see <http://www.gnu.org/licenses/>.
%
%
%%%%%%%%%%%%%%%%%%%%%%%%%%%%%%%%%%%%%%%%%%%%%%%%%%%%%%%%%%%%%%%%%%%%%%%%%%%%%
% SEE pgfplots-macros.tex as well!
%\pdfminorversion=5 % to allow compression
%\pdfobjcompresslevel=2
\documentclass[a4paper,openany]{book}

\let\bookmaketitle=\maketitle

% -----------------------------------
% this here is from ltxdoc:
\usepackage{doc}[=v2]
\AtBeginDocument{\MakeShortVerb{\|}}
%\setlength{\textwidth}{355pt}
%\addtolength\marginparwidth{30pt}
%\addtolength\oddsidemargin{20pt}
%\addtolength\evensidemargin{20pt}
\def\cmd#1{\cs{\expandafter\cmd@to@cs\string#1}}
\def\cmd@to@cs#1#2{\char\number`#2\relax}
\DeclareRobustCommand\cs[1]{\texttt{\char`\\#1}}
\providecommand\marg[1]{%
  {\ttfamily\char`\{}\meta{#1}{\ttfamily\char`\}}}
\providecommand\oarg[1]{%
  {\ttfamily[}\meta{#1}{\ttfamily]}}
\providecommand\parg[1]{%
  {\ttfamily(}\meta{#1}{\ttfamily)}}
\raggedbottom

% -----------------------------------
\input{pgfplots.preamble.tex}

\makeatletter
% I want a two-column index just as in pgfmanual styles. This here
% was the best way to get one:
\def\index@prologue{\section*{Index}\addcontentsline{toc}{chapter}{Index}
}
\makeatother

%\RequirePackage[german,english,francais]{babel}

\def\matlabcolormaptext{This colormap is similar to one shipped with Matlab$^\text{\textregistered}$ under a similar name.}

\IfFileExists{tikzlibraryspy.code.tex}{%
\usetikzlibrary{spy}
}{%
    \message{ERROR: tikz SPY library NOT available. The manual will only compile partially.^^J}%
}%

\usepackage{xparse}% for colorbrewer manual

\usetikzlibrary{
    decorations.markings,
    decorations.footprints,
    shapes.arrows,
    matrix,
    positioning,
}

\usepgfplotslibrary{
    fillbetween,
    ternary,
    smithchart,
    patchplots,
    polar,
    colormaps,
    colorbrewer,
    colortol,           % docu not ready yet
}
\pgfqkeys{/codeexample}{%
    every codeexample/.append style={
        /pgfplots/every ternary axis/.append style={
            /pgfplots/legend style={fill=graphicbackground},
        }
    },
    tabsize=4,
}

\pgfplotsmanualenableexternalizationofexpensive

%\usetikzlibrary{external}
%\tikzexternalize[prefix=figures/]{pgfplots}

\title{%
    Manual for Package \PGFPlots{}\\
    {\small 2D/3D Plots in \LaTeX{}, Version \pgfplotsversion}\\
    {\small\url{https://github.com/pgf-tikz/pgfplots}}
    %\\{\small Attention: you are using an unstable development version.}
}

\makeatletter
\long\def\abstractsmuggle{%
    \centering
    \textbf{Abstract}\\[0.5cm]

    \begin{minipage}{12cm}
        \PGFPlots{} draws high-quality function plots in normal or logarithmic
        scaling with a user-friendly interface directly in \TeX{}. The user
        supplies axis labels, legend entries and the plot coordinates for one or
        more plots and \PGFPlots{} applies axis scaling, computes any logarithms
        and axis ticks and draws the plots. It supports line plots, scatter
        plots, piecewise constant plots, bar plots, area plots, mesh and surface
        plots, patch plots, contour plots, quiver plots, histogram plots, box
        plots, polar axes, ternary diagrams, smith charts and some more. It is
        based on Till Tantau's package \PGF{}/\Tikz{}.
    \end{minipage}

}%

\expandafter\date\expandafter{\@date\\[2cm]
    \abstractsmuggle
}%
\makeatother

%\includeonly{pgfplots.reference}


\begin{document}

\def\plotcoords{%
\addplot coordinates {
(5,8.312e-02)    (17,2.547e-02)   (49,7.407e-03)
(129,2.102e-03)  (321,5.874e-04)  (769,1.623e-04)
(1793,4.442e-05) (4097,1.207e-05) (9217,3.261e-06)
};

\addplot coordinates{
(7,8.472e-02)    (31,3.044e-02)    (111,1.022e-02)
(351,3.303e-03)  (1023,1.039e-03)  (2815,3.196e-04)
(7423,9.658e-05) (18943,2.873e-05) (47103,8.437e-06)
};

\addplot coordinates{
(9,7.881e-02)     (49,3.243e-02)    (209,1.232e-02)
(769,4.454e-03)   (2561,1.551e-03)  (7937,5.236e-04)
(23297,1.723e-04) (65537,5.545e-05) (178177,1.751e-05)
};

\addplot coordinates{
(11,6.887e-02)    (71,3.177e-02)     (351,1.341e-02)
(1471,5.334e-03)  (5503,2.027e-03)   (18943,7.415e-04)
(61183,2.628e-04) (187903,9.063e-05) (553983,3.053e-05)
};

\addplot coordinates{
(13,5.755e-02)     (97,2.925e-02)     (545,1.351e-02)
(2561,5.842e-03)   (10625,2.397e-03)  (40193,9.414e-04)
(141569,3.564e-04) (471041,1.308e-04)
(1496065,4.670e-05)
};
}%


\bookmaketitle
\tableofcontents
\include{pgfplots.title_abstract_intro}
\include{pgfplots.preliminaries}
\include{pgfplots.intro}
\include{pgfplots.reference}
\include{pgfplots.libs}
\include{pgfplots.resources}
\include{pgfplots.importexport}
\include{pgfplots.basic.reference}

\printindex

\bibliographystyle{abbrv} %gerapali} %gerabbrv} %gerunsrt.bst} %gerabbrv}% gerplain}
\nocite{pgfplotstable}
\nocite{programmingnotes}
\bibliography{pgfplots}
\end{document}

% -----------------------------------------------------------------------------
% For Stefan Pinnow as reminder on what to look for when editing the manual
% -----------------------------------------------------------------------------
% There should be no line breaks in the following environments
% - |...|
% - \declareandlabel{...}
% - \verbpdfref{...}
% If "MakeTikzPictures" isn't running through check one of the externalized
% LOG files what is the cause of that.
% -----------------------------------------------------------------------------


%%%%%%%%%%%%%%%%%%%%%%%%%%%%%%%%%%%%%%%%%%%%%%%%%%%%%%%%%%%%%%%%%%%%%%%%%%%%%
%
% Package pgfplots.sty documentation.
%
% Copyright 2007/2008 by Christian Feuersaenger.
%
% This program is free software: you can redistribute it and/or modify
% it under the terms of the GNU General Public License as published by
% the Free Software Foundation, either version 3 of the License, or
% (at your option) any later version.
%
% This program is distributed in the hope that it will be useful,
% but WITHOUT ANY WARRANTY; without even the implied warranty of
% MERCHANTABILITY or FITNESS FOR A PARTICULAR PURPOSE.  See the
% GNU General Public License for more details.
%
% You should have received a copy of the GNU General Public License
% along with this program.  If not, see <http://www.gnu.org/licenses/>.
%
%
%%%%%%%%%%%%%%%%%%%%%%%%%%%%%%%%%%%%%%%%%%%%%%%%%%%%%%%%%%%%%%%%%%%%%%%%%%%%%
% SEE pgfplots-macros.tex as well!
%\pdfminorversion=5 % to allow compression
%\pdfobjcompresslevel=2
\documentclass[a4paper,openany]{book}

\let\bookmaketitle=\maketitle

% -----------------------------------
% this here is from ltxdoc:
\usepackage{doc}[=v2]
\AtBeginDocument{\MakeShortVerb{\|}}
%\setlength{\textwidth}{355pt}
%\addtolength\marginparwidth{30pt}
%\addtolength\oddsidemargin{20pt}
%\addtolength\evensidemargin{20pt}
\def\cmd#1{\cs{\expandafter\cmd@to@cs\string#1}}
\def\cmd@to@cs#1#2{\char\number`#2\relax}
\DeclareRobustCommand\cs[1]{\texttt{\char`\\#1}}
\providecommand\marg[1]{%
  {\ttfamily\char`\{}\meta{#1}{\ttfamily\char`\}}}
\providecommand\oarg[1]{%
  {\ttfamily[}\meta{#1}{\ttfamily]}}
\providecommand\parg[1]{%
  {\ttfamily(}\meta{#1}{\ttfamily)}}
\raggedbottom

% -----------------------------------
\input{pgfplots.preamble.tex}

\makeatletter
% I want a two-column index just as in pgfmanual styles. This here
% was the best way to get one:
\def\index@prologue{\section*{Index}\addcontentsline{toc}{chapter}{Index}
}
\makeatother

%\RequirePackage[german,english,francais]{babel}

\def\matlabcolormaptext{This colormap is similar to one shipped with Matlab$^\text{\textregistered}$ under a similar name.}

\IfFileExists{tikzlibraryspy.code.tex}{%
\usetikzlibrary{spy}
}{%
    \message{ERROR: tikz SPY library NOT available. The manual will only compile partially.^^J}%
}%

\usepackage{xparse}% for colorbrewer manual

\usetikzlibrary{
    decorations.markings,
    decorations.footprints,
    shapes.arrows,
    matrix,
    positioning,
}

\usepgfplotslibrary{
    fillbetween,
    ternary,
    smithchart,
    patchplots,
    polar,
    colormaps,
    colorbrewer,
    colortol,           % docu not ready yet
}
\pgfqkeys{/codeexample}{%
    every codeexample/.append style={
        /pgfplots/every ternary axis/.append style={
            /pgfplots/legend style={fill=graphicbackground},
        }
    },
    tabsize=4,
}

\pgfplotsmanualenableexternalizationofexpensive

%\usetikzlibrary{external}
%\tikzexternalize[prefix=figures/]{pgfplots}

\title{%
    Manual for Package \PGFPlots{}\\
    {\small 2D/3D Plots in \LaTeX{}, Version \pgfplotsversion}\\
    {\small\url{https://github.com/pgf-tikz/pgfplots}}
    %\\{\small Attention: you are using an unstable development version.}
}

\makeatletter
\long\def\abstractsmuggle{%
    \centering
    \textbf{Abstract}\\[0.5cm]

    \begin{minipage}{12cm}
        \PGFPlots{} draws high-quality function plots in normal or logarithmic
        scaling with a user-friendly interface directly in \TeX{}. The user
        supplies axis labels, legend entries and the plot coordinates for one or
        more plots and \PGFPlots{} applies axis scaling, computes any logarithms
        and axis ticks and draws the plots. It supports line plots, scatter
        plots, piecewise constant plots, bar plots, area plots, mesh and surface
        plots, patch plots, contour plots, quiver plots, histogram plots, box
        plots, polar axes, ternary diagrams, smith charts and some more. It is
        based on Till Tantau's package \PGF{}/\Tikz{}.
    \end{minipage}

}%

\expandafter\date\expandafter{\@date\\[2cm]
    \abstractsmuggle
}%
\makeatother

%\includeonly{pgfplots.reference}


\begin{document}

\def\plotcoords{%
\addplot coordinates {
(5,8.312e-02)    (17,2.547e-02)   (49,7.407e-03)
(129,2.102e-03)  (321,5.874e-04)  (769,1.623e-04)
(1793,4.442e-05) (4097,1.207e-05) (9217,3.261e-06)
};

\addplot coordinates{
(7,8.472e-02)    (31,3.044e-02)    (111,1.022e-02)
(351,3.303e-03)  (1023,1.039e-03)  (2815,3.196e-04)
(7423,9.658e-05) (18943,2.873e-05) (47103,8.437e-06)
};

\addplot coordinates{
(9,7.881e-02)     (49,3.243e-02)    (209,1.232e-02)
(769,4.454e-03)   (2561,1.551e-03)  (7937,5.236e-04)
(23297,1.723e-04) (65537,5.545e-05) (178177,1.751e-05)
};

\addplot coordinates{
(11,6.887e-02)    (71,3.177e-02)     (351,1.341e-02)
(1471,5.334e-03)  (5503,2.027e-03)   (18943,7.415e-04)
(61183,2.628e-04) (187903,9.063e-05) (553983,3.053e-05)
};

\addplot coordinates{
(13,5.755e-02)     (97,2.925e-02)     (545,1.351e-02)
(2561,5.842e-03)   (10625,2.397e-03)  (40193,9.414e-04)
(141569,3.564e-04) (471041,1.308e-04)
(1496065,4.670e-05)
};
}%


\bookmaketitle
\tableofcontents
\include{pgfplots.title_abstract_intro}
\include{pgfplots.preliminaries}
\include{pgfplots.intro}
\include{pgfplots.reference}
\include{pgfplots.libs}
\include{pgfplots.resources}
\include{pgfplots.importexport}
\include{pgfplots.basic.reference}

\printindex

\bibliographystyle{abbrv} %gerapali} %gerabbrv} %gerunsrt.bst} %gerabbrv}% gerplain}
\nocite{pgfplotstable}
\nocite{programmingnotes}
\bibliography{pgfplots}
\end{document}

% -----------------------------------------------------------------------------
% For Stefan Pinnow as reminder on what to look for when editing the manual
% -----------------------------------------------------------------------------
% There should be no line breaks in the following environments
% - |...|
% - \declareandlabel{...}
% - \verbpdfref{...}
% If "MakeTikzPictures" isn't running through check one of the externalized
% LOG files what is the cause of that.
% -----------------------------------------------------------------------------


%%%%%%%%%%%%%%%%%%%%%%%%%%%%%%%%%%%%%%%%%%%%%%%%%%%%%%%%%%%%%%%%%%%%%%%%%%%%%
%
% Package pgfplots.sty documentation.
%
% Copyright 2007/2008 by Christian Feuersaenger.
%
% This program is free software: you can redistribute it and/or modify
% it under the terms of the GNU General Public License as published by
% the Free Software Foundation, either version 3 of the License, or
% (at your option) any later version.
%
% This program is distributed in the hope that it will be useful,
% but WITHOUT ANY WARRANTY; without even the implied warranty of
% MERCHANTABILITY or FITNESS FOR A PARTICULAR PURPOSE.  See the
% GNU General Public License for more details.
%
% You should have received a copy of the GNU General Public License
% along with this program.  If not, see <http://www.gnu.org/licenses/>.
%
%
%%%%%%%%%%%%%%%%%%%%%%%%%%%%%%%%%%%%%%%%%%%%%%%%%%%%%%%%%%%%%%%%%%%%%%%%%%%%%
% SEE pgfplots-macros.tex as well!
%\pdfminorversion=5 % to allow compression
%\pdfobjcompresslevel=2
\documentclass[a4paper,openany]{book}

\let\bookmaketitle=\maketitle

% -----------------------------------
% this here is from ltxdoc:
\usepackage{doc}[=v2]
\AtBeginDocument{\MakeShortVerb{\|}}
%\setlength{\textwidth}{355pt}
%\addtolength\marginparwidth{30pt}
%\addtolength\oddsidemargin{20pt}
%\addtolength\evensidemargin{20pt}
\def\cmd#1{\cs{\expandafter\cmd@to@cs\string#1}}
\def\cmd@to@cs#1#2{\char\number`#2\relax}
\DeclareRobustCommand\cs[1]{\texttt{\char`\\#1}}
\providecommand\marg[1]{%
  {\ttfamily\char`\{}\meta{#1}{\ttfamily\char`\}}}
\providecommand\oarg[1]{%
  {\ttfamily[}\meta{#1}{\ttfamily]}}
\providecommand\parg[1]{%
  {\ttfamily(}\meta{#1}{\ttfamily)}}
\raggedbottom

% -----------------------------------
\input{pgfplots.preamble.tex}

\makeatletter
% I want a two-column index just as in pgfmanual styles. This here
% was the best way to get one:
\def\index@prologue{\section*{Index}\addcontentsline{toc}{chapter}{Index}
}
\makeatother

%\RequirePackage[german,english,francais]{babel}

\def\matlabcolormaptext{This colormap is similar to one shipped with Matlab$^\text{\textregistered}$ under a similar name.}

\IfFileExists{tikzlibraryspy.code.tex}{%
\usetikzlibrary{spy}
}{%
    \message{ERROR: tikz SPY library NOT available. The manual will only compile partially.^^J}%
}%

\usepackage{xparse}% for colorbrewer manual

\usetikzlibrary{
    decorations.markings,
    decorations.footprints,
    shapes.arrows,
    matrix,
    positioning,
}

\usepgfplotslibrary{
    fillbetween,
    ternary,
    smithchart,
    patchplots,
    polar,
    colormaps,
    colorbrewer,
    colortol,           % docu not ready yet
}
\pgfqkeys{/codeexample}{%
    every codeexample/.append style={
        /pgfplots/every ternary axis/.append style={
            /pgfplots/legend style={fill=graphicbackground},
        }
    },
    tabsize=4,
}

\pgfplotsmanualenableexternalizationofexpensive

%\usetikzlibrary{external}
%\tikzexternalize[prefix=figures/]{pgfplots}

\title{%
    Manual for Package \PGFPlots{}\\
    {\small 2D/3D Plots in \LaTeX{}, Version \pgfplotsversion}\\
    {\small\url{https://github.com/pgf-tikz/pgfplots}}
    %\\{\small Attention: you are using an unstable development version.}
}

\makeatletter
\long\def\abstractsmuggle{%
    \centering
    \textbf{Abstract}\\[0.5cm]

    \begin{minipage}{12cm}
        \PGFPlots{} draws high-quality function plots in normal or logarithmic
        scaling with a user-friendly interface directly in \TeX{}. The user
        supplies axis labels, legend entries and the plot coordinates for one or
        more plots and \PGFPlots{} applies axis scaling, computes any logarithms
        and axis ticks and draws the plots. It supports line plots, scatter
        plots, piecewise constant plots, bar plots, area plots, mesh and surface
        plots, patch plots, contour plots, quiver plots, histogram plots, box
        plots, polar axes, ternary diagrams, smith charts and some more. It is
        based on Till Tantau's package \PGF{}/\Tikz{}.
    \end{minipage}

}%

\expandafter\date\expandafter{\@date\\[2cm]
    \abstractsmuggle
}%
\makeatother

%\includeonly{pgfplots.reference}


\begin{document}

\def\plotcoords{%
\addplot coordinates {
(5,8.312e-02)    (17,2.547e-02)   (49,7.407e-03)
(129,2.102e-03)  (321,5.874e-04)  (769,1.623e-04)
(1793,4.442e-05) (4097,1.207e-05) (9217,3.261e-06)
};

\addplot coordinates{
(7,8.472e-02)    (31,3.044e-02)    (111,1.022e-02)
(351,3.303e-03)  (1023,1.039e-03)  (2815,3.196e-04)
(7423,9.658e-05) (18943,2.873e-05) (47103,8.437e-06)
};

\addplot coordinates{
(9,7.881e-02)     (49,3.243e-02)    (209,1.232e-02)
(769,4.454e-03)   (2561,1.551e-03)  (7937,5.236e-04)
(23297,1.723e-04) (65537,5.545e-05) (178177,1.751e-05)
};

\addplot coordinates{
(11,6.887e-02)    (71,3.177e-02)     (351,1.341e-02)
(1471,5.334e-03)  (5503,2.027e-03)   (18943,7.415e-04)
(61183,2.628e-04) (187903,9.063e-05) (553983,3.053e-05)
};

\addplot coordinates{
(13,5.755e-02)     (97,2.925e-02)     (545,1.351e-02)
(2561,5.842e-03)   (10625,2.397e-03)  (40193,9.414e-04)
(141569,3.564e-04) (471041,1.308e-04)
(1496065,4.670e-05)
};
}%


\bookmaketitle
\tableofcontents
\include{pgfplots.title_abstract_intro}
\include{pgfplots.preliminaries}
\include{pgfplots.intro}
\include{pgfplots.reference}
\include{pgfplots.libs}
\include{pgfplots.resources}
\include{pgfplots.importexport}
\include{pgfplots.basic.reference}

\printindex

\bibliographystyle{abbrv} %gerapali} %gerabbrv} %gerunsrt.bst} %gerabbrv}% gerplain}
\nocite{pgfplotstable}
\nocite{programmingnotes}
\bibliography{pgfplots}
\end{document}

% -----------------------------------------------------------------------------
% For Stefan Pinnow as reminder on what to look for when editing the manual
% -----------------------------------------------------------------------------
% There should be no line breaks in the following environments
% - |...|
% - \declareandlabel{...}
% - \verbpdfref{...}
% If "MakeTikzPictures" isn't running through check one of the externalized
% LOG files what is the cause of that.
% -----------------------------------------------------------------------------

\include{pgfplots.basic.reference}

\printindex

\bibliographystyle{abbrv} %gerapali} %gerabbrv} %gerunsrt.bst} %gerabbrv}% gerplain}
\nocite{pgfplotstable}
\nocite{programmingnotes}
\bibliography{pgfplots}
\end{document}

% -----------------------------------------------------------------------------
% For Stefan Pinnow as reminder on what to look for when editing the manual
% -----------------------------------------------------------------------------
% There should be no line breaks in the following environments
% - |...|
% - \declareandlabel{...}
% - \verbpdfref{...}
% If "MakeTikzPictures" isn't running through check one of the externalized
% LOG files what is the cause of that.
% -----------------------------------------------------------------------------


%%%%%%%%%%%%%%%%%%%%%%%%%%%%%%%%%%%%%%%%%%%%%%%%%%%%%%%%%%%%%%%%%%%%%%%%%%%%%
%
% Package pgfplots.sty documentation.
%
% Copyright 2007/2008 by Christian Feuersaenger.
%
% This program is free software: you can redistribute it and/or modify
% it under the terms of the GNU General Public License as published by
% the Free Software Foundation, either version 3 of the License, or
% (at your option) any later version.
%
% This program is distributed in the hope that it will be useful,
% but WITHOUT ANY WARRANTY; without even the implied warranty of
% MERCHANTABILITY or FITNESS FOR A PARTICULAR PURPOSE.  See the
% GNU General Public License for more details.
%
% You should have received a copy of the GNU General Public License
% along with this program.  If not, see <http://www.gnu.org/licenses/>.
%
%
%%%%%%%%%%%%%%%%%%%%%%%%%%%%%%%%%%%%%%%%%%%%%%%%%%%%%%%%%%%%%%%%%%%%%%%%%%%%%
% SEE pgfplots-macros.tex as well!
%\pdfminorversion=5 % to allow compression
%\pdfobjcompresslevel=2
\documentclass[a4paper,openany]{book}

\let\bookmaketitle=\maketitle

% -----------------------------------
% this here is from ltxdoc:
\usepackage{doc}[=v2]
\AtBeginDocument{\MakeShortVerb{\|}}
%\setlength{\textwidth}{355pt}
%\addtolength\marginparwidth{30pt}
%\addtolength\oddsidemargin{20pt}
%\addtolength\evensidemargin{20pt}
\def\cmd#1{\cs{\expandafter\cmd@to@cs\string#1}}
\def\cmd@to@cs#1#2{\char\number`#2\relax}
\DeclareRobustCommand\cs[1]{\texttt{\char`\\#1}}
\providecommand\marg[1]{%
  {\ttfamily\char`\{}\meta{#1}{\ttfamily\char`\}}}
\providecommand\oarg[1]{%
  {\ttfamily[}\meta{#1}{\ttfamily]}}
\providecommand\parg[1]{%
  {\ttfamily(}\meta{#1}{\ttfamily)}}
\raggedbottom

% -----------------------------------
\input{pgfplots.preamble.tex}

\makeatletter
% I want a two-column index just as in pgfmanual styles. This here
% was the best way to get one:
\def\index@prologue{\section*{Index}\addcontentsline{toc}{chapter}{Index}
}
\makeatother

%\RequirePackage[german,english,francais]{babel}

\def\matlabcolormaptext{This colormap is similar to one shipped with Matlab$^\text{\textregistered}$ under a similar name.}

\IfFileExists{tikzlibraryspy.code.tex}{%
\usetikzlibrary{spy}
}{%
    \message{ERROR: tikz SPY library NOT available. The manual will only compile partially.^^J}%
}%

\usepackage{xparse}% for colorbrewer manual

\usetikzlibrary{
    decorations.markings,
    decorations.footprints,
    shapes.arrows,
    matrix,
    positioning,
}

\usepgfplotslibrary{
    fillbetween,
    ternary,
    smithchart,
    patchplots,
    polar,
    colormaps,
    colorbrewer,
    colortol,           % docu not ready yet
}
\pgfqkeys{/codeexample}{%
    every codeexample/.append style={
        /pgfplots/every ternary axis/.append style={
            /pgfplots/legend style={fill=graphicbackground},
        }
    },
    tabsize=4,
}

\pgfplotsmanualenableexternalizationofexpensive

%\usetikzlibrary{external}
%\tikzexternalize[prefix=figures/]{pgfplots}

\title{%
    Manual for Package \PGFPlots{}\\
    {\small 2D/3D Plots in \LaTeX{}, Version \pgfplotsversion}\\
    {\small\url{https://github.com/pgf-tikz/pgfplots}}
    %\\{\small Attention: you are using an unstable development version.}
}

\makeatletter
\long\def\abstractsmuggle{%
    \centering
    \textbf{Abstract}\\[0.5cm]

    \begin{minipage}{12cm}
        \PGFPlots{} draws high-quality function plots in normal or logarithmic
        scaling with a user-friendly interface directly in \TeX{}. The user
        supplies axis labels, legend entries and the plot coordinates for one or
        more plots and \PGFPlots{} applies axis scaling, computes any logarithms
        and axis ticks and draws the plots. It supports line plots, scatter
        plots, piecewise constant plots, bar plots, area plots, mesh and surface
        plots, patch plots, contour plots, quiver plots, histogram plots, box
        plots, polar axes, ternary diagrams, smith charts and some more. It is
        based on Till Tantau's package \PGF{}/\Tikz{}.
    \end{minipage}

}%

\expandafter\date\expandafter{\@date\\[2cm]
    \abstractsmuggle
}%
\makeatother

%\includeonly{pgfplots.reference}


\begin{document}

\def\plotcoords{%
\addplot coordinates {
(5,8.312e-02)    (17,2.547e-02)   (49,7.407e-03)
(129,2.102e-03)  (321,5.874e-04)  (769,1.623e-04)
(1793,4.442e-05) (4097,1.207e-05) (9217,3.261e-06)
};

\addplot coordinates{
(7,8.472e-02)    (31,3.044e-02)    (111,1.022e-02)
(351,3.303e-03)  (1023,1.039e-03)  (2815,3.196e-04)
(7423,9.658e-05) (18943,2.873e-05) (47103,8.437e-06)
};

\addplot coordinates{
(9,7.881e-02)     (49,3.243e-02)    (209,1.232e-02)
(769,4.454e-03)   (2561,1.551e-03)  (7937,5.236e-04)
(23297,1.723e-04) (65537,5.545e-05) (178177,1.751e-05)
};

\addplot coordinates{
(11,6.887e-02)    (71,3.177e-02)     (351,1.341e-02)
(1471,5.334e-03)  (5503,2.027e-03)   (18943,7.415e-04)
(61183,2.628e-04) (187903,9.063e-05) (553983,3.053e-05)
};

\addplot coordinates{
(13,5.755e-02)     (97,2.925e-02)     (545,1.351e-02)
(2561,5.842e-03)   (10625,2.397e-03)  (40193,9.414e-04)
(141569,3.564e-04) (471041,1.308e-04)
(1496065,4.670e-05)
};
}%


\bookmaketitle
\tableofcontents

%%%%%%%%%%%%%%%%%%%%%%%%%%%%%%%%%%%%%%%%%%%%%%%%%%%%%%%%%%%%%%%%%%%%%%%%%%%%%
%
% Package pgfplots.sty documentation.
%
% Copyright 2007/2008 by Christian Feuersaenger.
%
% This program is free software: you can redistribute it and/or modify
% it under the terms of the GNU General Public License as published by
% the Free Software Foundation, either version 3 of the License, or
% (at your option) any later version.
%
% This program is distributed in the hope that it will be useful,
% but WITHOUT ANY WARRANTY; without even the implied warranty of
% MERCHANTABILITY or FITNESS FOR A PARTICULAR PURPOSE.  See the
% GNU General Public License for more details.
%
% You should have received a copy of the GNU General Public License
% along with this program.  If not, see <http://www.gnu.org/licenses/>.
%
%
%%%%%%%%%%%%%%%%%%%%%%%%%%%%%%%%%%%%%%%%%%%%%%%%%%%%%%%%%%%%%%%%%%%%%%%%%%%%%
% SEE pgfplots-macros.tex as well!
%\pdfminorversion=5 % to allow compression
%\pdfobjcompresslevel=2
\documentclass[a4paper,openany]{book}

\let\bookmaketitle=\maketitle

% -----------------------------------
% this here is from ltxdoc:
\usepackage{doc}[=v2]
\AtBeginDocument{\MakeShortVerb{\|}}
%\setlength{\textwidth}{355pt}
%\addtolength\marginparwidth{30pt}
%\addtolength\oddsidemargin{20pt}
%\addtolength\evensidemargin{20pt}
\def\cmd#1{\cs{\expandafter\cmd@to@cs\string#1}}
\def\cmd@to@cs#1#2{\char\number`#2\relax}
\DeclareRobustCommand\cs[1]{\texttt{\char`\\#1}}
\providecommand\marg[1]{%
  {\ttfamily\char`\{}\meta{#1}{\ttfamily\char`\}}}
\providecommand\oarg[1]{%
  {\ttfamily[}\meta{#1}{\ttfamily]}}
\providecommand\parg[1]{%
  {\ttfamily(}\meta{#1}{\ttfamily)}}
\raggedbottom

% -----------------------------------
\input{pgfplots.preamble.tex}

\makeatletter
% I want a two-column index just as in pgfmanual styles. This here
% was the best way to get one:
\def\index@prologue{\section*{Index}\addcontentsline{toc}{chapter}{Index}
}
\makeatother

%\RequirePackage[german,english,francais]{babel}

\def\matlabcolormaptext{This colormap is similar to one shipped with Matlab$^\text{\textregistered}$ under a similar name.}

\IfFileExists{tikzlibraryspy.code.tex}{%
\usetikzlibrary{spy}
}{%
    \message{ERROR: tikz SPY library NOT available. The manual will only compile partially.^^J}%
}%

\usepackage{xparse}% for colorbrewer manual

\usetikzlibrary{
    decorations.markings,
    decorations.footprints,
    shapes.arrows,
    matrix,
    positioning,
}

\usepgfplotslibrary{
    fillbetween,
    ternary,
    smithchart,
    patchplots,
    polar,
    colormaps,
    colorbrewer,
    colortol,           % docu not ready yet
}
\pgfqkeys{/codeexample}{%
    every codeexample/.append style={
        /pgfplots/every ternary axis/.append style={
            /pgfplots/legend style={fill=graphicbackground},
        }
    },
    tabsize=4,
}

\pgfplotsmanualenableexternalizationofexpensive

%\usetikzlibrary{external}
%\tikzexternalize[prefix=figures/]{pgfplots}

\title{%
    Manual for Package \PGFPlots{}\\
    {\small 2D/3D Plots in \LaTeX{}, Version \pgfplotsversion}\\
    {\small\url{https://github.com/pgf-tikz/pgfplots}}
    %\\{\small Attention: you are using an unstable development version.}
}

\makeatletter
\long\def\abstractsmuggle{%
    \centering
    \textbf{Abstract}\\[0.5cm]

    \begin{minipage}{12cm}
        \PGFPlots{} draws high-quality function plots in normal or logarithmic
        scaling with a user-friendly interface directly in \TeX{}. The user
        supplies axis labels, legend entries and the plot coordinates for one or
        more plots and \PGFPlots{} applies axis scaling, computes any logarithms
        and axis ticks and draws the plots. It supports line plots, scatter
        plots, piecewise constant plots, bar plots, area plots, mesh and surface
        plots, patch plots, contour plots, quiver plots, histogram plots, box
        plots, polar axes, ternary diagrams, smith charts and some more. It is
        based on Till Tantau's package \PGF{}/\Tikz{}.
    \end{minipage}

}%

\expandafter\date\expandafter{\@date\\[2cm]
    \abstractsmuggle
}%
\makeatother

%\includeonly{pgfplots.reference}


\begin{document}

\def\plotcoords{%
\addplot coordinates {
(5,8.312e-02)    (17,2.547e-02)   (49,7.407e-03)
(129,2.102e-03)  (321,5.874e-04)  (769,1.623e-04)
(1793,4.442e-05) (4097,1.207e-05) (9217,3.261e-06)
};

\addplot coordinates{
(7,8.472e-02)    (31,3.044e-02)    (111,1.022e-02)
(351,3.303e-03)  (1023,1.039e-03)  (2815,3.196e-04)
(7423,9.658e-05) (18943,2.873e-05) (47103,8.437e-06)
};

\addplot coordinates{
(9,7.881e-02)     (49,3.243e-02)    (209,1.232e-02)
(769,4.454e-03)   (2561,1.551e-03)  (7937,5.236e-04)
(23297,1.723e-04) (65537,5.545e-05) (178177,1.751e-05)
};

\addplot coordinates{
(11,6.887e-02)    (71,3.177e-02)     (351,1.341e-02)
(1471,5.334e-03)  (5503,2.027e-03)   (18943,7.415e-04)
(61183,2.628e-04) (187903,9.063e-05) (553983,3.053e-05)
};

\addplot coordinates{
(13,5.755e-02)     (97,2.925e-02)     (545,1.351e-02)
(2561,5.842e-03)   (10625,2.397e-03)  (40193,9.414e-04)
(141569,3.564e-04) (471041,1.308e-04)
(1496065,4.670e-05)
};
}%


\bookmaketitle
\tableofcontents
\include{pgfplots.title_abstract_intro}
\include{pgfplots.preliminaries}
\include{pgfplots.intro}
\include{pgfplots.reference}
\include{pgfplots.libs}
\include{pgfplots.resources}
\include{pgfplots.importexport}
\include{pgfplots.basic.reference}

\printindex

\bibliographystyle{abbrv} %gerapali} %gerabbrv} %gerunsrt.bst} %gerabbrv}% gerplain}
\nocite{pgfplotstable}
\nocite{programmingnotes}
\bibliography{pgfplots}
\end{document}

% -----------------------------------------------------------------------------
% For Stefan Pinnow as reminder on what to look for when editing the manual
% -----------------------------------------------------------------------------
% There should be no line breaks in the following environments
% - |...|
% - \declareandlabel{...}
% - \verbpdfref{...}
% If "MakeTikzPictures" isn't running through check one of the externalized
% LOG files what is the cause of that.
% -----------------------------------------------------------------------------


%%%%%%%%%%%%%%%%%%%%%%%%%%%%%%%%%%%%%%%%%%%%%%%%%%%%%%%%%%%%%%%%%%%%%%%%%%%%%
%
% Package pgfplots.sty documentation.
%
% Copyright 2007/2008 by Christian Feuersaenger.
%
% This program is free software: you can redistribute it and/or modify
% it under the terms of the GNU General Public License as published by
% the Free Software Foundation, either version 3 of the License, or
% (at your option) any later version.
%
% This program is distributed in the hope that it will be useful,
% but WITHOUT ANY WARRANTY; without even the implied warranty of
% MERCHANTABILITY or FITNESS FOR A PARTICULAR PURPOSE.  See the
% GNU General Public License for more details.
%
% You should have received a copy of the GNU General Public License
% along with this program.  If not, see <http://www.gnu.org/licenses/>.
%
%
%%%%%%%%%%%%%%%%%%%%%%%%%%%%%%%%%%%%%%%%%%%%%%%%%%%%%%%%%%%%%%%%%%%%%%%%%%%%%
% SEE pgfplots-macros.tex as well!
%\pdfminorversion=5 % to allow compression
%\pdfobjcompresslevel=2
\documentclass[a4paper,openany]{book}

\let\bookmaketitle=\maketitle

% -----------------------------------
% this here is from ltxdoc:
\usepackage{doc}[=v2]
\AtBeginDocument{\MakeShortVerb{\|}}
%\setlength{\textwidth}{355pt}
%\addtolength\marginparwidth{30pt}
%\addtolength\oddsidemargin{20pt}
%\addtolength\evensidemargin{20pt}
\def\cmd#1{\cs{\expandafter\cmd@to@cs\string#1}}
\def\cmd@to@cs#1#2{\char\number`#2\relax}
\DeclareRobustCommand\cs[1]{\texttt{\char`\\#1}}
\providecommand\marg[1]{%
  {\ttfamily\char`\{}\meta{#1}{\ttfamily\char`\}}}
\providecommand\oarg[1]{%
  {\ttfamily[}\meta{#1}{\ttfamily]}}
\providecommand\parg[1]{%
  {\ttfamily(}\meta{#1}{\ttfamily)}}
\raggedbottom

% -----------------------------------
\input{pgfplots.preamble.tex}

\makeatletter
% I want a two-column index just as in pgfmanual styles. This here
% was the best way to get one:
\def\index@prologue{\section*{Index}\addcontentsline{toc}{chapter}{Index}
}
\makeatother

%\RequirePackage[german,english,francais]{babel}

\def\matlabcolormaptext{This colormap is similar to one shipped with Matlab$^\text{\textregistered}$ under a similar name.}

\IfFileExists{tikzlibraryspy.code.tex}{%
\usetikzlibrary{spy}
}{%
    \message{ERROR: tikz SPY library NOT available. The manual will only compile partially.^^J}%
}%

\usepackage{xparse}% for colorbrewer manual

\usetikzlibrary{
    decorations.markings,
    decorations.footprints,
    shapes.arrows,
    matrix,
    positioning,
}

\usepgfplotslibrary{
    fillbetween,
    ternary,
    smithchart,
    patchplots,
    polar,
    colormaps,
    colorbrewer,
    colortol,           % docu not ready yet
}
\pgfqkeys{/codeexample}{%
    every codeexample/.append style={
        /pgfplots/every ternary axis/.append style={
            /pgfplots/legend style={fill=graphicbackground},
        }
    },
    tabsize=4,
}

\pgfplotsmanualenableexternalizationofexpensive

%\usetikzlibrary{external}
%\tikzexternalize[prefix=figures/]{pgfplots}

\title{%
    Manual for Package \PGFPlots{}\\
    {\small 2D/3D Plots in \LaTeX{}, Version \pgfplotsversion}\\
    {\small\url{https://github.com/pgf-tikz/pgfplots}}
    %\\{\small Attention: you are using an unstable development version.}
}

\makeatletter
\long\def\abstractsmuggle{%
    \centering
    \textbf{Abstract}\\[0.5cm]

    \begin{minipage}{12cm}
        \PGFPlots{} draws high-quality function plots in normal or logarithmic
        scaling with a user-friendly interface directly in \TeX{}. The user
        supplies axis labels, legend entries and the plot coordinates for one or
        more plots and \PGFPlots{} applies axis scaling, computes any logarithms
        and axis ticks and draws the plots. It supports line plots, scatter
        plots, piecewise constant plots, bar plots, area plots, mesh and surface
        plots, patch plots, contour plots, quiver plots, histogram plots, box
        plots, polar axes, ternary diagrams, smith charts and some more. It is
        based on Till Tantau's package \PGF{}/\Tikz{}.
    \end{minipage}

}%

\expandafter\date\expandafter{\@date\\[2cm]
    \abstractsmuggle
}%
\makeatother

%\includeonly{pgfplots.reference}


\begin{document}

\def\plotcoords{%
\addplot coordinates {
(5,8.312e-02)    (17,2.547e-02)   (49,7.407e-03)
(129,2.102e-03)  (321,5.874e-04)  (769,1.623e-04)
(1793,4.442e-05) (4097,1.207e-05) (9217,3.261e-06)
};

\addplot coordinates{
(7,8.472e-02)    (31,3.044e-02)    (111,1.022e-02)
(351,3.303e-03)  (1023,1.039e-03)  (2815,3.196e-04)
(7423,9.658e-05) (18943,2.873e-05) (47103,8.437e-06)
};

\addplot coordinates{
(9,7.881e-02)     (49,3.243e-02)    (209,1.232e-02)
(769,4.454e-03)   (2561,1.551e-03)  (7937,5.236e-04)
(23297,1.723e-04) (65537,5.545e-05) (178177,1.751e-05)
};

\addplot coordinates{
(11,6.887e-02)    (71,3.177e-02)     (351,1.341e-02)
(1471,5.334e-03)  (5503,2.027e-03)   (18943,7.415e-04)
(61183,2.628e-04) (187903,9.063e-05) (553983,3.053e-05)
};

\addplot coordinates{
(13,5.755e-02)     (97,2.925e-02)     (545,1.351e-02)
(2561,5.842e-03)   (10625,2.397e-03)  (40193,9.414e-04)
(141569,3.564e-04) (471041,1.308e-04)
(1496065,4.670e-05)
};
}%


\bookmaketitle
\tableofcontents
\include{pgfplots.title_abstract_intro}
\include{pgfplots.preliminaries}
\include{pgfplots.intro}
\include{pgfplots.reference}
\include{pgfplots.libs}
\include{pgfplots.resources}
\include{pgfplots.importexport}
\include{pgfplots.basic.reference}

\printindex

\bibliographystyle{abbrv} %gerapali} %gerabbrv} %gerunsrt.bst} %gerabbrv}% gerplain}
\nocite{pgfplotstable}
\nocite{programmingnotes}
\bibliography{pgfplots}
\end{document}

% -----------------------------------------------------------------------------
% For Stefan Pinnow as reminder on what to look for when editing the manual
% -----------------------------------------------------------------------------
% There should be no line breaks in the following environments
% - |...|
% - \declareandlabel{...}
% - \verbpdfref{...}
% If "MakeTikzPictures" isn't running through check one of the externalized
% LOG files what is the cause of that.
% -----------------------------------------------------------------------------


%%%%%%%%%%%%%%%%%%%%%%%%%%%%%%%%%%%%%%%%%%%%%%%%%%%%%%%%%%%%%%%%%%%%%%%%%%%%%
%
% Package pgfplots.sty documentation.
%
% Copyright 2007/2008 by Christian Feuersaenger.
%
% This program is free software: you can redistribute it and/or modify
% it under the terms of the GNU General Public License as published by
% the Free Software Foundation, either version 3 of the License, or
% (at your option) any later version.
%
% This program is distributed in the hope that it will be useful,
% but WITHOUT ANY WARRANTY; without even the implied warranty of
% MERCHANTABILITY or FITNESS FOR A PARTICULAR PURPOSE.  See the
% GNU General Public License for more details.
%
% You should have received a copy of the GNU General Public License
% along with this program.  If not, see <http://www.gnu.org/licenses/>.
%
%
%%%%%%%%%%%%%%%%%%%%%%%%%%%%%%%%%%%%%%%%%%%%%%%%%%%%%%%%%%%%%%%%%%%%%%%%%%%%%
% SEE pgfplots-macros.tex as well!
%\pdfminorversion=5 % to allow compression
%\pdfobjcompresslevel=2
\documentclass[a4paper,openany]{book}

\let\bookmaketitle=\maketitle

% -----------------------------------
% this here is from ltxdoc:
\usepackage{doc}[=v2]
\AtBeginDocument{\MakeShortVerb{\|}}
%\setlength{\textwidth}{355pt}
%\addtolength\marginparwidth{30pt}
%\addtolength\oddsidemargin{20pt}
%\addtolength\evensidemargin{20pt}
\def\cmd#1{\cs{\expandafter\cmd@to@cs\string#1}}
\def\cmd@to@cs#1#2{\char\number`#2\relax}
\DeclareRobustCommand\cs[1]{\texttt{\char`\\#1}}
\providecommand\marg[1]{%
  {\ttfamily\char`\{}\meta{#1}{\ttfamily\char`\}}}
\providecommand\oarg[1]{%
  {\ttfamily[}\meta{#1}{\ttfamily]}}
\providecommand\parg[1]{%
  {\ttfamily(}\meta{#1}{\ttfamily)}}
\raggedbottom

% -----------------------------------
\input{pgfplots.preamble.tex}

\makeatletter
% I want a two-column index just as in pgfmanual styles. This here
% was the best way to get one:
\def\index@prologue{\section*{Index}\addcontentsline{toc}{chapter}{Index}
}
\makeatother

%\RequirePackage[german,english,francais]{babel}

\def\matlabcolormaptext{This colormap is similar to one shipped with Matlab$^\text{\textregistered}$ under a similar name.}

\IfFileExists{tikzlibraryspy.code.tex}{%
\usetikzlibrary{spy}
}{%
    \message{ERROR: tikz SPY library NOT available. The manual will only compile partially.^^J}%
}%

\usepackage{xparse}% for colorbrewer manual

\usetikzlibrary{
    decorations.markings,
    decorations.footprints,
    shapes.arrows,
    matrix,
    positioning,
}

\usepgfplotslibrary{
    fillbetween,
    ternary,
    smithchart,
    patchplots,
    polar,
    colormaps,
    colorbrewer,
    colortol,           % docu not ready yet
}
\pgfqkeys{/codeexample}{%
    every codeexample/.append style={
        /pgfplots/every ternary axis/.append style={
            /pgfplots/legend style={fill=graphicbackground},
        }
    },
    tabsize=4,
}

\pgfplotsmanualenableexternalizationofexpensive

%\usetikzlibrary{external}
%\tikzexternalize[prefix=figures/]{pgfplots}

\title{%
    Manual for Package \PGFPlots{}\\
    {\small 2D/3D Plots in \LaTeX{}, Version \pgfplotsversion}\\
    {\small\url{https://github.com/pgf-tikz/pgfplots}}
    %\\{\small Attention: you are using an unstable development version.}
}

\makeatletter
\long\def\abstractsmuggle{%
    \centering
    \textbf{Abstract}\\[0.5cm]

    \begin{minipage}{12cm}
        \PGFPlots{} draws high-quality function plots in normal or logarithmic
        scaling with a user-friendly interface directly in \TeX{}. The user
        supplies axis labels, legend entries and the plot coordinates for one or
        more plots and \PGFPlots{} applies axis scaling, computes any logarithms
        and axis ticks and draws the plots. It supports line plots, scatter
        plots, piecewise constant plots, bar plots, area plots, mesh and surface
        plots, patch plots, contour plots, quiver plots, histogram plots, box
        plots, polar axes, ternary diagrams, smith charts and some more. It is
        based on Till Tantau's package \PGF{}/\Tikz{}.
    \end{minipage}

}%

\expandafter\date\expandafter{\@date\\[2cm]
    \abstractsmuggle
}%
\makeatother

%\includeonly{pgfplots.reference}


\begin{document}

\def\plotcoords{%
\addplot coordinates {
(5,8.312e-02)    (17,2.547e-02)   (49,7.407e-03)
(129,2.102e-03)  (321,5.874e-04)  (769,1.623e-04)
(1793,4.442e-05) (4097,1.207e-05) (9217,3.261e-06)
};

\addplot coordinates{
(7,8.472e-02)    (31,3.044e-02)    (111,1.022e-02)
(351,3.303e-03)  (1023,1.039e-03)  (2815,3.196e-04)
(7423,9.658e-05) (18943,2.873e-05) (47103,8.437e-06)
};

\addplot coordinates{
(9,7.881e-02)     (49,3.243e-02)    (209,1.232e-02)
(769,4.454e-03)   (2561,1.551e-03)  (7937,5.236e-04)
(23297,1.723e-04) (65537,5.545e-05) (178177,1.751e-05)
};

\addplot coordinates{
(11,6.887e-02)    (71,3.177e-02)     (351,1.341e-02)
(1471,5.334e-03)  (5503,2.027e-03)   (18943,7.415e-04)
(61183,2.628e-04) (187903,9.063e-05) (553983,3.053e-05)
};

\addplot coordinates{
(13,5.755e-02)     (97,2.925e-02)     (545,1.351e-02)
(2561,5.842e-03)   (10625,2.397e-03)  (40193,9.414e-04)
(141569,3.564e-04) (471041,1.308e-04)
(1496065,4.670e-05)
};
}%


\bookmaketitle
\tableofcontents
\include{pgfplots.title_abstract_intro}
\include{pgfplots.preliminaries}
\include{pgfplots.intro}
\include{pgfplots.reference}
\include{pgfplots.libs}
\include{pgfplots.resources}
\include{pgfplots.importexport}
\include{pgfplots.basic.reference}

\printindex

\bibliographystyle{abbrv} %gerapali} %gerabbrv} %gerunsrt.bst} %gerabbrv}% gerplain}
\nocite{pgfplotstable}
\nocite{programmingnotes}
\bibliography{pgfplots}
\end{document}

% -----------------------------------------------------------------------------
% For Stefan Pinnow as reminder on what to look for when editing the manual
% -----------------------------------------------------------------------------
% There should be no line breaks in the following environments
% - |...|
% - \declareandlabel{...}
% - \verbpdfref{...}
% If "MakeTikzPictures" isn't running through check one of the externalized
% LOG files what is the cause of that.
% -----------------------------------------------------------------------------


%%%%%%%%%%%%%%%%%%%%%%%%%%%%%%%%%%%%%%%%%%%%%%%%%%%%%%%%%%%%%%%%%%%%%%%%%%%%%
%
% Package pgfplots.sty documentation.
%
% Copyright 2007/2008 by Christian Feuersaenger.
%
% This program is free software: you can redistribute it and/or modify
% it under the terms of the GNU General Public License as published by
% the Free Software Foundation, either version 3 of the License, or
% (at your option) any later version.
%
% This program is distributed in the hope that it will be useful,
% but WITHOUT ANY WARRANTY; without even the implied warranty of
% MERCHANTABILITY or FITNESS FOR A PARTICULAR PURPOSE.  See the
% GNU General Public License for more details.
%
% You should have received a copy of the GNU General Public License
% along with this program.  If not, see <http://www.gnu.org/licenses/>.
%
%
%%%%%%%%%%%%%%%%%%%%%%%%%%%%%%%%%%%%%%%%%%%%%%%%%%%%%%%%%%%%%%%%%%%%%%%%%%%%%
% SEE pgfplots-macros.tex as well!
%\pdfminorversion=5 % to allow compression
%\pdfobjcompresslevel=2
\documentclass[a4paper,openany]{book}

\let\bookmaketitle=\maketitle

% -----------------------------------
% this here is from ltxdoc:
\usepackage{doc}[=v2]
\AtBeginDocument{\MakeShortVerb{\|}}
%\setlength{\textwidth}{355pt}
%\addtolength\marginparwidth{30pt}
%\addtolength\oddsidemargin{20pt}
%\addtolength\evensidemargin{20pt}
\def\cmd#1{\cs{\expandafter\cmd@to@cs\string#1}}
\def\cmd@to@cs#1#2{\char\number`#2\relax}
\DeclareRobustCommand\cs[1]{\texttt{\char`\\#1}}
\providecommand\marg[1]{%
  {\ttfamily\char`\{}\meta{#1}{\ttfamily\char`\}}}
\providecommand\oarg[1]{%
  {\ttfamily[}\meta{#1}{\ttfamily]}}
\providecommand\parg[1]{%
  {\ttfamily(}\meta{#1}{\ttfamily)}}
\raggedbottom

% -----------------------------------
\input{pgfplots.preamble.tex}

\makeatletter
% I want a two-column index just as in pgfmanual styles. This here
% was the best way to get one:
\def\index@prologue{\section*{Index}\addcontentsline{toc}{chapter}{Index}
}
\makeatother

%\RequirePackage[german,english,francais]{babel}

\def\matlabcolormaptext{This colormap is similar to one shipped with Matlab$^\text{\textregistered}$ under a similar name.}

\IfFileExists{tikzlibraryspy.code.tex}{%
\usetikzlibrary{spy}
}{%
    \message{ERROR: tikz SPY library NOT available. The manual will only compile partially.^^J}%
}%

\usepackage{xparse}% for colorbrewer manual

\usetikzlibrary{
    decorations.markings,
    decorations.footprints,
    shapes.arrows,
    matrix,
    positioning,
}

\usepgfplotslibrary{
    fillbetween,
    ternary,
    smithchart,
    patchplots,
    polar,
    colormaps,
    colorbrewer,
    colortol,           % docu not ready yet
}
\pgfqkeys{/codeexample}{%
    every codeexample/.append style={
        /pgfplots/every ternary axis/.append style={
            /pgfplots/legend style={fill=graphicbackground},
        }
    },
    tabsize=4,
}

\pgfplotsmanualenableexternalizationofexpensive

%\usetikzlibrary{external}
%\tikzexternalize[prefix=figures/]{pgfplots}

\title{%
    Manual for Package \PGFPlots{}\\
    {\small 2D/3D Plots in \LaTeX{}, Version \pgfplotsversion}\\
    {\small\url{https://github.com/pgf-tikz/pgfplots}}
    %\\{\small Attention: you are using an unstable development version.}
}

\makeatletter
\long\def\abstractsmuggle{%
    \centering
    \textbf{Abstract}\\[0.5cm]

    \begin{minipage}{12cm}
        \PGFPlots{} draws high-quality function plots in normal or logarithmic
        scaling with a user-friendly interface directly in \TeX{}. The user
        supplies axis labels, legend entries and the plot coordinates for one or
        more plots and \PGFPlots{} applies axis scaling, computes any logarithms
        and axis ticks and draws the plots. It supports line plots, scatter
        plots, piecewise constant plots, bar plots, area plots, mesh and surface
        plots, patch plots, contour plots, quiver plots, histogram plots, box
        plots, polar axes, ternary diagrams, smith charts and some more. It is
        based on Till Tantau's package \PGF{}/\Tikz{}.
    \end{minipage}

}%

\expandafter\date\expandafter{\@date\\[2cm]
    \abstractsmuggle
}%
\makeatother

%\includeonly{pgfplots.reference}


\begin{document}

\def\plotcoords{%
\addplot coordinates {
(5,8.312e-02)    (17,2.547e-02)   (49,7.407e-03)
(129,2.102e-03)  (321,5.874e-04)  (769,1.623e-04)
(1793,4.442e-05) (4097,1.207e-05) (9217,3.261e-06)
};

\addplot coordinates{
(7,8.472e-02)    (31,3.044e-02)    (111,1.022e-02)
(351,3.303e-03)  (1023,1.039e-03)  (2815,3.196e-04)
(7423,9.658e-05) (18943,2.873e-05) (47103,8.437e-06)
};

\addplot coordinates{
(9,7.881e-02)     (49,3.243e-02)    (209,1.232e-02)
(769,4.454e-03)   (2561,1.551e-03)  (7937,5.236e-04)
(23297,1.723e-04) (65537,5.545e-05) (178177,1.751e-05)
};

\addplot coordinates{
(11,6.887e-02)    (71,3.177e-02)     (351,1.341e-02)
(1471,5.334e-03)  (5503,2.027e-03)   (18943,7.415e-04)
(61183,2.628e-04) (187903,9.063e-05) (553983,3.053e-05)
};

\addplot coordinates{
(13,5.755e-02)     (97,2.925e-02)     (545,1.351e-02)
(2561,5.842e-03)   (10625,2.397e-03)  (40193,9.414e-04)
(141569,3.564e-04) (471041,1.308e-04)
(1496065,4.670e-05)
};
}%


\bookmaketitle
\tableofcontents
\include{pgfplots.title_abstract_intro}
\include{pgfplots.preliminaries}
\include{pgfplots.intro}
\include{pgfplots.reference}
\include{pgfplots.libs}
\include{pgfplots.resources}
\include{pgfplots.importexport}
\include{pgfplots.basic.reference}

\printindex

\bibliographystyle{abbrv} %gerapali} %gerabbrv} %gerunsrt.bst} %gerabbrv}% gerplain}
\nocite{pgfplotstable}
\nocite{programmingnotes}
\bibliography{pgfplots}
\end{document}

% -----------------------------------------------------------------------------
% For Stefan Pinnow as reminder on what to look for when editing the manual
% -----------------------------------------------------------------------------
% There should be no line breaks in the following environments
% - |...|
% - \declareandlabel{...}
% - \verbpdfref{...}
% If "MakeTikzPictures" isn't running through check one of the externalized
% LOG files what is the cause of that.
% -----------------------------------------------------------------------------


%%%%%%%%%%%%%%%%%%%%%%%%%%%%%%%%%%%%%%%%%%%%%%%%%%%%%%%%%%%%%%%%%%%%%%%%%%%%%
%
% Package pgfplots.sty documentation.
%
% Copyright 2007/2008 by Christian Feuersaenger.
%
% This program is free software: you can redistribute it and/or modify
% it under the terms of the GNU General Public License as published by
% the Free Software Foundation, either version 3 of the License, or
% (at your option) any later version.
%
% This program is distributed in the hope that it will be useful,
% but WITHOUT ANY WARRANTY; without even the implied warranty of
% MERCHANTABILITY or FITNESS FOR A PARTICULAR PURPOSE.  See the
% GNU General Public License for more details.
%
% You should have received a copy of the GNU General Public License
% along with this program.  If not, see <http://www.gnu.org/licenses/>.
%
%
%%%%%%%%%%%%%%%%%%%%%%%%%%%%%%%%%%%%%%%%%%%%%%%%%%%%%%%%%%%%%%%%%%%%%%%%%%%%%
% SEE pgfplots-macros.tex as well!
%\pdfminorversion=5 % to allow compression
%\pdfobjcompresslevel=2
\documentclass[a4paper,openany]{book}

\let\bookmaketitle=\maketitle

% -----------------------------------
% this here is from ltxdoc:
\usepackage{doc}[=v2]
\AtBeginDocument{\MakeShortVerb{\|}}
%\setlength{\textwidth}{355pt}
%\addtolength\marginparwidth{30pt}
%\addtolength\oddsidemargin{20pt}
%\addtolength\evensidemargin{20pt}
\def\cmd#1{\cs{\expandafter\cmd@to@cs\string#1}}
\def\cmd@to@cs#1#2{\char\number`#2\relax}
\DeclareRobustCommand\cs[1]{\texttt{\char`\\#1}}
\providecommand\marg[1]{%
  {\ttfamily\char`\{}\meta{#1}{\ttfamily\char`\}}}
\providecommand\oarg[1]{%
  {\ttfamily[}\meta{#1}{\ttfamily]}}
\providecommand\parg[1]{%
  {\ttfamily(}\meta{#1}{\ttfamily)}}
\raggedbottom

% -----------------------------------
\input{pgfplots.preamble.tex}

\makeatletter
% I want a two-column index just as in pgfmanual styles. This here
% was the best way to get one:
\def\index@prologue{\section*{Index}\addcontentsline{toc}{chapter}{Index}
}
\makeatother

%\RequirePackage[german,english,francais]{babel}

\def\matlabcolormaptext{This colormap is similar to one shipped with Matlab$^\text{\textregistered}$ under a similar name.}

\IfFileExists{tikzlibraryspy.code.tex}{%
\usetikzlibrary{spy}
}{%
    \message{ERROR: tikz SPY library NOT available. The manual will only compile partially.^^J}%
}%

\usepackage{xparse}% for colorbrewer manual

\usetikzlibrary{
    decorations.markings,
    decorations.footprints,
    shapes.arrows,
    matrix,
    positioning,
}

\usepgfplotslibrary{
    fillbetween,
    ternary,
    smithchart,
    patchplots,
    polar,
    colormaps,
    colorbrewer,
    colortol,           % docu not ready yet
}
\pgfqkeys{/codeexample}{%
    every codeexample/.append style={
        /pgfplots/every ternary axis/.append style={
            /pgfplots/legend style={fill=graphicbackground},
        }
    },
    tabsize=4,
}

\pgfplotsmanualenableexternalizationofexpensive

%\usetikzlibrary{external}
%\tikzexternalize[prefix=figures/]{pgfplots}

\title{%
    Manual for Package \PGFPlots{}\\
    {\small 2D/3D Plots in \LaTeX{}, Version \pgfplotsversion}\\
    {\small\url{https://github.com/pgf-tikz/pgfplots}}
    %\\{\small Attention: you are using an unstable development version.}
}

\makeatletter
\long\def\abstractsmuggle{%
    \centering
    \textbf{Abstract}\\[0.5cm]

    \begin{minipage}{12cm}
        \PGFPlots{} draws high-quality function plots in normal or logarithmic
        scaling with a user-friendly interface directly in \TeX{}. The user
        supplies axis labels, legend entries and the plot coordinates for one or
        more plots and \PGFPlots{} applies axis scaling, computes any logarithms
        and axis ticks and draws the plots. It supports line plots, scatter
        plots, piecewise constant plots, bar plots, area plots, mesh and surface
        plots, patch plots, contour plots, quiver plots, histogram plots, box
        plots, polar axes, ternary diagrams, smith charts and some more. It is
        based on Till Tantau's package \PGF{}/\Tikz{}.
    \end{minipage}

}%

\expandafter\date\expandafter{\@date\\[2cm]
    \abstractsmuggle
}%
\makeatother

%\includeonly{pgfplots.reference}


\begin{document}

\def\plotcoords{%
\addplot coordinates {
(5,8.312e-02)    (17,2.547e-02)   (49,7.407e-03)
(129,2.102e-03)  (321,5.874e-04)  (769,1.623e-04)
(1793,4.442e-05) (4097,1.207e-05) (9217,3.261e-06)
};

\addplot coordinates{
(7,8.472e-02)    (31,3.044e-02)    (111,1.022e-02)
(351,3.303e-03)  (1023,1.039e-03)  (2815,3.196e-04)
(7423,9.658e-05) (18943,2.873e-05) (47103,8.437e-06)
};

\addplot coordinates{
(9,7.881e-02)     (49,3.243e-02)    (209,1.232e-02)
(769,4.454e-03)   (2561,1.551e-03)  (7937,5.236e-04)
(23297,1.723e-04) (65537,5.545e-05) (178177,1.751e-05)
};

\addplot coordinates{
(11,6.887e-02)    (71,3.177e-02)     (351,1.341e-02)
(1471,5.334e-03)  (5503,2.027e-03)   (18943,7.415e-04)
(61183,2.628e-04) (187903,9.063e-05) (553983,3.053e-05)
};

\addplot coordinates{
(13,5.755e-02)     (97,2.925e-02)     (545,1.351e-02)
(2561,5.842e-03)   (10625,2.397e-03)  (40193,9.414e-04)
(141569,3.564e-04) (471041,1.308e-04)
(1496065,4.670e-05)
};
}%


\bookmaketitle
\tableofcontents
\include{pgfplots.title_abstract_intro}
\include{pgfplots.preliminaries}
\include{pgfplots.intro}
\include{pgfplots.reference}
\include{pgfplots.libs}
\include{pgfplots.resources}
\include{pgfplots.importexport}
\include{pgfplots.basic.reference}

\printindex

\bibliographystyle{abbrv} %gerapali} %gerabbrv} %gerunsrt.bst} %gerabbrv}% gerplain}
\nocite{pgfplotstable}
\nocite{programmingnotes}
\bibliography{pgfplots}
\end{document}

% -----------------------------------------------------------------------------
% For Stefan Pinnow as reminder on what to look for when editing the manual
% -----------------------------------------------------------------------------
% There should be no line breaks in the following environments
% - |...|
% - \declareandlabel{...}
% - \verbpdfref{...}
% If "MakeTikzPictures" isn't running through check one of the externalized
% LOG files what is the cause of that.
% -----------------------------------------------------------------------------


%%%%%%%%%%%%%%%%%%%%%%%%%%%%%%%%%%%%%%%%%%%%%%%%%%%%%%%%%%%%%%%%%%%%%%%%%%%%%
%
% Package pgfplots.sty documentation.
%
% Copyright 2007/2008 by Christian Feuersaenger.
%
% This program is free software: you can redistribute it and/or modify
% it under the terms of the GNU General Public License as published by
% the Free Software Foundation, either version 3 of the License, or
% (at your option) any later version.
%
% This program is distributed in the hope that it will be useful,
% but WITHOUT ANY WARRANTY; without even the implied warranty of
% MERCHANTABILITY or FITNESS FOR A PARTICULAR PURPOSE.  See the
% GNU General Public License for more details.
%
% You should have received a copy of the GNU General Public License
% along with this program.  If not, see <http://www.gnu.org/licenses/>.
%
%
%%%%%%%%%%%%%%%%%%%%%%%%%%%%%%%%%%%%%%%%%%%%%%%%%%%%%%%%%%%%%%%%%%%%%%%%%%%%%
% SEE pgfplots-macros.tex as well!
%\pdfminorversion=5 % to allow compression
%\pdfobjcompresslevel=2
\documentclass[a4paper,openany]{book}

\let\bookmaketitle=\maketitle

% -----------------------------------
% this here is from ltxdoc:
\usepackage{doc}[=v2]
\AtBeginDocument{\MakeShortVerb{\|}}
%\setlength{\textwidth}{355pt}
%\addtolength\marginparwidth{30pt}
%\addtolength\oddsidemargin{20pt}
%\addtolength\evensidemargin{20pt}
\def\cmd#1{\cs{\expandafter\cmd@to@cs\string#1}}
\def\cmd@to@cs#1#2{\char\number`#2\relax}
\DeclareRobustCommand\cs[1]{\texttt{\char`\\#1}}
\providecommand\marg[1]{%
  {\ttfamily\char`\{}\meta{#1}{\ttfamily\char`\}}}
\providecommand\oarg[1]{%
  {\ttfamily[}\meta{#1}{\ttfamily]}}
\providecommand\parg[1]{%
  {\ttfamily(}\meta{#1}{\ttfamily)}}
\raggedbottom

% -----------------------------------
\input{pgfplots.preamble.tex}

\makeatletter
% I want a two-column index just as in pgfmanual styles. This here
% was the best way to get one:
\def\index@prologue{\section*{Index}\addcontentsline{toc}{chapter}{Index}
}
\makeatother

%\RequirePackage[german,english,francais]{babel}

\def\matlabcolormaptext{This colormap is similar to one shipped with Matlab$^\text{\textregistered}$ under a similar name.}

\IfFileExists{tikzlibraryspy.code.tex}{%
\usetikzlibrary{spy}
}{%
    \message{ERROR: tikz SPY library NOT available. The manual will only compile partially.^^J}%
}%

\usepackage{xparse}% for colorbrewer manual

\usetikzlibrary{
    decorations.markings,
    decorations.footprints,
    shapes.arrows,
    matrix,
    positioning,
}

\usepgfplotslibrary{
    fillbetween,
    ternary,
    smithchart,
    patchplots,
    polar,
    colormaps,
    colorbrewer,
    colortol,           % docu not ready yet
}
\pgfqkeys{/codeexample}{%
    every codeexample/.append style={
        /pgfplots/every ternary axis/.append style={
            /pgfplots/legend style={fill=graphicbackground},
        }
    },
    tabsize=4,
}

\pgfplotsmanualenableexternalizationofexpensive

%\usetikzlibrary{external}
%\tikzexternalize[prefix=figures/]{pgfplots}

\title{%
    Manual for Package \PGFPlots{}\\
    {\small 2D/3D Plots in \LaTeX{}, Version \pgfplotsversion}\\
    {\small\url{https://github.com/pgf-tikz/pgfplots}}
    %\\{\small Attention: you are using an unstable development version.}
}

\makeatletter
\long\def\abstractsmuggle{%
    \centering
    \textbf{Abstract}\\[0.5cm]

    \begin{minipage}{12cm}
        \PGFPlots{} draws high-quality function plots in normal or logarithmic
        scaling with a user-friendly interface directly in \TeX{}. The user
        supplies axis labels, legend entries and the plot coordinates for one or
        more plots and \PGFPlots{} applies axis scaling, computes any logarithms
        and axis ticks and draws the plots. It supports line plots, scatter
        plots, piecewise constant plots, bar plots, area plots, mesh and surface
        plots, patch plots, contour plots, quiver plots, histogram plots, box
        plots, polar axes, ternary diagrams, smith charts and some more. It is
        based on Till Tantau's package \PGF{}/\Tikz{}.
    \end{minipage}

}%

\expandafter\date\expandafter{\@date\\[2cm]
    \abstractsmuggle
}%
\makeatother

%\includeonly{pgfplots.reference}


\begin{document}

\def\plotcoords{%
\addplot coordinates {
(5,8.312e-02)    (17,2.547e-02)   (49,7.407e-03)
(129,2.102e-03)  (321,5.874e-04)  (769,1.623e-04)
(1793,4.442e-05) (4097,1.207e-05) (9217,3.261e-06)
};

\addplot coordinates{
(7,8.472e-02)    (31,3.044e-02)    (111,1.022e-02)
(351,3.303e-03)  (1023,1.039e-03)  (2815,3.196e-04)
(7423,9.658e-05) (18943,2.873e-05) (47103,8.437e-06)
};

\addplot coordinates{
(9,7.881e-02)     (49,3.243e-02)    (209,1.232e-02)
(769,4.454e-03)   (2561,1.551e-03)  (7937,5.236e-04)
(23297,1.723e-04) (65537,5.545e-05) (178177,1.751e-05)
};

\addplot coordinates{
(11,6.887e-02)    (71,3.177e-02)     (351,1.341e-02)
(1471,5.334e-03)  (5503,2.027e-03)   (18943,7.415e-04)
(61183,2.628e-04) (187903,9.063e-05) (553983,3.053e-05)
};

\addplot coordinates{
(13,5.755e-02)     (97,2.925e-02)     (545,1.351e-02)
(2561,5.842e-03)   (10625,2.397e-03)  (40193,9.414e-04)
(141569,3.564e-04) (471041,1.308e-04)
(1496065,4.670e-05)
};
}%


\bookmaketitle
\tableofcontents
\include{pgfplots.title_abstract_intro}
\include{pgfplots.preliminaries}
\include{pgfplots.intro}
\include{pgfplots.reference}
\include{pgfplots.libs}
\include{pgfplots.resources}
\include{pgfplots.importexport}
\include{pgfplots.basic.reference}

\printindex

\bibliographystyle{abbrv} %gerapali} %gerabbrv} %gerunsrt.bst} %gerabbrv}% gerplain}
\nocite{pgfplotstable}
\nocite{programmingnotes}
\bibliography{pgfplots}
\end{document}

% -----------------------------------------------------------------------------
% For Stefan Pinnow as reminder on what to look for when editing the manual
% -----------------------------------------------------------------------------
% There should be no line breaks in the following environments
% - |...|
% - \declareandlabel{...}
% - \verbpdfref{...}
% If "MakeTikzPictures" isn't running through check one of the externalized
% LOG files what is the cause of that.
% -----------------------------------------------------------------------------


%%%%%%%%%%%%%%%%%%%%%%%%%%%%%%%%%%%%%%%%%%%%%%%%%%%%%%%%%%%%%%%%%%%%%%%%%%%%%
%
% Package pgfplots.sty documentation.
%
% Copyright 2007/2008 by Christian Feuersaenger.
%
% This program is free software: you can redistribute it and/or modify
% it under the terms of the GNU General Public License as published by
% the Free Software Foundation, either version 3 of the License, or
% (at your option) any later version.
%
% This program is distributed in the hope that it will be useful,
% but WITHOUT ANY WARRANTY; without even the implied warranty of
% MERCHANTABILITY or FITNESS FOR A PARTICULAR PURPOSE.  See the
% GNU General Public License for more details.
%
% You should have received a copy of the GNU General Public License
% along with this program.  If not, see <http://www.gnu.org/licenses/>.
%
%
%%%%%%%%%%%%%%%%%%%%%%%%%%%%%%%%%%%%%%%%%%%%%%%%%%%%%%%%%%%%%%%%%%%%%%%%%%%%%
% SEE pgfplots-macros.tex as well!
%\pdfminorversion=5 % to allow compression
%\pdfobjcompresslevel=2
\documentclass[a4paper,openany]{book}

\let\bookmaketitle=\maketitle

% -----------------------------------
% this here is from ltxdoc:
\usepackage{doc}[=v2]
\AtBeginDocument{\MakeShortVerb{\|}}
%\setlength{\textwidth}{355pt}
%\addtolength\marginparwidth{30pt}
%\addtolength\oddsidemargin{20pt}
%\addtolength\evensidemargin{20pt}
\def\cmd#1{\cs{\expandafter\cmd@to@cs\string#1}}
\def\cmd@to@cs#1#2{\char\number`#2\relax}
\DeclareRobustCommand\cs[1]{\texttt{\char`\\#1}}
\providecommand\marg[1]{%
  {\ttfamily\char`\{}\meta{#1}{\ttfamily\char`\}}}
\providecommand\oarg[1]{%
  {\ttfamily[}\meta{#1}{\ttfamily]}}
\providecommand\parg[1]{%
  {\ttfamily(}\meta{#1}{\ttfamily)}}
\raggedbottom

% -----------------------------------
\input{pgfplots.preamble.tex}

\makeatletter
% I want a two-column index just as in pgfmanual styles. This here
% was the best way to get one:
\def\index@prologue{\section*{Index}\addcontentsline{toc}{chapter}{Index}
}
\makeatother

%\RequirePackage[german,english,francais]{babel}

\def\matlabcolormaptext{This colormap is similar to one shipped with Matlab$^\text{\textregistered}$ under a similar name.}

\IfFileExists{tikzlibraryspy.code.tex}{%
\usetikzlibrary{spy}
}{%
    \message{ERROR: tikz SPY library NOT available. The manual will only compile partially.^^J}%
}%

\usepackage{xparse}% for colorbrewer manual

\usetikzlibrary{
    decorations.markings,
    decorations.footprints,
    shapes.arrows,
    matrix,
    positioning,
}

\usepgfplotslibrary{
    fillbetween,
    ternary,
    smithchart,
    patchplots,
    polar,
    colormaps,
    colorbrewer,
    colortol,           % docu not ready yet
}
\pgfqkeys{/codeexample}{%
    every codeexample/.append style={
        /pgfplots/every ternary axis/.append style={
            /pgfplots/legend style={fill=graphicbackground},
        }
    },
    tabsize=4,
}

\pgfplotsmanualenableexternalizationofexpensive

%\usetikzlibrary{external}
%\tikzexternalize[prefix=figures/]{pgfplots}

\title{%
    Manual for Package \PGFPlots{}\\
    {\small 2D/3D Plots in \LaTeX{}, Version \pgfplotsversion}\\
    {\small\url{https://github.com/pgf-tikz/pgfplots}}
    %\\{\small Attention: you are using an unstable development version.}
}

\makeatletter
\long\def\abstractsmuggle{%
    \centering
    \textbf{Abstract}\\[0.5cm]

    \begin{minipage}{12cm}
        \PGFPlots{} draws high-quality function plots in normal or logarithmic
        scaling with a user-friendly interface directly in \TeX{}. The user
        supplies axis labels, legend entries and the plot coordinates for one or
        more plots and \PGFPlots{} applies axis scaling, computes any logarithms
        and axis ticks and draws the plots. It supports line plots, scatter
        plots, piecewise constant plots, bar plots, area plots, mesh and surface
        plots, patch plots, contour plots, quiver plots, histogram plots, box
        plots, polar axes, ternary diagrams, smith charts and some more. It is
        based on Till Tantau's package \PGF{}/\Tikz{}.
    \end{minipage}

}%

\expandafter\date\expandafter{\@date\\[2cm]
    \abstractsmuggle
}%
\makeatother

%\includeonly{pgfplots.reference}


\begin{document}

\def\plotcoords{%
\addplot coordinates {
(5,8.312e-02)    (17,2.547e-02)   (49,7.407e-03)
(129,2.102e-03)  (321,5.874e-04)  (769,1.623e-04)
(1793,4.442e-05) (4097,1.207e-05) (9217,3.261e-06)
};

\addplot coordinates{
(7,8.472e-02)    (31,3.044e-02)    (111,1.022e-02)
(351,3.303e-03)  (1023,1.039e-03)  (2815,3.196e-04)
(7423,9.658e-05) (18943,2.873e-05) (47103,8.437e-06)
};

\addplot coordinates{
(9,7.881e-02)     (49,3.243e-02)    (209,1.232e-02)
(769,4.454e-03)   (2561,1.551e-03)  (7937,5.236e-04)
(23297,1.723e-04) (65537,5.545e-05) (178177,1.751e-05)
};

\addplot coordinates{
(11,6.887e-02)    (71,3.177e-02)     (351,1.341e-02)
(1471,5.334e-03)  (5503,2.027e-03)   (18943,7.415e-04)
(61183,2.628e-04) (187903,9.063e-05) (553983,3.053e-05)
};

\addplot coordinates{
(13,5.755e-02)     (97,2.925e-02)     (545,1.351e-02)
(2561,5.842e-03)   (10625,2.397e-03)  (40193,9.414e-04)
(141569,3.564e-04) (471041,1.308e-04)
(1496065,4.670e-05)
};
}%


\bookmaketitle
\tableofcontents
\include{pgfplots.title_abstract_intro}
\include{pgfplots.preliminaries}
\include{pgfplots.intro}
\include{pgfplots.reference}
\include{pgfplots.libs}
\include{pgfplots.resources}
\include{pgfplots.importexport}
\include{pgfplots.basic.reference}

\printindex

\bibliographystyle{abbrv} %gerapali} %gerabbrv} %gerunsrt.bst} %gerabbrv}% gerplain}
\nocite{pgfplotstable}
\nocite{programmingnotes}
\bibliography{pgfplots}
\end{document}

% -----------------------------------------------------------------------------
% For Stefan Pinnow as reminder on what to look for when editing the manual
% -----------------------------------------------------------------------------
% There should be no line breaks in the following environments
% - |...|
% - \declareandlabel{...}
% - \verbpdfref{...}
% If "MakeTikzPictures" isn't running through check one of the externalized
% LOG files what is the cause of that.
% -----------------------------------------------------------------------------

\include{pgfplots.basic.reference}

\printindex

\bibliographystyle{abbrv} %gerapali} %gerabbrv} %gerunsrt.bst} %gerabbrv}% gerplain}
\nocite{pgfplotstable}
\nocite{programmingnotes}
\bibliography{pgfplots}
\end{document}

% -----------------------------------------------------------------------------
% For Stefan Pinnow as reminder on what to look for when editing the manual
% -----------------------------------------------------------------------------
% There should be no line breaks in the following environments
% - |...|
% - \declareandlabel{...}
% - \verbpdfref{...}
% If "MakeTikzPictures" isn't running through check one of the externalized
% LOG files what is the cause of that.
% -----------------------------------------------------------------------------


%%%%%%%%%%%%%%%%%%%%%%%%%%%%%%%%%%%%%%%%%%%%%%%%%%%%%%%%%%%%%%%%%%%%%%%%%%%%%
%
% Package pgfplots.sty documentation.
%
% Copyright 2007/2008 by Christian Feuersaenger.
%
% This program is free software: you can redistribute it and/or modify
% it under the terms of the GNU General Public License as published by
% the Free Software Foundation, either version 3 of the License, or
% (at your option) any later version.
%
% This program is distributed in the hope that it will be useful,
% but WITHOUT ANY WARRANTY; without even the implied warranty of
% MERCHANTABILITY or FITNESS FOR A PARTICULAR PURPOSE.  See the
% GNU General Public License for more details.
%
% You should have received a copy of the GNU General Public License
% along with this program.  If not, see <http://www.gnu.org/licenses/>.
%
%
%%%%%%%%%%%%%%%%%%%%%%%%%%%%%%%%%%%%%%%%%%%%%%%%%%%%%%%%%%%%%%%%%%%%%%%%%%%%%
% SEE pgfplots-macros.tex as well!
%\pdfminorversion=5 % to allow compression
%\pdfobjcompresslevel=2
\documentclass[a4paper,openany]{book}

\let\bookmaketitle=\maketitle

% -----------------------------------
% this here is from ltxdoc:
\usepackage{doc}[=v2]
\AtBeginDocument{\MakeShortVerb{\|}}
%\setlength{\textwidth}{355pt}
%\addtolength\marginparwidth{30pt}
%\addtolength\oddsidemargin{20pt}
%\addtolength\evensidemargin{20pt}
\def\cmd#1{\cs{\expandafter\cmd@to@cs\string#1}}
\def\cmd@to@cs#1#2{\char\number`#2\relax}
\DeclareRobustCommand\cs[1]{\texttt{\char`\\#1}}
\providecommand\marg[1]{%
  {\ttfamily\char`\{}\meta{#1}{\ttfamily\char`\}}}
\providecommand\oarg[1]{%
  {\ttfamily[}\meta{#1}{\ttfamily]}}
\providecommand\parg[1]{%
  {\ttfamily(}\meta{#1}{\ttfamily)}}
\raggedbottom

% -----------------------------------
\input{pgfplots.preamble.tex}

\makeatletter
% I want a two-column index just as in pgfmanual styles. This here
% was the best way to get one:
\def\index@prologue{\section*{Index}\addcontentsline{toc}{chapter}{Index}
}
\makeatother

%\RequirePackage[german,english,francais]{babel}

\def\matlabcolormaptext{This colormap is similar to one shipped with Matlab$^\text{\textregistered}$ under a similar name.}

\IfFileExists{tikzlibraryspy.code.tex}{%
\usetikzlibrary{spy}
}{%
    \message{ERROR: tikz SPY library NOT available. The manual will only compile partially.^^J}%
}%

\usepackage{xparse}% for colorbrewer manual

\usetikzlibrary{
    decorations.markings,
    decorations.footprints,
    shapes.arrows,
    matrix,
    positioning,
}

\usepgfplotslibrary{
    fillbetween,
    ternary,
    smithchart,
    patchplots,
    polar,
    colormaps,
    colorbrewer,
    colortol,           % docu not ready yet
}
\pgfqkeys{/codeexample}{%
    every codeexample/.append style={
        /pgfplots/every ternary axis/.append style={
            /pgfplots/legend style={fill=graphicbackground},
        }
    },
    tabsize=4,
}

\pgfplotsmanualenableexternalizationofexpensive

%\usetikzlibrary{external}
%\tikzexternalize[prefix=figures/]{pgfplots}

\title{%
    Manual for Package \PGFPlots{}\\
    {\small 2D/3D Plots in \LaTeX{}, Version \pgfplotsversion}\\
    {\small\url{https://github.com/pgf-tikz/pgfplots}}
    %\\{\small Attention: you are using an unstable development version.}
}

\makeatletter
\long\def\abstractsmuggle{%
    \centering
    \textbf{Abstract}\\[0.5cm]

    \begin{minipage}{12cm}
        \PGFPlots{} draws high-quality function plots in normal or logarithmic
        scaling with a user-friendly interface directly in \TeX{}. The user
        supplies axis labels, legend entries and the plot coordinates for one or
        more plots and \PGFPlots{} applies axis scaling, computes any logarithms
        and axis ticks and draws the plots. It supports line plots, scatter
        plots, piecewise constant plots, bar plots, area plots, mesh and surface
        plots, patch plots, contour plots, quiver plots, histogram plots, box
        plots, polar axes, ternary diagrams, smith charts and some more. It is
        based on Till Tantau's package \PGF{}/\Tikz{}.
    \end{minipage}

}%

\expandafter\date\expandafter{\@date\\[2cm]
    \abstractsmuggle
}%
\makeatother

%\includeonly{pgfplots.reference}


\begin{document}

\def\plotcoords{%
\addplot coordinates {
(5,8.312e-02)    (17,2.547e-02)   (49,7.407e-03)
(129,2.102e-03)  (321,5.874e-04)  (769,1.623e-04)
(1793,4.442e-05) (4097,1.207e-05) (9217,3.261e-06)
};

\addplot coordinates{
(7,8.472e-02)    (31,3.044e-02)    (111,1.022e-02)
(351,3.303e-03)  (1023,1.039e-03)  (2815,3.196e-04)
(7423,9.658e-05) (18943,2.873e-05) (47103,8.437e-06)
};

\addplot coordinates{
(9,7.881e-02)     (49,3.243e-02)    (209,1.232e-02)
(769,4.454e-03)   (2561,1.551e-03)  (7937,5.236e-04)
(23297,1.723e-04) (65537,5.545e-05) (178177,1.751e-05)
};

\addplot coordinates{
(11,6.887e-02)    (71,3.177e-02)     (351,1.341e-02)
(1471,5.334e-03)  (5503,2.027e-03)   (18943,7.415e-04)
(61183,2.628e-04) (187903,9.063e-05) (553983,3.053e-05)
};

\addplot coordinates{
(13,5.755e-02)     (97,2.925e-02)     (545,1.351e-02)
(2561,5.842e-03)   (10625,2.397e-03)  (40193,9.414e-04)
(141569,3.564e-04) (471041,1.308e-04)
(1496065,4.670e-05)
};
}%


\bookmaketitle
\tableofcontents

%%%%%%%%%%%%%%%%%%%%%%%%%%%%%%%%%%%%%%%%%%%%%%%%%%%%%%%%%%%%%%%%%%%%%%%%%%%%%
%
% Package pgfplots.sty documentation.
%
% Copyright 2007/2008 by Christian Feuersaenger.
%
% This program is free software: you can redistribute it and/or modify
% it under the terms of the GNU General Public License as published by
% the Free Software Foundation, either version 3 of the License, or
% (at your option) any later version.
%
% This program is distributed in the hope that it will be useful,
% but WITHOUT ANY WARRANTY; without even the implied warranty of
% MERCHANTABILITY or FITNESS FOR A PARTICULAR PURPOSE.  See the
% GNU General Public License for more details.
%
% You should have received a copy of the GNU General Public License
% along with this program.  If not, see <http://www.gnu.org/licenses/>.
%
%
%%%%%%%%%%%%%%%%%%%%%%%%%%%%%%%%%%%%%%%%%%%%%%%%%%%%%%%%%%%%%%%%%%%%%%%%%%%%%
% SEE pgfplots-macros.tex as well!
%\pdfminorversion=5 % to allow compression
%\pdfobjcompresslevel=2
\documentclass[a4paper,openany]{book}

\let\bookmaketitle=\maketitle

% -----------------------------------
% this here is from ltxdoc:
\usepackage{doc}[=v2]
\AtBeginDocument{\MakeShortVerb{\|}}
%\setlength{\textwidth}{355pt}
%\addtolength\marginparwidth{30pt}
%\addtolength\oddsidemargin{20pt}
%\addtolength\evensidemargin{20pt}
\def\cmd#1{\cs{\expandafter\cmd@to@cs\string#1}}
\def\cmd@to@cs#1#2{\char\number`#2\relax}
\DeclareRobustCommand\cs[1]{\texttt{\char`\\#1}}
\providecommand\marg[1]{%
  {\ttfamily\char`\{}\meta{#1}{\ttfamily\char`\}}}
\providecommand\oarg[1]{%
  {\ttfamily[}\meta{#1}{\ttfamily]}}
\providecommand\parg[1]{%
  {\ttfamily(}\meta{#1}{\ttfamily)}}
\raggedbottom

% -----------------------------------
\input{pgfplots.preamble.tex}

\makeatletter
% I want a two-column index just as in pgfmanual styles. This here
% was the best way to get one:
\def\index@prologue{\section*{Index}\addcontentsline{toc}{chapter}{Index}
}
\makeatother

%\RequirePackage[german,english,francais]{babel}

\def\matlabcolormaptext{This colormap is similar to one shipped with Matlab$^\text{\textregistered}$ under a similar name.}

\IfFileExists{tikzlibraryspy.code.tex}{%
\usetikzlibrary{spy}
}{%
    \message{ERROR: tikz SPY library NOT available. The manual will only compile partially.^^J}%
}%

\usepackage{xparse}% for colorbrewer manual

\usetikzlibrary{
    decorations.markings,
    decorations.footprints,
    shapes.arrows,
    matrix,
    positioning,
}

\usepgfplotslibrary{
    fillbetween,
    ternary,
    smithchart,
    patchplots,
    polar,
    colormaps,
    colorbrewer,
    colortol,           % docu not ready yet
}
\pgfqkeys{/codeexample}{%
    every codeexample/.append style={
        /pgfplots/every ternary axis/.append style={
            /pgfplots/legend style={fill=graphicbackground},
        }
    },
    tabsize=4,
}

\pgfplotsmanualenableexternalizationofexpensive

%\usetikzlibrary{external}
%\tikzexternalize[prefix=figures/]{pgfplots}

\title{%
    Manual for Package \PGFPlots{}\\
    {\small 2D/3D Plots in \LaTeX{}, Version \pgfplotsversion}\\
    {\small\url{https://github.com/pgf-tikz/pgfplots}}
    %\\{\small Attention: you are using an unstable development version.}
}

\makeatletter
\long\def\abstractsmuggle{%
    \centering
    \textbf{Abstract}\\[0.5cm]

    \begin{minipage}{12cm}
        \PGFPlots{} draws high-quality function plots in normal or logarithmic
        scaling with a user-friendly interface directly in \TeX{}. The user
        supplies axis labels, legend entries and the plot coordinates for one or
        more plots and \PGFPlots{} applies axis scaling, computes any logarithms
        and axis ticks and draws the plots. It supports line plots, scatter
        plots, piecewise constant plots, bar plots, area plots, mesh and surface
        plots, patch plots, contour plots, quiver plots, histogram plots, box
        plots, polar axes, ternary diagrams, smith charts and some more. It is
        based on Till Tantau's package \PGF{}/\Tikz{}.
    \end{minipage}

}%

\expandafter\date\expandafter{\@date\\[2cm]
    \abstractsmuggle
}%
\makeatother

%\includeonly{pgfplots.reference}


\begin{document}

\def\plotcoords{%
\addplot coordinates {
(5,8.312e-02)    (17,2.547e-02)   (49,7.407e-03)
(129,2.102e-03)  (321,5.874e-04)  (769,1.623e-04)
(1793,4.442e-05) (4097,1.207e-05) (9217,3.261e-06)
};

\addplot coordinates{
(7,8.472e-02)    (31,3.044e-02)    (111,1.022e-02)
(351,3.303e-03)  (1023,1.039e-03)  (2815,3.196e-04)
(7423,9.658e-05) (18943,2.873e-05) (47103,8.437e-06)
};

\addplot coordinates{
(9,7.881e-02)     (49,3.243e-02)    (209,1.232e-02)
(769,4.454e-03)   (2561,1.551e-03)  (7937,5.236e-04)
(23297,1.723e-04) (65537,5.545e-05) (178177,1.751e-05)
};

\addplot coordinates{
(11,6.887e-02)    (71,3.177e-02)     (351,1.341e-02)
(1471,5.334e-03)  (5503,2.027e-03)   (18943,7.415e-04)
(61183,2.628e-04) (187903,9.063e-05) (553983,3.053e-05)
};

\addplot coordinates{
(13,5.755e-02)     (97,2.925e-02)     (545,1.351e-02)
(2561,5.842e-03)   (10625,2.397e-03)  (40193,9.414e-04)
(141569,3.564e-04) (471041,1.308e-04)
(1496065,4.670e-05)
};
}%


\bookmaketitle
\tableofcontents
\include{pgfplots.title_abstract_intro}
\include{pgfplots.preliminaries}
\include{pgfplots.intro}
\include{pgfplots.reference}
\include{pgfplots.libs}
\include{pgfplots.resources}
\include{pgfplots.importexport}
\include{pgfplots.basic.reference}

\printindex

\bibliographystyle{abbrv} %gerapali} %gerabbrv} %gerunsrt.bst} %gerabbrv}% gerplain}
\nocite{pgfplotstable}
\nocite{programmingnotes}
\bibliography{pgfplots}
\end{document}

% -----------------------------------------------------------------------------
% For Stefan Pinnow as reminder on what to look for when editing the manual
% -----------------------------------------------------------------------------
% There should be no line breaks in the following environments
% - |...|
% - \declareandlabel{...}
% - \verbpdfref{...}
% If "MakeTikzPictures" isn't running through check one of the externalized
% LOG files what is the cause of that.
% -----------------------------------------------------------------------------


%%%%%%%%%%%%%%%%%%%%%%%%%%%%%%%%%%%%%%%%%%%%%%%%%%%%%%%%%%%%%%%%%%%%%%%%%%%%%
%
% Package pgfplots.sty documentation.
%
% Copyright 2007/2008 by Christian Feuersaenger.
%
% This program is free software: you can redistribute it and/or modify
% it under the terms of the GNU General Public License as published by
% the Free Software Foundation, either version 3 of the License, or
% (at your option) any later version.
%
% This program is distributed in the hope that it will be useful,
% but WITHOUT ANY WARRANTY; without even the implied warranty of
% MERCHANTABILITY or FITNESS FOR A PARTICULAR PURPOSE.  See the
% GNU General Public License for more details.
%
% You should have received a copy of the GNU General Public License
% along with this program.  If not, see <http://www.gnu.org/licenses/>.
%
%
%%%%%%%%%%%%%%%%%%%%%%%%%%%%%%%%%%%%%%%%%%%%%%%%%%%%%%%%%%%%%%%%%%%%%%%%%%%%%
% SEE pgfplots-macros.tex as well!
%\pdfminorversion=5 % to allow compression
%\pdfobjcompresslevel=2
\documentclass[a4paper,openany]{book}

\let\bookmaketitle=\maketitle

% -----------------------------------
% this here is from ltxdoc:
\usepackage{doc}[=v2]
\AtBeginDocument{\MakeShortVerb{\|}}
%\setlength{\textwidth}{355pt}
%\addtolength\marginparwidth{30pt}
%\addtolength\oddsidemargin{20pt}
%\addtolength\evensidemargin{20pt}
\def\cmd#1{\cs{\expandafter\cmd@to@cs\string#1}}
\def\cmd@to@cs#1#2{\char\number`#2\relax}
\DeclareRobustCommand\cs[1]{\texttt{\char`\\#1}}
\providecommand\marg[1]{%
  {\ttfamily\char`\{}\meta{#1}{\ttfamily\char`\}}}
\providecommand\oarg[1]{%
  {\ttfamily[}\meta{#1}{\ttfamily]}}
\providecommand\parg[1]{%
  {\ttfamily(}\meta{#1}{\ttfamily)}}
\raggedbottom

% -----------------------------------
\input{pgfplots.preamble.tex}

\makeatletter
% I want a two-column index just as in pgfmanual styles. This here
% was the best way to get one:
\def\index@prologue{\section*{Index}\addcontentsline{toc}{chapter}{Index}
}
\makeatother

%\RequirePackage[german,english,francais]{babel}

\def\matlabcolormaptext{This colormap is similar to one shipped with Matlab$^\text{\textregistered}$ under a similar name.}

\IfFileExists{tikzlibraryspy.code.tex}{%
\usetikzlibrary{spy}
}{%
    \message{ERROR: tikz SPY library NOT available. The manual will only compile partially.^^J}%
}%

\usepackage{xparse}% for colorbrewer manual

\usetikzlibrary{
    decorations.markings,
    decorations.footprints,
    shapes.arrows,
    matrix,
    positioning,
}

\usepgfplotslibrary{
    fillbetween,
    ternary,
    smithchart,
    patchplots,
    polar,
    colormaps,
    colorbrewer,
    colortol,           % docu not ready yet
}
\pgfqkeys{/codeexample}{%
    every codeexample/.append style={
        /pgfplots/every ternary axis/.append style={
            /pgfplots/legend style={fill=graphicbackground},
        }
    },
    tabsize=4,
}

\pgfplotsmanualenableexternalizationofexpensive

%\usetikzlibrary{external}
%\tikzexternalize[prefix=figures/]{pgfplots}

\title{%
    Manual for Package \PGFPlots{}\\
    {\small 2D/3D Plots in \LaTeX{}, Version \pgfplotsversion}\\
    {\small\url{https://github.com/pgf-tikz/pgfplots}}
    %\\{\small Attention: you are using an unstable development version.}
}

\makeatletter
\long\def\abstractsmuggle{%
    \centering
    \textbf{Abstract}\\[0.5cm]

    \begin{minipage}{12cm}
        \PGFPlots{} draws high-quality function plots in normal or logarithmic
        scaling with a user-friendly interface directly in \TeX{}. The user
        supplies axis labels, legend entries and the plot coordinates for one or
        more plots and \PGFPlots{} applies axis scaling, computes any logarithms
        and axis ticks and draws the plots. It supports line plots, scatter
        plots, piecewise constant plots, bar plots, area plots, mesh and surface
        plots, patch plots, contour plots, quiver plots, histogram plots, box
        plots, polar axes, ternary diagrams, smith charts and some more. It is
        based on Till Tantau's package \PGF{}/\Tikz{}.
    \end{minipage}

}%

\expandafter\date\expandafter{\@date\\[2cm]
    \abstractsmuggle
}%
\makeatother

%\includeonly{pgfplots.reference}


\begin{document}

\def\plotcoords{%
\addplot coordinates {
(5,8.312e-02)    (17,2.547e-02)   (49,7.407e-03)
(129,2.102e-03)  (321,5.874e-04)  (769,1.623e-04)
(1793,4.442e-05) (4097,1.207e-05) (9217,3.261e-06)
};

\addplot coordinates{
(7,8.472e-02)    (31,3.044e-02)    (111,1.022e-02)
(351,3.303e-03)  (1023,1.039e-03)  (2815,3.196e-04)
(7423,9.658e-05) (18943,2.873e-05) (47103,8.437e-06)
};

\addplot coordinates{
(9,7.881e-02)     (49,3.243e-02)    (209,1.232e-02)
(769,4.454e-03)   (2561,1.551e-03)  (7937,5.236e-04)
(23297,1.723e-04) (65537,5.545e-05) (178177,1.751e-05)
};

\addplot coordinates{
(11,6.887e-02)    (71,3.177e-02)     (351,1.341e-02)
(1471,5.334e-03)  (5503,2.027e-03)   (18943,7.415e-04)
(61183,2.628e-04) (187903,9.063e-05) (553983,3.053e-05)
};

\addplot coordinates{
(13,5.755e-02)     (97,2.925e-02)     (545,1.351e-02)
(2561,5.842e-03)   (10625,2.397e-03)  (40193,9.414e-04)
(141569,3.564e-04) (471041,1.308e-04)
(1496065,4.670e-05)
};
}%


\bookmaketitle
\tableofcontents
\include{pgfplots.title_abstract_intro}
\include{pgfplots.preliminaries}
\include{pgfplots.intro}
\include{pgfplots.reference}
\include{pgfplots.libs}
\include{pgfplots.resources}
\include{pgfplots.importexport}
\include{pgfplots.basic.reference}

\printindex

\bibliographystyle{abbrv} %gerapali} %gerabbrv} %gerunsrt.bst} %gerabbrv}% gerplain}
\nocite{pgfplotstable}
\nocite{programmingnotes}
\bibliography{pgfplots}
\end{document}

% -----------------------------------------------------------------------------
% For Stefan Pinnow as reminder on what to look for when editing the manual
% -----------------------------------------------------------------------------
% There should be no line breaks in the following environments
% - |...|
% - \declareandlabel{...}
% - \verbpdfref{...}
% If "MakeTikzPictures" isn't running through check one of the externalized
% LOG files what is the cause of that.
% -----------------------------------------------------------------------------


%%%%%%%%%%%%%%%%%%%%%%%%%%%%%%%%%%%%%%%%%%%%%%%%%%%%%%%%%%%%%%%%%%%%%%%%%%%%%
%
% Package pgfplots.sty documentation.
%
% Copyright 2007/2008 by Christian Feuersaenger.
%
% This program is free software: you can redistribute it and/or modify
% it under the terms of the GNU General Public License as published by
% the Free Software Foundation, either version 3 of the License, or
% (at your option) any later version.
%
% This program is distributed in the hope that it will be useful,
% but WITHOUT ANY WARRANTY; without even the implied warranty of
% MERCHANTABILITY or FITNESS FOR A PARTICULAR PURPOSE.  See the
% GNU General Public License for more details.
%
% You should have received a copy of the GNU General Public License
% along with this program.  If not, see <http://www.gnu.org/licenses/>.
%
%
%%%%%%%%%%%%%%%%%%%%%%%%%%%%%%%%%%%%%%%%%%%%%%%%%%%%%%%%%%%%%%%%%%%%%%%%%%%%%
% SEE pgfplots-macros.tex as well!
%\pdfminorversion=5 % to allow compression
%\pdfobjcompresslevel=2
\documentclass[a4paper,openany]{book}

\let\bookmaketitle=\maketitle

% -----------------------------------
% this here is from ltxdoc:
\usepackage{doc}[=v2]
\AtBeginDocument{\MakeShortVerb{\|}}
%\setlength{\textwidth}{355pt}
%\addtolength\marginparwidth{30pt}
%\addtolength\oddsidemargin{20pt}
%\addtolength\evensidemargin{20pt}
\def\cmd#1{\cs{\expandafter\cmd@to@cs\string#1}}
\def\cmd@to@cs#1#2{\char\number`#2\relax}
\DeclareRobustCommand\cs[1]{\texttt{\char`\\#1}}
\providecommand\marg[1]{%
  {\ttfamily\char`\{}\meta{#1}{\ttfamily\char`\}}}
\providecommand\oarg[1]{%
  {\ttfamily[}\meta{#1}{\ttfamily]}}
\providecommand\parg[1]{%
  {\ttfamily(}\meta{#1}{\ttfamily)}}
\raggedbottom

% -----------------------------------
\input{pgfplots.preamble.tex}

\makeatletter
% I want a two-column index just as in pgfmanual styles. This here
% was the best way to get one:
\def\index@prologue{\section*{Index}\addcontentsline{toc}{chapter}{Index}
}
\makeatother

%\RequirePackage[german,english,francais]{babel}

\def\matlabcolormaptext{This colormap is similar to one shipped with Matlab$^\text{\textregistered}$ under a similar name.}

\IfFileExists{tikzlibraryspy.code.tex}{%
\usetikzlibrary{spy}
}{%
    \message{ERROR: tikz SPY library NOT available. The manual will only compile partially.^^J}%
}%

\usepackage{xparse}% for colorbrewer manual

\usetikzlibrary{
    decorations.markings,
    decorations.footprints,
    shapes.arrows,
    matrix,
    positioning,
}

\usepgfplotslibrary{
    fillbetween,
    ternary,
    smithchart,
    patchplots,
    polar,
    colormaps,
    colorbrewer,
    colortol,           % docu not ready yet
}
\pgfqkeys{/codeexample}{%
    every codeexample/.append style={
        /pgfplots/every ternary axis/.append style={
            /pgfplots/legend style={fill=graphicbackground},
        }
    },
    tabsize=4,
}

\pgfplotsmanualenableexternalizationofexpensive

%\usetikzlibrary{external}
%\tikzexternalize[prefix=figures/]{pgfplots}

\title{%
    Manual for Package \PGFPlots{}\\
    {\small 2D/3D Plots in \LaTeX{}, Version \pgfplotsversion}\\
    {\small\url{https://github.com/pgf-tikz/pgfplots}}
    %\\{\small Attention: you are using an unstable development version.}
}

\makeatletter
\long\def\abstractsmuggle{%
    \centering
    \textbf{Abstract}\\[0.5cm]

    \begin{minipage}{12cm}
        \PGFPlots{} draws high-quality function plots in normal or logarithmic
        scaling with a user-friendly interface directly in \TeX{}. The user
        supplies axis labels, legend entries and the plot coordinates for one or
        more plots and \PGFPlots{} applies axis scaling, computes any logarithms
        and axis ticks and draws the plots. It supports line plots, scatter
        plots, piecewise constant plots, bar plots, area plots, mesh and surface
        plots, patch plots, contour plots, quiver plots, histogram plots, box
        plots, polar axes, ternary diagrams, smith charts and some more. It is
        based on Till Tantau's package \PGF{}/\Tikz{}.
    \end{minipage}

}%

\expandafter\date\expandafter{\@date\\[2cm]
    \abstractsmuggle
}%
\makeatother

%\includeonly{pgfplots.reference}


\begin{document}

\def\plotcoords{%
\addplot coordinates {
(5,8.312e-02)    (17,2.547e-02)   (49,7.407e-03)
(129,2.102e-03)  (321,5.874e-04)  (769,1.623e-04)
(1793,4.442e-05) (4097,1.207e-05) (9217,3.261e-06)
};

\addplot coordinates{
(7,8.472e-02)    (31,3.044e-02)    (111,1.022e-02)
(351,3.303e-03)  (1023,1.039e-03)  (2815,3.196e-04)
(7423,9.658e-05) (18943,2.873e-05) (47103,8.437e-06)
};

\addplot coordinates{
(9,7.881e-02)     (49,3.243e-02)    (209,1.232e-02)
(769,4.454e-03)   (2561,1.551e-03)  (7937,5.236e-04)
(23297,1.723e-04) (65537,5.545e-05) (178177,1.751e-05)
};

\addplot coordinates{
(11,6.887e-02)    (71,3.177e-02)     (351,1.341e-02)
(1471,5.334e-03)  (5503,2.027e-03)   (18943,7.415e-04)
(61183,2.628e-04) (187903,9.063e-05) (553983,3.053e-05)
};

\addplot coordinates{
(13,5.755e-02)     (97,2.925e-02)     (545,1.351e-02)
(2561,5.842e-03)   (10625,2.397e-03)  (40193,9.414e-04)
(141569,3.564e-04) (471041,1.308e-04)
(1496065,4.670e-05)
};
}%


\bookmaketitle
\tableofcontents
\include{pgfplots.title_abstract_intro}
\include{pgfplots.preliminaries}
\include{pgfplots.intro}
\include{pgfplots.reference}
\include{pgfplots.libs}
\include{pgfplots.resources}
\include{pgfplots.importexport}
\include{pgfplots.basic.reference}

\printindex

\bibliographystyle{abbrv} %gerapali} %gerabbrv} %gerunsrt.bst} %gerabbrv}% gerplain}
\nocite{pgfplotstable}
\nocite{programmingnotes}
\bibliography{pgfplots}
\end{document}

% -----------------------------------------------------------------------------
% For Stefan Pinnow as reminder on what to look for when editing the manual
% -----------------------------------------------------------------------------
% There should be no line breaks in the following environments
% - |...|
% - \declareandlabel{...}
% - \verbpdfref{...}
% If "MakeTikzPictures" isn't running through check one of the externalized
% LOG files what is the cause of that.
% -----------------------------------------------------------------------------


%%%%%%%%%%%%%%%%%%%%%%%%%%%%%%%%%%%%%%%%%%%%%%%%%%%%%%%%%%%%%%%%%%%%%%%%%%%%%
%
% Package pgfplots.sty documentation.
%
% Copyright 2007/2008 by Christian Feuersaenger.
%
% This program is free software: you can redistribute it and/or modify
% it under the terms of the GNU General Public License as published by
% the Free Software Foundation, either version 3 of the License, or
% (at your option) any later version.
%
% This program is distributed in the hope that it will be useful,
% but WITHOUT ANY WARRANTY; without even the implied warranty of
% MERCHANTABILITY or FITNESS FOR A PARTICULAR PURPOSE.  See the
% GNU General Public License for more details.
%
% You should have received a copy of the GNU General Public License
% along with this program.  If not, see <http://www.gnu.org/licenses/>.
%
%
%%%%%%%%%%%%%%%%%%%%%%%%%%%%%%%%%%%%%%%%%%%%%%%%%%%%%%%%%%%%%%%%%%%%%%%%%%%%%
% SEE pgfplots-macros.tex as well!
%\pdfminorversion=5 % to allow compression
%\pdfobjcompresslevel=2
\documentclass[a4paper,openany]{book}

\let\bookmaketitle=\maketitle

% -----------------------------------
% this here is from ltxdoc:
\usepackage{doc}[=v2]
\AtBeginDocument{\MakeShortVerb{\|}}
%\setlength{\textwidth}{355pt}
%\addtolength\marginparwidth{30pt}
%\addtolength\oddsidemargin{20pt}
%\addtolength\evensidemargin{20pt}
\def\cmd#1{\cs{\expandafter\cmd@to@cs\string#1}}
\def\cmd@to@cs#1#2{\char\number`#2\relax}
\DeclareRobustCommand\cs[1]{\texttt{\char`\\#1}}
\providecommand\marg[1]{%
  {\ttfamily\char`\{}\meta{#1}{\ttfamily\char`\}}}
\providecommand\oarg[1]{%
  {\ttfamily[}\meta{#1}{\ttfamily]}}
\providecommand\parg[1]{%
  {\ttfamily(}\meta{#1}{\ttfamily)}}
\raggedbottom

% -----------------------------------
\input{pgfplots.preamble.tex}

\makeatletter
% I want a two-column index just as in pgfmanual styles. This here
% was the best way to get one:
\def\index@prologue{\section*{Index}\addcontentsline{toc}{chapter}{Index}
}
\makeatother

%\RequirePackage[german,english,francais]{babel}

\def\matlabcolormaptext{This colormap is similar to one shipped with Matlab$^\text{\textregistered}$ under a similar name.}

\IfFileExists{tikzlibraryspy.code.tex}{%
\usetikzlibrary{spy}
}{%
    \message{ERROR: tikz SPY library NOT available. The manual will only compile partially.^^J}%
}%

\usepackage{xparse}% for colorbrewer manual

\usetikzlibrary{
    decorations.markings,
    decorations.footprints,
    shapes.arrows,
    matrix,
    positioning,
}

\usepgfplotslibrary{
    fillbetween,
    ternary,
    smithchart,
    patchplots,
    polar,
    colormaps,
    colorbrewer,
    colortol,           % docu not ready yet
}
\pgfqkeys{/codeexample}{%
    every codeexample/.append style={
        /pgfplots/every ternary axis/.append style={
            /pgfplots/legend style={fill=graphicbackground},
        }
    },
    tabsize=4,
}

\pgfplotsmanualenableexternalizationofexpensive

%\usetikzlibrary{external}
%\tikzexternalize[prefix=figures/]{pgfplots}

\title{%
    Manual for Package \PGFPlots{}\\
    {\small 2D/3D Plots in \LaTeX{}, Version \pgfplotsversion}\\
    {\small\url{https://github.com/pgf-tikz/pgfplots}}
    %\\{\small Attention: you are using an unstable development version.}
}

\makeatletter
\long\def\abstractsmuggle{%
    \centering
    \textbf{Abstract}\\[0.5cm]

    \begin{minipage}{12cm}
        \PGFPlots{} draws high-quality function plots in normal or logarithmic
        scaling with a user-friendly interface directly in \TeX{}. The user
        supplies axis labels, legend entries and the plot coordinates for one or
        more plots and \PGFPlots{} applies axis scaling, computes any logarithms
        and axis ticks and draws the plots. It supports line plots, scatter
        plots, piecewise constant plots, bar plots, area plots, mesh and surface
        plots, patch plots, contour plots, quiver plots, histogram plots, box
        plots, polar axes, ternary diagrams, smith charts and some more. It is
        based on Till Tantau's package \PGF{}/\Tikz{}.
    \end{minipage}

}%

\expandafter\date\expandafter{\@date\\[2cm]
    \abstractsmuggle
}%
\makeatother

%\includeonly{pgfplots.reference}


\begin{document}

\def\plotcoords{%
\addplot coordinates {
(5,8.312e-02)    (17,2.547e-02)   (49,7.407e-03)
(129,2.102e-03)  (321,5.874e-04)  (769,1.623e-04)
(1793,4.442e-05) (4097,1.207e-05) (9217,3.261e-06)
};

\addplot coordinates{
(7,8.472e-02)    (31,3.044e-02)    (111,1.022e-02)
(351,3.303e-03)  (1023,1.039e-03)  (2815,3.196e-04)
(7423,9.658e-05) (18943,2.873e-05) (47103,8.437e-06)
};

\addplot coordinates{
(9,7.881e-02)     (49,3.243e-02)    (209,1.232e-02)
(769,4.454e-03)   (2561,1.551e-03)  (7937,5.236e-04)
(23297,1.723e-04) (65537,5.545e-05) (178177,1.751e-05)
};

\addplot coordinates{
(11,6.887e-02)    (71,3.177e-02)     (351,1.341e-02)
(1471,5.334e-03)  (5503,2.027e-03)   (18943,7.415e-04)
(61183,2.628e-04) (187903,9.063e-05) (553983,3.053e-05)
};

\addplot coordinates{
(13,5.755e-02)     (97,2.925e-02)     (545,1.351e-02)
(2561,5.842e-03)   (10625,2.397e-03)  (40193,9.414e-04)
(141569,3.564e-04) (471041,1.308e-04)
(1496065,4.670e-05)
};
}%


\bookmaketitle
\tableofcontents
\include{pgfplots.title_abstract_intro}
\include{pgfplots.preliminaries}
\include{pgfplots.intro}
\include{pgfplots.reference}
\include{pgfplots.libs}
\include{pgfplots.resources}
\include{pgfplots.importexport}
\include{pgfplots.basic.reference}

\printindex

\bibliographystyle{abbrv} %gerapali} %gerabbrv} %gerunsrt.bst} %gerabbrv}% gerplain}
\nocite{pgfplotstable}
\nocite{programmingnotes}
\bibliography{pgfplots}
\end{document}

% -----------------------------------------------------------------------------
% For Stefan Pinnow as reminder on what to look for when editing the manual
% -----------------------------------------------------------------------------
% There should be no line breaks in the following environments
% - |...|
% - \declareandlabel{...}
% - \verbpdfref{...}
% If "MakeTikzPictures" isn't running through check one of the externalized
% LOG files what is the cause of that.
% -----------------------------------------------------------------------------


%%%%%%%%%%%%%%%%%%%%%%%%%%%%%%%%%%%%%%%%%%%%%%%%%%%%%%%%%%%%%%%%%%%%%%%%%%%%%
%
% Package pgfplots.sty documentation.
%
% Copyright 2007/2008 by Christian Feuersaenger.
%
% This program is free software: you can redistribute it and/or modify
% it under the terms of the GNU General Public License as published by
% the Free Software Foundation, either version 3 of the License, or
% (at your option) any later version.
%
% This program is distributed in the hope that it will be useful,
% but WITHOUT ANY WARRANTY; without even the implied warranty of
% MERCHANTABILITY or FITNESS FOR A PARTICULAR PURPOSE.  See the
% GNU General Public License for more details.
%
% You should have received a copy of the GNU General Public License
% along with this program.  If not, see <http://www.gnu.org/licenses/>.
%
%
%%%%%%%%%%%%%%%%%%%%%%%%%%%%%%%%%%%%%%%%%%%%%%%%%%%%%%%%%%%%%%%%%%%%%%%%%%%%%
% SEE pgfplots-macros.tex as well!
%\pdfminorversion=5 % to allow compression
%\pdfobjcompresslevel=2
\documentclass[a4paper,openany]{book}

\let\bookmaketitle=\maketitle

% -----------------------------------
% this here is from ltxdoc:
\usepackage{doc}[=v2]
\AtBeginDocument{\MakeShortVerb{\|}}
%\setlength{\textwidth}{355pt}
%\addtolength\marginparwidth{30pt}
%\addtolength\oddsidemargin{20pt}
%\addtolength\evensidemargin{20pt}
\def\cmd#1{\cs{\expandafter\cmd@to@cs\string#1}}
\def\cmd@to@cs#1#2{\char\number`#2\relax}
\DeclareRobustCommand\cs[1]{\texttt{\char`\\#1}}
\providecommand\marg[1]{%
  {\ttfamily\char`\{}\meta{#1}{\ttfamily\char`\}}}
\providecommand\oarg[1]{%
  {\ttfamily[}\meta{#1}{\ttfamily]}}
\providecommand\parg[1]{%
  {\ttfamily(}\meta{#1}{\ttfamily)}}
\raggedbottom

% -----------------------------------
\input{pgfplots.preamble.tex}

\makeatletter
% I want a two-column index just as in pgfmanual styles. This here
% was the best way to get one:
\def\index@prologue{\section*{Index}\addcontentsline{toc}{chapter}{Index}
}
\makeatother

%\RequirePackage[german,english,francais]{babel}

\def\matlabcolormaptext{This colormap is similar to one shipped with Matlab$^\text{\textregistered}$ under a similar name.}

\IfFileExists{tikzlibraryspy.code.tex}{%
\usetikzlibrary{spy}
}{%
    \message{ERROR: tikz SPY library NOT available. The manual will only compile partially.^^J}%
}%

\usepackage{xparse}% for colorbrewer manual

\usetikzlibrary{
    decorations.markings,
    decorations.footprints,
    shapes.arrows,
    matrix,
    positioning,
}

\usepgfplotslibrary{
    fillbetween,
    ternary,
    smithchart,
    patchplots,
    polar,
    colormaps,
    colorbrewer,
    colortol,           % docu not ready yet
}
\pgfqkeys{/codeexample}{%
    every codeexample/.append style={
        /pgfplots/every ternary axis/.append style={
            /pgfplots/legend style={fill=graphicbackground},
        }
    },
    tabsize=4,
}

\pgfplotsmanualenableexternalizationofexpensive

%\usetikzlibrary{external}
%\tikzexternalize[prefix=figures/]{pgfplots}

\title{%
    Manual for Package \PGFPlots{}\\
    {\small 2D/3D Plots in \LaTeX{}, Version \pgfplotsversion}\\
    {\small\url{https://github.com/pgf-tikz/pgfplots}}
    %\\{\small Attention: you are using an unstable development version.}
}

\makeatletter
\long\def\abstractsmuggle{%
    \centering
    \textbf{Abstract}\\[0.5cm]

    \begin{minipage}{12cm}
        \PGFPlots{} draws high-quality function plots in normal or logarithmic
        scaling with a user-friendly interface directly in \TeX{}. The user
        supplies axis labels, legend entries and the plot coordinates for one or
        more plots and \PGFPlots{} applies axis scaling, computes any logarithms
        and axis ticks and draws the plots. It supports line plots, scatter
        plots, piecewise constant plots, bar plots, area plots, mesh and surface
        plots, patch plots, contour plots, quiver plots, histogram plots, box
        plots, polar axes, ternary diagrams, smith charts and some more. It is
        based on Till Tantau's package \PGF{}/\Tikz{}.
    \end{minipage}

}%

\expandafter\date\expandafter{\@date\\[2cm]
    \abstractsmuggle
}%
\makeatother

%\includeonly{pgfplots.reference}


\begin{document}

\def\plotcoords{%
\addplot coordinates {
(5,8.312e-02)    (17,2.547e-02)   (49,7.407e-03)
(129,2.102e-03)  (321,5.874e-04)  (769,1.623e-04)
(1793,4.442e-05) (4097,1.207e-05) (9217,3.261e-06)
};

\addplot coordinates{
(7,8.472e-02)    (31,3.044e-02)    (111,1.022e-02)
(351,3.303e-03)  (1023,1.039e-03)  (2815,3.196e-04)
(7423,9.658e-05) (18943,2.873e-05) (47103,8.437e-06)
};

\addplot coordinates{
(9,7.881e-02)     (49,3.243e-02)    (209,1.232e-02)
(769,4.454e-03)   (2561,1.551e-03)  (7937,5.236e-04)
(23297,1.723e-04) (65537,5.545e-05) (178177,1.751e-05)
};

\addplot coordinates{
(11,6.887e-02)    (71,3.177e-02)     (351,1.341e-02)
(1471,5.334e-03)  (5503,2.027e-03)   (18943,7.415e-04)
(61183,2.628e-04) (187903,9.063e-05) (553983,3.053e-05)
};

\addplot coordinates{
(13,5.755e-02)     (97,2.925e-02)     (545,1.351e-02)
(2561,5.842e-03)   (10625,2.397e-03)  (40193,9.414e-04)
(141569,3.564e-04) (471041,1.308e-04)
(1496065,4.670e-05)
};
}%


\bookmaketitle
\tableofcontents
\include{pgfplots.title_abstract_intro}
\include{pgfplots.preliminaries}
\include{pgfplots.intro}
\include{pgfplots.reference}
\include{pgfplots.libs}
\include{pgfplots.resources}
\include{pgfplots.importexport}
\include{pgfplots.basic.reference}

\printindex

\bibliographystyle{abbrv} %gerapali} %gerabbrv} %gerunsrt.bst} %gerabbrv}% gerplain}
\nocite{pgfplotstable}
\nocite{programmingnotes}
\bibliography{pgfplots}
\end{document}

% -----------------------------------------------------------------------------
% For Stefan Pinnow as reminder on what to look for when editing the manual
% -----------------------------------------------------------------------------
% There should be no line breaks in the following environments
% - |...|
% - \declareandlabel{...}
% - \verbpdfref{...}
% If "MakeTikzPictures" isn't running through check one of the externalized
% LOG files what is the cause of that.
% -----------------------------------------------------------------------------


%%%%%%%%%%%%%%%%%%%%%%%%%%%%%%%%%%%%%%%%%%%%%%%%%%%%%%%%%%%%%%%%%%%%%%%%%%%%%
%
% Package pgfplots.sty documentation.
%
% Copyright 2007/2008 by Christian Feuersaenger.
%
% This program is free software: you can redistribute it and/or modify
% it under the terms of the GNU General Public License as published by
% the Free Software Foundation, either version 3 of the License, or
% (at your option) any later version.
%
% This program is distributed in the hope that it will be useful,
% but WITHOUT ANY WARRANTY; without even the implied warranty of
% MERCHANTABILITY or FITNESS FOR A PARTICULAR PURPOSE.  See the
% GNU General Public License for more details.
%
% You should have received a copy of the GNU General Public License
% along with this program.  If not, see <http://www.gnu.org/licenses/>.
%
%
%%%%%%%%%%%%%%%%%%%%%%%%%%%%%%%%%%%%%%%%%%%%%%%%%%%%%%%%%%%%%%%%%%%%%%%%%%%%%
% SEE pgfplots-macros.tex as well!
%\pdfminorversion=5 % to allow compression
%\pdfobjcompresslevel=2
\documentclass[a4paper,openany]{book}

\let\bookmaketitle=\maketitle

% -----------------------------------
% this here is from ltxdoc:
\usepackage{doc}[=v2]
\AtBeginDocument{\MakeShortVerb{\|}}
%\setlength{\textwidth}{355pt}
%\addtolength\marginparwidth{30pt}
%\addtolength\oddsidemargin{20pt}
%\addtolength\evensidemargin{20pt}
\def\cmd#1{\cs{\expandafter\cmd@to@cs\string#1}}
\def\cmd@to@cs#1#2{\char\number`#2\relax}
\DeclareRobustCommand\cs[1]{\texttt{\char`\\#1}}
\providecommand\marg[1]{%
  {\ttfamily\char`\{}\meta{#1}{\ttfamily\char`\}}}
\providecommand\oarg[1]{%
  {\ttfamily[}\meta{#1}{\ttfamily]}}
\providecommand\parg[1]{%
  {\ttfamily(}\meta{#1}{\ttfamily)}}
\raggedbottom

% -----------------------------------
\input{pgfplots.preamble.tex}

\makeatletter
% I want a two-column index just as in pgfmanual styles. This here
% was the best way to get one:
\def\index@prologue{\section*{Index}\addcontentsline{toc}{chapter}{Index}
}
\makeatother

%\RequirePackage[german,english,francais]{babel}

\def\matlabcolormaptext{This colormap is similar to one shipped with Matlab$^\text{\textregistered}$ under a similar name.}

\IfFileExists{tikzlibraryspy.code.tex}{%
\usetikzlibrary{spy}
}{%
    \message{ERROR: tikz SPY library NOT available. The manual will only compile partially.^^J}%
}%

\usepackage{xparse}% for colorbrewer manual

\usetikzlibrary{
    decorations.markings,
    decorations.footprints,
    shapes.arrows,
    matrix,
    positioning,
}

\usepgfplotslibrary{
    fillbetween,
    ternary,
    smithchart,
    patchplots,
    polar,
    colormaps,
    colorbrewer,
    colortol,           % docu not ready yet
}
\pgfqkeys{/codeexample}{%
    every codeexample/.append style={
        /pgfplots/every ternary axis/.append style={
            /pgfplots/legend style={fill=graphicbackground},
        }
    },
    tabsize=4,
}

\pgfplotsmanualenableexternalizationofexpensive

%\usetikzlibrary{external}
%\tikzexternalize[prefix=figures/]{pgfplots}

\title{%
    Manual for Package \PGFPlots{}\\
    {\small 2D/3D Plots in \LaTeX{}, Version \pgfplotsversion}\\
    {\small\url{https://github.com/pgf-tikz/pgfplots}}
    %\\{\small Attention: you are using an unstable development version.}
}

\makeatletter
\long\def\abstractsmuggle{%
    \centering
    \textbf{Abstract}\\[0.5cm]

    \begin{minipage}{12cm}
        \PGFPlots{} draws high-quality function plots in normal or logarithmic
        scaling with a user-friendly interface directly in \TeX{}. The user
        supplies axis labels, legend entries and the plot coordinates for one or
        more plots and \PGFPlots{} applies axis scaling, computes any logarithms
        and axis ticks and draws the plots. It supports line plots, scatter
        plots, piecewise constant plots, bar plots, area plots, mesh and surface
        plots, patch plots, contour plots, quiver plots, histogram plots, box
        plots, polar axes, ternary diagrams, smith charts and some more. It is
        based on Till Tantau's package \PGF{}/\Tikz{}.
    \end{minipage}

}%

\expandafter\date\expandafter{\@date\\[2cm]
    \abstractsmuggle
}%
\makeatother

%\includeonly{pgfplots.reference}


\begin{document}

\def\plotcoords{%
\addplot coordinates {
(5,8.312e-02)    (17,2.547e-02)   (49,7.407e-03)
(129,2.102e-03)  (321,5.874e-04)  (769,1.623e-04)
(1793,4.442e-05) (4097,1.207e-05) (9217,3.261e-06)
};

\addplot coordinates{
(7,8.472e-02)    (31,3.044e-02)    (111,1.022e-02)
(351,3.303e-03)  (1023,1.039e-03)  (2815,3.196e-04)
(7423,9.658e-05) (18943,2.873e-05) (47103,8.437e-06)
};

\addplot coordinates{
(9,7.881e-02)     (49,3.243e-02)    (209,1.232e-02)
(769,4.454e-03)   (2561,1.551e-03)  (7937,5.236e-04)
(23297,1.723e-04) (65537,5.545e-05) (178177,1.751e-05)
};

\addplot coordinates{
(11,6.887e-02)    (71,3.177e-02)     (351,1.341e-02)
(1471,5.334e-03)  (5503,2.027e-03)   (18943,7.415e-04)
(61183,2.628e-04) (187903,9.063e-05) (553983,3.053e-05)
};

\addplot coordinates{
(13,5.755e-02)     (97,2.925e-02)     (545,1.351e-02)
(2561,5.842e-03)   (10625,2.397e-03)  (40193,9.414e-04)
(141569,3.564e-04) (471041,1.308e-04)
(1496065,4.670e-05)
};
}%


\bookmaketitle
\tableofcontents
\include{pgfplots.title_abstract_intro}
\include{pgfplots.preliminaries}
\include{pgfplots.intro}
\include{pgfplots.reference}
\include{pgfplots.libs}
\include{pgfplots.resources}
\include{pgfplots.importexport}
\include{pgfplots.basic.reference}

\printindex

\bibliographystyle{abbrv} %gerapali} %gerabbrv} %gerunsrt.bst} %gerabbrv}% gerplain}
\nocite{pgfplotstable}
\nocite{programmingnotes}
\bibliography{pgfplots}
\end{document}

% -----------------------------------------------------------------------------
% For Stefan Pinnow as reminder on what to look for when editing the manual
% -----------------------------------------------------------------------------
% There should be no line breaks in the following environments
% - |...|
% - \declareandlabel{...}
% - \verbpdfref{...}
% If "MakeTikzPictures" isn't running through check one of the externalized
% LOG files what is the cause of that.
% -----------------------------------------------------------------------------


%%%%%%%%%%%%%%%%%%%%%%%%%%%%%%%%%%%%%%%%%%%%%%%%%%%%%%%%%%%%%%%%%%%%%%%%%%%%%
%
% Package pgfplots.sty documentation.
%
% Copyright 2007/2008 by Christian Feuersaenger.
%
% This program is free software: you can redistribute it and/or modify
% it under the terms of the GNU General Public License as published by
% the Free Software Foundation, either version 3 of the License, or
% (at your option) any later version.
%
% This program is distributed in the hope that it will be useful,
% but WITHOUT ANY WARRANTY; without even the implied warranty of
% MERCHANTABILITY or FITNESS FOR A PARTICULAR PURPOSE.  See the
% GNU General Public License for more details.
%
% You should have received a copy of the GNU General Public License
% along with this program.  If not, see <http://www.gnu.org/licenses/>.
%
%
%%%%%%%%%%%%%%%%%%%%%%%%%%%%%%%%%%%%%%%%%%%%%%%%%%%%%%%%%%%%%%%%%%%%%%%%%%%%%
% SEE pgfplots-macros.tex as well!
%\pdfminorversion=5 % to allow compression
%\pdfobjcompresslevel=2
\documentclass[a4paper,openany]{book}

\let\bookmaketitle=\maketitle

% -----------------------------------
% this here is from ltxdoc:
\usepackage{doc}[=v2]
\AtBeginDocument{\MakeShortVerb{\|}}
%\setlength{\textwidth}{355pt}
%\addtolength\marginparwidth{30pt}
%\addtolength\oddsidemargin{20pt}
%\addtolength\evensidemargin{20pt}
\def\cmd#1{\cs{\expandafter\cmd@to@cs\string#1}}
\def\cmd@to@cs#1#2{\char\number`#2\relax}
\DeclareRobustCommand\cs[1]{\texttt{\char`\\#1}}
\providecommand\marg[1]{%
  {\ttfamily\char`\{}\meta{#1}{\ttfamily\char`\}}}
\providecommand\oarg[1]{%
  {\ttfamily[}\meta{#1}{\ttfamily]}}
\providecommand\parg[1]{%
  {\ttfamily(}\meta{#1}{\ttfamily)}}
\raggedbottom

% -----------------------------------
\input{pgfplots.preamble.tex}

\makeatletter
% I want a two-column index just as in pgfmanual styles. This here
% was the best way to get one:
\def\index@prologue{\section*{Index}\addcontentsline{toc}{chapter}{Index}
}
\makeatother

%\RequirePackage[german,english,francais]{babel}

\def\matlabcolormaptext{This colormap is similar to one shipped with Matlab$^\text{\textregistered}$ under a similar name.}

\IfFileExists{tikzlibraryspy.code.tex}{%
\usetikzlibrary{spy}
}{%
    \message{ERROR: tikz SPY library NOT available. The manual will only compile partially.^^J}%
}%

\usepackage{xparse}% for colorbrewer manual

\usetikzlibrary{
    decorations.markings,
    decorations.footprints,
    shapes.arrows,
    matrix,
    positioning,
}

\usepgfplotslibrary{
    fillbetween,
    ternary,
    smithchart,
    patchplots,
    polar,
    colormaps,
    colorbrewer,
    colortol,           % docu not ready yet
}
\pgfqkeys{/codeexample}{%
    every codeexample/.append style={
        /pgfplots/every ternary axis/.append style={
            /pgfplots/legend style={fill=graphicbackground},
        }
    },
    tabsize=4,
}

\pgfplotsmanualenableexternalizationofexpensive

%\usetikzlibrary{external}
%\tikzexternalize[prefix=figures/]{pgfplots}

\title{%
    Manual for Package \PGFPlots{}\\
    {\small 2D/3D Plots in \LaTeX{}, Version \pgfplotsversion}\\
    {\small\url{https://github.com/pgf-tikz/pgfplots}}
    %\\{\small Attention: you are using an unstable development version.}
}

\makeatletter
\long\def\abstractsmuggle{%
    \centering
    \textbf{Abstract}\\[0.5cm]

    \begin{minipage}{12cm}
        \PGFPlots{} draws high-quality function plots in normal or logarithmic
        scaling with a user-friendly interface directly in \TeX{}. The user
        supplies axis labels, legend entries and the plot coordinates for one or
        more plots and \PGFPlots{} applies axis scaling, computes any logarithms
        and axis ticks and draws the plots. It supports line plots, scatter
        plots, piecewise constant plots, bar plots, area plots, mesh and surface
        plots, patch plots, contour plots, quiver plots, histogram plots, box
        plots, polar axes, ternary diagrams, smith charts and some more. It is
        based on Till Tantau's package \PGF{}/\Tikz{}.
    \end{minipage}

}%

\expandafter\date\expandafter{\@date\\[2cm]
    \abstractsmuggle
}%
\makeatother

%\includeonly{pgfplots.reference}


\begin{document}

\def\plotcoords{%
\addplot coordinates {
(5,8.312e-02)    (17,2.547e-02)   (49,7.407e-03)
(129,2.102e-03)  (321,5.874e-04)  (769,1.623e-04)
(1793,4.442e-05) (4097,1.207e-05) (9217,3.261e-06)
};

\addplot coordinates{
(7,8.472e-02)    (31,3.044e-02)    (111,1.022e-02)
(351,3.303e-03)  (1023,1.039e-03)  (2815,3.196e-04)
(7423,9.658e-05) (18943,2.873e-05) (47103,8.437e-06)
};

\addplot coordinates{
(9,7.881e-02)     (49,3.243e-02)    (209,1.232e-02)
(769,4.454e-03)   (2561,1.551e-03)  (7937,5.236e-04)
(23297,1.723e-04) (65537,5.545e-05) (178177,1.751e-05)
};

\addplot coordinates{
(11,6.887e-02)    (71,3.177e-02)     (351,1.341e-02)
(1471,5.334e-03)  (5503,2.027e-03)   (18943,7.415e-04)
(61183,2.628e-04) (187903,9.063e-05) (553983,3.053e-05)
};

\addplot coordinates{
(13,5.755e-02)     (97,2.925e-02)     (545,1.351e-02)
(2561,5.842e-03)   (10625,2.397e-03)  (40193,9.414e-04)
(141569,3.564e-04) (471041,1.308e-04)
(1496065,4.670e-05)
};
}%


\bookmaketitle
\tableofcontents
\include{pgfplots.title_abstract_intro}
\include{pgfplots.preliminaries}
\include{pgfplots.intro}
\include{pgfplots.reference}
\include{pgfplots.libs}
\include{pgfplots.resources}
\include{pgfplots.importexport}
\include{pgfplots.basic.reference}

\printindex

\bibliographystyle{abbrv} %gerapali} %gerabbrv} %gerunsrt.bst} %gerabbrv}% gerplain}
\nocite{pgfplotstable}
\nocite{programmingnotes}
\bibliography{pgfplots}
\end{document}

% -----------------------------------------------------------------------------
% For Stefan Pinnow as reminder on what to look for when editing the manual
% -----------------------------------------------------------------------------
% There should be no line breaks in the following environments
% - |...|
% - \declareandlabel{...}
% - \verbpdfref{...}
% If "MakeTikzPictures" isn't running through check one of the externalized
% LOG files what is the cause of that.
% -----------------------------------------------------------------------------

\include{pgfplots.basic.reference}

\printindex

\bibliographystyle{abbrv} %gerapali} %gerabbrv} %gerunsrt.bst} %gerabbrv}% gerplain}
\nocite{pgfplotstable}
\nocite{programmingnotes}
\bibliography{pgfplots}
\end{document}

% -----------------------------------------------------------------------------
% For Stefan Pinnow as reminder on what to look for when editing the manual
% -----------------------------------------------------------------------------
% There should be no line breaks in the following environments
% - |...|
% - \declareandlabel{...}
% - \verbpdfref{...}
% If "MakeTikzPictures" isn't running through check one of the externalized
% LOG files what is the cause of that.
% -----------------------------------------------------------------------------


%%%%%%%%%%%%%%%%%%%%%%%%%%%%%%%%%%%%%%%%%%%%%%%%%%%%%%%%%%%%%%%%%%%%%%%%%%%%%
%
% Package pgfplots.sty documentation.
%
% Copyright 2007/2008 by Christian Feuersaenger.
%
% This program is free software: you can redistribute it and/or modify
% it under the terms of the GNU General Public License as published by
% the Free Software Foundation, either version 3 of the License, or
% (at your option) any later version.
%
% This program is distributed in the hope that it will be useful,
% but WITHOUT ANY WARRANTY; without even the implied warranty of
% MERCHANTABILITY or FITNESS FOR A PARTICULAR PURPOSE.  See the
% GNU General Public License for more details.
%
% You should have received a copy of the GNU General Public License
% along with this program.  If not, see <http://www.gnu.org/licenses/>.
%
%
%%%%%%%%%%%%%%%%%%%%%%%%%%%%%%%%%%%%%%%%%%%%%%%%%%%%%%%%%%%%%%%%%%%%%%%%%%%%%
% SEE pgfplots-macros.tex as well!
%\pdfminorversion=5 % to allow compression
%\pdfobjcompresslevel=2
\documentclass[a4paper,openany]{book}

\let\bookmaketitle=\maketitle

% -----------------------------------
% this here is from ltxdoc:
\usepackage{doc}[=v2]
\AtBeginDocument{\MakeShortVerb{\|}}
%\setlength{\textwidth}{355pt}
%\addtolength\marginparwidth{30pt}
%\addtolength\oddsidemargin{20pt}
%\addtolength\evensidemargin{20pt}
\def\cmd#1{\cs{\expandafter\cmd@to@cs\string#1}}
\def\cmd@to@cs#1#2{\char\number`#2\relax}
\DeclareRobustCommand\cs[1]{\texttt{\char`\\#1}}
\providecommand\marg[1]{%
  {\ttfamily\char`\{}\meta{#1}{\ttfamily\char`\}}}
\providecommand\oarg[1]{%
  {\ttfamily[}\meta{#1}{\ttfamily]}}
\providecommand\parg[1]{%
  {\ttfamily(}\meta{#1}{\ttfamily)}}
\raggedbottom

% -----------------------------------
\input{pgfplots.preamble.tex}

\makeatletter
% I want a two-column index just as in pgfmanual styles. This here
% was the best way to get one:
\def\index@prologue{\section*{Index}\addcontentsline{toc}{chapter}{Index}
}
\makeatother

%\RequirePackage[german,english,francais]{babel}

\def\matlabcolormaptext{This colormap is similar to one shipped with Matlab$^\text{\textregistered}$ under a similar name.}

\IfFileExists{tikzlibraryspy.code.tex}{%
\usetikzlibrary{spy}
}{%
    \message{ERROR: tikz SPY library NOT available. The manual will only compile partially.^^J}%
}%

\usepackage{xparse}% for colorbrewer manual

\usetikzlibrary{
    decorations.markings,
    decorations.footprints,
    shapes.arrows,
    matrix,
    positioning,
}

\usepgfplotslibrary{
    fillbetween,
    ternary,
    smithchart,
    patchplots,
    polar,
    colormaps,
    colorbrewer,
    colortol,           % docu not ready yet
}
\pgfqkeys{/codeexample}{%
    every codeexample/.append style={
        /pgfplots/every ternary axis/.append style={
            /pgfplots/legend style={fill=graphicbackground},
        }
    },
    tabsize=4,
}

\pgfplotsmanualenableexternalizationofexpensive

%\usetikzlibrary{external}
%\tikzexternalize[prefix=figures/]{pgfplots}

\title{%
    Manual for Package \PGFPlots{}\\
    {\small 2D/3D Plots in \LaTeX{}, Version \pgfplotsversion}\\
    {\small\url{https://github.com/pgf-tikz/pgfplots}}
    %\\{\small Attention: you are using an unstable development version.}
}

\makeatletter
\long\def\abstractsmuggle{%
    \centering
    \textbf{Abstract}\\[0.5cm]

    \begin{minipage}{12cm}
        \PGFPlots{} draws high-quality function plots in normal or logarithmic
        scaling with a user-friendly interface directly in \TeX{}. The user
        supplies axis labels, legend entries and the plot coordinates for one or
        more plots and \PGFPlots{} applies axis scaling, computes any logarithms
        and axis ticks and draws the plots. It supports line plots, scatter
        plots, piecewise constant plots, bar plots, area plots, mesh and surface
        plots, patch plots, contour plots, quiver plots, histogram plots, box
        plots, polar axes, ternary diagrams, smith charts and some more. It is
        based on Till Tantau's package \PGF{}/\Tikz{}.
    \end{minipage}

}%

\expandafter\date\expandafter{\@date\\[2cm]
    \abstractsmuggle
}%
\makeatother

%\includeonly{pgfplots.reference}


\begin{document}

\def\plotcoords{%
\addplot coordinates {
(5,8.312e-02)    (17,2.547e-02)   (49,7.407e-03)
(129,2.102e-03)  (321,5.874e-04)  (769,1.623e-04)
(1793,4.442e-05) (4097,1.207e-05) (9217,3.261e-06)
};

\addplot coordinates{
(7,8.472e-02)    (31,3.044e-02)    (111,1.022e-02)
(351,3.303e-03)  (1023,1.039e-03)  (2815,3.196e-04)
(7423,9.658e-05) (18943,2.873e-05) (47103,8.437e-06)
};

\addplot coordinates{
(9,7.881e-02)     (49,3.243e-02)    (209,1.232e-02)
(769,4.454e-03)   (2561,1.551e-03)  (7937,5.236e-04)
(23297,1.723e-04) (65537,5.545e-05) (178177,1.751e-05)
};

\addplot coordinates{
(11,6.887e-02)    (71,3.177e-02)     (351,1.341e-02)
(1471,5.334e-03)  (5503,2.027e-03)   (18943,7.415e-04)
(61183,2.628e-04) (187903,9.063e-05) (553983,3.053e-05)
};

\addplot coordinates{
(13,5.755e-02)     (97,2.925e-02)     (545,1.351e-02)
(2561,5.842e-03)   (10625,2.397e-03)  (40193,9.414e-04)
(141569,3.564e-04) (471041,1.308e-04)
(1496065,4.670e-05)
};
}%


\bookmaketitle
\tableofcontents

%%%%%%%%%%%%%%%%%%%%%%%%%%%%%%%%%%%%%%%%%%%%%%%%%%%%%%%%%%%%%%%%%%%%%%%%%%%%%
%
% Package pgfplots.sty documentation.
%
% Copyright 2007/2008 by Christian Feuersaenger.
%
% This program is free software: you can redistribute it and/or modify
% it under the terms of the GNU General Public License as published by
% the Free Software Foundation, either version 3 of the License, or
% (at your option) any later version.
%
% This program is distributed in the hope that it will be useful,
% but WITHOUT ANY WARRANTY; without even the implied warranty of
% MERCHANTABILITY or FITNESS FOR A PARTICULAR PURPOSE.  See the
% GNU General Public License for more details.
%
% You should have received a copy of the GNU General Public License
% along with this program.  If not, see <http://www.gnu.org/licenses/>.
%
%
%%%%%%%%%%%%%%%%%%%%%%%%%%%%%%%%%%%%%%%%%%%%%%%%%%%%%%%%%%%%%%%%%%%%%%%%%%%%%
% SEE pgfplots-macros.tex as well!
%\pdfminorversion=5 % to allow compression
%\pdfobjcompresslevel=2
\documentclass[a4paper,openany]{book}

\let\bookmaketitle=\maketitle

% -----------------------------------
% this here is from ltxdoc:
\usepackage{doc}[=v2]
\AtBeginDocument{\MakeShortVerb{\|}}
%\setlength{\textwidth}{355pt}
%\addtolength\marginparwidth{30pt}
%\addtolength\oddsidemargin{20pt}
%\addtolength\evensidemargin{20pt}
\def\cmd#1{\cs{\expandafter\cmd@to@cs\string#1}}
\def\cmd@to@cs#1#2{\char\number`#2\relax}
\DeclareRobustCommand\cs[1]{\texttt{\char`\\#1}}
\providecommand\marg[1]{%
  {\ttfamily\char`\{}\meta{#1}{\ttfamily\char`\}}}
\providecommand\oarg[1]{%
  {\ttfamily[}\meta{#1}{\ttfamily]}}
\providecommand\parg[1]{%
  {\ttfamily(}\meta{#1}{\ttfamily)}}
\raggedbottom

% -----------------------------------
\input{pgfplots.preamble.tex}

\makeatletter
% I want a two-column index just as in pgfmanual styles. This here
% was the best way to get one:
\def\index@prologue{\section*{Index}\addcontentsline{toc}{chapter}{Index}
}
\makeatother

%\RequirePackage[german,english,francais]{babel}

\def\matlabcolormaptext{This colormap is similar to one shipped with Matlab$^\text{\textregistered}$ under a similar name.}

\IfFileExists{tikzlibraryspy.code.tex}{%
\usetikzlibrary{spy}
}{%
    \message{ERROR: tikz SPY library NOT available. The manual will only compile partially.^^J}%
}%

\usepackage{xparse}% for colorbrewer manual

\usetikzlibrary{
    decorations.markings,
    decorations.footprints,
    shapes.arrows,
    matrix,
    positioning,
}

\usepgfplotslibrary{
    fillbetween,
    ternary,
    smithchart,
    patchplots,
    polar,
    colormaps,
    colorbrewer,
    colortol,           % docu not ready yet
}
\pgfqkeys{/codeexample}{%
    every codeexample/.append style={
        /pgfplots/every ternary axis/.append style={
            /pgfplots/legend style={fill=graphicbackground},
        }
    },
    tabsize=4,
}

\pgfplotsmanualenableexternalizationofexpensive

%\usetikzlibrary{external}
%\tikzexternalize[prefix=figures/]{pgfplots}

\title{%
    Manual for Package \PGFPlots{}\\
    {\small 2D/3D Plots in \LaTeX{}, Version \pgfplotsversion}\\
    {\small\url{https://github.com/pgf-tikz/pgfplots}}
    %\\{\small Attention: you are using an unstable development version.}
}

\makeatletter
\long\def\abstractsmuggle{%
    \centering
    \textbf{Abstract}\\[0.5cm]

    \begin{minipage}{12cm}
        \PGFPlots{} draws high-quality function plots in normal or logarithmic
        scaling with a user-friendly interface directly in \TeX{}. The user
        supplies axis labels, legend entries and the plot coordinates for one or
        more plots and \PGFPlots{} applies axis scaling, computes any logarithms
        and axis ticks and draws the plots. It supports line plots, scatter
        plots, piecewise constant plots, bar plots, area plots, mesh and surface
        plots, patch plots, contour plots, quiver plots, histogram plots, box
        plots, polar axes, ternary diagrams, smith charts and some more. It is
        based on Till Tantau's package \PGF{}/\Tikz{}.
    \end{minipage}

}%

\expandafter\date\expandafter{\@date\\[2cm]
    \abstractsmuggle
}%
\makeatother

%\includeonly{pgfplots.reference}


\begin{document}

\def\plotcoords{%
\addplot coordinates {
(5,8.312e-02)    (17,2.547e-02)   (49,7.407e-03)
(129,2.102e-03)  (321,5.874e-04)  (769,1.623e-04)
(1793,4.442e-05) (4097,1.207e-05) (9217,3.261e-06)
};

\addplot coordinates{
(7,8.472e-02)    (31,3.044e-02)    (111,1.022e-02)
(351,3.303e-03)  (1023,1.039e-03)  (2815,3.196e-04)
(7423,9.658e-05) (18943,2.873e-05) (47103,8.437e-06)
};

\addplot coordinates{
(9,7.881e-02)     (49,3.243e-02)    (209,1.232e-02)
(769,4.454e-03)   (2561,1.551e-03)  (7937,5.236e-04)
(23297,1.723e-04) (65537,5.545e-05) (178177,1.751e-05)
};

\addplot coordinates{
(11,6.887e-02)    (71,3.177e-02)     (351,1.341e-02)
(1471,5.334e-03)  (5503,2.027e-03)   (18943,7.415e-04)
(61183,2.628e-04) (187903,9.063e-05) (553983,3.053e-05)
};

\addplot coordinates{
(13,5.755e-02)     (97,2.925e-02)     (545,1.351e-02)
(2561,5.842e-03)   (10625,2.397e-03)  (40193,9.414e-04)
(141569,3.564e-04) (471041,1.308e-04)
(1496065,4.670e-05)
};
}%


\bookmaketitle
\tableofcontents
\include{pgfplots.title_abstract_intro}
\include{pgfplots.preliminaries}
\include{pgfplots.intro}
\include{pgfplots.reference}
\include{pgfplots.libs}
\include{pgfplots.resources}
\include{pgfplots.importexport}
\include{pgfplots.basic.reference}

\printindex

\bibliographystyle{abbrv} %gerapali} %gerabbrv} %gerunsrt.bst} %gerabbrv}% gerplain}
\nocite{pgfplotstable}
\nocite{programmingnotes}
\bibliography{pgfplots}
\end{document}

% -----------------------------------------------------------------------------
% For Stefan Pinnow as reminder on what to look for when editing the manual
% -----------------------------------------------------------------------------
% There should be no line breaks in the following environments
% - |...|
% - \declareandlabel{...}
% - \verbpdfref{...}
% If "MakeTikzPictures" isn't running through check one of the externalized
% LOG files what is the cause of that.
% -----------------------------------------------------------------------------


%%%%%%%%%%%%%%%%%%%%%%%%%%%%%%%%%%%%%%%%%%%%%%%%%%%%%%%%%%%%%%%%%%%%%%%%%%%%%
%
% Package pgfplots.sty documentation.
%
% Copyright 2007/2008 by Christian Feuersaenger.
%
% This program is free software: you can redistribute it and/or modify
% it under the terms of the GNU General Public License as published by
% the Free Software Foundation, either version 3 of the License, or
% (at your option) any later version.
%
% This program is distributed in the hope that it will be useful,
% but WITHOUT ANY WARRANTY; without even the implied warranty of
% MERCHANTABILITY or FITNESS FOR A PARTICULAR PURPOSE.  See the
% GNU General Public License for more details.
%
% You should have received a copy of the GNU General Public License
% along with this program.  If not, see <http://www.gnu.org/licenses/>.
%
%
%%%%%%%%%%%%%%%%%%%%%%%%%%%%%%%%%%%%%%%%%%%%%%%%%%%%%%%%%%%%%%%%%%%%%%%%%%%%%
% SEE pgfplots-macros.tex as well!
%\pdfminorversion=5 % to allow compression
%\pdfobjcompresslevel=2
\documentclass[a4paper,openany]{book}

\let\bookmaketitle=\maketitle

% -----------------------------------
% this here is from ltxdoc:
\usepackage{doc}[=v2]
\AtBeginDocument{\MakeShortVerb{\|}}
%\setlength{\textwidth}{355pt}
%\addtolength\marginparwidth{30pt}
%\addtolength\oddsidemargin{20pt}
%\addtolength\evensidemargin{20pt}
\def\cmd#1{\cs{\expandafter\cmd@to@cs\string#1}}
\def\cmd@to@cs#1#2{\char\number`#2\relax}
\DeclareRobustCommand\cs[1]{\texttt{\char`\\#1}}
\providecommand\marg[1]{%
  {\ttfamily\char`\{}\meta{#1}{\ttfamily\char`\}}}
\providecommand\oarg[1]{%
  {\ttfamily[}\meta{#1}{\ttfamily]}}
\providecommand\parg[1]{%
  {\ttfamily(}\meta{#1}{\ttfamily)}}
\raggedbottom

% -----------------------------------
\input{pgfplots.preamble.tex}

\makeatletter
% I want a two-column index just as in pgfmanual styles. This here
% was the best way to get one:
\def\index@prologue{\section*{Index}\addcontentsline{toc}{chapter}{Index}
}
\makeatother

%\RequirePackage[german,english,francais]{babel}

\def\matlabcolormaptext{This colormap is similar to one shipped with Matlab$^\text{\textregistered}$ under a similar name.}

\IfFileExists{tikzlibraryspy.code.tex}{%
\usetikzlibrary{spy}
}{%
    \message{ERROR: tikz SPY library NOT available. The manual will only compile partially.^^J}%
}%

\usepackage{xparse}% for colorbrewer manual

\usetikzlibrary{
    decorations.markings,
    decorations.footprints,
    shapes.arrows,
    matrix,
    positioning,
}

\usepgfplotslibrary{
    fillbetween,
    ternary,
    smithchart,
    patchplots,
    polar,
    colormaps,
    colorbrewer,
    colortol,           % docu not ready yet
}
\pgfqkeys{/codeexample}{%
    every codeexample/.append style={
        /pgfplots/every ternary axis/.append style={
            /pgfplots/legend style={fill=graphicbackground},
        }
    },
    tabsize=4,
}

\pgfplotsmanualenableexternalizationofexpensive

%\usetikzlibrary{external}
%\tikzexternalize[prefix=figures/]{pgfplots}

\title{%
    Manual for Package \PGFPlots{}\\
    {\small 2D/3D Plots in \LaTeX{}, Version \pgfplotsversion}\\
    {\small\url{https://github.com/pgf-tikz/pgfplots}}
    %\\{\small Attention: you are using an unstable development version.}
}

\makeatletter
\long\def\abstractsmuggle{%
    \centering
    \textbf{Abstract}\\[0.5cm]

    \begin{minipage}{12cm}
        \PGFPlots{} draws high-quality function plots in normal or logarithmic
        scaling with a user-friendly interface directly in \TeX{}. The user
        supplies axis labels, legend entries and the plot coordinates for one or
        more plots and \PGFPlots{} applies axis scaling, computes any logarithms
        and axis ticks and draws the plots. It supports line plots, scatter
        plots, piecewise constant plots, bar plots, area plots, mesh and surface
        plots, patch plots, contour plots, quiver plots, histogram plots, box
        plots, polar axes, ternary diagrams, smith charts and some more. It is
        based on Till Tantau's package \PGF{}/\Tikz{}.
    \end{minipage}

}%

\expandafter\date\expandafter{\@date\\[2cm]
    \abstractsmuggle
}%
\makeatother

%\includeonly{pgfplots.reference}


\begin{document}

\def\plotcoords{%
\addplot coordinates {
(5,8.312e-02)    (17,2.547e-02)   (49,7.407e-03)
(129,2.102e-03)  (321,5.874e-04)  (769,1.623e-04)
(1793,4.442e-05) (4097,1.207e-05) (9217,3.261e-06)
};

\addplot coordinates{
(7,8.472e-02)    (31,3.044e-02)    (111,1.022e-02)
(351,3.303e-03)  (1023,1.039e-03)  (2815,3.196e-04)
(7423,9.658e-05) (18943,2.873e-05) (47103,8.437e-06)
};

\addplot coordinates{
(9,7.881e-02)     (49,3.243e-02)    (209,1.232e-02)
(769,4.454e-03)   (2561,1.551e-03)  (7937,5.236e-04)
(23297,1.723e-04) (65537,5.545e-05) (178177,1.751e-05)
};

\addplot coordinates{
(11,6.887e-02)    (71,3.177e-02)     (351,1.341e-02)
(1471,5.334e-03)  (5503,2.027e-03)   (18943,7.415e-04)
(61183,2.628e-04) (187903,9.063e-05) (553983,3.053e-05)
};

\addplot coordinates{
(13,5.755e-02)     (97,2.925e-02)     (545,1.351e-02)
(2561,5.842e-03)   (10625,2.397e-03)  (40193,9.414e-04)
(141569,3.564e-04) (471041,1.308e-04)
(1496065,4.670e-05)
};
}%


\bookmaketitle
\tableofcontents
\include{pgfplots.title_abstract_intro}
\include{pgfplots.preliminaries}
\include{pgfplots.intro}
\include{pgfplots.reference}
\include{pgfplots.libs}
\include{pgfplots.resources}
\include{pgfplots.importexport}
\include{pgfplots.basic.reference}

\printindex

\bibliographystyle{abbrv} %gerapali} %gerabbrv} %gerunsrt.bst} %gerabbrv}% gerplain}
\nocite{pgfplotstable}
\nocite{programmingnotes}
\bibliography{pgfplots}
\end{document}

% -----------------------------------------------------------------------------
% For Stefan Pinnow as reminder on what to look for when editing the manual
% -----------------------------------------------------------------------------
% There should be no line breaks in the following environments
% - |...|
% - \declareandlabel{...}
% - \verbpdfref{...}
% If "MakeTikzPictures" isn't running through check one of the externalized
% LOG files what is the cause of that.
% -----------------------------------------------------------------------------


%%%%%%%%%%%%%%%%%%%%%%%%%%%%%%%%%%%%%%%%%%%%%%%%%%%%%%%%%%%%%%%%%%%%%%%%%%%%%
%
% Package pgfplots.sty documentation.
%
% Copyright 2007/2008 by Christian Feuersaenger.
%
% This program is free software: you can redistribute it and/or modify
% it under the terms of the GNU General Public License as published by
% the Free Software Foundation, either version 3 of the License, or
% (at your option) any later version.
%
% This program is distributed in the hope that it will be useful,
% but WITHOUT ANY WARRANTY; without even the implied warranty of
% MERCHANTABILITY or FITNESS FOR A PARTICULAR PURPOSE.  See the
% GNU General Public License for more details.
%
% You should have received a copy of the GNU General Public License
% along with this program.  If not, see <http://www.gnu.org/licenses/>.
%
%
%%%%%%%%%%%%%%%%%%%%%%%%%%%%%%%%%%%%%%%%%%%%%%%%%%%%%%%%%%%%%%%%%%%%%%%%%%%%%
% SEE pgfplots-macros.tex as well!
%\pdfminorversion=5 % to allow compression
%\pdfobjcompresslevel=2
\documentclass[a4paper,openany]{book}

\let\bookmaketitle=\maketitle

% -----------------------------------
% this here is from ltxdoc:
\usepackage{doc}[=v2]
\AtBeginDocument{\MakeShortVerb{\|}}
%\setlength{\textwidth}{355pt}
%\addtolength\marginparwidth{30pt}
%\addtolength\oddsidemargin{20pt}
%\addtolength\evensidemargin{20pt}
\def\cmd#1{\cs{\expandafter\cmd@to@cs\string#1}}
\def\cmd@to@cs#1#2{\char\number`#2\relax}
\DeclareRobustCommand\cs[1]{\texttt{\char`\\#1}}
\providecommand\marg[1]{%
  {\ttfamily\char`\{}\meta{#1}{\ttfamily\char`\}}}
\providecommand\oarg[1]{%
  {\ttfamily[}\meta{#1}{\ttfamily]}}
\providecommand\parg[1]{%
  {\ttfamily(}\meta{#1}{\ttfamily)}}
\raggedbottom

% -----------------------------------
\input{pgfplots.preamble.tex}

\makeatletter
% I want a two-column index just as in pgfmanual styles. This here
% was the best way to get one:
\def\index@prologue{\section*{Index}\addcontentsline{toc}{chapter}{Index}
}
\makeatother

%\RequirePackage[german,english,francais]{babel}

\def\matlabcolormaptext{This colormap is similar to one shipped with Matlab$^\text{\textregistered}$ under a similar name.}

\IfFileExists{tikzlibraryspy.code.tex}{%
\usetikzlibrary{spy}
}{%
    \message{ERROR: tikz SPY library NOT available. The manual will only compile partially.^^J}%
}%

\usepackage{xparse}% for colorbrewer manual

\usetikzlibrary{
    decorations.markings,
    decorations.footprints,
    shapes.arrows,
    matrix,
    positioning,
}

\usepgfplotslibrary{
    fillbetween,
    ternary,
    smithchart,
    patchplots,
    polar,
    colormaps,
    colorbrewer,
    colortol,           % docu not ready yet
}
\pgfqkeys{/codeexample}{%
    every codeexample/.append style={
        /pgfplots/every ternary axis/.append style={
            /pgfplots/legend style={fill=graphicbackground},
        }
    },
    tabsize=4,
}

\pgfplotsmanualenableexternalizationofexpensive

%\usetikzlibrary{external}
%\tikzexternalize[prefix=figures/]{pgfplots}

\title{%
    Manual for Package \PGFPlots{}\\
    {\small 2D/3D Plots in \LaTeX{}, Version \pgfplotsversion}\\
    {\small\url{https://github.com/pgf-tikz/pgfplots}}
    %\\{\small Attention: you are using an unstable development version.}
}

\makeatletter
\long\def\abstractsmuggle{%
    \centering
    \textbf{Abstract}\\[0.5cm]

    \begin{minipage}{12cm}
        \PGFPlots{} draws high-quality function plots in normal or logarithmic
        scaling with a user-friendly interface directly in \TeX{}. The user
        supplies axis labels, legend entries and the plot coordinates for one or
        more plots and \PGFPlots{} applies axis scaling, computes any logarithms
        and axis ticks and draws the plots. It supports line plots, scatter
        plots, piecewise constant plots, bar plots, area plots, mesh and surface
        plots, patch plots, contour plots, quiver plots, histogram plots, box
        plots, polar axes, ternary diagrams, smith charts and some more. It is
        based on Till Tantau's package \PGF{}/\Tikz{}.
    \end{minipage}

}%

\expandafter\date\expandafter{\@date\\[2cm]
    \abstractsmuggle
}%
\makeatother

%\includeonly{pgfplots.reference}


\begin{document}

\def\plotcoords{%
\addplot coordinates {
(5,8.312e-02)    (17,2.547e-02)   (49,7.407e-03)
(129,2.102e-03)  (321,5.874e-04)  (769,1.623e-04)
(1793,4.442e-05) (4097,1.207e-05) (9217,3.261e-06)
};

\addplot coordinates{
(7,8.472e-02)    (31,3.044e-02)    (111,1.022e-02)
(351,3.303e-03)  (1023,1.039e-03)  (2815,3.196e-04)
(7423,9.658e-05) (18943,2.873e-05) (47103,8.437e-06)
};

\addplot coordinates{
(9,7.881e-02)     (49,3.243e-02)    (209,1.232e-02)
(769,4.454e-03)   (2561,1.551e-03)  (7937,5.236e-04)
(23297,1.723e-04) (65537,5.545e-05) (178177,1.751e-05)
};

\addplot coordinates{
(11,6.887e-02)    (71,3.177e-02)     (351,1.341e-02)
(1471,5.334e-03)  (5503,2.027e-03)   (18943,7.415e-04)
(61183,2.628e-04) (187903,9.063e-05) (553983,3.053e-05)
};

\addplot coordinates{
(13,5.755e-02)     (97,2.925e-02)     (545,1.351e-02)
(2561,5.842e-03)   (10625,2.397e-03)  (40193,9.414e-04)
(141569,3.564e-04) (471041,1.308e-04)
(1496065,4.670e-05)
};
}%


\bookmaketitle
\tableofcontents
\include{pgfplots.title_abstract_intro}
\include{pgfplots.preliminaries}
\include{pgfplots.intro}
\include{pgfplots.reference}
\include{pgfplots.libs}
\include{pgfplots.resources}
\include{pgfplots.importexport}
\include{pgfplots.basic.reference}

\printindex

\bibliographystyle{abbrv} %gerapali} %gerabbrv} %gerunsrt.bst} %gerabbrv}% gerplain}
\nocite{pgfplotstable}
\nocite{programmingnotes}
\bibliography{pgfplots}
\end{document}

% -----------------------------------------------------------------------------
% For Stefan Pinnow as reminder on what to look for when editing the manual
% -----------------------------------------------------------------------------
% There should be no line breaks in the following environments
% - |...|
% - \declareandlabel{...}
% - \verbpdfref{...}
% If "MakeTikzPictures" isn't running through check one of the externalized
% LOG files what is the cause of that.
% -----------------------------------------------------------------------------


%%%%%%%%%%%%%%%%%%%%%%%%%%%%%%%%%%%%%%%%%%%%%%%%%%%%%%%%%%%%%%%%%%%%%%%%%%%%%
%
% Package pgfplots.sty documentation.
%
% Copyright 2007/2008 by Christian Feuersaenger.
%
% This program is free software: you can redistribute it and/or modify
% it under the terms of the GNU General Public License as published by
% the Free Software Foundation, either version 3 of the License, or
% (at your option) any later version.
%
% This program is distributed in the hope that it will be useful,
% but WITHOUT ANY WARRANTY; without even the implied warranty of
% MERCHANTABILITY or FITNESS FOR A PARTICULAR PURPOSE.  See the
% GNU General Public License for more details.
%
% You should have received a copy of the GNU General Public License
% along with this program.  If not, see <http://www.gnu.org/licenses/>.
%
%
%%%%%%%%%%%%%%%%%%%%%%%%%%%%%%%%%%%%%%%%%%%%%%%%%%%%%%%%%%%%%%%%%%%%%%%%%%%%%
% SEE pgfplots-macros.tex as well!
%\pdfminorversion=5 % to allow compression
%\pdfobjcompresslevel=2
\documentclass[a4paper,openany]{book}

\let\bookmaketitle=\maketitle

% -----------------------------------
% this here is from ltxdoc:
\usepackage{doc}[=v2]
\AtBeginDocument{\MakeShortVerb{\|}}
%\setlength{\textwidth}{355pt}
%\addtolength\marginparwidth{30pt}
%\addtolength\oddsidemargin{20pt}
%\addtolength\evensidemargin{20pt}
\def\cmd#1{\cs{\expandafter\cmd@to@cs\string#1}}
\def\cmd@to@cs#1#2{\char\number`#2\relax}
\DeclareRobustCommand\cs[1]{\texttt{\char`\\#1}}
\providecommand\marg[1]{%
  {\ttfamily\char`\{}\meta{#1}{\ttfamily\char`\}}}
\providecommand\oarg[1]{%
  {\ttfamily[}\meta{#1}{\ttfamily]}}
\providecommand\parg[1]{%
  {\ttfamily(}\meta{#1}{\ttfamily)}}
\raggedbottom

% -----------------------------------
\input{pgfplots.preamble.tex}

\makeatletter
% I want a two-column index just as in pgfmanual styles. This here
% was the best way to get one:
\def\index@prologue{\section*{Index}\addcontentsline{toc}{chapter}{Index}
}
\makeatother

%\RequirePackage[german,english,francais]{babel}

\def\matlabcolormaptext{This colormap is similar to one shipped with Matlab$^\text{\textregistered}$ under a similar name.}

\IfFileExists{tikzlibraryspy.code.tex}{%
\usetikzlibrary{spy}
}{%
    \message{ERROR: tikz SPY library NOT available. The manual will only compile partially.^^J}%
}%

\usepackage{xparse}% for colorbrewer manual

\usetikzlibrary{
    decorations.markings,
    decorations.footprints,
    shapes.arrows,
    matrix,
    positioning,
}

\usepgfplotslibrary{
    fillbetween,
    ternary,
    smithchart,
    patchplots,
    polar,
    colormaps,
    colorbrewer,
    colortol,           % docu not ready yet
}
\pgfqkeys{/codeexample}{%
    every codeexample/.append style={
        /pgfplots/every ternary axis/.append style={
            /pgfplots/legend style={fill=graphicbackground},
        }
    },
    tabsize=4,
}

\pgfplotsmanualenableexternalizationofexpensive

%\usetikzlibrary{external}
%\tikzexternalize[prefix=figures/]{pgfplots}

\title{%
    Manual for Package \PGFPlots{}\\
    {\small 2D/3D Plots in \LaTeX{}, Version \pgfplotsversion}\\
    {\small\url{https://github.com/pgf-tikz/pgfplots}}
    %\\{\small Attention: you are using an unstable development version.}
}

\makeatletter
\long\def\abstractsmuggle{%
    \centering
    \textbf{Abstract}\\[0.5cm]

    \begin{minipage}{12cm}
        \PGFPlots{} draws high-quality function plots in normal or logarithmic
        scaling with a user-friendly interface directly in \TeX{}. The user
        supplies axis labels, legend entries and the plot coordinates for one or
        more plots and \PGFPlots{} applies axis scaling, computes any logarithms
        and axis ticks and draws the plots. It supports line plots, scatter
        plots, piecewise constant plots, bar plots, area plots, mesh and surface
        plots, patch plots, contour plots, quiver plots, histogram plots, box
        plots, polar axes, ternary diagrams, smith charts and some more. It is
        based on Till Tantau's package \PGF{}/\Tikz{}.
    \end{minipage}

}%

\expandafter\date\expandafter{\@date\\[2cm]
    \abstractsmuggle
}%
\makeatother

%\includeonly{pgfplots.reference}


\begin{document}

\def\plotcoords{%
\addplot coordinates {
(5,8.312e-02)    (17,2.547e-02)   (49,7.407e-03)
(129,2.102e-03)  (321,5.874e-04)  (769,1.623e-04)
(1793,4.442e-05) (4097,1.207e-05) (9217,3.261e-06)
};

\addplot coordinates{
(7,8.472e-02)    (31,3.044e-02)    (111,1.022e-02)
(351,3.303e-03)  (1023,1.039e-03)  (2815,3.196e-04)
(7423,9.658e-05) (18943,2.873e-05) (47103,8.437e-06)
};

\addplot coordinates{
(9,7.881e-02)     (49,3.243e-02)    (209,1.232e-02)
(769,4.454e-03)   (2561,1.551e-03)  (7937,5.236e-04)
(23297,1.723e-04) (65537,5.545e-05) (178177,1.751e-05)
};

\addplot coordinates{
(11,6.887e-02)    (71,3.177e-02)     (351,1.341e-02)
(1471,5.334e-03)  (5503,2.027e-03)   (18943,7.415e-04)
(61183,2.628e-04) (187903,9.063e-05) (553983,3.053e-05)
};

\addplot coordinates{
(13,5.755e-02)     (97,2.925e-02)     (545,1.351e-02)
(2561,5.842e-03)   (10625,2.397e-03)  (40193,9.414e-04)
(141569,3.564e-04) (471041,1.308e-04)
(1496065,4.670e-05)
};
}%


\bookmaketitle
\tableofcontents
\include{pgfplots.title_abstract_intro}
\include{pgfplots.preliminaries}
\include{pgfplots.intro}
\include{pgfplots.reference}
\include{pgfplots.libs}
\include{pgfplots.resources}
\include{pgfplots.importexport}
\include{pgfplots.basic.reference}

\printindex

\bibliographystyle{abbrv} %gerapali} %gerabbrv} %gerunsrt.bst} %gerabbrv}% gerplain}
\nocite{pgfplotstable}
\nocite{programmingnotes}
\bibliography{pgfplots}
\end{document}

% -----------------------------------------------------------------------------
% For Stefan Pinnow as reminder on what to look for when editing the manual
% -----------------------------------------------------------------------------
% There should be no line breaks in the following environments
% - |...|
% - \declareandlabel{...}
% - \verbpdfref{...}
% If "MakeTikzPictures" isn't running through check one of the externalized
% LOG files what is the cause of that.
% -----------------------------------------------------------------------------


%%%%%%%%%%%%%%%%%%%%%%%%%%%%%%%%%%%%%%%%%%%%%%%%%%%%%%%%%%%%%%%%%%%%%%%%%%%%%
%
% Package pgfplots.sty documentation.
%
% Copyright 2007/2008 by Christian Feuersaenger.
%
% This program is free software: you can redistribute it and/or modify
% it under the terms of the GNU General Public License as published by
% the Free Software Foundation, either version 3 of the License, or
% (at your option) any later version.
%
% This program is distributed in the hope that it will be useful,
% but WITHOUT ANY WARRANTY; without even the implied warranty of
% MERCHANTABILITY or FITNESS FOR A PARTICULAR PURPOSE.  See the
% GNU General Public License for more details.
%
% You should have received a copy of the GNU General Public License
% along with this program.  If not, see <http://www.gnu.org/licenses/>.
%
%
%%%%%%%%%%%%%%%%%%%%%%%%%%%%%%%%%%%%%%%%%%%%%%%%%%%%%%%%%%%%%%%%%%%%%%%%%%%%%
% SEE pgfplots-macros.tex as well!
%\pdfminorversion=5 % to allow compression
%\pdfobjcompresslevel=2
\documentclass[a4paper,openany]{book}

\let\bookmaketitle=\maketitle

% -----------------------------------
% this here is from ltxdoc:
\usepackage{doc}[=v2]
\AtBeginDocument{\MakeShortVerb{\|}}
%\setlength{\textwidth}{355pt}
%\addtolength\marginparwidth{30pt}
%\addtolength\oddsidemargin{20pt}
%\addtolength\evensidemargin{20pt}
\def\cmd#1{\cs{\expandafter\cmd@to@cs\string#1}}
\def\cmd@to@cs#1#2{\char\number`#2\relax}
\DeclareRobustCommand\cs[1]{\texttt{\char`\\#1}}
\providecommand\marg[1]{%
  {\ttfamily\char`\{}\meta{#1}{\ttfamily\char`\}}}
\providecommand\oarg[1]{%
  {\ttfamily[}\meta{#1}{\ttfamily]}}
\providecommand\parg[1]{%
  {\ttfamily(}\meta{#1}{\ttfamily)}}
\raggedbottom

% -----------------------------------
\input{pgfplots.preamble.tex}

\makeatletter
% I want a two-column index just as in pgfmanual styles. This here
% was the best way to get one:
\def\index@prologue{\section*{Index}\addcontentsline{toc}{chapter}{Index}
}
\makeatother

%\RequirePackage[german,english,francais]{babel}

\def\matlabcolormaptext{This colormap is similar to one shipped with Matlab$^\text{\textregistered}$ under a similar name.}

\IfFileExists{tikzlibraryspy.code.tex}{%
\usetikzlibrary{spy}
}{%
    \message{ERROR: tikz SPY library NOT available. The manual will only compile partially.^^J}%
}%

\usepackage{xparse}% for colorbrewer manual

\usetikzlibrary{
    decorations.markings,
    decorations.footprints,
    shapes.arrows,
    matrix,
    positioning,
}

\usepgfplotslibrary{
    fillbetween,
    ternary,
    smithchart,
    patchplots,
    polar,
    colormaps,
    colorbrewer,
    colortol,           % docu not ready yet
}
\pgfqkeys{/codeexample}{%
    every codeexample/.append style={
        /pgfplots/every ternary axis/.append style={
            /pgfplots/legend style={fill=graphicbackground},
        }
    },
    tabsize=4,
}

\pgfplotsmanualenableexternalizationofexpensive

%\usetikzlibrary{external}
%\tikzexternalize[prefix=figures/]{pgfplots}

\title{%
    Manual for Package \PGFPlots{}\\
    {\small 2D/3D Plots in \LaTeX{}, Version \pgfplotsversion}\\
    {\small\url{https://github.com/pgf-tikz/pgfplots}}
    %\\{\small Attention: you are using an unstable development version.}
}

\makeatletter
\long\def\abstractsmuggle{%
    \centering
    \textbf{Abstract}\\[0.5cm]

    \begin{minipage}{12cm}
        \PGFPlots{} draws high-quality function plots in normal or logarithmic
        scaling with a user-friendly interface directly in \TeX{}. The user
        supplies axis labels, legend entries and the plot coordinates for one or
        more plots and \PGFPlots{} applies axis scaling, computes any logarithms
        and axis ticks and draws the plots. It supports line plots, scatter
        plots, piecewise constant plots, bar plots, area plots, mesh and surface
        plots, patch plots, contour plots, quiver plots, histogram plots, box
        plots, polar axes, ternary diagrams, smith charts and some more. It is
        based on Till Tantau's package \PGF{}/\Tikz{}.
    \end{minipage}

}%

\expandafter\date\expandafter{\@date\\[2cm]
    \abstractsmuggle
}%
\makeatother

%\includeonly{pgfplots.reference}


\begin{document}

\def\plotcoords{%
\addplot coordinates {
(5,8.312e-02)    (17,2.547e-02)   (49,7.407e-03)
(129,2.102e-03)  (321,5.874e-04)  (769,1.623e-04)
(1793,4.442e-05) (4097,1.207e-05) (9217,3.261e-06)
};

\addplot coordinates{
(7,8.472e-02)    (31,3.044e-02)    (111,1.022e-02)
(351,3.303e-03)  (1023,1.039e-03)  (2815,3.196e-04)
(7423,9.658e-05) (18943,2.873e-05) (47103,8.437e-06)
};

\addplot coordinates{
(9,7.881e-02)     (49,3.243e-02)    (209,1.232e-02)
(769,4.454e-03)   (2561,1.551e-03)  (7937,5.236e-04)
(23297,1.723e-04) (65537,5.545e-05) (178177,1.751e-05)
};

\addplot coordinates{
(11,6.887e-02)    (71,3.177e-02)     (351,1.341e-02)
(1471,5.334e-03)  (5503,2.027e-03)   (18943,7.415e-04)
(61183,2.628e-04) (187903,9.063e-05) (553983,3.053e-05)
};

\addplot coordinates{
(13,5.755e-02)     (97,2.925e-02)     (545,1.351e-02)
(2561,5.842e-03)   (10625,2.397e-03)  (40193,9.414e-04)
(141569,3.564e-04) (471041,1.308e-04)
(1496065,4.670e-05)
};
}%


\bookmaketitle
\tableofcontents
\include{pgfplots.title_abstract_intro}
\include{pgfplots.preliminaries}
\include{pgfplots.intro}
\include{pgfplots.reference}
\include{pgfplots.libs}
\include{pgfplots.resources}
\include{pgfplots.importexport}
\include{pgfplots.basic.reference}

\printindex

\bibliographystyle{abbrv} %gerapali} %gerabbrv} %gerunsrt.bst} %gerabbrv}% gerplain}
\nocite{pgfplotstable}
\nocite{programmingnotes}
\bibliography{pgfplots}
\end{document}

% -----------------------------------------------------------------------------
% For Stefan Pinnow as reminder on what to look for when editing the manual
% -----------------------------------------------------------------------------
% There should be no line breaks in the following environments
% - |...|
% - \declareandlabel{...}
% - \verbpdfref{...}
% If "MakeTikzPictures" isn't running through check one of the externalized
% LOG files what is the cause of that.
% -----------------------------------------------------------------------------


%%%%%%%%%%%%%%%%%%%%%%%%%%%%%%%%%%%%%%%%%%%%%%%%%%%%%%%%%%%%%%%%%%%%%%%%%%%%%
%
% Package pgfplots.sty documentation.
%
% Copyright 2007/2008 by Christian Feuersaenger.
%
% This program is free software: you can redistribute it and/or modify
% it under the terms of the GNU General Public License as published by
% the Free Software Foundation, either version 3 of the License, or
% (at your option) any later version.
%
% This program is distributed in the hope that it will be useful,
% but WITHOUT ANY WARRANTY; without even the implied warranty of
% MERCHANTABILITY or FITNESS FOR A PARTICULAR PURPOSE.  See the
% GNU General Public License for more details.
%
% You should have received a copy of the GNU General Public License
% along with this program.  If not, see <http://www.gnu.org/licenses/>.
%
%
%%%%%%%%%%%%%%%%%%%%%%%%%%%%%%%%%%%%%%%%%%%%%%%%%%%%%%%%%%%%%%%%%%%%%%%%%%%%%
% SEE pgfplots-macros.tex as well!
%\pdfminorversion=5 % to allow compression
%\pdfobjcompresslevel=2
\documentclass[a4paper,openany]{book}

\let\bookmaketitle=\maketitle

% -----------------------------------
% this here is from ltxdoc:
\usepackage{doc}[=v2]
\AtBeginDocument{\MakeShortVerb{\|}}
%\setlength{\textwidth}{355pt}
%\addtolength\marginparwidth{30pt}
%\addtolength\oddsidemargin{20pt}
%\addtolength\evensidemargin{20pt}
\def\cmd#1{\cs{\expandafter\cmd@to@cs\string#1}}
\def\cmd@to@cs#1#2{\char\number`#2\relax}
\DeclareRobustCommand\cs[1]{\texttt{\char`\\#1}}
\providecommand\marg[1]{%
  {\ttfamily\char`\{}\meta{#1}{\ttfamily\char`\}}}
\providecommand\oarg[1]{%
  {\ttfamily[}\meta{#1}{\ttfamily]}}
\providecommand\parg[1]{%
  {\ttfamily(}\meta{#1}{\ttfamily)}}
\raggedbottom

% -----------------------------------
\input{pgfplots.preamble.tex}

\makeatletter
% I want a two-column index just as in pgfmanual styles. This here
% was the best way to get one:
\def\index@prologue{\section*{Index}\addcontentsline{toc}{chapter}{Index}
}
\makeatother

%\RequirePackage[german,english,francais]{babel}

\def\matlabcolormaptext{This colormap is similar to one shipped with Matlab$^\text{\textregistered}$ under a similar name.}

\IfFileExists{tikzlibraryspy.code.tex}{%
\usetikzlibrary{spy}
}{%
    \message{ERROR: tikz SPY library NOT available. The manual will only compile partially.^^J}%
}%

\usepackage{xparse}% for colorbrewer manual

\usetikzlibrary{
    decorations.markings,
    decorations.footprints,
    shapes.arrows,
    matrix,
    positioning,
}

\usepgfplotslibrary{
    fillbetween,
    ternary,
    smithchart,
    patchplots,
    polar,
    colormaps,
    colorbrewer,
    colortol,           % docu not ready yet
}
\pgfqkeys{/codeexample}{%
    every codeexample/.append style={
        /pgfplots/every ternary axis/.append style={
            /pgfplots/legend style={fill=graphicbackground},
        }
    },
    tabsize=4,
}

\pgfplotsmanualenableexternalizationofexpensive

%\usetikzlibrary{external}
%\tikzexternalize[prefix=figures/]{pgfplots}

\title{%
    Manual for Package \PGFPlots{}\\
    {\small 2D/3D Plots in \LaTeX{}, Version \pgfplotsversion}\\
    {\small\url{https://github.com/pgf-tikz/pgfplots}}
    %\\{\small Attention: you are using an unstable development version.}
}

\makeatletter
\long\def\abstractsmuggle{%
    \centering
    \textbf{Abstract}\\[0.5cm]

    \begin{minipage}{12cm}
        \PGFPlots{} draws high-quality function plots in normal or logarithmic
        scaling with a user-friendly interface directly in \TeX{}. The user
        supplies axis labels, legend entries and the plot coordinates for one or
        more plots and \PGFPlots{} applies axis scaling, computes any logarithms
        and axis ticks and draws the plots. It supports line plots, scatter
        plots, piecewise constant plots, bar plots, area plots, mesh and surface
        plots, patch plots, contour plots, quiver plots, histogram plots, box
        plots, polar axes, ternary diagrams, smith charts and some more. It is
        based on Till Tantau's package \PGF{}/\Tikz{}.
    \end{minipage}

}%

\expandafter\date\expandafter{\@date\\[2cm]
    \abstractsmuggle
}%
\makeatother

%\includeonly{pgfplots.reference}


\begin{document}

\def\plotcoords{%
\addplot coordinates {
(5,8.312e-02)    (17,2.547e-02)   (49,7.407e-03)
(129,2.102e-03)  (321,5.874e-04)  (769,1.623e-04)
(1793,4.442e-05) (4097,1.207e-05) (9217,3.261e-06)
};

\addplot coordinates{
(7,8.472e-02)    (31,3.044e-02)    (111,1.022e-02)
(351,3.303e-03)  (1023,1.039e-03)  (2815,3.196e-04)
(7423,9.658e-05) (18943,2.873e-05) (47103,8.437e-06)
};

\addplot coordinates{
(9,7.881e-02)     (49,3.243e-02)    (209,1.232e-02)
(769,4.454e-03)   (2561,1.551e-03)  (7937,5.236e-04)
(23297,1.723e-04) (65537,5.545e-05) (178177,1.751e-05)
};

\addplot coordinates{
(11,6.887e-02)    (71,3.177e-02)     (351,1.341e-02)
(1471,5.334e-03)  (5503,2.027e-03)   (18943,7.415e-04)
(61183,2.628e-04) (187903,9.063e-05) (553983,3.053e-05)
};

\addplot coordinates{
(13,5.755e-02)     (97,2.925e-02)     (545,1.351e-02)
(2561,5.842e-03)   (10625,2.397e-03)  (40193,9.414e-04)
(141569,3.564e-04) (471041,1.308e-04)
(1496065,4.670e-05)
};
}%


\bookmaketitle
\tableofcontents
\include{pgfplots.title_abstract_intro}
\include{pgfplots.preliminaries}
\include{pgfplots.intro}
\include{pgfplots.reference}
\include{pgfplots.libs}
\include{pgfplots.resources}
\include{pgfplots.importexport}
\include{pgfplots.basic.reference}

\printindex

\bibliographystyle{abbrv} %gerapali} %gerabbrv} %gerunsrt.bst} %gerabbrv}% gerplain}
\nocite{pgfplotstable}
\nocite{programmingnotes}
\bibliography{pgfplots}
\end{document}

% -----------------------------------------------------------------------------
% For Stefan Pinnow as reminder on what to look for when editing the manual
% -----------------------------------------------------------------------------
% There should be no line breaks in the following environments
% - |...|
% - \declareandlabel{...}
% - \verbpdfref{...}
% If "MakeTikzPictures" isn't running through check one of the externalized
% LOG files what is the cause of that.
% -----------------------------------------------------------------------------


%%%%%%%%%%%%%%%%%%%%%%%%%%%%%%%%%%%%%%%%%%%%%%%%%%%%%%%%%%%%%%%%%%%%%%%%%%%%%
%
% Package pgfplots.sty documentation.
%
% Copyright 2007/2008 by Christian Feuersaenger.
%
% This program is free software: you can redistribute it and/or modify
% it under the terms of the GNU General Public License as published by
% the Free Software Foundation, either version 3 of the License, or
% (at your option) any later version.
%
% This program is distributed in the hope that it will be useful,
% but WITHOUT ANY WARRANTY; without even the implied warranty of
% MERCHANTABILITY or FITNESS FOR A PARTICULAR PURPOSE.  See the
% GNU General Public License for more details.
%
% You should have received a copy of the GNU General Public License
% along with this program.  If not, see <http://www.gnu.org/licenses/>.
%
%
%%%%%%%%%%%%%%%%%%%%%%%%%%%%%%%%%%%%%%%%%%%%%%%%%%%%%%%%%%%%%%%%%%%%%%%%%%%%%
% SEE pgfplots-macros.tex as well!
%\pdfminorversion=5 % to allow compression
%\pdfobjcompresslevel=2
\documentclass[a4paper,openany]{book}

\let\bookmaketitle=\maketitle

% -----------------------------------
% this here is from ltxdoc:
\usepackage{doc}[=v2]
\AtBeginDocument{\MakeShortVerb{\|}}
%\setlength{\textwidth}{355pt}
%\addtolength\marginparwidth{30pt}
%\addtolength\oddsidemargin{20pt}
%\addtolength\evensidemargin{20pt}
\def\cmd#1{\cs{\expandafter\cmd@to@cs\string#1}}
\def\cmd@to@cs#1#2{\char\number`#2\relax}
\DeclareRobustCommand\cs[1]{\texttt{\char`\\#1}}
\providecommand\marg[1]{%
  {\ttfamily\char`\{}\meta{#1}{\ttfamily\char`\}}}
\providecommand\oarg[1]{%
  {\ttfamily[}\meta{#1}{\ttfamily]}}
\providecommand\parg[1]{%
  {\ttfamily(}\meta{#1}{\ttfamily)}}
\raggedbottom

% -----------------------------------
\input{pgfplots.preamble.tex}

\makeatletter
% I want a two-column index just as in pgfmanual styles. This here
% was the best way to get one:
\def\index@prologue{\section*{Index}\addcontentsline{toc}{chapter}{Index}
}
\makeatother

%\RequirePackage[german,english,francais]{babel}

\def\matlabcolormaptext{This colormap is similar to one shipped with Matlab$^\text{\textregistered}$ under a similar name.}

\IfFileExists{tikzlibraryspy.code.tex}{%
\usetikzlibrary{spy}
}{%
    \message{ERROR: tikz SPY library NOT available. The manual will only compile partially.^^J}%
}%

\usepackage{xparse}% for colorbrewer manual

\usetikzlibrary{
    decorations.markings,
    decorations.footprints,
    shapes.arrows,
    matrix,
    positioning,
}

\usepgfplotslibrary{
    fillbetween,
    ternary,
    smithchart,
    patchplots,
    polar,
    colormaps,
    colorbrewer,
    colortol,           % docu not ready yet
}
\pgfqkeys{/codeexample}{%
    every codeexample/.append style={
        /pgfplots/every ternary axis/.append style={
            /pgfplots/legend style={fill=graphicbackground},
        }
    },
    tabsize=4,
}

\pgfplotsmanualenableexternalizationofexpensive

%\usetikzlibrary{external}
%\tikzexternalize[prefix=figures/]{pgfplots}

\title{%
    Manual for Package \PGFPlots{}\\
    {\small 2D/3D Plots in \LaTeX{}, Version \pgfplotsversion}\\
    {\small\url{https://github.com/pgf-tikz/pgfplots}}
    %\\{\small Attention: you are using an unstable development version.}
}

\makeatletter
\long\def\abstractsmuggle{%
    \centering
    \textbf{Abstract}\\[0.5cm]

    \begin{minipage}{12cm}
        \PGFPlots{} draws high-quality function plots in normal or logarithmic
        scaling with a user-friendly interface directly in \TeX{}. The user
        supplies axis labels, legend entries and the plot coordinates for one or
        more plots and \PGFPlots{} applies axis scaling, computes any logarithms
        and axis ticks and draws the plots. It supports line plots, scatter
        plots, piecewise constant plots, bar plots, area plots, mesh and surface
        plots, patch plots, contour plots, quiver plots, histogram plots, box
        plots, polar axes, ternary diagrams, smith charts and some more. It is
        based on Till Tantau's package \PGF{}/\Tikz{}.
    \end{minipage}

}%

\expandafter\date\expandafter{\@date\\[2cm]
    \abstractsmuggle
}%
\makeatother

%\includeonly{pgfplots.reference}


\begin{document}

\def\plotcoords{%
\addplot coordinates {
(5,8.312e-02)    (17,2.547e-02)   (49,7.407e-03)
(129,2.102e-03)  (321,5.874e-04)  (769,1.623e-04)
(1793,4.442e-05) (4097,1.207e-05) (9217,3.261e-06)
};

\addplot coordinates{
(7,8.472e-02)    (31,3.044e-02)    (111,1.022e-02)
(351,3.303e-03)  (1023,1.039e-03)  (2815,3.196e-04)
(7423,9.658e-05) (18943,2.873e-05) (47103,8.437e-06)
};

\addplot coordinates{
(9,7.881e-02)     (49,3.243e-02)    (209,1.232e-02)
(769,4.454e-03)   (2561,1.551e-03)  (7937,5.236e-04)
(23297,1.723e-04) (65537,5.545e-05) (178177,1.751e-05)
};

\addplot coordinates{
(11,6.887e-02)    (71,3.177e-02)     (351,1.341e-02)
(1471,5.334e-03)  (5503,2.027e-03)   (18943,7.415e-04)
(61183,2.628e-04) (187903,9.063e-05) (553983,3.053e-05)
};

\addplot coordinates{
(13,5.755e-02)     (97,2.925e-02)     (545,1.351e-02)
(2561,5.842e-03)   (10625,2.397e-03)  (40193,9.414e-04)
(141569,3.564e-04) (471041,1.308e-04)
(1496065,4.670e-05)
};
}%


\bookmaketitle
\tableofcontents
\include{pgfplots.title_abstract_intro}
\include{pgfplots.preliminaries}
\include{pgfplots.intro}
\include{pgfplots.reference}
\include{pgfplots.libs}
\include{pgfplots.resources}
\include{pgfplots.importexport}
\include{pgfplots.basic.reference}

\printindex

\bibliographystyle{abbrv} %gerapali} %gerabbrv} %gerunsrt.bst} %gerabbrv}% gerplain}
\nocite{pgfplotstable}
\nocite{programmingnotes}
\bibliography{pgfplots}
\end{document}

% -----------------------------------------------------------------------------
% For Stefan Pinnow as reminder on what to look for when editing the manual
% -----------------------------------------------------------------------------
% There should be no line breaks in the following environments
% - |...|
% - \declareandlabel{...}
% - \verbpdfref{...}
% If "MakeTikzPictures" isn't running through check one of the externalized
% LOG files what is the cause of that.
% -----------------------------------------------------------------------------

\include{pgfplots.basic.reference}

\printindex

\bibliographystyle{abbrv} %gerapali} %gerabbrv} %gerunsrt.bst} %gerabbrv}% gerplain}
\nocite{pgfplotstable}
\nocite{programmingnotes}
\bibliography{pgfplots}
\end{document}

% -----------------------------------------------------------------------------
% For Stefan Pinnow as reminder on what to look for when editing the manual
% -----------------------------------------------------------------------------
% There should be no line breaks in the following environments
% - |...|
% - \declareandlabel{...}
% - \verbpdfref{...}
% If "MakeTikzPictures" isn't running through check one of the externalized
% LOG files what is the cause of that.
% -----------------------------------------------------------------------------


%%%%%%%%%%%%%%%%%%%%%%%%%%%%%%%%%%%%%%%%%%%%%%%%%%%%%%%%%%%%%%%%%%%%%%%%%%%%%
%
% Package pgfplots.sty documentation.
%
% Copyright 2007/2008 by Christian Feuersaenger.
%
% This program is free software: you can redistribute it and/or modify
% it under the terms of the GNU General Public License as published by
% the Free Software Foundation, either version 3 of the License, or
% (at your option) any later version.
%
% This program is distributed in the hope that it will be useful,
% but WITHOUT ANY WARRANTY; without even the implied warranty of
% MERCHANTABILITY or FITNESS FOR A PARTICULAR PURPOSE.  See the
% GNU General Public License for more details.
%
% You should have received a copy of the GNU General Public License
% along with this program.  If not, see <http://www.gnu.org/licenses/>.
%
%
%%%%%%%%%%%%%%%%%%%%%%%%%%%%%%%%%%%%%%%%%%%%%%%%%%%%%%%%%%%%%%%%%%%%%%%%%%%%%
% SEE pgfplots-macros.tex as well!
%\pdfminorversion=5 % to allow compression
%\pdfobjcompresslevel=2
\documentclass[a4paper,openany]{book}

\let\bookmaketitle=\maketitle

% -----------------------------------
% this here is from ltxdoc:
\usepackage{doc}[=v2]
\AtBeginDocument{\MakeShortVerb{\|}}
%\setlength{\textwidth}{355pt}
%\addtolength\marginparwidth{30pt}
%\addtolength\oddsidemargin{20pt}
%\addtolength\evensidemargin{20pt}
\def\cmd#1{\cs{\expandafter\cmd@to@cs\string#1}}
\def\cmd@to@cs#1#2{\char\number`#2\relax}
\DeclareRobustCommand\cs[1]{\texttt{\char`\\#1}}
\providecommand\marg[1]{%
  {\ttfamily\char`\{}\meta{#1}{\ttfamily\char`\}}}
\providecommand\oarg[1]{%
  {\ttfamily[}\meta{#1}{\ttfamily]}}
\providecommand\parg[1]{%
  {\ttfamily(}\meta{#1}{\ttfamily)}}
\raggedbottom

% -----------------------------------
\input{pgfplots.preamble.tex}

\makeatletter
% I want a two-column index just as in pgfmanual styles. This here
% was the best way to get one:
\def\index@prologue{\section*{Index}\addcontentsline{toc}{chapter}{Index}
}
\makeatother

%\RequirePackage[german,english,francais]{babel}

\def\matlabcolormaptext{This colormap is similar to one shipped with Matlab$^\text{\textregistered}$ under a similar name.}

\IfFileExists{tikzlibraryspy.code.tex}{%
\usetikzlibrary{spy}
}{%
    \message{ERROR: tikz SPY library NOT available. The manual will only compile partially.^^J}%
}%

\usepackage{xparse}% for colorbrewer manual

\usetikzlibrary{
    decorations.markings,
    decorations.footprints,
    shapes.arrows,
    matrix,
    positioning,
}

\usepgfplotslibrary{
    fillbetween,
    ternary,
    smithchart,
    patchplots,
    polar,
    colormaps,
    colorbrewer,
    colortol,           % docu not ready yet
}
\pgfqkeys{/codeexample}{%
    every codeexample/.append style={
        /pgfplots/every ternary axis/.append style={
            /pgfplots/legend style={fill=graphicbackground},
        }
    },
    tabsize=4,
}

\pgfplotsmanualenableexternalizationofexpensive

%\usetikzlibrary{external}
%\tikzexternalize[prefix=figures/]{pgfplots}

\title{%
    Manual for Package \PGFPlots{}\\
    {\small 2D/3D Plots in \LaTeX{}, Version \pgfplotsversion}\\
    {\small\url{https://github.com/pgf-tikz/pgfplots}}
    %\\{\small Attention: you are using an unstable development version.}
}

\makeatletter
\long\def\abstractsmuggle{%
    \centering
    \textbf{Abstract}\\[0.5cm]

    \begin{minipage}{12cm}
        \PGFPlots{} draws high-quality function plots in normal or logarithmic
        scaling with a user-friendly interface directly in \TeX{}. The user
        supplies axis labels, legend entries and the plot coordinates for one or
        more plots and \PGFPlots{} applies axis scaling, computes any logarithms
        and axis ticks and draws the plots. It supports line plots, scatter
        plots, piecewise constant plots, bar plots, area plots, mesh and surface
        plots, patch plots, contour plots, quiver plots, histogram plots, box
        plots, polar axes, ternary diagrams, smith charts and some more. It is
        based on Till Tantau's package \PGF{}/\Tikz{}.
    \end{minipage}

}%

\expandafter\date\expandafter{\@date\\[2cm]
    \abstractsmuggle
}%
\makeatother

%\includeonly{pgfplots.reference}


\begin{document}

\def\plotcoords{%
\addplot coordinates {
(5,8.312e-02)    (17,2.547e-02)   (49,7.407e-03)
(129,2.102e-03)  (321,5.874e-04)  (769,1.623e-04)
(1793,4.442e-05) (4097,1.207e-05) (9217,3.261e-06)
};

\addplot coordinates{
(7,8.472e-02)    (31,3.044e-02)    (111,1.022e-02)
(351,3.303e-03)  (1023,1.039e-03)  (2815,3.196e-04)
(7423,9.658e-05) (18943,2.873e-05) (47103,8.437e-06)
};

\addplot coordinates{
(9,7.881e-02)     (49,3.243e-02)    (209,1.232e-02)
(769,4.454e-03)   (2561,1.551e-03)  (7937,5.236e-04)
(23297,1.723e-04) (65537,5.545e-05) (178177,1.751e-05)
};

\addplot coordinates{
(11,6.887e-02)    (71,3.177e-02)     (351,1.341e-02)
(1471,5.334e-03)  (5503,2.027e-03)   (18943,7.415e-04)
(61183,2.628e-04) (187903,9.063e-05) (553983,3.053e-05)
};

\addplot coordinates{
(13,5.755e-02)     (97,2.925e-02)     (545,1.351e-02)
(2561,5.842e-03)   (10625,2.397e-03)  (40193,9.414e-04)
(141569,3.564e-04) (471041,1.308e-04)
(1496065,4.670e-05)
};
}%


\bookmaketitle
\tableofcontents

%%%%%%%%%%%%%%%%%%%%%%%%%%%%%%%%%%%%%%%%%%%%%%%%%%%%%%%%%%%%%%%%%%%%%%%%%%%%%
%
% Package pgfplots.sty documentation.
%
% Copyright 2007/2008 by Christian Feuersaenger.
%
% This program is free software: you can redistribute it and/or modify
% it under the terms of the GNU General Public License as published by
% the Free Software Foundation, either version 3 of the License, or
% (at your option) any later version.
%
% This program is distributed in the hope that it will be useful,
% but WITHOUT ANY WARRANTY; without even the implied warranty of
% MERCHANTABILITY or FITNESS FOR A PARTICULAR PURPOSE.  See the
% GNU General Public License for more details.
%
% You should have received a copy of the GNU General Public License
% along with this program.  If not, see <http://www.gnu.org/licenses/>.
%
%
%%%%%%%%%%%%%%%%%%%%%%%%%%%%%%%%%%%%%%%%%%%%%%%%%%%%%%%%%%%%%%%%%%%%%%%%%%%%%
% SEE pgfplots-macros.tex as well!
%\pdfminorversion=5 % to allow compression
%\pdfobjcompresslevel=2
\documentclass[a4paper,openany]{book}

\let\bookmaketitle=\maketitle

% -----------------------------------
% this here is from ltxdoc:
\usepackage{doc}[=v2]
\AtBeginDocument{\MakeShortVerb{\|}}
%\setlength{\textwidth}{355pt}
%\addtolength\marginparwidth{30pt}
%\addtolength\oddsidemargin{20pt}
%\addtolength\evensidemargin{20pt}
\def\cmd#1{\cs{\expandafter\cmd@to@cs\string#1}}
\def\cmd@to@cs#1#2{\char\number`#2\relax}
\DeclareRobustCommand\cs[1]{\texttt{\char`\\#1}}
\providecommand\marg[1]{%
  {\ttfamily\char`\{}\meta{#1}{\ttfamily\char`\}}}
\providecommand\oarg[1]{%
  {\ttfamily[}\meta{#1}{\ttfamily]}}
\providecommand\parg[1]{%
  {\ttfamily(}\meta{#1}{\ttfamily)}}
\raggedbottom

% -----------------------------------
\input{pgfplots.preamble.tex}

\makeatletter
% I want a two-column index just as in pgfmanual styles. This here
% was the best way to get one:
\def\index@prologue{\section*{Index}\addcontentsline{toc}{chapter}{Index}
}
\makeatother

%\RequirePackage[german,english,francais]{babel}

\def\matlabcolormaptext{This colormap is similar to one shipped with Matlab$^\text{\textregistered}$ under a similar name.}

\IfFileExists{tikzlibraryspy.code.tex}{%
\usetikzlibrary{spy}
}{%
    \message{ERROR: tikz SPY library NOT available. The manual will only compile partially.^^J}%
}%

\usepackage{xparse}% for colorbrewer manual

\usetikzlibrary{
    decorations.markings,
    decorations.footprints,
    shapes.arrows,
    matrix,
    positioning,
}

\usepgfplotslibrary{
    fillbetween,
    ternary,
    smithchart,
    patchplots,
    polar,
    colormaps,
    colorbrewer,
    colortol,           % docu not ready yet
}
\pgfqkeys{/codeexample}{%
    every codeexample/.append style={
        /pgfplots/every ternary axis/.append style={
            /pgfplots/legend style={fill=graphicbackground},
        }
    },
    tabsize=4,
}

\pgfplotsmanualenableexternalizationofexpensive

%\usetikzlibrary{external}
%\tikzexternalize[prefix=figures/]{pgfplots}

\title{%
    Manual for Package \PGFPlots{}\\
    {\small 2D/3D Plots in \LaTeX{}, Version \pgfplotsversion}\\
    {\small\url{https://github.com/pgf-tikz/pgfplots}}
    %\\{\small Attention: you are using an unstable development version.}
}

\makeatletter
\long\def\abstractsmuggle{%
    \centering
    \textbf{Abstract}\\[0.5cm]

    \begin{minipage}{12cm}
        \PGFPlots{} draws high-quality function plots in normal or logarithmic
        scaling with a user-friendly interface directly in \TeX{}. The user
        supplies axis labels, legend entries and the plot coordinates for one or
        more plots and \PGFPlots{} applies axis scaling, computes any logarithms
        and axis ticks and draws the plots. It supports line plots, scatter
        plots, piecewise constant plots, bar plots, area plots, mesh and surface
        plots, patch plots, contour plots, quiver plots, histogram plots, box
        plots, polar axes, ternary diagrams, smith charts and some more. It is
        based on Till Tantau's package \PGF{}/\Tikz{}.
    \end{minipage}

}%

\expandafter\date\expandafter{\@date\\[2cm]
    \abstractsmuggle
}%
\makeatother

%\includeonly{pgfplots.reference}


\begin{document}

\def\plotcoords{%
\addplot coordinates {
(5,8.312e-02)    (17,2.547e-02)   (49,7.407e-03)
(129,2.102e-03)  (321,5.874e-04)  (769,1.623e-04)
(1793,4.442e-05) (4097,1.207e-05) (9217,3.261e-06)
};

\addplot coordinates{
(7,8.472e-02)    (31,3.044e-02)    (111,1.022e-02)
(351,3.303e-03)  (1023,1.039e-03)  (2815,3.196e-04)
(7423,9.658e-05) (18943,2.873e-05) (47103,8.437e-06)
};

\addplot coordinates{
(9,7.881e-02)     (49,3.243e-02)    (209,1.232e-02)
(769,4.454e-03)   (2561,1.551e-03)  (7937,5.236e-04)
(23297,1.723e-04) (65537,5.545e-05) (178177,1.751e-05)
};

\addplot coordinates{
(11,6.887e-02)    (71,3.177e-02)     (351,1.341e-02)
(1471,5.334e-03)  (5503,2.027e-03)   (18943,7.415e-04)
(61183,2.628e-04) (187903,9.063e-05) (553983,3.053e-05)
};

\addplot coordinates{
(13,5.755e-02)     (97,2.925e-02)     (545,1.351e-02)
(2561,5.842e-03)   (10625,2.397e-03)  (40193,9.414e-04)
(141569,3.564e-04) (471041,1.308e-04)
(1496065,4.670e-05)
};
}%


\bookmaketitle
\tableofcontents
\include{pgfplots.title_abstract_intro}
\include{pgfplots.preliminaries}
\include{pgfplots.intro}
\include{pgfplots.reference}
\include{pgfplots.libs}
\include{pgfplots.resources}
\include{pgfplots.importexport}
\include{pgfplots.basic.reference}

\printindex

\bibliographystyle{abbrv} %gerapali} %gerabbrv} %gerunsrt.bst} %gerabbrv}% gerplain}
\nocite{pgfplotstable}
\nocite{programmingnotes}
\bibliography{pgfplots}
\end{document}

% -----------------------------------------------------------------------------
% For Stefan Pinnow as reminder on what to look for when editing the manual
% -----------------------------------------------------------------------------
% There should be no line breaks in the following environments
% - |...|
% - \declareandlabel{...}
% - \verbpdfref{...}
% If "MakeTikzPictures" isn't running through check one of the externalized
% LOG files what is the cause of that.
% -----------------------------------------------------------------------------


%%%%%%%%%%%%%%%%%%%%%%%%%%%%%%%%%%%%%%%%%%%%%%%%%%%%%%%%%%%%%%%%%%%%%%%%%%%%%
%
% Package pgfplots.sty documentation.
%
% Copyright 2007/2008 by Christian Feuersaenger.
%
% This program is free software: you can redistribute it and/or modify
% it under the terms of the GNU General Public License as published by
% the Free Software Foundation, either version 3 of the License, or
% (at your option) any later version.
%
% This program is distributed in the hope that it will be useful,
% but WITHOUT ANY WARRANTY; without even the implied warranty of
% MERCHANTABILITY or FITNESS FOR A PARTICULAR PURPOSE.  See the
% GNU General Public License for more details.
%
% You should have received a copy of the GNU General Public License
% along with this program.  If not, see <http://www.gnu.org/licenses/>.
%
%
%%%%%%%%%%%%%%%%%%%%%%%%%%%%%%%%%%%%%%%%%%%%%%%%%%%%%%%%%%%%%%%%%%%%%%%%%%%%%
% SEE pgfplots-macros.tex as well!
%\pdfminorversion=5 % to allow compression
%\pdfobjcompresslevel=2
\documentclass[a4paper,openany]{book}

\let\bookmaketitle=\maketitle

% -----------------------------------
% this here is from ltxdoc:
\usepackage{doc}[=v2]
\AtBeginDocument{\MakeShortVerb{\|}}
%\setlength{\textwidth}{355pt}
%\addtolength\marginparwidth{30pt}
%\addtolength\oddsidemargin{20pt}
%\addtolength\evensidemargin{20pt}
\def\cmd#1{\cs{\expandafter\cmd@to@cs\string#1}}
\def\cmd@to@cs#1#2{\char\number`#2\relax}
\DeclareRobustCommand\cs[1]{\texttt{\char`\\#1}}
\providecommand\marg[1]{%
  {\ttfamily\char`\{}\meta{#1}{\ttfamily\char`\}}}
\providecommand\oarg[1]{%
  {\ttfamily[}\meta{#1}{\ttfamily]}}
\providecommand\parg[1]{%
  {\ttfamily(}\meta{#1}{\ttfamily)}}
\raggedbottom

% -----------------------------------
\input{pgfplots.preamble.tex}

\makeatletter
% I want a two-column index just as in pgfmanual styles. This here
% was the best way to get one:
\def\index@prologue{\section*{Index}\addcontentsline{toc}{chapter}{Index}
}
\makeatother

%\RequirePackage[german,english,francais]{babel}

\def\matlabcolormaptext{This colormap is similar to one shipped with Matlab$^\text{\textregistered}$ under a similar name.}

\IfFileExists{tikzlibraryspy.code.tex}{%
\usetikzlibrary{spy}
}{%
    \message{ERROR: tikz SPY library NOT available. The manual will only compile partially.^^J}%
}%

\usepackage{xparse}% for colorbrewer manual

\usetikzlibrary{
    decorations.markings,
    decorations.footprints,
    shapes.arrows,
    matrix,
    positioning,
}

\usepgfplotslibrary{
    fillbetween,
    ternary,
    smithchart,
    patchplots,
    polar,
    colormaps,
    colorbrewer,
    colortol,           % docu not ready yet
}
\pgfqkeys{/codeexample}{%
    every codeexample/.append style={
        /pgfplots/every ternary axis/.append style={
            /pgfplots/legend style={fill=graphicbackground},
        }
    },
    tabsize=4,
}

\pgfplotsmanualenableexternalizationofexpensive

%\usetikzlibrary{external}
%\tikzexternalize[prefix=figures/]{pgfplots}

\title{%
    Manual for Package \PGFPlots{}\\
    {\small 2D/3D Plots in \LaTeX{}, Version \pgfplotsversion}\\
    {\small\url{https://github.com/pgf-tikz/pgfplots}}
    %\\{\small Attention: you are using an unstable development version.}
}

\makeatletter
\long\def\abstractsmuggle{%
    \centering
    \textbf{Abstract}\\[0.5cm]

    \begin{minipage}{12cm}
        \PGFPlots{} draws high-quality function plots in normal or logarithmic
        scaling with a user-friendly interface directly in \TeX{}. The user
        supplies axis labels, legend entries and the plot coordinates for one or
        more plots and \PGFPlots{} applies axis scaling, computes any logarithms
        and axis ticks and draws the plots. It supports line plots, scatter
        plots, piecewise constant plots, bar plots, area plots, mesh and surface
        plots, patch plots, contour plots, quiver plots, histogram plots, box
        plots, polar axes, ternary diagrams, smith charts and some more. It is
        based on Till Tantau's package \PGF{}/\Tikz{}.
    \end{minipage}

}%

\expandafter\date\expandafter{\@date\\[2cm]
    \abstractsmuggle
}%
\makeatother

%\includeonly{pgfplots.reference}


\begin{document}

\def\plotcoords{%
\addplot coordinates {
(5,8.312e-02)    (17,2.547e-02)   (49,7.407e-03)
(129,2.102e-03)  (321,5.874e-04)  (769,1.623e-04)
(1793,4.442e-05) (4097,1.207e-05) (9217,3.261e-06)
};

\addplot coordinates{
(7,8.472e-02)    (31,3.044e-02)    (111,1.022e-02)
(351,3.303e-03)  (1023,1.039e-03)  (2815,3.196e-04)
(7423,9.658e-05) (18943,2.873e-05) (47103,8.437e-06)
};

\addplot coordinates{
(9,7.881e-02)     (49,3.243e-02)    (209,1.232e-02)
(769,4.454e-03)   (2561,1.551e-03)  (7937,5.236e-04)
(23297,1.723e-04) (65537,5.545e-05) (178177,1.751e-05)
};

\addplot coordinates{
(11,6.887e-02)    (71,3.177e-02)     (351,1.341e-02)
(1471,5.334e-03)  (5503,2.027e-03)   (18943,7.415e-04)
(61183,2.628e-04) (187903,9.063e-05) (553983,3.053e-05)
};

\addplot coordinates{
(13,5.755e-02)     (97,2.925e-02)     (545,1.351e-02)
(2561,5.842e-03)   (10625,2.397e-03)  (40193,9.414e-04)
(141569,3.564e-04) (471041,1.308e-04)
(1496065,4.670e-05)
};
}%


\bookmaketitle
\tableofcontents
\include{pgfplots.title_abstract_intro}
\include{pgfplots.preliminaries}
\include{pgfplots.intro}
\include{pgfplots.reference}
\include{pgfplots.libs}
\include{pgfplots.resources}
\include{pgfplots.importexport}
\include{pgfplots.basic.reference}

\printindex

\bibliographystyle{abbrv} %gerapali} %gerabbrv} %gerunsrt.bst} %gerabbrv}% gerplain}
\nocite{pgfplotstable}
\nocite{programmingnotes}
\bibliography{pgfplots}
\end{document}

% -----------------------------------------------------------------------------
% For Stefan Pinnow as reminder on what to look for when editing the manual
% -----------------------------------------------------------------------------
% There should be no line breaks in the following environments
% - |...|
% - \declareandlabel{...}
% - \verbpdfref{...}
% If "MakeTikzPictures" isn't running through check one of the externalized
% LOG files what is the cause of that.
% -----------------------------------------------------------------------------


%%%%%%%%%%%%%%%%%%%%%%%%%%%%%%%%%%%%%%%%%%%%%%%%%%%%%%%%%%%%%%%%%%%%%%%%%%%%%
%
% Package pgfplots.sty documentation.
%
% Copyright 2007/2008 by Christian Feuersaenger.
%
% This program is free software: you can redistribute it and/or modify
% it under the terms of the GNU General Public License as published by
% the Free Software Foundation, either version 3 of the License, or
% (at your option) any later version.
%
% This program is distributed in the hope that it will be useful,
% but WITHOUT ANY WARRANTY; without even the implied warranty of
% MERCHANTABILITY or FITNESS FOR A PARTICULAR PURPOSE.  See the
% GNU General Public License for more details.
%
% You should have received a copy of the GNU General Public License
% along with this program.  If not, see <http://www.gnu.org/licenses/>.
%
%
%%%%%%%%%%%%%%%%%%%%%%%%%%%%%%%%%%%%%%%%%%%%%%%%%%%%%%%%%%%%%%%%%%%%%%%%%%%%%
% SEE pgfplots-macros.tex as well!
%\pdfminorversion=5 % to allow compression
%\pdfobjcompresslevel=2
\documentclass[a4paper,openany]{book}

\let\bookmaketitle=\maketitle

% -----------------------------------
% this here is from ltxdoc:
\usepackage{doc}[=v2]
\AtBeginDocument{\MakeShortVerb{\|}}
%\setlength{\textwidth}{355pt}
%\addtolength\marginparwidth{30pt}
%\addtolength\oddsidemargin{20pt}
%\addtolength\evensidemargin{20pt}
\def\cmd#1{\cs{\expandafter\cmd@to@cs\string#1}}
\def\cmd@to@cs#1#2{\char\number`#2\relax}
\DeclareRobustCommand\cs[1]{\texttt{\char`\\#1}}
\providecommand\marg[1]{%
  {\ttfamily\char`\{}\meta{#1}{\ttfamily\char`\}}}
\providecommand\oarg[1]{%
  {\ttfamily[}\meta{#1}{\ttfamily]}}
\providecommand\parg[1]{%
  {\ttfamily(}\meta{#1}{\ttfamily)}}
\raggedbottom

% -----------------------------------
\input{pgfplots.preamble.tex}

\makeatletter
% I want a two-column index just as in pgfmanual styles. This here
% was the best way to get one:
\def\index@prologue{\section*{Index}\addcontentsline{toc}{chapter}{Index}
}
\makeatother

%\RequirePackage[german,english,francais]{babel}

\def\matlabcolormaptext{This colormap is similar to one shipped with Matlab$^\text{\textregistered}$ under a similar name.}

\IfFileExists{tikzlibraryspy.code.tex}{%
\usetikzlibrary{spy}
}{%
    \message{ERROR: tikz SPY library NOT available. The manual will only compile partially.^^J}%
}%

\usepackage{xparse}% for colorbrewer manual

\usetikzlibrary{
    decorations.markings,
    decorations.footprints,
    shapes.arrows,
    matrix,
    positioning,
}

\usepgfplotslibrary{
    fillbetween,
    ternary,
    smithchart,
    patchplots,
    polar,
    colormaps,
    colorbrewer,
    colortol,           % docu not ready yet
}
\pgfqkeys{/codeexample}{%
    every codeexample/.append style={
        /pgfplots/every ternary axis/.append style={
            /pgfplots/legend style={fill=graphicbackground},
        }
    },
    tabsize=4,
}

\pgfplotsmanualenableexternalizationofexpensive

%\usetikzlibrary{external}
%\tikzexternalize[prefix=figures/]{pgfplots}

\title{%
    Manual for Package \PGFPlots{}\\
    {\small 2D/3D Plots in \LaTeX{}, Version \pgfplotsversion}\\
    {\small\url{https://github.com/pgf-tikz/pgfplots}}
    %\\{\small Attention: you are using an unstable development version.}
}

\makeatletter
\long\def\abstractsmuggle{%
    \centering
    \textbf{Abstract}\\[0.5cm]

    \begin{minipage}{12cm}
        \PGFPlots{} draws high-quality function plots in normal or logarithmic
        scaling with a user-friendly interface directly in \TeX{}. The user
        supplies axis labels, legend entries and the plot coordinates for one or
        more plots and \PGFPlots{} applies axis scaling, computes any logarithms
        and axis ticks and draws the plots. It supports line plots, scatter
        plots, piecewise constant plots, bar plots, area plots, mesh and surface
        plots, patch plots, contour plots, quiver plots, histogram plots, box
        plots, polar axes, ternary diagrams, smith charts and some more. It is
        based on Till Tantau's package \PGF{}/\Tikz{}.
    \end{minipage}

}%

\expandafter\date\expandafter{\@date\\[2cm]
    \abstractsmuggle
}%
\makeatother

%\includeonly{pgfplots.reference}


\begin{document}

\def\plotcoords{%
\addplot coordinates {
(5,8.312e-02)    (17,2.547e-02)   (49,7.407e-03)
(129,2.102e-03)  (321,5.874e-04)  (769,1.623e-04)
(1793,4.442e-05) (4097,1.207e-05) (9217,3.261e-06)
};

\addplot coordinates{
(7,8.472e-02)    (31,3.044e-02)    (111,1.022e-02)
(351,3.303e-03)  (1023,1.039e-03)  (2815,3.196e-04)
(7423,9.658e-05) (18943,2.873e-05) (47103,8.437e-06)
};

\addplot coordinates{
(9,7.881e-02)     (49,3.243e-02)    (209,1.232e-02)
(769,4.454e-03)   (2561,1.551e-03)  (7937,5.236e-04)
(23297,1.723e-04) (65537,5.545e-05) (178177,1.751e-05)
};

\addplot coordinates{
(11,6.887e-02)    (71,3.177e-02)     (351,1.341e-02)
(1471,5.334e-03)  (5503,2.027e-03)   (18943,7.415e-04)
(61183,2.628e-04) (187903,9.063e-05) (553983,3.053e-05)
};

\addplot coordinates{
(13,5.755e-02)     (97,2.925e-02)     (545,1.351e-02)
(2561,5.842e-03)   (10625,2.397e-03)  (40193,9.414e-04)
(141569,3.564e-04) (471041,1.308e-04)
(1496065,4.670e-05)
};
}%


\bookmaketitle
\tableofcontents
\include{pgfplots.title_abstract_intro}
\include{pgfplots.preliminaries}
\include{pgfplots.intro}
\include{pgfplots.reference}
\include{pgfplots.libs}
\include{pgfplots.resources}
\include{pgfplots.importexport}
\include{pgfplots.basic.reference}

\printindex

\bibliographystyle{abbrv} %gerapali} %gerabbrv} %gerunsrt.bst} %gerabbrv}% gerplain}
\nocite{pgfplotstable}
\nocite{programmingnotes}
\bibliography{pgfplots}
\end{document}

% -----------------------------------------------------------------------------
% For Stefan Pinnow as reminder on what to look for when editing the manual
% -----------------------------------------------------------------------------
% There should be no line breaks in the following environments
% - |...|
% - \declareandlabel{...}
% - \verbpdfref{...}
% If "MakeTikzPictures" isn't running through check one of the externalized
% LOG files what is the cause of that.
% -----------------------------------------------------------------------------


%%%%%%%%%%%%%%%%%%%%%%%%%%%%%%%%%%%%%%%%%%%%%%%%%%%%%%%%%%%%%%%%%%%%%%%%%%%%%
%
% Package pgfplots.sty documentation.
%
% Copyright 2007/2008 by Christian Feuersaenger.
%
% This program is free software: you can redistribute it and/or modify
% it under the terms of the GNU General Public License as published by
% the Free Software Foundation, either version 3 of the License, or
% (at your option) any later version.
%
% This program is distributed in the hope that it will be useful,
% but WITHOUT ANY WARRANTY; without even the implied warranty of
% MERCHANTABILITY or FITNESS FOR A PARTICULAR PURPOSE.  See the
% GNU General Public License for more details.
%
% You should have received a copy of the GNU General Public License
% along with this program.  If not, see <http://www.gnu.org/licenses/>.
%
%
%%%%%%%%%%%%%%%%%%%%%%%%%%%%%%%%%%%%%%%%%%%%%%%%%%%%%%%%%%%%%%%%%%%%%%%%%%%%%
% SEE pgfplots-macros.tex as well!
%\pdfminorversion=5 % to allow compression
%\pdfobjcompresslevel=2
\documentclass[a4paper,openany]{book}

\let\bookmaketitle=\maketitle

% -----------------------------------
% this here is from ltxdoc:
\usepackage{doc}[=v2]
\AtBeginDocument{\MakeShortVerb{\|}}
%\setlength{\textwidth}{355pt}
%\addtolength\marginparwidth{30pt}
%\addtolength\oddsidemargin{20pt}
%\addtolength\evensidemargin{20pt}
\def\cmd#1{\cs{\expandafter\cmd@to@cs\string#1}}
\def\cmd@to@cs#1#2{\char\number`#2\relax}
\DeclareRobustCommand\cs[1]{\texttt{\char`\\#1}}
\providecommand\marg[1]{%
  {\ttfamily\char`\{}\meta{#1}{\ttfamily\char`\}}}
\providecommand\oarg[1]{%
  {\ttfamily[}\meta{#1}{\ttfamily]}}
\providecommand\parg[1]{%
  {\ttfamily(}\meta{#1}{\ttfamily)}}
\raggedbottom

% -----------------------------------
\input{pgfplots.preamble.tex}

\makeatletter
% I want a two-column index just as in pgfmanual styles. This here
% was the best way to get one:
\def\index@prologue{\section*{Index}\addcontentsline{toc}{chapter}{Index}
}
\makeatother

%\RequirePackage[german,english,francais]{babel}

\def\matlabcolormaptext{This colormap is similar to one shipped with Matlab$^\text{\textregistered}$ under a similar name.}

\IfFileExists{tikzlibraryspy.code.tex}{%
\usetikzlibrary{spy}
}{%
    \message{ERROR: tikz SPY library NOT available. The manual will only compile partially.^^J}%
}%

\usepackage{xparse}% for colorbrewer manual

\usetikzlibrary{
    decorations.markings,
    decorations.footprints,
    shapes.arrows,
    matrix,
    positioning,
}

\usepgfplotslibrary{
    fillbetween,
    ternary,
    smithchart,
    patchplots,
    polar,
    colormaps,
    colorbrewer,
    colortol,           % docu not ready yet
}
\pgfqkeys{/codeexample}{%
    every codeexample/.append style={
        /pgfplots/every ternary axis/.append style={
            /pgfplots/legend style={fill=graphicbackground},
        }
    },
    tabsize=4,
}

\pgfplotsmanualenableexternalizationofexpensive

%\usetikzlibrary{external}
%\tikzexternalize[prefix=figures/]{pgfplots}

\title{%
    Manual for Package \PGFPlots{}\\
    {\small 2D/3D Plots in \LaTeX{}, Version \pgfplotsversion}\\
    {\small\url{https://github.com/pgf-tikz/pgfplots}}
    %\\{\small Attention: you are using an unstable development version.}
}

\makeatletter
\long\def\abstractsmuggle{%
    \centering
    \textbf{Abstract}\\[0.5cm]

    \begin{minipage}{12cm}
        \PGFPlots{} draws high-quality function plots in normal or logarithmic
        scaling with a user-friendly interface directly in \TeX{}. The user
        supplies axis labels, legend entries and the plot coordinates for one or
        more plots and \PGFPlots{} applies axis scaling, computes any logarithms
        and axis ticks and draws the plots. It supports line plots, scatter
        plots, piecewise constant plots, bar plots, area plots, mesh and surface
        plots, patch plots, contour plots, quiver plots, histogram plots, box
        plots, polar axes, ternary diagrams, smith charts and some more. It is
        based on Till Tantau's package \PGF{}/\Tikz{}.
    \end{minipage}

}%

\expandafter\date\expandafter{\@date\\[2cm]
    \abstractsmuggle
}%
\makeatother

%\includeonly{pgfplots.reference}


\begin{document}

\def\plotcoords{%
\addplot coordinates {
(5,8.312e-02)    (17,2.547e-02)   (49,7.407e-03)
(129,2.102e-03)  (321,5.874e-04)  (769,1.623e-04)
(1793,4.442e-05) (4097,1.207e-05) (9217,3.261e-06)
};

\addplot coordinates{
(7,8.472e-02)    (31,3.044e-02)    (111,1.022e-02)
(351,3.303e-03)  (1023,1.039e-03)  (2815,3.196e-04)
(7423,9.658e-05) (18943,2.873e-05) (47103,8.437e-06)
};

\addplot coordinates{
(9,7.881e-02)     (49,3.243e-02)    (209,1.232e-02)
(769,4.454e-03)   (2561,1.551e-03)  (7937,5.236e-04)
(23297,1.723e-04) (65537,5.545e-05) (178177,1.751e-05)
};

\addplot coordinates{
(11,6.887e-02)    (71,3.177e-02)     (351,1.341e-02)
(1471,5.334e-03)  (5503,2.027e-03)   (18943,7.415e-04)
(61183,2.628e-04) (187903,9.063e-05) (553983,3.053e-05)
};

\addplot coordinates{
(13,5.755e-02)     (97,2.925e-02)     (545,1.351e-02)
(2561,5.842e-03)   (10625,2.397e-03)  (40193,9.414e-04)
(141569,3.564e-04) (471041,1.308e-04)
(1496065,4.670e-05)
};
}%


\bookmaketitle
\tableofcontents
\include{pgfplots.title_abstract_intro}
\include{pgfplots.preliminaries}
\include{pgfplots.intro}
\include{pgfplots.reference}
\include{pgfplots.libs}
\include{pgfplots.resources}
\include{pgfplots.importexport}
\include{pgfplots.basic.reference}

\printindex

\bibliographystyle{abbrv} %gerapali} %gerabbrv} %gerunsrt.bst} %gerabbrv}% gerplain}
\nocite{pgfplotstable}
\nocite{programmingnotes}
\bibliography{pgfplots}
\end{document}

% -----------------------------------------------------------------------------
% For Stefan Pinnow as reminder on what to look for when editing the manual
% -----------------------------------------------------------------------------
% There should be no line breaks in the following environments
% - |...|
% - \declareandlabel{...}
% - \verbpdfref{...}
% If "MakeTikzPictures" isn't running through check one of the externalized
% LOG files what is the cause of that.
% -----------------------------------------------------------------------------


%%%%%%%%%%%%%%%%%%%%%%%%%%%%%%%%%%%%%%%%%%%%%%%%%%%%%%%%%%%%%%%%%%%%%%%%%%%%%
%
% Package pgfplots.sty documentation.
%
% Copyright 2007/2008 by Christian Feuersaenger.
%
% This program is free software: you can redistribute it and/or modify
% it under the terms of the GNU General Public License as published by
% the Free Software Foundation, either version 3 of the License, or
% (at your option) any later version.
%
% This program is distributed in the hope that it will be useful,
% but WITHOUT ANY WARRANTY; without even the implied warranty of
% MERCHANTABILITY or FITNESS FOR A PARTICULAR PURPOSE.  See the
% GNU General Public License for more details.
%
% You should have received a copy of the GNU General Public License
% along with this program.  If not, see <http://www.gnu.org/licenses/>.
%
%
%%%%%%%%%%%%%%%%%%%%%%%%%%%%%%%%%%%%%%%%%%%%%%%%%%%%%%%%%%%%%%%%%%%%%%%%%%%%%
% SEE pgfplots-macros.tex as well!
%\pdfminorversion=5 % to allow compression
%\pdfobjcompresslevel=2
\documentclass[a4paper,openany]{book}

\let\bookmaketitle=\maketitle

% -----------------------------------
% this here is from ltxdoc:
\usepackage{doc}[=v2]
\AtBeginDocument{\MakeShortVerb{\|}}
%\setlength{\textwidth}{355pt}
%\addtolength\marginparwidth{30pt}
%\addtolength\oddsidemargin{20pt}
%\addtolength\evensidemargin{20pt}
\def\cmd#1{\cs{\expandafter\cmd@to@cs\string#1}}
\def\cmd@to@cs#1#2{\char\number`#2\relax}
\DeclareRobustCommand\cs[1]{\texttt{\char`\\#1}}
\providecommand\marg[1]{%
  {\ttfamily\char`\{}\meta{#1}{\ttfamily\char`\}}}
\providecommand\oarg[1]{%
  {\ttfamily[}\meta{#1}{\ttfamily]}}
\providecommand\parg[1]{%
  {\ttfamily(}\meta{#1}{\ttfamily)}}
\raggedbottom

% -----------------------------------
\input{pgfplots.preamble.tex}

\makeatletter
% I want a two-column index just as in pgfmanual styles. This here
% was the best way to get one:
\def\index@prologue{\section*{Index}\addcontentsline{toc}{chapter}{Index}
}
\makeatother

%\RequirePackage[german,english,francais]{babel}

\def\matlabcolormaptext{This colormap is similar to one shipped with Matlab$^\text{\textregistered}$ under a similar name.}

\IfFileExists{tikzlibraryspy.code.tex}{%
\usetikzlibrary{spy}
}{%
    \message{ERROR: tikz SPY library NOT available. The manual will only compile partially.^^J}%
}%

\usepackage{xparse}% for colorbrewer manual

\usetikzlibrary{
    decorations.markings,
    decorations.footprints,
    shapes.arrows,
    matrix,
    positioning,
}

\usepgfplotslibrary{
    fillbetween,
    ternary,
    smithchart,
    patchplots,
    polar,
    colormaps,
    colorbrewer,
    colortol,           % docu not ready yet
}
\pgfqkeys{/codeexample}{%
    every codeexample/.append style={
        /pgfplots/every ternary axis/.append style={
            /pgfplots/legend style={fill=graphicbackground},
        }
    },
    tabsize=4,
}

\pgfplotsmanualenableexternalizationofexpensive

%\usetikzlibrary{external}
%\tikzexternalize[prefix=figures/]{pgfplots}

\title{%
    Manual for Package \PGFPlots{}\\
    {\small 2D/3D Plots in \LaTeX{}, Version \pgfplotsversion}\\
    {\small\url{https://github.com/pgf-tikz/pgfplots}}
    %\\{\small Attention: you are using an unstable development version.}
}

\makeatletter
\long\def\abstractsmuggle{%
    \centering
    \textbf{Abstract}\\[0.5cm]

    \begin{minipage}{12cm}
        \PGFPlots{} draws high-quality function plots in normal or logarithmic
        scaling with a user-friendly interface directly in \TeX{}. The user
        supplies axis labels, legend entries and the plot coordinates for one or
        more plots and \PGFPlots{} applies axis scaling, computes any logarithms
        and axis ticks and draws the plots. It supports line plots, scatter
        plots, piecewise constant plots, bar plots, area plots, mesh and surface
        plots, patch plots, contour plots, quiver plots, histogram plots, box
        plots, polar axes, ternary diagrams, smith charts and some more. It is
        based on Till Tantau's package \PGF{}/\Tikz{}.
    \end{minipage}

}%

\expandafter\date\expandafter{\@date\\[2cm]
    \abstractsmuggle
}%
\makeatother

%\includeonly{pgfplots.reference}


\begin{document}

\def\plotcoords{%
\addplot coordinates {
(5,8.312e-02)    (17,2.547e-02)   (49,7.407e-03)
(129,2.102e-03)  (321,5.874e-04)  (769,1.623e-04)
(1793,4.442e-05) (4097,1.207e-05) (9217,3.261e-06)
};

\addplot coordinates{
(7,8.472e-02)    (31,3.044e-02)    (111,1.022e-02)
(351,3.303e-03)  (1023,1.039e-03)  (2815,3.196e-04)
(7423,9.658e-05) (18943,2.873e-05) (47103,8.437e-06)
};

\addplot coordinates{
(9,7.881e-02)     (49,3.243e-02)    (209,1.232e-02)
(769,4.454e-03)   (2561,1.551e-03)  (7937,5.236e-04)
(23297,1.723e-04) (65537,5.545e-05) (178177,1.751e-05)
};

\addplot coordinates{
(11,6.887e-02)    (71,3.177e-02)     (351,1.341e-02)
(1471,5.334e-03)  (5503,2.027e-03)   (18943,7.415e-04)
(61183,2.628e-04) (187903,9.063e-05) (553983,3.053e-05)
};

\addplot coordinates{
(13,5.755e-02)     (97,2.925e-02)     (545,1.351e-02)
(2561,5.842e-03)   (10625,2.397e-03)  (40193,9.414e-04)
(141569,3.564e-04) (471041,1.308e-04)
(1496065,4.670e-05)
};
}%


\bookmaketitle
\tableofcontents
\include{pgfplots.title_abstract_intro}
\include{pgfplots.preliminaries}
\include{pgfplots.intro}
\include{pgfplots.reference}
\include{pgfplots.libs}
\include{pgfplots.resources}
\include{pgfplots.importexport}
\include{pgfplots.basic.reference}

\printindex

\bibliographystyle{abbrv} %gerapali} %gerabbrv} %gerunsrt.bst} %gerabbrv}% gerplain}
\nocite{pgfplotstable}
\nocite{programmingnotes}
\bibliography{pgfplots}
\end{document}

% -----------------------------------------------------------------------------
% For Stefan Pinnow as reminder on what to look for when editing the manual
% -----------------------------------------------------------------------------
% There should be no line breaks in the following environments
% - |...|
% - \declareandlabel{...}
% - \verbpdfref{...}
% If "MakeTikzPictures" isn't running through check one of the externalized
% LOG files what is the cause of that.
% -----------------------------------------------------------------------------


%%%%%%%%%%%%%%%%%%%%%%%%%%%%%%%%%%%%%%%%%%%%%%%%%%%%%%%%%%%%%%%%%%%%%%%%%%%%%
%
% Package pgfplots.sty documentation.
%
% Copyright 2007/2008 by Christian Feuersaenger.
%
% This program is free software: you can redistribute it and/or modify
% it under the terms of the GNU General Public License as published by
% the Free Software Foundation, either version 3 of the License, or
% (at your option) any later version.
%
% This program is distributed in the hope that it will be useful,
% but WITHOUT ANY WARRANTY; without even the implied warranty of
% MERCHANTABILITY or FITNESS FOR A PARTICULAR PURPOSE.  See the
% GNU General Public License for more details.
%
% You should have received a copy of the GNU General Public License
% along with this program.  If not, see <http://www.gnu.org/licenses/>.
%
%
%%%%%%%%%%%%%%%%%%%%%%%%%%%%%%%%%%%%%%%%%%%%%%%%%%%%%%%%%%%%%%%%%%%%%%%%%%%%%
% SEE pgfplots-macros.tex as well!
%\pdfminorversion=5 % to allow compression
%\pdfobjcompresslevel=2
\documentclass[a4paper,openany]{book}

\let\bookmaketitle=\maketitle

% -----------------------------------
% this here is from ltxdoc:
\usepackage{doc}[=v2]
\AtBeginDocument{\MakeShortVerb{\|}}
%\setlength{\textwidth}{355pt}
%\addtolength\marginparwidth{30pt}
%\addtolength\oddsidemargin{20pt}
%\addtolength\evensidemargin{20pt}
\def\cmd#1{\cs{\expandafter\cmd@to@cs\string#1}}
\def\cmd@to@cs#1#2{\char\number`#2\relax}
\DeclareRobustCommand\cs[1]{\texttt{\char`\\#1}}
\providecommand\marg[1]{%
  {\ttfamily\char`\{}\meta{#1}{\ttfamily\char`\}}}
\providecommand\oarg[1]{%
  {\ttfamily[}\meta{#1}{\ttfamily]}}
\providecommand\parg[1]{%
  {\ttfamily(}\meta{#1}{\ttfamily)}}
\raggedbottom

% -----------------------------------
\input{pgfplots.preamble.tex}

\makeatletter
% I want a two-column index just as in pgfmanual styles. This here
% was the best way to get one:
\def\index@prologue{\section*{Index}\addcontentsline{toc}{chapter}{Index}
}
\makeatother

%\RequirePackage[german,english,francais]{babel}

\def\matlabcolormaptext{This colormap is similar to one shipped with Matlab$^\text{\textregistered}$ under a similar name.}

\IfFileExists{tikzlibraryspy.code.tex}{%
\usetikzlibrary{spy}
}{%
    \message{ERROR: tikz SPY library NOT available. The manual will only compile partially.^^J}%
}%

\usepackage{xparse}% for colorbrewer manual

\usetikzlibrary{
    decorations.markings,
    decorations.footprints,
    shapes.arrows,
    matrix,
    positioning,
}

\usepgfplotslibrary{
    fillbetween,
    ternary,
    smithchart,
    patchplots,
    polar,
    colormaps,
    colorbrewer,
    colortol,           % docu not ready yet
}
\pgfqkeys{/codeexample}{%
    every codeexample/.append style={
        /pgfplots/every ternary axis/.append style={
            /pgfplots/legend style={fill=graphicbackground},
        }
    },
    tabsize=4,
}

\pgfplotsmanualenableexternalizationofexpensive

%\usetikzlibrary{external}
%\tikzexternalize[prefix=figures/]{pgfplots}

\title{%
    Manual for Package \PGFPlots{}\\
    {\small 2D/3D Plots in \LaTeX{}, Version \pgfplotsversion}\\
    {\small\url{https://github.com/pgf-tikz/pgfplots}}
    %\\{\small Attention: you are using an unstable development version.}
}

\makeatletter
\long\def\abstractsmuggle{%
    \centering
    \textbf{Abstract}\\[0.5cm]

    \begin{minipage}{12cm}
        \PGFPlots{} draws high-quality function plots in normal or logarithmic
        scaling with a user-friendly interface directly in \TeX{}. The user
        supplies axis labels, legend entries and the plot coordinates for one or
        more plots and \PGFPlots{} applies axis scaling, computes any logarithms
        and axis ticks and draws the plots. It supports line plots, scatter
        plots, piecewise constant plots, bar plots, area plots, mesh and surface
        plots, patch plots, contour plots, quiver plots, histogram plots, box
        plots, polar axes, ternary diagrams, smith charts and some more. It is
        based on Till Tantau's package \PGF{}/\Tikz{}.
    \end{minipage}

}%

\expandafter\date\expandafter{\@date\\[2cm]
    \abstractsmuggle
}%
\makeatother

%\includeonly{pgfplots.reference}


\begin{document}

\def\plotcoords{%
\addplot coordinates {
(5,8.312e-02)    (17,2.547e-02)   (49,7.407e-03)
(129,2.102e-03)  (321,5.874e-04)  (769,1.623e-04)
(1793,4.442e-05) (4097,1.207e-05) (9217,3.261e-06)
};

\addplot coordinates{
(7,8.472e-02)    (31,3.044e-02)    (111,1.022e-02)
(351,3.303e-03)  (1023,1.039e-03)  (2815,3.196e-04)
(7423,9.658e-05) (18943,2.873e-05) (47103,8.437e-06)
};

\addplot coordinates{
(9,7.881e-02)     (49,3.243e-02)    (209,1.232e-02)
(769,4.454e-03)   (2561,1.551e-03)  (7937,5.236e-04)
(23297,1.723e-04) (65537,5.545e-05) (178177,1.751e-05)
};

\addplot coordinates{
(11,6.887e-02)    (71,3.177e-02)     (351,1.341e-02)
(1471,5.334e-03)  (5503,2.027e-03)   (18943,7.415e-04)
(61183,2.628e-04) (187903,9.063e-05) (553983,3.053e-05)
};

\addplot coordinates{
(13,5.755e-02)     (97,2.925e-02)     (545,1.351e-02)
(2561,5.842e-03)   (10625,2.397e-03)  (40193,9.414e-04)
(141569,3.564e-04) (471041,1.308e-04)
(1496065,4.670e-05)
};
}%


\bookmaketitle
\tableofcontents
\include{pgfplots.title_abstract_intro}
\include{pgfplots.preliminaries}
\include{pgfplots.intro}
\include{pgfplots.reference}
\include{pgfplots.libs}
\include{pgfplots.resources}
\include{pgfplots.importexport}
\include{pgfplots.basic.reference}

\printindex

\bibliographystyle{abbrv} %gerapali} %gerabbrv} %gerunsrt.bst} %gerabbrv}% gerplain}
\nocite{pgfplotstable}
\nocite{programmingnotes}
\bibliography{pgfplots}
\end{document}

% -----------------------------------------------------------------------------
% For Stefan Pinnow as reminder on what to look for when editing the manual
% -----------------------------------------------------------------------------
% There should be no line breaks in the following environments
% - |...|
% - \declareandlabel{...}
% - \verbpdfref{...}
% If "MakeTikzPictures" isn't running through check one of the externalized
% LOG files what is the cause of that.
% -----------------------------------------------------------------------------


%%%%%%%%%%%%%%%%%%%%%%%%%%%%%%%%%%%%%%%%%%%%%%%%%%%%%%%%%%%%%%%%%%%%%%%%%%%%%
%
% Package pgfplots.sty documentation.
%
% Copyright 2007/2008 by Christian Feuersaenger.
%
% This program is free software: you can redistribute it and/or modify
% it under the terms of the GNU General Public License as published by
% the Free Software Foundation, either version 3 of the License, or
% (at your option) any later version.
%
% This program is distributed in the hope that it will be useful,
% but WITHOUT ANY WARRANTY; without even the implied warranty of
% MERCHANTABILITY or FITNESS FOR A PARTICULAR PURPOSE.  See the
% GNU General Public License for more details.
%
% You should have received a copy of the GNU General Public License
% along with this program.  If not, see <http://www.gnu.org/licenses/>.
%
%
%%%%%%%%%%%%%%%%%%%%%%%%%%%%%%%%%%%%%%%%%%%%%%%%%%%%%%%%%%%%%%%%%%%%%%%%%%%%%
% SEE pgfplots-macros.tex as well!
%\pdfminorversion=5 % to allow compression
%\pdfobjcompresslevel=2
\documentclass[a4paper,openany]{book}

\let\bookmaketitle=\maketitle

% -----------------------------------
% this here is from ltxdoc:
\usepackage{doc}[=v2]
\AtBeginDocument{\MakeShortVerb{\|}}
%\setlength{\textwidth}{355pt}
%\addtolength\marginparwidth{30pt}
%\addtolength\oddsidemargin{20pt}
%\addtolength\evensidemargin{20pt}
\def\cmd#1{\cs{\expandafter\cmd@to@cs\string#1}}
\def\cmd@to@cs#1#2{\char\number`#2\relax}
\DeclareRobustCommand\cs[1]{\texttt{\char`\\#1}}
\providecommand\marg[1]{%
  {\ttfamily\char`\{}\meta{#1}{\ttfamily\char`\}}}
\providecommand\oarg[1]{%
  {\ttfamily[}\meta{#1}{\ttfamily]}}
\providecommand\parg[1]{%
  {\ttfamily(}\meta{#1}{\ttfamily)}}
\raggedbottom

% -----------------------------------
\input{pgfplots.preamble.tex}

\makeatletter
% I want a two-column index just as in pgfmanual styles. This here
% was the best way to get one:
\def\index@prologue{\section*{Index}\addcontentsline{toc}{chapter}{Index}
}
\makeatother

%\RequirePackage[german,english,francais]{babel}

\def\matlabcolormaptext{This colormap is similar to one shipped with Matlab$^\text{\textregistered}$ under a similar name.}

\IfFileExists{tikzlibraryspy.code.tex}{%
\usetikzlibrary{spy}
}{%
    \message{ERROR: tikz SPY library NOT available. The manual will only compile partially.^^J}%
}%

\usepackage{xparse}% for colorbrewer manual

\usetikzlibrary{
    decorations.markings,
    decorations.footprints,
    shapes.arrows,
    matrix,
    positioning,
}

\usepgfplotslibrary{
    fillbetween,
    ternary,
    smithchart,
    patchplots,
    polar,
    colormaps,
    colorbrewer,
    colortol,           % docu not ready yet
}
\pgfqkeys{/codeexample}{%
    every codeexample/.append style={
        /pgfplots/every ternary axis/.append style={
            /pgfplots/legend style={fill=graphicbackground},
        }
    },
    tabsize=4,
}

\pgfplotsmanualenableexternalizationofexpensive

%\usetikzlibrary{external}
%\tikzexternalize[prefix=figures/]{pgfplots}

\title{%
    Manual for Package \PGFPlots{}\\
    {\small 2D/3D Plots in \LaTeX{}, Version \pgfplotsversion}\\
    {\small\url{https://github.com/pgf-tikz/pgfplots}}
    %\\{\small Attention: you are using an unstable development version.}
}

\makeatletter
\long\def\abstractsmuggle{%
    \centering
    \textbf{Abstract}\\[0.5cm]

    \begin{minipage}{12cm}
        \PGFPlots{} draws high-quality function plots in normal or logarithmic
        scaling with a user-friendly interface directly in \TeX{}. The user
        supplies axis labels, legend entries and the plot coordinates for one or
        more plots and \PGFPlots{} applies axis scaling, computes any logarithms
        and axis ticks and draws the plots. It supports line plots, scatter
        plots, piecewise constant plots, bar plots, area plots, mesh and surface
        plots, patch plots, contour plots, quiver plots, histogram plots, box
        plots, polar axes, ternary diagrams, smith charts and some more. It is
        based on Till Tantau's package \PGF{}/\Tikz{}.
    \end{minipage}

}%

\expandafter\date\expandafter{\@date\\[2cm]
    \abstractsmuggle
}%
\makeatother

%\includeonly{pgfplots.reference}


\begin{document}

\def\plotcoords{%
\addplot coordinates {
(5,8.312e-02)    (17,2.547e-02)   (49,7.407e-03)
(129,2.102e-03)  (321,5.874e-04)  (769,1.623e-04)
(1793,4.442e-05) (4097,1.207e-05) (9217,3.261e-06)
};

\addplot coordinates{
(7,8.472e-02)    (31,3.044e-02)    (111,1.022e-02)
(351,3.303e-03)  (1023,1.039e-03)  (2815,3.196e-04)
(7423,9.658e-05) (18943,2.873e-05) (47103,8.437e-06)
};

\addplot coordinates{
(9,7.881e-02)     (49,3.243e-02)    (209,1.232e-02)
(769,4.454e-03)   (2561,1.551e-03)  (7937,5.236e-04)
(23297,1.723e-04) (65537,5.545e-05) (178177,1.751e-05)
};

\addplot coordinates{
(11,6.887e-02)    (71,3.177e-02)     (351,1.341e-02)
(1471,5.334e-03)  (5503,2.027e-03)   (18943,7.415e-04)
(61183,2.628e-04) (187903,9.063e-05) (553983,3.053e-05)
};

\addplot coordinates{
(13,5.755e-02)     (97,2.925e-02)     (545,1.351e-02)
(2561,5.842e-03)   (10625,2.397e-03)  (40193,9.414e-04)
(141569,3.564e-04) (471041,1.308e-04)
(1496065,4.670e-05)
};
}%


\bookmaketitle
\tableofcontents
\include{pgfplots.title_abstract_intro}
\include{pgfplots.preliminaries}
\include{pgfplots.intro}
\include{pgfplots.reference}
\include{pgfplots.libs}
\include{pgfplots.resources}
\include{pgfplots.importexport}
\include{pgfplots.basic.reference}

\printindex

\bibliographystyle{abbrv} %gerapali} %gerabbrv} %gerunsrt.bst} %gerabbrv}% gerplain}
\nocite{pgfplotstable}
\nocite{programmingnotes}
\bibliography{pgfplots}
\end{document}

% -----------------------------------------------------------------------------
% For Stefan Pinnow as reminder on what to look for when editing the manual
% -----------------------------------------------------------------------------
% There should be no line breaks in the following environments
% - |...|
% - \declareandlabel{...}
% - \verbpdfref{...}
% If "MakeTikzPictures" isn't running through check one of the externalized
% LOG files what is the cause of that.
% -----------------------------------------------------------------------------

\include{pgfplots.basic.reference}

\printindex

\bibliographystyle{abbrv} %gerapali} %gerabbrv} %gerunsrt.bst} %gerabbrv}% gerplain}
\nocite{pgfplotstable}
\nocite{programmingnotes}
\bibliography{pgfplots}
\end{document}

% -----------------------------------------------------------------------------
% For Stefan Pinnow as reminder on what to look for when editing the manual
% -----------------------------------------------------------------------------
% There should be no line breaks in the following environments
% - |...|
% - \declareandlabel{...}
% - \verbpdfref{...}
% If "MakeTikzPictures" isn't running through check one of the externalized
% LOG files what is the cause of that.
% -----------------------------------------------------------------------------


%%%%%%%%%%%%%%%%%%%%%%%%%%%%%%%%%%%%%%%%%%%%%%%%%%%%%%%%%%%%%%%%%%%%%%%%%%%%%
%
% Package pgfplots.sty documentation.
%
% Copyright 2007/2008 by Christian Feuersaenger.
%
% This program is free software: you can redistribute it and/or modify
% it under the terms of the GNU General Public License as published by
% the Free Software Foundation, either version 3 of the License, or
% (at your option) any later version.
%
% This program is distributed in the hope that it will be useful,
% but WITHOUT ANY WARRANTY; without even the implied warranty of
% MERCHANTABILITY or FITNESS FOR A PARTICULAR PURPOSE.  See the
% GNU General Public License for more details.
%
% You should have received a copy of the GNU General Public License
% along with this program.  If not, see <http://www.gnu.org/licenses/>.
%
%
%%%%%%%%%%%%%%%%%%%%%%%%%%%%%%%%%%%%%%%%%%%%%%%%%%%%%%%%%%%%%%%%%%%%%%%%%%%%%
% SEE pgfplots-macros.tex as well!
%\pdfminorversion=5 % to allow compression
%\pdfobjcompresslevel=2
\documentclass[a4paper,openany]{book}

\let\bookmaketitle=\maketitle

% -----------------------------------
% this here is from ltxdoc:
\usepackage{doc}[=v2]
\AtBeginDocument{\MakeShortVerb{\|}}
%\setlength{\textwidth}{355pt}
%\addtolength\marginparwidth{30pt}
%\addtolength\oddsidemargin{20pt}
%\addtolength\evensidemargin{20pt}
\def\cmd#1{\cs{\expandafter\cmd@to@cs\string#1}}
\def\cmd@to@cs#1#2{\char\number`#2\relax}
\DeclareRobustCommand\cs[1]{\texttt{\char`\\#1}}
\providecommand\marg[1]{%
  {\ttfamily\char`\{}\meta{#1}{\ttfamily\char`\}}}
\providecommand\oarg[1]{%
  {\ttfamily[}\meta{#1}{\ttfamily]}}
\providecommand\parg[1]{%
  {\ttfamily(}\meta{#1}{\ttfamily)}}
\raggedbottom

% -----------------------------------
\input{pgfplots.preamble.tex}

\makeatletter
% I want a two-column index just as in pgfmanual styles. This here
% was the best way to get one:
\def\index@prologue{\section*{Index}\addcontentsline{toc}{chapter}{Index}
}
\makeatother

%\RequirePackage[german,english,francais]{babel}

\def\matlabcolormaptext{This colormap is similar to one shipped with Matlab$^\text{\textregistered}$ under a similar name.}

\IfFileExists{tikzlibraryspy.code.tex}{%
\usetikzlibrary{spy}
}{%
    \message{ERROR: tikz SPY library NOT available. The manual will only compile partially.^^J}%
}%

\usepackage{xparse}% for colorbrewer manual

\usetikzlibrary{
    decorations.markings,
    decorations.footprints,
    shapes.arrows,
    matrix,
    positioning,
}

\usepgfplotslibrary{
    fillbetween,
    ternary,
    smithchart,
    patchplots,
    polar,
    colormaps,
    colorbrewer,
    colortol,           % docu not ready yet
}
\pgfqkeys{/codeexample}{%
    every codeexample/.append style={
        /pgfplots/every ternary axis/.append style={
            /pgfplots/legend style={fill=graphicbackground},
        }
    },
    tabsize=4,
}

\pgfplotsmanualenableexternalizationofexpensive

%\usetikzlibrary{external}
%\tikzexternalize[prefix=figures/]{pgfplots}

\title{%
    Manual for Package \PGFPlots{}\\
    {\small 2D/3D Plots in \LaTeX{}, Version \pgfplotsversion}\\
    {\small\url{https://github.com/pgf-tikz/pgfplots}}
    %\\{\small Attention: you are using an unstable development version.}
}

\makeatletter
\long\def\abstractsmuggle{%
    \centering
    \textbf{Abstract}\\[0.5cm]

    \begin{minipage}{12cm}
        \PGFPlots{} draws high-quality function plots in normal or logarithmic
        scaling with a user-friendly interface directly in \TeX{}. The user
        supplies axis labels, legend entries and the plot coordinates for one or
        more plots and \PGFPlots{} applies axis scaling, computes any logarithms
        and axis ticks and draws the plots. It supports line plots, scatter
        plots, piecewise constant plots, bar plots, area plots, mesh and surface
        plots, patch plots, contour plots, quiver plots, histogram plots, box
        plots, polar axes, ternary diagrams, smith charts and some more. It is
        based on Till Tantau's package \PGF{}/\Tikz{}.
    \end{minipage}

}%

\expandafter\date\expandafter{\@date\\[2cm]
    \abstractsmuggle
}%
\makeatother

%\includeonly{pgfplots.reference}


\begin{document}

\def\plotcoords{%
\addplot coordinates {
(5,8.312e-02)    (17,2.547e-02)   (49,7.407e-03)
(129,2.102e-03)  (321,5.874e-04)  (769,1.623e-04)
(1793,4.442e-05) (4097,1.207e-05) (9217,3.261e-06)
};

\addplot coordinates{
(7,8.472e-02)    (31,3.044e-02)    (111,1.022e-02)
(351,3.303e-03)  (1023,1.039e-03)  (2815,3.196e-04)
(7423,9.658e-05) (18943,2.873e-05) (47103,8.437e-06)
};

\addplot coordinates{
(9,7.881e-02)     (49,3.243e-02)    (209,1.232e-02)
(769,4.454e-03)   (2561,1.551e-03)  (7937,5.236e-04)
(23297,1.723e-04) (65537,5.545e-05) (178177,1.751e-05)
};

\addplot coordinates{
(11,6.887e-02)    (71,3.177e-02)     (351,1.341e-02)
(1471,5.334e-03)  (5503,2.027e-03)   (18943,7.415e-04)
(61183,2.628e-04) (187903,9.063e-05) (553983,3.053e-05)
};

\addplot coordinates{
(13,5.755e-02)     (97,2.925e-02)     (545,1.351e-02)
(2561,5.842e-03)   (10625,2.397e-03)  (40193,9.414e-04)
(141569,3.564e-04) (471041,1.308e-04)
(1496065,4.670e-05)
};
}%


\bookmaketitle
\tableofcontents

%%%%%%%%%%%%%%%%%%%%%%%%%%%%%%%%%%%%%%%%%%%%%%%%%%%%%%%%%%%%%%%%%%%%%%%%%%%%%
%
% Package pgfplots.sty documentation.
%
% Copyright 2007/2008 by Christian Feuersaenger.
%
% This program is free software: you can redistribute it and/or modify
% it under the terms of the GNU General Public License as published by
% the Free Software Foundation, either version 3 of the License, or
% (at your option) any later version.
%
% This program is distributed in the hope that it will be useful,
% but WITHOUT ANY WARRANTY; without even the implied warranty of
% MERCHANTABILITY or FITNESS FOR A PARTICULAR PURPOSE.  See the
% GNU General Public License for more details.
%
% You should have received a copy of the GNU General Public License
% along with this program.  If not, see <http://www.gnu.org/licenses/>.
%
%
%%%%%%%%%%%%%%%%%%%%%%%%%%%%%%%%%%%%%%%%%%%%%%%%%%%%%%%%%%%%%%%%%%%%%%%%%%%%%
% SEE pgfplots-macros.tex as well!
%\pdfminorversion=5 % to allow compression
%\pdfobjcompresslevel=2
\documentclass[a4paper,openany]{book}

\let\bookmaketitle=\maketitle

% -----------------------------------
% this here is from ltxdoc:
\usepackage{doc}[=v2]
\AtBeginDocument{\MakeShortVerb{\|}}
%\setlength{\textwidth}{355pt}
%\addtolength\marginparwidth{30pt}
%\addtolength\oddsidemargin{20pt}
%\addtolength\evensidemargin{20pt}
\def\cmd#1{\cs{\expandafter\cmd@to@cs\string#1}}
\def\cmd@to@cs#1#2{\char\number`#2\relax}
\DeclareRobustCommand\cs[1]{\texttt{\char`\\#1}}
\providecommand\marg[1]{%
  {\ttfamily\char`\{}\meta{#1}{\ttfamily\char`\}}}
\providecommand\oarg[1]{%
  {\ttfamily[}\meta{#1}{\ttfamily]}}
\providecommand\parg[1]{%
  {\ttfamily(}\meta{#1}{\ttfamily)}}
\raggedbottom

% -----------------------------------
\input{pgfplots.preamble.tex}

\makeatletter
% I want a two-column index just as in pgfmanual styles. This here
% was the best way to get one:
\def\index@prologue{\section*{Index}\addcontentsline{toc}{chapter}{Index}
}
\makeatother

%\RequirePackage[german,english,francais]{babel}

\def\matlabcolormaptext{This colormap is similar to one shipped with Matlab$^\text{\textregistered}$ under a similar name.}

\IfFileExists{tikzlibraryspy.code.tex}{%
\usetikzlibrary{spy}
}{%
    \message{ERROR: tikz SPY library NOT available. The manual will only compile partially.^^J}%
}%

\usepackage{xparse}% for colorbrewer manual

\usetikzlibrary{
    decorations.markings,
    decorations.footprints,
    shapes.arrows,
    matrix,
    positioning,
}

\usepgfplotslibrary{
    fillbetween,
    ternary,
    smithchart,
    patchplots,
    polar,
    colormaps,
    colorbrewer,
    colortol,           % docu not ready yet
}
\pgfqkeys{/codeexample}{%
    every codeexample/.append style={
        /pgfplots/every ternary axis/.append style={
            /pgfplots/legend style={fill=graphicbackground},
        }
    },
    tabsize=4,
}

\pgfplotsmanualenableexternalizationofexpensive

%\usetikzlibrary{external}
%\tikzexternalize[prefix=figures/]{pgfplots}

\title{%
    Manual for Package \PGFPlots{}\\
    {\small 2D/3D Plots in \LaTeX{}, Version \pgfplotsversion}\\
    {\small\url{https://github.com/pgf-tikz/pgfplots}}
    %\\{\small Attention: you are using an unstable development version.}
}

\makeatletter
\long\def\abstractsmuggle{%
    \centering
    \textbf{Abstract}\\[0.5cm]

    \begin{minipage}{12cm}
        \PGFPlots{} draws high-quality function plots in normal or logarithmic
        scaling with a user-friendly interface directly in \TeX{}. The user
        supplies axis labels, legend entries and the plot coordinates for one or
        more plots and \PGFPlots{} applies axis scaling, computes any logarithms
        and axis ticks and draws the plots. It supports line plots, scatter
        plots, piecewise constant plots, bar plots, area plots, mesh and surface
        plots, patch plots, contour plots, quiver plots, histogram plots, box
        plots, polar axes, ternary diagrams, smith charts and some more. It is
        based on Till Tantau's package \PGF{}/\Tikz{}.
    \end{minipage}

}%

\expandafter\date\expandafter{\@date\\[2cm]
    \abstractsmuggle
}%
\makeatother

%\includeonly{pgfplots.reference}


\begin{document}

\def\plotcoords{%
\addplot coordinates {
(5,8.312e-02)    (17,2.547e-02)   (49,7.407e-03)
(129,2.102e-03)  (321,5.874e-04)  (769,1.623e-04)
(1793,4.442e-05) (4097,1.207e-05) (9217,3.261e-06)
};

\addplot coordinates{
(7,8.472e-02)    (31,3.044e-02)    (111,1.022e-02)
(351,3.303e-03)  (1023,1.039e-03)  (2815,3.196e-04)
(7423,9.658e-05) (18943,2.873e-05) (47103,8.437e-06)
};

\addplot coordinates{
(9,7.881e-02)     (49,3.243e-02)    (209,1.232e-02)
(769,4.454e-03)   (2561,1.551e-03)  (7937,5.236e-04)
(23297,1.723e-04) (65537,5.545e-05) (178177,1.751e-05)
};

\addplot coordinates{
(11,6.887e-02)    (71,3.177e-02)     (351,1.341e-02)
(1471,5.334e-03)  (5503,2.027e-03)   (18943,7.415e-04)
(61183,2.628e-04) (187903,9.063e-05) (553983,3.053e-05)
};

\addplot coordinates{
(13,5.755e-02)     (97,2.925e-02)     (545,1.351e-02)
(2561,5.842e-03)   (10625,2.397e-03)  (40193,9.414e-04)
(141569,3.564e-04) (471041,1.308e-04)
(1496065,4.670e-05)
};
}%


\bookmaketitle
\tableofcontents
\include{pgfplots.title_abstract_intro}
\include{pgfplots.preliminaries}
\include{pgfplots.intro}
\include{pgfplots.reference}
\include{pgfplots.libs}
\include{pgfplots.resources}
\include{pgfplots.importexport}
\include{pgfplots.basic.reference}

\printindex

\bibliographystyle{abbrv} %gerapali} %gerabbrv} %gerunsrt.bst} %gerabbrv}% gerplain}
\nocite{pgfplotstable}
\nocite{programmingnotes}
\bibliography{pgfplots}
\end{document}

% -----------------------------------------------------------------------------
% For Stefan Pinnow as reminder on what to look for when editing the manual
% -----------------------------------------------------------------------------
% There should be no line breaks in the following environments
% - |...|
% - \declareandlabel{...}
% - \verbpdfref{...}
% If "MakeTikzPictures" isn't running through check one of the externalized
% LOG files what is the cause of that.
% -----------------------------------------------------------------------------


%%%%%%%%%%%%%%%%%%%%%%%%%%%%%%%%%%%%%%%%%%%%%%%%%%%%%%%%%%%%%%%%%%%%%%%%%%%%%
%
% Package pgfplots.sty documentation.
%
% Copyright 2007/2008 by Christian Feuersaenger.
%
% This program is free software: you can redistribute it and/or modify
% it under the terms of the GNU General Public License as published by
% the Free Software Foundation, either version 3 of the License, or
% (at your option) any later version.
%
% This program is distributed in the hope that it will be useful,
% but WITHOUT ANY WARRANTY; without even the implied warranty of
% MERCHANTABILITY or FITNESS FOR A PARTICULAR PURPOSE.  See the
% GNU General Public License for more details.
%
% You should have received a copy of the GNU General Public License
% along with this program.  If not, see <http://www.gnu.org/licenses/>.
%
%
%%%%%%%%%%%%%%%%%%%%%%%%%%%%%%%%%%%%%%%%%%%%%%%%%%%%%%%%%%%%%%%%%%%%%%%%%%%%%
% SEE pgfplots-macros.tex as well!
%\pdfminorversion=5 % to allow compression
%\pdfobjcompresslevel=2
\documentclass[a4paper,openany]{book}

\let\bookmaketitle=\maketitle

% -----------------------------------
% this here is from ltxdoc:
\usepackage{doc}[=v2]
\AtBeginDocument{\MakeShortVerb{\|}}
%\setlength{\textwidth}{355pt}
%\addtolength\marginparwidth{30pt}
%\addtolength\oddsidemargin{20pt}
%\addtolength\evensidemargin{20pt}
\def\cmd#1{\cs{\expandafter\cmd@to@cs\string#1}}
\def\cmd@to@cs#1#2{\char\number`#2\relax}
\DeclareRobustCommand\cs[1]{\texttt{\char`\\#1}}
\providecommand\marg[1]{%
  {\ttfamily\char`\{}\meta{#1}{\ttfamily\char`\}}}
\providecommand\oarg[1]{%
  {\ttfamily[}\meta{#1}{\ttfamily]}}
\providecommand\parg[1]{%
  {\ttfamily(}\meta{#1}{\ttfamily)}}
\raggedbottom

% -----------------------------------
\input{pgfplots.preamble.tex}

\makeatletter
% I want a two-column index just as in pgfmanual styles. This here
% was the best way to get one:
\def\index@prologue{\section*{Index}\addcontentsline{toc}{chapter}{Index}
}
\makeatother

%\RequirePackage[german,english,francais]{babel}

\def\matlabcolormaptext{This colormap is similar to one shipped with Matlab$^\text{\textregistered}$ under a similar name.}

\IfFileExists{tikzlibraryspy.code.tex}{%
\usetikzlibrary{spy}
}{%
    \message{ERROR: tikz SPY library NOT available. The manual will only compile partially.^^J}%
}%

\usepackage{xparse}% for colorbrewer manual

\usetikzlibrary{
    decorations.markings,
    decorations.footprints,
    shapes.arrows,
    matrix,
    positioning,
}

\usepgfplotslibrary{
    fillbetween,
    ternary,
    smithchart,
    patchplots,
    polar,
    colormaps,
    colorbrewer,
    colortol,           % docu not ready yet
}
\pgfqkeys{/codeexample}{%
    every codeexample/.append style={
        /pgfplots/every ternary axis/.append style={
            /pgfplots/legend style={fill=graphicbackground},
        }
    },
    tabsize=4,
}

\pgfplotsmanualenableexternalizationofexpensive

%\usetikzlibrary{external}
%\tikzexternalize[prefix=figures/]{pgfplots}

\title{%
    Manual for Package \PGFPlots{}\\
    {\small 2D/3D Plots in \LaTeX{}, Version \pgfplotsversion}\\
    {\small\url{https://github.com/pgf-tikz/pgfplots}}
    %\\{\small Attention: you are using an unstable development version.}
}

\makeatletter
\long\def\abstractsmuggle{%
    \centering
    \textbf{Abstract}\\[0.5cm]

    \begin{minipage}{12cm}
        \PGFPlots{} draws high-quality function plots in normal or logarithmic
        scaling with a user-friendly interface directly in \TeX{}. The user
        supplies axis labels, legend entries and the plot coordinates for one or
        more plots and \PGFPlots{} applies axis scaling, computes any logarithms
        and axis ticks and draws the plots. It supports line plots, scatter
        plots, piecewise constant plots, bar plots, area plots, mesh and surface
        plots, patch plots, contour plots, quiver plots, histogram plots, box
        plots, polar axes, ternary diagrams, smith charts and some more. It is
        based on Till Tantau's package \PGF{}/\Tikz{}.
    \end{minipage}

}%

\expandafter\date\expandafter{\@date\\[2cm]
    \abstractsmuggle
}%
\makeatother

%\includeonly{pgfplots.reference}


\begin{document}

\def\plotcoords{%
\addplot coordinates {
(5,8.312e-02)    (17,2.547e-02)   (49,7.407e-03)
(129,2.102e-03)  (321,5.874e-04)  (769,1.623e-04)
(1793,4.442e-05) (4097,1.207e-05) (9217,3.261e-06)
};

\addplot coordinates{
(7,8.472e-02)    (31,3.044e-02)    (111,1.022e-02)
(351,3.303e-03)  (1023,1.039e-03)  (2815,3.196e-04)
(7423,9.658e-05) (18943,2.873e-05) (47103,8.437e-06)
};

\addplot coordinates{
(9,7.881e-02)     (49,3.243e-02)    (209,1.232e-02)
(769,4.454e-03)   (2561,1.551e-03)  (7937,5.236e-04)
(23297,1.723e-04) (65537,5.545e-05) (178177,1.751e-05)
};

\addplot coordinates{
(11,6.887e-02)    (71,3.177e-02)     (351,1.341e-02)
(1471,5.334e-03)  (5503,2.027e-03)   (18943,7.415e-04)
(61183,2.628e-04) (187903,9.063e-05) (553983,3.053e-05)
};

\addplot coordinates{
(13,5.755e-02)     (97,2.925e-02)     (545,1.351e-02)
(2561,5.842e-03)   (10625,2.397e-03)  (40193,9.414e-04)
(141569,3.564e-04) (471041,1.308e-04)
(1496065,4.670e-05)
};
}%


\bookmaketitle
\tableofcontents
\include{pgfplots.title_abstract_intro}
\include{pgfplots.preliminaries}
\include{pgfplots.intro}
\include{pgfplots.reference}
\include{pgfplots.libs}
\include{pgfplots.resources}
\include{pgfplots.importexport}
\include{pgfplots.basic.reference}

\printindex

\bibliographystyle{abbrv} %gerapali} %gerabbrv} %gerunsrt.bst} %gerabbrv}% gerplain}
\nocite{pgfplotstable}
\nocite{programmingnotes}
\bibliography{pgfplots}
\end{document}

% -----------------------------------------------------------------------------
% For Stefan Pinnow as reminder on what to look for when editing the manual
% -----------------------------------------------------------------------------
% There should be no line breaks in the following environments
% - |...|
% - \declareandlabel{...}
% - \verbpdfref{...}
% If "MakeTikzPictures" isn't running through check one of the externalized
% LOG files what is the cause of that.
% -----------------------------------------------------------------------------


%%%%%%%%%%%%%%%%%%%%%%%%%%%%%%%%%%%%%%%%%%%%%%%%%%%%%%%%%%%%%%%%%%%%%%%%%%%%%
%
% Package pgfplots.sty documentation.
%
% Copyright 2007/2008 by Christian Feuersaenger.
%
% This program is free software: you can redistribute it and/or modify
% it under the terms of the GNU General Public License as published by
% the Free Software Foundation, either version 3 of the License, or
% (at your option) any later version.
%
% This program is distributed in the hope that it will be useful,
% but WITHOUT ANY WARRANTY; without even the implied warranty of
% MERCHANTABILITY or FITNESS FOR A PARTICULAR PURPOSE.  See the
% GNU General Public License for more details.
%
% You should have received a copy of the GNU General Public License
% along with this program.  If not, see <http://www.gnu.org/licenses/>.
%
%
%%%%%%%%%%%%%%%%%%%%%%%%%%%%%%%%%%%%%%%%%%%%%%%%%%%%%%%%%%%%%%%%%%%%%%%%%%%%%
% SEE pgfplots-macros.tex as well!
%\pdfminorversion=5 % to allow compression
%\pdfobjcompresslevel=2
\documentclass[a4paper,openany]{book}

\let\bookmaketitle=\maketitle

% -----------------------------------
% this here is from ltxdoc:
\usepackage{doc}[=v2]
\AtBeginDocument{\MakeShortVerb{\|}}
%\setlength{\textwidth}{355pt}
%\addtolength\marginparwidth{30pt}
%\addtolength\oddsidemargin{20pt}
%\addtolength\evensidemargin{20pt}
\def\cmd#1{\cs{\expandafter\cmd@to@cs\string#1}}
\def\cmd@to@cs#1#2{\char\number`#2\relax}
\DeclareRobustCommand\cs[1]{\texttt{\char`\\#1}}
\providecommand\marg[1]{%
  {\ttfamily\char`\{}\meta{#1}{\ttfamily\char`\}}}
\providecommand\oarg[1]{%
  {\ttfamily[}\meta{#1}{\ttfamily]}}
\providecommand\parg[1]{%
  {\ttfamily(}\meta{#1}{\ttfamily)}}
\raggedbottom

% -----------------------------------
\input{pgfplots.preamble.tex}

\makeatletter
% I want a two-column index just as in pgfmanual styles. This here
% was the best way to get one:
\def\index@prologue{\section*{Index}\addcontentsline{toc}{chapter}{Index}
}
\makeatother

%\RequirePackage[german,english,francais]{babel}

\def\matlabcolormaptext{This colormap is similar to one shipped with Matlab$^\text{\textregistered}$ under a similar name.}

\IfFileExists{tikzlibraryspy.code.tex}{%
\usetikzlibrary{spy}
}{%
    \message{ERROR: tikz SPY library NOT available. The manual will only compile partially.^^J}%
}%

\usepackage{xparse}% for colorbrewer manual

\usetikzlibrary{
    decorations.markings,
    decorations.footprints,
    shapes.arrows,
    matrix,
    positioning,
}

\usepgfplotslibrary{
    fillbetween,
    ternary,
    smithchart,
    patchplots,
    polar,
    colormaps,
    colorbrewer,
    colortol,           % docu not ready yet
}
\pgfqkeys{/codeexample}{%
    every codeexample/.append style={
        /pgfplots/every ternary axis/.append style={
            /pgfplots/legend style={fill=graphicbackground},
        }
    },
    tabsize=4,
}

\pgfplotsmanualenableexternalizationofexpensive

%\usetikzlibrary{external}
%\tikzexternalize[prefix=figures/]{pgfplots}

\title{%
    Manual for Package \PGFPlots{}\\
    {\small 2D/3D Plots in \LaTeX{}, Version \pgfplotsversion}\\
    {\small\url{https://github.com/pgf-tikz/pgfplots}}
    %\\{\small Attention: you are using an unstable development version.}
}

\makeatletter
\long\def\abstractsmuggle{%
    \centering
    \textbf{Abstract}\\[0.5cm]

    \begin{minipage}{12cm}
        \PGFPlots{} draws high-quality function plots in normal or logarithmic
        scaling with a user-friendly interface directly in \TeX{}. The user
        supplies axis labels, legend entries and the plot coordinates for one or
        more plots and \PGFPlots{} applies axis scaling, computes any logarithms
        and axis ticks and draws the plots. It supports line plots, scatter
        plots, piecewise constant plots, bar plots, area plots, mesh and surface
        plots, patch plots, contour plots, quiver plots, histogram plots, box
        plots, polar axes, ternary diagrams, smith charts and some more. It is
        based on Till Tantau's package \PGF{}/\Tikz{}.
    \end{minipage}

}%

\expandafter\date\expandafter{\@date\\[2cm]
    \abstractsmuggle
}%
\makeatother

%\includeonly{pgfplots.reference}


\begin{document}

\def\plotcoords{%
\addplot coordinates {
(5,8.312e-02)    (17,2.547e-02)   (49,7.407e-03)
(129,2.102e-03)  (321,5.874e-04)  (769,1.623e-04)
(1793,4.442e-05) (4097,1.207e-05) (9217,3.261e-06)
};

\addplot coordinates{
(7,8.472e-02)    (31,3.044e-02)    (111,1.022e-02)
(351,3.303e-03)  (1023,1.039e-03)  (2815,3.196e-04)
(7423,9.658e-05) (18943,2.873e-05) (47103,8.437e-06)
};

\addplot coordinates{
(9,7.881e-02)     (49,3.243e-02)    (209,1.232e-02)
(769,4.454e-03)   (2561,1.551e-03)  (7937,5.236e-04)
(23297,1.723e-04) (65537,5.545e-05) (178177,1.751e-05)
};

\addplot coordinates{
(11,6.887e-02)    (71,3.177e-02)     (351,1.341e-02)
(1471,5.334e-03)  (5503,2.027e-03)   (18943,7.415e-04)
(61183,2.628e-04) (187903,9.063e-05) (553983,3.053e-05)
};

\addplot coordinates{
(13,5.755e-02)     (97,2.925e-02)     (545,1.351e-02)
(2561,5.842e-03)   (10625,2.397e-03)  (40193,9.414e-04)
(141569,3.564e-04) (471041,1.308e-04)
(1496065,4.670e-05)
};
}%


\bookmaketitle
\tableofcontents
\include{pgfplots.title_abstract_intro}
\include{pgfplots.preliminaries}
\include{pgfplots.intro}
\include{pgfplots.reference}
\include{pgfplots.libs}
\include{pgfplots.resources}
\include{pgfplots.importexport}
\include{pgfplots.basic.reference}

\printindex

\bibliographystyle{abbrv} %gerapali} %gerabbrv} %gerunsrt.bst} %gerabbrv}% gerplain}
\nocite{pgfplotstable}
\nocite{programmingnotes}
\bibliography{pgfplots}
\end{document}

% -----------------------------------------------------------------------------
% For Stefan Pinnow as reminder on what to look for when editing the manual
% -----------------------------------------------------------------------------
% There should be no line breaks in the following environments
% - |...|
% - \declareandlabel{...}
% - \verbpdfref{...}
% If "MakeTikzPictures" isn't running through check one of the externalized
% LOG files what is the cause of that.
% -----------------------------------------------------------------------------


%%%%%%%%%%%%%%%%%%%%%%%%%%%%%%%%%%%%%%%%%%%%%%%%%%%%%%%%%%%%%%%%%%%%%%%%%%%%%
%
% Package pgfplots.sty documentation.
%
% Copyright 2007/2008 by Christian Feuersaenger.
%
% This program is free software: you can redistribute it and/or modify
% it under the terms of the GNU General Public License as published by
% the Free Software Foundation, either version 3 of the License, or
% (at your option) any later version.
%
% This program is distributed in the hope that it will be useful,
% but WITHOUT ANY WARRANTY; without even the implied warranty of
% MERCHANTABILITY or FITNESS FOR A PARTICULAR PURPOSE.  See the
% GNU General Public License for more details.
%
% You should have received a copy of the GNU General Public License
% along with this program.  If not, see <http://www.gnu.org/licenses/>.
%
%
%%%%%%%%%%%%%%%%%%%%%%%%%%%%%%%%%%%%%%%%%%%%%%%%%%%%%%%%%%%%%%%%%%%%%%%%%%%%%
% SEE pgfplots-macros.tex as well!
%\pdfminorversion=5 % to allow compression
%\pdfobjcompresslevel=2
\documentclass[a4paper,openany]{book}

\let\bookmaketitle=\maketitle

% -----------------------------------
% this here is from ltxdoc:
\usepackage{doc}[=v2]
\AtBeginDocument{\MakeShortVerb{\|}}
%\setlength{\textwidth}{355pt}
%\addtolength\marginparwidth{30pt}
%\addtolength\oddsidemargin{20pt}
%\addtolength\evensidemargin{20pt}
\def\cmd#1{\cs{\expandafter\cmd@to@cs\string#1}}
\def\cmd@to@cs#1#2{\char\number`#2\relax}
\DeclareRobustCommand\cs[1]{\texttt{\char`\\#1}}
\providecommand\marg[1]{%
  {\ttfamily\char`\{}\meta{#1}{\ttfamily\char`\}}}
\providecommand\oarg[1]{%
  {\ttfamily[}\meta{#1}{\ttfamily]}}
\providecommand\parg[1]{%
  {\ttfamily(}\meta{#1}{\ttfamily)}}
\raggedbottom

% -----------------------------------
\input{pgfplots.preamble.tex}

\makeatletter
% I want a two-column index just as in pgfmanual styles. This here
% was the best way to get one:
\def\index@prologue{\section*{Index}\addcontentsline{toc}{chapter}{Index}
}
\makeatother

%\RequirePackage[german,english,francais]{babel}

\def\matlabcolormaptext{This colormap is similar to one shipped with Matlab$^\text{\textregistered}$ under a similar name.}

\IfFileExists{tikzlibraryspy.code.tex}{%
\usetikzlibrary{spy}
}{%
    \message{ERROR: tikz SPY library NOT available. The manual will only compile partially.^^J}%
}%

\usepackage{xparse}% for colorbrewer manual

\usetikzlibrary{
    decorations.markings,
    decorations.footprints,
    shapes.arrows,
    matrix,
    positioning,
}

\usepgfplotslibrary{
    fillbetween,
    ternary,
    smithchart,
    patchplots,
    polar,
    colormaps,
    colorbrewer,
    colortol,           % docu not ready yet
}
\pgfqkeys{/codeexample}{%
    every codeexample/.append style={
        /pgfplots/every ternary axis/.append style={
            /pgfplots/legend style={fill=graphicbackground},
        }
    },
    tabsize=4,
}

\pgfplotsmanualenableexternalizationofexpensive

%\usetikzlibrary{external}
%\tikzexternalize[prefix=figures/]{pgfplots}

\title{%
    Manual for Package \PGFPlots{}\\
    {\small 2D/3D Plots in \LaTeX{}, Version \pgfplotsversion}\\
    {\small\url{https://github.com/pgf-tikz/pgfplots}}
    %\\{\small Attention: you are using an unstable development version.}
}

\makeatletter
\long\def\abstractsmuggle{%
    \centering
    \textbf{Abstract}\\[0.5cm]

    \begin{minipage}{12cm}
        \PGFPlots{} draws high-quality function plots in normal or logarithmic
        scaling with a user-friendly interface directly in \TeX{}. The user
        supplies axis labels, legend entries and the plot coordinates for one or
        more plots and \PGFPlots{} applies axis scaling, computes any logarithms
        and axis ticks and draws the plots. It supports line plots, scatter
        plots, piecewise constant plots, bar plots, area plots, mesh and surface
        plots, patch plots, contour plots, quiver plots, histogram plots, box
        plots, polar axes, ternary diagrams, smith charts and some more. It is
        based on Till Tantau's package \PGF{}/\Tikz{}.
    \end{minipage}

}%

\expandafter\date\expandafter{\@date\\[2cm]
    \abstractsmuggle
}%
\makeatother

%\includeonly{pgfplots.reference}


\begin{document}

\def\plotcoords{%
\addplot coordinates {
(5,8.312e-02)    (17,2.547e-02)   (49,7.407e-03)
(129,2.102e-03)  (321,5.874e-04)  (769,1.623e-04)
(1793,4.442e-05) (4097,1.207e-05) (9217,3.261e-06)
};

\addplot coordinates{
(7,8.472e-02)    (31,3.044e-02)    (111,1.022e-02)
(351,3.303e-03)  (1023,1.039e-03)  (2815,3.196e-04)
(7423,9.658e-05) (18943,2.873e-05) (47103,8.437e-06)
};

\addplot coordinates{
(9,7.881e-02)     (49,3.243e-02)    (209,1.232e-02)
(769,4.454e-03)   (2561,1.551e-03)  (7937,5.236e-04)
(23297,1.723e-04) (65537,5.545e-05) (178177,1.751e-05)
};

\addplot coordinates{
(11,6.887e-02)    (71,3.177e-02)     (351,1.341e-02)
(1471,5.334e-03)  (5503,2.027e-03)   (18943,7.415e-04)
(61183,2.628e-04) (187903,9.063e-05) (553983,3.053e-05)
};

\addplot coordinates{
(13,5.755e-02)     (97,2.925e-02)     (545,1.351e-02)
(2561,5.842e-03)   (10625,2.397e-03)  (40193,9.414e-04)
(141569,3.564e-04) (471041,1.308e-04)
(1496065,4.670e-05)
};
}%


\bookmaketitle
\tableofcontents
\include{pgfplots.title_abstract_intro}
\include{pgfplots.preliminaries}
\include{pgfplots.intro}
\include{pgfplots.reference}
\include{pgfplots.libs}
\include{pgfplots.resources}
\include{pgfplots.importexport}
\include{pgfplots.basic.reference}

\printindex

\bibliographystyle{abbrv} %gerapali} %gerabbrv} %gerunsrt.bst} %gerabbrv}% gerplain}
\nocite{pgfplotstable}
\nocite{programmingnotes}
\bibliography{pgfplots}
\end{document}

% -----------------------------------------------------------------------------
% For Stefan Pinnow as reminder on what to look for when editing the manual
% -----------------------------------------------------------------------------
% There should be no line breaks in the following environments
% - |...|
% - \declareandlabel{...}
% - \verbpdfref{...}
% If "MakeTikzPictures" isn't running through check one of the externalized
% LOG files what is the cause of that.
% -----------------------------------------------------------------------------


%%%%%%%%%%%%%%%%%%%%%%%%%%%%%%%%%%%%%%%%%%%%%%%%%%%%%%%%%%%%%%%%%%%%%%%%%%%%%
%
% Package pgfplots.sty documentation.
%
% Copyright 2007/2008 by Christian Feuersaenger.
%
% This program is free software: you can redistribute it and/or modify
% it under the terms of the GNU General Public License as published by
% the Free Software Foundation, either version 3 of the License, or
% (at your option) any later version.
%
% This program is distributed in the hope that it will be useful,
% but WITHOUT ANY WARRANTY; without even the implied warranty of
% MERCHANTABILITY or FITNESS FOR A PARTICULAR PURPOSE.  See the
% GNU General Public License for more details.
%
% You should have received a copy of the GNU General Public License
% along with this program.  If not, see <http://www.gnu.org/licenses/>.
%
%
%%%%%%%%%%%%%%%%%%%%%%%%%%%%%%%%%%%%%%%%%%%%%%%%%%%%%%%%%%%%%%%%%%%%%%%%%%%%%
% SEE pgfplots-macros.tex as well!
%\pdfminorversion=5 % to allow compression
%\pdfobjcompresslevel=2
\documentclass[a4paper,openany]{book}

\let\bookmaketitle=\maketitle

% -----------------------------------
% this here is from ltxdoc:
\usepackage{doc}[=v2]
\AtBeginDocument{\MakeShortVerb{\|}}
%\setlength{\textwidth}{355pt}
%\addtolength\marginparwidth{30pt}
%\addtolength\oddsidemargin{20pt}
%\addtolength\evensidemargin{20pt}
\def\cmd#1{\cs{\expandafter\cmd@to@cs\string#1}}
\def\cmd@to@cs#1#2{\char\number`#2\relax}
\DeclareRobustCommand\cs[1]{\texttt{\char`\\#1}}
\providecommand\marg[1]{%
  {\ttfamily\char`\{}\meta{#1}{\ttfamily\char`\}}}
\providecommand\oarg[1]{%
  {\ttfamily[}\meta{#1}{\ttfamily]}}
\providecommand\parg[1]{%
  {\ttfamily(}\meta{#1}{\ttfamily)}}
\raggedbottom

% -----------------------------------
\input{pgfplots.preamble.tex}

\makeatletter
% I want a two-column index just as in pgfmanual styles. This here
% was the best way to get one:
\def\index@prologue{\section*{Index}\addcontentsline{toc}{chapter}{Index}
}
\makeatother

%\RequirePackage[german,english,francais]{babel}

\def\matlabcolormaptext{This colormap is similar to one shipped with Matlab$^\text{\textregistered}$ under a similar name.}

\IfFileExists{tikzlibraryspy.code.tex}{%
\usetikzlibrary{spy}
}{%
    \message{ERROR: tikz SPY library NOT available. The manual will only compile partially.^^J}%
}%

\usepackage{xparse}% for colorbrewer manual

\usetikzlibrary{
    decorations.markings,
    decorations.footprints,
    shapes.arrows,
    matrix,
    positioning,
}

\usepgfplotslibrary{
    fillbetween,
    ternary,
    smithchart,
    patchplots,
    polar,
    colormaps,
    colorbrewer,
    colortol,           % docu not ready yet
}
\pgfqkeys{/codeexample}{%
    every codeexample/.append style={
        /pgfplots/every ternary axis/.append style={
            /pgfplots/legend style={fill=graphicbackground},
        }
    },
    tabsize=4,
}

\pgfplotsmanualenableexternalizationofexpensive

%\usetikzlibrary{external}
%\tikzexternalize[prefix=figures/]{pgfplots}

\title{%
    Manual for Package \PGFPlots{}\\
    {\small 2D/3D Plots in \LaTeX{}, Version \pgfplotsversion}\\
    {\small\url{https://github.com/pgf-tikz/pgfplots}}
    %\\{\small Attention: you are using an unstable development version.}
}

\makeatletter
\long\def\abstractsmuggle{%
    \centering
    \textbf{Abstract}\\[0.5cm]

    \begin{minipage}{12cm}
        \PGFPlots{} draws high-quality function plots in normal or logarithmic
        scaling with a user-friendly interface directly in \TeX{}. The user
        supplies axis labels, legend entries and the plot coordinates for one or
        more plots and \PGFPlots{} applies axis scaling, computes any logarithms
        and axis ticks and draws the plots. It supports line plots, scatter
        plots, piecewise constant plots, bar plots, area plots, mesh and surface
        plots, patch plots, contour plots, quiver plots, histogram plots, box
        plots, polar axes, ternary diagrams, smith charts and some more. It is
        based on Till Tantau's package \PGF{}/\Tikz{}.
    \end{minipage}

}%

\expandafter\date\expandafter{\@date\\[2cm]
    \abstractsmuggle
}%
\makeatother

%\includeonly{pgfplots.reference}


\begin{document}

\def\plotcoords{%
\addplot coordinates {
(5,8.312e-02)    (17,2.547e-02)   (49,7.407e-03)
(129,2.102e-03)  (321,5.874e-04)  (769,1.623e-04)
(1793,4.442e-05) (4097,1.207e-05) (9217,3.261e-06)
};

\addplot coordinates{
(7,8.472e-02)    (31,3.044e-02)    (111,1.022e-02)
(351,3.303e-03)  (1023,1.039e-03)  (2815,3.196e-04)
(7423,9.658e-05) (18943,2.873e-05) (47103,8.437e-06)
};

\addplot coordinates{
(9,7.881e-02)     (49,3.243e-02)    (209,1.232e-02)
(769,4.454e-03)   (2561,1.551e-03)  (7937,5.236e-04)
(23297,1.723e-04) (65537,5.545e-05) (178177,1.751e-05)
};

\addplot coordinates{
(11,6.887e-02)    (71,3.177e-02)     (351,1.341e-02)
(1471,5.334e-03)  (5503,2.027e-03)   (18943,7.415e-04)
(61183,2.628e-04) (187903,9.063e-05) (553983,3.053e-05)
};

\addplot coordinates{
(13,5.755e-02)     (97,2.925e-02)     (545,1.351e-02)
(2561,5.842e-03)   (10625,2.397e-03)  (40193,9.414e-04)
(141569,3.564e-04) (471041,1.308e-04)
(1496065,4.670e-05)
};
}%


\bookmaketitle
\tableofcontents
\include{pgfplots.title_abstract_intro}
\include{pgfplots.preliminaries}
\include{pgfplots.intro}
\include{pgfplots.reference}
\include{pgfplots.libs}
\include{pgfplots.resources}
\include{pgfplots.importexport}
\include{pgfplots.basic.reference}

\printindex

\bibliographystyle{abbrv} %gerapali} %gerabbrv} %gerunsrt.bst} %gerabbrv}% gerplain}
\nocite{pgfplotstable}
\nocite{programmingnotes}
\bibliography{pgfplots}
\end{document}

% -----------------------------------------------------------------------------
% For Stefan Pinnow as reminder on what to look for when editing the manual
% -----------------------------------------------------------------------------
% There should be no line breaks in the following environments
% - |...|
% - \declareandlabel{...}
% - \verbpdfref{...}
% If "MakeTikzPictures" isn't running through check one of the externalized
% LOG files what is the cause of that.
% -----------------------------------------------------------------------------


%%%%%%%%%%%%%%%%%%%%%%%%%%%%%%%%%%%%%%%%%%%%%%%%%%%%%%%%%%%%%%%%%%%%%%%%%%%%%
%
% Package pgfplots.sty documentation.
%
% Copyright 2007/2008 by Christian Feuersaenger.
%
% This program is free software: you can redistribute it and/or modify
% it under the terms of the GNU General Public License as published by
% the Free Software Foundation, either version 3 of the License, or
% (at your option) any later version.
%
% This program is distributed in the hope that it will be useful,
% but WITHOUT ANY WARRANTY; without even the implied warranty of
% MERCHANTABILITY or FITNESS FOR A PARTICULAR PURPOSE.  See the
% GNU General Public License for more details.
%
% You should have received a copy of the GNU General Public License
% along with this program.  If not, see <http://www.gnu.org/licenses/>.
%
%
%%%%%%%%%%%%%%%%%%%%%%%%%%%%%%%%%%%%%%%%%%%%%%%%%%%%%%%%%%%%%%%%%%%%%%%%%%%%%
% SEE pgfplots-macros.tex as well!
%\pdfminorversion=5 % to allow compression
%\pdfobjcompresslevel=2
\documentclass[a4paper,openany]{book}

\let\bookmaketitle=\maketitle

% -----------------------------------
% this here is from ltxdoc:
\usepackage{doc}[=v2]
\AtBeginDocument{\MakeShortVerb{\|}}
%\setlength{\textwidth}{355pt}
%\addtolength\marginparwidth{30pt}
%\addtolength\oddsidemargin{20pt}
%\addtolength\evensidemargin{20pt}
\def\cmd#1{\cs{\expandafter\cmd@to@cs\string#1}}
\def\cmd@to@cs#1#2{\char\number`#2\relax}
\DeclareRobustCommand\cs[1]{\texttt{\char`\\#1}}
\providecommand\marg[1]{%
  {\ttfamily\char`\{}\meta{#1}{\ttfamily\char`\}}}
\providecommand\oarg[1]{%
  {\ttfamily[}\meta{#1}{\ttfamily]}}
\providecommand\parg[1]{%
  {\ttfamily(}\meta{#1}{\ttfamily)}}
\raggedbottom

% -----------------------------------
\input{pgfplots.preamble.tex}

\makeatletter
% I want a two-column index just as in pgfmanual styles. This here
% was the best way to get one:
\def\index@prologue{\section*{Index}\addcontentsline{toc}{chapter}{Index}
}
\makeatother

%\RequirePackage[german,english,francais]{babel}

\def\matlabcolormaptext{This colormap is similar to one shipped with Matlab$^\text{\textregistered}$ under a similar name.}

\IfFileExists{tikzlibraryspy.code.tex}{%
\usetikzlibrary{spy}
}{%
    \message{ERROR: tikz SPY library NOT available. The manual will only compile partially.^^J}%
}%

\usepackage{xparse}% for colorbrewer manual

\usetikzlibrary{
    decorations.markings,
    decorations.footprints,
    shapes.arrows,
    matrix,
    positioning,
}

\usepgfplotslibrary{
    fillbetween,
    ternary,
    smithchart,
    patchplots,
    polar,
    colormaps,
    colorbrewer,
    colortol,           % docu not ready yet
}
\pgfqkeys{/codeexample}{%
    every codeexample/.append style={
        /pgfplots/every ternary axis/.append style={
            /pgfplots/legend style={fill=graphicbackground},
        }
    },
    tabsize=4,
}

\pgfplotsmanualenableexternalizationofexpensive

%\usetikzlibrary{external}
%\tikzexternalize[prefix=figures/]{pgfplots}

\title{%
    Manual for Package \PGFPlots{}\\
    {\small 2D/3D Plots in \LaTeX{}, Version \pgfplotsversion}\\
    {\small\url{https://github.com/pgf-tikz/pgfplots}}
    %\\{\small Attention: you are using an unstable development version.}
}

\makeatletter
\long\def\abstractsmuggle{%
    \centering
    \textbf{Abstract}\\[0.5cm]

    \begin{minipage}{12cm}
        \PGFPlots{} draws high-quality function plots in normal or logarithmic
        scaling with a user-friendly interface directly in \TeX{}. The user
        supplies axis labels, legend entries and the plot coordinates for one or
        more plots and \PGFPlots{} applies axis scaling, computes any logarithms
        and axis ticks and draws the plots. It supports line plots, scatter
        plots, piecewise constant plots, bar plots, area plots, mesh and surface
        plots, patch plots, contour plots, quiver plots, histogram plots, box
        plots, polar axes, ternary diagrams, smith charts and some more. It is
        based on Till Tantau's package \PGF{}/\Tikz{}.
    \end{minipage}

}%

\expandafter\date\expandafter{\@date\\[2cm]
    \abstractsmuggle
}%
\makeatother

%\includeonly{pgfplots.reference}


\begin{document}

\def\plotcoords{%
\addplot coordinates {
(5,8.312e-02)    (17,2.547e-02)   (49,7.407e-03)
(129,2.102e-03)  (321,5.874e-04)  (769,1.623e-04)
(1793,4.442e-05) (4097,1.207e-05) (9217,3.261e-06)
};

\addplot coordinates{
(7,8.472e-02)    (31,3.044e-02)    (111,1.022e-02)
(351,3.303e-03)  (1023,1.039e-03)  (2815,3.196e-04)
(7423,9.658e-05) (18943,2.873e-05) (47103,8.437e-06)
};

\addplot coordinates{
(9,7.881e-02)     (49,3.243e-02)    (209,1.232e-02)
(769,4.454e-03)   (2561,1.551e-03)  (7937,5.236e-04)
(23297,1.723e-04) (65537,5.545e-05) (178177,1.751e-05)
};

\addplot coordinates{
(11,6.887e-02)    (71,3.177e-02)     (351,1.341e-02)
(1471,5.334e-03)  (5503,2.027e-03)   (18943,7.415e-04)
(61183,2.628e-04) (187903,9.063e-05) (553983,3.053e-05)
};

\addplot coordinates{
(13,5.755e-02)     (97,2.925e-02)     (545,1.351e-02)
(2561,5.842e-03)   (10625,2.397e-03)  (40193,9.414e-04)
(141569,3.564e-04) (471041,1.308e-04)
(1496065,4.670e-05)
};
}%


\bookmaketitle
\tableofcontents
\include{pgfplots.title_abstract_intro}
\include{pgfplots.preliminaries}
\include{pgfplots.intro}
\include{pgfplots.reference}
\include{pgfplots.libs}
\include{pgfplots.resources}
\include{pgfplots.importexport}
\include{pgfplots.basic.reference}

\printindex

\bibliographystyle{abbrv} %gerapali} %gerabbrv} %gerunsrt.bst} %gerabbrv}% gerplain}
\nocite{pgfplotstable}
\nocite{programmingnotes}
\bibliography{pgfplots}
\end{document}

% -----------------------------------------------------------------------------
% For Stefan Pinnow as reminder on what to look for when editing the manual
% -----------------------------------------------------------------------------
% There should be no line breaks in the following environments
% - |...|
% - \declareandlabel{...}
% - \verbpdfref{...}
% If "MakeTikzPictures" isn't running through check one of the externalized
% LOG files what is the cause of that.
% -----------------------------------------------------------------------------


%%%%%%%%%%%%%%%%%%%%%%%%%%%%%%%%%%%%%%%%%%%%%%%%%%%%%%%%%%%%%%%%%%%%%%%%%%%%%
%
% Package pgfplots.sty documentation.
%
% Copyright 2007/2008 by Christian Feuersaenger.
%
% This program is free software: you can redistribute it and/or modify
% it under the terms of the GNU General Public License as published by
% the Free Software Foundation, either version 3 of the License, or
% (at your option) any later version.
%
% This program is distributed in the hope that it will be useful,
% but WITHOUT ANY WARRANTY; without even the implied warranty of
% MERCHANTABILITY or FITNESS FOR A PARTICULAR PURPOSE.  See the
% GNU General Public License for more details.
%
% You should have received a copy of the GNU General Public License
% along with this program.  If not, see <http://www.gnu.org/licenses/>.
%
%
%%%%%%%%%%%%%%%%%%%%%%%%%%%%%%%%%%%%%%%%%%%%%%%%%%%%%%%%%%%%%%%%%%%%%%%%%%%%%
% SEE pgfplots-macros.tex as well!
%\pdfminorversion=5 % to allow compression
%\pdfobjcompresslevel=2
\documentclass[a4paper,openany]{book}

\let\bookmaketitle=\maketitle

% -----------------------------------
% this here is from ltxdoc:
\usepackage{doc}[=v2]
\AtBeginDocument{\MakeShortVerb{\|}}
%\setlength{\textwidth}{355pt}
%\addtolength\marginparwidth{30pt}
%\addtolength\oddsidemargin{20pt}
%\addtolength\evensidemargin{20pt}
\def\cmd#1{\cs{\expandafter\cmd@to@cs\string#1}}
\def\cmd@to@cs#1#2{\char\number`#2\relax}
\DeclareRobustCommand\cs[1]{\texttt{\char`\\#1}}
\providecommand\marg[1]{%
  {\ttfamily\char`\{}\meta{#1}{\ttfamily\char`\}}}
\providecommand\oarg[1]{%
  {\ttfamily[}\meta{#1}{\ttfamily]}}
\providecommand\parg[1]{%
  {\ttfamily(}\meta{#1}{\ttfamily)}}
\raggedbottom

% -----------------------------------
\input{pgfplots.preamble.tex}

\makeatletter
% I want a two-column index just as in pgfmanual styles. This here
% was the best way to get one:
\def\index@prologue{\section*{Index}\addcontentsline{toc}{chapter}{Index}
}
\makeatother

%\RequirePackage[german,english,francais]{babel}

\def\matlabcolormaptext{This colormap is similar to one shipped with Matlab$^\text{\textregistered}$ under a similar name.}

\IfFileExists{tikzlibraryspy.code.tex}{%
\usetikzlibrary{spy}
}{%
    \message{ERROR: tikz SPY library NOT available. The manual will only compile partially.^^J}%
}%

\usepackage{xparse}% for colorbrewer manual

\usetikzlibrary{
    decorations.markings,
    decorations.footprints,
    shapes.arrows,
    matrix,
    positioning,
}

\usepgfplotslibrary{
    fillbetween,
    ternary,
    smithchart,
    patchplots,
    polar,
    colormaps,
    colorbrewer,
    colortol,           % docu not ready yet
}
\pgfqkeys{/codeexample}{%
    every codeexample/.append style={
        /pgfplots/every ternary axis/.append style={
            /pgfplots/legend style={fill=graphicbackground},
        }
    },
    tabsize=4,
}

\pgfplotsmanualenableexternalizationofexpensive

%\usetikzlibrary{external}
%\tikzexternalize[prefix=figures/]{pgfplots}

\title{%
    Manual for Package \PGFPlots{}\\
    {\small 2D/3D Plots in \LaTeX{}, Version \pgfplotsversion}\\
    {\small\url{https://github.com/pgf-tikz/pgfplots}}
    %\\{\small Attention: you are using an unstable development version.}
}

\makeatletter
\long\def\abstractsmuggle{%
    \centering
    \textbf{Abstract}\\[0.5cm]

    \begin{minipage}{12cm}
        \PGFPlots{} draws high-quality function plots in normal or logarithmic
        scaling with a user-friendly interface directly in \TeX{}. The user
        supplies axis labels, legend entries and the plot coordinates for one or
        more plots and \PGFPlots{} applies axis scaling, computes any logarithms
        and axis ticks and draws the plots. It supports line plots, scatter
        plots, piecewise constant plots, bar plots, area plots, mesh and surface
        plots, patch plots, contour plots, quiver plots, histogram plots, box
        plots, polar axes, ternary diagrams, smith charts and some more. It is
        based on Till Tantau's package \PGF{}/\Tikz{}.
    \end{minipage}

}%

\expandafter\date\expandafter{\@date\\[2cm]
    \abstractsmuggle
}%
\makeatother

%\includeonly{pgfplots.reference}


\begin{document}

\def\plotcoords{%
\addplot coordinates {
(5,8.312e-02)    (17,2.547e-02)   (49,7.407e-03)
(129,2.102e-03)  (321,5.874e-04)  (769,1.623e-04)
(1793,4.442e-05) (4097,1.207e-05) (9217,3.261e-06)
};

\addplot coordinates{
(7,8.472e-02)    (31,3.044e-02)    (111,1.022e-02)
(351,3.303e-03)  (1023,1.039e-03)  (2815,3.196e-04)
(7423,9.658e-05) (18943,2.873e-05) (47103,8.437e-06)
};

\addplot coordinates{
(9,7.881e-02)     (49,3.243e-02)    (209,1.232e-02)
(769,4.454e-03)   (2561,1.551e-03)  (7937,5.236e-04)
(23297,1.723e-04) (65537,5.545e-05) (178177,1.751e-05)
};

\addplot coordinates{
(11,6.887e-02)    (71,3.177e-02)     (351,1.341e-02)
(1471,5.334e-03)  (5503,2.027e-03)   (18943,7.415e-04)
(61183,2.628e-04) (187903,9.063e-05) (553983,3.053e-05)
};

\addplot coordinates{
(13,5.755e-02)     (97,2.925e-02)     (545,1.351e-02)
(2561,5.842e-03)   (10625,2.397e-03)  (40193,9.414e-04)
(141569,3.564e-04) (471041,1.308e-04)
(1496065,4.670e-05)
};
}%


\bookmaketitle
\tableofcontents
\include{pgfplots.title_abstract_intro}
\include{pgfplots.preliminaries}
\include{pgfplots.intro}
\include{pgfplots.reference}
\include{pgfplots.libs}
\include{pgfplots.resources}
\include{pgfplots.importexport}
\include{pgfplots.basic.reference}

\printindex

\bibliographystyle{abbrv} %gerapali} %gerabbrv} %gerunsrt.bst} %gerabbrv}% gerplain}
\nocite{pgfplotstable}
\nocite{programmingnotes}
\bibliography{pgfplots}
\end{document}

% -----------------------------------------------------------------------------
% For Stefan Pinnow as reminder on what to look for when editing the manual
% -----------------------------------------------------------------------------
% There should be no line breaks in the following environments
% - |...|
% - \declareandlabel{...}
% - \verbpdfref{...}
% If "MakeTikzPictures" isn't running through check one of the externalized
% LOG files what is the cause of that.
% -----------------------------------------------------------------------------

\include{pgfplots.basic.reference}

\printindex

\bibliographystyle{abbrv} %gerapali} %gerabbrv} %gerunsrt.bst} %gerabbrv}% gerplain}
\nocite{pgfplotstable}
\nocite{programmingnotes}
\bibliography{pgfplots}
\end{document}

% -----------------------------------------------------------------------------
% For Stefan Pinnow as reminder on what to look for when editing the manual
% -----------------------------------------------------------------------------
% There should be no line breaks in the following environments
% - |...|
% - \declareandlabel{...}
% - \verbpdfref{...}
% If "MakeTikzPictures" isn't running through check one of the externalized
% LOG files what is the cause of that.
% -----------------------------------------------------------------------------


%%%%%%%%%%%%%%%%%%%%%%%%%%%%%%%%%%%%%%%%%%%%%%%%%%%%%%%%%%%%%%%%%%%%%%%%%%%%%
%
% Package pgfplots.sty documentation.
%
% Copyright 2007/2008 by Christian Feuersaenger.
%
% This program is free software: you can redistribute it and/or modify
% it under the terms of the GNU General Public License as published by
% the Free Software Foundation, either version 3 of the License, or
% (at your option) any later version.
%
% This program is distributed in the hope that it will be useful,
% but WITHOUT ANY WARRANTY; without even the implied warranty of
% MERCHANTABILITY or FITNESS FOR A PARTICULAR PURPOSE.  See the
% GNU General Public License for more details.
%
% You should have received a copy of the GNU General Public License
% along with this program.  If not, see <http://www.gnu.org/licenses/>.
%
%
%%%%%%%%%%%%%%%%%%%%%%%%%%%%%%%%%%%%%%%%%%%%%%%%%%%%%%%%%%%%%%%%%%%%%%%%%%%%%
% SEE pgfplots-macros.tex as well!
%\pdfminorversion=5 % to allow compression
%\pdfobjcompresslevel=2
\documentclass[a4paper,openany]{book}

\let\bookmaketitle=\maketitle

% -----------------------------------
% this here is from ltxdoc:
\usepackage{doc}[=v2]
\AtBeginDocument{\MakeShortVerb{\|}}
%\setlength{\textwidth}{355pt}
%\addtolength\marginparwidth{30pt}
%\addtolength\oddsidemargin{20pt}
%\addtolength\evensidemargin{20pt}
\def\cmd#1{\cs{\expandafter\cmd@to@cs\string#1}}
\def\cmd@to@cs#1#2{\char\number`#2\relax}
\DeclareRobustCommand\cs[1]{\texttt{\char`\\#1}}
\providecommand\marg[1]{%
  {\ttfamily\char`\{}\meta{#1}{\ttfamily\char`\}}}
\providecommand\oarg[1]{%
  {\ttfamily[}\meta{#1}{\ttfamily]}}
\providecommand\parg[1]{%
  {\ttfamily(}\meta{#1}{\ttfamily)}}
\raggedbottom

% -----------------------------------
\input{pgfplots.preamble.tex}

\makeatletter
% I want a two-column index just as in pgfmanual styles. This here
% was the best way to get one:
\def\index@prologue{\section*{Index}\addcontentsline{toc}{chapter}{Index}
}
\makeatother

%\RequirePackage[german,english,francais]{babel}

\def\matlabcolormaptext{This colormap is similar to one shipped with Matlab$^\text{\textregistered}$ under a similar name.}

\IfFileExists{tikzlibraryspy.code.tex}{%
\usetikzlibrary{spy}
}{%
    \message{ERROR: tikz SPY library NOT available. The manual will only compile partially.^^J}%
}%

\usepackage{xparse}% for colorbrewer manual

\usetikzlibrary{
    decorations.markings,
    decorations.footprints,
    shapes.arrows,
    matrix,
    positioning,
}

\usepgfplotslibrary{
    fillbetween,
    ternary,
    smithchart,
    patchplots,
    polar,
    colormaps,
    colorbrewer,
    colortol,           % docu not ready yet
}
\pgfqkeys{/codeexample}{%
    every codeexample/.append style={
        /pgfplots/every ternary axis/.append style={
            /pgfplots/legend style={fill=graphicbackground},
        }
    },
    tabsize=4,
}

\pgfplotsmanualenableexternalizationofexpensive

%\usetikzlibrary{external}
%\tikzexternalize[prefix=figures/]{pgfplots}

\title{%
    Manual for Package \PGFPlots{}\\
    {\small 2D/3D Plots in \LaTeX{}, Version \pgfplotsversion}\\
    {\small\url{https://github.com/pgf-tikz/pgfplots}}
    %\\{\small Attention: you are using an unstable development version.}
}

\makeatletter
\long\def\abstractsmuggle{%
    \centering
    \textbf{Abstract}\\[0.5cm]

    \begin{minipage}{12cm}
        \PGFPlots{} draws high-quality function plots in normal or logarithmic
        scaling with a user-friendly interface directly in \TeX{}. The user
        supplies axis labels, legend entries and the plot coordinates for one or
        more plots and \PGFPlots{} applies axis scaling, computes any logarithms
        and axis ticks and draws the plots. It supports line plots, scatter
        plots, piecewise constant plots, bar plots, area plots, mesh and surface
        plots, patch plots, contour plots, quiver plots, histogram plots, box
        plots, polar axes, ternary diagrams, smith charts and some more. It is
        based on Till Tantau's package \PGF{}/\Tikz{}.
    \end{minipage}

}%

\expandafter\date\expandafter{\@date\\[2cm]
    \abstractsmuggle
}%
\makeatother

%\includeonly{pgfplots.reference}


\begin{document}

\def\plotcoords{%
\addplot coordinates {
(5,8.312e-02)    (17,2.547e-02)   (49,7.407e-03)
(129,2.102e-03)  (321,5.874e-04)  (769,1.623e-04)
(1793,4.442e-05) (4097,1.207e-05) (9217,3.261e-06)
};

\addplot coordinates{
(7,8.472e-02)    (31,3.044e-02)    (111,1.022e-02)
(351,3.303e-03)  (1023,1.039e-03)  (2815,3.196e-04)
(7423,9.658e-05) (18943,2.873e-05) (47103,8.437e-06)
};

\addplot coordinates{
(9,7.881e-02)     (49,3.243e-02)    (209,1.232e-02)
(769,4.454e-03)   (2561,1.551e-03)  (7937,5.236e-04)
(23297,1.723e-04) (65537,5.545e-05) (178177,1.751e-05)
};

\addplot coordinates{
(11,6.887e-02)    (71,3.177e-02)     (351,1.341e-02)
(1471,5.334e-03)  (5503,2.027e-03)   (18943,7.415e-04)
(61183,2.628e-04) (187903,9.063e-05) (553983,3.053e-05)
};

\addplot coordinates{
(13,5.755e-02)     (97,2.925e-02)     (545,1.351e-02)
(2561,5.842e-03)   (10625,2.397e-03)  (40193,9.414e-04)
(141569,3.564e-04) (471041,1.308e-04)
(1496065,4.670e-05)
};
}%


\bookmaketitle
\tableofcontents

%%%%%%%%%%%%%%%%%%%%%%%%%%%%%%%%%%%%%%%%%%%%%%%%%%%%%%%%%%%%%%%%%%%%%%%%%%%%%
%
% Package pgfplots.sty documentation.
%
% Copyright 2007/2008 by Christian Feuersaenger.
%
% This program is free software: you can redistribute it and/or modify
% it under the terms of the GNU General Public License as published by
% the Free Software Foundation, either version 3 of the License, or
% (at your option) any later version.
%
% This program is distributed in the hope that it will be useful,
% but WITHOUT ANY WARRANTY; without even the implied warranty of
% MERCHANTABILITY or FITNESS FOR A PARTICULAR PURPOSE.  See the
% GNU General Public License for more details.
%
% You should have received a copy of the GNU General Public License
% along with this program.  If not, see <http://www.gnu.org/licenses/>.
%
%
%%%%%%%%%%%%%%%%%%%%%%%%%%%%%%%%%%%%%%%%%%%%%%%%%%%%%%%%%%%%%%%%%%%%%%%%%%%%%
% SEE pgfplots-macros.tex as well!
%\pdfminorversion=5 % to allow compression
%\pdfobjcompresslevel=2
\documentclass[a4paper,openany]{book}

\let\bookmaketitle=\maketitle

% -----------------------------------
% this here is from ltxdoc:
\usepackage{doc}[=v2]
\AtBeginDocument{\MakeShortVerb{\|}}
%\setlength{\textwidth}{355pt}
%\addtolength\marginparwidth{30pt}
%\addtolength\oddsidemargin{20pt}
%\addtolength\evensidemargin{20pt}
\def\cmd#1{\cs{\expandafter\cmd@to@cs\string#1}}
\def\cmd@to@cs#1#2{\char\number`#2\relax}
\DeclareRobustCommand\cs[1]{\texttt{\char`\\#1}}
\providecommand\marg[1]{%
  {\ttfamily\char`\{}\meta{#1}{\ttfamily\char`\}}}
\providecommand\oarg[1]{%
  {\ttfamily[}\meta{#1}{\ttfamily]}}
\providecommand\parg[1]{%
  {\ttfamily(}\meta{#1}{\ttfamily)}}
\raggedbottom

% -----------------------------------
\input{pgfplots.preamble.tex}

\makeatletter
% I want a two-column index just as in pgfmanual styles. This here
% was the best way to get one:
\def\index@prologue{\section*{Index}\addcontentsline{toc}{chapter}{Index}
}
\makeatother

%\RequirePackage[german,english,francais]{babel}

\def\matlabcolormaptext{This colormap is similar to one shipped with Matlab$^\text{\textregistered}$ under a similar name.}

\IfFileExists{tikzlibraryspy.code.tex}{%
\usetikzlibrary{spy}
}{%
    \message{ERROR: tikz SPY library NOT available. The manual will only compile partially.^^J}%
}%

\usepackage{xparse}% for colorbrewer manual

\usetikzlibrary{
    decorations.markings,
    decorations.footprints,
    shapes.arrows,
    matrix,
    positioning,
}

\usepgfplotslibrary{
    fillbetween,
    ternary,
    smithchart,
    patchplots,
    polar,
    colormaps,
    colorbrewer,
    colortol,           % docu not ready yet
}
\pgfqkeys{/codeexample}{%
    every codeexample/.append style={
        /pgfplots/every ternary axis/.append style={
            /pgfplots/legend style={fill=graphicbackground},
        }
    },
    tabsize=4,
}

\pgfplotsmanualenableexternalizationofexpensive

%\usetikzlibrary{external}
%\tikzexternalize[prefix=figures/]{pgfplots}

\title{%
    Manual for Package \PGFPlots{}\\
    {\small 2D/3D Plots in \LaTeX{}, Version \pgfplotsversion}\\
    {\small\url{https://github.com/pgf-tikz/pgfplots}}
    %\\{\small Attention: you are using an unstable development version.}
}

\makeatletter
\long\def\abstractsmuggle{%
    \centering
    \textbf{Abstract}\\[0.5cm]

    \begin{minipage}{12cm}
        \PGFPlots{} draws high-quality function plots in normal or logarithmic
        scaling with a user-friendly interface directly in \TeX{}. The user
        supplies axis labels, legend entries and the plot coordinates for one or
        more plots and \PGFPlots{} applies axis scaling, computes any logarithms
        and axis ticks and draws the plots. It supports line plots, scatter
        plots, piecewise constant plots, bar plots, area plots, mesh and surface
        plots, patch plots, contour plots, quiver plots, histogram plots, box
        plots, polar axes, ternary diagrams, smith charts and some more. It is
        based on Till Tantau's package \PGF{}/\Tikz{}.
    \end{minipage}

}%

\expandafter\date\expandafter{\@date\\[2cm]
    \abstractsmuggle
}%
\makeatother

%\includeonly{pgfplots.reference}


\begin{document}

\def\plotcoords{%
\addplot coordinates {
(5,8.312e-02)    (17,2.547e-02)   (49,7.407e-03)
(129,2.102e-03)  (321,5.874e-04)  (769,1.623e-04)
(1793,4.442e-05) (4097,1.207e-05) (9217,3.261e-06)
};

\addplot coordinates{
(7,8.472e-02)    (31,3.044e-02)    (111,1.022e-02)
(351,3.303e-03)  (1023,1.039e-03)  (2815,3.196e-04)
(7423,9.658e-05) (18943,2.873e-05) (47103,8.437e-06)
};

\addplot coordinates{
(9,7.881e-02)     (49,3.243e-02)    (209,1.232e-02)
(769,4.454e-03)   (2561,1.551e-03)  (7937,5.236e-04)
(23297,1.723e-04) (65537,5.545e-05) (178177,1.751e-05)
};

\addplot coordinates{
(11,6.887e-02)    (71,3.177e-02)     (351,1.341e-02)
(1471,5.334e-03)  (5503,2.027e-03)   (18943,7.415e-04)
(61183,2.628e-04) (187903,9.063e-05) (553983,3.053e-05)
};

\addplot coordinates{
(13,5.755e-02)     (97,2.925e-02)     (545,1.351e-02)
(2561,5.842e-03)   (10625,2.397e-03)  (40193,9.414e-04)
(141569,3.564e-04) (471041,1.308e-04)
(1496065,4.670e-05)
};
}%


\bookmaketitle
\tableofcontents
\include{pgfplots.title_abstract_intro}
\include{pgfplots.preliminaries}
\include{pgfplots.intro}
\include{pgfplots.reference}
\include{pgfplots.libs}
\include{pgfplots.resources}
\include{pgfplots.importexport}
\include{pgfplots.basic.reference}

\printindex

\bibliographystyle{abbrv} %gerapali} %gerabbrv} %gerunsrt.bst} %gerabbrv}% gerplain}
\nocite{pgfplotstable}
\nocite{programmingnotes}
\bibliography{pgfplots}
\end{document}

% -----------------------------------------------------------------------------
% For Stefan Pinnow as reminder on what to look for when editing the manual
% -----------------------------------------------------------------------------
% There should be no line breaks in the following environments
% - |...|
% - \declareandlabel{...}
% - \verbpdfref{...}
% If "MakeTikzPictures" isn't running through check one of the externalized
% LOG files what is the cause of that.
% -----------------------------------------------------------------------------


%%%%%%%%%%%%%%%%%%%%%%%%%%%%%%%%%%%%%%%%%%%%%%%%%%%%%%%%%%%%%%%%%%%%%%%%%%%%%
%
% Package pgfplots.sty documentation.
%
% Copyright 2007/2008 by Christian Feuersaenger.
%
% This program is free software: you can redistribute it and/or modify
% it under the terms of the GNU General Public License as published by
% the Free Software Foundation, either version 3 of the License, or
% (at your option) any later version.
%
% This program is distributed in the hope that it will be useful,
% but WITHOUT ANY WARRANTY; without even the implied warranty of
% MERCHANTABILITY or FITNESS FOR A PARTICULAR PURPOSE.  See the
% GNU General Public License for more details.
%
% You should have received a copy of the GNU General Public License
% along with this program.  If not, see <http://www.gnu.org/licenses/>.
%
%
%%%%%%%%%%%%%%%%%%%%%%%%%%%%%%%%%%%%%%%%%%%%%%%%%%%%%%%%%%%%%%%%%%%%%%%%%%%%%
% SEE pgfplots-macros.tex as well!
%\pdfminorversion=5 % to allow compression
%\pdfobjcompresslevel=2
\documentclass[a4paper,openany]{book}

\let\bookmaketitle=\maketitle

% -----------------------------------
% this here is from ltxdoc:
\usepackage{doc}[=v2]
\AtBeginDocument{\MakeShortVerb{\|}}
%\setlength{\textwidth}{355pt}
%\addtolength\marginparwidth{30pt}
%\addtolength\oddsidemargin{20pt}
%\addtolength\evensidemargin{20pt}
\def\cmd#1{\cs{\expandafter\cmd@to@cs\string#1}}
\def\cmd@to@cs#1#2{\char\number`#2\relax}
\DeclareRobustCommand\cs[1]{\texttt{\char`\\#1}}
\providecommand\marg[1]{%
  {\ttfamily\char`\{}\meta{#1}{\ttfamily\char`\}}}
\providecommand\oarg[1]{%
  {\ttfamily[}\meta{#1}{\ttfamily]}}
\providecommand\parg[1]{%
  {\ttfamily(}\meta{#1}{\ttfamily)}}
\raggedbottom

% -----------------------------------
\input{pgfplots.preamble.tex}

\makeatletter
% I want a two-column index just as in pgfmanual styles. This here
% was the best way to get one:
\def\index@prologue{\section*{Index}\addcontentsline{toc}{chapter}{Index}
}
\makeatother

%\RequirePackage[german,english,francais]{babel}

\def\matlabcolormaptext{This colormap is similar to one shipped with Matlab$^\text{\textregistered}$ under a similar name.}

\IfFileExists{tikzlibraryspy.code.tex}{%
\usetikzlibrary{spy}
}{%
    \message{ERROR: tikz SPY library NOT available. The manual will only compile partially.^^J}%
}%

\usepackage{xparse}% for colorbrewer manual

\usetikzlibrary{
    decorations.markings,
    decorations.footprints,
    shapes.arrows,
    matrix,
    positioning,
}

\usepgfplotslibrary{
    fillbetween,
    ternary,
    smithchart,
    patchplots,
    polar,
    colormaps,
    colorbrewer,
    colortol,           % docu not ready yet
}
\pgfqkeys{/codeexample}{%
    every codeexample/.append style={
        /pgfplots/every ternary axis/.append style={
            /pgfplots/legend style={fill=graphicbackground},
        }
    },
    tabsize=4,
}

\pgfplotsmanualenableexternalizationofexpensive

%\usetikzlibrary{external}
%\tikzexternalize[prefix=figures/]{pgfplots}

\title{%
    Manual for Package \PGFPlots{}\\
    {\small 2D/3D Plots in \LaTeX{}, Version \pgfplotsversion}\\
    {\small\url{https://github.com/pgf-tikz/pgfplots}}
    %\\{\small Attention: you are using an unstable development version.}
}

\makeatletter
\long\def\abstractsmuggle{%
    \centering
    \textbf{Abstract}\\[0.5cm]

    \begin{minipage}{12cm}
        \PGFPlots{} draws high-quality function plots in normal or logarithmic
        scaling with a user-friendly interface directly in \TeX{}. The user
        supplies axis labels, legend entries and the plot coordinates for one or
        more plots and \PGFPlots{} applies axis scaling, computes any logarithms
        and axis ticks and draws the plots. It supports line plots, scatter
        plots, piecewise constant plots, bar plots, area plots, mesh and surface
        plots, patch plots, contour plots, quiver plots, histogram plots, box
        plots, polar axes, ternary diagrams, smith charts and some more. It is
        based on Till Tantau's package \PGF{}/\Tikz{}.
    \end{minipage}

}%

\expandafter\date\expandafter{\@date\\[2cm]
    \abstractsmuggle
}%
\makeatother

%\includeonly{pgfplots.reference}


\begin{document}

\def\plotcoords{%
\addplot coordinates {
(5,8.312e-02)    (17,2.547e-02)   (49,7.407e-03)
(129,2.102e-03)  (321,5.874e-04)  (769,1.623e-04)
(1793,4.442e-05) (4097,1.207e-05) (9217,3.261e-06)
};

\addplot coordinates{
(7,8.472e-02)    (31,3.044e-02)    (111,1.022e-02)
(351,3.303e-03)  (1023,1.039e-03)  (2815,3.196e-04)
(7423,9.658e-05) (18943,2.873e-05) (47103,8.437e-06)
};

\addplot coordinates{
(9,7.881e-02)     (49,3.243e-02)    (209,1.232e-02)
(769,4.454e-03)   (2561,1.551e-03)  (7937,5.236e-04)
(23297,1.723e-04) (65537,5.545e-05) (178177,1.751e-05)
};

\addplot coordinates{
(11,6.887e-02)    (71,3.177e-02)     (351,1.341e-02)
(1471,5.334e-03)  (5503,2.027e-03)   (18943,7.415e-04)
(61183,2.628e-04) (187903,9.063e-05) (553983,3.053e-05)
};

\addplot coordinates{
(13,5.755e-02)     (97,2.925e-02)     (545,1.351e-02)
(2561,5.842e-03)   (10625,2.397e-03)  (40193,9.414e-04)
(141569,3.564e-04) (471041,1.308e-04)
(1496065,4.670e-05)
};
}%


\bookmaketitle
\tableofcontents
\include{pgfplots.title_abstract_intro}
\include{pgfplots.preliminaries}
\include{pgfplots.intro}
\include{pgfplots.reference}
\include{pgfplots.libs}
\include{pgfplots.resources}
\include{pgfplots.importexport}
\include{pgfplots.basic.reference}

\printindex

\bibliographystyle{abbrv} %gerapali} %gerabbrv} %gerunsrt.bst} %gerabbrv}% gerplain}
\nocite{pgfplotstable}
\nocite{programmingnotes}
\bibliography{pgfplots}
\end{document}

% -----------------------------------------------------------------------------
% For Stefan Pinnow as reminder on what to look for when editing the manual
% -----------------------------------------------------------------------------
% There should be no line breaks in the following environments
% - |...|
% - \declareandlabel{...}
% - \verbpdfref{...}
% If "MakeTikzPictures" isn't running through check one of the externalized
% LOG files what is the cause of that.
% -----------------------------------------------------------------------------


%%%%%%%%%%%%%%%%%%%%%%%%%%%%%%%%%%%%%%%%%%%%%%%%%%%%%%%%%%%%%%%%%%%%%%%%%%%%%
%
% Package pgfplots.sty documentation.
%
% Copyright 2007/2008 by Christian Feuersaenger.
%
% This program is free software: you can redistribute it and/or modify
% it under the terms of the GNU General Public License as published by
% the Free Software Foundation, either version 3 of the License, or
% (at your option) any later version.
%
% This program is distributed in the hope that it will be useful,
% but WITHOUT ANY WARRANTY; without even the implied warranty of
% MERCHANTABILITY or FITNESS FOR A PARTICULAR PURPOSE.  See the
% GNU General Public License for more details.
%
% You should have received a copy of the GNU General Public License
% along with this program.  If not, see <http://www.gnu.org/licenses/>.
%
%
%%%%%%%%%%%%%%%%%%%%%%%%%%%%%%%%%%%%%%%%%%%%%%%%%%%%%%%%%%%%%%%%%%%%%%%%%%%%%
% SEE pgfplots-macros.tex as well!
%\pdfminorversion=5 % to allow compression
%\pdfobjcompresslevel=2
\documentclass[a4paper,openany]{book}

\let\bookmaketitle=\maketitle

% -----------------------------------
% this here is from ltxdoc:
\usepackage{doc}[=v2]
\AtBeginDocument{\MakeShortVerb{\|}}
%\setlength{\textwidth}{355pt}
%\addtolength\marginparwidth{30pt}
%\addtolength\oddsidemargin{20pt}
%\addtolength\evensidemargin{20pt}
\def\cmd#1{\cs{\expandafter\cmd@to@cs\string#1}}
\def\cmd@to@cs#1#2{\char\number`#2\relax}
\DeclareRobustCommand\cs[1]{\texttt{\char`\\#1}}
\providecommand\marg[1]{%
  {\ttfamily\char`\{}\meta{#1}{\ttfamily\char`\}}}
\providecommand\oarg[1]{%
  {\ttfamily[}\meta{#1}{\ttfamily]}}
\providecommand\parg[1]{%
  {\ttfamily(}\meta{#1}{\ttfamily)}}
\raggedbottom

% -----------------------------------
\input{pgfplots.preamble.tex}

\makeatletter
% I want a two-column index just as in pgfmanual styles. This here
% was the best way to get one:
\def\index@prologue{\section*{Index}\addcontentsline{toc}{chapter}{Index}
}
\makeatother

%\RequirePackage[german,english,francais]{babel}

\def\matlabcolormaptext{This colormap is similar to one shipped with Matlab$^\text{\textregistered}$ under a similar name.}

\IfFileExists{tikzlibraryspy.code.tex}{%
\usetikzlibrary{spy}
}{%
    \message{ERROR: tikz SPY library NOT available. The manual will only compile partially.^^J}%
}%

\usepackage{xparse}% for colorbrewer manual

\usetikzlibrary{
    decorations.markings,
    decorations.footprints,
    shapes.arrows,
    matrix,
    positioning,
}

\usepgfplotslibrary{
    fillbetween,
    ternary,
    smithchart,
    patchplots,
    polar,
    colormaps,
    colorbrewer,
    colortol,           % docu not ready yet
}
\pgfqkeys{/codeexample}{%
    every codeexample/.append style={
        /pgfplots/every ternary axis/.append style={
            /pgfplots/legend style={fill=graphicbackground},
        }
    },
    tabsize=4,
}

\pgfplotsmanualenableexternalizationofexpensive

%\usetikzlibrary{external}
%\tikzexternalize[prefix=figures/]{pgfplots}

\title{%
    Manual for Package \PGFPlots{}\\
    {\small 2D/3D Plots in \LaTeX{}, Version \pgfplotsversion}\\
    {\small\url{https://github.com/pgf-tikz/pgfplots}}
    %\\{\small Attention: you are using an unstable development version.}
}

\makeatletter
\long\def\abstractsmuggle{%
    \centering
    \textbf{Abstract}\\[0.5cm]

    \begin{minipage}{12cm}
        \PGFPlots{} draws high-quality function plots in normal or logarithmic
        scaling with a user-friendly interface directly in \TeX{}. The user
        supplies axis labels, legend entries and the plot coordinates for one or
        more plots and \PGFPlots{} applies axis scaling, computes any logarithms
        and axis ticks and draws the plots. It supports line plots, scatter
        plots, piecewise constant plots, bar plots, area plots, mesh and surface
        plots, patch plots, contour plots, quiver plots, histogram plots, box
        plots, polar axes, ternary diagrams, smith charts and some more. It is
        based on Till Tantau's package \PGF{}/\Tikz{}.
    \end{minipage}

}%

\expandafter\date\expandafter{\@date\\[2cm]
    \abstractsmuggle
}%
\makeatother

%\includeonly{pgfplots.reference}


\begin{document}

\def\plotcoords{%
\addplot coordinates {
(5,8.312e-02)    (17,2.547e-02)   (49,7.407e-03)
(129,2.102e-03)  (321,5.874e-04)  (769,1.623e-04)
(1793,4.442e-05) (4097,1.207e-05) (9217,3.261e-06)
};

\addplot coordinates{
(7,8.472e-02)    (31,3.044e-02)    (111,1.022e-02)
(351,3.303e-03)  (1023,1.039e-03)  (2815,3.196e-04)
(7423,9.658e-05) (18943,2.873e-05) (47103,8.437e-06)
};

\addplot coordinates{
(9,7.881e-02)     (49,3.243e-02)    (209,1.232e-02)
(769,4.454e-03)   (2561,1.551e-03)  (7937,5.236e-04)
(23297,1.723e-04) (65537,5.545e-05) (178177,1.751e-05)
};

\addplot coordinates{
(11,6.887e-02)    (71,3.177e-02)     (351,1.341e-02)
(1471,5.334e-03)  (5503,2.027e-03)   (18943,7.415e-04)
(61183,2.628e-04) (187903,9.063e-05) (553983,3.053e-05)
};

\addplot coordinates{
(13,5.755e-02)     (97,2.925e-02)     (545,1.351e-02)
(2561,5.842e-03)   (10625,2.397e-03)  (40193,9.414e-04)
(141569,3.564e-04) (471041,1.308e-04)
(1496065,4.670e-05)
};
}%


\bookmaketitle
\tableofcontents
\include{pgfplots.title_abstract_intro}
\include{pgfplots.preliminaries}
\include{pgfplots.intro}
\include{pgfplots.reference}
\include{pgfplots.libs}
\include{pgfplots.resources}
\include{pgfplots.importexport}
\include{pgfplots.basic.reference}

\printindex

\bibliographystyle{abbrv} %gerapali} %gerabbrv} %gerunsrt.bst} %gerabbrv}% gerplain}
\nocite{pgfplotstable}
\nocite{programmingnotes}
\bibliography{pgfplots}
\end{document}

% -----------------------------------------------------------------------------
% For Stefan Pinnow as reminder on what to look for when editing the manual
% -----------------------------------------------------------------------------
% There should be no line breaks in the following environments
% - |...|
% - \declareandlabel{...}
% - \verbpdfref{...}
% If "MakeTikzPictures" isn't running through check one of the externalized
% LOG files what is the cause of that.
% -----------------------------------------------------------------------------


%%%%%%%%%%%%%%%%%%%%%%%%%%%%%%%%%%%%%%%%%%%%%%%%%%%%%%%%%%%%%%%%%%%%%%%%%%%%%
%
% Package pgfplots.sty documentation.
%
% Copyright 2007/2008 by Christian Feuersaenger.
%
% This program is free software: you can redistribute it and/or modify
% it under the terms of the GNU General Public License as published by
% the Free Software Foundation, either version 3 of the License, or
% (at your option) any later version.
%
% This program is distributed in the hope that it will be useful,
% but WITHOUT ANY WARRANTY; without even the implied warranty of
% MERCHANTABILITY or FITNESS FOR A PARTICULAR PURPOSE.  See the
% GNU General Public License for more details.
%
% You should have received a copy of the GNU General Public License
% along with this program.  If not, see <http://www.gnu.org/licenses/>.
%
%
%%%%%%%%%%%%%%%%%%%%%%%%%%%%%%%%%%%%%%%%%%%%%%%%%%%%%%%%%%%%%%%%%%%%%%%%%%%%%
% SEE pgfplots-macros.tex as well!
%\pdfminorversion=5 % to allow compression
%\pdfobjcompresslevel=2
\documentclass[a4paper,openany]{book}

\let\bookmaketitle=\maketitle

% -----------------------------------
% this here is from ltxdoc:
\usepackage{doc}[=v2]
\AtBeginDocument{\MakeShortVerb{\|}}
%\setlength{\textwidth}{355pt}
%\addtolength\marginparwidth{30pt}
%\addtolength\oddsidemargin{20pt}
%\addtolength\evensidemargin{20pt}
\def\cmd#1{\cs{\expandafter\cmd@to@cs\string#1}}
\def\cmd@to@cs#1#2{\char\number`#2\relax}
\DeclareRobustCommand\cs[1]{\texttt{\char`\\#1}}
\providecommand\marg[1]{%
  {\ttfamily\char`\{}\meta{#1}{\ttfamily\char`\}}}
\providecommand\oarg[1]{%
  {\ttfamily[}\meta{#1}{\ttfamily]}}
\providecommand\parg[1]{%
  {\ttfamily(}\meta{#1}{\ttfamily)}}
\raggedbottom

% -----------------------------------
\input{pgfplots.preamble.tex}

\makeatletter
% I want a two-column index just as in pgfmanual styles. This here
% was the best way to get one:
\def\index@prologue{\section*{Index}\addcontentsline{toc}{chapter}{Index}
}
\makeatother

%\RequirePackage[german,english,francais]{babel}

\def\matlabcolormaptext{This colormap is similar to one shipped with Matlab$^\text{\textregistered}$ under a similar name.}

\IfFileExists{tikzlibraryspy.code.tex}{%
\usetikzlibrary{spy}
}{%
    \message{ERROR: tikz SPY library NOT available. The manual will only compile partially.^^J}%
}%

\usepackage{xparse}% for colorbrewer manual

\usetikzlibrary{
    decorations.markings,
    decorations.footprints,
    shapes.arrows,
    matrix,
    positioning,
}

\usepgfplotslibrary{
    fillbetween,
    ternary,
    smithchart,
    patchplots,
    polar,
    colormaps,
    colorbrewer,
    colortol,           % docu not ready yet
}
\pgfqkeys{/codeexample}{%
    every codeexample/.append style={
        /pgfplots/every ternary axis/.append style={
            /pgfplots/legend style={fill=graphicbackground},
        }
    },
    tabsize=4,
}

\pgfplotsmanualenableexternalizationofexpensive

%\usetikzlibrary{external}
%\tikzexternalize[prefix=figures/]{pgfplots}

\title{%
    Manual for Package \PGFPlots{}\\
    {\small 2D/3D Plots in \LaTeX{}, Version \pgfplotsversion}\\
    {\small\url{https://github.com/pgf-tikz/pgfplots}}
    %\\{\small Attention: you are using an unstable development version.}
}

\makeatletter
\long\def\abstractsmuggle{%
    \centering
    \textbf{Abstract}\\[0.5cm]

    \begin{minipage}{12cm}
        \PGFPlots{} draws high-quality function plots in normal or logarithmic
        scaling with a user-friendly interface directly in \TeX{}. The user
        supplies axis labels, legend entries and the plot coordinates for one or
        more plots and \PGFPlots{} applies axis scaling, computes any logarithms
        and axis ticks and draws the plots. It supports line plots, scatter
        plots, piecewise constant plots, bar plots, area plots, mesh and surface
        plots, patch plots, contour plots, quiver plots, histogram plots, box
        plots, polar axes, ternary diagrams, smith charts and some more. It is
        based on Till Tantau's package \PGF{}/\Tikz{}.
    \end{minipage}

}%

\expandafter\date\expandafter{\@date\\[2cm]
    \abstractsmuggle
}%
\makeatother

%\includeonly{pgfplots.reference}


\begin{document}

\def\plotcoords{%
\addplot coordinates {
(5,8.312e-02)    (17,2.547e-02)   (49,7.407e-03)
(129,2.102e-03)  (321,5.874e-04)  (769,1.623e-04)
(1793,4.442e-05) (4097,1.207e-05) (9217,3.261e-06)
};

\addplot coordinates{
(7,8.472e-02)    (31,3.044e-02)    (111,1.022e-02)
(351,3.303e-03)  (1023,1.039e-03)  (2815,3.196e-04)
(7423,9.658e-05) (18943,2.873e-05) (47103,8.437e-06)
};

\addplot coordinates{
(9,7.881e-02)     (49,3.243e-02)    (209,1.232e-02)
(769,4.454e-03)   (2561,1.551e-03)  (7937,5.236e-04)
(23297,1.723e-04) (65537,5.545e-05) (178177,1.751e-05)
};

\addplot coordinates{
(11,6.887e-02)    (71,3.177e-02)     (351,1.341e-02)
(1471,5.334e-03)  (5503,2.027e-03)   (18943,7.415e-04)
(61183,2.628e-04) (187903,9.063e-05) (553983,3.053e-05)
};

\addplot coordinates{
(13,5.755e-02)     (97,2.925e-02)     (545,1.351e-02)
(2561,5.842e-03)   (10625,2.397e-03)  (40193,9.414e-04)
(141569,3.564e-04) (471041,1.308e-04)
(1496065,4.670e-05)
};
}%


\bookmaketitle
\tableofcontents
\include{pgfplots.title_abstract_intro}
\include{pgfplots.preliminaries}
\include{pgfplots.intro}
\include{pgfplots.reference}
\include{pgfplots.libs}
\include{pgfplots.resources}
\include{pgfplots.importexport}
\include{pgfplots.basic.reference}

\printindex

\bibliographystyle{abbrv} %gerapali} %gerabbrv} %gerunsrt.bst} %gerabbrv}% gerplain}
\nocite{pgfplotstable}
\nocite{programmingnotes}
\bibliography{pgfplots}
\end{document}

% -----------------------------------------------------------------------------
% For Stefan Pinnow as reminder on what to look for when editing the manual
% -----------------------------------------------------------------------------
% There should be no line breaks in the following environments
% - |...|
% - \declareandlabel{...}
% - \verbpdfref{...}
% If "MakeTikzPictures" isn't running through check one of the externalized
% LOG files what is the cause of that.
% -----------------------------------------------------------------------------


%%%%%%%%%%%%%%%%%%%%%%%%%%%%%%%%%%%%%%%%%%%%%%%%%%%%%%%%%%%%%%%%%%%%%%%%%%%%%
%
% Package pgfplots.sty documentation.
%
% Copyright 2007/2008 by Christian Feuersaenger.
%
% This program is free software: you can redistribute it and/or modify
% it under the terms of the GNU General Public License as published by
% the Free Software Foundation, either version 3 of the License, or
% (at your option) any later version.
%
% This program is distributed in the hope that it will be useful,
% but WITHOUT ANY WARRANTY; without even the implied warranty of
% MERCHANTABILITY or FITNESS FOR A PARTICULAR PURPOSE.  See the
% GNU General Public License for more details.
%
% You should have received a copy of the GNU General Public License
% along with this program.  If not, see <http://www.gnu.org/licenses/>.
%
%
%%%%%%%%%%%%%%%%%%%%%%%%%%%%%%%%%%%%%%%%%%%%%%%%%%%%%%%%%%%%%%%%%%%%%%%%%%%%%
% SEE pgfplots-macros.tex as well!
%\pdfminorversion=5 % to allow compression
%\pdfobjcompresslevel=2
\documentclass[a4paper,openany]{book}

\let\bookmaketitle=\maketitle

% -----------------------------------
% this here is from ltxdoc:
\usepackage{doc}[=v2]
\AtBeginDocument{\MakeShortVerb{\|}}
%\setlength{\textwidth}{355pt}
%\addtolength\marginparwidth{30pt}
%\addtolength\oddsidemargin{20pt}
%\addtolength\evensidemargin{20pt}
\def\cmd#1{\cs{\expandafter\cmd@to@cs\string#1}}
\def\cmd@to@cs#1#2{\char\number`#2\relax}
\DeclareRobustCommand\cs[1]{\texttt{\char`\\#1}}
\providecommand\marg[1]{%
  {\ttfamily\char`\{}\meta{#1}{\ttfamily\char`\}}}
\providecommand\oarg[1]{%
  {\ttfamily[}\meta{#1}{\ttfamily]}}
\providecommand\parg[1]{%
  {\ttfamily(}\meta{#1}{\ttfamily)}}
\raggedbottom

% -----------------------------------
\input{pgfplots.preamble.tex}

\makeatletter
% I want a two-column index just as in pgfmanual styles. This here
% was the best way to get one:
\def\index@prologue{\section*{Index}\addcontentsline{toc}{chapter}{Index}
}
\makeatother

%\RequirePackage[german,english,francais]{babel}

\def\matlabcolormaptext{This colormap is similar to one shipped with Matlab$^\text{\textregistered}$ under a similar name.}

\IfFileExists{tikzlibraryspy.code.tex}{%
\usetikzlibrary{spy}
}{%
    \message{ERROR: tikz SPY library NOT available. The manual will only compile partially.^^J}%
}%

\usepackage{xparse}% for colorbrewer manual

\usetikzlibrary{
    decorations.markings,
    decorations.footprints,
    shapes.arrows,
    matrix,
    positioning,
}

\usepgfplotslibrary{
    fillbetween,
    ternary,
    smithchart,
    patchplots,
    polar,
    colormaps,
    colorbrewer,
    colortol,           % docu not ready yet
}
\pgfqkeys{/codeexample}{%
    every codeexample/.append style={
        /pgfplots/every ternary axis/.append style={
            /pgfplots/legend style={fill=graphicbackground},
        }
    },
    tabsize=4,
}

\pgfplotsmanualenableexternalizationofexpensive

%\usetikzlibrary{external}
%\tikzexternalize[prefix=figures/]{pgfplots}

\title{%
    Manual for Package \PGFPlots{}\\
    {\small 2D/3D Plots in \LaTeX{}, Version \pgfplotsversion}\\
    {\small\url{https://github.com/pgf-tikz/pgfplots}}
    %\\{\small Attention: you are using an unstable development version.}
}

\makeatletter
\long\def\abstractsmuggle{%
    \centering
    \textbf{Abstract}\\[0.5cm]

    \begin{minipage}{12cm}
        \PGFPlots{} draws high-quality function plots in normal or logarithmic
        scaling with a user-friendly interface directly in \TeX{}. The user
        supplies axis labels, legend entries and the plot coordinates for one or
        more plots and \PGFPlots{} applies axis scaling, computes any logarithms
        and axis ticks and draws the plots. It supports line plots, scatter
        plots, piecewise constant plots, bar plots, area plots, mesh and surface
        plots, patch plots, contour plots, quiver plots, histogram plots, box
        plots, polar axes, ternary diagrams, smith charts and some more. It is
        based on Till Tantau's package \PGF{}/\Tikz{}.
    \end{minipage}

}%

\expandafter\date\expandafter{\@date\\[2cm]
    \abstractsmuggle
}%
\makeatother

%\includeonly{pgfplots.reference}


\begin{document}

\def\plotcoords{%
\addplot coordinates {
(5,8.312e-02)    (17,2.547e-02)   (49,7.407e-03)
(129,2.102e-03)  (321,5.874e-04)  (769,1.623e-04)
(1793,4.442e-05) (4097,1.207e-05) (9217,3.261e-06)
};

\addplot coordinates{
(7,8.472e-02)    (31,3.044e-02)    (111,1.022e-02)
(351,3.303e-03)  (1023,1.039e-03)  (2815,3.196e-04)
(7423,9.658e-05) (18943,2.873e-05) (47103,8.437e-06)
};

\addplot coordinates{
(9,7.881e-02)     (49,3.243e-02)    (209,1.232e-02)
(769,4.454e-03)   (2561,1.551e-03)  (7937,5.236e-04)
(23297,1.723e-04) (65537,5.545e-05) (178177,1.751e-05)
};

\addplot coordinates{
(11,6.887e-02)    (71,3.177e-02)     (351,1.341e-02)
(1471,5.334e-03)  (5503,2.027e-03)   (18943,7.415e-04)
(61183,2.628e-04) (187903,9.063e-05) (553983,3.053e-05)
};

\addplot coordinates{
(13,5.755e-02)     (97,2.925e-02)     (545,1.351e-02)
(2561,5.842e-03)   (10625,2.397e-03)  (40193,9.414e-04)
(141569,3.564e-04) (471041,1.308e-04)
(1496065,4.670e-05)
};
}%


\bookmaketitle
\tableofcontents
\include{pgfplots.title_abstract_intro}
\include{pgfplots.preliminaries}
\include{pgfplots.intro}
\include{pgfplots.reference}
\include{pgfplots.libs}
\include{pgfplots.resources}
\include{pgfplots.importexport}
\include{pgfplots.basic.reference}

\printindex

\bibliographystyle{abbrv} %gerapali} %gerabbrv} %gerunsrt.bst} %gerabbrv}% gerplain}
\nocite{pgfplotstable}
\nocite{programmingnotes}
\bibliography{pgfplots}
\end{document}

% -----------------------------------------------------------------------------
% For Stefan Pinnow as reminder on what to look for when editing the manual
% -----------------------------------------------------------------------------
% There should be no line breaks in the following environments
% - |...|
% - \declareandlabel{...}
% - \verbpdfref{...}
% If "MakeTikzPictures" isn't running through check one of the externalized
% LOG files what is the cause of that.
% -----------------------------------------------------------------------------


%%%%%%%%%%%%%%%%%%%%%%%%%%%%%%%%%%%%%%%%%%%%%%%%%%%%%%%%%%%%%%%%%%%%%%%%%%%%%
%
% Package pgfplots.sty documentation.
%
% Copyright 2007/2008 by Christian Feuersaenger.
%
% This program is free software: you can redistribute it and/or modify
% it under the terms of the GNU General Public License as published by
% the Free Software Foundation, either version 3 of the License, or
% (at your option) any later version.
%
% This program is distributed in the hope that it will be useful,
% but WITHOUT ANY WARRANTY; without even the implied warranty of
% MERCHANTABILITY or FITNESS FOR A PARTICULAR PURPOSE.  See the
% GNU General Public License for more details.
%
% You should have received a copy of the GNU General Public License
% along with this program.  If not, see <http://www.gnu.org/licenses/>.
%
%
%%%%%%%%%%%%%%%%%%%%%%%%%%%%%%%%%%%%%%%%%%%%%%%%%%%%%%%%%%%%%%%%%%%%%%%%%%%%%
% SEE pgfplots-macros.tex as well!
%\pdfminorversion=5 % to allow compression
%\pdfobjcompresslevel=2
\documentclass[a4paper,openany]{book}

\let\bookmaketitle=\maketitle

% -----------------------------------
% this here is from ltxdoc:
\usepackage{doc}[=v2]
\AtBeginDocument{\MakeShortVerb{\|}}
%\setlength{\textwidth}{355pt}
%\addtolength\marginparwidth{30pt}
%\addtolength\oddsidemargin{20pt}
%\addtolength\evensidemargin{20pt}
\def\cmd#1{\cs{\expandafter\cmd@to@cs\string#1}}
\def\cmd@to@cs#1#2{\char\number`#2\relax}
\DeclareRobustCommand\cs[1]{\texttt{\char`\\#1}}
\providecommand\marg[1]{%
  {\ttfamily\char`\{}\meta{#1}{\ttfamily\char`\}}}
\providecommand\oarg[1]{%
  {\ttfamily[}\meta{#1}{\ttfamily]}}
\providecommand\parg[1]{%
  {\ttfamily(}\meta{#1}{\ttfamily)}}
\raggedbottom

% -----------------------------------
\input{pgfplots.preamble.tex}

\makeatletter
% I want a two-column index just as in pgfmanual styles. This here
% was the best way to get one:
\def\index@prologue{\section*{Index}\addcontentsline{toc}{chapter}{Index}
}
\makeatother

%\RequirePackage[german,english,francais]{babel}

\def\matlabcolormaptext{This colormap is similar to one shipped with Matlab$^\text{\textregistered}$ under a similar name.}

\IfFileExists{tikzlibraryspy.code.tex}{%
\usetikzlibrary{spy}
}{%
    \message{ERROR: tikz SPY library NOT available. The manual will only compile partially.^^J}%
}%

\usepackage{xparse}% for colorbrewer manual

\usetikzlibrary{
    decorations.markings,
    decorations.footprints,
    shapes.arrows,
    matrix,
    positioning,
}

\usepgfplotslibrary{
    fillbetween,
    ternary,
    smithchart,
    patchplots,
    polar,
    colormaps,
    colorbrewer,
    colortol,           % docu not ready yet
}
\pgfqkeys{/codeexample}{%
    every codeexample/.append style={
        /pgfplots/every ternary axis/.append style={
            /pgfplots/legend style={fill=graphicbackground},
        }
    },
    tabsize=4,
}

\pgfplotsmanualenableexternalizationofexpensive

%\usetikzlibrary{external}
%\tikzexternalize[prefix=figures/]{pgfplots}

\title{%
    Manual for Package \PGFPlots{}\\
    {\small 2D/3D Plots in \LaTeX{}, Version \pgfplotsversion}\\
    {\small\url{https://github.com/pgf-tikz/pgfplots}}
    %\\{\small Attention: you are using an unstable development version.}
}

\makeatletter
\long\def\abstractsmuggle{%
    \centering
    \textbf{Abstract}\\[0.5cm]

    \begin{minipage}{12cm}
        \PGFPlots{} draws high-quality function plots in normal or logarithmic
        scaling with a user-friendly interface directly in \TeX{}. The user
        supplies axis labels, legend entries and the plot coordinates for one or
        more plots and \PGFPlots{} applies axis scaling, computes any logarithms
        and axis ticks and draws the plots. It supports line plots, scatter
        plots, piecewise constant plots, bar plots, area plots, mesh and surface
        plots, patch plots, contour plots, quiver plots, histogram plots, box
        plots, polar axes, ternary diagrams, smith charts and some more. It is
        based on Till Tantau's package \PGF{}/\Tikz{}.
    \end{minipage}

}%

\expandafter\date\expandafter{\@date\\[2cm]
    \abstractsmuggle
}%
\makeatother

%\includeonly{pgfplots.reference}


\begin{document}

\def\plotcoords{%
\addplot coordinates {
(5,8.312e-02)    (17,2.547e-02)   (49,7.407e-03)
(129,2.102e-03)  (321,5.874e-04)  (769,1.623e-04)
(1793,4.442e-05) (4097,1.207e-05) (9217,3.261e-06)
};

\addplot coordinates{
(7,8.472e-02)    (31,3.044e-02)    (111,1.022e-02)
(351,3.303e-03)  (1023,1.039e-03)  (2815,3.196e-04)
(7423,9.658e-05) (18943,2.873e-05) (47103,8.437e-06)
};

\addplot coordinates{
(9,7.881e-02)     (49,3.243e-02)    (209,1.232e-02)
(769,4.454e-03)   (2561,1.551e-03)  (7937,5.236e-04)
(23297,1.723e-04) (65537,5.545e-05) (178177,1.751e-05)
};

\addplot coordinates{
(11,6.887e-02)    (71,3.177e-02)     (351,1.341e-02)
(1471,5.334e-03)  (5503,2.027e-03)   (18943,7.415e-04)
(61183,2.628e-04) (187903,9.063e-05) (553983,3.053e-05)
};

\addplot coordinates{
(13,5.755e-02)     (97,2.925e-02)     (545,1.351e-02)
(2561,5.842e-03)   (10625,2.397e-03)  (40193,9.414e-04)
(141569,3.564e-04) (471041,1.308e-04)
(1496065,4.670e-05)
};
}%


\bookmaketitle
\tableofcontents
\include{pgfplots.title_abstract_intro}
\include{pgfplots.preliminaries}
\include{pgfplots.intro}
\include{pgfplots.reference}
\include{pgfplots.libs}
\include{pgfplots.resources}
\include{pgfplots.importexport}
\include{pgfplots.basic.reference}

\printindex

\bibliographystyle{abbrv} %gerapali} %gerabbrv} %gerunsrt.bst} %gerabbrv}% gerplain}
\nocite{pgfplotstable}
\nocite{programmingnotes}
\bibliography{pgfplots}
\end{document}

% -----------------------------------------------------------------------------
% For Stefan Pinnow as reminder on what to look for when editing the manual
% -----------------------------------------------------------------------------
% There should be no line breaks in the following environments
% - |...|
% - \declareandlabel{...}
% - \verbpdfref{...}
% If "MakeTikzPictures" isn't running through check one of the externalized
% LOG files what is the cause of that.
% -----------------------------------------------------------------------------


%%%%%%%%%%%%%%%%%%%%%%%%%%%%%%%%%%%%%%%%%%%%%%%%%%%%%%%%%%%%%%%%%%%%%%%%%%%%%
%
% Package pgfplots.sty documentation.
%
% Copyright 2007/2008 by Christian Feuersaenger.
%
% This program is free software: you can redistribute it and/or modify
% it under the terms of the GNU General Public License as published by
% the Free Software Foundation, either version 3 of the License, or
% (at your option) any later version.
%
% This program is distributed in the hope that it will be useful,
% but WITHOUT ANY WARRANTY; without even the implied warranty of
% MERCHANTABILITY or FITNESS FOR A PARTICULAR PURPOSE.  See the
% GNU General Public License for more details.
%
% You should have received a copy of the GNU General Public License
% along with this program.  If not, see <http://www.gnu.org/licenses/>.
%
%
%%%%%%%%%%%%%%%%%%%%%%%%%%%%%%%%%%%%%%%%%%%%%%%%%%%%%%%%%%%%%%%%%%%%%%%%%%%%%
% SEE pgfplots-macros.tex as well!
%\pdfminorversion=5 % to allow compression
%\pdfobjcompresslevel=2
\documentclass[a4paper,openany]{book}

\let\bookmaketitle=\maketitle

% -----------------------------------
% this here is from ltxdoc:
\usepackage{doc}[=v2]
\AtBeginDocument{\MakeShortVerb{\|}}
%\setlength{\textwidth}{355pt}
%\addtolength\marginparwidth{30pt}
%\addtolength\oddsidemargin{20pt}
%\addtolength\evensidemargin{20pt}
\def\cmd#1{\cs{\expandafter\cmd@to@cs\string#1}}
\def\cmd@to@cs#1#2{\char\number`#2\relax}
\DeclareRobustCommand\cs[1]{\texttt{\char`\\#1}}
\providecommand\marg[1]{%
  {\ttfamily\char`\{}\meta{#1}{\ttfamily\char`\}}}
\providecommand\oarg[1]{%
  {\ttfamily[}\meta{#1}{\ttfamily]}}
\providecommand\parg[1]{%
  {\ttfamily(}\meta{#1}{\ttfamily)}}
\raggedbottom

% -----------------------------------
\input{pgfplots.preamble.tex}

\makeatletter
% I want a two-column index just as in pgfmanual styles. This here
% was the best way to get one:
\def\index@prologue{\section*{Index}\addcontentsline{toc}{chapter}{Index}
}
\makeatother

%\RequirePackage[german,english,francais]{babel}

\def\matlabcolormaptext{This colormap is similar to one shipped with Matlab$^\text{\textregistered}$ under a similar name.}

\IfFileExists{tikzlibraryspy.code.tex}{%
\usetikzlibrary{spy}
}{%
    \message{ERROR: tikz SPY library NOT available. The manual will only compile partially.^^J}%
}%

\usepackage{xparse}% for colorbrewer manual

\usetikzlibrary{
    decorations.markings,
    decorations.footprints,
    shapes.arrows,
    matrix,
    positioning,
}

\usepgfplotslibrary{
    fillbetween,
    ternary,
    smithchart,
    patchplots,
    polar,
    colormaps,
    colorbrewer,
    colortol,           % docu not ready yet
}
\pgfqkeys{/codeexample}{%
    every codeexample/.append style={
        /pgfplots/every ternary axis/.append style={
            /pgfplots/legend style={fill=graphicbackground},
        }
    },
    tabsize=4,
}

\pgfplotsmanualenableexternalizationofexpensive

%\usetikzlibrary{external}
%\tikzexternalize[prefix=figures/]{pgfplots}

\title{%
    Manual for Package \PGFPlots{}\\
    {\small 2D/3D Plots in \LaTeX{}, Version \pgfplotsversion}\\
    {\small\url{https://github.com/pgf-tikz/pgfplots}}
    %\\{\small Attention: you are using an unstable development version.}
}

\makeatletter
\long\def\abstractsmuggle{%
    \centering
    \textbf{Abstract}\\[0.5cm]

    \begin{minipage}{12cm}
        \PGFPlots{} draws high-quality function plots in normal or logarithmic
        scaling with a user-friendly interface directly in \TeX{}. The user
        supplies axis labels, legend entries and the plot coordinates for one or
        more plots and \PGFPlots{} applies axis scaling, computes any logarithms
        and axis ticks and draws the plots. It supports line plots, scatter
        plots, piecewise constant plots, bar plots, area plots, mesh and surface
        plots, patch plots, contour plots, quiver plots, histogram plots, box
        plots, polar axes, ternary diagrams, smith charts and some more. It is
        based on Till Tantau's package \PGF{}/\Tikz{}.
    \end{minipage}

}%

\expandafter\date\expandafter{\@date\\[2cm]
    \abstractsmuggle
}%
\makeatother

%\includeonly{pgfplots.reference}


\begin{document}

\def\plotcoords{%
\addplot coordinates {
(5,8.312e-02)    (17,2.547e-02)   (49,7.407e-03)
(129,2.102e-03)  (321,5.874e-04)  (769,1.623e-04)
(1793,4.442e-05) (4097,1.207e-05) (9217,3.261e-06)
};

\addplot coordinates{
(7,8.472e-02)    (31,3.044e-02)    (111,1.022e-02)
(351,3.303e-03)  (1023,1.039e-03)  (2815,3.196e-04)
(7423,9.658e-05) (18943,2.873e-05) (47103,8.437e-06)
};

\addplot coordinates{
(9,7.881e-02)     (49,3.243e-02)    (209,1.232e-02)
(769,4.454e-03)   (2561,1.551e-03)  (7937,5.236e-04)
(23297,1.723e-04) (65537,5.545e-05) (178177,1.751e-05)
};

\addplot coordinates{
(11,6.887e-02)    (71,3.177e-02)     (351,1.341e-02)
(1471,5.334e-03)  (5503,2.027e-03)   (18943,7.415e-04)
(61183,2.628e-04) (187903,9.063e-05) (553983,3.053e-05)
};

\addplot coordinates{
(13,5.755e-02)     (97,2.925e-02)     (545,1.351e-02)
(2561,5.842e-03)   (10625,2.397e-03)  (40193,9.414e-04)
(141569,3.564e-04) (471041,1.308e-04)
(1496065,4.670e-05)
};
}%


\bookmaketitle
\tableofcontents
\include{pgfplots.title_abstract_intro}
\include{pgfplots.preliminaries}
\include{pgfplots.intro}
\include{pgfplots.reference}
\include{pgfplots.libs}
\include{pgfplots.resources}
\include{pgfplots.importexport}
\include{pgfplots.basic.reference}

\printindex

\bibliographystyle{abbrv} %gerapali} %gerabbrv} %gerunsrt.bst} %gerabbrv}% gerplain}
\nocite{pgfplotstable}
\nocite{programmingnotes}
\bibliography{pgfplots}
\end{document}

% -----------------------------------------------------------------------------
% For Stefan Pinnow as reminder on what to look for when editing the manual
% -----------------------------------------------------------------------------
% There should be no line breaks in the following environments
% - |...|
% - \declareandlabel{...}
% - \verbpdfref{...}
% If "MakeTikzPictures" isn't running through check one of the externalized
% LOG files what is the cause of that.
% -----------------------------------------------------------------------------

\include{pgfplots.basic.reference}

\printindex

\bibliographystyle{abbrv} %gerapali} %gerabbrv} %gerunsrt.bst} %gerabbrv}% gerplain}
\nocite{pgfplotstable}
\nocite{programmingnotes}
\bibliography{pgfplots}
\end{document}

% -----------------------------------------------------------------------------
% For Stefan Pinnow as reminder on what to look for when editing the manual
% -----------------------------------------------------------------------------
% There should be no line breaks in the following environments
% - |...|
% - \declareandlabel{...}
% - \verbpdfref{...}
% If "MakeTikzPictures" isn't running through check one of the externalized
% LOG files what is the cause of that.
% -----------------------------------------------------------------------------

\include{pgfplots.basic.reference}

\printindex

\bibliographystyle{abbrv} %gerapali} %gerabbrv} %gerunsrt.bst} %gerabbrv}% gerplain}
\nocite{pgfplotstable}
\nocite{programmingnotes}
\bibliography{pgfplots}
\end{document}

% -----------------------------------------------------------------------------
% For Stefan Pinnow as reminder on what to look for when editing the manual
% -----------------------------------------------------------------------------
% There should be no line breaks in the following environments
% - |...|
% - \declareandlabel{...}
% - \verbpdfref{...}
% If "MakeTikzPictures" isn't running through check one of the externalized
% LOG files what is the cause of that.
% -----------------------------------------------------------------------------


%%%%%%%%%%%%%%%%%%%%%%%%%%%%%%%%%%%%%%%%%%%%%%%%%%%%%%%%%%%%%%%%%%%%%%%%%%%%%
%
% Package pgfplots.sty documentation.
%
% Copyright 2007/2008 by Christian Feuersaenger.
%
% This program is free software: you can redistribute it and/or modify
% it under the terms of the GNU General Public License as published by
% the Free Software Foundation, either version 3 of the License, or
% (at your option) any later version.
%
% This program is distributed in the hope that it will be useful,
% but WITHOUT ANY WARRANTY; without even the implied warranty of
% MERCHANTABILITY or FITNESS FOR A PARTICULAR PURPOSE.  See the
% GNU General Public License for more details.
%
% You should have received a copy of the GNU General Public License
% along with this program.  If not, see <http://www.gnu.org/licenses/>.
%
%
%%%%%%%%%%%%%%%%%%%%%%%%%%%%%%%%%%%%%%%%%%%%%%%%%%%%%%%%%%%%%%%%%%%%%%%%%%%%%
% SEE pgfplots-macros.tex as well!
%\pdfminorversion=5 % to allow compression
%\pdfobjcompresslevel=2
\documentclass[a4paper,openany]{book}

\let\bookmaketitle=\maketitle

% -----------------------------------
% this here is from ltxdoc:
\usepackage{doc}[=v2]
\AtBeginDocument{\MakeShortVerb{\|}}
%\setlength{\textwidth}{355pt}
%\addtolength\marginparwidth{30pt}
%\addtolength\oddsidemargin{20pt}
%\addtolength\evensidemargin{20pt}
\def\cmd#1{\cs{\expandafter\cmd@to@cs\string#1}}
\def\cmd@to@cs#1#2{\char\number`#2\relax}
\DeclareRobustCommand\cs[1]{\texttt{\char`\\#1}}
\providecommand\marg[1]{%
  {\ttfamily\char`\{}\meta{#1}{\ttfamily\char`\}}}
\providecommand\oarg[1]{%
  {\ttfamily[}\meta{#1}{\ttfamily]}}
\providecommand\parg[1]{%
  {\ttfamily(}\meta{#1}{\ttfamily)}}
\raggedbottom

% -----------------------------------
\input{pgfplots.preamble.tex}

\makeatletter
% I want a two-column index just as in pgfmanual styles. This here
% was the best way to get one:
\def\index@prologue{\section*{Index}\addcontentsline{toc}{chapter}{Index}
}
\makeatother

%\RequirePackage[german,english,francais]{babel}

\def\matlabcolormaptext{This colormap is similar to one shipped with Matlab$^\text{\textregistered}$ under a similar name.}

\IfFileExists{tikzlibraryspy.code.tex}{%
\usetikzlibrary{spy}
}{%
    \message{ERROR: tikz SPY library NOT available. The manual will only compile partially.^^J}%
}%

\usepackage{xparse}% for colorbrewer manual

\usetikzlibrary{
    decorations.markings,
    decorations.footprints,
    shapes.arrows,
    matrix,
    positioning,
}

\usepgfplotslibrary{
    fillbetween,
    ternary,
    smithchart,
    patchplots,
    polar,
    colormaps,
    colorbrewer,
    colortol,           % docu not ready yet
}
\pgfqkeys{/codeexample}{%
    every codeexample/.append style={
        /pgfplots/every ternary axis/.append style={
            /pgfplots/legend style={fill=graphicbackground},
        }
    },
    tabsize=4,
}

\pgfplotsmanualenableexternalizationofexpensive

%\usetikzlibrary{external}
%\tikzexternalize[prefix=figures/]{pgfplots}

\title{%
    Manual for Package \PGFPlots{}\\
    {\small 2D/3D Plots in \LaTeX{}, Version \pgfplotsversion}\\
    {\small\url{https://github.com/pgf-tikz/pgfplots}}
    %\\{\small Attention: you are using an unstable development version.}
}

\makeatletter
\long\def\abstractsmuggle{%
    \centering
    \textbf{Abstract}\\[0.5cm]

    \begin{minipage}{12cm}
        \PGFPlots{} draws high-quality function plots in normal or logarithmic
        scaling with a user-friendly interface directly in \TeX{}. The user
        supplies axis labels, legend entries and the plot coordinates for one or
        more plots and \PGFPlots{} applies axis scaling, computes any logarithms
        and axis ticks and draws the plots. It supports line plots, scatter
        plots, piecewise constant plots, bar plots, area plots, mesh and surface
        plots, patch plots, contour plots, quiver plots, histogram plots, box
        plots, polar axes, ternary diagrams, smith charts and some more. It is
        based on Till Tantau's package \PGF{}/\Tikz{}.
    \end{minipage}

}%

\expandafter\date\expandafter{\@date\\[2cm]
    \abstractsmuggle
}%
\makeatother

%\includeonly{pgfplots.reference}


\begin{document}

\def\plotcoords{%
\addplot coordinates {
(5,8.312e-02)    (17,2.547e-02)   (49,7.407e-03)
(129,2.102e-03)  (321,5.874e-04)  (769,1.623e-04)
(1793,4.442e-05) (4097,1.207e-05) (9217,3.261e-06)
};

\addplot coordinates{
(7,8.472e-02)    (31,3.044e-02)    (111,1.022e-02)
(351,3.303e-03)  (1023,1.039e-03)  (2815,3.196e-04)
(7423,9.658e-05) (18943,2.873e-05) (47103,8.437e-06)
};

\addplot coordinates{
(9,7.881e-02)     (49,3.243e-02)    (209,1.232e-02)
(769,4.454e-03)   (2561,1.551e-03)  (7937,5.236e-04)
(23297,1.723e-04) (65537,5.545e-05) (178177,1.751e-05)
};

\addplot coordinates{
(11,6.887e-02)    (71,3.177e-02)     (351,1.341e-02)
(1471,5.334e-03)  (5503,2.027e-03)   (18943,7.415e-04)
(61183,2.628e-04) (187903,9.063e-05) (553983,3.053e-05)
};

\addplot coordinates{
(13,5.755e-02)     (97,2.925e-02)     (545,1.351e-02)
(2561,5.842e-03)   (10625,2.397e-03)  (40193,9.414e-04)
(141569,3.564e-04) (471041,1.308e-04)
(1496065,4.670e-05)
};
}%


\bookmaketitle
\tableofcontents

%%%%%%%%%%%%%%%%%%%%%%%%%%%%%%%%%%%%%%%%%%%%%%%%%%%%%%%%%%%%%%%%%%%%%%%%%%%%%
%
% Package pgfplots.sty documentation.
%
% Copyright 2007/2008 by Christian Feuersaenger.
%
% This program is free software: you can redistribute it and/or modify
% it under the terms of the GNU General Public License as published by
% the Free Software Foundation, either version 3 of the License, or
% (at your option) any later version.
%
% This program is distributed in the hope that it will be useful,
% but WITHOUT ANY WARRANTY; without even the implied warranty of
% MERCHANTABILITY or FITNESS FOR A PARTICULAR PURPOSE.  See the
% GNU General Public License for more details.
%
% You should have received a copy of the GNU General Public License
% along with this program.  If not, see <http://www.gnu.org/licenses/>.
%
%
%%%%%%%%%%%%%%%%%%%%%%%%%%%%%%%%%%%%%%%%%%%%%%%%%%%%%%%%%%%%%%%%%%%%%%%%%%%%%
% SEE pgfplots-macros.tex as well!
%\pdfminorversion=5 % to allow compression
%\pdfobjcompresslevel=2
\documentclass[a4paper,openany]{book}

\let\bookmaketitle=\maketitle

% -----------------------------------
% this here is from ltxdoc:
\usepackage{doc}[=v2]
\AtBeginDocument{\MakeShortVerb{\|}}
%\setlength{\textwidth}{355pt}
%\addtolength\marginparwidth{30pt}
%\addtolength\oddsidemargin{20pt}
%\addtolength\evensidemargin{20pt}
\def\cmd#1{\cs{\expandafter\cmd@to@cs\string#1}}
\def\cmd@to@cs#1#2{\char\number`#2\relax}
\DeclareRobustCommand\cs[1]{\texttt{\char`\\#1}}
\providecommand\marg[1]{%
  {\ttfamily\char`\{}\meta{#1}{\ttfamily\char`\}}}
\providecommand\oarg[1]{%
  {\ttfamily[}\meta{#1}{\ttfamily]}}
\providecommand\parg[1]{%
  {\ttfamily(}\meta{#1}{\ttfamily)}}
\raggedbottom

% -----------------------------------
\input{pgfplots.preamble.tex}

\makeatletter
% I want a two-column index just as in pgfmanual styles. This here
% was the best way to get one:
\def\index@prologue{\section*{Index}\addcontentsline{toc}{chapter}{Index}
}
\makeatother

%\RequirePackage[german,english,francais]{babel}

\def\matlabcolormaptext{This colormap is similar to one shipped with Matlab$^\text{\textregistered}$ under a similar name.}

\IfFileExists{tikzlibraryspy.code.tex}{%
\usetikzlibrary{spy}
}{%
    \message{ERROR: tikz SPY library NOT available. The manual will only compile partially.^^J}%
}%

\usepackage{xparse}% for colorbrewer manual

\usetikzlibrary{
    decorations.markings,
    decorations.footprints,
    shapes.arrows,
    matrix,
    positioning,
}

\usepgfplotslibrary{
    fillbetween,
    ternary,
    smithchart,
    patchplots,
    polar,
    colormaps,
    colorbrewer,
    colortol,           % docu not ready yet
}
\pgfqkeys{/codeexample}{%
    every codeexample/.append style={
        /pgfplots/every ternary axis/.append style={
            /pgfplots/legend style={fill=graphicbackground},
        }
    },
    tabsize=4,
}

\pgfplotsmanualenableexternalizationofexpensive

%\usetikzlibrary{external}
%\tikzexternalize[prefix=figures/]{pgfplots}

\title{%
    Manual for Package \PGFPlots{}\\
    {\small 2D/3D Plots in \LaTeX{}, Version \pgfplotsversion}\\
    {\small\url{https://github.com/pgf-tikz/pgfplots}}
    %\\{\small Attention: you are using an unstable development version.}
}

\makeatletter
\long\def\abstractsmuggle{%
    \centering
    \textbf{Abstract}\\[0.5cm]

    \begin{minipage}{12cm}
        \PGFPlots{} draws high-quality function plots in normal or logarithmic
        scaling with a user-friendly interface directly in \TeX{}. The user
        supplies axis labels, legend entries and the plot coordinates for one or
        more plots and \PGFPlots{} applies axis scaling, computes any logarithms
        and axis ticks and draws the plots. It supports line plots, scatter
        plots, piecewise constant plots, bar plots, area plots, mesh and surface
        plots, patch plots, contour plots, quiver plots, histogram plots, box
        plots, polar axes, ternary diagrams, smith charts and some more. It is
        based on Till Tantau's package \PGF{}/\Tikz{}.
    \end{minipage}

}%

\expandafter\date\expandafter{\@date\\[2cm]
    \abstractsmuggle
}%
\makeatother

%\includeonly{pgfplots.reference}


\begin{document}

\def\plotcoords{%
\addplot coordinates {
(5,8.312e-02)    (17,2.547e-02)   (49,7.407e-03)
(129,2.102e-03)  (321,5.874e-04)  (769,1.623e-04)
(1793,4.442e-05) (4097,1.207e-05) (9217,3.261e-06)
};

\addplot coordinates{
(7,8.472e-02)    (31,3.044e-02)    (111,1.022e-02)
(351,3.303e-03)  (1023,1.039e-03)  (2815,3.196e-04)
(7423,9.658e-05) (18943,2.873e-05) (47103,8.437e-06)
};

\addplot coordinates{
(9,7.881e-02)     (49,3.243e-02)    (209,1.232e-02)
(769,4.454e-03)   (2561,1.551e-03)  (7937,5.236e-04)
(23297,1.723e-04) (65537,5.545e-05) (178177,1.751e-05)
};

\addplot coordinates{
(11,6.887e-02)    (71,3.177e-02)     (351,1.341e-02)
(1471,5.334e-03)  (5503,2.027e-03)   (18943,7.415e-04)
(61183,2.628e-04) (187903,9.063e-05) (553983,3.053e-05)
};

\addplot coordinates{
(13,5.755e-02)     (97,2.925e-02)     (545,1.351e-02)
(2561,5.842e-03)   (10625,2.397e-03)  (40193,9.414e-04)
(141569,3.564e-04) (471041,1.308e-04)
(1496065,4.670e-05)
};
}%


\bookmaketitle
\tableofcontents

%%%%%%%%%%%%%%%%%%%%%%%%%%%%%%%%%%%%%%%%%%%%%%%%%%%%%%%%%%%%%%%%%%%%%%%%%%%%%
%
% Package pgfplots.sty documentation.
%
% Copyright 2007/2008 by Christian Feuersaenger.
%
% This program is free software: you can redistribute it and/or modify
% it under the terms of the GNU General Public License as published by
% the Free Software Foundation, either version 3 of the License, or
% (at your option) any later version.
%
% This program is distributed in the hope that it will be useful,
% but WITHOUT ANY WARRANTY; without even the implied warranty of
% MERCHANTABILITY or FITNESS FOR A PARTICULAR PURPOSE.  See the
% GNU General Public License for more details.
%
% You should have received a copy of the GNU General Public License
% along with this program.  If not, see <http://www.gnu.org/licenses/>.
%
%
%%%%%%%%%%%%%%%%%%%%%%%%%%%%%%%%%%%%%%%%%%%%%%%%%%%%%%%%%%%%%%%%%%%%%%%%%%%%%
% SEE pgfplots-macros.tex as well!
%\pdfminorversion=5 % to allow compression
%\pdfobjcompresslevel=2
\documentclass[a4paper,openany]{book}

\let\bookmaketitle=\maketitle

% -----------------------------------
% this here is from ltxdoc:
\usepackage{doc}[=v2]
\AtBeginDocument{\MakeShortVerb{\|}}
%\setlength{\textwidth}{355pt}
%\addtolength\marginparwidth{30pt}
%\addtolength\oddsidemargin{20pt}
%\addtolength\evensidemargin{20pt}
\def\cmd#1{\cs{\expandafter\cmd@to@cs\string#1}}
\def\cmd@to@cs#1#2{\char\number`#2\relax}
\DeclareRobustCommand\cs[1]{\texttt{\char`\\#1}}
\providecommand\marg[1]{%
  {\ttfamily\char`\{}\meta{#1}{\ttfamily\char`\}}}
\providecommand\oarg[1]{%
  {\ttfamily[}\meta{#1}{\ttfamily]}}
\providecommand\parg[1]{%
  {\ttfamily(}\meta{#1}{\ttfamily)}}
\raggedbottom

% -----------------------------------
\input{pgfplots.preamble.tex}

\makeatletter
% I want a two-column index just as in pgfmanual styles. This here
% was the best way to get one:
\def\index@prologue{\section*{Index}\addcontentsline{toc}{chapter}{Index}
}
\makeatother

%\RequirePackage[german,english,francais]{babel}

\def\matlabcolormaptext{This colormap is similar to one shipped with Matlab$^\text{\textregistered}$ under a similar name.}

\IfFileExists{tikzlibraryspy.code.tex}{%
\usetikzlibrary{spy}
}{%
    \message{ERROR: tikz SPY library NOT available. The manual will only compile partially.^^J}%
}%

\usepackage{xparse}% for colorbrewer manual

\usetikzlibrary{
    decorations.markings,
    decorations.footprints,
    shapes.arrows,
    matrix,
    positioning,
}

\usepgfplotslibrary{
    fillbetween,
    ternary,
    smithchart,
    patchplots,
    polar,
    colormaps,
    colorbrewer,
    colortol,           % docu not ready yet
}
\pgfqkeys{/codeexample}{%
    every codeexample/.append style={
        /pgfplots/every ternary axis/.append style={
            /pgfplots/legend style={fill=graphicbackground},
        }
    },
    tabsize=4,
}

\pgfplotsmanualenableexternalizationofexpensive

%\usetikzlibrary{external}
%\tikzexternalize[prefix=figures/]{pgfplots}

\title{%
    Manual for Package \PGFPlots{}\\
    {\small 2D/3D Plots in \LaTeX{}, Version \pgfplotsversion}\\
    {\small\url{https://github.com/pgf-tikz/pgfplots}}
    %\\{\small Attention: you are using an unstable development version.}
}

\makeatletter
\long\def\abstractsmuggle{%
    \centering
    \textbf{Abstract}\\[0.5cm]

    \begin{minipage}{12cm}
        \PGFPlots{} draws high-quality function plots in normal or logarithmic
        scaling with a user-friendly interface directly in \TeX{}. The user
        supplies axis labels, legend entries and the plot coordinates for one or
        more plots and \PGFPlots{} applies axis scaling, computes any logarithms
        and axis ticks and draws the plots. It supports line plots, scatter
        plots, piecewise constant plots, bar plots, area plots, mesh and surface
        plots, patch plots, contour plots, quiver plots, histogram plots, box
        plots, polar axes, ternary diagrams, smith charts and some more. It is
        based on Till Tantau's package \PGF{}/\Tikz{}.
    \end{minipage}

}%

\expandafter\date\expandafter{\@date\\[2cm]
    \abstractsmuggle
}%
\makeatother

%\includeonly{pgfplots.reference}


\begin{document}

\def\plotcoords{%
\addplot coordinates {
(5,8.312e-02)    (17,2.547e-02)   (49,7.407e-03)
(129,2.102e-03)  (321,5.874e-04)  (769,1.623e-04)
(1793,4.442e-05) (4097,1.207e-05) (9217,3.261e-06)
};

\addplot coordinates{
(7,8.472e-02)    (31,3.044e-02)    (111,1.022e-02)
(351,3.303e-03)  (1023,1.039e-03)  (2815,3.196e-04)
(7423,9.658e-05) (18943,2.873e-05) (47103,8.437e-06)
};

\addplot coordinates{
(9,7.881e-02)     (49,3.243e-02)    (209,1.232e-02)
(769,4.454e-03)   (2561,1.551e-03)  (7937,5.236e-04)
(23297,1.723e-04) (65537,5.545e-05) (178177,1.751e-05)
};

\addplot coordinates{
(11,6.887e-02)    (71,3.177e-02)     (351,1.341e-02)
(1471,5.334e-03)  (5503,2.027e-03)   (18943,7.415e-04)
(61183,2.628e-04) (187903,9.063e-05) (553983,3.053e-05)
};

\addplot coordinates{
(13,5.755e-02)     (97,2.925e-02)     (545,1.351e-02)
(2561,5.842e-03)   (10625,2.397e-03)  (40193,9.414e-04)
(141569,3.564e-04) (471041,1.308e-04)
(1496065,4.670e-05)
};
}%


\bookmaketitle
\tableofcontents
\include{pgfplots.title_abstract_intro}
\include{pgfplots.preliminaries}
\include{pgfplots.intro}
\include{pgfplots.reference}
\include{pgfplots.libs}
\include{pgfplots.resources}
\include{pgfplots.importexport}
\include{pgfplots.basic.reference}

\printindex

\bibliographystyle{abbrv} %gerapali} %gerabbrv} %gerunsrt.bst} %gerabbrv}% gerplain}
\nocite{pgfplotstable}
\nocite{programmingnotes}
\bibliography{pgfplots}
\end{document}

% -----------------------------------------------------------------------------
% For Stefan Pinnow as reminder on what to look for when editing the manual
% -----------------------------------------------------------------------------
% There should be no line breaks in the following environments
% - |...|
% - \declareandlabel{...}
% - \verbpdfref{...}
% If "MakeTikzPictures" isn't running through check one of the externalized
% LOG files what is the cause of that.
% -----------------------------------------------------------------------------


%%%%%%%%%%%%%%%%%%%%%%%%%%%%%%%%%%%%%%%%%%%%%%%%%%%%%%%%%%%%%%%%%%%%%%%%%%%%%
%
% Package pgfplots.sty documentation.
%
% Copyright 2007/2008 by Christian Feuersaenger.
%
% This program is free software: you can redistribute it and/or modify
% it under the terms of the GNU General Public License as published by
% the Free Software Foundation, either version 3 of the License, or
% (at your option) any later version.
%
% This program is distributed in the hope that it will be useful,
% but WITHOUT ANY WARRANTY; without even the implied warranty of
% MERCHANTABILITY or FITNESS FOR A PARTICULAR PURPOSE.  See the
% GNU General Public License for more details.
%
% You should have received a copy of the GNU General Public License
% along with this program.  If not, see <http://www.gnu.org/licenses/>.
%
%
%%%%%%%%%%%%%%%%%%%%%%%%%%%%%%%%%%%%%%%%%%%%%%%%%%%%%%%%%%%%%%%%%%%%%%%%%%%%%
% SEE pgfplots-macros.tex as well!
%\pdfminorversion=5 % to allow compression
%\pdfobjcompresslevel=2
\documentclass[a4paper,openany]{book}

\let\bookmaketitle=\maketitle

% -----------------------------------
% this here is from ltxdoc:
\usepackage{doc}[=v2]
\AtBeginDocument{\MakeShortVerb{\|}}
%\setlength{\textwidth}{355pt}
%\addtolength\marginparwidth{30pt}
%\addtolength\oddsidemargin{20pt}
%\addtolength\evensidemargin{20pt}
\def\cmd#1{\cs{\expandafter\cmd@to@cs\string#1}}
\def\cmd@to@cs#1#2{\char\number`#2\relax}
\DeclareRobustCommand\cs[1]{\texttt{\char`\\#1}}
\providecommand\marg[1]{%
  {\ttfamily\char`\{}\meta{#1}{\ttfamily\char`\}}}
\providecommand\oarg[1]{%
  {\ttfamily[}\meta{#1}{\ttfamily]}}
\providecommand\parg[1]{%
  {\ttfamily(}\meta{#1}{\ttfamily)}}
\raggedbottom

% -----------------------------------
\input{pgfplots.preamble.tex}

\makeatletter
% I want a two-column index just as in pgfmanual styles. This here
% was the best way to get one:
\def\index@prologue{\section*{Index}\addcontentsline{toc}{chapter}{Index}
}
\makeatother

%\RequirePackage[german,english,francais]{babel}

\def\matlabcolormaptext{This colormap is similar to one shipped with Matlab$^\text{\textregistered}$ under a similar name.}

\IfFileExists{tikzlibraryspy.code.tex}{%
\usetikzlibrary{spy}
}{%
    \message{ERROR: tikz SPY library NOT available. The manual will only compile partially.^^J}%
}%

\usepackage{xparse}% for colorbrewer manual

\usetikzlibrary{
    decorations.markings,
    decorations.footprints,
    shapes.arrows,
    matrix,
    positioning,
}

\usepgfplotslibrary{
    fillbetween,
    ternary,
    smithchart,
    patchplots,
    polar,
    colormaps,
    colorbrewer,
    colortol,           % docu not ready yet
}
\pgfqkeys{/codeexample}{%
    every codeexample/.append style={
        /pgfplots/every ternary axis/.append style={
            /pgfplots/legend style={fill=graphicbackground},
        }
    },
    tabsize=4,
}

\pgfplotsmanualenableexternalizationofexpensive

%\usetikzlibrary{external}
%\tikzexternalize[prefix=figures/]{pgfplots}

\title{%
    Manual for Package \PGFPlots{}\\
    {\small 2D/3D Plots in \LaTeX{}, Version \pgfplotsversion}\\
    {\small\url{https://github.com/pgf-tikz/pgfplots}}
    %\\{\small Attention: you are using an unstable development version.}
}

\makeatletter
\long\def\abstractsmuggle{%
    \centering
    \textbf{Abstract}\\[0.5cm]

    \begin{minipage}{12cm}
        \PGFPlots{} draws high-quality function plots in normal or logarithmic
        scaling with a user-friendly interface directly in \TeX{}. The user
        supplies axis labels, legend entries and the plot coordinates for one or
        more plots and \PGFPlots{} applies axis scaling, computes any logarithms
        and axis ticks and draws the plots. It supports line plots, scatter
        plots, piecewise constant plots, bar plots, area plots, mesh and surface
        plots, patch plots, contour plots, quiver plots, histogram plots, box
        plots, polar axes, ternary diagrams, smith charts and some more. It is
        based on Till Tantau's package \PGF{}/\Tikz{}.
    \end{minipage}

}%

\expandafter\date\expandafter{\@date\\[2cm]
    \abstractsmuggle
}%
\makeatother

%\includeonly{pgfplots.reference}


\begin{document}

\def\plotcoords{%
\addplot coordinates {
(5,8.312e-02)    (17,2.547e-02)   (49,7.407e-03)
(129,2.102e-03)  (321,5.874e-04)  (769,1.623e-04)
(1793,4.442e-05) (4097,1.207e-05) (9217,3.261e-06)
};

\addplot coordinates{
(7,8.472e-02)    (31,3.044e-02)    (111,1.022e-02)
(351,3.303e-03)  (1023,1.039e-03)  (2815,3.196e-04)
(7423,9.658e-05) (18943,2.873e-05) (47103,8.437e-06)
};

\addplot coordinates{
(9,7.881e-02)     (49,3.243e-02)    (209,1.232e-02)
(769,4.454e-03)   (2561,1.551e-03)  (7937,5.236e-04)
(23297,1.723e-04) (65537,5.545e-05) (178177,1.751e-05)
};

\addplot coordinates{
(11,6.887e-02)    (71,3.177e-02)     (351,1.341e-02)
(1471,5.334e-03)  (5503,2.027e-03)   (18943,7.415e-04)
(61183,2.628e-04) (187903,9.063e-05) (553983,3.053e-05)
};

\addplot coordinates{
(13,5.755e-02)     (97,2.925e-02)     (545,1.351e-02)
(2561,5.842e-03)   (10625,2.397e-03)  (40193,9.414e-04)
(141569,3.564e-04) (471041,1.308e-04)
(1496065,4.670e-05)
};
}%


\bookmaketitle
\tableofcontents
\include{pgfplots.title_abstract_intro}
\include{pgfplots.preliminaries}
\include{pgfplots.intro}
\include{pgfplots.reference}
\include{pgfplots.libs}
\include{pgfplots.resources}
\include{pgfplots.importexport}
\include{pgfplots.basic.reference}

\printindex

\bibliographystyle{abbrv} %gerapali} %gerabbrv} %gerunsrt.bst} %gerabbrv}% gerplain}
\nocite{pgfplotstable}
\nocite{programmingnotes}
\bibliography{pgfplots}
\end{document}

% -----------------------------------------------------------------------------
% For Stefan Pinnow as reminder on what to look for when editing the manual
% -----------------------------------------------------------------------------
% There should be no line breaks in the following environments
% - |...|
% - \declareandlabel{...}
% - \verbpdfref{...}
% If "MakeTikzPictures" isn't running through check one of the externalized
% LOG files what is the cause of that.
% -----------------------------------------------------------------------------


%%%%%%%%%%%%%%%%%%%%%%%%%%%%%%%%%%%%%%%%%%%%%%%%%%%%%%%%%%%%%%%%%%%%%%%%%%%%%
%
% Package pgfplots.sty documentation.
%
% Copyright 2007/2008 by Christian Feuersaenger.
%
% This program is free software: you can redistribute it and/or modify
% it under the terms of the GNU General Public License as published by
% the Free Software Foundation, either version 3 of the License, or
% (at your option) any later version.
%
% This program is distributed in the hope that it will be useful,
% but WITHOUT ANY WARRANTY; without even the implied warranty of
% MERCHANTABILITY or FITNESS FOR A PARTICULAR PURPOSE.  See the
% GNU General Public License for more details.
%
% You should have received a copy of the GNU General Public License
% along with this program.  If not, see <http://www.gnu.org/licenses/>.
%
%
%%%%%%%%%%%%%%%%%%%%%%%%%%%%%%%%%%%%%%%%%%%%%%%%%%%%%%%%%%%%%%%%%%%%%%%%%%%%%
% SEE pgfplots-macros.tex as well!
%\pdfminorversion=5 % to allow compression
%\pdfobjcompresslevel=2
\documentclass[a4paper,openany]{book}

\let\bookmaketitle=\maketitle

% -----------------------------------
% this here is from ltxdoc:
\usepackage{doc}[=v2]
\AtBeginDocument{\MakeShortVerb{\|}}
%\setlength{\textwidth}{355pt}
%\addtolength\marginparwidth{30pt}
%\addtolength\oddsidemargin{20pt}
%\addtolength\evensidemargin{20pt}
\def\cmd#1{\cs{\expandafter\cmd@to@cs\string#1}}
\def\cmd@to@cs#1#2{\char\number`#2\relax}
\DeclareRobustCommand\cs[1]{\texttt{\char`\\#1}}
\providecommand\marg[1]{%
  {\ttfamily\char`\{}\meta{#1}{\ttfamily\char`\}}}
\providecommand\oarg[1]{%
  {\ttfamily[}\meta{#1}{\ttfamily]}}
\providecommand\parg[1]{%
  {\ttfamily(}\meta{#1}{\ttfamily)}}
\raggedbottom

% -----------------------------------
\input{pgfplots.preamble.tex}

\makeatletter
% I want a two-column index just as in pgfmanual styles. This here
% was the best way to get one:
\def\index@prologue{\section*{Index}\addcontentsline{toc}{chapter}{Index}
}
\makeatother

%\RequirePackage[german,english,francais]{babel}

\def\matlabcolormaptext{This colormap is similar to one shipped with Matlab$^\text{\textregistered}$ under a similar name.}

\IfFileExists{tikzlibraryspy.code.tex}{%
\usetikzlibrary{spy}
}{%
    \message{ERROR: tikz SPY library NOT available. The manual will only compile partially.^^J}%
}%

\usepackage{xparse}% for colorbrewer manual

\usetikzlibrary{
    decorations.markings,
    decorations.footprints,
    shapes.arrows,
    matrix,
    positioning,
}

\usepgfplotslibrary{
    fillbetween,
    ternary,
    smithchart,
    patchplots,
    polar,
    colormaps,
    colorbrewer,
    colortol,           % docu not ready yet
}
\pgfqkeys{/codeexample}{%
    every codeexample/.append style={
        /pgfplots/every ternary axis/.append style={
            /pgfplots/legend style={fill=graphicbackground},
        }
    },
    tabsize=4,
}

\pgfplotsmanualenableexternalizationofexpensive

%\usetikzlibrary{external}
%\tikzexternalize[prefix=figures/]{pgfplots}

\title{%
    Manual for Package \PGFPlots{}\\
    {\small 2D/3D Plots in \LaTeX{}, Version \pgfplotsversion}\\
    {\small\url{https://github.com/pgf-tikz/pgfplots}}
    %\\{\small Attention: you are using an unstable development version.}
}

\makeatletter
\long\def\abstractsmuggle{%
    \centering
    \textbf{Abstract}\\[0.5cm]

    \begin{minipage}{12cm}
        \PGFPlots{} draws high-quality function plots in normal or logarithmic
        scaling with a user-friendly interface directly in \TeX{}. The user
        supplies axis labels, legend entries and the plot coordinates for one or
        more plots and \PGFPlots{} applies axis scaling, computes any logarithms
        and axis ticks and draws the plots. It supports line plots, scatter
        plots, piecewise constant plots, bar plots, area plots, mesh and surface
        plots, patch plots, contour plots, quiver plots, histogram plots, box
        plots, polar axes, ternary diagrams, smith charts and some more. It is
        based on Till Tantau's package \PGF{}/\Tikz{}.
    \end{minipage}

}%

\expandafter\date\expandafter{\@date\\[2cm]
    \abstractsmuggle
}%
\makeatother

%\includeonly{pgfplots.reference}


\begin{document}

\def\plotcoords{%
\addplot coordinates {
(5,8.312e-02)    (17,2.547e-02)   (49,7.407e-03)
(129,2.102e-03)  (321,5.874e-04)  (769,1.623e-04)
(1793,4.442e-05) (4097,1.207e-05) (9217,3.261e-06)
};

\addplot coordinates{
(7,8.472e-02)    (31,3.044e-02)    (111,1.022e-02)
(351,3.303e-03)  (1023,1.039e-03)  (2815,3.196e-04)
(7423,9.658e-05) (18943,2.873e-05) (47103,8.437e-06)
};

\addplot coordinates{
(9,7.881e-02)     (49,3.243e-02)    (209,1.232e-02)
(769,4.454e-03)   (2561,1.551e-03)  (7937,5.236e-04)
(23297,1.723e-04) (65537,5.545e-05) (178177,1.751e-05)
};

\addplot coordinates{
(11,6.887e-02)    (71,3.177e-02)     (351,1.341e-02)
(1471,5.334e-03)  (5503,2.027e-03)   (18943,7.415e-04)
(61183,2.628e-04) (187903,9.063e-05) (553983,3.053e-05)
};

\addplot coordinates{
(13,5.755e-02)     (97,2.925e-02)     (545,1.351e-02)
(2561,5.842e-03)   (10625,2.397e-03)  (40193,9.414e-04)
(141569,3.564e-04) (471041,1.308e-04)
(1496065,4.670e-05)
};
}%


\bookmaketitle
\tableofcontents
\include{pgfplots.title_abstract_intro}
\include{pgfplots.preliminaries}
\include{pgfplots.intro}
\include{pgfplots.reference}
\include{pgfplots.libs}
\include{pgfplots.resources}
\include{pgfplots.importexport}
\include{pgfplots.basic.reference}

\printindex

\bibliographystyle{abbrv} %gerapali} %gerabbrv} %gerunsrt.bst} %gerabbrv}% gerplain}
\nocite{pgfplotstable}
\nocite{programmingnotes}
\bibliography{pgfplots}
\end{document}

% -----------------------------------------------------------------------------
% For Stefan Pinnow as reminder on what to look for when editing the manual
% -----------------------------------------------------------------------------
% There should be no line breaks in the following environments
% - |...|
% - \declareandlabel{...}
% - \verbpdfref{...}
% If "MakeTikzPictures" isn't running through check one of the externalized
% LOG files what is the cause of that.
% -----------------------------------------------------------------------------


%%%%%%%%%%%%%%%%%%%%%%%%%%%%%%%%%%%%%%%%%%%%%%%%%%%%%%%%%%%%%%%%%%%%%%%%%%%%%
%
% Package pgfplots.sty documentation.
%
% Copyright 2007/2008 by Christian Feuersaenger.
%
% This program is free software: you can redistribute it and/or modify
% it under the terms of the GNU General Public License as published by
% the Free Software Foundation, either version 3 of the License, or
% (at your option) any later version.
%
% This program is distributed in the hope that it will be useful,
% but WITHOUT ANY WARRANTY; without even the implied warranty of
% MERCHANTABILITY or FITNESS FOR A PARTICULAR PURPOSE.  See the
% GNU General Public License for more details.
%
% You should have received a copy of the GNU General Public License
% along with this program.  If not, see <http://www.gnu.org/licenses/>.
%
%
%%%%%%%%%%%%%%%%%%%%%%%%%%%%%%%%%%%%%%%%%%%%%%%%%%%%%%%%%%%%%%%%%%%%%%%%%%%%%
% SEE pgfplots-macros.tex as well!
%\pdfminorversion=5 % to allow compression
%\pdfobjcompresslevel=2
\documentclass[a4paper,openany]{book}

\let\bookmaketitle=\maketitle

% -----------------------------------
% this here is from ltxdoc:
\usepackage{doc}[=v2]
\AtBeginDocument{\MakeShortVerb{\|}}
%\setlength{\textwidth}{355pt}
%\addtolength\marginparwidth{30pt}
%\addtolength\oddsidemargin{20pt}
%\addtolength\evensidemargin{20pt}
\def\cmd#1{\cs{\expandafter\cmd@to@cs\string#1}}
\def\cmd@to@cs#1#2{\char\number`#2\relax}
\DeclareRobustCommand\cs[1]{\texttt{\char`\\#1}}
\providecommand\marg[1]{%
  {\ttfamily\char`\{}\meta{#1}{\ttfamily\char`\}}}
\providecommand\oarg[1]{%
  {\ttfamily[}\meta{#1}{\ttfamily]}}
\providecommand\parg[1]{%
  {\ttfamily(}\meta{#1}{\ttfamily)}}
\raggedbottom

% -----------------------------------
\input{pgfplots.preamble.tex}

\makeatletter
% I want a two-column index just as in pgfmanual styles. This here
% was the best way to get one:
\def\index@prologue{\section*{Index}\addcontentsline{toc}{chapter}{Index}
}
\makeatother

%\RequirePackage[german,english,francais]{babel}

\def\matlabcolormaptext{This colormap is similar to one shipped with Matlab$^\text{\textregistered}$ under a similar name.}

\IfFileExists{tikzlibraryspy.code.tex}{%
\usetikzlibrary{spy}
}{%
    \message{ERROR: tikz SPY library NOT available. The manual will only compile partially.^^J}%
}%

\usepackage{xparse}% for colorbrewer manual

\usetikzlibrary{
    decorations.markings,
    decorations.footprints,
    shapes.arrows,
    matrix,
    positioning,
}

\usepgfplotslibrary{
    fillbetween,
    ternary,
    smithchart,
    patchplots,
    polar,
    colormaps,
    colorbrewer,
    colortol,           % docu not ready yet
}
\pgfqkeys{/codeexample}{%
    every codeexample/.append style={
        /pgfplots/every ternary axis/.append style={
            /pgfplots/legend style={fill=graphicbackground},
        }
    },
    tabsize=4,
}

\pgfplotsmanualenableexternalizationofexpensive

%\usetikzlibrary{external}
%\tikzexternalize[prefix=figures/]{pgfplots}

\title{%
    Manual for Package \PGFPlots{}\\
    {\small 2D/3D Plots in \LaTeX{}, Version \pgfplotsversion}\\
    {\small\url{https://github.com/pgf-tikz/pgfplots}}
    %\\{\small Attention: you are using an unstable development version.}
}

\makeatletter
\long\def\abstractsmuggle{%
    \centering
    \textbf{Abstract}\\[0.5cm]

    \begin{minipage}{12cm}
        \PGFPlots{} draws high-quality function plots in normal or logarithmic
        scaling with a user-friendly interface directly in \TeX{}. The user
        supplies axis labels, legend entries and the plot coordinates for one or
        more plots and \PGFPlots{} applies axis scaling, computes any logarithms
        and axis ticks and draws the plots. It supports line plots, scatter
        plots, piecewise constant plots, bar plots, area plots, mesh and surface
        plots, patch plots, contour plots, quiver plots, histogram plots, box
        plots, polar axes, ternary diagrams, smith charts and some more. It is
        based on Till Tantau's package \PGF{}/\Tikz{}.
    \end{minipage}

}%

\expandafter\date\expandafter{\@date\\[2cm]
    \abstractsmuggle
}%
\makeatother

%\includeonly{pgfplots.reference}


\begin{document}

\def\plotcoords{%
\addplot coordinates {
(5,8.312e-02)    (17,2.547e-02)   (49,7.407e-03)
(129,2.102e-03)  (321,5.874e-04)  (769,1.623e-04)
(1793,4.442e-05) (4097,1.207e-05) (9217,3.261e-06)
};

\addplot coordinates{
(7,8.472e-02)    (31,3.044e-02)    (111,1.022e-02)
(351,3.303e-03)  (1023,1.039e-03)  (2815,3.196e-04)
(7423,9.658e-05) (18943,2.873e-05) (47103,8.437e-06)
};

\addplot coordinates{
(9,7.881e-02)     (49,3.243e-02)    (209,1.232e-02)
(769,4.454e-03)   (2561,1.551e-03)  (7937,5.236e-04)
(23297,1.723e-04) (65537,5.545e-05) (178177,1.751e-05)
};

\addplot coordinates{
(11,6.887e-02)    (71,3.177e-02)     (351,1.341e-02)
(1471,5.334e-03)  (5503,2.027e-03)   (18943,7.415e-04)
(61183,2.628e-04) (187903,9.063e-05) (553983,3.053e-05)
};

\addplot coordinates{
(13,5.755e-02)     (97,2.925e-02)     (545,1.351e-02)
(2561,5.842e-03)   (10625,2.397e-03)  (40193,9.414e-04)
(141569,3.564e-04) (471041,1.308e-04)
(1496065,4.670e-05)
};
}%


\bookmaketitle
\tableofcontents
\include{pgfplots.title_abstract_intro}
\include{pgfplots.preliminaries}
\include{pgfplots.intro}
\include{pgfplots.reference}
\include{pgfplots.libs}
\include{pgfplots.resources}
\include{pgfplots.importexport}
\include{pgfplots.basic.reference}

\printindex

\bibliographystyle{abbrv} %gerapali} %gerabbrv} %gerunsrt.bst} %gerabbrv}% gerplain}
\nocite{pgfplotstable}
\nocite{programmingnotes}
\bibliography{pgfplots}
\end{document}

% -----------------------------------------------------------------------------
% For Stefan Pinnow as reminder on what to look for when editing the manual
% -----------------------------------------------------------------------------
% There should be no line breaks in the following environments
% - |...|
% - \declareandlabel{...}
% - \verbpdfref{...}
% If "MakeTikzPictures" isn't running through check one of the externalized
% LOG files what is the cause of that.
% -----------------------------------------------------------------------------


%%%%%%%%%%%%%%%%%%%%%%%%%%%%%%%%%%%%%%%%%%%%%%%%%%%%%%%%%%%%%%%%%%%%%%%%%%%%%
%
% Package pgfplots.sty documentation.
%
% Copyright 2007/2008 by Christian Feuersaenger.
%
% This program is free software: you can redistribute it and/or modify
% it under the terms of the GNU General Public License as published by
% the Free Software Foundation, either version 3 of the License, or
% (at your option) any later version.
%
% This program is distributed in the hope that it will be useful,
% but WITHOUT ANY WARRANTY; without even the implied warranty of
% MERCHANTABILITY or FITNESS FOR A PARTICULAR PURPOSE.  See the
% GNU General Public License for more details.
%
% You should have received a copy of the GNU General Public License
% along with this program.  If not, see <http://www.gnu.org/licenses/>.
%
%
%%%%%%%%%%%%%%%%%%%%%%%%%%%%%%%%%%%%%%%%%%%%%%%%%%%%%%%%%%%%%%%%%%%%%%%%%%%%%
% SEE pgfplots-macros.tex as well!
%\pdfminorversion=5 % to allow compression
%\pdfobjcompresslevel=2
\documentclass[a4paper,openany]{book}

\let\bookmaketitle=\maketitle

% -----------------------------------
% this here is from ltxdoc:
\usepackage{doc}[=v2]
\AtBeginDocument{\MakeShortVerb{\|}}
%\setlength{\textwidth}{355pt}
%\addtolength\marginparwidth{30pt}
%\addtolength\oddsidemargin{20pt}
%\addtolength\evensidemargin{20pt}
\def\cmd#1{\cs{\expandafter\cmd@to@cs\string#1}}
\def\cmd@to@cs#1#2{\char\number`#2\relax}
\DeclareRobustCommand\cs[1]{\texttt{\char`\\#1}}
\providecommand\marg[1]{%
  {\ttfamily\char`\{}\meta{#1}{\ttfamily\char`\}}}
\providecommand\oarg[1]{%
  {\ttfamily[}\meta{#1}{\ttfamily]}}
\providecommand\parg[1]{%
  {\ttfamily(}\meta{#1}{\ttfamily)}}
\raggedbottom

% -----------------------------------
\input{pgfplots.preamble.tex}

\makeatletter
% I want a two-column index just as in pgfmanual styles. This here
% was the best way to get one:
\def\index@prologue{\section*{Index}\addcontentsline{toc}{chapter}{Index}
}
\makeatother

%\RequirePackage[german,english,francais]{babel}

\def\matlabcolormaptext{This colormap is similar to one shipped with Matlab$^\text{\textregistered}$ under a similar name.}

\IfFileExists{tikzlibraryspy.code.tex}{%
\usetikzlibrary{spy}
}{%
    \message{ERROR: tikz SPY library NOT available. The manual will only compile partially.^^J}%
}%

\usepackage{xparse}% for colorbrewer manual

\usetikzlibrary{
    decorations.markings,
    decorations.footprints,
    shapes.arrows,
    matrix,
    positioning,
}

\usepgfplotslibrary{
    fillbetween,
    ternary,
    smithchart,
    patchplots,
    polar,
    colormaps,
    colorbrewer,
    colortol,           % docu not ready yet
}
\pgfqkeys{/codeexample}{%
    every codeexample/.append style={
        /pgfplots/every ternary axis/.append style={
            /pgfplots/legend style={fill=graphicbackground},
        }
    },
    tabsize=4,
}

\pgfplotsmanualenableexternalizationofexpensive

%\usetikzlibrary{external}
%\tikzexternalize[prefix=figures/]{pgfplots}

\title{%
    Manual for Package \PGFPlots{}\\
    {\small 2D/3D Plots in \LaTeX{}, Version \pgfplotsversion}\\
    {\small\url{https://github.com/pgf-tikz/pgfplots}}
    %\\{\small Attention: you are using an unstable development version.}
}

\makeatletter
\long\def\abstractsmuggle{%
    \centering
    \textbf{Abstract}\\[0.5cm]

    \begin{minipage}{12cm}
        \PGFPlots{} draws high-quality function plots in normal or logarithmic
        scaling with a user-friendly interface directly in \TeX{}. The user
        supplies axis labels, legend entries and the plot coordinates for one or
        more plots and \PGFPlots{} applies axis scaling, computes any logarithms
        and axis ticks and draws the plots. It supports line plots, scatter
        plots, piecewise constant plots, bar plots, area plots, mesh and surface
        plots, patch plots, contour plots, quiver plots, histogram plots, box
        plots, polar axes, ternary diagrams, smith charts and some more. It is
        based on Till Tantau's package \PGF{}/\Tikz{}.
    \end{minipage}

}%

\expandafter\date\expandafter{\@date\\[2cm]
    \abstractsmuggle
}%
\makeatother

%\includeonly{pgfplots.reference}


\begin{document}

\def\plotcoords{%
\addplot coordinates {
(5,8.312e-02)    (17,2.547e-02)   (49,7.407e-03)
(129,2.102e-03)  (321,5.874e-04)  (769,1.623e-04)
(1793,4.442e-05) (4097,1.207e-05) (9217,3.261e-06)
};

\addplot coordinates{
(7,8.472e-02)    (31,3.044e-02)    (111,1.022e-02)
(351,3.303e-03)  (1023,1.039e-03)  (2815,3.196e-04)
(7423,9.658e-05) (18943,2.873e-05) (47103,8.437e-06)
};

\addplot coordinates{
(9,7.881e-02)     (49,3.243e-02)    (209,1.232e-02)
(769,4.454e-03)   (2561,1.551e-03)  (7937,5.236e-04)
(23297,1.723e-04) (65537,5.545e-05) (178177,1.751e-05)
};

\addplot coordinates{
(11,6.887e-02)    (71,3.177e-02)     (351,1.341e-02)
(1471,5.334e-03)  (5503,2.027e-03)   (18943,7.415e-04)
(61183,2.628e-04) (187903,9.063e-05) (553983,3.053e-05)
};

\addplot coordinates{
(13,5.755e-02)     (97,2.925e-02)     (545,1.351e-02)
(2561,5.842e-03)   (10625,2.397e-03)  (40193,9.414e-04)
(141569,3.564e-04) (471041,1.308e-04)
(1496065,4.670e-05)
};
}%


\bookmaketitle
\tableofcontents
\include{pgfplots.title_abstract_intro}
\include{pgfplots.preliminaries}
\include{pgfplots.intro}
\include{pgfplots.reference}
\include{pgfplots.libs}
\include{pgfplots.resources}
\include{pgfplots.importexport}
\include{pgfplots.basic.reference}

\printindex

\bibliographystyle{abbrv} %gerapali} %gerabbrv} %gerunsrt.bst} %gerabbrv}% gerplain}
\nocite{pgfplotstable}
\nocite{programmingnotes}
\bibliography{pgfplots}
\end{document}

% -----------------------------------------------------------------------------
% For Stefan Pinnow as reminder on what to look for when editing the manual
% -----------------------------------------------------------------------------
% There should be no line breaks in the following environments
% - |...|
% - \declareandlabel{...}
% - \verbpdfref{...}
% If "MakeTikzPictures" isn't running through check one of the externalized
% LOG files what is the cause of that.
% -----------------------------------------------------------------------------


%%%%%%%%%%%%%%%%%%%%%%%%%%%%%%%%%%%%%%%%%%%%%%%%%%%%%%%%%%%%%%%%%%%%%%%%%%%%%
%
% Package pgfplots.sty documentation.
%
% Copyright 2007/2008 by Christian Feuersaenger.
%
% This program is free software: you can redistribute it and/or modify
% it under the terms of the GNU General Public License as published by
% the Free Software Foundation, either version 3 of the License, or
% (at your option) any later version.
%
% This program is distributed in the hope that it will be useful,
% but WITHOUT ANY WARRANTY; without even the implied warranty of
% MERCHANTABILITY or FITNESS FOR A PARTICULAR PURPOSE.  See the
% GNU General Public License for more details.
%
% You should have received a copy of the GNU General Public License
% along with this program.  If not, see <http://www.gnu.org/licenses/>.
%
%
%%%%%%%%%%%%%%%%%%%%%%%%%%%%%%%%%%%%%%%%%%%%%%%%%%%%%%%%%%%%%%%%%%%%%%%%%%%%%
% SEE pgfplots-macros.tex as well!
%\pdfminorversion=5 % to allow compression
%\pdfobjcompresslevel=2
\documentclass[a4paper,openany]{book}

\let\bookmaketitle=\maketitle

% -----------------------------------
% this here is from ltxdoc:
\usepackage{doc}[=v2]
\AtBeginDocument{\MakeShortVerb{\|}}
%\setlength{\textwidth}{355pt}
%\addtolength\marginparwidth{30pt}
%\addtolength\oddsidemargin{20pt}
%\addtolength\evensidemargin{20pt}
\def\cmd#1{\cs{\expandafter\cmd@to@cs\string#1}}
\def\cmd@to@cs#1#2{\char\number`#2\relax}
\DeclareRobustCommand\cs[1]{\texttt{\char`\\#1}}
\providecommand\marg[1]{%
  {\ttfamily\char`\{}\meta{#1}{\ttfamily\char`\}}}
\providecommand\oarg[1]{%
  {\ttfamily[}\meta{#1}{\ttfamily]}}
\providecommand\parg[1]{%
  {\ttfamily(}\meta{#1}{\ttfamily)}}
\raggedbottom

% -----------------------------------
\input{pgfplots.preamble.tex}

\makeatletter
% I want a two-column index just as in pgfmanual styles. This here
% was the best way to get one:
\def\index@prologue{\section*{Index}\addcontentsline{toc}{chapter}{Index}
}
\makeatother

%\RequirePackage[german,english,francais]{babel}

\def\matlabcolormaptext{This colormap is similar to one shipped with Matlab$^\text{\textregistered}$ under a similar name.}

\IfFileExists{tikzlibraryspy.code.tex}{%
\usetikzlibrary{spy}
}{%
    \message{ERROR: tikz SPY library NOT available. The manual will only compile partially.^^J}%
}%

\usepackage{xparse}% for colorbrewer manual

\usetikzlibrary{
    decorations.markings,
    decorations.footprints,
    shapes.arrows,
    matrix,
    positioning,
}

\usepgfplotslibrary{
    fillbetween,
    ternary,
    smithchart,
    patchplots,
    polar,
    colormaps,
    colorbrewer,
    colortol,           % docu not ready yet
}
\pgfqkeys{/codeexample}{%
    every codeexample/.append style={
        /pgfplots/every ternary axis/.append style={
            /pgfplots/legend style={fill=graphicbackground},
        }
    },
    tabsize=4,
}

\pgfplotsmanualenableexternalizationofexpensive

%\usetikzlibrary{external}
%\tikzexternalize[prefix=figures/]{pgfplots}

\title{%
    Manual for Package \PGFPlots{}\\
    {\small 2D/3D Plots in \LaTeX{}, Version \pgfplotsversion}\\
    {\small\url{https://github.com/pgf-tikz/pgfplots}}
    %\\{\small Attention: you are using an unstable development version.}
}

\makeatletter
\long\def\abstractsmuggle{%
    \centering
    \textbf{Abstract}\\[0.5cm]

    \begin{minipage}{12cm}
        \PGFPlots{} draws high-quality function plots in normal or logarithmic
        scaling with a user-friendly interface directly in \TeX{}. The user
        supplies axis labels, legend entries and the plot coordinates for one or
        more plots and \PGFPlots{} applies axis scaling, computes any logarithms
        and axis ticks and draws the plots. It supports line plots, scatter
        plots, piecewise constant plots, bar plots, area plots, mesh and surface
        plots, patch plots, contour plots, quiver plots, histogram plots, box
        plots, polar axes, ternary diagrams, smith charts and some more. It is
        based on Till Tantau's package \PGF{}/\Tikz{}.
    \end{minipage}

}%

\expandafter\date\expandafter{\@date\\[2cm]
    \abstractsmuggle
}%
\makeatother

%\includeonly{pgfplots.reference}


\begin{document}

\def\plotcoords{%
\addplot coordinates {
(5,8.312e-02)    (17,2.547e-02)   (49,7.407e-03)
(129,2.102e-03)  (321,5.874e-04)  (769,1.623e-04)
(1793,4.442e-05) (4097,1.207e-05) (9217,3.261e-06)
};

\addplot coordinates{
(7,8.472e-02)    (31,3.044e-02)    (111,1.022e-02)
(351,3.303e-03)  (1023,1.039e-03)  (2815,3.196e-04)
(7423,9.658e-05) (18943,2.873e-05) (47103,8.437e-06)
};

\addplot coordinates{
(9,7.881e-02)     (49,3.243e-02)    (209,1.232e-02)
(769,4.454e-03)   (2561,1.551e-03)  (7937,5.236e-04)
(23297,1.723e-04) (65537,5.545e-05) (178177,1.751e-05)
};

\addplot coordinates{
(11,6.887e-02)    (71,3.177e-02)     (351,1.341e-02)
(1471,5.334e-03)  (5503,2.027e-03)   (18943,7.415e-04)
(61183,2.628e-04) (187903,9.063e-05) (553983,3.053e-05)
};

\addplot coordinates{
(13,5.755e-02)     (97,2.925e-02)     (545,1.351e-02)
(2561,5.842e-03)   (10625,2.397e-03)  (40193,9.414e-04)
(141569,3.564e-04) (471041,1.308e-04)
(1496065,4.670e-05)
};
}%


\bookmaketitle
\tableofcontents
\include{pgfplots.title_abstract_intro}
\include{pgfplots.preliminaries}
\include{pgfplots.intro}
\include{pgfplots.reference}
\include{pgfplots.libs}
\include{pgfplots.resources}
\include{pgfplots.importexport}
\include{pgfplots.basic.reference}

\printindex

\bibliographystyle{abbrv} %gerapali} %gerabbrv} %gerunsrt.bst} %gerabbrv}% gerplain}
\nocite{pgfplotstable}
\nocite{programmingnotes}
\bibliography{pgfplots}
\end{document}

% -----------------------------------------------------------------------------
% For Stefan Pinnow as reminder on what to look for when editing the manual
% -----------------------------------------------------------------------------
% There should be no line breaks in the following environments
% - |...|
% - \declareandlabel{...}
% - \verbpdfref{...}
% If "MakeTikzPictures" isn't running through check one of the externalized
% LOG files what is the cause of that.
% -----------------------------------------------------------------------------


%%%%%%%%%%%%%%%%%%%%%%%%%%%%%%%%%%%%%%%%%%%%%%%%%%%%%%%%%%%%%%%%%%%%%%%%%%%%%
%
% Package pgfplots.sty documentation.
%
% Copyright 2007/2008 by Christian Feuersaenger.
%
% This program is free software: you can redistribute it and/or modify
% it under the terms of the GNU General Public License as published by
% the Free Software Foundation, either version 3 of the License, or
% (at your option) any later version.
%
% This program is distributed in the hope that it will be useful,
% but WITHOUT ANY WARRANTY; without even the implied warranty of
% MERCHANTABILITY or FITNESS FOR A PARTICULAR PURPOSE.  See the
% GNU General Public License for more details.
%
% You should have received a copy of the GNU General Public License
% along with this program.  If not, see <http://www.gnu.org/licenses/>.
%
%
%%%%%%%%%%%%%%%%%%%%%%%%%%%%%%%%%%%%%%%%%%%%%%%%%%%%%%%%%%%%%%%%%%%%%%%%%%%%%
% SEE pgfplots-macros.tex as well!
%\pdfminorversion=5 % to allow compression
%\pdfobjcompresslevel=2
\documentclass[a4paper,openany]{book}

\let\bookmaketitle=\maketitle

% -----------------------------------
% this here is from ltxdoc:
\usepackage{doc}[=v2]
\AtBeginDocument{\MakeShortVerb{\|}}
%\setlength{\textwidth}{355pt}
%\addtolength\marginparwidth{30pt}
%\addtolength\oddsidemargin{20pt}
%\addtolength\evensidemargin{20pt}
\def\cmd#1{\cs{\expandafter\cmd@to@cs\string#1}}
\def\cmd@to@cs#1#2{\char\number`#2\relax}
\DeclareRobustCommand\cs[1]{\texttt{\char`\\#1}}
\providecommand\marg[1]{%
  {\ttfamily\char`\{}\meta{#1}{\ttfamily\char`\}}}
\providecommand\oarg[1]{%
  {\ttfamily[}\meta{#1}{\ttfamily]}}
\providecommand\parg[1]{%
  {\ttfamily(}\meta{#1}{\ttfamily)}}
\raggedbottom

% -----------------------------------
\input{pgfplots.preamble.tex}

\makeatletter
% I want a two-column index just as in pgfmanual styles. This here
% was the best way to get one:
\def\index@prologue{\section*{Index}\addcontentsline{toc}{chapter}{Index}
}
\makeatother

%\RequirePackage[german,english,francais]{babel}

\def\matlabcolormaptext{This colormap is similar to one shipped with Matlab$^\text{\textregistered}$ under a similar name.}

\IfFileExists{tikzlibraryspy.code.tex}{%
\usetikzlibrary{spy}
}{%
    \message{ERROR: tikz SPY library NOT available. The manual will only compile partially.^^J}%
}%

\usepackage{xparse}% for colorbrewer manual

\usetikzlibrary{
    decorations.markings,
    decorations.footprints,
    shapes.arrows,
    matrix,
    positioning,
}

\usepgfplotslibrary{
    fillbetween,
    ternary,
    smithchart,
    patchplots,
    polar,
    colormaps,
    colorbrewer,
    colortol,           % docu not ready yet
}
\pgfqkeys{/codeexample}{%
    every codeexample/.append style={
        /pgfplots/every ternary axis/.append style={
            /pgfplots/legend style={fill=graphicbackground},
        }
    },
    tabsize=4,
}

\pgfplotsmanualenableexternalizationofexpensive

%\usetikzlibrary{external}
%\tikzexternalize[prefix=figures/]{pgfplots}

\title{%
    Manual for Package \PGFPlots{}\\
    {\small 2D/3D Plots in \LaTeX{}, Version \pgfplotsversion}\\
    {\small\url{https://github.com/pgf-tikz/pgfplots}}
    %\\{\small Attention: you are using an unstable development version.}
}

\makeatletter
\long\def\abstractsmuggle{%
    \centering
    \textbf{Abstract}\\[0.5cm]

    \begin{minipage}{12cm}
        \PGFPlots{} draws high-quality function plots in normal or logarithmic
        scaling with a user-friendly interface directly in \TeX{}. The user
        supplies axis labels, legend entries and the plot coordinates for one or
        more plots and \PGFPlots{} applies axis scaling, computes any logarithms
        and axis ticks and draws the plots. It supports line plots, scatter
        plots, piecewise constant plots, bar plots, area plots, mesh and surface
        plots, patch plots, contour plots, quiver plots, histogram plots, box
        plots, polar axes, ternary diagrams, smith charts and some more. It is
        based on Till Tantau's package \PGF{}/\Tikz{}.
    \end{minipage}

}%

\expandafter\date\expandafter{\@date\\[2cm]
    \abstractsmuggle
}%
\makeatother

%\includeonly{pgfplots.reference}


\begin{document}

\def\plotcoords{%
\addplot coordinates {
(5,8.312e-02)    (17,2.547e-02)   (49,7.407e-03)
(129,2.102e-03)  (321,5.874e-04)  (769,1.623e-04)
(1793,4.442e-05) (4097,1.207e-05) (9217,3.261e-06)
};

\addplot coordinates{
(7,8.472e-02)    (31,3.044e-02)    (111,1.022e-02)
(351,3.303e-03)  (1023,1.039e-03)  (2815,3.196e-04)
(7423,9.658e-05) (18943,2.873e-05) (47103,8.437e-06)
};

\addplot coordinates{
(9,7.881e-02)     (49,3.243e-02)    (209,1.232e-02)
(769,4.454e-03)   (2561,1.551e-03)  (7937,5.236e-04)
(23297,1.723e-04) (65537,5.545e-05) (178177,1.751e-05)
};

\addplot coordinates{
(11,6.887e-02)    (71,3.177e-02)     (351,1.341e-02)
(1471,5.334e-03)  (5503,2.027e-03)   (18943,7.415e-04)
(61183,2.628e-04) (187903,9.063e-05) (553983,3.053e-05)
};

\addplot coordinates{
(13,5.755e-02)     (97,2.925e-02)     (545,1.351e-02)
(2561,5.842e-03)   (10625,2.397e-03)  (40193,9.414e-04)
(141569,3.564e-04) (471041,1.308e-04)
(1496065,4.670e-05)
};
}%


\bookmaketitle
\tableofcontents
\include{pgfplots.title_abstract_intro}
\include{pgfplots.preliminaries}
\include{pgfplots.intro}
\include{pgfplots.reference}
\include{pgfplots.libs}
\include{pgfplots.resources}
\include{pgfplots.importexport}
\include{pgfplots.basic.reference}

\printindex

\bibliographystyle{abbrv} %gerapali} %gerabbrv} %gerunsrt.bst} %gerabbrv}% gerplain}
\nocite{pgfplotstable}
\nocite{programmingnotes}
\bibliography{pgfplots}
\end{document}

% -----------------------------------------------------------------------------
% For Stefan Pinnow as reminder on what to look for when editing the manual
% -----------------------------------------------------------------------------
% There should be no line breaks in the following environments
% - |...|
% - \declareandlabel{...}
% - \verbpdfref{...}
% If "MakeTikzPictures" isn't running through check one of the externalized
% LOG files what is the cause of that.
% -----------------------------------------------------------------------------

\include{pgfplots.basic.reference}

\printindex

\bibliographystyle{abbrv} %gerapali} %gerabbrv} %gerunsrt.bst} %gerabbrv}% gerplain}
\nocite{pgfplotstable}
\nocite{programmingnotes}
\bibliography{pgfplots}
\end{document}

% -----------------------------------------------------------------------------
% For Stefan Pinnow as reminder on what to look for when editing the manual
% -----------------------------------------------------------------------------
% There should be no line breaks in the following environments
% - |...|
% - \declareandlabel{...}
% - \verbpdfref{...}
% If "MakeTikzPictures" isn't running through check one of the externalized
% LOG files what is the cause of that.
% -----------------------------------------------------------------------------


%%%%%%%%%%%%%%%%%%%%%%%%%%%%%%%%%%%%%%%%%%%%%%%%%%%%%%%%%%%%%%%%%%%%%%%%%%%%%
%
% Package pgfplots.sty documentation.
%
% Copyright 2007/2008 by Christian Feuersaenger.
%
% This program is free software: you can redistribute it and/or modify
% it under the terms of the GNU General Public License as published by
% the Free Software Foundation, either version 3 of the License, or
% (at your option) any later version.
%
% This program is distributed in the hope that it will be useful,
% but WITHOUT ANY WARRANTY; without even the implied warranty of
% MERCHANTABILITY or FITNESS FOR A PARTICULAR PURPOSE.  See the
% GNU General Public License for more details.
%
% You should have received a copy of the GNU General Public License
% along with this program.  If not, see <http://www.gnu.org/licenses/>.
%
%
%%%%%%%%%%%%%%%%%%%%%%%%%%%%%%%%%%%%%%%%%%%%%%%%%%%%%%%%%%%%%%%%%%%%%%%%%%%%%
% SEE pgfplots-macros.tex as well!
%\pdfminorversion=5 % to allow compression
%\pdfobjcompresslevel=2
\documentclass[a4paper,openany]{book}

\let\bookmaketitle=\maketitle

% -----------------------------------
% this here is from ltxdoc:
\usepackage{doc}[=v2]
\AtBeginDocument{\MakeShortVerb{\|}}
%\setlength{\textwidth}{355pt}
%\addtolength\marginparwidth{30pt}
%\addtolength\oddsidemargin{20pt}
%\addtolength\evensidemargin{20pt}
\def\cmd#1{\cs{\expandafter\cmd@to@cs\string#1}}
\def\cmd@to@cs#1#2{\char\number`#2\relax}
\DeclareRobustCommand\cs[1]{\texttt{\char`\\#1}}
\providecommand\marg[1]{%
  {\ttfamily\char`\{}\meta{#1}{\ttfamily\char`\}}}
\providecommand\oarg[1]{%
  {\ttfamily[}\meta{#1}{\ttfamily]}}
\providecommand\parg[1]{%
  {\ttfamily(}\meta{#1}{\ttfamily)}}
\raggedbottom

% -----------------------------------
\input{pgfplots.preamble.tex}

\makeatletter
% I want a two-column index just as in pgfmanual styles. This here
% was the best way to get one:
\def\index@prologue{\section*{Index}\addcontentsline{toc}{chapter}{Index}
}
\makeatother

%\RequirePackage[german,english,francais]{babel}

\def\matlabcolormaptext{This colormap is similar to one shipped with Matlab$^\text{\textregistered}$ under a similar name.}

\IfFileExists{tikzlibraryspy.code.tex}{%
\usetikzlibrary{spy}
}{%
    \message{ERROR: tikz SPY library NOT available. The manual will only compile partially.^^J}%
}%

\usepackage{xparse}% for colorbrewer manual

\usetikzlibrary{
    decorations.markings,
    decorations.footprints,
    shapes.arrows,
    matrix,
    positioning,
}

\usepgfplotslibrary{
    fillbetween,
    ternary,
    smithchart,
    patchplots,
    polar,
    colormaps,
    colorbrewer,
    colortol,           % docu not ready yet
}
\pgfqkeys{/codeexample}{%
    every codeexample/.append style={
        /pgfplots/every ternary axis/.append style={
            /pgfplots/legend style={fill=graphicbackground},
        }
    },
    tabsize=4,
}

\pgfplotsmanualenableexternalizationofexpensive

%\usetikzlibrary{external}
%\tikzexternalize[prefix=figures/]{pgfplots}

\title{%
    Manual for Package \PGFPlots{}\\
    {\small 2D/3D Plots in \LaTeX{}, Version \pgfplotsversion}\\
    {\small\url{https://github.com/pgf-tikz/pgfplots}}
    %\\{\small Attention: you are using an unstable development version.}
}

\makeatletter
\long\def\abstractsmuggle{%
    \centering
    \textbf{Abstract}\\[0.5cm]

    \begin{minipage}{12cm}
        \PGFPlots{} draws high-quality function plots in normal or logarithmic
        scaling with a user-friendly interface directly in \TeX{}. The user
        supplies axis labels, legend entries and the plot coordinates for one or
        more plots and \PGFPlots{} applies axis scaling, computes any logarithms
        and axis ticks and draws the plots. It supports line plots, scatter
        plots, piecewise constant plots, bar plots, area plots, mesh and surface
        plots, patch plots, contour plots, quiver plots, histogram plots, box
        plots, polar axes, ternary diagrams, smith charts and some more. It is
        based on Till Tantau's package \PGF{}/\Tikz{}.
    \end{minipage}

}%

\expandafter\date\expandafter{\@date\\[2cm]
    \abstractsmuggle
}%
\makeatother

%\includeonly{pgfplots.reference}


\begin{document}

\def\plotcoords{%
\addplot coordinates {
(5,8.312e-02)    (17,2.547e-02)   (49,7.407e-03)
(129,2.102e-03)  (321,5.874e-04)  (769,1.623e-04)
(1793,4.442e-05) (4097,1.207e-05) (9217,3.261e-06)
};

\addplot coordinates{
(7,8.472e-02)    (31,3.044e-02)    (111,1.022e-02)
(351,3.303e-03)  (1023,1.039e-03)  (2815,3.196e-04)
(7423,9.658e-05) (18943,2.873e-05) (47103,8.437e-06)
};

\addplot coordinates{
(9,7.881e-02)     (49,3.243e-02)    (209,1.232e-02)
(769,4.454e-03)   (2561,1.551e-03)  (7937,5.236e-04)
(23297,1.723e-04) (65537,5.545e-05) (178177,1.751e-05)
};

\addplot coordinates{
(11,6.887e-02)    (71,3.177e-02)     (351,1.341e-02)
(1471,5.334e-03)  (5503,2.027e-03)   (18943,7.415e-04)
(61183,2.628e-04) (187903,9.063e-05) (553983,3.053e-05)
};

\addplot coordinates{
(13,5.755e-02)     (97,2.925e-02)     (545,1.351e-02)
(2561,5.842e-03)   (10625,2.397e-03)  (40193,9.414e-04)
(141569,3.564e-04) (471041,1.308e-04)
(1496065,4.670e-05)
};
}%


\bookmaketitle
\tableofcontents

%%%%%%%%%%%%%%%%%%%%%%%%%%%%%%%%%%%%%%%%%%%%%%%%%%%%%%%%%%%%%%%%%%%%%%%%%%%%%
%
% Package pgfplots.sty documentation.
%
% Copyright 2007/2008 by Christian Feuersaenger.
%
% This program is free software: you can redistribute it and/or modify
% it under the terms of the GNU General Public License as published by
% the Free Software Foundation, either version 3 of the License, or
% (at your option) any later version.
%
% This program is distributed in the hope that it will be useful,
% but WITHOUT ANY WARRANTY; without even the implied warranty of
% MERCHANTABILITY or FITNESS FOR A PARTICULAR PURPOSE.  See the
% GNU General Public License for more details.
%
% You should have received a copy of the GNU General Public License
% along with this program.  If not, see <http://www.gnu.org/licenses/>.
%
%
%%%%%%%%%%%%%%%%%%%%%%%%%%%%%%%%%%%%%%%%%%%%%%%%%%%%%%%%%%%%%%%%%%%%%%%%%%%%%
% SEE pgfplots-macros.tex as well!
%\pdfminorversion=5 % to allow compression
%\pdfobjcompresslevel=2
\documentclass[a4paper,openany]{book}

\let\bookmaketitle=\maketitle

% -----------------------------------
% this here is from ltxdoc:
\usepackage{doc}[=v2]
\AtBeginDocument{\MakeShortVerb{\|}}
%\setlength{\textwidth}{355pt}
%\addtolength\marginparwidth{30pt}
%\addtolength\oddsidemargin{20pt}
%\addtolength\evensidemargin{20pt}
\def\cmd#1{\cs{\expandafter\cmd@to@cs\string#1}}
\def\cmd@to@cs#1#2{\char\number`#2\relax}
\DeclareRobustCommand\cs[1]{\texttt{\char`\\#1}}
\providecommand\marg[1]{%
  {\ttfamily\char`\{}\meta{#1}{\ttfamily\char`\}}}
\providecommand\oarg[1]{%
  {\ttfamily[}\meta{#1}{\ttfamily]}}
\providecommand\parg[1]{%
  {\ttfamily(}\meta{#1}{\ttfamily)}}
\raggedbottom

% -----------------------------------
\input{pgfplots.preamble.tex}

\makeatletter
% I want a two-column index just as in pgfmanual styles. This here
% was the best way to get one:
\def\index@prologue{\section*{Index}\addcontentsline{toc}{chapter}{Index}
}
\makeatother

%\RequirePackage[german,english,francais]{babel}

\def\matlabcolormaptext{This colormap is similar to one shipped with Matlab$^\text{\textregistered}$ under a similar name.}

\IfFileExists{tikzlibraryspy.code.tex}{%
\usetikzlibrary{spy}
}{%
    \message{ERROR: tikz SPY library NOT available. The manual will only compile partially.^^J}%
}%

\usepackage{xparse}% for colorbrewer manual

\usetikzlibrary{
    decorations.markings,
    decorations.footprints,
    shapes.arrows,
    matrix,
    positioning,
}

\usepgfplotslibrary{
    fillbetween,
    ternary,
    smithchart,
    patchplots,
    polar,
    colormaps,
    colorbrewer,
    colortol,           % docu not ready yet
}
\pgfqkeys{/codeexample}{%
    every codeexample/.append style={
        /pgfplots/every ternary axis/.append style={
            /pgfplots/legend style={fill=graphicbackground},
        }
    },
    tabsize=4,
}

\pgfplotsmanualenableexternalizationofexpensive

%\usetikzlibrary{external}
%\tikzexternalize[prefix=figures/]{pgfplots}

\title{%
    Manual for Package \PGFPlots{}\\
    {\small 2D/3D Plots in \LaTeX{}, Version \pgfplotsversion}\\
    {\small\url{https://github.com/pgf-tikz/pgfplots}}
    %\\{\small Attention: you are using an unstable development version.}
}

\makeatletter
\long\def\abstractsmuggle{%
    \centering
    \textbf{Abstract}\\[0.5cm]

    \begin{minipage}{12cm}
        \PGFPlots{} draws high-quality function plots in normal or logarithmic
        scaling with a user-friendly interface directly in \TeX{}. The user
        supplies axis labels, legend entries and the plot coordinates for one or
        more plots and \PGFPlots{} applies axis scaling, computes any logarithms
        and axis ticks and draws the plots. It supports line plots, scatter
        plots, piecewise constant plots, bar plots, area plots, mesh and surface
        plots, patch plots, contour plots, quiver plots, histogram plots, box
        plots, polar axes, ternary diagrams, smith charts and some more. It is
        based on Till Tantau's package \PGF{}/\Tikz{}.
    \end{minipage}

}%

\expandafter\date\expandafter{\@date\\[2cm]
    \abstractsmuggle
}%
\makeatother

%\includeonly{pgfplots.reference}


\begin{document}

\def\plotcoords{%
\addplot coordinates {
(5,8.312e-02)    (17,2.547e-02)   (49,7.407e-03)
(129,2.102e-03)  (321,5.874e-04)  (769,1.623e-04)
(1793,4.442e-05) (4097,1.207e-05) (9217,3.261e-06)
};

\addplot coordinates{
(7,8.472e-02)    (31,3.044e-02)    (111,1.022e-02)
(351,3.303e-03)  (1023,1.039e-03)  (2815,3.196e-04)
(7423,9.658e-05) (18943,2.873e-05) (47103,8.437e-06)
};

\addplot coordinates{
(9,7.881e-02)     (49,3.243e-02)    (209,1.232e-02)
(769,4.454e-03)   (2561,1.551e-03)  (7937,5.236e-04)
(23297,1.723e-04) (65537,5.545e-05) (178177,1.751e-05)
};

\addplot coordinates{
(11,6.887e-02)    (71,3.177e-02)     (351,1.341e-02)
(1471,5.334e-03)  (5503,2.027e-03)   (18943,7.415e-04)
(61183,2.628e-04) (187903,9.063e-05) (553983,3.053e-05)
};

\addplot coordinates{
(13,5.755e-02)     (97,2.925e-02)     (545,1.351e-02)
(2561,5.842e-03)   (10625,2.397e-03)  (40193,9.414e-04)
(141569,3.564e-04) (471041,1.308e-04)
(1496065,4.670e-05)
};
}%


\bookmaketitle
\tableofcontents
\include{pgfplots.title_abstract_intro}
\include{pgfplots.preliminaries}
\include{pgfplots.intro}
\include{pgfplots.reference}
\include{pgfplots.libs}
\include{pgfplots.resources}
\include{pgfplots.importexport}
\include{pgfplots.basic.reference}

\printindex

\bibliographystyle{abbrv} %gerapali} %gerabbrv} %gerunsrt.bst} %gerabbrv}% gerplain}
\nocite{pgfplotstable}
\nocite{programmingnotes}
\bibliography{pgfplots}
\end{document}

% -----------------------------------------------------------------------------
% For Stefan Pinnow as reminder on what to look for when editing the manual
% -----------------------------------------------------------------------------
% There should be no line breaks in the following environments
% - |...|
% - \declareandlabel{...}
% - \verbpdfref{...}
% If "MakeTikzPictures" isn't running through check one of the externalized
% LOG files what is the cause of that.
% -----------------------------------------------------------------------------


%%%%%%%%%%%%%%%%%%%%%%%%%%%%%%%%%%%%%%%%%%%%%%%%%%%%%%%%%%%%%%%%%%%%%%%%%%%%%
%
% Package pgfplots.sty documentation.
%
% Copyright 2007/2008 by Christian Feuersaenger.
%
% This program is free software: you can redistribute it and/or modify
% it under the terms of the GNU General Public License as published by
% the Free Software Foundation, either version 3 of the License, or
% (at your option) any later version.
%
% This program is distributed in the hope that it will be useful,
% but WITHOUT ANY WARRANTY; without even the implied warranty of
% MERCHANTABILITY or FITNESS FOR A PARTICULAR PURPOSE.  See the
% GNU General Public License for more details.
%
% You should have received a copy of the GNU General Public License
% along with this program.  If not, see <http://www.gnu.org/licenses/>.
%
%
%%%%%%%%%%%%%%%%%%%%%%%%%%%%%%%%%%%%%%%%%%%%%%%%%%%%%%%%%%%%%%%%%%%%%%%%%%%%%
% SEE pgfplots-macros.tex as well!
%\pdfminorversion=5 % to allow compression
%\pdfobjcompresslevel=2
\documentclass[a4paper,openany]{book}

\let\bookmaketitle=\maketitle

% -----------------------------------
% this here is from ltxdoc:
\usepackage{doc}[=v2]
\AtBeginDocument{\MakeShortVerb{\|}}
%\setlength{\textwidth}{355pt}
%\addtolength\marginparwidth{30pt}
%\addtolength\oddsidemargin{20pt}
%\addtolength\evensidemargin{20pt}
\def\cmd#1{\cs{\expandafter\cmd@to@cs\string#1}}
\def\cmd@to@cs#1#2{\char\number`#2\relax}
\DeclareRobustCommand\cs[1]{\texttt{\char`\\#1}}
\providecommand\marg[1]{%
  {\ttfamily\char`\{}\meta{#1}{\ttfamily\char`\}}}
\providecommand\oarg[1]{%
  {\ttfamily[}\meta{#1}{\ttfamily]}}
\providecommand\parg[1]{%
  {\ttfamily(}\meta{#1}{\ttfamily)}}
\raggedbottom

% -----------------------------------
\input{pgfplots.preamble.tex}

\makeatletter
% I want a two-column index just as in pgfmanual styles. This here
% was the best way to get one:
\def\index@prologue{\section*{Index}\addcontentsline{toc}{chapter}{Index}
}
\makeatother

%\RequirePackage[german,english,francais]{babel}

\def\matlabcolormaptext{This colormap is similar to one shipped with Matlab$^\text{\textregistered}$ under a similar name.}

\IfFileExists{tikzlibraryspy.code.tex}{%
\usetikzlibrary{spy}
}{%
    \message{ERROR: tikz SPY library NOT available. The manual will only compile partially.^^J}%
}%

\usepackage{xparse}% for colorbrewer manual

\usetikzlibrary{
    decorations.markings,
    decorations.footprints,
    shapes.arrows,
    matrix,
    positioning,
}

\usepgfplotslibrary{
    fillbetween,
    ternary,
    smithchart,
    patchplots,
    polar,
    colormaps,
    colorbrewer,
    colortol,           % docu not ready yet
}
\pgfqkeys{/codeexample}{%
    every codeexample/.append style={
        /pgfplots/every ternary axis/.append style={
            /pgfplots/legend style={fill=graphicbackground},
        }
    },
    tabsize=4,
}

\pgfplotsmanualenableexternalizationofexpensive

%\usetikzlibrary{external}
%\tikzexternalize[prefix=figures/]{pgfplots}

\title{%
    Manual for Package \PGFPlots{}\\
    {\small 2D/3D Plots in \LaTeX{}, Version \pgfplotsversion}\\
    {\small\url{https://github.com/pgf-tikz/pgfplots}}
    %\\{\small Attention: you are using an unstable development version.}
}

\makeatletter
\long\def\abstractsmuggle{%
    \centering
    \textbf{Abstract}\\[0.5cm]

    \begin{minipage}{12cm}
        \PGFPlots{} draws high-quality function plots in normal or logarithmic
        scaling with a user-friendly interface directly in \TeX{}. The user
        supplies axis labels, legend entries and the plot coordinates for one or
        more plots and \PGFPlots{} applies axis scaling, computes any logarithms
        and axis ticks and draws the plots. It supports line plots, scatter
        plots, piecewise constant plots, bar plots, area plots, mesh and surface
        plots, patch plots, contour plots, quiver plots, histogram plots, box
        plots, polar axes, ternary diagrams, smith charts and some more. It is
        based on Till Tantau's package \PGF{}/\Tikz{}.
    \end{minipage}

}%

\expandafter\date\expandafter{\@date\\[2cm]
    \abstractsmuggle
}%
\makeatother

%\includeonly{pgfplots.reference}


\begin{document}

\def\plotcoords{%
\addplot coordinates {
(5,8.312e-02)    (17,2.547e-02)   (49,7.407e-03)
(129,2.102e-03)  (321,5.874e-04)  (769,1.623e-04)
(1793,4.442e-05) (4097,1.207e-05) (9217,3.261e-06)
};

\addplot coordinates{
(7,8.472e-02)    (31,3.044e-02)    (111,1.022e-02)
(351,3.303e-03)  (1023,1.039e-03)  (2815,3.196e-04)
(7423,9.658e-05) (18943,2.873e-05) (47103,8.437e-06)
};

\addplot coordinates{
(9,7.881e-02)     (49,3.243e-02)    (209,1.232e-02)
(769,4.454e-03)   (2561,1.551e-03)  (7937,5.236e-04)
(23297,1.723e-04) (65537,5.545e-05) (178177,1.751e-05)
};

\addplot coordinates{
(11,6.887e-02)    (71,3.177e-02)     (351,1.341e-02)
(1471,5.334e-03)  (5503,2.027e-03)   (18943,7.415e-04)
(61183,2.628e-04) (187903,9.063e-05) (553983,3.053e-05)
};

\addplot coordinates{
(13,5.755e-02)     (97,2.925e-02)     (545,1.351e-02)
(2561,5.842e-03)   (10625,2.397e-03)  (40193,9.414e-04)
(141569,3.564e-04) (471041,1.308e-04)
(1496065,4.670e-05)
};
}%


\bookmaketitle
\tableofcontents
\include{pgfplots.title_abstract_intro}
\include{pgfplots.preliminaries}
\include{pgfplots.intro}
\include{pgfplots.reference}
\include{pgfplots.libs}
\include{pgfplots.resources}
\include{pgfplots.importexport}
\include{pgfplots.basic.reference}

\printindex

\bibliographystyle{abbrv} %gerapali} %gerabbrv} %gerunsrt.bst} %gerabbrv}% gerplain}
\nocite{pgfplotstable}
\nocite{programmingnotes}
\bibliography{pgfplots}
\end{document}

% -----------------------------------------------------------------------------
% For Stefan Pinnow as reminder on what to look for when editing the manual
% -----------------------------------------------------------------------------
% There should be no line breaks in the following environments
% - |...|
% - \declareandlabel{...}
% - \verbpdfref{...}
% If "MakeTikzPictures" isn't running through check one of the externalized
% LOG files what is the cause of that.
% -----------------------------------------------------------------------------


%%%%%%%%%%%%%%%%%%%%%%%%%%%%%%%%%%%%%%%%%%%%%%%%%%%%%%%%%%%%%%%%%%%%%%%%%%%%%
%
% Package pgfplots.sty documentation.
%
% Copyright 2007/2008 by Christian Feuersaenger.
%
% This program is free software: you can redistribute it and/or modify
% it under the terms of the GNU General Public License as published by
% the Free Software Foundation, either version 3 of the License, or
% (at your option) any later version.
%
% This program is distributed in the hope that it will be useful,
% but WITHOUT ANY WARRANTY; without even the implied warranty of
% MERCHANTABILITY or FITNESS FOR A PARTICULAR PURPOSE.  See the
% GNU General Public License for more details.
%
% You should have received a copy of the GNU General Public License
% along with this program.  If not, see <http://www.gnu.org/licenses/>.
%
%
%%%%%%%%%%%%%%%%%%%%%%%%%%%%%%%%%%%%%%%%%%%%%%%%%%%%%%%%%%%%%%%%%%%%%%%%%%%%%
% SEE pgfplots-macros.tex as well!
%\pdfminorversion=5 % to allow compression
%\pdfobjcompresslevel=2
\documentclass[a4paper,openany]{book}

\let\bookmaketitle=\maketitle

% -----------------------------------
% this here is from ltxdoc:
\usepackage{doc}[=v2]
\AtBeginDocument{\MakeShortVerb{\|}}
%\setlength{\textwidth}{355pt}
%\addtolength\marginparwidth{30pt}
%\addtolength\oddsidemargin{20pt}
%\addtolength\evensidemargin{20pt}
\def\cmd#1{\cs{\expandafter\cmd@to@cs\string#1}}
\def\cmd@to@cs#1#2{\char\number`#2\relax}
\DeclareRobustCommand\cs[1]{\texttt{\char`\\#1}}
\providecommand\marg[1]{%
  {\ttfamily\char`\{}\meta{#1}{\ttfamily\char`\}}}
\providecommand\oarg[1]{%
  {\ttfamily[}\meta{#1}{\ttfamily]}}
\providecommand\parg[1]{%
  {\ttfamily(}\meta{#1}{\ttfamily)}}
\raggedbottom

% -----------------------------------
\input{pgfplots.preamble.tex}

\makeatletter
% I want a two-column index just as in pgfmanual styles. This here
% was the best way to get one:
\def\index@prologue{\section*{Index}\addcontentsline{toc}{chapter}{Index}
}
\makeatother

%\RequirePackage[german,english,francais]{babel}

\def\matlabcolormaptext{This colormap is similar to one shipped with Matlab$^\text{\textregistered}$ under a similar name.}

\IfFileExists{tikzlibraryspy.code.tex}{%
\usetikzlibrary{spy}
}{%
    \message{ERROR: tikz SPY library NOT available. The manual will only compile partially.^^J}%
}%

\usepackage{xparse}% for colorbrewer manual

\usetikzlibrary{
    decorations.markings,
    decorations.footprints,
    shapes.arrows,
    matrix,
    positioning,
}

\usepgfplotslibrary{
    fillbetween,
    ternary,
    smithchart,
    patchplots,
    polar,
    colormaps,
    colorbrewer,
    colortol,           % docu not ready yet
}
\pgfqkeys{/codeexample}{%
    every codeexample/.append style={
        /pgfplots/every ternary axis/.append style={
            /pgfplots/legend style={fill=graphicbackground},
        }
    },
    tabsize=4,
}

\pgfplotsmanualenableexternalizationofexpensive

%\usetikzlibrary{external}
%\tikzexternalize[prefix=figures/]{pgfplots}

\title{%
    Manual for Package \PGFPlots{}\\
    {\small 2D/3D Plots in \LaTeX{}, Version \pgfplotsversion}\\
    {\small\url{https://github.com/pgf-tikz/pgfplots}}
    %\\{\small Attention: you are using an unstable development version.}
}

\makeatletter
\long\def\abstractsmuggle{%
    \centering
    \textbf{Abstract}\\[0.5cm]

    \begin{minipage}{12cm}
        \PGFPlots{} draws high-quality function plots in normal or logarithmic
        scaling with a user-friendly interface directly in \TeX{}. The user
        supplies axis labels, legend entries and the plot coordinates for one or
        more plots and \PGFPlots{} applies axis scaling, computes any logarithms
        and axis ticks and draws the plots. It supports line plots, scatter
        plots, piecewise constant plots, bar plots, area plots, mesh and surface
        plots, patch plots, contour plots, quiver plots, histogram plots, box
        plots, polar axes, ternary diagrams, smith charts and some more. It is
        based on Till Tantau's package \PGF{}/\Tikz{}.
    \end{minipage}

}%

\expandafter\date\expandafter{\@date\\[2cm]
    \abstractsmuggle
}%
\makeatother

%\includeonly{pgfplots.reference}


\begin{document}

\def\plotcoords{%
\addplot coordinates {
(5,8.312e-02)    (17,2.547e-02)   (49,7.407e-03)
(129,2.102e-03)  (321,5.874e-04)  (769,1.623e-04)
(1793,4.442e-05) (4097,1.207e-05) (9217,3.261e-06)
};

\addplot coordinates{
(7,8.472e-02)    (31,3.044e-02)    (111,1.022e-02)
(351,3.303e-03)  (1023,1.039e-03)  (2815,3.196e-04)
(7423,9.658e-05) (18943,2.873e-05) (47103,8.437e-06)
};

\addplot coordinates{
(9,7.881e-02)     (49,3.243e-02)    (209,1.232e-02)
(769,4.454e-03)   (2561,1.551e-03)  (7937,5.236e-04)
(23297,1.723e-04) (65537,5.545e-05) (178177,1.751e-05)
};

\addplot coordinates{
(11,6.887e-02)    (71,3.177e-02)     (351,1.341e-02)
(1471,5.334e-03)  (5503,2.027e-03)   (18943,7.415e-04)
(61183,2.628e-04) (187903,9.063e-05) (553983,3.053e-05)
};

\addplot coordinates{
(13,5.755e-02)     (97,2.925e-02)     (545,1.351e-02)
(2561,5.842e-03)   (10625,2.397e-03)  (40193,9.414e-04)
(141569,3.564e-04) (471041,1.308e-04)
(1496065,4.670e-05)
};
}%


\bookmaketitle
\tableofcontents
\include{pgfplots.title_abstract_intro}
\include{pgfplots.preliminaries}
\include{pgfplots.intro}
\include{pgfplots.reference}
\include{pgfplots.libs}
\include{pgfplots.resources}
\include{pgfplots.importexport}
\include{pgfplots.basic.reference}

\printindex

\bibliographystyle{abbrv} %gerapali} %gerabbrv} %gerunsrt.bst} %gerabbrv}% gerplain}
\nocite{pgfplotstable}
\nocite{programmingnotes}
\bibliography{pgfplots}
\end{document}

% -----------------------------------------------------------------------------
% For Stefan Pinnow as reminder on what to look for when editing the manual
% -----------------------------------------------------------------------------
% There should be no line breaks in the following environments
% - |...|
% - \declareandlabel{...}
% - \verbpdfref{...}
% If "MakeTikzPictures" isn't running through check one of the externalized
% LOG files what is the cause of that.
% -----------------------------------------------------------------------------


%%%%%%%%%%%%%%%%%%%%%%%%%%%%%%%%%%%%%%%%%%%%%%%%%%%%%%%%%%%%%%%%%%%%%%%%%%%%%
%
% Package pgfplots.sty documentation.
%
% Copyright 2007/2008 by Christian Feuersaenger.
%
% This program is free software: you can redistribute it and/or modify
% it under the terms of the GNU General Public License as published by
% the Free Software Foundation, either version 3 of the License, or
% (at your option) any later version.
%
% This program is distributed in the hope that it will be useful,
% but WITHOUT ANY WARRANTY; without even the implied warranty of
% MERCHANTABILITY or FITNESS FOR A PARTICULAR PURPOSE.  See the
% GNU General Public License for more details.
%
% You should have received a copy of the GNU General Public License
% along with this program.  If not, see <http://www.gnu.org/licenses/>.
%
%
%%%%%%%%%%%%%%%%%%%%%%%%%%%%%%%%%%%%%%%%%%%%%%%%%%%%%%%%%%%%%%%%%%%%%%%%%%%%%
% SEE pgfplots-macros.tex as well!
%\pdfminorversion=5 % to allow compression
%\pdfobjcompresslevel=2
\documentclass[a4paper,openany]{book}

\let\bookmaketitle=\maketitle

% -----------------------------------
% this here is from ltxdoc:
\usepackage{doc}[=v2]
\AtBeginDocument{\MakeShortVerb{\|}}
%\setlength{\textwidth}{355pt}
%\addtolength\marginparwidth{30pt}
%\addtolength\oddsidemargin{20pt}
%\addtolength\evensidemargin{20pt}
\def\cmd#1{\cs{\expandafter\cmd@to@cs\string#1}}
\def\cmd@to@cs#1#2{\char\number`#2\relax}
\DeclareRobustCommand\cs[1]{\texttt{\char`\\#1}}
\providecommand\marg[1]{%
  {\ttfamily\char`\{}\meta{#1}{\ttfamily\char`\}}}
\providecommand\oarg[1]{%
  {\ttfamily[}\meta{#1}{\ttfamily]}}
\providecommand\parg[1]{%
  {\ttfamily(}\meta{#1}{\ttfamily)}}
\raggedbottom

% -----------------------------------
\input{pgfplots.preamble.tex}

\makeatletter
% I want a two-column index just as in pgfmanual styles. This here
% was the best way to get one:
\def\index@prologue{\section*{Index}\addcontentsline{toc}{chapter}{Index}
}
\makeatother

%\RequirePackage[german,english,francais]{babel}

\def\matlabcolormaptext{This colormap is similar to one shipped with Matlab$^\text{\textregistered}$ under a similar name.}

\IfFileExists{tikzlibraryspy.code.tex}{%
\usetikzlibrary{spy}
}{%
    \message{ERROR: tikz SPY library NOT available. The manual will only compile partially.^^J}%
}%

\usepackage{xparse}% for colorbrewer manual

\usetikzlibrary{
    decorations.markings,
    decorations.footprints,
    shapes.arrows,
    matrix,
    positioning,
}

\usepgfplotslibrary{
    fillbetween,
    ternary,
    smithchart,
    patchplots,
    polar,
    colormaps,
    colorbrewer,
    colortol,           % docu not ready yet
}
\pgfqkeys{/codeexample}{%
    every codeexample/.append style={
        /pgfplots/every ternary axis/.append style={
            /pgfplots/legend style={fill=graphicbackground},
        }
    },
    tabsize=4,
}

\pgfplotsmanualenableexternalizationofexpensive

%\usetikzlibrary{external}
%\tikzexternalize[prefix=figures/]{pgfplots}

\title{%
    Manual for Package \PGFPlots{}\\
    {\small 2D/3D Plots in \LaTeX{}, Version \pgfplotsversion}\\
    {\small\url{https://github.com/pgf-tikz/pgfplots}}
    %\\{\small Attention: you are using an unstable development version.}
}

\makeatletter
\long\def\abstractsmuggle{%
    \centering
    \textbf{Abstract}\\[0.5cm]

    \begin{minipage}{12cm}
        \PGFPlots{} draws high-quality function plots in normal or logarithmic
        scaling with a user-friendly interface directly in \TeX{}. The user
        supplies axis labels, legend entries and the plot coordinates for one or
        more plots and \PGFPlots{} applies axis scaling, computes any logarithms
        and axis ticks and draws the plots. It supports line plots, scatter
        plots, piecewise constant plots, bar plots, area plots, mesh and surface
        plots, patch plots, contour plots, quiver plots, histogram plots, box
        plots, polar axes, ternary diagrams, smith charts and some more. It is
        based on Till Tantau's package \PGF{}/\Tikz{}.
    \end{minipage}

}%

\expandafter\date\expandafter{\@date\\[2cm]
    \abstractsmuggle
}%
\makeatother

%\includeonly{pgfplots.reference}


\begin{document}

\def\plotcoords{%
\addplot coordinates {
(5,8.312e-02)    (17,2.547e-02)   (49,7.407e-03)
(129,2.102e-03)  (321,5.874e-04)  (769,1.623e-04)
(1793,4.442e-05) (4097,1.207e-05) (9217,3.261e-06)
};

\addplot coordinates{
(7,8.472e-02)    (31,3.044e-02)    (111,1.022e-02)
(351,3.303e-03)  (1023,1.039e-03)  (2815,3.196e-04)
(7423,9.658e-05) (18943,2.873e-05) (47103,8.437e-06)
};

\addplot coordinates{
(9,7.881e-02)     (49,3.243e-02)    (209,1.232e-02)
(769,4.454e-03)   (2561,1.551e-03)  (7937,5.236e-04)
(23297,1.723e-04) (65537,5.545e-05) (178177,1.751e-05)
};

\addplot coordinates{
(11,6.887e-02)    (71,3.177e-02)     (351,1.341e-02)
(1471,5.334e-03)  (5503,2.027e-03)   (18943,7.415e-04)
(61183,2.628e-04) (187903,9.063e-05) (553983,3.053e-05)
};

\addplot coordinates{
(13,5.755e-02)     (97,2.925e-02)     (545,1.351e-02)
(2561,5.842e-03)   (10625,2.397e-03)  (40193,9.414e-04)
(141569,3.564e-04) (471041,1.308e-04)
(1496065,4.670e-05)
};
}%


\bookmaketitle
\tableofcontents
\include{pgfplots.title_abstract_intro}
\include{pgfplots.preliminaries}
\include{pgfplots.intro}
\include{pgfplots.reference}
\include{pgfplots.libs}
\include{pgfplots.resources}
\include{pgfplots.importexport}
\include{pgfplots.basic.reference}

\printindex

\bibliographystyle{abbrv} %gerapali} %gerabbrv} %gerunsrt.bst} %gerabbrv}% gerplain}
\nocite{pgfplotstable}
\nocite{programmingnotes}
\bibliography{pgfplots}
\end{document}

% -----------------------------------------------------------------------------
% For Stefan Pinnow as reminder on what to look for when editing the manual
% -----------------------------------------------------------------------------
% There should be no line breaks in the following environments
% - |...|
% - \declareandlabel{...}
% - \verbpdfref{...}
% If "MakeTikzPictures" isn't running through check one of the externalized
% LOG files what is the cause of that.
% -----------------------------------------------------------------------------


%%%%%%%%%%%%%%%%%%%%%%%%%%%%%%%%%%%%%%%%%%%%%%%%%%%%%%%%%%%%%%%%%%%%%%%%%%%%%
%
% Package pgfplots.sty documentation.
%
% Copyright 2007/2008 by Christian Feuersaenger.
%
% This program is free software: you can redistribute it and/or modify
% it under the terms of the GNU General Public License as published by
% the Free Software Foundation, either version 3 of the License, or
% (at your option) any later version.
%
% This program is distributed in the hope that it will be useful,
% but WITHOUT ANY WARRANTY; without even the implied warranty of
% MERCHANTABILITY or FITNESS FOR A PARTICULAR PURPOSE.  See the
% GNU General Public License for more details.
%
% You should have received a copy of the GNU General Public License
% along with this program.  If not, see <http://www.gnu.org/licenses/>.
%
%
%%%%%%%%%%%%%%%%%%%%%%%%%%%%%%%%%%%%%%%%%%%%%%%%%%%%%%%%%%%%%%%%%%%%%%%%%%%%%
% SEE pgfplots-macros.tex as well!
%\pdfminorversion=5 % to allow compression
%\pdfobjcompresslevel=2
\documentclass[a4paper,openany]{book}

\let\bookmaketitle=\maketitle

% -----------------------------------
% this here is from ltxdoc:
\usepackage{doc}[=v2]
\AtBeginDocument{\MakeShortVerb{\|}}
%\setlength{\textwidth}{355pt}
%\addtolength\marginparwidth{30pt}
%\addtolength\oddsidemargin{20pt}
%\addtolength\evensidemargin{20pt}
\def\cmd#1{\cs{\expandafter\cmd@to@cs\string#1}}
\def\cmd@to@cs#1#2{\char\number`#2\relax}
\DeclareRobustCommand\cs[1]{\texttt{\char`\\#1}}
\providecommand\marg[1]{%
  {\ttfamily\char`\{}\meta{#1}{\ttfamily\char`\}}}
\providecommand\oarg[1]{%
  {\ttfamily[}\meta{#1}{\ttfamily]}}
\providecommand\parg[1]{%
  {\ttfamily(}\meta{#1}{\ttfamily)}}
\raggedbottom

% -----------------------------------
\input{pgfplots.preamble.tex}

\makeatletter
% I want a two-column index just as in pgfmanual styles. This here
% was the best way to get one:
\def\index@prologue{\section*{Index}\addcontentsline{toc}{chapter}{Index}
}
\makeatother

%\RequirePackage[german,english,francais]{babel}

\def\matlabcolormaptext{This colormap is similar to one shipped with Matlab$^\text{\textregistered}$ under a similar name.}

\IfFileExists{tikzlibraryspy.code.tex}{%
\usetikzlibrary{spy}
}{%
    \message{ERROR: tikz SPY library NOT available. The manual will only compile partially.^^J}%
}%

\usepackage{xparse}% for colorbrewer manual

\usetikzlibrary{
    decorations.markings,
    decorations.footprints,
    shapes.arrows,
    matrix,
    positioning,
}

\usepgfplotslibrary{
    fillbetween,
    ternary,
    smithchart,
    patchplots,
    polar,
    colormaps,
    colorbrewer,
    colortol,           % docu not ready yet
}
\pgfqkeys{/codeexample}{%
    every codeexample/.append style={
        /pgfplots/every ternary axis/.append style={
            /pgfplots/legend style={fill=graphicbackground},
        }
    },
    tabsize=4,
}

\pgfplotsmanualenableexternalizationofexpensive

%\usetikzlibrary{external}
%\tikzexternalize[prefix=figures/]{pgfplots}

\title{%
    Manual for Package \PGFPlots{}\\
    {\small 2D/3D Plots in \LaTeX{}, Version \pgfplotsversion}\\
    {\small\url{https://github.com/pgf-tikz/pgfplots}}
    %\\{\small Attention: you are using an unstable development version.}
}

\makeatletter
\long\def\abstractsmuggle{%
    \centering
    \textbf{Abstract}\\[0.5cm]

    \begin{minipage}{12cm}
        \PGFPlots{} draws high-quality function plots in normal or logarithmic
        scaling with a user-friendly interface directly in \TeX{}. The user
        supplies axis labels, legend entries and the plot coordinates for one or
        more plots and \PGFPlots{} applies axis scaling, computes any logarithms
        and axis ticks and draws the plots. It supports line plots, scatter
        plots, piecewise constant plots, bar plots, area plots, mesh and surface
        plots, patch plots, contour plots, quiver plots, histogram plots, box
        plots, polar axes, ternary diagrams, smith charts and some more. It is
        based on Till Tantau's package \PGF{}/\Tikz{}.
    \end{minipage}

}%

\expandafter\date\expandafter{\@date\\[2cm]
    \abstractsmuggle
}%
\makeatother

%\includeonly{pgfplots.reference}


\begin{document}

\def\plotcoords{%
\addplot coordinates {
(5,8.312e-02)    (17,2.547e-02)   (49,7.407e-03)
(129,2.102e-03)  (321,5.874e-04)  (769,1.623e-04)
(1793,4.442e-05) (4097,1.207e-05) (9217,3.261e-06)
};

\addplot coordinates{
(7,8.472e-02)    (31,3.044e-02)    (111,1.022e-02)
(351,3.303e-03)  (1023,1.039e-03)  (2815,3.196e-04)
(7423,9.658e-05) (18943,2.873e-05) (47103,8.437e-06)
};

\addplot coordinates{
(9,7.881e-02)     (49,3.243e-02)    (209,1.232e-02)
(769,4.454e-03)   (2561,1.551e-03)  (7937,5.236e-04)
(23297,1.723e-04) (65537,5.545e-05) (178177,1.751e-05)
};

\addplot coordinates{
(11,6.887e-02)    (71,3.177e-02)     (351,1.341e-02)
(1471,5.334e-03)  (5503,2.027e-03)   (18943,7.415e-04)
(61183,2.628e-04) (187903,9.063e-05) (553983,3.053e-05)
};

\addplot coordinates{
(13,5.755e-02)     (97,2.925e-02)     (545,1.351e-02)
(2561,5.842e-03)   (10625,2.397e-03)  (40193,9.414e-04)
(141569,3.564e-04) (471041,1.308e-04)
(1496065,4.670e-05)
};
}%


\bookmaketitle
\tableofcontents
\include{pgfplots.title_abstract_intro}
\include{pgfplots.preliminaries}
\include{pgfplots.intro}
\include{pgfplots.reference}
\include{pgfplots.libs}
\include{pgfplots.resources}
\include{pgfplots.importexport}
\include{pgfplots.basic.reference}

\printindex

\bibliographystyle{abbrv} %gerapali} %gerabbrv} %gerunsrt.bst} %gerabbrv}% gerplain}
\nocite{pgfplotstable}
\nocite{programmingnotes}
\bibliography{pgfplots}
\end{document}

% -----------------------------------------------------------------------------
% For Stefan Pinnow as reminder on what to look for when editing the manual
% -----------------------------------------------------------------------------
% There should be no line breaks in the following environments
% - |...|
% - \declareandlabel{...}
% - \verbpdfref{...}
% If "MakeTikzPictures" isn't running through check one of the externalized
% LOG files what is the cause of that.
% -----------------------------------------------------------------------------


%%%%%%%%%%%%%%%%%%%%%%%%%%%%%%%%%%%%%%%%%%%%%%%%%%%%%%%%%%%%%%%%%%%%%%%%%%%%%
%
% Package pgfplots.sty documentation.
%
% Copyright 2007/2008 by Christian Feuersaenger.
%
% This program is free software: you can redistribute it and/or modify
% it under the terms of the GNU General Public License as published by
% the Free Software Foundation, either version 3 of the License, or
% (at your option) any later version.
%
% This program is distributed in the hope that it will be useful,
% but WITHOUT ANY WARRANTY; without even the implied warranty of
% MERCHANTABILITY or FITNESS FOR A PARTICULAR PURPOSE.  See the
% GNU General Public License for more details.
%
% You should have received a copy of the GNU General Public License
% along with this program.  If not, see <http://www.gnu.org/licenses/>.
%
%
%%%%%%%%%%%%%%%%%%%%%%%%%%%%%%%%%%%%%%%%%%%%%%%%%%%%%%%%%%%%%%%%%%%%%%%%%%%%%
% SEE pgfplots-macros.tex as well!
%\pdfminorversion=5 % to allow compression
%\pdfobjcompresslevel=2
\documentclass[a4paper,openany]{book}

\let\bookmaketitle=\maketitle

% -----------------------------------
% this here is from ltxdoc:
\usepackage{doc}[=v2]
\AtBeginDocument{\MakeShortVerb{\|}}
%\setlength{\textwidth}{355pt}
%\addtolength\marginparwidth{30pt}
%\addtolength\oddsidemargin{20pt}
%\addtolength\evensidemargin{20pt}
\def\cmd#1{\cs{\expandafter\cmd@to@cs\string#1}}
\def\cmd@to@cs#1#2{\char\number`#2\relax}
\DeclareRobustCommand\cs[1]{\texttt{\char`\\#1}}
\providecommand\marg[1]{%
  {\ttfamily\char`\{}\meta{#1}{\ttfamily\char`\}}}
\providecommand\oarg[1]{%
  {\ttfamily[}\meta{#1}{\ttfamily]}}
\providecommand\parg[1]{%
  {\ttfamily(}\meta{#1}{\ttfamily)}}
\raggedbottom

% -----------------------------------
\input{pgfplots.preamble.tex}

\makeatletter
% I want a two-column index just as in pgfmanual styles. This here
% was the best way to get one:
\def\index@prologue{\section*{Index}\addcontentsline{toc}{chapter}{Index}
}
\makeatother

%\RequirePackage[german,english,francais]{babel}

\def\matlabcolormaptext{This colormap is similar to one shipped with Matlab$^\text{\textregistered}$ under a similar name.}

\IfFileExists{tikzlibraryspy.code.tex}{%
\usetikzlibrary{spy}
}{%
    \message{ERROR: tikz SPY library NOT available. The manual will only compile partially.^^J}%
}%

\usepackage{xparse}% for colorbrewer manual

\usetikzlibrary{
    decorations.markings,
    decorations.footprints,
    shapes.arrows,
    matrix,
    positioning,
}

\usepgfplotslibrary{
    fillbetween,
    ternary,
    smithchart,
    patchplots,
    polar,
    colormaps,
    colorbrewer,
    colortol,           % docu not ready yet
}
\pgfqkeys{/codeexample}{%
    every codeexample/.append style={
        /pgfplots/every ternary axis/.append style={
            /pgfplots/legend style={fill=graphicbackground},
        }
    },
    tabsize=4,
}

\pgfplotsmanualenableexternalizationofexpensive

%\usetikzlibrary{external}
%\tikzexternalize[prefix=figures/]{pgfplots}

\title{%
    Manual for Package \PGFPlots{}\\
    {\small 2D/3D Plots in \LaTeX{}, Version \pgfplotsversion}\\
    {\small\url{https://github.com/pgf-tikz/pgfplots}}
    %\\{\small Attention: you are using an unstable development version.}
}

\makeatletter
\long\def\abstractsmuggle{%
    \centering
    \textbf{Abstract}\\[0.5cm]

    \begin{minipage}{12cm}
        \PGFPlots{} draws high-quality function plots in normal or logarithmic
        scaling with a user-friendly interface directly in \TeX{}. The user
        supplies axis labels, legend entries and the plot coordinates for one or
        more plots and \PGFPlots{} applies axis scaling, computes any logarithms
        and axis ticks and draws the plots. It supports line plots, scatter
        plots, piecewise constant plots, bar plots, area plots, mesh and surface
        plots, patch plots, contour plots, quiver plots, histogram plots, box
        plots, polar axes, ternary diagrams, smith charts and some more. It is
        based on Till Tantau's package \PGF{}/\Tikz{}.
    \end{minipage}

}%

\expandafter\date\expandafter{\@date\\[2cm]
    \abstractsmuggle
}%
\makeatother

%\includeonly{pgfplots.reference}


\begin{document}

\def\plotcoords{%
\addplot coordinates {
(5,8.312e-02)    (17,2.547e-02)   (49,7.407e-03)
(129,2.102e-03)  (321,5.874e-04)  (769,1.623e-04)
(1793,4.442e-05) (4097,1.207e-05) (9217,3.261e-06)
};

\addplot coordinates{
(7,8.472e-02)    (31,3.044e-02)    (111,1.022e-02)
(351,3.303e-03)  (1023,1.039e-03)  (2815,3.196e-04)
(7423,9.658e-05) (18943,2.873e-05) (47103,8.437e-06)
};

\addplot coordinates{
(9,7.881e-02)     (49,3.243e-02)    (209,1.232e-02)
(769,4.454e-03)   (2561,1.551e-03)  (7937,5.236e-04)
(23297,1.723e-04) (65537,5.545e-05) (178177,1.751e-05)
};

\addplot coordinates{
(11,6.887e-02)    (71,3.177e-02)     (351,1.341e-02)
(1471,5.334e-03)  (5503,2.027e-03)   (18943,7.415e-04)
(61183,2.628e-04) (187903,9.063e-05) (553983,3.053e-05)
};

\addplot coordinates{
(13,5.755e-02)     (97,2.925e-02)     (545,1.351e-02)
(2561,5.842e-03)   (10625,2.397e-03)  (40193,9.414e-04)
(141569,3.564e-04) (471041,1.308e-04)
(1496065,4.670e-05)
};
}%


\bookmaketitle
\tableofcontents
\include{pgfplots.title_abstract_intro}
\include{pgfplots.preliminaries}
\include{pgfplots.intro}
\include{pgfplots.reference}
\include{pgfplots.libs}
\include{pgfplots.resources}
\include{pgfplots.importexport}
\include{pgfplots.basic.reference}

\printindex

\bibliographystyle{abbrv} %gerapali} %gerabbrv} %gerunsrt.bst} %gerabbrv}% gerplain}
\nocite{pgfplotstable}
\nocite{programmingnotes}
\bibliography{pgfplots}
\end{document}

% -----------------------------------------------------------------------------
% For Stefan Pinnow as reminder on what to look for when editing the manual
% -----------------------------------------------------------------------------
% There should be no line breaks in the following environments
% - |...|
% - \declareandlabel{...}
% - \verbpdfref{...}
% If "MakeTikzPictures" isn't running through check one of the externalized
% LOG files what is the cause of that.
% -----------------------------------------------------------------------------


%%%%%%%%%%%%%%%%%%%%%%%%%%%%%%%%%%%%%%%%%%%%%%%%%%%%%%%%%%%%%%%%%%%%%%%%%%%%%
%
% Package pgfplots.sty documentation.
%
% Copyright 2007/2008 by Christian Feuersaenger.
%
% This program is free software: you can redistribute it and/or modify
% it under the terms of the GNU General Public License as published by
% the Free Software Foundation, either version 3 of the License, or
% (at your option) any later version.
%
% This program is distributed in the hope that it will be useful,
% but WITHOUT ANY WARRANTY; without even the implied warranty of
% MERCHANTABILITY or FITNESS FOR A PARTICULAR PURPOSE.  See the
% GNU General Public License for more details.
%
% You should have received a copy of the GNU General Public License
% along with this program.  If not, see <http://www.gnu.org/licenses/>.
%
%
%%%%%%%%%%%%%%%%%%%%%%%%%%%%%%%%%%%%%%%%%%%%%%%%%%%%%%%%%%%%%%%%%%%%%%%%%%%%%
% SEE pgfplots-macros.tex as well!
%\pdfminorversion=5 % to allow compression
%\pdfobjcompresslevel=2
\documentclass[a4paper,openany]{book}

\let\bookmaketitle=\maketitle

% -----------------------------------
% this here is from ltxdoc:
\usepackage{doc}[=v2]
\AtBeginDocument{\MakeShortVerb{\|}}
%\setlength{\textwidth}{355pt}
%\addtolength\marginparwidth{30pt}
%\addtolength\oddsidemargin{20pt}
%\addtolength\evensidemargin{20pt}
\def\cmd#1{\cs{\expandafter\cmd@to@cs\string#1}}
\def\cmd@to@cs#1#2{\char\number`#2\relax}
\DeclareRobustCommand\cs[1]{\texttt{\char`\\#1}}
\providecommand\marg[1]{%
  {\ttfamily\char`\{}\meta{#1}{\ttfamily\char`\}}}
\providecommand\oarg[1]{%
  {\ttfamily[}\meta{#1}{\ttfamily]}}
\providecommand\parg[1]{%
  {\ttfamily(}\meta{#1}{\ttfamily)}}
\raggedbottom

% -----------------------------------
\input{pgfplots.preamble.tex}

\makeatletter
% I want a two-column index just as in pgfmanual styles. This here
% was the best way to get one:
\def\index@prologue{\section*{Index}\addcontentsline{toc}{chapter}{Index}
}
\makeatother

%\RequirePackage[german,english,francais]{babel}

\def\matlabcolormaptext{This colormap is similar to one shipped with Matlab$^\text{\textregistered}$ under a similar name.}

\IfFileExists{tikzlibraryspy.code.tex}{%
\usetikzlibrary{spy}
}{%
    \message{ERROR: tikz SPY library NOT available. The manual will only compile partially.^^J}%
}%

\usepackage{xparse}% for colorbrewer manual

\usetikzlibrary{
    decorations.markings,
    decorations.footprints,
    shapes.arrows,
    matrix,
    positioning,
}

\usepgfplotslibrary{
    fillbetween,
    ternary,
    smithchart,
    patchplots,
    polar,
    colormaps,
    colorbrewer,
    colortol,           % docu not ready yet
}
\pgfqkeys{/codeexample}{%
    every codeexample/.append style={
        /pgfplots/every ternary axis/.append style={
            /pgfplots/legend style={fill=graphicbackground},
        }
    },
    tabsize=4,
}

\pgfplotsmanualenableexternalizationofexpensive

%\usetikzlibrary{external}
%\tikzexternalize[prefix=figures/]{pgfplots}

\title{%
    Manual for Package \PGFPlots{}\\
    {\small 2D/3D Plots in \LaTeX{}, Version \pgfplotsversion}\\
    {\small\url{https://github.com/pgf-tikz/pgfplots}}
    %\\{\small Attention: you are using an unstable development version.}
}

\makeatletter
\long\def\abstractsmuggle{%
    \centering
    \textbf{Abstract}\\[0.5cm]

    \begin{minipage}{12cm}
        \PGFPlots{} draws high-quality function plots in normal or logarithmic
        scaling with a user-friendly interface directly in \TeX{}. The user
        supplies axis labels, legend entries and the plot coordinates for one or
        more plots and \PGFPlots{} applies axis scaling, computes any logarithms
        and axis ticks and draws the plots. It supports line plots, scatter
        plots, piecewise constant plots, bar plots, area plots, mesh and surface
        plots, patch plots, contour plots, quiver plots, histogram plots, box
        plots, polar axes, ternary diagrams, smith charts and some more. It is
        based on Till Tantau's package \PGF{}/\Tikz{}.
    \end{minipage}

}%

\expandafter\date\expandafter{\@date\\[2cm]
    \abstractsmuggle
}%
\makeatother

%\includeonly{pgfplots.reference}


\begin{document}

\def\plotcoords{%
\addplot coordinates {
(5,8.312e-02)    (17,2.547e-02)   (49,7.407e-03)
(129,2.102e-03)  (321,5.874e-04)  (769,1.623e-04)
(1793,4.442e-05) (4097,1.207e-05) (9217,3.261e-06)
};

\addplot coordinates{
(7,8.472e-02)    (31,3.044e-02)    (111,1.022e-02)
(351,3.303e-03)  (1023,1.039e-03)  (2815,3.196e-04)
(7423,9.658e-05) (18943,2.873e-05) (47103,8.437e-06)
};

\addplot coordinates{
(9,7.881e-02)     (49,3.243e-02)    (209,1.232e-02)
(769,4.454e-03)   (2561,1.551e-03)  (7937,5.236e-04)
(23297,1.723e-04) (65537,5.545e-05) (178177,1.751e-05)
};

\addplot coordinates{
(11,6.887e-02)    (71,3.177e-02)     (351,1.341e-02)
(1471,5.334e-03)  (5503,2.027e-03)   (18943,7.415e-04)
(61183,2.628e-04) (187903,9.063e-05) (553983,3.053e-05)
};

\addplot coordinates{
(13,5.755e-02)     (97,2.925e-02)     (545,1.351e-02)
(2561,5.842e-03)   (10625,2.397e-03)  (40193,9.414e-04)
(141569,3.564e-04) (471041,1.308e-04)
(1496065,4.670e-05)
};
}%


\bookmaketitle
\tableofcontents
\include{pgfplots.title_abstract_intro}
\include{pgfplots.preliminaries}
\include{pgfplots.intro}
\include{pgfplots.reference}
\include{pgfplots.libs}
\include{pgfplots.resources}
\include{pgfplots.importexport}
\include{pgfplots.basic.reference}

\printindex

\bibliographystyle{abbrv} %gerapali} %gerabbrv} %gerunsrt.bst} %gerabbrv}% gerplain}
\nocite{pgfplotstable}
\nocite{programmingnotes}
\bibliography{pgfplots}
\end{document}

% -----------------------------------------------------------------------------
% For Stefan Pinnow as reminder on what to look for when editing the manual
% -----------------------------------------------------------------------------
% There should be no line breaks in the following environments
% - |...|
% - \declareandlabel{...}
% - \verbpdfref{...}
% If "MakeTikzPictures" isn't running through check one of the externalized
% LOG files what is the cause of that.
% -----------------------------------------------------------------------------

\include{pgfplots.basic.reference}

\printindex

\bibliographystyle{abbrv} %gerapali} %gerabbrv} %gerunsrt.bst} %gerabbrv}% gerplain}
\nocite{pgfplotstable}
\nocite{programmingnotes}
\bibliography{pgfplots}
\end{document}

% -----------------------------------------------------------------------------
% For Stefan Pinnow as reminder on what to look for when editing the manual
% -----------------------------------------------------------------------------
% There should be no line breaks in the following environments
% - |...|
% - \declareandlabel{...}
% - \verbpdfref{...}
% If "MakeTikzPictures" isn't running through check one of the externalized
% LOG files what is the cause of that.
% -----------------------------------------------------------------------------


%%%%%%%%%%%%%%%%%%%%%%%%%%%%%%%%%%%%%%%%%%%%%%%%%%%%%%%%%%%%%%%%%%%%%%%%%%%%%
%
% Package pgfplots.sty documentation.
%
% Copyright 2007/2008 by Christian Feuersaenger.
%
% This program is free software: you can redistribute it and/or modify
% it under the terms of the GNU General Public License as published by
% the Free Software Foundation, either version 3 of the License, or
% (at your option) any later version.
%
% This program is distributed in the hope that it will be useful,
% but WITHOUT ANY WARRANTY; without even the implied warranty of
% MERCHANTABILITY or FITNESS FOR A PARTICULAR PURPOSE.  See the
% GNU General Public License for more details.
%
% You should have received a copy of the GNU General Public License
% along with this program.  If not, see <http://www.gnu.org/licenses/>.
%
%
%%%%%%%%%%%%%%%%%%%%%%%%%%%%%%%%%%%%%%%%%%%%%%%%%%%%%%%%%%%%%%%%%%%%%%%%%%%%%
% SEE pgfplots-macros.tex as well!
%\pdfminorversion=5 % to allow compression
%\pdfobjcompresslevel=2
\documentclass[a4paper,openany]{book}

\let\bookmaketitle=\maketitle

% -----------------------------------
% this here is from ltxdoc:
\usepackage{doc}[=v2]
\AtBeginDocument{\MakeShortVerb{\|}}
%\setlength{\textwidth}{355pt}
%\addtolength\marginparwidth{30pt}
%\addtolength\oddsidemargin{20pt}
%\addtolength\evensidemargin{20pt}
\def\cmd#1{\cs{\expandafter\cmd@to@cs\string#1}}
\def\cmd@to@cs#1#2{\char\number`#2\relax}
\DeclareRobustCommand\cs[1]{\texttt{\char`\\#1}}
\providecommand\marg[1]{%
  {\ttfamily\char`\{}\meta{#1}{\ttfamily\char`\}}}
\providecommand\oarg[1]{%
  {\ttfamily[}\meta{#1}{\ttfamily]}}
\providecommand\parg[1]{%
  {\ttfamily(}\meta{#1}{\ttfamily)}}
\raggedbottom

% -----------------------------------
\input{pgfplots.preamble.tex}

\makeatletter
% I want a two-column index just as in pgfmanual styles. This here
% was the best way to get one:
\def\index@prologue{\section*{Index}\addcontentsline{toc}{chapter}{Index}
}
\makeatother

%\RequirePackage[german,english,francais]{babel}

\def\matlabcolormaptext{This colormap is similar to one shipped with Matlab$^\text{\textregistered}$ under a similar name.}

\IfFileExists{tikzlibraryspy.code.tex}{%
\usetikzlibrary{spy}
}{%
    \message{ERROR: tikz SPY library NOT available. The manual will only compile partially.^^J}%
}%

\usepackage{xparse}% for colorbrewer manual

\usetikzlibrary{
    decorations.markings,
    decorations.footprints,
    shapes.arrows,
    matrix,
    positioning,
}

\usepgfplotslibrary{
    fillbetween,
    ternary,
    smithchart,
    patchplots,
    polar,
    colormaps,
    colorbrewer,
    colortol,           % docu not ready yet
}
\pgfqkeys{/codeexample}{%
    every codeexample/.append style={
        /pgfplots/every ternary axis/.append style={
            /pgfplots/legend style={fill=graphicbackground},
        }
    },
    tabsize=4,
}

\pgfplotsmanualenableexternalizationofexpensive

%\usetikzlibrary{external}
%\tikzexternalize[prefix=figures/]{pgfplots}

\title{%
    Manual for Package \PGFPlots{}\\
    {\small 2D/3D Plots in \LaTeX{}, Version \pgfplotsversion}\\
    {\small\url{https://github.com/pgf-tikz/pgfplots}}
    %\\{\small Attention: you are using an unstable development version.}
}

\makeatletter
\long\def\abstractsmuggle{%
    \centering
    \textbf{Abstract}\\[0.5cm]

    \begin{minipage}{12cm}
        \PGFPlots{} draws high-quality function plots in normal or logarithmic
        scaling with a user-friendly interface directly in \TeX{}. The user
        supplies axis labels, legend entries and the plot coordinates for one or
        more plots and \PGFPlots{} applies axis scaling, computes any logarithms
        and axis ticks and draws the plots. It supports line plots, scatter
        plots, piecewise constant plots, bar plots, area plots, mesh and surface
        plots, patch plots, contour plots, quiver plots, histogram plots, box
        plots, polar axes, ternary diagrams, smith charts and some more. It is
        based on Till Tantau's package \PGF{}/\Tikz{}.
    \end{minipage}

}%

\expandafter\date\expandafter{\@date\\[2cm]
    \abstractsmuggle
}%
\makeatother

%\includeonly{pgfplots.reference}


\begin{document}

\def\plotcoords{%
\addplot coordinates {
(5,8.312e-02)    (17,2.547e-02)   (49,7.407e-03)
(129,2.102e-03)  (321,5.874e-04)  (769,1.623e-04)
(1793,4.442e-05) (4097,1.207e-05) (9217,3.261e-06)
};

\addplot coordinates{
(7,8.472e-02)    (31,3.044e-02)    (111,1.022e-02)
(351,3.303e-03)  (1023,1.039e-03)  (2815,3.196e-04)
(7423,9.658e-05) (18943,2.873e-05) (47103,8.437e-06)
};

\addplot coordinates{
(9,7.881e-02)     (49,3.243e-02)    (209,1.232e-02)
(769,4.454e-03)   (2561,1.551e-03)  (7937,5.236e-04)
(23297,1.723e-04) (65537,5.545e-05) (178177,1.751e-05)
};

\addplot coordinates{
(11,6.887e-02)    (71,3.177e-02)     (351,1.341e-02)
(1471,5.334e-03)  (5503,2.027e-03)   (18943,7.415e-04)
(61183,2.628e-04) (187903,9.063e-05) (553983,3.053e-05)
};

\addplot coordinates{
(13,5.755e-02)     (97,2.925e-02)     (545,1.351e-02)
(2561,5.842e-03)   (10625,2.397e-03)  (40193,9.414e-04)
(141569,3.564e-04) (471041,1.308e-04)
(1496065,4.670e-05)
};
}%


\bookmaketitle
\tableofcontents

%%%%%%%%%%%%%%%%%%%%%%%%%%%%%%%%%%%%%%%%%%%%%%%%%%%%%%%%%%%%%%%%%%%%%%%%%%%%%
%
% Package pgfplots.sty documentation.
%
% Copyright 2007/2008 by Christian Feuersaenger.
%
% This program is free software: you can redistribute it and/or modify
% it under the terms of the GNU General Public License as published by
% the Free Software Foundation, either version 3 of the License, or
% (at your option) any later version.
%
% This program is distributed in the hope that it will be useful,
% but WITHOUT ANY WARRANTY; without even the implied warranty of
% MERCHANTABILITY or FITNESS FOR A PARTICULAR PURPOSE.  See the
% GNU General Public License for more details.
%
% You should have received a copy of the GNU General Public License
% along with this program.  If not, see <http://www.gnu.org/licenses/>.
%
%
%%%%%%%%%%%%%%%%%%%%%%%%%%%%%%%%%%%%%%%%%%%%%%%%%%%%%%%%%%%%%%%%%%%%%%%%%%%%%
% SEE pgfplots-macros.tex as well!
%\pdfminorversion=5 % to allow compression
%\pdfobjcompresslevel=2
\documentclass[a4paper,openany]{book}

\let\bookmaketitle=\maketitle

% -----------------------------------
% this here is from ltxdoc:
\usepackage{doc}[=v2]
\AtBeginDocument{\MakeShortVerb{\|}}
%\setlength{\textwidth}{355pt}
%\addtolength\marginparwidth{30pt}
%\addtolength\oddsidemargin{20pt}
%\addtolength\evensidemargin{20pt}
\def\cmd#1{\cs{\expandafter\cmd@to@cs\string#1}}
\def\cmd@to@cs#1#2{\char\number`#2\relax}
\DeclareRobustCommand\cs[1]{\texttt{\char`\\#1}}
\providecommand\marg[1]{%
  {\ttfamily\char`\{}\meta{#1}{\ttfamily\char`\}}}
\providecommand\oarg[1]{%
  {\ttfamily[}\meta{#1}{\ttfamily]}}
\providecommand\parg[1]{%
  {\ttfamily(}\meta{#1}{\ttfamily)}}
\raggedbottom

% -----------------------------------
\input{pgfplots.preamble.tex}

\makeatletter
% I want a two-column index just as in pgfmanual styles. This here
% was the best way to get one:
\def\index@prologue{\section*{Index}\addcontentsline{toc}{chapter}{Index}
}
\makeatother

%\RequirePackage[german,english,francais]{babel}

\def\matlabcolormaptext{This colormap is similar to one shipped with Matlab$^\text{\textregistered}$ under a similar name.}

\IfFileExists{tikzlibraryspy.code.tex}{%
\usetikzlibrary{spy}
}{%
    \message{ERROR: tikz SPY library NOT available. The manual will only compile partially.^^J}%
}%

\usepackage{xparse}% for colorbrewer manual

\usetikzlibrary{
    decorations.markings,
    decorations.footprints,
    shapes.arrows,
    matrix,
    positioning,
}

\usepgfplotslibrary{
    fillbetween,
    ternary,
    smithchart,
    patchplots,
    polar,
    colormaps,
    colorbrewer,
    colortol,           % docu not ready yet
}
\pgfqkeys{/codeexample}{%
    every codeexample/.append style={
        /pgfplots/every ternary axis/.append style={
            /pgfplots/legend style={fill=graphicbackground},
        }
    },
    tabsize=4,
}

\pgfplotsmanualenableexternalizationofexpensive

%\usetikzlibrary{external}
%\tikzexternalize[prefix=figures/]{pgfplots}

\title{%
    Manual for Package \PGFPlots{}\\
    {\small 2D/3D Plots in \LaTeX{}, Version \pgfplotsversion}\\
    {\small\url{https://github.com/pgf-tikz/pgfplots}}
    %\\{\small Attention: you are using an unstable development version.}
}

\makeatletter
\long\def\abstractsmuggle{%
    \centering
    \textbf{Abstract}\\[0.5cm]

    \begin{minipage}{12cm}
        \PGFPlots{} draws high-quality function plots in normal or logarithmic
        scaling with a user-friendly interface directly in \TeX{}. The user
        supplies axis labels, legend entries and the plot coordinates for one or
        more plots and \PGFPlots{} applies axis scaling, computes any logarithms
        and axis ticks and draws the plots. It supports line plots, scatter
        plots, piecewise constant plots, bar plots, area plots, mesh and surface
        plots, patch plots, contour plots, quiver plots, histogram plots, box
        plots, polar axes, ternary diagrams, smith charts and some more. It is
        based on Till Tantau's package \PGF{}/\Tikz{}.
    \end{minipage}

}%

\expandafter\date\expandafter{\@date\\[2cm]
    \abstractsmuggle
}%
\makeatother

%\includeonly{pgfplots.reference}


\begin{document}

\def\plotcoords{%
\addplot coordinates {
(5,8.312e-02)    (17,2.547e-02)   (49,7.407e-03)
(129,2.102e-03)  (321,5.874e-04)  (769,1.623e-04)
(1793,4.442e-05) (4097,1.207e-05) (9217,3.261e-06)
};

\addplot coordinates{
(7,8.472e-02)    (31,3.044e-02)    (111,1.022e-02)
(351,3.303e-03)  (1023,1.039e-03)  (2815,3.196e-04)
(7423,9.658e-05) (18943,2.873e-05) (47103,8.437e-06)
};

\addplot coordinates{
(9,7.881e-02)     (49,3.243e-02)    (209,1.232e-02)
(769,4.454e-03)   (2561,1.551e-03)  (7937,5.236e-04)
(23297,1.723e-04) (65537,5.545e-05) (178177,1.751e-05)
};

\addplot coordinates{
(11,6.887e-02)    (71,3.177e-02)     (351,1.341e-02)
(1471,5.334e-03)  (5503,2.027e-03)   (18943,7.415e-04)
(61183,2.628e-04) (187903,9.063e-05) (553983,3.053e-05)
};

\addplot coordinates{
(13,5.755e-02)     (97,2.925e-02)     (545,1.351e-02)
(2561,5.842e-03)   (10625,2.397e-03)  (40193,9.414e-04)
(141569,3.564e-04) (471041,1.308e-04)
(1496065,4.670e-05)
};
}%


\bookmaketitle
\tableofcontents
\include{pgfplots.title_abstract_intro}
\include{pgfplots.preliminaries}
\include{pgfplots.intro}
\include{pgfplots.reference}
\include{pgfplots.libs}
\include{pgfplots.resources}
\include{pgfplots.importexport}
\include{pgfplots.basic.reference}

\printindex

\bibliographystyle{abbrv} %gerapali} %gerabbrv} %gerunsrt.bst} %gerabbrv}% gerplain}
\nocite{pgfplotstable}
\nocite{programmingnotes}
\bibliography{pgfplots}
\end{document}

% -----------------------------------------------------------------------------
% For Stefan Pinnow as reminder on what to look for when editing the manual
% -----------------------------------------------------------------------------
% There should be no line breaks in the following environments
% - |...|
% - \declareandlabel{...}
% - \verbpdfref{...}
% If "MakeTikzPictures" isn't running through check one of the externalized
% LOG files what is the cause of that.
% -----------------------------------------------------------------------------


%%%%%%%%%%%%%%%%%%%%%%%%%%%%%%%%%%%%%%%%%%%%%%%%%%%%%%%%%%%%%%%%%%%%%%%%%%%%%
%
% Package pgfplots.sty documentation.
%
% Copyright 2007/2008 by Christian Feuersaenger.
%
% This program is free software: you can redistribute it and/or modify
% it under the terms of the GNU General Public License as published by
% the Free Software Foundation, either version 3 of the License, or
% (at your option) any later version.
%
% This program is distributed in the hope that it will be useful,
% but WITHOUT ANY WARRANTY; without even the implied warranty of
% MERCHANTABILITY or FITNESS FOR A PARTICULAR PURPOSE.  See the
% GNU General Public License for more details.
%
% You should have received a copy of the GNU General Public License
% along with this program.  If not, see <http://www.gnu.org/licenses/>.
%
%
%%%%%%%%%%%%%%%%%%%%%%%%%%%%%%%%%%%%%%%%%%%%%%%%%%%%%%%%%%%%%%%%%%%%%%%%%%%%%
% SEE pgfplots-macros.tex as well!
%\pdfminorversion=5 % to allow compression
%\pdfobjcompresslevel=2
\documentclass[a4paper,openany]{book}

\let\bookmaketitle=\maketitle

% -----------------------------------
% this here is from ltxdoc:
\usepackage{doc}[=v2]
\AtBeginDocument{\MakeShortVerb{\|}}
%\setlength{\textwidth}{355pt}
%\addtolength\marginparwidth{30pt}
%\addtolength\oddsidemargin{20pt}
%\addtolength\evensidemargin{20pt}
\def\cmd#1{\cs{\expandafter\cmd@to@cs\string#1}}
\def\cmd@to@cs#1#2{\char\number`#2\relax}
\DeclareRobustCommand\cs[1]{\texttt{\char`\\#1}}
\providecommand\marg[1]{%
  {\ttfamily\char`\{}\meta{#1}{\ttfamily\char`\}}}
\providecommand\oarg[1]{%
  {\ttfamily[}\meta{#1}{\ttfamily]}}
\providecommand\parg[1]{%
  {\ttfamily(}\meta{#1}{\ttfamily)}}
\raggedbottom

% -----------------------------------
\input{pgfplots.preamble.tex}

\makeatletter
% I want a two-column index just as in pgfmanual styles. This here
% was the best way to get one:
\def\index@prologue{\section*{Index}\addcontentsline{toc}{chapter}{Index}
}
\makeatother

%\RequirePackage[german,english,francais]{babel}

\def\matlabcolormaptext{This colormap is similar to one shipped with Matlab$^\text{\textregistered}$ under a similar name.}

\IfFileExists{tikzlibraryspy.code.tex}{%
\usetikzlibrary{spy}
}{%
    \message{ERROR: tikz SPY library NOT available. The manual will only compile partially.^^J}%
}%

\usepackage{xparse}% for colorbrewer manual

\usetikzlibrary{
    decorations.markings,
    decorations.footprints,
    shapes.arrows,
    matrix,
    positioning,
}

\usepgfplotslibrary{
    fillbetween,
    ternary,
    smithchart,
    patchplots,
    polar,
    colormaps,
    colorbrewer,
    colortol,           % docu not ready yet
}
\pgfqkeys{/codeexample}{%
    every codeexample/.append style={
        /pgfplots/every ternary axis/.append style={
            /pgfplots/legend style={fill=graphicbackground},
        }
    },
    tabsize=4,
}

\pgfplotsmanualenableexternalizationofexpensive

%\usetikzlibrary{external}
%\tikzexternalize[prefix=figures/]{pgfplots}

\title{%
    Manual for Package \PGFPlots{}\\
    {\small 2D/3D Plots in \LaTeX{}, Version \pgfplotsversion}\\
    {\small\url{https://github.com/pgf-tikz/pgfplots}}
    %\\{\small Attention: you are using an unstable development version.}
}

\makeatletter
\long\def\abstractsmuggle{%
    \centering
    \textbf{Abstract}\\[0.5cm]

    \begin{minipage}{12cm}
        \PGFPlots{} draws high-quality function plots in normal or logarithmic
        scaling with a user-friendly interface directly in \TeX{}. The user
        supplies axis labels, legend entries and the plot coordinates for one or
        more plots and \PGFPlots{} applies axis scaling, computes any logarithms
        and axis ticks and draws the plots. It supports line plots, scatter
        plots, piecewise constant plots, bar plots, area plots, mesh and surface
        plots, patch plots, contour plots, quiver plots, histogram plots, box
        plots, polar axes, ternary diagrams, smith charts and some more. It is
        based on Till Tantau's package \PGF{}/\Tikz{}.
    \end{minipage}

}%

\expandafter\date\expandafter{\@date\\[2cm]
    \abstractsmuggle
}%
\makeatother

%\includeonly{pgfplots.reference}


\begin{document}

\def\plotcoords{%
\addplot coordinates {
(5,8.312e-02)    (17,2.547e-02)   (49,7.407e-03)
(129,2.102e-03)  (321,5.874e-04)  (769,1.623e-04)
(1793,4.442e-05) (4097,1.207e-05) (9217,3.261e-06)
};

\addplot coordinates{
(7,8.472e-02)    (31,3.044e-02)    (111,1.022e-02)
(351,3.303e-03)  (1023,1.039e-03)  (2815,3.196e-04)
(7423,9.658e-05) (18943,2.873e-05) (47103,8.437e-06)
};

\addplot coordinates{
(9,7.881e-02)     (49,3.243e-02)    (209,1.232e-02)
(769,4.454e-03)   (2561,1.551e-03)  (7937,5.236e-04)
(23297,1.723e-04) (65537,5.545e-05) (178177,1.751e-05)
};

\addplot coordinates{
(11,6.887e-02)    (71,3.177e-02)     (351,1.341e-02)
(1471,5.334e-03)  (5503,2.027e-03)   (18943,7.415e-04)
(61183,2.628e-04) (187903,9.063e-05) (553983,3.053e-05)
};

\addplot coordinates{
(13,5.755e-02)     (97,2.925e-02)     (545,1.351e-02)
(2561,5.842e-03)   (10625,2.397e-03)  (40193,9.414e-04)
(141569,3.564e-04) (471041,1.308e-04)
(1496065,4.670e-05)
};
}%


\bookmaketitle
\tableofcontents
\include{pgfplots.title_abstract_intro}
\include{pgfplots.preliminaries}
\include{pgfplots.intro}
\include{pgfplots.reference}
\include{pgfplots.libs}
\include{pgfplots.resources}
\include{pgfplots.importexport}
\include{pgfplots.basic.reference}

\printindex

\bibliographystyle{abbrv} %gerapali} %gerabbrv} %gerunsrt.bst} %gerabbrv}% gerplain}
\nocite{pgfplotstable}
\nocite{programmingnotes}
\bibliography{pgfplots}
\end{document}

% -----------------------------------------------------------------------------
% For Stefan Pinnow as reminder on what to look for when editing the manual
% -----------------------------------------------------------------------------
% There should be no line breaks in the following environments
% - |...|
% - \declareandlabel{...}
% - \verbpdfref{...}
% If "MakeTikzPictures" isn't running through check one of the externalized
% LOG files what is the cause of that.
% -----------------------------------------------------------------------------


%%%%%%%%%%%%%%%%%%%%%%%%%%%%%%%%%%%%%%%%%%%%%%%%%%%%%%%%%%%%%%%%%%%%%%%%%%%%%
%
% Package pgfplots.sty documentation.
%
% Copyright 2007/2008 by Christian Feuersaenger.
%
% This program is free software: you can redistribute it and/or modify
% it under the terms of the GNU General Public License as published by
% the Free Software Foundation, either version 3 of the License, or
% (at your option) any later version.
%
% This program is distributed in the hope that it will be useful,
% but WITHOUT ANY WARRANTY; without even the implied warranty of
% MERCHANTABILITY or FITNESS FOR A PARTICULAR PURPOSE.  See the
% GNU General Public License for more details.
%
% You should have received a copy of the GNU General Public License
% along with this program.  If not, see <http://www.gnu.org/licenses/>.
%
%
%%%%%%%%%%%%%%%%%%%%%%%%%%%%%%%%%%%%%%%%%%%%%%%%%%%%%%%%%%%%%%%%%%%%%%%%%%%%%
% SEE pgfplots-macros.tex as well!
%\pdfminorversion=5 % to allow compression
%\pdfobjcompresslevel=2
\documentclass[a4paper,openany]{book}

\let\bookmaketitle=\maketitle

% -----------------------------------
% this here is from ltxdoc:
\usepackage{doc}[=v2]
\AtBeginDocument{\MakeShortVerb{\|}}
%\setlength{\textwidth}{355pt}
%\addtolength\marginparwidth{30pt}
%\addtolength\oddsidemargin{20pt}
%\addtolength\evensidemargin{20pt}
\def\cmd#1{\cs{\expandafter\cmd@to@cs\string#1}}
\def\cmd@to@cs#1#2{\char\number`#2\relax}
\DeclareRobustCommand\cs[1]{\texttt{\char`\\#1}}
\providecommand\marg[1]{%
  {\ttfamily\char`\{}\meta{#1}{\ttfamily\char`\}}}
\providecommand\oarg[1]{%
  {\ttfamily[}\meta{#1}{\ttfamily]}}
\providecommand\parg[1]{%
  {\ttfamily(}\meta{#1}{\ttfamily)}}
\raggedbottom

% -----------------------------------
\input{pgfplots.preamble.tex}

\makeatletter
% I want a two-column index just as in pgfmanual styles. This here
% was the best way to get one:
\def\index@prologue{\section*{Index}\addcontentsline{toc}{chapter}{Index}
}
\makeatother

%\RequirePackage[german,english,francais]{babel}

\def\matlabcolormaptext{This colormap is similar to one shipped with Matlab$^\text{\textregistered}$ under a similar name.}

\IfFileExists{tikzlibraryspy.code.tex}{%
\usetikzlibrary{spy}
}{%
    \message{ERROR: tikz SPY library NOT available. The manual will only compile partially.^^J}%
}%

\usepackage{xparse}% for colorbrewer manual

\usetikzlibrary{
    decorations.markings,
    decorations.footprints,
    shapes.arrows,
    matrix,
    positioning,
}

\usepgfplotslibrary{
    fillbetween,
    ternary,
    smithchart,
    patchplots,
    polar,
    colormaps,
    colorbrewer,
    colortol,           % docu not ready yet
}
\pgfqkeys{/codeexample}{%
    every codeexample/.append style={
        /pgfplots/every ternary axis/.append style={
            /pgfplots/legend style={fill=graphicbackground},
        }
    },
    tabsize=4,
}

\pgfplotsmanualenableexternalizationofexpensive

%\usetikzlibrary{external}
%\tikzexternalize[prefix=figures/]{pgfplots}

\title{%
    Manual for Package \PGFPlots{}\\
    {\small 2D/3D Plots in \LaTeX{}, Version \pgfplotsversion}\\
    {\small\url{https://github.com/pgf-tikz/pgfplots}}
    %\\{\small Attention: you are using an unstable development version.}
}

\makeatletter
\long\def\abstractsmuggle{%
    \centering
    \textbf{Abstract}\\[0.5cm]

    \begin{minipage}{12cm}
        \PGFPlots{} draws high-quality function plots in normal or logarithmic
        scaling with a user-friendly interface directly in \TeX{}. The user
        supplies axis labels, legend entries and the plot coordinates for one or
        more plots and \PGFPlots{} applies axis scaling, computes any logarithms
        and axis ticks and draws the plots. It supports line plots, scatter
        plots, piecewise constant plots, bar plots, area plots, mesh and surface
        plots, patch plots, contour plots, quiver plots, histogram plots, box
        plots, polar axes, ternary diagrams, smith charts and some more. It is
        based on Till Tantau's package \PGF{}/\Tikz{}.
    \end{minipage}

}%

\expandafter\date\expandafter{\@date\\[2cm]
    \abstractsmuggle
}%
\makeatother

%\includeonly{pgfplots.reference}


\begin{document}

\def\plotcoords{%
\addplot coordinates {
(5,8.312e-02)    (17,2.547e-02)   (49,7.407e-03)
(129,2.102e-03)  (321,5.874e-04)  (769,1.623e-04)
(1793,4.442e-05) (4097,1.207e-05) (9217,3.261e-06)
};

\addplot coordinates{
(7,8.472e-02)    (31,3.044e-02)    (111,1.022e-02)
(351,3.303e-03)  (1023,1.039e-03)  (2815,3.196e-04)
(7423,9.658e-05) (18943,2.873e-05) (47103,8.437e-06)
};

\addplot coordinates{
(9,7.881e-02)     (49,3.243e-02)    (209,1.232e-02)
(769,4.454e-03)   (2561,1.551e-03)  (7937,5.236e-04)
(23297,1.723e-04) (65537,5.545e-05) (178177,1.751e-05)
};

\addplot coordinates{
(11,6.887e-02)    (71,3.177e-02)     (351,1.341e-02)
(1471,5.334e-03)  (5503,2.027e-03)   (18943,7.415e-04)
(61183,2.628e-04) (187903,9.063e-05) (553983,3.053e-05)
};

\addplot coordinates{
(13,5.755e-02)     (97,2.925e-02)     (545,1.351e-02)
(2561,5.842e-03)   (10625,2.397e-03)  (40193,9.414e-04)
(141569,3.564e-04) (471041,1.308e-04)
(1496065,4.670e-05)
};
}%


\bookmaketitle
\tableofcontents
\include{pgfplots.title_abstract_intro}
\include{pgfplots.preliminaries}
\include{pgfplots.intro}
\include{pgfplots.reference}
\include{pgfplots.libs}
\include{pgfplots.resources}
\include{pgfplots.importexport}
\include{pgfplots.basic.reference}

\printindex

\bibliographystyle{abbrv} %gerapali} %gerabbrv} %gerunsrt.bst} %gerabbrv}% gerplain}
\nocite{pgfplotstable}
\nocite{programmingnotes}
\bibliography{pgfplots}
\end{document}

% -----------------------------------------------------------------------------
% For Stefan Pinnow as reminder on what to look for when editing the manual
% -----------------------------------------------------------------------------
% There should be no line breaks in the following environments
% - |...|
% - \declareandlabel{...}
% - \verbpdfref{...}
% If "MakeTikzPictures" isn't running through check one of the externalized
% LOG files what is the cause of that.
% -----------------------------------------------------------------------------


%%%%%%%%%%%%%%%%%%%%%%%%%%%%%%%%%%%%%%%%%%%%%%%%%%%%%%%%%%%%%%%%%%%%%%%%%%%%%
%
% Package pgfplots.sty documentation.
%
% Copyright 2007/2008 by Christian Feuersaenger.
%
% This program is free software: you can redistribute it and/or modify
% it under the terms of the GNU General Public License as published by
% the Free Software Foundation, either version 3 of the License, or
% (at your option) any later version.
%
% This program is distributed in the hope that it will be useful,
% but WITHOUT ANY WARRANTY; without even the implied warranty of
% MERCHANTABILITY or FITNESS FOR A PARTICULAR PURPOSE.  See the
% GNU General Public License for more details.
%
% You should have received a copy of the GNU General Public License
% along with this program.  If not, see <http://www.gnu.org/licenses/>.
%
%
%%%%%%%%%%%%%%%%%%%%%%%%%%%%%%%%%%%%%%%%%%%%%%%%%%%%%%%%%%%%%%%%%%%%%%%%%%%%%
% SEE pgfplots-macros.tex as well!
%\pdfminorversion=5 % to allow compression
%\pdfobjcompresslevel=2
\documentclass[a4paper,openany]{book}

\let\bookmaketitle=\maketitle

% -----------------------------------
% this here is from ltxdoc:
\usepackage{doc}[=v2]
\AtBeginDocument{\MakeShortVerb{\|}}
%\setlength{\textwidth}{355pt}
%\addtolength\marginparwidth{30pt}
%\addtolength\oddsidemargin{20pt}
%\addtolength\evensidemargin{20pt}
\def\cmd#1{\cs{\expandafter\cmd@to@cs\string#1}}
\def\cmd@to@cs#1#2{\char\number`#2\relax}
\DeclareRobustCommand\cs[1]{\texttt{\char`\\#1}}
\providecommand\marg[1]{%
  {\ttfamily\char`\{}\meta{#1}{\ttfamily\char`\}}}
\providecommand\oarg[1]{%
  {\ttfamily[}\meta{#1}{\ttfamily]}}
\providecommand\parg[1]{%
  {\ttfamily(}\meta{#1}{\ttfamily)}}
\raggedbottom

% -----------------------------------
\input{pgfplots.preamble.tex}

\makeatletter
% I want a two-column index just as in pgfmanual styles. This here
% was the best way to get one:
\def\index@prologue{\section*{Index}\addcontentsline{toc}{chapter}{Index}
}
\makeatother

%\RequirePackage[german,english,francais]{babel}

\def\matlabcolormaptext{This colormap is similar to one shipped with Matlab$^\text{\textregistered}$ under a similar name.}

\IfFileExists{tikzlibraryspy.code.tex}{%
\usetikzlibrary{spy}
}{%
    \message{ERROR: tikz SPY library NOT available. The manual will only compile partially.^^J}%
}%

\usepackage{xparse}% for colorbrewer manual

\usetikzlibrary{
    decorations.markings,
    decorations.footprints,
    shapes.arrows,
    matrix,
    positioning,
}

\usepgfplotslibrary{
    fillbetween,
    ternary,
    smithchart,
    patchplots,
    polar,
    colormaps,
    colorbrewer,
    colortol,           % docu not ready yet
}
\pgfqkeys{/codeexample}{%
    every codeexample/.append style={
        /pgfplots/every ternary axis/.append style={
            /pgfplots/legend style={fill=graphicbackground},
        }
    },
    tabsize=4,
}

\pgfplotsmanualenableexternalizationofexpensive

%\usetikzlibrary{external}
%\tikzexternalize[prefix=figures/]{pgfplots}

\title{%
    Manual for Package \PGFPlots{}\\
    {\small 2D/3D Plots in \LaTeX{}, Version \pgfplotsversion}\\
    {\small\url{https://github.com/pgf-tikz/pgfplots}}
    %\\{\small Attention: you are using an unstable development version.}
}

\makeatletter
\long\def\abstractsmuggle{%
    \centering
    \textbf{Abstract}\\[0.5cm]

    \begin{minipage}{12cm}
        \PGFPlots{} draws high-quality function plots in normal or logarithmic
        scaling with a user-friendly interface directly in \TeX{}. The user
        supplies axis labels, legend entries and the plot coordinates for one or
        more plots and \PGFPlots{} applies axis scaling, computes any logarithms
        and axis ticks and draws the plots. It supports line plots, scatter
        plots, piecewise constant plots, bar plots, area plots, mesh and surface
        plots, patch plots, contour plots, quiver plots, histogram plots, box
        plots, polar axes, ternary diagrams, smith charts and some more. It is
        based on Till Tantau's package \PGF{}/\Tikz{}.
    \end{minipage}

}%

\expandafter\date\expandafter{\@date\\[2cm]
    \abstractsmuggle
}%
\makeatother

%\includeonly{pgfplots.reference}


\begin{document}

\def\plotcoords{%
\addplot coordinates {
(5,8.312e-02)    (17,2.547e-02)   (49,7.407e-03)
(129,2.102e-03)  (321,5.874e-04)  (769,1.623e-04)
(1793,4.442e-05) (4097,1.207e-05) (9217,3.261e-06)
};

\addplot coordinates{
(7,8.472e-02)    (31,3.044e-02)    (111,1.022e-02)
(351,3.303e-03)  (1023,1.039e-03)  (2815,3.196e-04)
(7423,9.658e-05) (18943,2.873e-05) (47103,8.437e-06)
};

\addplot coordinates{
(9,7.881e-02)     (49,3.243e-02)    (209,1.232e-02)
(769,4.454e-03)   (2561,1.551e-03)  (7937,5.236e-04)
(23297,1.723e-04) (65537,5.545e-05) (178177,1.751e-05)
};

\addplot coordinates{
(11,6.887e-02)    (71,3.177e-02)     (351,1.341e-02)
(1471,5.334e-03)  (5503,2.027e-03)   (18943,7.415e-04)
(61183,2.628e-04) (187903,9.063e-05) (553983,3.053e-05)
};

\addplot coordinates{
(13,5.755e-02)     (97,2.925e-02)     (545,1.351e-02)
(2561,5.842e-03)   (10625,2.397e-03)  (40193,9.414e-04)
(141569,3.564e-04) (471041,1.308e-04)
(1496065,4.670e-05)
};
}%


\bookmaketitle
\tableofcontents
\include{pgfplots.title_abstract_intro}
\include{pgfplots.preliminaries}
\include{pgfplots.intro}
\include{pgfplots.reference}
\include{pgfplots.libs}
\include{pgfplots.resources}
\include{pgfplots.importexport}
\include{pgfplots.basic.reference}

\printindex

\bibliographystyle{abbrv} %gerapali} %gerabbrv} %gerunsrt.bst} %gerabbrv}% gerplain}
\nocite{pgfplotstable}
\nocite{programmingnotes}
\bibliography{pgfplots}
\end{document}

% -----------------------------------------------------------------------------
% For Stefan Pinnow as reminder on what to look for when editing the manual
% -----------------------------------------------------------------------------
% There should be no line breaks in the following environments
% - |...|
% - \declareandlabel{...}
% - \verbpdfref{...}
% If "MakeTikzPictures" isn't running through check one of the externalized
% LOG files what is the cause of that.
% -----------------------------------------------------------------------------


%%%%%%%%%%%%%%%%%%%%%%%%%%%%%%%%%%%%%%%%%%%%%%%%%%%%%%%%%%%%%%%%%%%%%%%%%%%%%
%
% Package pgfplots.sty documentation.
%
% Copyright 2007/2008 by Christian Feuersaenger.
%
% This program is free software: you can redistribute it and/or modify
% it under the terms of the GNU General Public License as published by
% the Free Software Foundation, either version 3 of the License, or
% (at your option) any later version.
%
% This program is distributed in the hope that it will be useful,
% but WITHOUT ANY WARRANTY; without even the implied warranty of
% MERCHANTABILITY or FITNESS FOR A PARTICULAR PURPOSE.  See the
% GNU General Public License for more details.
%
% You should have received a copy of the GNU General Public License
% along with this program.  If not, see <http://www.gnu.org/licenses/>.
%
%
%%%%%%%%%%%%%%%%%%%%%%%%%%%%%%%%%%%%%%%%%%%%%%%%%%%%%%%%%%%%%%%%%%%%%%%%%%%%%
% SEE pgfplots-macros.tex as well!
%\pdfminorversion=5 % to allow compression
%\pdfobjcompresslevel=2
\documentclass[a4paper,openany]{book}

\let\bookmaketitle=\maketitle

% -----------------------------------
% this here is from ltxdoc:
\usepackage{doc}[=v2]
\AtBeginDocument{\MakeShortVerb{\|}}
%\setlength{\textwidth}{355pt}
%\addtolength\marginparwidth{30pt}
%\addtolength\oddsidemargin{20pt}
%\addtolength\evensidemargin{20pt}
\def\cmd#1{\cs{\expandafter\cmd@to@cs\string#1}}
\def\cmd@to@cs#1#2{\char\number`#2\relax}
\DeclareRobustCommand\cs[1]{\texttt{\char`\\#1}}
\providecommand\marg[1]{%
  {\ttfamily\char`\{}\meta{#1}{\ttfamily\char`\}}}
\providecommand\oarg[1]{%
  {\ttfamily[}\meta{#1}{\ttfamily]}}
\providecommand\parg[1]{%
  {\ttfamily(}\meta{#1}{\ttfamily)}}
\raggedbottom

% -----------------------------------
\input{pgfplots.preamble.tex}

\makeatletter
% I want a two-column index just as in pgfmanual styles. This here
% was the best way to get one:
\def\index@prologue{\section*{Index}\addcontentsline{toc}{chapter}{Index}
}
\makeatother

%\RequirePackage[german,english,francais]{babel}

\def\matlabcolormaptext{This colormap is similar to one shipped with Matlab$^\text{\textregistered}$ under a similar name.}

\IfFileExists{tikzlibraryspy.code.tex}{%
\usetikzlibrary{spy}
}{%
    \message{ERROR: tikz SPY library NOT available. The manual will only compile partially.^^J}%
}%

\usepackage{xparse}% for colorbrewer manual

\usetikzlibrary{
    decorations.markings,
    decorations.footprints,
    shapes.arrows,
    matrix,
    positioning,
}

\usepgfplotslibrary{
    fillbetween,
    ternary,
    smithchart,
    patchplots,
    polar,
    colormaps,
    colorbrewer,
    colortol,           % docu not ready yet
}
\pgfqkeys{/codeexample}{%
    every codeexample/.append style={
        /pgfplots/every ternary axis/.append style={
            /pgfplots/legend style={fill=graphicbackground},
        }
    },
    tabsize=4,
}

\pgfplotsmanualenableexternalizationofexpensive

%\usetikzlibrary{external}
%\tikzexternalize[prefix=figures/]{pgfplots}

\title{%
    Manual for Package \PGFPlots{}\\
    {\small 2D/3D Plots in \LaTeX{}, Version \pgfplotsversion}\\
    {\small\url{https://github.com/pgf-tikz/pgfplots}}
    %\\{\small Attention: you are using an unstable development version.}
}

\makeatletter
\long\def\abstractsmuggle{%
    \centering
    \textbf{Abstract}\\[0.5cm]

    \begin{minipage}{12cm}
        \PGFPlots{} draws high-quality function plots in normal or logarithmic
        scaling with a user-friendly interface directly in \TeX{}. The user
        supplies axis labels, legend entries and the plot coordinates for one or
        more plots and \PGFPlots{} applies axis scaling, computes any logarithms
        and axis ticks and draws the plots. It supports line plots, scatter
        plots, piecewise constant plots, bar plots, area plots, mesh and surface
        plots, patch plots, contour plots, quiver plots, histogram plots, box
        plots, polar axes, ternary diagrams, smith charts and some more. It is
        based on Till Tantau's package \PGF{}/\Tikz{}.
    \end{minipage}

}%

\expandafter\date\expandafter{\@date\\[2cm]
    \abstractsmuggle
}%
\makeatother

%\includeonly{pgfplots.reference}


\begin{document}

\def\plotcoords{%
\addplot coordinates {
(5,8.312e-02)    (17,2.547e-02)   (49,7.407e-03)
(129,2.102e-03)  (321,5.874e-04)  (769,1.623e-04)
(1793,4.442e-05) (4097,1.207e-05) (9217,3.261e-06)
};

\addplot coordinates{
(7,8.472e-02)    (31,3.044e-02)    (111,1.022e-02)
(351,3.303e-03)  (1023,1.039e-03)  (2815,3.196e-04)
(7423,9.658e-05) (18943,2.873e-05) (47103,8.437e-06)
};

\addplot coordinates{
(9,7.881e-02)     (49,3.243e-02)    (209,1.232e-02)
(769,4.454e-03)   (2561,1.551e-03)  (7937,5.236e-04)
(23297,1.723e-04) (65537,5.545e-05) (178177,1.751e-05)
};

\addplot coordinates{
(11,6.887e-02)    (71,3.177e-02)     (351,1.341e-02)
(1471,5.334e-03)  (5503,2.027e-03)   (18943,7.415e-04)
(61183,2.628e-04) (187903,9.063e-05) (553983,3.053e-05)
};

\addplot coordinates{
(13,5.755e-02)     (97,2.925e-02)     (545,1.351e-02)
(2561,5.842e-03)   (10625,2.397e-03)  (40193,9.414e-04)
(141569,3.564e-04) (471041,1.308e-04)
(1496065,4.670e-05)
};
}%


\bookmaketitle
\tableofcontents
\include{pgfplots.title_abstract_intro}
\include{pgfplots.preliminaries}
\include{pgfplots.intro}
\include{pgfplots.reference}
\include{pgfplots.libs}
\include{pgfplots.resources}
\include{pgfplots.importexport}
\include{pgfplots.basic.reference}

\printindex

\bibliographystyle{abbrv} %gerapali} %gerabbrv} %gerunsrt.bst} %gerabbrv}% gerplain}
\nocite{pgfplotstable}
\nocite{programmingnotes}
\bibliography{pgfplots}
\end{document}

% -----------------------------------------------------------------------------
% For Stefan Pinnow as reminder on what to look for when editing the manual
% -----------------------------------------------------------------------------
% There should be no line breaks in the following environments
% - |...|
% - \declareandlabel{...}
% - \verbpdfref{...}
% If "MakeTikzPictures" isn't running through check one of the externalized
% LOG files what is the cause of that.
% -----------------------------------------------------------------------------


%%%%%%%%%%%%%%%%%%%%%%%%%%%%%%%%%%%%%%%%%%%%%%%%%%%%%%%%%%%%%%%%%%%%%%%%%%%%%
%
% Package pgfplots.sty documentation.
%
% Copyright 2007/2008 by Christian Feuersaenger.
%
% This program is free software: you can redistribute it and/or modify
% it under the terms of the GNU General Public License as published by
% the Free Software Foundation, either version 3 of the License, or
% (at your option) any later version.
%
% This program is distributed in the hope that it will be useful,
% but WITHOUT ANY WARRANTY; without even the implied warranty of
% MERCHANTABILITY or FITNESS FOR A PARTICULAR PURPOSE.  See the
% GNU General Public License for more details.
%
% You should have received a copy of the GNU General Public License
% along with this program.  If not, see <http://www.gnu.org/licenses/>.
%
%
%%%%%%%%%%%%%%%%%%%%%%%%%%%%%%%%%%%%%%%%%%%%%%%%%%%%%%%%%%%%%%%%%%%%%%%%%%%%%
% SEE pgfplots-macros.tex as well!
%\pdfminorversion=5 % to allow compression
%\pdfobjcompresslevel=2
\documentclass[a4paper,openany]{book}

\let\bookmaketitle=\maketitle

% -----------------------------------
% this here is from ltxdoc:
\usepackage{doc}[=v2]
\AtBeginDocument{\MakeShortVerb{\|}}
%\setlength{\textwidth}{355pt}
%\addtolength\marginparwidth{30pt}
%\addtolength\oddsidemargin{20pt}
%\addtolength\evensidemargin{20pt}
\def\cmd#1{\cs{\expandafter\cmd@to@cs\string#1}}
\def\cmd@to@cs#1#2{\char\number`#2\relax}
\DeclareRobustCommand\cs[1]{\texttt{\char`\\#1}}
\providecommand\marg[1]{%
  {\ttfamily\char`\{}\meta{#1}{\ttfamily\char`\}}}
\providecommand\oarg[1]{%
  {\ttfamily[}\meta{#1}{\ttfamily]}}
\providecommand\parg[1]{%
  {\ttfamily(}\meta{#1}{\ttfamily)}}
\raggedbottom

% -----------------------------------
\input{pgfplots.preamble.tex}

\makeatletter
% I want a two-column index just as in pgfmanual styles. This here
% was the best way to get one:
\def\index@prologue{\section*{Index}\addcontentsline{toc}{chapter}{Index}
}
\makeatother

%\RequirePackage[german,english,francais]{babel}

\def\matlabcolormaptext{This colormap is similar to one shipped with Matlab$^\text{\textregistered}$ under a similar name.}

\IfFileExists{tikzlibraryspy.code.tex}{%
\usetikzlibrary{spy}
}{%
    \message{ERROR: tikz SPY library NOT available. The manual will only compile partially.^^J}%
}%

\usepackage{xparse}% for colorbrewer manual

\usetikzlibrary{
    decorations.markings,
    decorations.footprints,
    shapes.arrows,
    matrix,
    positioning,
}

\usepgfplotslibrary{
    fillbetween,
    ternary,
    smithchart,
    patchplots,
    polar,
    colormaps,
    colorbrewer,
    colortol,           % docu not ready yet
}
\pgfqkeys{/codeexample}{%
    every codeexample/.append style={
        /pgfplots/every ternary axis/.append style={
            /pgfplots/legend style={fill=graphicbackground},
        }
    },
    tabsize=4,
}

\pgfplotsmanualenableexternalizationofexpensive

%\usetikzlibrary{external}
%\tikzexternalize[prefix=figures/]{pgfplots}

\title{%
    Manual for Package \PGFPlots{}\\
    {\small 2D/3D Plots in \LaTeX{}, Version \pgfplotsversion}\\
    {\small\url{https://github.com/pgf-tikz/pgfplots}}
    %\\{\small Attention: you are using an unstable development version.}
}

\makeatletter
\long\def\abstractsmuggle{%
    \centering
    \textbf{Abstract}\\[0.5cm]

    \begin{minipage}{12cm}
        \PGFPlots{} draws high-quality function plots in normal or logarithmic
        scaling with a user-friendly interface directly in \TeX{}. The user
        supplies axis labels, legend entries and the plot coordinates for one or
        more plots and \PGFPlots{} applies axis scaling, computes any logarithms
        and axis ticks and draws the plots. It supports line plots, scatter
        plots, piecewise constant plots, bar plots, area plots, mesh and surface
        plots, patch plots, contour plots, quiver plots, histogram plots, box
        plots, polar axes, ternary diagrams, smith charts and some more. It is
        based on Till Tantau's package \PGF{}/\Tikz{}.
    \end{minipage}

}%

\expandafter\date\expandafter{\@date\\[2cm]
    \abstractsmuggle
}%
\makeatother

%\includeonly{pgfplots.reference}


\begin{document}

\def\plotcoords{%
\addplot coordinates {
(5,8.312e-02)    (17,2.547e-02)   (49,7.407e-03)
(129,2.102e-03)  (321,5.874e-04)  (769,1.623e-04)
(1793,4.442e-05) (4097,1.207e-05) (9217,3.261e-06)
};

\addplot coordinates{
(7,8.472e-02)    (31,3.044e-02)    (111,1.022e-02)
(351,3.303e-03)  (1023,1.039e-03)  (2815,3.196e-04)
(7423,9.658e-05) (18943,2.873e-05) (47103,8.437e-06)
};

\addplot coordinates{
(9,7.881e-02)     (49,3.243e-02)    (209,1.232e-02)
(769,4.454e-03)   (2561,1.551e-03)  (7937,5.236e-04)
(23297,1.723e-04) (65537,5.545e-05) (178177,1.751e-05)
};

\addplot coordinates{
(11,6.887e-02)    (71,3.177e-02)     (351,1.341e-02)
(1471,5.334e-03)  (5503,2.027e-03)   (18943,7.415e-04)
(61183,2.628e-04) (187903,9.063e-05) (553983,3.053e-05)
};

\addplot coordinates{
(13,5.755e-02)     (97,2.925e-02)     (545,1.351e-02)
(2561,5.842e-03)   (10625,2.397e-03)  (40193,9.414e-04)
(141569,3.564e-04) (471041,1.308e-04)
(1496065,4.670e-05)
};
}%


\bookmaketitle
\tableofcontents
\include{pgfplots.title_abstract_intro}
\include{pgfplots.preliminaries}
\include{pgfplots.intro}
\include{pgfplots.reference}
\include{pgfplots.libs}
\include{pgfplots.resources}
\include{pgfplots.importexport}
\include{pgfplots.basic.reference}

\printindex

\bibliographystyle{abbrv} %gerapali} %gerabbrv} %gerunsrt.bst} %gerabbrv}% gerplain}
\nocite{pgfplotstable}
\nocite{programmingnotes}
\bibliography{pgfplots}
\end{document}

% -----------------------------------------------------------------------------
% For Stefan Pinnow as reminder on what to look for when editing the manual
% -----------------------------------------------------------------------------
% There should be no line breaks in the following environments
% - |...|
% - \declareandlabel{...}
% - \verbpdfref{...}
% If "MakeTikzPictures" isn't running through check one of the externalized
% LOG files what is the cause of that.
% -----------------------------------------------------------------------------


%%%%%%%%%%%%%%%%%%%%%%%%%%%%%%%%%%%%%%%%%%%%%%%%%%%%%%%%%%%%%%%%%%%%%%%%%%%%%
%
% Package pgfplots.sty documentation.
%
% Copyright 2007/2008 by Christian Feuersaenger.
%
% This program is free software: you can redistribute it and/or modify
% it under the terms of the GNU General Public License as published by
% the Free Software Foundation, either version 3 of the License, or
% (at your option) any later version.
%
% This program is distributed in the hope that it will be useful,
% but WITHOUT ANY WARRANTY; without even the implied warranty of
% MERCHANTABILITY or FITNESS FOR A PARTICULAR PURPOSE.  See the
% GNU General Public License for more details.
%
% You should have received a copy of the GNU General Public License
% along with this program.  If not, see <http://www.gnu.org/licenses/>.
%
%
%%%%%%%%%%%%%%%%%%%%%%%%%%%%%%%%%%%%%%%%%%%%%%%%%%%%%%%%%%%%%%%%%%%%%%%%%%%%%
% SEE pgfplots-macros.tex as well!
%\pdfminorversion=5 % to allow compression
%\pdfobjcompresslevel=2
\documentclass[a4paper,openany]{book}

\let\bookmaketitle=\maketitle

% -----------------------------------
% this here is from ltxdoc:
\usepackage{doc}[=v2]
\AtBeginDocument{\MakeShortVerb{\|}}
%\setlength{\textwidth}{355pt}
%\addtolength\marginparwidth{30pt}
%\addtolength\oddsidemargin{20pt}
%\addtolength\evensidemargin{20pt}
\def\cmd#1{\cs{\expandafter\cmd@to@cs\string#1}}
\def\cmd@to@cs#1#2{\char\number`#2\relax}
\DeclareRobustCommand\cs[1]{\texttt{\char`\\#1}}
\providecommand\marg[1]{%
  {\ttfamily\char`\{}\meta{#1}{\ttfamily\char`\}}}
\providecommand\oarg[1]{%
  {\ttfamily[}\meta{#1}{\ttfamily]}}
\providecommand\parg[1]{%
  {\ttfamily(}\meta{#1}{\ttfamily)}}
\raggedbottom

% -----------------------------------
\input{pgfplots.preamble.tex}

\makeatletter
% I want a two-column index just as in pgfmanual styles. This here
% was the best way to get one:
\def\index@prologue{\section*{Index}\addcontentsline{toc}{chapter}{Index}
}
\makeatother

%\RequirePackage[german,english,francais]{babel}

\def\matlabcolormaptext{This colormap is similar to one shipped with Matlab$^\text{\textregistered}$ under a similar name.}

\IfFileExists{tikzlibraryspy.code.tex}{%
\usetikzlibrary{spy}
}{%
    \message{ERROR: tikz SPY library NOT available. The manual will only compile partially.^^J}%
}%

\usepackage{xparse}% for colorbrewer manual

\usetikzlibrary{
    decorations.markings,
    decorations.footprints,
    shapes.arrows,
    matrix,
    positioning,
}

\usepgfplotslibrary{
    fillbetween,
    ternary,
    smithchart,
    patchplots,
    polar,
    colormaps,
    colorbrewer,
    colortol,           % docu not ready yet
}
\pgfqkeys{/codeexample}{%
    every codeexample/.append style={
        /pgfplots/every ternary axis/.append style={
            /pgfplots/legend style={fill=graphicbackground},
        }
    },
    tabsize=4,
}

\pgfplotsmanualenableexternalizationofexpensive

%\usetikzlibrary{external}
%\tikzexternalize[prefix=figures/]{pgfplots}

\title{%
    Manual for Package \PGFPlots{}\\
    {\small 2D/3D Plots in \LaTeX{}, Version \pgfplotsversion}\\
    {\small\url{https://github.com/pgf-tikz/pgfplots}}
    %\\{\small Attention: you are using an unstable development version.}
}

\makeatletter
\long\def\abstractsmuggle{%
    \centering
    \textbf{Abstract}\\[0.5cm]

    \begin{minipage}{12cm}
        \PGFPlots{} draws high-quality function plots in normal or logarithmic
        scaling with a user-friendly interface directly in \TeX{}. The user
        supplies axis labels, legend entries and the plot coordinates for one or
        more plots and \PGFPlots{} applies axis scaling, computes any logarithms
        and axis ticks and draws the plots. It supports line plots, scatter
        plots, piecewise constant plots, bar plots, area plots, mesh and surface
        plots, patch plots, contour plots, quiver plots, histogram plots, box
        plots, polar axes, ternary diagrams, smith charts and some more. It is
        based on Till Tantau's package \PGF{}/\Tikz{}.
    \end{minipage}

}%

\expandafter\date\expandafter{\@date\\[2cm]
    \abstractsmuggle
}%
\makeatother

%\includeonly{pgfplots.reference}


\begin{document}

\def\plotcoords{%
\addplot coordinates {
(5,8.312e-02)    (17,2.547e-02)   (49,7.407e-03)
(129,2.102e-03)  (321,5.874e-04)  (769,1.623e-04)
(1793,4.442e-05) (4097,1.207e-05) (9217,3.261e-06)
};

\addplot coordinates{
(7,8.472e-02)    (31,3.044e-02)    (111,1.022e-02)
(351,3.303e-03)  (1023,1.039e-03)  (2815,3.196e-04)
(7423,9.658e-05) (18943,2.873e-05) (47103,8.437e-06)
};

\addplot coordinates{
(9,7.881e-02)     (49,3.243e-02)    (209,1.232e-02)
(769,4.454e-03)   (2561,1.551e-03)  (7937,5.236e-04)
(23297,1.723e-04) (65537,5.545e-05) (178177,1.751e-05)
};

\addplot coordinates{
(11,6.887e-02)    (71,3.177e-02)     (351,1.341e-02)
(1471,5.334e-03)  (5503,2.027e-03)   (18943,7.415e-04)
(61183,2.628e-04) (187903,9.063e-05) (553983,3.053e-05)
};

\addplot coordinates{
(13,5.755e-02)     (97,2.925e-02)     (545,1.351e-02)
(2561,5.842e-03)   (10625,2.397e-03)  (40193,9.414e-04)
(141569,3.564e-04) (471041,1.308e-04)
(1496065,4.670e-05)
};
}%


\bookmaketitle
\tableofcontents
\include{pgfplots.title_abstract_intro}
\include{pgfplots.preliminaries}
\include{pgfplots.intro}
\include{pgfplots.reference}
\include{pgfplots.libs}
\include{pgfplots.resources}
\include{pgfplots.importexport}
\include{pgfplots.basic.reference}

\printindex

\bibliographystyle{abbrv} %gerapali} %gerabbrv} %gerunsrt.bst} %gerabbrv}% gerplain}
\nocite{pgfplotstable}
\nocite{programmingnotes}
\bibliography{pgfplots}
\end{document}

% -----------------------------------------------------------------------------
% For Stefan Pinnow as reminder on what to look for when editing the manual
% -----------------------------------------------------------------------------
% There should be no line breaks in the following environments
% - |...|
% - \declareandlabel{...}
% - \verbpdfref{...}
% If "MakeTikzPictures" isn't running through check one of the externalized
% LOG files what is the cause of that.
% -----------------------------------------------------------------------------

\include{pgfplots.basic.reference}

\printindex

\bibliographystyle{abbrv} %gerapali} %gerabbrv} %gerunsrt.bst} %gerabbrv}% gerplain}
\nocite{pgfplotstable}
\nocite{programmingnotes}
\bibliography{pgfplots}
\end{document}

% -----------------------------------------------------------------------------
% For Stefan Pinnow as reminder on what to look for when editing the manual
% -----------------------------------------------------------------------------
% There should be no line breaks in the following environments
% - |...|
% - \declareandlabel{...}
% - \verbpdfref{...}
% If "MakeTikzPictures" isn't running through check one of the externalized
% LOG files what is the cause of that.
% -----------------------------------------------------------------------------


%%%%%%%%%%%%%%%%%%%%%%%%%%%%%%%%%%%%%%%%%%%%%%%%%%%%%%%%%%%%%%%%%%%%%%%%%%%%%
%
% Package pgfplots.sty documentation.
%
% Copyright 2007/2008 by Christian Feuersaenger.
%
% This program is free software: you can redistribute it and/or modify
% it under the terms of the GNU General Public License as published by
% the Free Software Foundation, either version 3 of the License, or
% (at your option) any later version.
%
% This program is distributed in the hope that it will be useful,
% but WITHOUT ANY WARRANTY; without even the implied warranty of
% MERCHANTABILITY or FITNESS FOR A PARTICULAR PURPOSE.  See the
% GNU General Public License for more details.
%
% You should have received a copy of the GNU General Public License
% along with this program.  If not, see <http://www.gnu.org/licenses/>.
%
%
%%%%%%%%%%%%%%%%%%%%%%%%%%%%%%%%%%%%%%%%%%%%%%%%%%%%%%%%%%%%%%%%%%%%%%%%%%%%%
% SEE pgfplots-macros.tex as well!
%\pdfminorversion=5 % to allow compression
%\pdfobjcompresslevel=2
\documentclass[a4paper,openany]{book}

\let\bookmaketitle=\maketitle

% -----------------------------------
% this here is from ltxdoc:
\usepackage{doc}[=v2]
\AtBeginDocument{\MakeShortVerb{\|}}
%\setlength{\textwidth}{355pt}
%\addtolength\marginparwidth{30pt}
%\addtolength\oddsidemargin{20pt}
%\addtolength\evensidemargin{20pt}
\def\cmd#1{\cs{\expandafter\cmd@to@cs\string#1}}
\def\cmd@to@cs#1#2{\char\number`#2\relax}
\DeclareRobustCommand\cs[1]{\texttt{\char`\\#1}}
\providecommand\marg[1]{%
  {\ttfamily\char`\{}\meta{#1}{\ttfamily\char`\}}}
\providecommand\oarg[1]{%
  {\ttfamily[}\meta{#1}{\ttfamily]}}
\providecommand\parg[1]{%
  {\ttfamily(}\meta{#1}{\ttfamily)}}
\raggedbottom

% -----------------------------------
\input{pgfplots.preamble.tex}

\makeatletter
% I want a two-column index just as in pgfmanual styles. This here
% was the best way to get one:
\def\index@prologue{\section*{Index}\addcontentsline{toc}{chapter}{Index}
}
\makeatother

%\RequirePackage[german,english,francais]{babel}

\def\matlabcolormaptext{This colormap is similar to one shipped with Matlab$^\text{\textregistered}$ under a similar name.}

\IfFileExists{tikzlibraryspy.code.tex}{%
\usetikzlibrary{spy}
}{%
    \message{ERROR: tikz SPY library NOT available. The manual will only compile partially.^^J}%
}%

\usepackage{xparse}% for colorbrewer manual

\usetikzlibrary{
    decorations.markings,
    decorations.footprints,
    shapes.arrows,
    matrix,
    positioning,
}

\usepgfplotslibrary{
    fillbetween,
    ternary,
    smithchart,
    patchplots,
    polar,
    colormaps,
    colorbrewer,
    colortol,           % docu not ready yet
}
\pgfqkeys{/codeexample}{%
    every codeexample/.append style={
        /pgfplots/every ternary axis/.append style={
            /pgfplots/legend style={fill=graphicbackground},
        }
    },
    tabsize=4,
}

\pgfplotsmanualenableexternalizationofexpensive

%\usetikzlibrary{external}
%\tikzexternalize[prefix=figures/]{pgfplots}

\title{%
    Manual for Package \PGFPlots{}\\
    {\small 2D/3D Plots in \LaTeX{}, Version \pgfplotsversion}\\
    {\small\url{https://github.com/pgf-tikz/pgfplots}}
    %\\{\small Attention: you are using an unstable development version.}
}

\makeatletter
\long\def\abstractsmuggle{%
    \centering
    \textbf{Abstract}\\[0.5cm]

    \begin{minipage}{12cm}
        \PGFPlots{} draws high-quality function plots in normal or logarithmic
        scaling with a user-friendly interface directly in \TeX{}. The user
        supplies axis labels, legend entries and the plot coordinates for one or
        more plots and \PGFPlots{} applies axis scaling, computes any logarithms
        and axis ticks and draws the plots. It supports line plots, scatter
        plots, piecewise constant plots, bar plots, area plots, mesh and surface
        plots, patch plots, contour plots, quiver plots, histogram plots, box
        plots, polar axes, ternary diagrams, smith charts and some more. It is
        based on Till Tantau's package \PGF{}/\Tikz{}.
    \end{minipage}

}%

\expandafter\date\expandafter{\@date\\[2cm]
    \abstractsmuggle
}%
\makeatother

%\includeonly{pgfplots.reference}


\begin{document}

\def\plotcoords{%
\addplot coordinates {
(5,8.312e-02)    (17,2.547e-02)   (49,7.407e-03)
(129,2.102e-03)  (321,5.874e-04)  (769,1.623e-04)
(1793,4.442e-05) (4097,1.207e-05) (9217,3.261e-06)
};

\addplot coordinates{
(7,8.472e-02)    (31,3.044e-02)    (111,1.022e-02)
(351,3.303e-03)  (1023,1.039e-03)  (2815,3.196e-04)
(7423,9.658e-05) (18943,2.873e-05) (47103,8.437e-06)
};

\addplot coordinates{
(9,7.881e-02)     (49,3.243e-02)    (209,1.232e-02)
(769,4.454e-03)   (2561,1.551e-03)  (7937,5.236e-04)
(23297,1.723e-04) (65537,5.545e-05) (178177,1.751e-05)
};

\addplot coordinates{
(11,6.887e-02)    (71,3.177e-02)     (351,1.341e-02)
(1471,5.334e-03)  (5503,2.027e-03)   (18943,7.415e-04)
(61183,2.628e-04) (187903,9.063e-05) (553983,3.053e-05)
};

\addplot coordinates{
(13,5.755e-02)     (97,2.925e-02)     (545,1.351e-02)
(2561,5.842e-03)   (10625,2.397e-03)  (40193,9.414e-04)
(141569,3.564e-04) (471041,1.308e-04)
(1496065,4.670e-05)
};
}%


\bookmaketitle
\tableofcontents

%%%%%%%%%%%%%%%%%%%%%%%%%%%%%%%%%%%%%%%%%%%%%%%%%%%%%%%%%%%%%%%%%%%%%%%%%%%%%
%
% Package pgfplots.sty documentation.
%
% Copyright 2007/2008 by Christian Feuersaenger.
%
% This program is free software: you can redistribute it and/or modify
% it under the terms of the GNU General Public License as published by
% the Free Software Foundation, either version 3 of the License, or
% (at your option) any later version.
%
% This program is distributed in the hope that it will be useful,
% but WITHOUT ANY WARRANTY; without even the implied warranty of
% MERCHANTABILITY or FITNESS FOR A PARTICULAR PURPOSE.  See the
% GNU General Public License for more details.
%
% You should have received a copy of the GNU General Public License
% along with this program.  If not, see <http://www.gnu.org/licenses/>.
%
%
%%%%%%%%%%%%%%%%%%%%%%%%%%%%%%%%%%%%%%%%%%%%%%%%%%%%%%%%%%%%%%%%%%%%%%%%%%%%%
% SEE pgfplots-macros.tex as well!
%\pdfminorversion=5 % to allow compression
%\pdfobjcompresslevel=2
\documentclass[a4paper,openany]{book}

\let\bookmaketitle=\maketitle

% -----------------------------------
% this here is from ltxdoc:
\usepackage{doc}[=v2]
\AtBeginDocument{\MakeShortVerb{\|}}
%\setlength{\textwidth}{355pt}
%\addtolength\marginparwidth{30pt}
%\addtolength\oddsidemargin{20pt}
%\addtolength\evensidemargin{20pt}
\def\cmd#1{\cs{\expandafter\cmd@to@cs\string#1}}
\def\cmd@to@cs#1#2{\char\number`#2\relax}
\DeclareRobustCommand\cs[1]{\texttt{\char`\\#1}}
\providecommand\marg[1]{%
  {\ttfamily\char`\{}\meta{#1}{\ttfamily\char`\}}}
\providecommand\oarg[1]{%
  {\ttfamily[}\meta{#1}{\ttfamily]}}
\providecommand\parg[1]{%
  {\ttfamily(}\meta{#1}{\ttfamily)}}
\raggedbottom

% -----------------------------------
\input{pgfplots.preamble.tex}

\makeatletter
% I want a two-column index just as in pgfmanual styles. This here
% was the best way to get one:
\def\index@prologue{\section*{Index}\addcontentsline{toc}{chapter}{Index}
}
\makeatother

%\RequirePackage[german,english,francais]{babel}

\def\matlabcolormaptext{This colormap is similar to one shipped with Matlab$^\text{\textregistered}$ under a similar name.}

\IfFileExists{tikzlibraryspy.code.tex}{%
\usetikzlibrary{spy}
}{%
    \message{ERROR: tikz SPY library NOT available. The manual will only compile partially.^^J}%
}%

\usepackage{xparse}% for colorbrewer manual

\usetikzlibrary{
    decorations.markings,
    decorations.footprints,
    shapes.arrows,
    matrix,
    positioning,
}

\usepgfplotslibrary{
    fillbetween,
    ternary,
    smithchart,
    patchplots,
    polar,
    colormaps,
    colorbrewer,
    colortol,           % docu not ready yet
}
\pgfqkeys{/codeexample}{%
    every codeexample/.append style={
        /pgfplots/every ternary axis/.append style={
            /pgfplots/legend style={fill=graphicbackground},
        }
    },
    tabsize=4,
}

\pgfplotsmanualenableexternalizationofexpensive

%\usetikzlibrary{external}
%\tikzexternalize[prefix=figures/]{pgfplots}

\title{%
    Manual for Package \PGFPlots{}\\
    {\small 2D/3D Plots in \LaTeX{}, Version \pgfplotsversion}\\
    {\small\url{https://github.com/pgf-tikz/pgfplots}}
    %\\{\small Attention: you are using an unstable development version.}
}

\makeatletter
\long\def\abstractsmuggle{%
    \centering
    \textbf{Abstract}\\[0.5cm]

    \begin{minipage}{12cm}
        \PGFPlots{} draws high-quality function plots in normal or logarithmic
        scaling with a user-friendly interface directly in \TeX{}. The user
        supplies axis labels, legend entries and the plot coordinates for one or
        more plots and \PGFPlots{} applies axis scaling, computes any logarithms
        and axis ticks and draws the plots. It supports line plots, scatter
        plots, piecewise constant plots, bar plots, area plots, mesh and surface
        plots, patch plots, contour plots, quiver plots, histogram plots, box
        plots, polar axes, ternary diagrams, smith charts and some more. It is
        based on Till Tantau's package \PGF{}/\Tikz{}.
    \end{minipage}

}%

\expandafter\date\expandafter{\@date\\[2cm]
    \abstractsmuggle
}%
\makeatother

%\includeonly{pgfplots.reference}


\begin{document}

\def\plotcoords{%
\addplot coordinates {
(5,8.312e-02)    (17,2.547e-02)   (49,7.407e-03)
(129,2.102e-03)  (321,5.874e-04)  (769,1.623e-04)
(1793,4.442e-05) (4097,1.207e-05) (9217,3.261e-06)
};

\addplot coordinates{
(7,8.472e-02)    (31,3.044e-02)    (111,1.022e-02)
(351,3.303e-03)  (1023,1.039e-03)  (2815,3.196e-04)
(7423,9.658e-05) (18943,2.873e-05) (47103,8.437e-06)
};

\addplot coordinates{
(9,7.881e-02)     (49,3.243e-02)    (209,1.232e-02)
(769,4.454e-03)   (2561,1.551e-03)  (7937,5.236e-04)
(23297,1.723e-04) (65537,5.545e-05) (178177,1.751e-05)
};

\addplot coordinates{
(11,6.887e-02)    (71,3.177e-02)     (351,1.341e-02)
(1471,5.334e-03)  (5503,2.027e-03)   (18943,7.415e-04)
(61183,2.628e-04) (187903,9.063e-05) (553983,3.053e-05)
};

\addplot coordinates{
(13,5.755e-02)     (97,2.925e-02)     (545,1.351e-02)
(2561,5.842e-03)   (10625,2.397e-03)  (40193,9.414e-04)
(141569,3.564e-04) (471041,1.308e-04)
(1496065,4.670e-05)
};
}%


\bookmaketitle
\tableofcontents
\include{pgfplots.title_abstract_intro}
\include{pgfplots.preliminaries}
\include{pgfplots.intro}
\include{pgfplots.reference}
\include{pgfplots.libs}
\include{pgfplots.resources}
\include{pgfplots.importexport}
\include{pgfplots.basic.reference}

\printindex

\bibliographystyle{abbrv} %gerapali} %gerabbrv} %gerunsrt.bst} %gerabbrv}% gerplain}
\nocite{pgfplotstable}
\nocite{programmingnotes}
\bibliography{pgfplots}
\end{document}

% -----------------------------------------------------------------------------
% For Stefan Pinnow as reminder on what to look for when editing the manual
% -----------------------------------------------------------------------------
% There should be no line breaks in the following environments
% - |...|
% - \declareandlabel{...}
% - \verbpdfref{...}
% If "MakeTikzPictures" isn't running through check one of the externalized
% LOG files what is the cause of that.
% -----------------------------------------------------------------------------


%%%%%%%%%%%%%%%%%%%%%%%%%%%%%%%%%%%%%%%%%%%%%%%%%%%%%%%%%%%%%%%%%%%%%%%%%%%%%
%
% Package pgfplots.sty documentation.
%
% Copyright 2007/2008 by Christian Feuersaenger.
%
% This program is free software: you can redistribute it and/or modify
% it under the terms of the GNU General Public License as published by
% the Free Software Foundation, either version 3 of the License, or
% (at your option) any later version.
%
% This program is distributed in the hope that it will be useful,
% but WITHOUT ANY WARRANTY; without even the implied warranty of
% MERCHANTABILITY or FITNESS FOR A PARTICULAR PURPOSE.  See the
% GNU General Public License for more details.
%
% You should have received a copy of the GNU General Public License
% along with this program.  If not, see <http://www.gnu.org/licenses/>.
%
%
%%%%%%%%%%%%%%%%%%%%%%%%%%%%%%%%%%%%%%%%%%%%%%%%%%%%%%%%%%%%%%%%%%%%%%%%%%%%%
% SEE pgfplots-macros.tex as well!
%\pdfminorversion=5 % to allow compression
%\pdfobjcompresslevel=2
\documentclass[a4paper,openany]{book}

\let\bookmaketitle=\maketitle

% -----------------------------------
% this here is from ltxdoc:
\usepackage{doc}[=v2]
\AtBeginDocument{\MakeShortVerb{\|}}
%\setlength{\textwidth}{355pt}
%\addtolength\marginparwidth{30pt}
%\addtolength\oddsidemargin{20pt}
%\addtolength\evensidemargin{20pt}
\def\cmd#1{\cs{\expandafter\cmd@to@cs\string#1}}
\def\cmd@to@cs#1#2{\char\number`#2\relax}
\DeclareRobustCommand\cs[1]{\texttt{\char`\\#1}}
\providecommand\marg[1]{%
  {\ttfamily\char`\{}\meta{#1}{\ttfamily\char`\}}}
\providecommand\oarg[1]{%
  {\ttfamily[}\meta{#1}{\ttfamily]}}
\providecommand\parg[1]{%
  {\ttfamily(}\meta{#1}{\ttfamily)}}
\raggedbottom

% -----------------------------------
\input{pgfplots.preamble.tex}

\makeatletter
% I want a two-column index just as in pgfmanual styles. This here
% was the best way to get one:
\def\index@prologue{\section*{Index}\addcontentsline{toc}{chapter}{Index}
}
\makeatother

%\RequirePackage[german,english,francais]{babel}

\def\matlabcolormaptext{This colormap is similar to one shipped with Matlab$^\text{\textregistered}$ under a similar name.}

\IfFileExists{tikzlibraryspy.code.tex}{%
\usetikzlibrary{spy}
}{%
    \message{ERROR: tikz SPY library NOT available. The manual will only compile partially.^^J}%
}%

\usepackage{xparse}% for colorbrewer manual

\usetikzlibrary{
    decorations.markings,
    decorations.footprints,
    shapes.arrows,
    matrix,
    positioning,
}

\usepgfplotslibrary{
    fillbetween,
    ternary,
    smithchart,
    patchplots,
    polar,
    colormaps,
    colorbrewer,
    colortol,           % docu not ready yet
}
\pgfqkeys{/codeexample}{%
    every codeexample/.append style={
        /pgfplots/every ternary axis/.append style={
            /pgfplots/legend style={fill=graphicbackground},
        }
    },
    tabsize=4,
}

\pgfplotsmanualenableexternalizationofexpensive

%\usetikzlibrary{external}
%\tikzexternalize[prefix=figures/]{pgfplots}

\title{%
    Manual for Package \PGFPlots{}\\
    {\small 2D/3D Plots in \LaTeX{}, Version \pgfplotsversion}\\
    {\small\url{https://github.com/pgf-tikz/pgfplots}}
    %\\{\small Attention: you are using an unstable development version.}
}

\makeatletter
\long\def\abstractsmuggle{%
    \centering
    \textbf{Abstract}\\[0.5cm]

    \begin{minipage}{12cm}
        \PGFPlots{} draws high-quality function plots in normal or logarithmic
        scaling with a user-friendly interface directly in \TeX{}. The user
        supplies axis labels, legend entries and the plot coordinates for one or
        more plots and \PGFPlots{} applies axis scaling, computes any logarithms
        and axis ticks and draws the plots. It supports line plots, scatter
        plots, piecewise constant plots, bar plots, area plots, mesh and surface
        plots, patch plots, contour plots, quiver plots, histogram plots, box
        plots, polar axes, ternary diagrams, smith charts and some more. It is
        based on Till Tantau's package \PGF{}/\Tikz{}.
    \end{minipage}

}%

\expandafter\date\expandafter{\@date\\[2cm]
    \abstractsmuggle
}%
\makeatother

%\includeonly{pgfplots.reference}


\begin{document}

\def\plotcoords{%
\addplot coordinates {
(5,8.312e-02)    (17,2.547e-02)   (49,7.407e-03)
(129,2.102e-03)  (321,5.874e-04)  (769,1.623e-04)
(1793,4.442e-05) (4097,1.207e-05) (9217,3.261e-06)
};

\addplot coordinates{
(7,8.472e-02)    (31,3.044e-02)    (111,1.022e-02)
(351,3.303e-03)  (1023,1.039e-03)  (2815,3.196e-04)
(7423,9.658e-05) (18943,2.873e-05) (47103,8.437e-06)
};

\addplot coordinates{
(9,7.881e-02)     (49,3.243e-02)    (209,1.232e-02)
(769,4.454e-03)   (2561,1.551e-03)  (7937,5.236e-04)
(23297,1.723e-04) (65537,5.545e-05) (178177,1.751e-05)
};

\addplot coordinates{
(11,6.887e-02)    (71,3.177e-02)     (351,1.341e-02)
(1471,5.334e-03)  (5503,2.027e-03)   (18943,7.415e-04)
(61183,2.628e-04) (187903,9.063e-05) (553983,3.053e-05)
};

\addplot coordinates{
(13,5.755e-02)     (97,2.925e-02)     (545,1.351e-02)
(2561,5.842e-03)   (10625,2.397e-03)  (40193,9.414e-04)
(141569,3.564e-04) (471041,1.308e-04)
(1496065,4.670e-05)
};
}%


\bookmaketitle
\tableofcontents
\include{pgfplots.title_abstract_intro}
\include{pgfplots.preliminaries}
\include{pgfplots.intro}
\include{pgfplots.reference}
\include{pgfplots.libs}
\include{pgfplots.resources}
\include{pgfplots.importexport}
\include{pgfplots.basic.reference}

\printindex

\bibliographystyle{abbrv} %gerapali} %gerabbrv} %gerunsrt.bst} %gerabbrv}% gerplain}
\nocite{pgfplotstable}
\nocite{programmingnotes}
\bibliography{pgfplots}
\end{document}

% -----------------------------------------------------------------------------
% For Stefan Pinnow as reminder on what to look for when editing the manual
% -----------------------------------------------------------------------------
% There should be no line breaks in the following environments
% - |...|
% - \declareandlabel{...}
% - \verbpdfref{...}
% If "MakeTikzPictures" isn't running through check one of the externalized
% LOG files what is the cause of that.
% -----------------------------------------------------------------------------


%%%%%%%%%%%%%%%%%%%%%%%%%%%%%%%%%%%%%%%%%%%%%%%%%%%%%%%%%%%%%%%%%%%%%%%%%%%%%
%
% Package pgfplots.sty documentation.
%
% Copyright 2007/2008 by Christian Feuersaenger.
%
% This program is free software: you can redistribute it and/or modify
% it under the terms of the GNU General Public License as published by
% the Free Software Foundation, either version 3 of the License, or
% (at your option) any later version.
%
% This program is distributed in the hope that it will be useful,
% but WITHOUT ANY WARRANTY; without even the implied warranty of
% MERCHANTABILITY or FITNESS FOR A PARTICULAR PURPOSE.  See the
% GNU General Public License for more details.
%
% You should have received a copy of the GNU General Public License
% along with this program.  If not, see <http://www.gnu.org/licenses/>.
%
%
%%%%%%%%%%%%%%%%%%%%%%%%%%%%%%%%%%%%%%%%%%%%%%%%%%%%%%%%%%%%%%%%%%%%%%%%%%%%%
% SEE pgfplots-macros.tex as well!
%\pdfminorversion=5 % to allow compression
%\pdfobjcompresslevel=2
\documentclass[a4paper,openany]{book}

\let\bookmaketitle=\maketitle

% -----------------------------------
% this here is from ltxdoc:
\usepackage{doc}[=v2]
\AtBeginDocument{\MakeShortVerb{\|}}
%\setlength{\textwidth}{355pt}
%\addtolength\marginparwidth{30pt}
%\addtolength\oddsidemargin{20pt}
%\addtolength\evensidemargin{20pt}
\def\cmd#1{\cs{\expandafter\cmd@to@cs\string#1}}
\def\cmd@to@cs#1#2{\char\number`#2\relax}
\DeclareRobustCommand\cs[1]{\texttt{\char`\\#1}}
\providecommand\marg[1]{%
  {\ttfamily\char`\{}\meta{#1}{\ttfamily\char`\}}}
\providecommand\oarg[1]{%
  {\ttfamily[}\meta{#1}{\ttfamily]}}
\providecommand\parg[1]{%
  {\ttfamily(}\meta{#1}{\ttfamily)}}
\raggedbottom

% -----------------------------------
\input{pgfplots.preamble.tex}

\makeatletter
% I want a two-column index just as in pgfmanual styles. This here
% was the best way to get one:
\def\index@prologue{\section*{Index}\addcontentsline{toc}{chapter}{Index}
}
\makeatother

%\RequirePackage[german,english,francais]{babel}

\def\matlabcolormaptext{This colormap is similar to one shipped with Matlab$^\text{\textregistered}$ under a similar name.}

\IfFileExists{tikzlibraryspy.code.tex}{%
\usetikzlibrary{spy}
}{%
    \message{ERROR: tikz SPY library NOT available. The manual will only compile partially.^^J}%
}%

\usepackage{xparse}% for colorbrewer manual

\usetikzlibrary{
    decorations.markings,
    decorations.footprints,
    shapes.arrows,
    matrix,
    positioning,
}

\usepgfplotslibrary{
    fillbetween,
    ternary,
    smithchart,
    patchplots,
    polar,
    colormaps,
    colorbrewer,
    colortol,           % docu not ready yet
}
\pgfqkeys{/codeexample}{%
    every codeexample/.append style={
        /pgfplots/every ternary axis/.append style={
            /pgfplots/legend style={fill=graphicbackground},
        }
    },
    tabsize=4,
}

\pgfplotsmanualenableexternalizationofexpensive

%\usetikzlibrary{external}
%\tikzexternalize[prefix=figures/]{pgfplots}

\title{%
    Manual for Package \PGFPlots{}\\
    {\small 2D/3D Plots in \LaTeX{}, Version \pgfplotsversion}\\
    {\small\url{https://github.com/pgf-tikz/pgfplots}}
    %\\{\small Attention: you are using an unstable development version.}
}

\makeatletter
\long\def\abstractsmuggle{%
    \centering
    \textbf{Abstract}\\[0.5cm]

    \begin{minipage}{12cm}
        \PGFPlots{} draws high-quality function plots in normal or logarithmic
        scaling with a user-friendly interface directly in \TeX{}. The user
        supplies axis labels, legend entries and the plot coordinates for one or
        more plots and \PGFPlots{} applies axis scaling, computes any logarithms
        and axis ticks and draws the plots. It supports line plots, scatter
        plots, piecewise constant plots, bar plots, area plots, mesh and surface
        plots, patch plots, contour plots, quiver plots, histogram plots, box
        plots, polar axes, ternary diagrams, smith charts and some more. It is
        based on Till Tantau's package \PGF{}/\Tikz{}.
    \end{minipage}

}%

\expandafter\date\expandafter{\@date\\[2cm]
    \abstractsmuggle
}%
\makeatother

%\includeonly{pgfplots.reference}


\begin{document}

\def\plotcoords{%
\addplot coordinates {
(5,8.312e-02)    (17,2.547e-02)   (49,7.407e-03)
(129,2.102e-03)  (321,5.874e-04)  (769,1.623e-04)
(1793,4.442e-05) (4097,1.207e-05) (9217,3.261e-06)
};

\addplot coordinates{
(7,8.472e-02)    (31,3.044e-02)    (111,1.022e-02)
(351,3.303e-03)  (1023,1.039e-03)  (2815,3.196e-04)
(7423,9.658e-05) (18943,2.873e-05) (47103,8.437e-06)
};

\addplot coordinates{
(9,7.881e-02)     (49,3.243e-02)    (209,1.232e-02)
(769,4.454e-03)   (2561,1.551e-03)  (7937,5.236e-04)
(23297,1.723e-04) (65537,5.545e-05) (178177,1.751e-05)
};

\addplot coordinates{
(11,6.887e-02)    (71,3.177e-02)     (351,1.341e-02)
(1471,5.334e-03)  (5503,2.027e-03)   (18943,7.415e-04)
(61183,2.628e-04) (187903,9.063e-05) (553983,3.053e-05)
};

\addplot coordinates{
(13,5.755e-02)     (97,2.925e-02)     (545,1.351e-02)
(2561,5.842e-03)   (10625,2.397e-03)  (40193,9.414e-04)
(141569,3.564e-04) (471041,1.308e-04)
(1496065,4.670e-05)
};
}%


\bookmaketitle
\tableofcontents
\include{pgfplots.title_abstract_intro}
\include{pgfplots.preliminaries}
\include{pgfplots.intro}
\include{pgfplots.reference}
\include{pgfplots.libs}
\include{pgfplots.resources}
\include{pgfplots.importexport}
\include{pgfplots.basic.reference}

\printindex

\bibliographystyle{abbrv} %gerapali} %gerabbrv} %gerunsrt.bst} %gerabbrv}% gerplain}
\nocite{pgfplotstable}
\nocite{programmingnotes}
\bibliography{pgfplots}
\end{document}

% -----------------------------------------------------------------------------
% For Stefan Pinnow as reminder on what to look for when editing the manual
% -----------------------------------------------------------------------------
% There should be no line breaks in the following environments
% - |...|
% - \declareandlabel{...}
% - \verbpdfref{...}
% If "MakeTikzPictures" isn't running through check one of the externalized
% LOG files what is the cause of that.
% -----------------------------------------------------------------------------


%%%%%%%%%%%%%%%%%%%%%%%%%%%%%%%%%%%%%%%%%%%%%%%%%%%%%%%%%%%%%%%%%%%%%%%%%%%%%
%
% Package pgfplots.sty documentation.
%
% Copyright 2007/2008 by Christian Feuersaenger.
%
% This program is free software: you can redistribute it and/or modify
% it under the terms of the GNU General Public License as published by
% the Free Software Foundation, either version 3 of the License, or
% (at your option) any later version.
%
% This program is distributed in the hope that it will be useful,
% but WITHOUT ANY WARRANTY; without even the implied warranty of
% MERCHANTABILITY or FITNESS FOR A PARTICULAR PURPOSE.  See the
% GNU General Public License for more details.
%
% You should have received a copy of the GNU General Public License
% along with this program.  If not, see <http://www.gnu.org/licenses/>.
%
%
%%%%%%%%%%%%%%%%%%%%%%%%%%%%%%%%%%%%%%%%%%%%%%%%%%%%%%%%%%%%%%%%%%%%%%%%%%%%%
% SEE pgfplots-macros.tex as well!
%\pdfminorversion=5 % to allow compression
%\pdfobjcompresslevel=2
\documentclass[a4paper,openany]{book}

\let\bookmaketitle=\maketitle

% -----------------------------------
% this here is from ltxdoc:
\usepackage{doc}[=v2]
\AtBeginDocument{\MakeShortVerb{\|}}
%\setlength{\textwidth}{355pt}
%\addtolength\marginparwidth{30pt}
%\addtolength\oddsidemargin{20pt}
%\addtolength\evensidemargin{20pt}
\def\cmd#1{\cs{\expandafter\cmd@to@cs\string#1}}
\def\cmd@to@cs#1#2{\char\number`#2\relax}
\DeclareRobustCommand\cs[1]{\texttt{\char`\\#1}}
\providecommand\marg[1]{%
  {\ttfamily\char`\{}\meta{#1}{\ttfamily\char`\}}}
\providecommand\oarg[1]{%
  {\ttfamily[}\meta{#1}{\ttfamily]}}
\providecommand\parg[1]{%
  {\ttfamily(}\meta{#1}{\ttfamily)}}
\raggedbottom

% -----------------------------------
\input{pgfplots.preamble.tex}

\makeatletter
% I want a two-column index just as in pgfmanual styles. This here
% was the best way to get one:
\def\index@prologue{\section*{Index}\addcontentsline{toc}{chapter}{Index}
}
\makeatother

%\RequirePackage[german,english,francais]{babel}

\def\matlabcolormaptext{This colormap is similar to one shipped with Matlab$^\text{\textregistered}$ under a similar name.}

\IfFileExists{tikzlibraryspy.code.tex}{%
\usetikzlibrary{spy}
}{%
    \message{ERROR: tikz SPY library NOT available. The manual will only compile partially.^^J}%
}%

\usepackage{xparse}% for colorbrewer manual

\usetikzlibrary{
    decorations.markings,
    decorations.footprints,
    shapes.arrows,
    matrix,
    positioning,
}

\usepgfplotslibrary{
    fillbetween,
    ternary,
    smithchart,
    patchplots,
    polar,
    colormaps,
    colorbrewer,
    colortol,           % docu not ready yet
}
\pgfqkeys{/codeexample}{%
    every codeexample/.append style={
        /pgfplots/every ternary axis/.append style={
            /pgfplots/legend style={fill=graphicbackground},
        }
    },
    tabsize=4,
}

\pgfplotsmanualenableexternalizationofexpensive

%\usetikzlibrary{external}
%\tikzexternalize[prefix=figures/]{pgfplots}

\title{%
    Manual for Package \PGFPlots{}\\
    {\small 2D/3D Plots in \LaTeX{}, Version \pgfplotsversion}\\
    {\small\url{https://github.com/pgf-tikz/pgfplots}}
    %\\{\small Attention: you are using an unstable development version.}
}

\makeatletter
\long\def\abstractsmuggle{%
    \centering
    \textbf{Abstract}\\[0.5cm]

    \begin{minipage}{12cm}
        \PGFPlots{} draws high-quality function plots in normal or logarithmic
        scaling with a user-friendly interface directly in \TeX{}. The user
        supplies axis labels, legend entries and the plot coordinates for one or
        more plots and \PGFPlots{} applies axis scaling, computes any logarithms
        and axis ticks and draws the plots. It supports line plots, scatter
        plots, piecewise constant plots, bar plots, area plots, mesh and surface
        plots, patch plots, contour plots, quiver plots, histogram plots, box
        plots, polar axes, ternary diagrams, smith charts and some more. It is
        based on Till Tantau's package \PGF{}/\Tikz{}.
    \end{minipage}

}%

\expandafter\date\expandafter{\@date\\[2cm]
    \abstractsmuggle
}%
\makeatother

%\includeonly{pgfplots.reference}


\begin{document}

\def\plotcoords{%
\addplot coordinates {
(5,8.312e-02)    (17,2.547e-02)   (49,7.407e-03)
(129,2.102e-03)  (321,5.874e-04)  (769,1.623e-04)
(1793,4.442e-05) (4097,1.207e-05) (9217,3.261e-06)
};

\addplot coordinates{
(7,8.472e-02)    (31,3.044e-02)    (111,1.022e-02)
(351,3.303e-03)  (1023,1.039e-03)  (2815,3.196e-04)
(7423,9.658e-05) (18943,2.873e-05) (47103,8.437e-06)
};

\addplot coordinates{
(9,7.881e-02)     (49,3.243e-02)    (209,1.232e-02)
(769,4.454e-03)   (2561,1.551e-03)  (7937,5.236e-04)
(23297,1.723e-04) (65537,5.545e-05) (178177,1.751e-05)
};

\addplot coordinates{
(11,6.887e-02)    (71,3.177e-02)     (351,1.341e-02)
(1471,5.334e-03)  (5503,2.027e-03)   (18943,7.415e-04)
(61183,2.628e-04) (187903,9.063e-05) (553983,3.053e-05)
};

\addplot coordinates{
(13,5.755e-02)     (97,2.925e-02)     (545,1.351e-02)
(2561,5.842e-03)   (10625,2.397e-03)  (40193,9.414e-04)
(141569,3.564e-04) (471041,1.308e-04)
(1496065,4.670e-05)
};
}%


\bookmaketitle
\tableofcontents
\include{pgfplots.title_abstract_intro}
\include{pgfplots.preliminaries}
\include{pgfplots.intro}
\include{pgfplots.reference}
\include{pgfplots.libs}
\include{pgfplots.resources}
\include{pgfplots.importexport}
\include{pgfplots.basic.reference}

\printindex

\bibliographystyle{abbrv} %gerapali} %gerabbrv} %gerunsrt.bst} %gerabbrv}% gerplain}
\nocite{pgfplotstable}
\nocite{programmingnotes}
\bibliography{pgfplots}
\end{document}

% -----------------------------------------------------------------------------
% For Stefan Pinnow as reminder on what to look for when editing the manual
% -----------------------------------------------------------------------------
% There should be no line breaks in the following environments
% - |...|
% - \declareandlabel{...}
% - \verbpdfref{...}
% If "MakeTikzPictures" isn't running through check one of the externalized
% LOG files what is the cause of that.
% -----------------------------------------------------------------------------


%%%%%%%%%%%%%%%%%%%%%%%%%%%%%%%%%%%%%%%%%%%%%%%%%%%%%%%%%%%%%%%%%%%%%%%%%%%%%
%
% Package pgfplots.sty documentation.
%
% Copyright 2007/2008 by Christian Feuersaenger.
%
% This program is free software: you can redistribute it and/or modify
% it under the terms of the GNU General Public License as published by
% the Free Software Foundation, either version 3 of the License, or
% (at your option) any later version.
%
% This program is distributed in the hope that it will be useful,
% but WITHOUT ANY WARRANTY; without even the implied warranty of
% MERCHANTABILITY or FITNESS FOR A PARTICULAR PURPOSE.  See the
% GNU General Public License for more details.
%
% You should have received a copy of the GNU General Public License
% along with this program.  If not, see <http://www.gnu.org/licenses/>.
%
%
%%%%%%%%%%%%%%%%%%%%%%%%%%%%%%%%%%%%%%%%%%%%%%%%%%%%%%%%%%%%%%%%%%%%%%%%%%%%%
% SEE pgfplots-macros.tex as well!
%\pdfminorversion=5 % to allow compression
%\pdfobjcompresslevel=2
\documentclass[a4paper,openany]{book}

\let\bookmaketitle=\maketitle

% -----------------------------------
% this here is from ltxdoc:
\usepackage{doc}[=v2]
\AtBeginDocument{\MakeShortVerb{\|}}
%\setlength{\textwidth}{355pt}
%\addtolength\marginparwidth{30pt}
%\addtolength\oddsidemargin{20pt}
%\addtolength\evensidemargin{20pt}
\def\cmd#1{\cs{\expandafter\cmd@to@cs\string#1}}
\def\cmd@to@cs#1#2{\char\number`#2\relax}
\DeclareRobustCommand\cs[1]{\texttt{\char`\\#1}}
\providecommand\marg[1]{%
  {\ttfamily\char`\{}\meta{#1}{\ttfamily\char`\}}}
\providecommand\oarg[1]{%
  {\ttfamily[}\meta{#1}{\ttfamily]}}
\providecommand\parg[1]{%
  {\ttfamily(}\meta{#1}{\ttfamily)}}
\raggedbottom

% -----------------------------------
\input{pgfplots.preamble.tex}

\makeatletter
% I want a two-column index just as in pgfmanual styles. This here
% was the best way to get one:
\def\index@prologue{\section*{Index}\addcontentsline{toc}{chapter}{Index}
}
\makeatother

%\RequirePackage[german,english,francais]{babel}

\def\matlabcolormaptext{This colormap is similar to one shipped with Matlab$^\text{\textregistered}$ under a similar name.}

\IfFileExists{tikzlibraryspy.code.tex}{%
\usetikzlibrary{spy}
}{%
    \message{ERROR: tikz SPY library NOT available. The manual will only compile partially.^^J}%
}%

\usepackage{xparse}% for colorbrewer manual

\usetikzlibrary{
    decorations.markings,
    decorations.footprints,
    shapes.arrows,
    matrix,
    positioning,
}

\usepgfplotslibrary{
    fillbetween,
    ternary,
    smithchart,
    patchplots,
    polar,
    colormaps,
    colorbrewer,
    colortol,           % docu not ready yet
}
\pgfqkeys{/codeexample}{%
    every codeexample/.append style={
        /pgfplots/every ternary axis/.append style={
            /pgfplots/legend style={fill=graphicbackground},
        }
    },
    tabsize=4,
}

\pgfplotsmanualenableexternalizationofexpensive

%\usetikzlibrary{external}
%\tikzexternalize[prefix=figures/]{pgfplots}

\title{%
    Manual for Package \PGFPlots{}\\
    {\small 2D/3D Plots in \LaTeX{}, Version \pgfplotsversion}\\
    {\small\url{https://github.com/pgf-tikz/pgfplots}}
    %\\{\small Attention: you are using an unstable development version.}
}

\makeatletter
\long\def\abstractsmuggle{%
    \centering
    \textbf{Abstract}\\[0.5cm]

    \begin{minipage}{12cm}
        \PGFPlots{} draws high-quality function plots in normal or logarithmic
        scaling with a user-friendly interface directly in \TeX{}. The user
        supplies axis labels, legend entries and the plot coordinates for one or
        more plots and \PGFPlots{} applies axis scaling, computes any logarithms
        and axis ticks and draws the plots. It supports line plots, scatter
        plots, piecewise constant plots, bar plots, area plots, mesh and surface
        plots, patch plots, contour plots, quiver plots, histogram plots, box
        plots, polar axes, ternary diagrams, smith charts and some more. It is
        based on Till Tantau's package \PGF{}/\Tikz{}.
    \end{minipage}

}%

\expandafter\date\expandafter{\@date\\[2cm]
    \abstractsmuggle
}%
\makeatother

%\includeonly{pgfplots.reference}


\begin{document}

\def\plotcoords{%
\addplot coordinates {
(5,8.312e-02)    (17,2.547e-02)   (49,7.407e-03)
(129,2.102e-03)  (321,5.874e-04)  (769,1.623e-04)
(1793,4.442e-05) (4097,1.207e-05) (9217,3.261e-06)
};

\addplot coordinates{
(7,8.472e-02)    (31,3.044e-02)    (111,1.022e-02)
(351,3.303e-03)  (1023,1.039e-03)  (2815,3.196e-04)
(7423,9.658e-05) (18943,2.873e-05) (47103,8.437e-06)
};

\addplot coordinates{
(9,7.881e-02)     (49,3.243e-02)    (209,1.232e-02)
(769,4.454e-03)   (2561,1.551e-03)  (7937,5.236e-04)
(23297,1.723e-04) (65537,5.545e-05) (178177,1.751e-05)
};

\addplot coordinates{
(11,6.887e-02)    (71,3.177e-02)     (351,1.341e-02)
(1471,5.334e-03)  (5503,2.027e-03)   (18943,7.415e-04)
(61183,2.628e-04) (187903,9.063e-05) (553983,3.053e-05)
};

\addplot coordinates{
(13,5.755e-02)     (97,2.925e-02)     (545,1.351e-02)
(2561,5.842e-03)   (10625,2.397e-03)  (40193,9.414e-04)
(141569,3.564e-04) (471041,1.308e-04)
(1496065,4.670e-05)
};
}%


\bookmaketitle
\tableofcontents
\include{pgfplots.title_abstract_intro}
\include{pgfplots.preliminaries}
\include{pgfplots.intro}
\include{pgfplots.reference}
\include{pgfplots.libs}
\include{pgfplots.resources}
\include{pgfplots.importexport}
\include{pgfplots.basic.reference}

\printindex

\bibliographystyle{abbrv} %gerapali} %gerabbrv} %gerunsrt.bst} %gerabbrv}% gerplain}
\nocite{pgfplotstable}
\nocite{programmingnotes}
\bibliography{pgfplots}
\end{document}

% -----------------------------------------------------------------------------
% For Stefan Pinnow as reminder on what to look for when editing the manual
% -----------------------------------------------------------------------------
% There should be no line breaks in the following environments
% - |...|
% - \declareandlabel{...}
% - \verbpdfref{...}
% If "MakeTikzPictures" isn't running through check one of the externalized
% LOG files what is the cause of that.
% -----------------------------------------------------------------------------


%%%%%%%%%%%%%%%%%%%%%%%%%%%%%%%%%%%%%%%%%%%%%%%%%%%%%%%%%%%%%%%%%%%%%%%%%%%%%
%
% Package pgfplots.sty documentation.
%
% Copyright 2007/2008 by Christian Feuersaenger.
%
% This program is free software: you can redistribute it and/or modify
% it under the terms of the GNU General Public License as published by
% the Free Software Foundation, either version 3 of the License, or
% (at your option) any later version.
%
% This program is distributed in the hope that it will be useful,
% but WITHOUT ANY WARRANTY; without even the implied warranty of
% MERCHANTABILITY or FITNESS FOR A PARTICULAR PURPOSE.  See the
% GNU General Public License for more details.
%
% You should have received a copy of the GNU General Public License
% along with this program.  If not, see <http://www.gnu.org/licenses/>.
%
%
%%%%%%%%%%%%%%%%%%%%%%%%%%%%%%%%%%%%%%%%%%%%%%%%%%%%%%%%%%%%%%%%%%%%%%%%%%%%%
% SEE pgfplots-macros.tex as well!
%\pdfminorversion=5 % to allow compression
%\pdfobjcompresslevel=2
\documentclass[a4paper,openany]{book}

\let\bookmaketitle=\maketitle

% -----------------------------------
% this here is from ltxdoc:
\usepackage{doc}[=v2]
\AtBeginDocument{\MakeShortVerb{\|}}
%\setlength{\textwidth}{355pt}
%\addtolength\marginparwidth{30pt}
%\addtolength\oddsidemargin{20pt}
%\addtolength\evensidemargin{20pt}
\def\cmd#1{\cs{\expandafter\cmd@to@cs\string#1}}
\def\cmd@to@cs#1#2{\char\number`#2\relax}
\DeclareRobustCommand\cs[1]{\texttt{\char`\\#1}}
\providecommand\marg[1]{%
  {\ttfamily\char`\{}\meta{#1}{\ttfamily\char`\}}}
\providecommand\oarg[1]{%
  {\ttfamily[}\meta{#1}{\ttfamily]}}
\providecommand\parg[1]{%
  {\ttfamily(}\meta{#1}{\ttfamily)}}
\raggedbottom

% -----------------------------------
\input{pgfplots.preamble.tex}

\makeatletter
% I want a two-column index just as in pgfmanual styles. This here
% was the best way to get one:
\def\index@prologue{\section*{Index}\addcontentsline{toc}{chapter}{Index}
}
\makeatother

%\RequirePackage[german,english,francais]{babel}

\def\matlabcolormaptext{This colormap is similar to one shipped with Matlab$^\text{\textregistered}$ under a similar name.}

\IfFileExists{tikzlibraryspy.code.tex}{%
\usetikzlibrary{spy}
}{%
    \message{ERROR: tikz SPY library NOT available. The manual will only compile partially.^^J}%
}%

\usepackage{xparse}% for colorbrewer manual

\usetikzlibrary{
    decorations.markings,
    decorations.footprints,
    shapes.arrows,
    matrix,
    positioning,
}

\usepgfplotslibrary{
    fillbetween,
    ternary,
    smithchart,
    patchplots,
    polar,
    colormaps,
    colorbrewer,
    colortol,           % docu not ready yet
}
\pgfqkeys{/codeexample}{%
    every codeexample/.append style={
        /pgfplots/every ternary axis/.append style={
            /pgfplots/legend style={fill=graphicbackground},
        }
    },
    tabsize=4,
}

\pgfplotsmanualenableexternalizationofexpensive

%\usetikzlibrary{external}
%\tikzexternalize[prefix=figures/]{pgfplots}

\title{%
    Manual for Package \PGFPlots{}\\
    {\small 2D/3D Plots in \LaTeX{}, Version \pgfplotsversion}\\
    {\small\url{https://github.com/pgf-tikz/pgfplots}}
    %\\{\small Attention: you are using an unstable development version.}
}

\makeatletter
\long\def\abstractsmuggle{%
    \centering
    \textbf{Abstract}\\[0.5cm]

    \begin{minipage}{12cm}
        \PGFPlots{} draws high-quality function plots in normal or logarithmic
        scaling with a user-friendly interface directly in \TeX{}. The user
        supplies axis labels, legend entries and the plot coordinates for one or
        more plots and \PGFPlots{} applies axis scaling, computes any logarithms
        and axis ticks and draws the plots. It supports line plots, scatter
        plots, piecewise constant plots, bar plots, area plots, mesh and surface
        plots, patch plots, contour plots, quiver plots, histogram plots, box
        plots, polar axes, ternary diagrams, smith charts and some more. It is
        based on Till Tantau's package \PGF{}/\Tikz{}.
    \end{minipage}

}%

\expandafter\date\expandafter{\@date\\[2cm]
    \abstractsmuggle
}%
\makeatother

%\includeonly{pgfplots.reference}


\begin{document}

\def\plotcoords{%
\addplot coordinates {
(5,8.312e-02)    (17,2.547e-02)   (49,7.407e-03)
(129,2.102e-03)  (321,5.874e-04)  (769,1.623e-04)
(1793,4.442e-05) (4097,1.207e-05) (9217,3.261e-06)
};

\addplot coordinates{
(7,8.472e-02)    (31,3.044e-02)    (111,1.022e-02)
(351,3.303e-03)  (1023,1.039e-03)  (2815,3.196e-04)
(7423,9.658e-05) (18943,2.873e-05) (47103,8.437e-06)
};

\addplot coordinates{
(9,7.881e-02)     (49,3.243e-02)    (209,1.232e-02)
(769,4.454e-03)   (2561,1.551e-03)  (7937,5.236e-04)
(23297,1.723e-04) (65537,5.545e-05) (178177,1.751e-05)
};

\addplot coordinates{
(11,6.887e-02)    (71,3.177e-02)     (351,1.341e-02)
(1471,5.334e-03)  (5503,2.027e-03)   (18943,7.415e-04)
(61183,2.628e-04) (187903,9.063e-05) (553983,3.053e-05)
};

\addplot coordinates{
(13,5.755e-02)     (97,2.925e-02)     (545,1.351e-02)
(2561,5.842e-03)   (10625,2.397e-03)  (40193,9.414e-04)
(141569,3.564e-04) (471041,1.308e-04)
(1496065,4.670e-05)
};
}%


\bookmaketitle
\tableofcontents
\include{pgfplots.title_abstract_intro}
\include{pgfplots.preliminaries}
\include{pgfplots.intro}
\include{pgfplots.reference}
\include{pgfplots.libs}
\include{pgfplots.resources}
\include{pgfplots.importexport}
\include{pgfplots.basic.reference}

\printindex

\bibliographystyle{abbrv} %gerapali} %gerabbrv} %gerunsrt.bst} %gerabbrv}% gerplain}
\nocite{pgfplotstable}
\nocite{programmingnotes}
\bibliography{pgfplots}
\end{document}

% -----------------------------------------------------------------------------
% For Stefan Pinnow as reminder on what to look for when editing the manual
% -----------------------------------------------------------------------------
% There should be no line breaks in the following environments
% - |...|
% - \declareandlabel{...}
% - \verbpdfref{...}
% If "MakeTikzPictures" isn't running through check one of the externalized
% LOG files what is the cause of that.
% -----------------------------------------------------------------------------


%%%%%%%%%%%%%%%%%%%%%%%%%%%%%%%%%%%%%%%%%%%%%%%%%%%%%%%%%%%%%%%%%%%%%%%%%%%%%
%
% Package pgfplots.sty documentation.
%
% Copyright 2007/2008 by Christian Feuersaenger.
%
% This program is free software: you can redistribute it and/or modify
% it under the terms of the GNU General Public License as published by
% the Free Software Foundation, either version 3 of the License, or
% (at your option) any later version.
%
% This program is distributed in the hope that it will be useful,
% but WITHOUT ANY WARRANTY; without even the implied warranty of
% MERCHANTABILITY or FITNESS FOR A PARTICULAR PURPOSE.  See the
% GNU General Public License for more details.
%
% You should have received a copy of the GNU General Public License
% along with this program.  If not, see <http://www.gnu.org/licenses/>.
%
%
%%%%%%%%%%%%%%%%%%%%%%%%%%%%%%%%%%%%%%%%%%%%%%%%%%%%%%%%%%%%%%%%%%%%%%%%%%%%%
% SEE pgfplots-macros.tex as well!
%\pdfminorversion=5 % to allow compression
%\pdfobjcompresslevel=2
\documentclass[a4paper,openany]{book}

\let\bookmaketitle=\maketitle

% -----------------------------------
% this here is from ltxdoc:
\usepackage{doc}[=v2]
\AtBeginDocument{\MakeShortVerb{\|}}
%\setlength{\textwidth}{355pt}
%\addtolength\marginparwidth{30pt}
%\addtolength\oddsidemargin{20pt}
%\addtolength\evensidemargin{20pt}
\def\cmd#1{\cs{\expandafter\cmd@to@cs\string#1}}
\def\cmd@to@cs#1#2{\char\number`#2\relax}
\DeclareRobustCommand\cs[1]{\texttt{\char`\\#1}}
\providecommand\marg[1]{%
  {\ttfamily\char`\{}\meta{#1}{\ttfamily\char`\}}}
\providecommand\oarg[1]{%
  {\ttfamily[}\meta{#1}{\ttfamily]}}
\providecommand\parg[1]{%
  {\ttfamily(}\meta{#1}{\ttfamily)}}
\raggedbottom

% -----------------------------------
\input{pgfplots.preamble.tex}

\makeatletter
% I want a two-column index just as in pgfmanual styles. This here
% was the best way to get one:
\def\index@prologue{\section*{Index}\addcontentsline{toc}{chapter}{Index}
}
\makeatother

%\RequirePackage[german,english,francais]{babel}

\def\matlabcolormaptext{This colormap is similar to one shipped with Matlab$^\text{\textregistered}$ under a similar name.}

\IfFileExists{tikzlibraryspy.code.tex}{%
\usetikzlibrary{spy}
}{%
    \message{ERROR: tikz SPY library NOT available. The manual will only compile partially.^^J}%
}%

\usepackage{xparse}% for colorbrewer manual

\usetikzlibrary{
    decorations.markings,
    decorations.footprints,
    shapes.arrows,
    matrix,
    positioning,
}

\usepgfplotslibrary{
    fillbetween,
    ternary,
    smithchart,
    patchplots,
    polar,
    colormaps,
    colorbrewer,
    colortol,           % docu not ready yet
}
\pgfqkeys{/codeexample}{%
    every codeexample/.append style={
        /pgfplots/every ternary axis/.append style={
            /pgfplots/legend style={fill=graphicbackground},
        }
    },
    tabsize=4,
}

\pgfplotsmanualenableexternalizationofexpensive

%\usetikzlibrary{external}
%\tikzexternalize[prefix=figures/]{pgfplots}

\title{%
    Manual for Package \PGFPlots{}\\
    {\small 2D/3D Plots in \LaTeX{}, Version \pgfplotsversion}\\
    {\small\url{https://github.com/pgf-tikz/pgfplots}}
    %\\{\small Attention: you are using an unstable development version.}
}

\makeatletter
\long\def\abstractsmuggle{%
    \centering
    \textbf{Abstract}\\[0.5cm]

    \begin{minipage}{12cm}
        \PGFPlots{} draws high-quality function plots in normal or logarithmic
        scaling with a user-friendly interface directly in \TeX{}. The user
        supplies axis labels, legend entries and the plot coordinates for one or
        more plots and \PGFPlots{} applies axis scaling, computes any logarithms
        and axis ticks and draws the plots. It supports line plots, scatter
        plots, piecewise constant plots, bar plots, area plots, mesh and surface
        plots, patch plots, contour plots, quiver plots, histogram plots, box
        plots, polar axes, ternary diagrams, smith charts and some more. It is
        based on Till Tantau's package \PGF{}/\Tikz{}.
    \end{minipage}

}%

\expandafter\date\expandafter{\@date\\[2cm]
    \abstractsmuggle
}%
\makeatother

%\includeonly{pgfplots.reference}


\begin{document}

\def\plotcoords{%
\addplot coordinates {
(5,8.312e-02)    (17,2.547e-02)   (49,7.407e-03)
(129,2.102e-03)  (321,5.874e-04)  (769,1.623e-04)
(1793,4.442e-05) (4097,1.207e-05) (9217,3.261e-06)
};

\addplot coordinates{
(7,8.472e-02)    (31,3.044e-02)    (111,1.022e-02)
(351,3.303e-03)  (1023,1.039e-03)  (2815,3.196e-04)
(7423,9.658e-05) (18943,2.873e-05) (47103,8.437e-06)
};

\addplot coordinates{
(9,7.881e-02)     (49,3.243e-02)    (209,1.232e-02)
(769,4.454e-03)   (2561,1.551e-03)  (7937,5.236e-04)
(23297,1.723e-04) (65537,5.545e-05) (178177,1.751e-05)
};

\addplot coordinates{
(11,6.887e-02)    (71,3.177e-02)     (351,1.341e-02)
(1471,5.334e-03)  (5503,2.027e-03)   (18943,7.415e-04)
(61183,2.628e-04) (187903,9.063e-05) (553983,3.053e-05)
};

\addplot coordinates{
(13,5.755e-02)     (97,2.925e-02)     (545,1.351e-02)
(2561,5.842e-03)   (10625,2.397e-03)  (40193,9.414e-04)
(141569,3.564e-04) (471041,1.308e-04)
(1496065,4.670e-05)
};
}%


\bookmaketitle
\tableofcontents
\include{pgfplots.title_abstract_intro}
\include{pgfplots.preliminaries}
\include{pgfplots.intro}
\include{pgfplots.reference}
\include{pgfplots.libs}
\include{pgfplots.resources}
\include{pgfplots.importexport}
\include{pgfplots.basic.reference}

\printindex

\bibliographystyle{abbrv} %gerapali} %gerabbrv} %gerunsrt.bst} %gerabbrv}% gerplain}
\nocite{pgfplotstable}
\nocite{programmingnotes}
\bibliography{pgfplots}
\end{document}

% -----------------------------------------------------------------------------
% For Stefan Pinnow as reminder on what to look for when editing the manual
% -----------------------------------------------------------------------------
% There should be no line breaks in the following environments
% - |...|
% - \declareandlabel{...}
% - \verbpdfref{...}
% If "MakeTikzPictures" isn't running through check one of the externalized
% LOG files what is the cause of that.
% -----------------------------------------------------------------------------

\include{pgfplots.basic.reference}

\printindex

\bibliographystyle{abbrv} %gerapali} %gerabbrv} %gerunsrt.bst} %gerabbrv}% gerplain}
\nocite{pgfplotstable}
\nocite{programmingnotes}
\bibliography{pgfplots}
\end{document}

% -----------------------------------------------------------------------------
% For Stefan Pinnow as reminder on what to look for when editing the manual
% -----------------------------------------------------------------------------
% There should be no line breaks in the following environments
% - |...|
% - \declareandlabel{...}
% - \verbpdfref{...}
% If "MakeTikzPictures" isn't running through check one of the externalized
% LOG files what is the cause of that.
% -----------------------------------------------------------------------------


%%%%%%%%%%%%%%%%%%%%%%%%%%%%%%%%%%%%%%%%%%%%%%%%%%%%%%%%%%%%%%%%%%%%%%%%%%%%%
%
% Package pgfplots.sty documentation.
%
% Copyright 2007/2008 by Christian Feuersaenger.
%
% This program is free software: you can redistribute it and/or modify
% it under the terms of the GNU General Public License as published by
% the Free Software Foundation, either version 3 of the License, or
% (at your option) any later version.
%
% This program is distributed in the hope that it will be useful,
% but WITHOUT ANY WARRANTY; without even the implied warranty of
% MERCHANTABILITY or FITNESS FOR A PARTICULAR PURPOSE.  See the
% GNU General Public License for more details.
%
% You should have received a copy of the GNU General Public License
% along with this program.  If not, see <http://www.gnu.org/licenses/>.
%
%
%%%%%%%%%%%%%%%%%%%%%%%%%%%%%%%%%%%%%%%%%%%%%%%%%%%%%%%%%%%%%%%%%%%%%%%%%%%%%
% SEE pgfplots-macros.tex as well!
%\pdfminorversion=5 % to allow compression
%\pdfobjcompresslevel=2
\documentclass[a4paper,openany]{book}

\let\bookmaketitle=\maketitle

% -----------------------------------
% this here is from ltxdoc:
\usepackage{doc}[=v2]
\AtBeginDocument{\MakeShortVerb{\|}}
%\setlength{\textwidth}{355pt}
%\addtolength\marginparwidth{30pt}
%\addtolength\oddsidemargin{20pt}
%\addtolength\evensidemargin{20pt}
\def\cmd#1{\cs{\expandafter\cmd@to@cs\string#1}}
\def\cmd@to@cs#1#2{\char\number`#2\relax}
\DeclareRobustCommand\cs[1]{\texttt{\char`\\#1}}
\providecommand\marg[1]{%
  {\ttfamily\char`\{}\meta{#1}{\ttfamily\char`\}}}
\providecommand\oarg[1]{%
  {\ttfamily[}\meta{#1}{\ttfamily]}}
\providecommand\parg[1]{%
  {\ttfamily(}\meta{#1}{\ttfamily)}}
\raggedbottom

% -----------------------------------
\input{pgfplots.preamble.tex}

\makeatletter
% I want a two-column index just as in pgfmanual styles. This here
% was the best way to get one:
\def\index@prologue{\section*{Index}\addcontentsline{toc}{chapter}{Index}
}
\makeatother

%\RequirePackage[german,english,francais]{babel}

\def\matlabcolormaptext{This colormap is similar to one shipped with Matlab$^\text{\textregistered}$ under a similar name.}

\IfFileExists{tikzlibraryspy.code.tex}{%
\usetikzlibrary{spy}
}{%
    \message{ERROR: tikz SPY library NOT available. The manual will only compile partially.^^J}%
}%

\usepackage{xparse}% for colorbrewer manual

\usetikzlibrary{
    decorations.markings,
    decorations.footprints,
    shapes.arrows,
    matrix,
    positioning,
}

\usepgfplotslibrary{
    fillbetween,
    ternary,
    smithchart,
    patchplots,
    polar,
    colormaps,
    colorbrewer,
    colortol,           % docu not ready yet
}
\pgfqkeys{/codeexample}{%
    every codeexample/.append style={
        /pgfplots/every ternary axis/.append style={
            /pgfplots/legend style={fill=graphicbackground},
        }
    },
    tabsize=4,
}

\pgfplotsmanualenableexternalizationofexpensive

%\usetikzlibrary{external}
%\tikzexternalize[prefix=figures/]{pgfplots}

\title{%
    Manual for Package \PGFPlots{}\\
    {\small 2D/3D Plots in \LaTeX{}, Version \pgfplotsversion}\\
    {\small\url{https://github.com/pgf-tikz/pgfplots}}
    %\\{\small Attention: you are using an unstable development version.}
}

\makeatletter
\long\def\abstractsmuggle{%
    \centering
    \textbf{Abstract}\\[0.5cm]

    \begin{minipage}{12cm}
        \PGFPlots{} draws high-quality function plots in normal or logarithmic
        scaling with a user-friendly interface directly in \TeX{}. The user
        supplies axis labels, legend entries and the plot coordinates for one or
        more plots and \PGFPlots{} applies axis scaling, computes any logarithms
        and axis ticks and draws the plots. It supports line plots, scatter
        plots, piecewise constant plots, bar plots, area plots, mesh and surface
        plots, patch plots, contour plots, quiver plots, histogram plots, box
        plots, polar axes, ternary diagrams, smith charts and some more. It is
        based on Till Tantau's package \PGF{}/\Tikz{}.
    \end{minipage}

}%

\expandafter\date\expandafter{\@date\\[2cm]
    \abstractsmuggle
}%
\makeatother

%\includeonly{pgfplots.reference}


\begin{document}

\def\plotcoords{%
\addplot coordinates {
(5,8.312e-02)    (17,2.547e-02)   (49,7.407e-03)
(129,2.102e-03)  (321,5.874e-04)  (769,1.623e-04)
(1793,4.442e-05) (4097,1.207e-05) (9217,3.261e-06)
};

\addplot coordinates{
(7,8.472e-02)    (31,3.044e-02)    (111,1.022e-02)
(351,3.303e-03)  (1023,1.039e-03)  (2815,3.196e-04)
(7423,9.658e-05) (18943,2.873e-05) (47103,8.437e-06)
};

\addplot coordinates{
(9,7.881e-02)     (49,3.243e-02)    (209,1.232e-02)
(769,4.454e-03)   (2561,1.551e-03)  (7937,5.236e-04)
(23297,1.723e-04) (65537,5.545e-05) (178177,1.751e-05)
};

\addplot coordinates{
(11,6.887e-02)    (71,3.177e-02)     (351,1.341e-02)
(1471,5.334e-03)  (5503,2.027e-03)   (18943,7.415e-04)
(61183,2.628e-04) (187903,9.063e-05) (553983,3.053e-05)
};

\addplot coordinates{
(13,5.755e-02)     (97,2.925e-02)     (545,1.351e-02)
(2561,5.842e-03)   (10625,2.397e-03)  (40193,9.414e-04)
(141569,3.564e-04) (471041,1.308e-04)
(1496065,4.670e-05)
};
}%


\bookmaketitle
\tableofcontents

%%%%%%%%%%%%%%%%%%%%%%%%%%%%%%%%%%%%%%%%%%%%%%%%%%%%%%%%%%%%%%%%%%%%%%%%%%%%%
%
% Package pgfplots.sty documentation.
%
% Copyright 2007/2008 by Christian Feuersaenger.
%
% This program is free software: you can redistribute it and/or modify
% it under the terms of the GNU General Public License as published by
% the Free Software Foundation, either version 3 of the License, or
% (at your option) any later version.
%
% This program is distributed in the hope that it will be useful,
% but WITHOUT ANY WARRANTY; without even the implied warranty of
% MERCHANTABILITY or FITNESS FOR A PARTICULAR PURPOSE.  See the
% GNU General Public License for more details.
%
% You should have received a copy of the GNU General Public License
% along with this program.  If not, see <http://www.gnu.org/licenses/>.
%
%
%%%%%%%%%%%%%%%%%%%%%%%%%%%%%%%%%%%%%%%%%%%%%%%%%%%%%%%%%%%%%%%%%%%%%%%%%%%%%
% SEE pgfplots-macros.tex as well!
%\pdfminorversion=5 % to allow compression
%\pdfobjcompresslevel=2
\documentclass[a4paper,openany]{book}

\let\bookmaketitle=\maketitle

% -----------------------------------
% this here is from ltxdoc:
\usepackage{doc}[=v2]
\AtBeginDocument{\MakeShortVerb{\|}}
%\setlength{\textwidth}{355pt}
%\addtolength\marginparwidth{30pt}
%\addtolength\oddsidemargin{20pt}
%\addtolength\evensidemargin{20pt}
\def\cmd#1{\cs{\expandafter\cmd@to@cs\string#1}}
\def\cmd@to@cs#1#2{\char\number`#2\relax}
\DeclareRobustCommand\cs[1]{\texttt{\char`\\#1}}
\providecommand\marg[1]{%
  {\ttfamily\char`\{}\meta{#1}{\ttfamily\char`\}}}
\providecommand\oarg[1]{%
  {\ttfamily[}\meta{#1}{\ttfamily]}}
\providecommand\parg[1]{%
  {\ttfamily(}\meta{#1}{\ttfamily)}}
\raggedbottom

% -----------------------------------
\input{pgfplots.preamble.tex}

\makeatletter
% I want a two-column index just as in pgfmanual styles. This here
% was the best way to get one:
\def\index@prologue{\section*{Index}\addcontentsline{toc}{chapter}{Index}
}
\makeatother

%\RequirePackage[german,english,francais]{babel}

\def\matlabcolormaptext{This colormap is similar to one shipped with Matlab$^\text{\textregistered}$ under a similar name.}

\IfFileExists{tikzlibraryspy.code.tex}{%
\usetikzlibrary{spy}
}{%
    \message{ERROR: tikz SPY library NOT available. The manual will only compile partially.^^J}%
}%

\usepackage{xparse}% for colorbrewer manual

\usetikzlibrary{
    decorations.markings,
    decorations.footprints,
    shapes.arrows,
    matrix,
    positioning,
}

\usepgfplotslibrary{
    fillbetween,
    ternary,
    smithchart,
    patchplots,
    polar,
    colormaps,
    colorbrewer,
    colortol,           % docu not ready yet
}
\pgfqkeys{/codeexample}{%
    every codeexample/.append style={
        /pgfplots/every ternary axis/.append style={
            /pgfplots/legend style={fill=graphicbackground},
        }
    },
    tabsize=4,
}

\pgfplotsmanualenableexternalizationofexpensive

%\usetikzlibrary{external}
%\tikzexternalize[prefix=figures/]{pgfplots}

\title{%
    Manual for Package \PGFPlots{}\\
    {\small 2D/3D Plots in \LaTeX{}, Version \pgfplotsversion}\\
    {\small\url{https://github.com/pgf-tikz/pgfplots}}
    %\\{\small Attention: you are using an unstable development version.}
}

\makeatletter
\long\def\abstractsmuggle{%
    \centering
    \textbf{Abstract}\\[0.5cm]

    \begin{minipage}{12cm}
        \PGFPlots{} draws high-quality function plots in normal or logarithmic
        scaling with a user-friendly interface directly in \TeX{}. The user
        supplies axis labels, legend entries and the plot coordinates for one or
        more plots and \PGFPlots{} applies axis scaling, computes any logarithms
        and axis ticks and draws the plots. It supports line plots, scatter
        plots, piecewise constant plots, bar plots, area plots, mesh and surface
        plots, patch plots, contour plots, quiver plots, histogram plots, box
        plots, polar axes, ternary diagrams, smith charts and some more. It is
        based on Till Tantau's package \PGF{}/\Tikz{}.
    \end{minipage}

}%

\expandafter\date\expandafter{\@date\\[2cm]
    \abstractsmuggle
}%
\makeatother

%\includeonly{pgfplots.reference}


\begin{document}

\def\plotcoords{%
\addplot coordinates {
(5,8.312e-02)    (17,2.547e-02)   (49,7.407e-03)
(129,2.102e-03)  (321,5.874e-04)  (769,1.623e-04)
(1793,4.442e-05) (4097,1.207e-05) (9217,3.261e-06)
};

\addplot coordinates{
(7,8.472e-02)    (31,3.044e-02)    (111,1.022e-02)
(351,3.303e-03)  (1023,1.039e-03)  (2815,3.196e-04)
(7423,9.658e-05) (18943,2.873e-05) (47103,8.437e-06)
};

\addplot coordinates{
(9,7.881e-02)     (49,3.243e-02)    (209,1.232e-02)
(769,4.454e-03)   (2561,1.551e-03)  (7937,5.236e-04)
(23297,1.723e-04) (65537,5.545e-05) (178177,1.751e-05)
};

\addplot coordinates{
(11,6.887e-02)    (71,3.177e-02)     (351,1.341e-02)
(1471,5.334e-03)  (5503,2.027e-03)   (18943,7.415e-04)
(61183,2.628e-04) (187903,9.063e-05) (553983,3.053e-05)
};

\addplot coordinates{
(13,5.755e-02)     (97,2.925e-02)     (545,1.351e-02)
(2561,5.842e-03)   (10625,2.397e-03)  (40193,9.414e-04)
(141569,3.564e-04) (471041,1.308e-04)
(1496065,4.670e-05)
};
}%


\bookmaketitle
\tableofcontents
\include{pgfplots.title_abstract_intro}
\include{pgfplots.preliminaries}
\include{pgfplots.intro}
\include{pgfplots.reference}
\include{pgfplots.libs}
\include{pgfplots.resources}
\include{pgfplots.importexport}
\include{pgfplots.basic.reference}

\printindex

\bibliographystyle{abbrv} %gerapali} %gerabbrv} %gerunsrt.bst} %gerabbrv}% gerplain}
\nocite{pgfplotstable}
\nocite{programmingnotes}
\bibliography{pgfplots}
\end{document}

% -----------------------------------------------------------------------------
% For Stefan Pinnow as reminder on what to look for when editing the manual
% -----------------------------------------------------------------------------
% There should be no line breaks in the following environments
% - |...|
% - \declareandlabel{...}
% - \verbpdfref{...}
% If "MakeTikzPictures" isn't running through check one of the externalized
% LOG files what is the cause of that.
% -----------------------------------------------------------------------------


%%%%%%%%%%%%%%%%%%%%%%%%%%%%%%%%%%%%%%%%%%%%%%%%%%%%%%%%%%%%%%%%%%%%%%%%%%%%%
%
% Package pgfplots.sty documentation.
%
% Copyright 2007/2008 by Christian Feuersaenger.
%
% This program is free software: you can redistribute it and/or modify
% it under the terms of the GNU General Public License as published by
% the Free Software Foundation, either version 3 of the License, or
% (at your option) any later version.
%
% This program is distributed in the hope that it will be useful,
% but WITHOUT ANY WARRANTY; without even the implied warranty of
% MERCHANTABILITY or FITNESS FOR A PARTICULAR PURPOSE.  See the
% GNU General Public License for more details.
%
% You should have received a copy of the GNU General Public License
% along with this program.  If not, see <http://www.gnu.org/licenses/>.
%
%
%%%%%%%%%%%%%%%%%%%%%%%%%%%%%%%%%%%%%%%%%%%%%%%%%%%%%%%%%%%%%%%%%%%%%%%%%%%%%
% SEE pgfplots-macros.tex as well!
%\pdfminorversion=5 % to allow compression
%\pdfobjcompresslevel=2
\documentclass[a4paper,openany]{book}

\let\bookmaketitle=\maketitle

% -----------------------------------
% this here is from ltxdoc:
\usepackage{doc}[=v2]
\AtBeginDocument{\MakeShortVerb{\|}}
%\setlength{\textwidth}{355pt}
%\addtolength\marginparwidth{30pt}
%\addtolength\oddsidemargin{20pt}
%\addtolength\evensidemargin{20pt}
\def\cmd#1{\cs{\expandafter\cmd@to@cs\string#1}}
\def\cmd@to@cs#1#2{\char\number`#2\relax}
\DeclareRobustCommand\cs[1]{\texttt{\char`\\#1}}
\providecommand\marg[1]{%
  {\ttfamily\char`\{}\meta{#1}{\ttfamily\char`\}}}
\providecommand\oarg[1]{%
  {\ttfamily[}\meta{#1}{\ttfamily]}}
\providecommand\parg[1]{%
  {\ttfamily(}\meta{#1}{\ttfamily)}}
\raggedbottom

% -----------------------------------
\input{pgfplots.preamble.tex}

\makeatletter
% I want a two-column index just as in pgfmanual styles. This here
% was the best way to get one:
\def\index@prologue{\section*{Index}\addcontentsline{toc}{chapter}{Index}
}
\makeatother

%\RequirePackage[german,english,francais]{babel}

\def\matlabcolormaptext{This colormap is similar to one shipped with Matlab$^\text{\textregistered}$ under a similar name.}

\IfFileExists{tikzlibraryspy.code.tex}{%
\usetikzlibrary{spy}
}{%
    \message{ERROR: tikz SPY library NOT available. The manual will only compile partially.^^J}%
}%

\usepackage{xparse}% for colorbrewer manual

\usetikzlibrary{
    decorations.markings,
    decorations.footprints,
    shapes.arrows,
    matrix,
    positioning,
}

\usepgfplotslibrary{
    fillbetween,
    ternary,
    smithchart,
    patchplots,
    polar,
    colormaps,
    colorbrewer,
    colortol,           % docu not ready yet
}
\pgfqkeys{/codeexample}{%
    every codeexample/.append style={
        /pgfplots/every ternary axis/.append style={
            /pgfplots/legend style={fill=graphicbackground},
        }
    },
    tabsize=4,
}

\pgfplotsmanualenableexternalizationofexpensive

%\usetikzlibrary{external}
%\tikzexternalize[prefix=figures/]{pgfplots}

\title{%
    Manual for Package \PGFPlots{}\\
    {\small 2D/3D Plots in \LaTeX{}, Version \pgfplotsversion}\\
    {\small\url{https://github.com/pgf-tikz/pgfplots}}
    %\\{\small Attention: you are using an unstable development version.}
}

\makeatletter
\long\def\abstractsmuggle{%
    \centering
    \textbf{Abstract}\\[0.5cm]

    \begin{minipage}{12cm}
        \PGFPlots{} draws high-quality function plots in normal or logarithmic
        scaling with a user-friendly interface directly in \TeX{}. The user
        supplies axis labels, legend entries and the plot coordinates for one or
        more plots and \PGFPlots{} applies axis scaling, computes any logarithms
        and axis ticks and draws the plots. It supports line plots, scatter
        plots, piecewise constant plots, bar plots, area plots, mesh and surface
        plots, patch plots, contour plots, quiver plots, histogram plots, box
        plots, polar axes, ternary diagrams, smith charts and some more. It is
        based on Till Tantau's package \PGF{}/\Tikz{}.
    \end{minipage}

}%

\expandafter\date\expandafter{\@date\\[2cm]
    \abstractsmuggle
}%
\makeatother

%\includeonly{pgfplots.reference}


\begin{document}

\def\plotcoords{%
\addplot coordinates {
(5,8.312e-02)    (17,2.547e-02)   (49,7.407e-03)
(129,2.102e-03)  (321,5.874e-04)  (769,1.623e-04)
(1793,4.442e-05) (4097,1.207e-05) (9217,3.261e-06)
};

\addplot coordinates{
(7,8.472e-02)    (31,3.044e-02)    (111,1.022e-02)
(351,3.303e-03)  (1023,1.039e-03)  (2815,3.196e-04)
(7423,9.658e-05) (18943,2.873e-05) (47103,8.437e-06)
};

\addplot coordinates{
(9,7.881e-02)     (49,3.243e-02)    (209,1.232e-02)
(769,4.454e-03)   (2561,1.551e-03)  (7937,5.236e-04)
(23297,1.723e-04) (65537,5.545e-05) (178177,1.751e-05)
};

\addplot coordinates{
(11,6.887e-02)    (71,3.177e-02)     (351,1.341e-02)
(1471,5.334e-03)  (5503,2.027e-03)   (18943,7.415e-04)
(61183,2.628e-04) (187903,9.063e-05) (553983,3.053e-05)
};

\addplot coordinates{
(13,5.755e-02)     (97,2.925e-02)     (545,1.351e-02)
(2561,5.842e-03)   (10625,2.397e-03)  (40193,9.414e-04)
(141569,3.564e-04) (471041,1.308e-04)
(1496065,4.670e-05)
};
}%


\bookmaketitle
\tableofcontents
\include{pgfplots.title_abstract_intro}
\include{pgfplots.preliminaries}
\include{pgfplots.intro}
\include{pgfplots.reference}
\include{pgfplots.libs}
\include{pgfplots.resources}
\include{pgfplots.importexport}
\include{pgfplots.basic.reference}

\printindex

\bibliographystyle{abbrv} %gerapali} %gerabbrv} %gerunsrt.bst} %gerabbrv}% gerplain}
\nocite{pgfplotstable}
\nocite{programmingnotes}
\bibliography{pgfplots}
\end{document}

% -----------------------------------------------------------------------------
% For Stefan Pinnow as reminder on what to look for when editing the manual
% -----------------------------------------------------------------------------
% There should be no line breaks in the following environments
% - |...|
% - \declareandlabel{...}
% - \verbpdfref{...}
% If "MakeTikzPictures" isn't running through check one of the externalized
% LOG files what is the cause of that.
% -----------------------------------------------------------------------------


%%%%%%%%%%%%%%%%%%%%%%%%%%%%%%%%%%%%%%%%%%%%%%%%%%%%%%%%%%%%%%%%%%%%%%%%%%%%%
%
% Package pgfplots.sty documentation.
%
% Copyright 2007/2008 by Christian Feuersaenger.
%
% This program is free software: you can redistribute it and/or modify
% it under the terms of the GNU General Public License as published by
% the Free Software Foundation, either version 3 of the License, or
% (at your option) any later version.
%
% This program is distributed in the hope that it will be useful,
% but WITHOUT ANY WARRANTY; without even the implied warranty of
% MERCHANTABILITY or FITNESS FOR A PARTICULAR PURPOSE.  See the
% GNU General Public License for more details.
%
% You should have received a copy of the GNU General Public License
% along with this program.  If not, see <http://www.gnu.org/licenses/>.
%
%
%%%%%%%%%%%%%%%%%%%%%%%%%%%%%%%%%%%%%%%%%%%%%%%%%%%%%%%%%%%%%%%%%%%%%%%%%%%%%
% SEE pgfplots-macros.tex as well!
%\pdfminorversion=5 % to allow compression
%\pdfobjcompresslevel=2
\documentclass[a4paper,openany]{book}

\let\bookmaketitle=\maketitle

% -----------------------------------
% this here is from ltxdoc:
\usepackage{doc}[=v2]
\AtBeginDocument{\MakeShortVerb{\|}}
%\setlength{\textwidth}{355pt}
%\addtolength\marginparwidth{30pt}
%\addtolength\oddsidemargin{20pt}
%\addtolength\evensidemargin{20pt}
\def\cmd#1{\cs{\expandafter\cmd@to@cs\string#1}}
\def\cmd@to@cs#1#2{\char\number`#2\relax}
\DeclareRobustCommand\cs[1]{\texttt{\char`\\#1}}
\providecommand\marg[1]{%
  {\ttfamily\char`\{}\meta{#1}{\ttfamily\char`\}}}
\providecommand\oarg[1]{%
  {\ttfamily[}\meta{#1}{\ttfamily]}}
\providecommand\parg[1]{%
  {\ttfamily(}\meta{#1}{\ttfamily)}}
\raggedbottom

% -----------------------------------
\input{pgfplots.preamble.tex}

\makeatletter
% I want a two-column index just as in pgfmanual styles. This here
% was the best way to get one:
\def\index@prologue{\section*{Index}\addcontentsline{toc}{chapter}{Index}
}
\makeatother

%\RequirePackage[german,english,francais]{babel}

\def\matlabcolormaptext{This colormap is similar to one shipped with Matlab$^\text{\textregistered}$ under a similar name.}

\IfFileExists{tikzlibraryspy.code.tex}{%
\usetikzlibrary{spy}
}{%
    \message{ERROR: tikz SPY library NOT available. The manual will only compile partially.^^J}%
}%

\usepackage{xparse}% for colorbrewer manual

\usetikzlibrary{
    decorations.markings,
    decorations.footprints,
    shapes.arrows,
    matrix,
    positioning,
}

\usepgfplotslibrary{
    fillbetween,
    ternary,
    smithchart,
    patchplots,
    polar,
    colormaps,
    colorbrewer,
    colortol,           % docu not ready yet
}
\pgfqkeys{/codeexample}{%
    every codeexample/.append style={
        /pgfplots/every ternary axis/.append style={
            /pgfplots/legend style={fill=graphicbackground},
        }
    },
    tabsize=4,
}

\pgfplotsmanualenableexternalizationofexpensive

%\usetikzlibrary{external}
%\tikzexternalize[prefix=figures/]{pgfplots}

\title{%
    Manual for Package \PGFPlots{}\\
    {\small 2D/3D Plots in \LaTeX{}, Version \pgfplotsversion}\\
    {\small\url{https://github.com/pgf-tikz/pgfplots}}
    %\\{\small Attention: you are using an unstable development version.}
}

\makeatletter
\long\def\abstractsmuggle{%
    \centering
    \textbf{Abstract}\\[0.5cm]

    \begin{minipage}{12cm}
        \PGFPlots{} draws high-quality function plots in normal or logarithmic
        scaling with a user-friendly interface directly in \TeX{}. The user
        supplies axis labels, legend entries and the plot coordinates for one or
        more plots and \PGFPlots{} applies axis scaling, computes any logarithms
        and axis ticks and draws the plots. It supports line plots, scatter
        plots, piecewise constant plots, bar plots, area plots, mesh and surface
        plots, patch plots, contour plots, quiver plots, histogram plots, box
        plots, polar axes, ternary diagrams, smith charts and some more. It is
        based on Till Tantau's package \PGF{}/\Tikz{}.
    \end{minipage}

}%

\expandafter\date\expandafter{\@date\\[2cm]
    \abstractsmuggle
}%
\makeatother

%\includeonly{pgfplots.reference}


\begin{document}

\def\plotcoords{%
\addplot coordinates {
(5,8.312e-02)    (17,2.547e-02)   (49,7.407e-03)
(129,2.102e-03)  (321,5.874e-04)  (769,1.623e-04)
(1793,4.442e-05) (4097,1.207e-05) (9217,3.261e-06)
};

\addplot coordinates{
(7,8.472e-02)    (31,3.044e-02)    (111,1.022e-02)
(351,3.303e-03)  (1023,1.039e-03)  (2815,3.196e-04)
(7423,9.658e-05) (18943,2.873e-05) (47103,8.437e-06)
};

\addplot coordinates{
(9,7.881e-02)     (49,3.243e-02)    (209,1.232e-02)
(769,4.454e-03)   (2561,1.551e-03)  (7937,5.236e-04)
(23297,1.723e-04) (65537,5.545e-05) (178177,1.751e-05)
};

\addplot coordinates{
(11,6.887e-02)    (71,3.177e-02)     (351,1.341e-02)
(1471,5.334e-03)  (5503,2.027e-03)   (18943,7.415e-04)
(61183,2.628e-04) (187903,9.063e-05) (553983,3.053e-05)
};

\addplot coordinates{
(13,5.755e-02)     (97,2.925e-02)     (545,1.351e-02)
(2561,5.842e-03)   (10625,2.397e-03)  (40193,9.414e-04)
(141569,3.564e-04) (471041,1.308e-04)
(1496065,4.670e-05)
};
}%


\bookmaketitle
\tableofcontents
\include{pgfplots.title_abstract_intro}
\include{pgfplots.preliminaries}
\include{pgfplots.intro}
\include{pgfplots.reference}
\include{pgfplots.libs}
\include{pgfplots.resources}
\include{pgfplots.importexport}
\include{pgfplots.basic.reference}

\printindex

\bibliographystyle{abbrv} %gerapali} %gerabbrv} %gerunsrt.bst} %gerabbrv}% gerplain}
\nocite{pgfplotstable}
\nocite{programmingnotes}
\bibliography{pgfplots}
\end{document}

% -----------------------------------------------------------------------------
% For Stefan Pinnow as reminder on what to look for when editing the manual
% -----------------------------------------------------------------------------
% There should be no line breaks in the following environments
% - |...|
% - \declareandlabel{...}
% - \verbpdfref{...}
% If "MakeTikzPictures" isn't running through check one of the externalized
% LOG files what is the cause of that.
% -----------------------------------------------------------------------------


%%%%%%%%%%%%%%%%%%%%%%%%%%%%%%%%%%%%%%%%%%%%%%%%%%%%%%%%%%%%%%%%%%%%%%%%%%%%%
%
% Package pgfplots.sty documentation.
%
% Copyright 2007/2008 by Christian Feuersaenger.
%
% This program is free software: you can redistribute it and/or modify
% it under the terms of the GNU General Public License as published by
% the Free Software Foundation, either version 3 of the License, or
% (at your option) any later version.
%
% This program is distributed in the hope that it will be useful,
% but WITHOUT ANY WARRANTY; without even the implied warranty of
% MERCHANTABILITY or FITNESS FOR A PARTICULAR PURPOSE.  See the
% GNU General Public License for more details.
%
% You should have received a copy of the GNU General Public License
% along with this program.  If not, see <http://www.gnu.org/licenses/>.
%
%
%%%%%%%%%%%%%%%%%%%%%%%%%%%%%%%%%%%%%%%%%%%%%%%%%%%%%%%%%%%%%%%%%%%%%%%%%%%%%
% SEE pgfplots-macros.tex as well!
%\pdfminorversion=5 % to allow compression
%\pdfobjcompresslevel=2
\documentclass[a4paper,openany]{book}

\let\bookmaketitle=\maketitle

% -----------------------------------
% this here is from ltxdoc:
\usepackage{doc}[=v2]
\AtBeginDocument{\MakeShortVerb{\|}}
%\setlength{\textwidth}{355pt}
%\addtolength\marginparwidth{30pt}
%\addtolength\oddsidemargin{20pt}
%\addtolength\evensidemargin{20pt}
\def\cmd#1{\cs{\expandafter\cmd@to@cs\string#1}}
\def\cmd@to@cs#1#2{\char\number`#2\relax}
\DeclareRobustCommand\cs[1]{\texttt{\char`\\#1}}
\providecommand\marg[1]{%
  {\ttfamily\char`\{}\meta{#1}{\ttfamily\char`\}}}
\providecommand\oarg[1]{%
  {\ttfamily[}\meta{#1}{\ttfamily]}}
\providecommand\parg[1]{%
  {\ttfamily(}\meta{#1}{\ttfamily)}}
\raggedbottom

% -----------------------------------
\input{pgfplots.preamble.tex}

\makeatletter
% I want a two-column index just as in pgfmanual styles. This here
% was the best way to get one:
\def\index@prologue{\section*{Index}\addcontentsline{toc}{chapter}{Index}
}
\makeatother

%\RequirePackage[german,english,francais]{babel}

\def\matlabcolormaptext{This colormap is similar to one shipped with Matlab$^\text{\textregistered}$ under a similar name.}

\IfFileExists{tikzlibraryspy.code.tex}{%
\usetikzlibrary{spy}
}{%
    \message{ERROR: tikz SPY library NOT available. The manual will only compile partially.^^J}%
}%

\usepackage{xparse}% for colorbrewer manual

\usetikzlibrary{
    decorations.markings,
    decorations.footprints,
    shapes.arrows,
    matrix,
    positioning,
}

\usepgfplotslibrary{
    fillbetween,
    ternary,
    smithchart,
    patchplots,
    polar,
    colormaps,
    colorbrewer,
    colortol,           % docu not ready yet
}
\pgfqkeys{/codeexample}{%
    every codeexample/.append style={
        /pgfplots/every ternary axis/.append style={
            /pgfplots/legend style={fill=graphicbackground},
        }
    },
    tabsize=4,
}

\pgfplotsmanualenableexternalizationofexpensive

%\usetikzlibrary{external}
%\tikzexternalize[prefix=figures/]{pgfplots}

\title{%
    Manual for Package \PGFPlots{}\\
    {\small 2D/3D Plots in \LaTeX{}, Version \pgfplotsversion}\\
    {\small\url{https://github.com/pgf-tikz/pgfplots}}
    %\\{\small Attention: you are using an unstable development version.}
}

\makeatletter
\long\def\abstractsmuggle{%
    \centering
    \textbf{Abstract}\\[0.5cm]

    \begin{minipage}{12cm}
        \PGFPlots{} draws high-quality function plots in normal or logarithmic
        scaling with a user-friendly interface directly in \TeX{}. The user
        supplies axis labels, legend entries and the plot coordinates for one or
        more plots and \PGFPlots{} applies axis scaling, computes any logarithms
        and axis ticks and draws the plots. It supports line plots, scatter
        plots, piecewise constant plots, bar plots, area plots, mesh and surface
        plots, patch plots, contour plots, quiver plots, histogram plots, box
        plots, polar axes, ternary diagrams, smith charts and some more. It is
        based on Till Tantau's package \PGF{}/\Tikz{}.
    \end{minipage}

}%

\expandafter\date\expandafter{\@date\\[2cm]
    \abstractsmuggle
}%
\makeatother

%\includeonly{pgfplots.reference}


\begin{document}

\def\plotcoords{%
\addplot coordinates {
(5,8.312e-02)    (17,2.547e-02)   (49,7.407e-03)
(129,2.102e-03)  (321,5.874e-04)  (769,1.623e-04)
(1793,4.442e-05) (4097,1.207e-05) (9217,3.261e-06)
};

\addplot coordinates{
(7,8.472e-02)    (31,3.044e-02)    (111,1.022e-02)
(351,3.303e-03)  (1023,1.039e-03)  (2815,3.196e-04)
(7423,9.658e-05) (18943,2.873e-05) (47103,8.437e-06)
};

\addplot coordinates{
(9,7.881e-02)     (49,3.243e-02)    (209,1.232e-02)
(769,4.454e-03)   (2561,1.551e-03)  (7937,5.236e-04)
(23297,1.723e-04) (65537,5.545e-05) (178177,1.751e-05)
};

\addplot coordinates{
(11,6.887e-02)    (71,3.177e-02)     (351,1.341e-02)
(1471,5.334e-03)  (5503,2.027e-03)   (18943,7.415e-04)
(61183,2.628e-04) (187903,9.063e-05) (553983,3.053e-05)
};

\addplot coordinates{
(13,5.755e-02)     (97,2.925e-02)     (545,1.351e-02)
(2561,5.842e-03)   (10625,2.397e-03)  (40193,9.414e-04)
(141569,3.564e-04) (471041,1.308e-04)
(1496065,4.670e-05)
};
}%


\bookmaketitle
\tableofcontents
\include{pgfplots.title_abstract_intro}
\include{pgfplots.preliminaries}
\include{pgfplots.intro}
\include{pgfplots.reference}
\include{pgfplots.libs}
\include{pgfplots.resources}
\include{pgfplots.importexport}
\include{pgfplots.basic.reference}

\printindex

\bibliographystyle{abbrv} %gerapali} %gerabbrv} %gerunsrt.bst} %gerabbrv}% gerplain}
\nocite{pgfplotstable}
\nocite{programmingnotes}
\bibliography{pgfplots}
\end{document}

% -----------------------------------------------------------------------------
% For Stefan Pinnow as reminder on what to look for when editing the manual
% -----------------------------------------------------------------------------
% There should be no line breaks in the following environments
% - |...|
% - \declareandlabel{...}
% - \verbpdfref{...}
% If "MakeTikzPictures" isn't running through check one of the externalized
% LOG files what is the cause of that.
% -----------------------------------------------------------------------------


%%%%%%%%%%%%%%%%%%%%%%%%%%%%%%%%%%%%%%%%%%%%%%%%%%%%%%%%%%%%%%%%%%%%%%%%%%%%%
%
% Package pgfplots.sty documentation.
%
% Copyright 2007/2008 by Christian Feuersaenger.
%
% This program is free software: you can redistribute it and/or modify
% it under the terms of the GNU General Public License as published by
% the Free Software Foundation, either version 3 of the License, or
% (at your option) any later version.
%
% This program is distributed in the hope that it will be useful,
% but WITHOUT ANY WARRANTY; without even the implied warranty of
% MERCHANTABILITY or FITNESS FOR A PARTICULAR PURPOSE.  See the
% GNU General Public License for more details.
%
% You should have received a copy of the GNU General Public License
% along with this program.  If not, see <http://www.gnu.org/licenses/>.
%
%
%%%%%%%%%%%%%%%%%%%%%%%%%%%%%%%%%%%%%%%%%%%%%%%%%%%%%%%%%%%%%%%%%%%%%%%%%%%%%
% SEE pgfplots-macros.tex as well!
%\pdfminorversion=5 % to allow compression
%\pdfobjcompresslevel=2
\documentclass[a4paper,openany]{book}

\let\bookmaketitle=\maketitle

% -----------------------------------
% this here is from ltxdoc:
\usepackage{doc}[=v2]
\AtBeginDocument{\MakeShortVerb{\|}}
%\setlength{\textwidth}{355pt}
%\addtolength\marginparwidth{30pt}
%\addtolength\oddsidemargin{20pt}
%\addtolength\evensidemargin{20pt}
\def\cmd#1{\cs{\expandafter\cmd@to@cs\string#1}}
\def\cmd@to@cs#1#2{\char\number`#2\relax}
\DeclareRobustCommand\cs[1]{\texttt{\char`\\#1}}
\providecommand\marg[1]{%
  {\ttfamily\char`\{}\meta{#1}{\ttfamily\char`\}}}
\providecommand\oarg[1]{%
  {\ttfamily[}\meta{#1}{\ttfamily]}}
\providecommand\parg[1]{%
  {\ttfamily(}\meta{#1}{\ttfamily)}}
\raggedbottom

% -----------------------------------
\input{pgfplots.preamble.tex}

\makeatletter
% I want a two-column index just as in pgfmanual styles. This here
% was the best way to get one:
\def\index@prologue{\section*{Index}\addcontentsline{toc}{chapter}{Index}
}
\makeatother

%\RequirePackage[german,english,francais]{babel}

\def\matlabcolormaptext{This colormap is similar to one shipped with Matlab$^\text{\textregistered}$ under a similar name.}

\IfFileExists{tikzlibraryspy.code.tex}{%
\usetikzlibrary{spy}
}{%
    \message{ERROR: tikz SPY library NOT available. The manual will only compile partially.^^J}%
}%

\usepackage{xparse}% for colorbrewer manual

\usetikzlibrary{
    decorations.markings,
    decorations.footprints,
    shapes.arrows,
    matrix,
    positioning,
}

\usepgfplotslibrary{
    fillbetween,
    ternary,
    smithchart,
    patchplots,
    polar,
    colormaps,
    colorbrewer,
    colortol,           % docu not ready yet
}
\pgfqkeys{/codeexample}{%
    every codeexample/.append style={
        /pgfplots/every ternary axis/.append style={
            /pgfplots/legend style={fill=graphicbackground},
        }
    },
    tabsize=4,
}

\pgfplotsmanualenableexternalizationofexpensive

%\usetikzlibrary{external}
%\tikzexternalize[prefix=figures/]{pgfplots}

\title{%
    Manual for Package \PGFPlots{}\\
    {\small 2D/3D Plots in \LaTeX{}, Version \pgfplotsversion}\\
    {\small\url{https://github.com/pgf-tikz/pgfplots}}
    %\\{\small Attention: you are using an unstable development version.}
}

\makeatletter
\long\def\abstractsmuggle{%
    \centering
    \textbf{Abstract}\\[0.5cm]

    \begin{minipage}{12cm}
        \PGFPlots{} draws high-quality function plots in normal or logarithmic
        scaling with a user-friendly interface directly in \TeX{}. The user
        supplies axis labels, legend entries and the plot coordinates for one or
        more plots and \PGFPlots{} applies axis scaling, computes any logarithms
        and axis ticks and draws the plots. It supports line plots, scatter
        plots, piecewise constant plots, bar plots, area plots, mesh and surface
        plots, patch plots, contour plots, quiver plots, histogram plots, box
        plots, polar axes, ternary diagrams, smith charts and some more. It is
        based on Till Tantau's package \PGF{}/\Tikz{}.
    \end{minipage}

}%

\expandafter\date\expandafter{\@date\\[2cm]
    \abstractsmuggle
}%
\makeatother

%\includeonly{pgfplots.reference}


\begin{document}

\def\plotcoords{%
\addplot coordinates {
(5,8.312e-02)    (17,2.547e-02)   (49,7.407e-03)
(129,2.102e-03)  (321,5.874e-04)  (769,1.623e-04)
(1793,4.442e-05) (4097,1.207e-05) (9217,3.261e-06)
};

\addplot coordinates{
(7,8.472e-02)    (31,3.044e-02)    (111,1.022e-02)
(351,3.303e-03)  (1023,1.039e-03)  (2815,3.196e-04)
(7423,9.658e-05) (18943,2.873e-05) (47103,8.437e-06)
};

\addplot coordinates{
(9,7.881e-02)     (49,3.243e-02)    (209,1.232e-02)
(769,4.454e-03)   (2561,1.551e-03)  (7937,5.236e-04)
(23297,1.723e-04) (65537,5.545e-05) (178177,1.751e-05)
};

\addplot coordinates{
(11,6.887e-02)    (71,3.177e-02)     (351,1.341e-02)
(1471,5.334e-03)  (5503,2.027e-03)   (18943,7.415e-04)
(61183,2.628e-04) (187903,9.063e-05) (553983,3.053e-05)
};

\addplot coordinates{
(13,5.755e-02)     (97,2.925e-02)     (545,1.351e-02)
(2561,5.842e-03)   (10625,2.397e-03)  (40193,9.414e-04)
(141569,3.564e-04) (471041,1.308e-04)
(1496065,4.670e-05)
};
}%


\bookmaketitle
\tableofcontents
\include{pgfplots.title_abstract_intro}
\include{pgfplots.preliminaries}
\include{pgfplots.intro}
\include{pgfplots.reference}
\include{pgfplots.libs}
\include{pgfplots.resources}
\include{pgfplots.importexport}
\include{pgfplots.basic.reference}

\printindex

\bibliographystyle{abbrv} %gerapali} %gerabbrv} %gerunsrt.bst} %gerabbrv}% gerplain}
\nocite{pgfplotstable}
\nocite{programmingnotes}
\bibliography{pgfplots}
\end{document}

% -----------------------------------------------------------------------------
% For Stefan Pinnow as reminder on what to look for when editing the manual
% -----------------------------------------------------------------------------
% There should be no line breaks in the following environments
% - |...|
% - \declareandlabel{...}
% - \verbpdfref{...}
% If "MakeTikzPictures" isn't running through check one of the externalized
% LOG files what is the cause of that.
% -----------------------------------------------------------------------------


%%%%%%%%%%%%%%%%%%%%%%%%%%%%%%%%%%%%%%%%%%%%%%%%%%%%%%%%%%%%%%%%%%%%%%%%%%%%%
%
% Package pgfplots.sty documentation.
%
% Copyright 2007/2008 by Christian Feuersaenger.
%
% This program is free software: you can redistribute it and/or modify
% it under the terms of the GNU General Public License as published by
% the Free Software Foundation, either version 3 of the License, or
% (at your option) any later version.
%
% This program is distributed in the hope that it will be useful,
% but WITHOUT ANY WARRANTY; without even the implied warranty of
% MERCHANTABILITY or FITNESS FOR A PARTICULAR PURPOSE.  See the
% GNU General Public License for more details.
%
% You should have received a copy of the GNU General Public License
% along with this program.  If not, see <http://www.gnu.org/licenses/>.
%
%
%%%%%%%%%%%%%%%%%%%%%%%%%%%%%%%%%%%%%%%%%%%%%%%%%%%%%%%%%%%%%%%%%%%%%%%%%%%%%
% SEE pgfplots-macros.tex as well!
%\pdfminorversion=5 % to allow compression
%\pdfobjcompresslevel=2
\documentclass[a4paper,openany]{book}

\let\bookmaketitle=\maketitle

% -----------------------------------
% this here is from ltxdoc:
\usepackage{doc}[=v2]
\AtBeginDocument{\MakeShortVerb{\|}}
%\setlength{\textwidth}{355pt}
%\addtolength\marginparwidth{30pt}
%\addtolength\oddsidemargin{20pt}
%\addtolength\evensidemargin{20pt}
\def\cmd#1{\cs{\expandafter\cmd@to@cs\string#1}}
\def\cmd@to@cs#1#2{\char\number`#2\relax}
\DeclareRobustCommand\cs[1]{\texttt{\char`\\#1}}
\providecommand\marg[1]{%
  {\ttfamily\char`\{}\meta{#1}{\ttfamily\char`\}}}
\providecommand\oarg[1]{%
  {\ttfamily[}\meta{#1}{\ttfamily]}}
\providecommand\parg[1]{%
  {\ttfamily(}\meta{#1}{\ttfamily)}}
\raggedbottom

% -----------------------------------
\input{pgfplots.preamble.tex}

\makeatletter
% I want a two-column index just as in pgfmanual styles. This here
% was the best way to get one:
\def\index@prologue{\section*{Index}\addcontentsline{toc}{chapter}{Index}
}
\makeatother

%\RequirePackage[german,english,francais]{babel}

\def\matlabcolormaptext{This colormap is similar to one shipped with Matlab$^\text{\textregistered}$ under a similar name.}

\IfFileExists{tikzlibraryspy.code.tex}{%
\usetikzlibrary{spy}
}{%
    \message{ERROR: tikz SPY library NOT available. The manual will only compile partially.^^J}%
}%

\usepackage{xparse}% for colorbrewer manual

\usetikzlibrary{
    decorations.markings,
    decorations.footprints,
    shapes.arrows,
    matrix,
    positioning,
}

\usepgfplotslibrary{
    fillbetween,
    ternary,
    smithchart,
    patchplots,
    polar,
    colormaps,
    colorbrewer,
    colortol,           % docu not ready yet
}
\pgfqkeys{/codeexample}{%
    every codeexample/.append style={
        /pgfplots/every ternary axis/.append style={
            /pgfplots/legend style={fill=graphicbackground},
        }
    },
    tabsize=4,
}

\pgfplotsmanualenableexternalizationofexpensive

%\usetikzlibrary{external}
%\tikzexternalize[prefix=figures/]{pgfplots}

\title{%
    Manual for Package \PGFPlots{}\\
    {\small 2D/3D Plots in \LaTeX{}, Version \pgfplotsversion}\\
    {\small\url{https://github.com/pgf-tikz/pgfplots}}
    %\\{\small Attention: you are using an unstable development version.}
}

\makeatletter
\long\def\abstractsmuggle{%
    \centering
    \textbf{Abstract}\\[0.5cm]

    \begin{minipage}{12cm}
        \PGFPlots{} draws high-quality function plots in normal or logarithmic
        scaling with a user-friendly interface directly in \TeX{}. The user
        supplies axis labels, legend entries and the plot coordinates for one or
        more plots and \PGFPlots{} applies axis scaling, computes any logarithms
        and axis ticks and draws the plots. It supports line plots, scatter
        plots, piecewise constant plots, bar plots, area plots, mesh and surface
        plots, patch plots, contour plots, quiver plots, histogram plots, box
        plots, polar axes, ternary diagrams, smith charts and some more. It is
        based on Till Tantau's package \PGF{}/\Tikz{}.
    \end{minipage}

}%

\expandafter\date\expandafter{\@date\\[2cm]
    \abstractsmuggle
}%
\makeatother

%\includeonly{pgfplots.reference}


\begin{document}

\def\plotcoords{%
\addplot coordinates {
(5,8.312e-02)    (17,2.547e-02)   (49,7.407e-03)
(129,2.102e-03)  (321,5.874e-04)  (769,1.623e-04)
(1793,4.442e-05) (4097,1.207e-05) (9217,3.261e-06)
};

\addplot coordinates{
(7,8.472e-02)    (31,3.044e-02)    (111,1.022e-02)
(351,3.303e-03)  (1023,1.039e-03)  (2815,3.196e-04)
(7423,9.658e-05) (18943,2.873e-05) (47103,8.437e-06)
};

\addplot coordinates{
(9,7.881e-02)     (49,3.243e-02)    (209,1.232e-02)
(769,4.454e-03)   (2561,1.551e-03)  (7937,5.236e-04)
(23297,1.723e-04) (65537,5.545e-05) (178177,1.751e-05)
};

\addplot coordinates{
(11,6.887e-02)    (71,3.177e-02)     (351,1.341e-02)
(1471,5.334e-03)  (5503,2.027e-03)   (18943,7.415e-04)
(61183,2.628e-04) (187903,9.063e-05) (553983,3.053e-05)
};

\addplot coordinates{
(13,5.755e-02)     (97,2.925e-02)     (545,1.351e-02)
(2561,5.842e-03)   (10625,2.397e-03)  (40193,9.414e-04)
(141569,3.564e-04) (471041,1.308e-04)
(1496065,4.670e-05)
};
}%


\bookmaketitle
\tableofcontents
\include{pgfplots.title_abstract_intro}
\include{pgfplots.preliminaries}
\include{pgfplots.intro}
\include{pgfplots.reference}
\include{pgfplots.libs}
\include{pgfplots.resources}
\include{pgfplots.importexport}
\include{pgfplots.basic.reference}

\printindex

\bibliographystyle{abbrv} %gerapali} %gerabbrv} %gerunsrt.bst} %gerabbrv}% gerplain}
\nocite{pgfplotstable}
\nocite{programmingnotes}
\bibliography{pgfplots}
\end{document}

% -----------------------------------------------------------------------------
% For Stefan Pinnow as reminder on what to look for when editing the manual
% -----------------------------------------------------------------------------
% There should be no line breaks in the following environments
% - |...|
% - \declareandlabel{...}
% - \verbpdfref{...}
% If "MakeTikzPictures" isn't running through check one of the externalized
% LOG files what is the cause of that.
% -----------------------------------------------------------------------------


%%%%%%%%%%%%%%%%%%%%%%%%%%%%%%%%%%%%%%%%%%%%%%%%%%%%%%%%%%%%%%%%%%%%%%%%%%%%%
%
% Package pgfplots.sty documentation.
%
% Copyright 2007/2008 by Christian Feuersaenger.
%
% This program is free software: you can redistribute it and/or modify
% it under the terms of the GNU General Public License as published by
% the Free Software Foundation, either version 3 of the License, or
% (at your option) any later version.
%
% This program is distributed in the hope that it will be useful,
% but WITHOUT ANY WARRANTY; without even the implied warranty of
% MERCHANTABILITY or FITNESS FOR A PARTICULAR PURPOSE.  See the
% GNU General Public License for more details.
%
% You should have received a copy of the GNU General Public License
% along with this program.  If not, see <http://www.gnu.org/licenses/>.
%
%
%%%%%%%%%%%%%%%%%%%%%%%%%%%%%%%%%%%%%%%%%%%%%%%%%%%%%%%%%%%%%%%%%%%%%%%%%%%%%
% SEE pgfplots-macros.tex as well!
%\pdfminorversion=5 % to allow compression
%\pdfobjcompresslevel=2
\documentclass[a4paper,openany]{book}

\let\bookmaketitle=\maketitle

% -----------------------------------
% this here is from ltxdoc:
\usepackage{doc}[=v2]
\AtBeginDocument{\MakeShortVerb{\|}}
%\setlength{\textwidth}{355pt}
%\addtolength\marginparwidth{30pt}
%\addtolength\oddsidemargin{20pt}
%\addtolength\evensidemargin{20pt}
\def\cmd#1{\cs{\expandafter\cmd@to@cs\string#1}}
\def\cmd@to@cs#1#2{\char\number`#2\relax}
\DeclareRobustCommand\cs[1]{\texttt{\char`\\#1}}
\providecommand\marg[1]{%
  {\ttfamily\char`\{}\meta{#1}{\ttfamily\char`\}}}
\providecommand\oarg[1]{%
  {\ttfamily[}\meta{#1}{\ttfamily]}}
\providecommand\parg[1]{%
  {\ttfamily(}\meta{#1}{\ttfamily)}}
\raggedbottom

% -----------------------------------
\input{pgfplots.preamble.tex}

\makeatletter
% I want a two-column index just as in pgfmanual styles. This here
% was the best way to get one:
\def\index@prologue{\section*{Index}\addcontentsline{toc}{chapter}{Index}
}
\makeatother

%\RequirePackage[german,english,francais]{babel}

\def\matlabcolormaptext{This colormap is similar to one shipped with Matlab$^\text{\textregistered}$ under a similar name.}

\IfFileExists{tikzlibraryspy.code.tex}{%
\usetikzlibrary{spy}
}{%
    \message{ERROR: tikz SPY library NOT available. The manual will only compile partially.^^J}%
}%

\usepackage{xparse}% for colorbrewer manual

\usetikzlibrary{
    decorations.markings,
    decorations.footprints,
    shapes.arrows,
    matrix,
    positioning,
}

\usepgfplotslibrary{
    fillbetween,
    ternary,
    smithchart,
    patchplots,
    polar,
    colormaps,
    colorbrewer,
    colortol,           % docu not ready yet
}
\pgfqkeys{/codeexample}{%
    every codeexample/.append style={
        /pgfplots/every ternary axis/.append style={
            /pgfplots/legend style={fill=graphicbackground},
        }
    },
    tabsize=4,
}

\pgfplotsmanualenableexternalizationofexpensive

%\usetikzlibrary{external}
%\tikzexternalize[prefix=figures/]{pgfplots}

\title{%
    Manual for Package \PGFPlots{}\\
    {\small 2D/3D Plots in \LaTeX{}, Version \pgfplotsversion}\\
    {\small\url{https://github.com/pgf-tikz/pgfplots}}
    %\\{\small Attention: you are using an unstable development version.}
}

\makeatletter
\long\def\abstractsmuggle{%
    \centering
    \textbf{Abstract}\\[0.5cm]

    \begin{minipage}{12cm}
        \PGFPlots{} draws high-quality function plots in normal or logarithmic
        scaling with a user-friendly interface directly in \TeX{}. The user
        supplies axis labels, legend entries and the plot coordinates for one or
        more plots and \PGFPlots{} applies axis scaling, computes any logarithms
        and axis ticks and draws the plots. It supports line plots, scatter
        plots, piecewise constant plots, bar plots, area plots, mesh and surface
        plots, patch plots, contour plots, quiver plots, histogram plots, box
        plots, polar axes, ternary diagrams, smith charts and some more. It is
        based on Till Tantau's package \PGF{}/\Tikz{}.
    \end{minipage}

}%

\expandafter\date\expandafter{\@date\\[2cm]
    \abstractsmuggle
}%
\makeatother

%\includeonly{pgfplots.reference}


\begin{document}

\def\plotcoords{%
\addplot coordinates {
(5,8.312e-02)    (17,2.547e-02)   (49,7.407e-03)
(129,2.102e-03)  (321,5.874e-04)  (769,1.623e-04)
(1793,4.442e-05) (4097,1.207e-05) (9217,3.261e-06)
};

\addplot coordinates{
(7,8.472e-02)    (31,3.044e-02)    (111,1.022e-02)
(351,3.303e-03)  (1023,1.039e-03)  (2815,3.196e-04)
(7423,9.658e-05) (18943,2.873e-05) (47103,8.437e-06)
};

\addplot coordinates{
(9,7.881e-02)     (49,3.243e-02)    (209,1.232e-02)
(769,4.454e-03)   (2561,1.551e-03)  (7937,5.236e-04)
(23297,1.723e-04) (65537,5.545e-05) (178177,1.751e-05)
};

\addplot coordinates{
(11,6.887e-02)    (71,3.177e-02)     (351,1.341e-02)
(1471,5.334e-03)  (5503,2.027e-03)   (18943,7.415e-04)
(61183,2.628e-04) (187903,9.063e-05) (553983,3.053e-05)
};

\addplot coordinates{
(13,5.755e-02)     (97,2.925e-02)     (545,1.351e-02)
(2561,5.842e-03)   (10625,2.397e-03)  (40193,9.414e-04)
(141569,3.564e-04) (471041,1.308e-04)
(1496065,4.670e-05)
};
}%


\bookmaketitle
\tableofcontents
\include{pgfplots.title_abstract_intro}
\include{pgfplots.preliminaries}
\include{pgfplots.intro}
\include{pgfplots.reference}
\include{pgfplots.libs}
\include{pgfplots.resources}
\include{pgfplots.importexport}
\include{pgfplots.basic.reference}

\printindex

\bibliographystyle{abbrv} %gerapali} %gerabbrv} %gerunsrt.bst} %gerabbrv}% gerplain}
\nocite{pgfplotstable}
\nocite{programmingnotes}
\bibliography{pgfplots}
\end{document}

% -----------------------------------------------------------------------------
% For Stefan Pinnow as reminder on what to look for when editing the manual
% -----------------------------------------------------------------------------
% There should be no line breaks in the following environments
% - |...|
% - \declareandlabel{...}
% - \verbpdfref{...}
% If "MakeTikzPictures" isn't running through check one of the externalized
% LOG files what is the cause of that.
% -----------------------------------------------------------------------------

\include{pgfplots.basic.reference}

\printindex

\bibliographystyle{abbrv} %gerapali} %gerabbrv} %gerunsrt.bst} %gerabbrv}% gerplain}
\nocite{pgfplotstable}
\nocite{programmingnotes}
\bibliography{pgfplots}
\end{document}

% -----------------------------------------------------------------------------
% For Stefan Pinnow as reminder on what to look for when editing the manual
% -----------------------------------------------------------------------------
% There should be no line breaks in the following environments
% - |...|
% - \declareandlabel{...}
% - \verbpdfref{...}
% If "MakeTikzPictures" isn't running through check one of the externalized
% LOG files what is the cause of that.
% -----------------------------------------------------------------------------


%%%%%%%%%%%%%%%%%%%%%%%%%%%%%%%%%%%%%%%%%%%%%%%%%%%%%%%%%%%%%%%%%%%%%%%%%%%%%
%
% Package pgfplots.sty documentation.
%
% Copyright 2007/2008 by Christian Feuersaenger.
%
% This program is free software: you can redistribute it and/or modify
% it under the terms of the GNU General Public License as published by
% the Free Software Foundation, either version 3 of the License, or
% (at your option) any later version.
%
% This program is distributed in the hope that it will be useful,
% but WITHOUT ANY WARRANTY; without even the implied warranty of
% MERCHANTABILITY or FITNESS FOR A PARTICULAR PURPOSE.  See the
% GNU General Public License for more details.
%
% You should have received a copy of the GNU General Public License
% along with this program.  If not, see <http://www.gnu.org/licenses/>.
%
%
%%%%%%%%%%%%%%%%%%%%%%%%%%%%%%%%%%%%%%%%%%%%%%%%%%%%%%%%%%%%%%%%%%%%%%%%%%%%%
% SEE pgfplots-macros.tex as well!
%\pdfminorversion=5 % to allow compression
%\pdfobjcompresslevel=2
\documentclass[a4paper,openany]{book}

\let\bookmaketitle=\maketitle

% -----------------------------------
% this here is from ltxdoc:
\usepackage{doc}[=v2]
\AtBeginDocument{\MakeShortVerb{\|}}
%\setlength{\textwidth}{355pt}
%\addtolength\marginparwidth{30pt}
%\addtolength\oddsidemargin{20pt}
%\addtolength\evensidemargin{20pt}
\def\cmd#1{\cs{\expandafter\cmd@to@cs\string#1}}
\def\cmd@to@cs#1#2{\char\number`#2\relax}
\DeclareRobustCommand\cs[1]{\texttt{\char`\\#1}}
\providecommand\marg[1]{%
  {\ttfamily\char`\{}\meta{#1}{\ttfamily\char`\}}}
\providecommand\oarg[1]{%
  {\ttfamily[}\meta{#1}{\ttfamily]}}
\providecommand\parg[1]{%
  {\ttfamily(}\meta{#1}{\ttfamily)}}
\raggedbottom

% -----------------------------------
\input{pgfplots.preamble.tex}

\makeatletter
% I want a two-column index just as in pgfmanual styles. This here
% was the best way to get one:
\def\index@prologue{\section*{Index}\addcontentsline{toc}{chapter}{Index}
}
\makeatother

%\RequirePackage[german,english,francais]{babel}

\def\matlabcolormaptext{This colormap is similar to one shipped with Matlab$^\text{\textregistered}$ under a similar name.}

\IfFileExists{tikzlibraryspy.code.tex}{%
\usetikzlibrary{spy}
}{%
    \message{ERROR: tikz SPY library NOT available. The manual will only compile partially.^^J}%
}%

\usepackage{xparse}% for colorbrewer manual

\usetikzlibrary{
    decorations.markings,
    decorations.footprints,
    shapes.arrows,
    matrix,
    positioning,
}

\usepgfplotslibrary{
    fillbetween,
    ternary,
    smithchart,
    patchplots,
    polar,
    colormaps,
    colorbrewer,
    colortol,           % docu not ready yet
}
\pgfqkeys{/codeexample}{%
    every codeexample/.append style={
        /pgfplots/every ternary axis/.append style={
            /pgfplots/legend style={fill=graphicbackground},
        }
    },
    tabsize=4,
}

\pgfplotsmanualenableexternalizationofexpensive

%\usetikzlibrary{external}
%\tikzexternalize[prefix=figures/]{pgfplots}

\title{%
    Manual for Package \PGFPlots{}\\
    {\small 2D/3D Plots in \LaTeX{}, Version \pgfplotsversion}\\
    {\small\url{https://github.com/pgf-tikz/pgfplots}}
    %\\{\small Attention: you are using an unstable development version.}
}

\makeatletter
\long\def\abstractsmuggle{%
    \centering
    \textbf{Abstract}\\[0.5cm]

    \begin{minipage}{12cm}
        \PGFPlots{} draws high-quality function plots in normal or logarithmic
        scaling with a user-friendly interface directly in \TeX{}. The user
        supplies axis labels, legend entries and the plot coordinates for one or
        more plots and \PGFPlots{} applies axis scaling, computes any logarithms
        and axis ticks and draws the plots. It supports line plots, scatter
        plots, piecewise constant plots, bar plots, area plots, mesh and surface
        plots, patch plots, contour plots, quiver plots, histogram plots, box
        plots, polar axes, ternary diagrams, smith charts and some more. It is
        based on Till Tantau's package \PGF{}/\Tikz{}.
    \end{minipage}

}%

\expandafter\date\expandafter{\@date\\[2cm]
    \abstractsmuggle
}%
\makeatother

%\includeonly{pgfplots.reference}


\begin{document}

\def\plotcoords{%
\addplot coordinates {
(5,8.312e-02)    (17,2.547e-02)   (49,7.407e-03)
(129,2.102e-03)  (321,5.874e-04)  (769,1.623e-04)
(1793,4.442e-05) (4097,1.207e-05) (9217,3.261e-06)
};

\addplot coordinates{
(7,8.472e-02)    (31,3.044e-02)    (111,1.022e-02)
(351,3.303e-03)  (1023,1.039e-03)  (2815,3.196e-04)
(7423,9.658e-05) (18943,2.873e-05) (47103,8.437e-06)
};

\addplot coordinates{
(9,7.881e-02)     (49,3.243e-02)    (209,1.232e-02)
(769,4.454e-03)   (2561,1.551e-03)  (7937,5.236e-04)
(23297,1.723e-04) (65537,5.545e-05) (178177,1.751e-05)
};

\addplot coordinates{
(11,6.887e-02)    (71,3.177e-02)     (351,1.341e-02)
(1471,5.334e-03)  (5503,2.027e-03)   (18943,7.415e-04)
(61183,2.628e-04) (187903,9.063e-05) (553983,3.053e-05)
};

\addplot coordinates{
(13,5.755e-02)     (97,2.925e-02)     (545,1.351e-02)
(2561,5.842e-03)   (10625,2.397e-03)  (40193,9.414e-04)
(141569,3.564e-04) (471041,1.308e-04)
(1496065,4.670e-05)
};
}%


\bookmaketitle
\tableofcontents

%%%%%%%%%%%%%%%%%%%%%%%%%%%%%%%%%%%%%%%%%%%%%%%%%%%%%%%%%%%%%%%%%%%%%%%%%%%%%
%
% Package pgfplots.sty documentation.
%
% Copyright 2007/2008 by Christian Feuersaenger.
%
% This program is free software: you can redistribute it and/or modify
% it under the terms of the GNU General Public License as published by
% the Free Software Foundation, either version 3 of the License, or
% (at your option) any later version.
%
% This program is distributed in the hope that it will be useful,
% but WITHOUT ANY WARRANTY; without even the implied warranty of
% MERCHANTABILITY or FITNESS FOR A PARTICULAR PURPOSE.  See the
% GNU General Public License for more details.
%
% You should have received a copy of the GNU General Public License
% along with this program.  If not, see <http://www.gnu.org/licenses/>.
%
%
%%%%%%%%%%%%%%%%%%%%%%%%%%%%%%%%%%%%%%%%%%%%%%%%%%%%%%%%%%%%%%%%%%%%%%%%%%%%%
% SEE pgfplots-macros.tex as well!
%\pdfminorversion=5 % to allow compression
%\pdfobjcompresslevel=2
\documentclass[a4paper,openany]{book}

\let\bookmaketitle=\maketitle

% -----------------------------------
% this here is from ltxdoc:
\usepackage{doc}[=v2]
\AtBeginDocument{\MakeShortVerb{\|}}
%\setlength{\textwidth}{355pt}
%\addtolength\marginparwidth{30pt}
%\addtolength\oddsidemargin{20pt}
%\addtolength\evensidemargin{20pt}
\def\cmd#1{\cs{\expandafter\cmd@to@cs\string#1}}
\def\cmd@to@cs#1#2{\char\number`#2\relax}
\DeclareRobustCommand\cs[1]{\texttt{\char`\\#1}}
\providecommand\marg[1]{%
  {\ttfamily\char`\{}\meta{#1}{\ttfamily\char`\}}}
\providecommand\oarg[1]{%
  {\ttfamily[}\meta{#1}{\ttfamily]}}
\providecommand\parg[1]{%
  {\ttfamily(}\meta{#1}{\ttfamily)}}
\raggedbottom

% -----------------------------------
\input{pgfplots.preamble.tex}

\makeatletter
% I want a two-column index just as in pgfmanual styles. This here
% was the best way to get one:
\def\index@prologue{\section*{Index}\addcontentsline{toc}{chapter}{Index}
}
\makeatother

%\RequirePackage[german,english,francais]{babel}

\def\matlabcolormaptext{This colormap is similar to one shipped with Matlab$^\text{\textregistered}$ under a similar name.}

\IfFileExists{tikzlibraryspy.code.tex}{%
\usetikzlibrary{spy}
}{%
    \message{ERROR: tikz SPY library NOT available. The manual will only compile partially.^^J}%
}%

\usepackage{xparse}% for colorbrewer manual

\usetikzlibrary{
    decorations.markings,
    decorations.footprints,
    shapes.arrows,
    matrix,
    positioning,
}

\usepgfplotslibrary{
    fillbetween,
    ternary,
    smithchart,
    patchplots,
    polar,
    colormaps,
    colorbrewer,
    colortol,           % docu not ready yet
}
\pgfqkeys{/codeexample}{%
    every codeexample/.append style={
        /pgfplots/every ternary axis/.append style={
            /pgfplots/legend style={fill=graphicbackground},
        }
    },
    tabsize=4,
}

\pgfplotsmanualenableexternalizationofexpensive

%\usetikzlibrary{external}
%\tikzexternalize[prefix=figures/]{pgfplots}

\title{%
    Manual for Package \PGFPlots{}\\
    {\small 2D/3D Plots in \LaTeX{}, Version \pgfplotsversion}\\
    {\small\url{https://github.com/pgf-tikz/pgfplots}}
    %\\{\small Attention: you are using an unstable development version.}
}

\makeatletter
\long\def\abstractsmuggle{%
    \centering
    \textbf{Abstract}\\[0.5cm]

    \begin{minipage}{12cm}
        \PGFPlots{} draws high-quality function plots in normal or logarithmic
        scaling with a user-friendly interface directly in \TeX{}. The user
        supplies axis labels, legend entries and the plot coordinates for one or
        more plots and \PGFPlots{} applies axis scaling, computes any logarithms
        and axis ticks and draws the plots. It supports line plots, scatter
        plots, piecewise constant plots, bar plots, area plots, mesh and surface
        plots, patch plots, contour plots, quiver plots, histogram plots, box
        plots, polar axes, ternary diagrams, smith charts and some more. It is
        based on Till Tantau's package \PGF{}/\Tikz{}.
    \end{minipage}

}%

\expandafter\date\expandafter{\@date\\[2cm]
    \abstractsmuggle
}%
\makeatother

%\includeonly{pgfplots.reference}


\begin{document}

\def\plotcoords{%
\addplot coordinates {
(5,8.312e-02)    (17,2.547e-02)   (49,7.407e-03)
(129,2.102e-03)  (321,5.874e-04)  (769,1.623e-04)
(1793,4.442e-05) (4097,1.207e-05) (9217,3.261e-06)
};

\addplot coordinates{
(7,8.472e-02)    (31,3.044e-02)    (111,1.022e-02)
(351,3.303e-03)  (1023,1.039e-03)  (2815,3.196e-04)
(7423,9.658e-05) (18943,2.873e-05) (47103,8.437e-06)
};

\addplot coordinates{
(9,7.881e-02)     (49,3.243e-02)    (209,1.232e-02)
(769,4.454e-03)   (2561,1.551e-03)  (7937,5.236e-04)
(23297,1.723e-04) (65537,5.545e-05) (178177,1.751e-05)
};

\addplot coordinates{
(11,6.887e-02)    (71,3.177e-02)     (351,1.341e-02)
(1471,5.334e-03)  (5503,2.027e-03)   (18943,7.415e-04)
(61183,2.628e-04) (187903,9.063e-05) (553983,3.053e-05)
};

\addplot coordinates{
(13,5.755e-02)     (97,2.925e-02)     (545,1.351e-02)
(2561,5.842e-03)   (10625,2.397e-03)  (40193,9.414e-04)
(141569,3.564e-04) (471041,1.308e-04)
(1496065,4.670e-05)
};
}%


\bookmaketitle
\tableofcontents
\include{pgfplots.title_abstract_intro}
\include{pgfplots.preliminaries}
\include{pgfplots.intro}
\include{pgfplots.reference}
\include{pgfplots.libs}
\include{pgfplots.resources}
\include{pgfplots.importexport}
\include{pgfplots.basic.reference}

\printindex

\bibliographystyle{abbrv} %gerapali} %gerabbrv} %gerunsrt.bst} %gerabbrv}% gerplain}
\nocite{pgfplotstable}
\nocite{programmingnotes}
\bibliography{pgfplots}
\end{document}

% -----------------------------------------------------------------------------
% For Stefan Pinnow as reminder on what to look for when editing the manual
% -----------------------------------------------------------------------------
% There should be no line breaks in the following environments
% - |...|
% - \declareandlabel{...}
% - \verbpdfref{...}
% If "MakeTikzPictures" isn't running through check one of the externalized
% LOG files what is the cause of that.
% -----------------------------------------------------------------------------


%%%%%%%%%%%%%%%%%%%%%%%%%%%%%%%%%%%%%%%%%%%%%%%%%%%%%%%%%%%%%%%%%%%%%%%%%%%%%
%
% Package pgfplots.sty documentation.
%
% Copyright 2007/2008 by Christian Feuersaenger.
%
% This program is free software: you can redistribute it and/or modify
% it under the terms of the GNU General Public License as published by
% the Free Software Foundation, either version 3 of the License, or
% (at your option) any later version.
%
% This program is distributed in the hope that it will be useful,
% but WITHOUT ANY WARRANTY; without even the implied warranty of
% MERCHANTABILITY or FITNESS FOR A PARTICULAR PURPOSE.  See the
% GNU General Public License for more details.
%
% You should have received a copy of the GNU General Public License
% along with this program.  If not, see <http://www.gnu.org/licenses/>.
%
%
%%%%%%%%%%%%%%%%%%%%%%%%%%%%%%%%%%%%%%%%%%%%%%%%%%%%%%%%%%%%%%%%%%%%%%%%%%%%%
% SEE pgfplots-macros.tex as well!
%\pdfminorversion=5 % to allow compression
%\pdfobjcompresslevel=2
\documentclass[a4paper,openany]{book}

\let\bookmaketitle=\maketitle

% -----------------------------------
% this here is from ltxdoc:
\usepackage{doc}[=v2]
\AtBeginDocument{\MakeShortVerb{\|}}
%\setlength{\textwidth}{355pt}
%\addtolength\marginparwidth{30pt}
%\addtolength\oddsidemargin{20pt}
%\addtolength\evensidemargin{20pt}
\def\cmd#1{\cs{\expandafter\cmd@to@cs\string#1}}
\def\cmd@to@cs#1#2{\char\number`#2\relax}
\DeclareRobustCommand\cs[1]{\texttt{\char`\\#1}}
\providecommand\marg[1]{%
  {\ttfamily\char`\{}\meta{#1}{\ttfamily\char`\}}}
\providecommand\oarg[1]{%
  {\ttfamily[}\meta{#1}{\ttfamily]}}
\providecommand\parg[1]{%
  {\ttfamily(}\meta{#1}{\ttfamily)}}
\raggedbottom

% -----------------------------------
\input{pgfplots.preamble.tex}

\makeatletter
% I want a two-column index just as in pgfmanual styles. This here
% was the best way to get one:
\def\index@prologue{\section*{Index}\addcontentsline{toc}{chapter}{Index}
}
\makeatother

%\RequirePackage[german,english,francais]{babel}

\def\matlabcolormaptext{This colormap is similar to one shipped with Matlab$^\text{\textregistered}$ under a similar name.}

\IfFileExists{tikzlibraryspy.code.tex}{%
\usetikzlibrary{spy}
}{%
    \message{ERROR: tikz SPY library NOT available. The manual will only compile partially.^^J}%
}%

\usepackage{xparse}% for colorbrewer manual

\usetikzlibrary{
    decorations.markings,
    decorations.footprints,
    shapes.arrows,
    matrix,
    positioning,
}

\usepgfplotslibrary{
    fillbetween,
    ternary,
    smithchart,
    patchplots,
    polar,
    colormaps,
    colorbrewer,
    colortol,           % docu not ready yet
}
\pgfqkeys{/codeexample}{%
    every codeexample/.append style={
        /pgfplots/every ternary axis/.append style={
            /pgfplots/legend style={fill=graphicbackground},
        }
    },
    tabsize=4,
}

\pgfplotsmanualenableexternalizationofexpensive

%\usetikzlibrary{external}
%\tikzexternalize[prefix=figures/]{pgfplots}

\title{%
    Manual for Package \PGFPlots{}\\
    {\small 2D/3D Plots in \LaTeX{}, Version \pgfplotsversion}\\
    {\small\url{https://github.com/pgf-tikz/pgfplots}}
    %\\{\small Attention: you are using an unstable development version.}
}

\makeatletter
\long\def\abstractsmuggle{%
    \centering
    \textbf{Abstract}\\[0.5cm]

    \begin{minipage}{12cm}
        \PGFPlots{} draws high-quality function plots in normal or logarithmic
        scaling with a user-friendly interface directly in \TeX{}. The user
        supplies axis labels, legend entries and the plot coordinates for one or
        more plots and \PGFPlots{} applies axis scaling, computes any logarithms
        and axis ticks and draws the plots. It supports line plots, scatter
        plots, piecewise constant plots, bar plots, area plots, mesh and surface
        plots, patch plots, contour plots, quiver plots, histogram plots, box
        plots, polar axes, ternary diagrams, smith charts and some more. It is
        based on Till Tantau's package \PGF{}/\Tikz{}.
    \end{minipage}

}%

\expandafter\date\expandafter{\@date\\[2cm]
    \abstractsmuggle
}%
\makeatother

%\includeonly{pgfplots.reference}


\begin{document}

\def\plotcoords{%
\addplot coordinates {
(5,8.312e-02)    (17,2.547e-02)   (49,7.407e-03)
(129,2.102e-03)  (321,5.874e-04)  (769,1.623e-04)
(1793,4.442e-05) (4097,1.207e-05) (9217,3.261e-06)
};

\addplot coordinates{
(7,8.472e-02)    (31,3.044e-02)    (111,1.022e-02)
(351,3.303e-03)  (1023,1.039e-03)  (2815,3.196e-04)
(7423,9.658e-05) (18943,2.873e-05) (47103,8.437e-06)
};

\addplot coordinates{
(9,7.881e-02)     (49,3.243e-02)    (209,1.232e-02)
(769,4.454e-03)   (2561,1.551e-03)  (7937,5.236e-04)
(23297,1.723e-04) (65537,5.545e-05) (178177,1.751e-05)
};

\addplot coordinates{
(11,6.887e-02)    (71,3.177e-02)     (351,1.341e-02)
(1471,5.334e-03)  (5503,2.027e-03)   (18943,7.415e-04)
(61183,2.628e-04) (187903,9.063e-05) (553983,3.053e-05)
};

\addplot coordinates{
(13,5.755e-02)     (97,2.925e-02)     (545,1.351e-02)
(2561,5.842e-03)   (10625,2.397e-03)  (40193,9.414e-04)
(141569,3.564e-04) (471041,1.308e-04)
(1496065,4.670e-05)
};
}%


\bookmaketitle
\tableofcontents
\include{pgfplots.title_abstract_intro}
\include{pgfplots.preliminaries}
\include{pgfplots.intro}
\include{pgfplots.reference}
\include{pgfplots.libs}
\include{pgfplots.resources}
\include{pgfplots.importexport}
\include{pgfplots.basic.reference}

\printindex

\bibliographystyle{abbrv} %gerapali} %gerabbrv} %gerunsrt.bst} %gerabbrv}% gerplain}
\nocite{pgfplotstable}
\nocite{programmingnotes}
\bibliography{pgfplots}
\end{document}

% -----------------------------------------------------------------------------
% For Stefan Pinnow as reminder on what to look for when editing the manual
% -----------------------------------------------------------------------------
% There should be no line breaks in the following environments
% - |...|
% - \declareandlabel{...}
% - \verbpdfref{...}
% If "MakeTikzPictures" isn't running through check one of the externalized
% LOG files what is the cause of that.
% -----------------------------------------------------------------------------


%%%%%%%%%%%%%%%%%%%%%%%%%%%%%%%%%%%%%%%%%%%%%%%%%%%%%%%%%%%%%%%%%%%%%%%%%%%%%
%
% Package pgfplots.sty documentation.
%
% Copyright 2007/2008 by Christian Feuersaenger.
%
% This program is free software: you can redistribute it and/or modify
% it under the terms of the GNU General Public License as published by
% the Free Software Foundation, either version 3 of the License, or
% (at your option) any later version.
%
% This program is distributed in the hope that it will be useful,
% but WITHOUT ANY WARRANTY; without even the implied warranty of
% MERCHANTABILITY or FITNESS FOR A PARTICULAR PURPOSE.  See the
% GNU General Public License for more details.
%
% You should have received a copy of the GNU General Public License
% along with this program.  If not, see <http://www.gnu.org/licenses/>.
%
%
%%%%%%%%%%%%%%%%%%%%%%%%%%%%%%%%%%%%%%%%%%%%%%%%%%%%%%%%%%%%%%%%%%%%%%%%%%%%%
% SEE pgfplots-macros.tex as well!
%\pdfminorversion=5 % to allow compression
%\pdfobjcompresslevel=2
\documentclass[a4paper,openany]{book}

\let\bookmaketitle=\maketitle

% -----------------------------------
% this here is from ltxdoc:
\usepackage{doc}[=v2]
\AtBeginDocument{\MakeShortVerb{\|}}
%\setlength{\textwidth}{355pt}
%\addtolength\marginparwidth{30pt}
%\addtolength\oddsidemargin{20pt}
%\addtolength\evensidemargin{20pt}
\def\cmd#1{\cs{\expandafter\cmd@to@cs\string#1}}
\def\cmd@to@cs#1#2{\char\number`#2\relax}
\DeclareRobustCommand\cs[1]{\texttt{\char`\\#1}}
\providecommand\marg[1]{%
  {\ttfamily\char`\{}\meta{#1}{\ttfamily\char`\}}}
\providecommand\oarg[1]{%
  {\ttfamily[}\meta{#1}{\ttfamily]}}
\providecommand\parg[1]{%
  {\ttfamily(}\meta{#1}{\ttfamily)}}
\raggedbottom

% -----------------------------------
\input{pgfplots.preamble.tex}

\makeatletter
% I want a two-column index just as in pgfmanual styles. This here
% was the best way to get one:
\def\index@prologue{\section*{Index}\addcontentsline{toc}{chapter}{Index}
}
\makeatother

%\RequirePackage[german,english,francais]{babel}

\def\matlabcolormaptext{This colormap is similar to one shipped with Matlab$^\text{\textregistered}$ under a similar name.}

\IfFileExists{tikzlibraryspy.code.tex}{%
\usetikzlibrary{spy}
}{%
    \message{ERROR: tikz SPY library NOT available. The manual will only compile partially.^^J}%
}%

\usepackage{xparse}% for colorbrewer manual

\usetikzlibrary{
    decorations.markings,
    decorations.footprints,
    shapes.arrows,
    matrix,
    positioning,
}

\usepgfplotslibrary{
    fillbetween,
    ternary,
    smithchart,
    patchplots,
    polar,
    colormaps,
    colorbrewer,
    colortol,           % docu not ready yet
}
\pgfqkeys{/codeexample}{%
    every codeexample/.append style={
        /pgfplots/every ternary axis/.append style={
            /pgfplots/legend style={fill=graphicbackground},
        }
    },
    tabsize=4,
}

\pgfplotsmanualenableexternalizationofexpensive

%\usetikzlibrary{external}
%\tikzexternalize[prefix=figures/]{pgfplots}

\title{%
    Manual for Package \PGFPlots{}\\
    {\small 2D/3D Plots in \LaTeX{}, Version \pgfplotsversion}\\
    {\small\url{https://github.com/pgf-tikz/pgfplots}}
    %\\{\small Attention: you are using an unstable development version.}
}

\makeatletter
\long\def\abstractsmuggle{%
    \centering
    \textbf{Abstract}\\[0.5cm]

    \begin{minipage}{12cm}
        \PGFPlots{} draws high-quality function plots in normal or logarithmic
        scaling with a user-friendly interface directly in \TeX{}. The user
        supplies axis labels, legend entries and the plot coordinates for one or
        more plots and \PGFPlots{} applies axis scaling, computes any logarithms
        and axis ticks and draws the plots. It supports line plots, scatter
        plots, piecewise constant plots, bar plots, area plots, mesh and surface
        plots, patch plots, contour plots, quiver plots, histogram plots, box
        plots, polar axes, ternary diagrams, smith charts and some more. It is
        based on Till Tantau's package \PGF{}/\Tikz{}.
    \end{minipage}

}%

\expandafter\date\expandafter{\@date\\[2cm]
    \abstractsmuggle
}%
\makeatother

%\includeonly{pgfplots.reference}


\begin{document}

\def\plotcoords{%
\addplot coordinates {
(5,8.312e-02)    (17,2.547e-02)   (49,7.407e-03)
(129,2.102e-03)  (321,5.874e-04)  (769,1.623e-04)
(1793,4.442e-05) (4097,1.207e-05) (9217,3.261e-06)
};

\addplot coordinates{
(7,8.472e-02)    (31,3.044e-02)    (111,1.022e-02)
(351,3.303e-03)  (1023,1.039e-03)  (2815,3.196e-04)
(7423,9.658e-05) (18943,2.873e-05) (47103,8.437e-06)
};

\addplot coordinates{
(9,7.881e-02)     (49,3.243e-02)    (209,1.232e-02)
(769,4.454e-03)   (2561,1.551e-03)  (7937,5.236e-04)
(23297,1.723e-04) (65537,5.545e-05) (178177,1.751e-05)
};

\addplot coordinates{
(11,6.887e-02)    (71,3.177e-02)     (351,1.341e-02)
(1471,5.334e-03)  (5503,2.027e-03)   (18943,7.415e-04)
(61183,2.628e-04) (187903,9.063e-05) (553983,3.053e-05)
};

\addplot coordinates{
(13,5.755e-02)     (97,2.925e-02)     (545,1.351e-02)
(2561,5.842e-03)   (10625,2.397e-03)  (40193,9.414e-04)
(141569,3.564e-04) (471041,1.308e-04)
(1496065,4.670e-05)
};
}%


\bookmaketitle
\tableofcontents
\include{pgfplots.title_abstract_intro}
\include{pgfplots.preliminaries}
\include{pgfplots.intro}
\include{pgfplots.reference}
\include{pgfplots.libs}
\include{pgfplots.resources}
\include{pgfplots.importexport}
\include{pgfplots.basic.reference}

\printindex

\bibliographystyle{abbrv} %gerapali} %gerabbrv} %gerunsrt.bst} %gerabbrv}% gerplain}
\nocite{pgfplotstable}
\nocite{programmingnotes}
\bibliography{pgfplots}
\end{document}

% -----------------------------------------------------------------------------
% For Stefan Pinnow as reminder on what to look for when editing the manual
% -----------------------------------------------------------------------------
% There should be no line breaks in the following environments
% - |...|
% - \declareandlabel{...}
% - \verbpdfref{...}
% If "MakeTikzPictures" isn't running through check one of the externalized
% LOG files what is the cause of that.
% -----------------------------------------------------------------------------


%%%%%%%%%%%%%%%%%%%%%%%%%%%%%%%%%%%%%%%%%%%%%%%%%%%%%%%%%%%%%%%%%%%%%%%%%%%%%
%
% Package pgfplots.sty documentation.
%
% Copyright 2007/2008 by Christian Feuersaenger.
%
% This program is free software: you can redistribute it and/or modify
% it under the terms of the GNU General Public License as published by
% the Free Software Foundation, either version 3 of the License, or
% (at your option) any later version.
%
% This program is distributed in the hope that it will be useful,
% but WITHOUT ANY WARRANTY; without even the implied warranty of
% MERCHANTABILITY or FITNESS FOR A PARTICULAR PURPOSE.  See the
% GNU General Public License for more details.
%
% You should have received a copy of the GNU General Public License
% along with this program.  If not, see <http://www.gnu.org/licenses/>.
%
%
%%%%%%%%%%%%%%%%%%%%%%%%%%%%%%%%%%%%%%%%%%%%%%%%%%%%%%%%%%%%%%%%%%%%%%%%%%%%%
% SEE pgfplots-macros.tex as well!
%\pdfminorversion=5 % to allow compression
%\pdfobjcompresslevel=2
\documentclass[a4paper,openany]{book}

\let\bookmaketitle=\maketitle

% -----------------------------------
% this here is from ltxdoc:
\usepackage{doc}[=v2]
\AtBeginDocument{\MakeShortVerb{\|}}
%\setlength{\textwidth}{355pt}
%\addtolength\marginparwidth{30pt}
%\addtolength\oddsidemargin{20pt}
%\addtolength\evensidemargin{20pt}
\def\cmd#1{\cs{\expandafter\cmd@to@cs\string#1}}
\def\cmd@to@cs#1#2{\char\number`#2\relax}
\DeclareRobustCommand\cs[1]{\texttt{\char`\\#1}}
\providecommand\marg[1]{%
  {\ttfamily\char`\{}\meta{#1}{\ttfamily\char`\}}}
\providecommand\oarg[1]{%
  {\ttfamily[}\meta{#1}{\ttfamily]}}
\providecommand\parg[1]{%
  {\ttfamily(}\meta{#1}{\ttfamily)}}
\raggedbottom

% -----------------------------------
\input{pgfplots.preamble.tex}

\makeatletter
% I want a two-column index just as in pgfmanual styles. This here
% was the best way to get one:
\def\index@prologue{\section*{Index}\addcontentsline{toc}{chapter}{Index}
}
\makeatother

%\RequirePackage[german,english,francais]{babel}

\def\matlabcolormaptext{This colormap is similar to one shipped with Matlab$^\text{\textregistered}$ under a similar name.}

\IfFileExists{tikzlibraryspy.code.tex}{%
\usetikzlibrary{spy}
}{%
    \message{ERROR: tikz SPY library NOT available. The manual will only compile partially.^^J}%
}%

\usepackage{xparse}% for colorbrewer manual

\usetikzlibrary{
    decorations.markings,
    decorations.footprints,
    shapes.arrows,
    matrix,
    positioning,
}

\usepgfplotslibrary{
    fillbetween,
    ternary,
    smithchart,
    patchplots,
    polar,
    colormaps,
    colorbrewer,
    colortol,           % docu not ready yet
}
\pgfqkeys{/codeexample}{%
    every codeexample/.append style={
        /pgfplots/every ternary axis/.append style={
            /pgfplots/legend style={fill=graphicbackground},
        }
    },
    tabsize=4,
}

\pgfplotsmanualenableexternalizationofexpensive

%\usetikzlibrary{external}
%\tikzexternalize[prefix=figures/]{pgfplots}

\title{%
    Manual for Package \PGFPlots{}\\
    {\small 2D/3D Plots in \LaTeX{}, Version \pgfplotsversion}\\
    {\small\url{https://github.com/pgf-tikz/pgfplots}}
    %\\{\small Attention: you are using an unstable development version.}
}

\makeatletter
\long\def\abstractsmuggle{%
    \centering
    \textbf{Abstract}\\[0.5cm]

    \begin{minipage}{12cm}
        \PGFPlots{} draws high-quality function plots in normal or logarithmic
        scaling with a user-friendly interface directly in \TeX{}. The user
        supplies axis labels, legend entries and the plot coordinates for one or
        more plots and \PGFPlots{} applies axis scaling, computes any logarithms
        and axis ticks and draws the plots. It supports line plots, scatter
        plots, piecewise constant plots, bar plots, area plots, mesh and surface
        plots, patch plots, contour plots, quiver plots, histogram plots, box
        plots, polar axes, ternary diagrams, smith charts and some more. It is
        based on Till Tantau's package \PGF{}/\Tikz{}.
    \end{minipage}

}%

\expandafter\date\expandafter{\@date\\[2cm]
    \abstractsmuggle
}%
\makeatother

%\includeonly{pgfplots.reference}


\begin{document}

\def\plotcoords{%
\addplot coordinates {
(5,8.312e-02)    (17,2.547e-02)   (49,7.407e-03)
(129,2.102e-03)  (321,5.874e-04)  (769,1.623e-04)
(1793,4.442e-05) (4097,1.207e-05) (9217,3.261e-06)
};

\addplot coordinates{
(7,8.472e-02)    (31,3.044e-02)    (111,1.022e-02)
(351,3.303e-03)  (1023,1.039e-03)  (2815,3.196e-04)
(7423,9.658e-05) (18943,2.873e-05) (47103,8.437e-06)
};

\addplot coordinates{
(9,7.881e-02)     (49,3.243e-02)    (209,1.232e-02)
(769,4.454e-03)   (2561,1.551e-03)  (7937,5.236e-04)
(23297,1.723e-04) (65537,5.545e-05) (178177,1.751e-05)
};

\addplot coordinates{
(11,6.887e-02)    (71,3.177e-02)     (351,1.341e-02)
(1471,5.334e-03)  (5503,2.027e-03)   (18943,7.415e-04)
(61183,2.628e-04) (187903,9.063e-05) (553983,3.053e-05)
};

\addplot coordinates{
(13,5.755e-02)     (97,2.925e-02)     (545,1.351e-02)
(2561,5.842e-03)   (10625,2.397e-03)  (40193,9.414e-04)
(141569,3.564e-04) (471041,1.308e-04)
(1496065,4.670e-05)
};
}%


\bookmaketitle
\tableofcontents
\include{pgfplots.title_abstract_intro}
\include{pgfplots.preliminaries}
\include{pgfplots.intro}
\include{pgfplots.reference}
\include{pgfplots.libs}
\include{pgfplots.resources}
\include{pgfplots.importexport}
\include{pgfplots.basic.reference}

\printindex

\bibliographystyle{abbrv} %gerapali} %gerabbrv} %gerunsrt.bst} %gerabbrv}% gerplain}
\nocite{pgfplotstable}
\nocite{programmingnotes}
\bibliography{pgfplots}
\end{document}

% -----------------------------------------------------------------------------
% For Stefan Pinnow as reminder on what to look for when editing the manual
% -----------------------------------------------------------------------------
% There should be no line breaks in the following environments
% - |...|
% - \declareandlabel{...}
% - \verbpdfref{...}
% If "MakeTikzPictures" isn't running through check one of the externalized
% LOG files what is the cause of that.
% -----------------------------------------------------------------------------


%%%%%%%%%%%%%%%%%%%%%%%%%%%%%%%%%%%%%%%%%%%%%%%%%%%%%%%%%%%%%%%%%%%%%%%%%%%%%
%
% Package pgfplots.sty documentation.
%
% Copyright 2007/2008 by Christian Feuersaenger.
%
% This program is free software: you can redistribute it and/or modify
% it under the terms of the GNU General Public License as published by
% the Free Software Foundation, either version 3 of the License, or
% (at your option) any later version.
%
% This program is distributed in the hope that it will be useful,
% but WITHOUT ANY WARRANTY; without even the implied warranty of
% MERCHANTABILITY or FITNESS FOR A PARTICULAR PURPOSE.  See the
% GNU General Public License for more details.
%
% You should have received a copy of the GNU General Public License
% along with this program.  If not, see <http://www.gnu.org/licenses/>.
%
%
%%%%%%%%%%%%%%%%%%%%%%%%%%%%%%%%%%%%%%%%%%%%%%%%%%%%%%%%%%%%%%%%%%%%%%%%%%%%%
% SEE pgfplots-macros.tex as well!
%\pdfminorversion=5 % to allow compression
%\pdfobjcompresslevel=2
\documentclass[a4paper,openany]{book}

\let\bookmaketitle=\maketitle

% -----------------------------------
% this here is from ltxdoc:
\usepackage{doc}[=v2]
\AtBeginDocument{\MakeShortVerb{\|}}
%\setlength{\textwidth}{355pt}
%\addtolength\marginparwidth{30pt}
%\addtolength\oddsidemargin{20pt}
%\addtolength\evensidemargin{20pt}
\def\cmd#1{\cs{\expandafter\cmd@to@cs\string#1}}
\def\cmd@to@cs#1#2{\char\number`#2\relax}
\DeclareRobustCommand\cs[1]{\texttt{\char`\\#1}}
\providecommand\marg[1]{%
  {\ttfamily\char`\{}\meta{#1}{\ttfamily\char`\}}}
\providecommand\oarg[1]{%
  {\ttfamily[}\meta{#1}{\ttfamily]}}
\providecommand\parg[1]{%
  {\ttfamily(}\meta{#1}{\ttfamily)}}
\raggedbottom

% -----------------------------------
\input{pgfplots.preamble.tex}

\makeatletter
% I want a two-column index just as in pgfmanual styles. This here
% was the best way to get one:
\def\index@prologue{\section*{Index}\addcontentsline{toc}{chapter}{Index}
}
\makeatother

%\RequirePackage[german,english,francais]{babel}

\def\matlabcolormaptext{This colormap is similar to one shipped with Matlab$^\text{\textregistered}$ under a similar name.}

\IfFileExists{tikzlibraryspy.code.tex}{%
\usetikzlibrary{spy}
}{%
    \message{ERROR: tikz SPY library NOT available. The manual will only compile partially.^^J}%
}%

\usepackage{xparse}% for colorbrewer manual

\usetikzlibrary{
    decorations.markings,
    decorations.footprints,
    shapes.arrows,
    matrix,
    positioning,
}

\usepgfplotslibrary{
    fillbetween,
    ternary,
    smithchart,
    patchplots,
    polar,
    colormaps,
    colorbrewer,
    colortol,           % docu not ready yet
}
\pgfqkeys{/codeexample}{%
    every codeexample/.append style={
        /pgfplots/every ternary axis/.append style={
            /pgfplots/legend style={fill=graphicbackground},
        }
    },
    tabsize=4,
}

\pgfplotsmanualenableexternalizationofexpensive

%\usetikzlibrary{external}
%\tikzexternalize[prefix=figures/]{pgfplots}

\title{%
    Manual for Package \PGFPlots{}\\
    {\small 2D/3D Plots in \LaTeX{}, Version \pgfplotsversion}\\
    {\small\url{https://github.com/pgf-tikz/pgfplots}}
    %\\{\small Attention: you are using an unstable development version.}
}

\makeatletter
\long\def\abstractsmuggle{%
    \centering
    \textbf{Abstract}\\[0.5cm]

    \begin{minipage}{12cm}
        \PGFPlots{} draws high-quality function plots in normal or logarithmic
        scaling with a user-friendly interface directly in \TeX{}. The user
        supplies axis labels, legend entries and the plot coordinates for one or
        more plots and \PGFPlots{} applies axis scaling, computes any logarithms
        and axis ticks and draws the plots. It supports line plots, scatter
        plots, piecewise constant plots, bar plots, area plots, mesh and surface
        plots, patch plots, contour plots, quiver plots, histogram plots, box
        plots, polar axes, ternary diagrams, smith charts and some more. It is
        based on Till Tantau's package \PGF{}/\Tikz{}.
    \end{minipage}

}%

\expandafter\date\expandafter{\@date\\[2cm]
    \abstractsmuggle
}%
\makeatother

%\includeonly{pgfplots.reference}


\begin{document}

\def\plotcoords{%
\addplot coordinates {
(5,8.312e-02)    (17,2.547e-02)   (49,7.407e-03)
(129,2.102e-03)  (321,5.874e-04)  (769,1.623e-04)
(1793,4.442e-05) (4097,1.207e-05) (9217,3.261e-06)
};

\addplot coordinates{
(7,8.472e-02)    (31,3.044e-02)    (111,1.022e-02)
(351,3.303e-03)  (1023,1.039e-03)  (2815,3.196e-04)
(7423,9.658e-05) (18943,2.873e-05) (47103,8.437e-06)
};

\addplot coordinates{
(9,7.881e-02)     (49,3.243e-02)    (209,1.232e-02)
(769,4.454e-03)   (2561,1.551e-03)  (7937,5.236e-04)
(23297,1.723e-04) (65537,5.545e-05) (178177,1.751e-05)
};

\addplot coordinates{
(11,6.887e-02)    (71,3.177e-02)     (351,1.341e-02)
(1471,5.334e-03)  (5503,2.027e-03)   (18943,7.415e-04)
(61183,2.628e-04) (187903,9.063e-05) (553983,3.053e-05)
};

\addplot coordinates{
(13,5.755e-02)     (97,2.925e-02)     (545,1.351e-02)
(2561,5.842e-03)   (10625,2.397e-03)  (40193,9.414e-04)
(141569,3.564e-04) (471041,1.308e-04)
(1496065,4.670e-05)
};
}%


\bookmaketitle
\tableofcontents
\include{pgfplots.title_abstract_intro}
\include{pgfplots.preliminaries}
\include{pgfplots.intro}
\include{pgfplots.reference}
\include{pgfplots.libs}
\include{pgfplots.resources}
\include{pgfplots.importexport}
\include{pgfplots.basic.reference}

\printindex

\bibliographystyle{abbrv} %gerapali} %gerabbrv} %gerunsrt.bst} %gerabbrv}% gerplain}
\nocite{pgfplotstable}
\nocite{programmingnotes}
\bibliography{pgfplots}
\end{document}

% -----------------------------------------------------------------------------
% For Stefan Pinnow as reminder on what to look for when editing the manual
% -----------------------------------------------------------------------------
% There should be no line breaks in the following environments
% - |...|
% - \declareandlabel{...}
% - \verbpdfref{...}
% If "MakeTikzPictures" isn't running through check one of the externalized
% LOG files what is the cause of that.
% -----------------------------------------------------------------------------


%%%%%%%%%%%%%%%%%%%%%%%%%%%%%%%%%%%%%%%%%%%%%%%%%%%%%%%%%%%%%%%%%%%%%%%%%%%%%
%
% Package pgfplots.sty documentation.
%
% Copyright 2007/2008 by Christian Feuersaenger.
%
% This program is free software: you can redistribute it and/or modify
% it under the terms of the GNU General Public License as published by
% the Free Software Foundation, either version 3 of the License, or
% (at your option) any later version.
%
% This program is distributed in the hope that it will be useful,
% but WITHOUT ANY WARRANTY; without even the implied warranty of
% MERCHANTABILITY or FITNESS FOR A PARTICULAR PURPOSE.  See the
% GNU General Public License for more details.
%
% You should have received a copy of the GNU General Public License
% along with this program.  If not, see <http://www.gnu.org/licenses/>.
%
%
%%%%%%%%%%%%%%%%%%%%%%%%%%%%%%%%%%%%%%%%%%%%%%%%%%%%%%%%%%%%%%%%%%%%%%%%%%%%%
% SEE pgfplots-macros.tex as well!
%\pdfminorversion=5 % to allow compression
%\pdfobjcompresslevel=2
\documentclass[a4paper,openany]{book}

\let\bookmaketitle=\maketitle

% -----------------------------------
% this here is from ltxdoc:
\usepackage{doc}[=v2]
\AtBeginDocument{\MakeShortVerb{\|}}
%\setlength{\textwidth}{355pt}
%\addtolength\marginparwidth{30pt}
%\addtolength\oddsidemargin{20pt}
%\addtolength\evensidemargin{20pt}
\def\cmd#1{\cs{\expandafter\cmd@to@cs\string#1}}
\def\cmd@to@cs#1#2{\char\number`#2\relax}
\DeclareRobustCommand\cs[1]{\texttt{\char`\\#1}}
\providecommand\marg[1]{%
  {\ttfamily\char`\{}\meta{#1}{\ttfamily\char`\}}}
\providecommand\oarg[1]{%
  {\ttfamily[}\meta{#1}{\ttfamily]}}
\providecommand\parg[1]{%
  {\ttfamily(}\meta{#1}{\ttfamily)}}
\raggedbottom

% -----------------------------------
\input{pgfplots.preamble.tex}

\makeatletter
% I want a two-column index just as in pgfmanual styles. This here
% was the best way to get one:
\def\index@prologue{\section*{Index}\addcontentsline{toc}{chapter}{Index}
}
\makeatother

%\RequirePackage[german,english,francais]{babel}

\def\matlabcolormaptext{This colormap is similar to one shipped with Matlab$^\text{\textregistered}$ under a similar name.}

\IfFileExists{tikzlibraryspy.code.tex}{%
\usetikzlibrary{spy}
}{%
    \message{ERROR: tikz SPY library NOT available. The manual will only compile partially.^^J}%
}%

\usepackage{xparse}% for colorbrewer manual

\usetikzlibrary{
    decorations.markings,
    decorations.footprints,
    shapes.arrows,
    matrix,
    positioning,
}

\usepgfplotslibrary{
    fillbetween,
    ternary,
    smithchart,
    patchplots,
    polar,
    colormaps,
    colorbrewer,
    colortol,           % docu not ready yet
}
\pgfqkeys{/codeexample}{%
    every codeexample/.append style={
        /pgfplots/every ternary axis/.append style={
            /pgfplots/legend style={fill=graphicbackground},
        }
    },
    tabsize=4,
}

\pgfplotsmanualenableexternalizationofexpensive

%\usetikzlibrary{external}
%\tikzexternalize[prefix=figures/]{pgfplots}

\title{%
    Manual for Package \PGFPlots{}\\
    {\small 2D/3D Plots in \LaTeX{}, Version \pgfplotsversion}\\
    {\small\url{https://github.com/pgf-tikz/pgfplots}}
    %\\{\small Attention: you are using an unstable development version.}
}

\makeatletter
\long\def\abstractsmuggle{%
    \centering
    \textbf{Abstract}\\[0.5cm]

    \begin{minipage}{12cm}
        \PGFPlots{} draws high-quality function plots in normal or logarithmic
        scaling with a user-friendly interface directly in \TeX{}. The user
        supplies axis labels, legend entries and the plot coordinates for one or
        more plots and \PGFPlots{} applies axis scaling, computes any logarithms
        and axis ticks and draws the plots. It supports line plots, scatter
        plots, piecewise constant plots, bar plots, area plots, mesh and surface
        plots, patch plots, contour plots, quiver plots, histogram plots, box
        plots, polar axes, ternary diagrams, smith charts and some more. It is
        based on Till Tantau's package \PGF{}/\Tikz{}.
    \end{minipage}

}%

\expandafter\date\expandafter{\@date\\[2cm]
    \abstractsmuggle
}%
\makeatother

%\includeonly{pgfplots.reference}


\begin{document}

\def\plotcoords{%
\addplot coordinates {
(5,8.312e-02)    (17,2.547e-02)   (49,7.407e-03)
(129,2.102e-03)  (321,5.874e-04)  (769,1.623e-04)
(1793,4.442e-05) (4097,1.207e-05) (9217,3.261e-06)
};

\addplot coordinates{
(7,8.472e-02)    (31,3.044e-02)    (111,1.022e-02)
(351,3.303e-03)  (1023,1.039e-03)  (2815,3.196e-04)
(7423,9.658e-05) (18943,2.873e-05) (47103,8.437e-06)
};

\addplot coordinates{
(9,7.881e-02)     (49,3.243e-02)    (209,1.232e-02)
(769,4.454e-03)   (2561,1.551e-03)  (7937,5.236e-04)
(23297,1.723e-04) (65537,5.545e-05) (178177,1.751e-05)
};

\addplot coordinates{
(11,6.887e-02)    (71,3.177e-02)     (351,1.341e-02)
(1471,5.334e-03)  (5503,2.027e-03)   (18943,7.415e-04)
(61183,2.628e-04) (187903,9.063e-05) (553983,3.053e-05)
};

\addplot coordinates{
(13,5.755e-02)     (97,2.925e-02)     (545,1.351e-02)
(2561,5.842e-03)   (10625,2.397e-03)  (40193,9.414e-04)
(141569,3.564e-04) (471041,1.308e-04)
(1496065,4.670e-05)
};
}%


\bookmaketitle
\tableofcontents
\include{pgfplots.title_abstract_intro}
\include{pgfplots.preliminaries}
\include{pgfplots.intro}
\include{pgfplots.reference}
\include{pgfplots.libs}
\include{pgfplots.resources}
\include{pgfplots.importexport}
\include{pgfplots.basic.reference}

\printindex

\bibliographystyle{abbrv} %gerapali} %gerabbrv} %gerunsrt.bst} %gerabbrv}% gerplain}
\nocite{pgfplotstable}
\nocite{programmingnotes}
\bibliography{pgfplots}
\end{document}

% -----------------------------------------------------------------------------
% For Stefan Pinnow as reminder on what to look for when editing the manual
% -----------------------------------------------------------------------------
% There should be no line breaks in the following environments
% - |...|
% - \declareandlabel{...}
% - \verbpdfref{...}
% If "MakeTikzPictures" isn't running through check one of the externalized
% LOG files what is the cause of that.
% -----------------------------------------------------------------------------


%%%%%%%%%%%%%%%%%%%%%%%%%%%%%%%%%%%%%%%%%%%%%%%%%%%%%%%%%%%%%%%%%%%%%%%%%%%%%
%
% Package pgfplots.sty documentation.
%
% Copyright 2007/2008 by Christian Feuersaenger.
%
% This program is free software: you can redistribute it and/or modify
% it under the terms of the GNU General Public License as published by
% the Free Software Foundation, either version 3 of the License, or
% (at your option) any later version.
%
% This program is distributed in the hope that it will be useful,
% but WITHOUT ANY WARRANTY; without even the implied warranty of
% MERCHANTABILITY or FITNESS FOR A PARTICULAR PURPOSE.  See the
% GNU General Public License for more details.
%
% You should have received a copy of the GNU General Public License
% along with this program.  If not, see <http://www.gnu.org/licenses/>.
%
%
%%%%%%%%%%%%%%%%%%%%%%%%%%%%%%%%%%%%%%%%%%%%%%%%%%%%%%%%%%%%%%%%%%%%%%%%%%%%%
% SEE pgfplots-macros.tex as well!
%\pdfminorversion=5 % to allow compression
%\pdfobjcompresslevel=2
\documentclass[a4paper,openany]{book}

\let\bookmaketitle=\maketitle

% -----------------------------------
% this here is from ltxdoc:
\usepackage{doc}[=v2]
\AtBeginDocument{\MakeShortVerb{\|}}
%\setlength{\textwidth}{355pt}
%\addtolength\marginparwidth{30pt}
%\addtolength\oddsidemargin{20pt}
%\addtolength\evensidemargin{20pt}
\def\cmd#1{\cs{\expandafter\cmd@to@cs\string#1}}
\def\cmd@to@cs#1#2{\char\number`#2\relax}
\DeclareRobustCommand\cs[1]{\texttt{\char`\\#1}}
\providecommand\marg[1]{%
  {\ttfamily\char`\{}\meta{#1}{\ttfamily\char`\}}}
\providecommand\oarg[1]{%
  {\ttfamily[}\meta{#1}{\ttfamily]}}
\providecommand\parg[1]{%
  {\ttfamily(}\meta{#1}{\ttfamily)}}
\raggedbottom

% -----------------------------------
\input{pgfplots.preamble.tex}

\makeatletter
% I want a two-column index just as in pgfmanual styles. This here
% was the best way to get one:
\def\index@prologue{\section*{Index}\addcontentsline{toc}{chapter}{Index}
}
\makeatother

%\RequirePackage[german,english,francais]{babel}

\def\matlabcolormaptext{This colormap is similar to one shipped with Matlab$^\text{\textregistered}$ under a similar name.}

\IfFileExists{tikzlibraryspy.code.tex}{%
\usetikzlibrary{spy}
}{%
    \message{ERROR: tikz SPY library NOT available. The manual will only compile partially.^^J}%
}%

\usepackage{xparse}% for colorbrewer manual

\usetikzlibrary{
    decorations.markings,
    decorations.footprints,
    shapes.arrows,
    matrix,
    positioning,
}

\usepgfplotslibrary{
    fillbetween,
    ternary,
    smithchart,
    patchplots,
    polar,
    colormaps,
    colorbrewer,
    colortol,           % docu not ready yet
}
\pgfqkeys{/codeexample}{%
    every codeexample/.append style={
        /pgfplots/every ternary axis/.append style={
            /pgfplots/legend style={fill=graphicbackground},
        }
    },
    tabsize=4,
}

\pgfplotsmanualenableexternalizationofexpensive

%\usetikzlibrary{external}
%\tikzexternalize[prefix=figures/]{pgfplots}

\title{%
    Manual for Package \PGFPlots{}\\
    {\small 2D/3D Plots in \LaTeX{}, Version \pgfplotsversion}\\
    {\small\url{https://github.com/pgf-tikz/pgfplots}}
    %\\{\small Attention: you are using an unstable development version.}
}

\makeatletter
\long\def\abstractsmuggle{%
    \centering
    \textbf{Abstract}\\[0.5cm]

    \begin{minipage}{12cm}
        \PGFPlots{} draws high-quality function plots in normal or logarithmic
        scaling with a user-friendly interface directly in \TeX{}. The user
        supplies axis labels, legend entries and the plot coordinates for one or
        more plots and \PGFPlots{} applies axis scaling, computes any logarithms
        and axis ticks and draws the plots. It supports line plots, scatter
        plots, piecewise constant plots, bar plots, area plots, mesh and surface
        plots, patch plots, contour plots, quiver plots, histogram plots, box
        plots, polar axes, ternary diagrams, smith charts and some more. It is
        based on Till Tantau's package \PGF{}/\Tikz{}.
    \end{minipage}

}%

\expandafter\date\expandafter{\@date\\[2cm]
    \abstractsmuggle
}%
\makeatother

%\includeonly{pgfplots.reference}


\begin{document}

\def\plotcoords{%
\addplot coordinates {
(5,8.312e-02)    (17,2.547e-02)   (49,7.407e-03)
(129,2.102e-03)  (321,5.874e-04)  (769,1.623e-04)
(1793,4.442e-05) (4097,1.207e-05) (9217,3.261e-06)
};

\addplot coordinates{
(7,8.472e-02)    (31,3.044e-02)    (111,1.022e-02)
(351,3.303e-03)  (1023,1.039e-03)  (2815,3.196e-04)
(7423,9.658e-05) (18943,2.873e-05) (47103,8.437e-06)
};

\addplot coordinates{
(9,7.881e-02)     (49,3.243e-02)    (209,1.232e-02)
(769,4.454e-03)   (2561,1.551e-03)  (7937,5.236e-04)
(23297,1.723e-04) (65537,5.545e-05) (178177,1.751e-05)
};

\addplot coordinates{
(11,6.887e-02)    (71,3.177e-02)     (351,1.341e-02)
(1471,5.334e-03)  (5503,2.027e-03)   (18943,7.415e-04)
(61183,2.628e-04) (187903,9.063e-05) (553983,3.053e-05)
};

\addplot coordinates{
(13,5.755e-02)     (97,2.925e-02)     (545,1.351e-02)
(2561,5.842e-03)   (10625,2.397e-03)  (40193,9.414e-04)
(141569,3.564e-04) (471041,1.308e-04)
(1496065,4.670e-05)
};
}%


\bookmaketitle
\tableofcontents
\include{pgfplots.title_abstract_intro}
\include{pgfplots.preliminaries}
\include{pgfplots.intro}
\include{pgfplots.reference}
\include{pgfplots.libs}
\include{pgfplots.resources}
\include{pgfplots.importexport}
\include{pgfplots.basic.reference}

\printindex

\bibliographystyle{abbrv} %gerapali} %gerabbrv} %gerunsrt.bst} %gerabbrv}% gerplain}
\nocite{pgfplotstable}
\nocite{programmingnotes}
\bibliography{pgfplots}
\end{document}

% -----------------------------------------------------------------------------
% For Stefan Pinnow as reminder on what to look for when editing the manual
% -----------------------------------------------------------------------------
% There should be no line breaks in the following environments
% - |...|
% - \declareandlabel{...}
% - \verbpdfref{...}
% If "MakeTikzPictures" isn't running through check one of the externalized
% LOG files what is the cause of that.
% -----------------------------------------------------------------------------

\include{pgfplots.basic.reference}

\printindex

\bibliographystyle{abbrv} %gerapali} %gerabbrv} %gerunsrt.bst} %gerabbrv}% gerplain}
\nocite{pgfplotstable}
\nocite{programmingnotes}
\bibliography{pgfplots}
\end{document}

% -----------------------------------------------------------------------------
% For Stefan Pinnow as reminder on what to look for when editing the manual
% -----------------------------------------------------------------------------
% There should be no line breaks in the following environments
% - |...|
% - \declareandlabel{...}
% - \verbpdfref{...}
% If "MakeTikzPictures" isn't running through check one of the externalized
% LOG files what is the cause of that.
% -----------------------------------------------------------------------------


%%%%%%%%%%%%%%%%%%%%%%%%%%%%%%%%%%%%%%%%%%%%%%%%%%%%%%%%%%%%%%%%%%%%%%%%%%%%%
%
% Package pgfplots.sty documentation.
%
% Copyright 2007/2008 by Christian Feuersaenger.
%
% This program is free software: you can redistribute it and/or modify
% it under the terms of the GNU General Public License as published by
% the Free Software Foundation, either version 3 of the License, or
% (at your option) any later version.
%
% This program is distributed in the hope that it will be useful,
% but WITHOUT ANY WARRANTY; without even the implied warranty of
% MERCHANTABILITY or FITNESS FOR A PARTICULAR PURPOSE.  See the
% GNU General Public License for more details.
%
% You should have received a copy of the GNU General Public License
% along with this program.  If not, see <http://www.gnu.org/licenses/>.
%
%
%%%%%%%%%%%%%%%%%%%%%%%%%%%%%%%%%%%%%%%%%%%%%%%%%%%%%%%%%%%%%%%%%%%%%%%%%%%%%
% SEE pgfplots-macros.tex as well!
%\pdfminorversion=5 % to allow compression
%\pdfobjcompresslevel=2
\documentclass[a4paper,openany]{book}

\let\bookmaketitle=\maketitle

% -----------------------------------
% this here is from ltxdoc:
\usepackage{doc}[=v2]
\AtBeginDocument{\MakeShortVerb{\|}}
%\setlength{\textwidth}{355pt}
%\addtolength\marginparwidth{30pt}
%\addtolength\oddsidemargin{20pt}
%\addtolength\evensidemargin{20pt}
\def\cmd#1{\cs{\expandafter\cmd@to@cs\string#1}}
\def\cmd@to@cs#1#2{\char\number`#2\relax}
\DeclareRobustCommand\cs[1]{\texttt{\char`\\#1}}
\providecommand\marg[1]{%
  {\ttfamily\char`\{}\meta{#1}{\ttfamily\char`\}}}
\providecommand\oarg[1]{%
  {\ttfamily[}\meta{#1}{\ttfamily]}}
\providecommand\parg[1]{%
  {\ttfamily(}\meta{#1}{\ttfamily)}}
\raggedbottom

% -----------------------------------
\input{pgfplots.preamble.tex}

\makeatletter
% I want a two-column index just as in pgfmanual styles. This here
% was the best way to get one:
\def\index@prologue{\section*{Index}\addcontentsline{toc}{chapter}{Index}
}
\makeatother

%\RequirePackage[german,english,francais]{babel}

\def\matlabcolormaptext{This colormap is similar to one shipped with Matlab$^\text{\textregistered}$ under a similar name.}

\IfFileExists{tikzlibraryspy.code.tex}{%
\usetikzlibrary{spy}
}{%
    \message{ERROR: tikz SPY library NOT available. The manual will only compile partially.^^J}%
}%

\usepackage{xparse}% for colorbrewer manual

\usetikzlibrary{
    decorations.markings,
    decorations.footprints,
    shapes.arrows,
    matrix,
    positioning,
}

\usepgfplotslibrary{
    fillbetween,
    ternary,
    smithchart,
    patchplots,
    polar,
    colormaps,
    colorbrewer,
    colortol,           % docu not ready yet
}
\pgfqkeys{/codeexample}{%
    every codeexample/.append style={
        /pgfplots/every ternary axis/.append style={
            /pgfplots/legend style={fill=graphicbackground},
        }
    },
    tabsize=4,
}

\pgfplotsmanualenableexternalizationofexpensive

%\usetikzlibrary{external}
%\tikzexternalize[prefix=figures/]{pgfplots}

\title{%
    Manual for Package \PGFPlots{}\\
    {\small 2D/3D Plots in \LaTeX{}, Version \pgfplotsversion}\\
    {\small\url{https://github.com/pgf-tikz/pgfplots}}
    %\\{\small Attention: you are using an unstable development version.}
}

\makeatletter
\long\def\abstractsmuggle{%
    \centering
    \textbf{Abstract}\\[0.5cm]

    \begin{minipage}{12cm}
        \PGFPlots{} draws high-quality function plots in normal or logarithmic
        scaling with a user-friendly interface directly in \TeX{}. The user
        supplies axis labels, legend entries and the plot coordinates for one or
        more plots and \PGFPlots{} applies axis scaling, computes any logarithms
        and axis ticks and draws the plots. It supports line plots, scatter
        plots, piecewise constant plots, bar plots, area plots, mesh and surface
        plots, patch plots, contour plots, quiver plots, histogram plots, box
        plots, polar axes, ternary diagrams, smith charts and some more. It is
        based on Till Tantau's package \PGF{}/\Tikz{}.
    \end{minipage}

}%

\expandafter\date\expandafter{\@date\\[2cm]
    \abstractsmuggle
}%
\makeatother

%\includeonly{pgfplots.reference}


\begin{document}

\def\plotcoords{%
\addplot coordinates {
(5,8.312e-02)    (17,2.547e-02)   (49,7.407e-03)
(129,2.102e-03)  (321,5.874e-04)  (769,1.623e-04)
(1793,4.442e-05) (4097,1.207e-05) (9217,3.261e-06)
};

\addplot coordinates{
(7,8.472e-02)    (31,3.044e-02)    (111,1.022e-02)
(351,3.303e-03)  (1023,1.039e-03)  (2815,3.196e-04)
(7423,9.658e-05) (18943,2.873e-05) (47103,8.437e-06)
};

\addplot coordinates{
(9,7.881e-02)     (49,3.243e-02)    (209,1.232e-02)
(769,4.454e-03)   (2561,1.551e-03)  (7937,5.236e-04)
(23297,1.723e-04) (65537,5.545e-05) (178177,1.751e-05)
};

\addplot coordinates{
(11,6.887e-02)    (71,3.177e-02)     (351,1.341e-02)
(1471,5.334e-03)  (5503,2.027e-03)   (18943,7.415e-04)
(61183,2.628e-04) (187903,9.063e-05) (553983,3.053e-05)
};

\addplot coordinates{
(13,5.755e-02)     (97,2.925e-02)     (545,1.351e-02)
(2561,5.842e-03)   (10625,2.397e-03)  (40193,9.414e-04)
(141569,3.564e-04) (471041,1.308e-04)
(1496065,4.670e-05)
};
}%


\bookmaketitle
\tableofcontents

%%%%%%%%%%%%%%%%%%%%%%%%%%%%%%%%%%%%%%%%%%%%%%%%%%%%%%%%%%%%%%%%%%%%%%%%%%%%%
%
% Package pgfplots.sty documentation.
%
% Copyright 2007/2008 by Christian Feuersaenger.
%
% This program is free software: you can redistribute it and/or modify
% it under the terms of the GNU General Public License as published by
% the Free Software Foundation, either version 3 of the License, or
% (at your option) any later version.
%
% This program is distributed in the hope that it will be useful,
% but WITHOUT ANY WARRANTY; without even the implied warranty of
% MERCHANTABILITY or FITNESS FOR A PARTICULAR PURPOSE.  See the
% GNU General Public License for more details.
%
% You should have received a copy of the GNU General Public License
% along with this program.  If not, see <http://www.gnu.org/licenses/>.
%
%
%%%%%%%%%%%%%%%%%%%%%%%%%%%%%%%%%%%%%%%%%%%%%%%%%%%%%%%%%%%%%%%%%%%%%%%%%%%%%
% SEE pgfplots-macros.tex as well!
%\pdfminorversion=5 % to allow compression
%\pdfobjcompresslevel=2
\documentclass[a4paper,openany]{book}

\let\bookmaketitle=\maketitle

% -----------------------------------
% this here is from ltxdoc:
\usepackage{doc}[=v2]
\AtBeginDocument{\MakeShortVerb{\|}}
%\setlength{\textwidth}{355pt}
%\addtolength\marginparwidth{30pt}
%\addtolength\oddsidemargin{20pt}
%\addtolength\evensidemargin{20pt}
\def\cmd#1{\cs{\expandafter\cmd@to@cs\string#1}}
\def\cmd@to@cs#1#2{\char\number`#2\relax}
\DeclareRobustCommand\cs[1]{\texttt{\char`\\#1}}
\providecommand\marg[1]{%
  {\ttfamily\char`\{}\meta{#1}{\ttfamily\char`\}}}
\providecommand\oarg[1]{%
  {\ttfamily[}\meta{#1}{\ttfamily]}}
\providecommand\parg[1]{%
  {\ttfamily(}\meta{#1}{\ttfamily)}}
\raggedbottom

% -----------------------------------
\input{pgfplots.preamble.tex}

\makeatletter
% I want a two-column index just as in pgfmanual styles. This here
% was the best way to get one:
\def\index@prologue{\section*{Index}\addcontentsline{toc}{chapter}{Index}
}
\makeatother

%\RequirePackage[german,english,francais]{babel}

\def\matlabcolormaptext{This colormap is similar to one shipped with Matlab$^\text{\textregistered}$ under a similar name.}

\IfFileExists{tikzlibraryspy.code.tex}{%
\usetikzlibrary{spy}
}{%
    \message{ERROR: tikz SPY library NOT available. The manual will only compile partially.^^J}%
}%

\usepackage{xparse}% for colorbrewer manual

\usetikzlibrary{
    decorations.markings,
    decorations.footprints,
    shapes.arrows,
    matrix,
    positioning,
}

\usepgfplotslibrary{
    fillbetween,
    ternary,
    smithchart,
    patchplots,
    polar,
    colormaps,
    colorbrewer,
    colortol,           % docu not ready yet
}
\pgfqkeys{/codeexample}{%
    every codeexample/.append style={
        /pgfplots/every ternary axis/.append style={
            /pgfplots/legend style={fill=graphicbackground},
        }
    },
    tabsize=4,
}

\pgfplotsmanualenableexternalizationofexpensive

%\usetikzlibrary{external}
%\tikzexternalize[prefix=figures/]{pgfplots}

\title{%
    Manual for Package \PGFPlots{}\\
    {\small 2D/3D Plots in \LaTeX{}, Version \pgfplotsversion}\\
    {\small\url{https://github.com/pgf-tikz/pgfplots}}
    %\\{\small Attention: you are using an unstable development version.}
}

\makeatletter
\long\def\abstractsmuggle{%
    \centering
    \textbf{Abstract}\\[0.5cm]

    \begin{minipage}{12cm}
        \PGFPlots{} draws high-quality function plots in normal or logarithmic
        scaling with a user-friendly interface directly in \TeX{}. The user
        supplies axis labels, legend entries and the plot coordinates for one or
        more plots and \PGFPlots{} applies axis scaling, computes any logarithms
        and axis ticks and draws the plots. It supports line plots, scatter
        plots, piecewise constant plots, bar plots, area plots, mesh and surface
        plots, patch plots, contour plots, quiver plots, histogram plots, box
        plots, polar axes, ternary diagrams, smith charts and some more. It is
        based on Till Tantau's package \PGF{}/\Tikz{}.
    \end{minipage}

}%

\expandafter\date\expandafter{\@date\\[2cm]
    \abstractsmuggle
}%
\makeatother

%\includeonly{pgfplots.reference}


\begin{document}

\def\plotcoords{%
\addplot coordinates {
(5,8.312e-02)    (17,2.547e-02)   (49,7.407e-03)
(129,2.102e-03)  (321,5.874e-04)  (769,1.623e-04)
(1793,4.442e-05) (4097,1.207e-05) (9217,3.261e-06)
};

\addplot coordinates{
(7,8.472e-02)    (31,3.044e-02)    (111,1.022e-02)
(351,3.303e-03)  (1023,1.039e-03)  (2815,3.196e-04)
(7423,9.658e-05) (18943,2.873e-05) (47103,8.437e-06)
};

\addplot coordinates{
(9,7.881e-02)     (49,3.243e-02)    (209,1.232e-02)
(769,4.454e-03)   (2561,1.551e-03)  (7937,5.236e-04)
(23297,1.723e-04) (65537,5.545e-05) (178177,1.751e-05)
};

\addplot coordinates{
(11,6.887e-02)    (71,3.177e-02)     (351,1.341e-02)
(1471,5.334e-03)  (5503,2.027e-03)   (18943,7.415e-04)
(61183,2.628e-04) (187903,9.063e-05) (553983,3.053e-05)
};

\addplot coordinates{
(13,5.755e-02)     (97,2.925e-02)     (545,1.351e-02)
(2561,5.842e-03)   (10625,2.397e-03)  (40193,9.414e-04)
(141569,3.564e-04) (471041,1.308e-04)
(1496065,4.670e-05)
};
}%


\bookmaketitle
\tableofcontents
\include{pgfplots.title_abstract_intro}
\include{pgfplots.preliminaries}
\include{pgfplots.intro}
\include{pgfplots.reference}
\include{pgfplots.libs}
\include{pgfplots.resources}
\include{pgfplots.importexport}
\include{pgfplots.basic.reference}

\printindex

\bibliographystyle{abbrv} %gerapali} %gerabbrv} %gerunsrt.bst} %gerabbrv}% gerplain}
\nocite{pgfplotstable}
\nocite{programmingnotes}
\bibliography{pgfplots}
\end{document}

% -----------------------------------------------------------------------------
% For Stefan Pinnow as reminder on what to look for when editing the manual
% -----------------------------------------------------------------------------
% There should be no line breaks in the following environments
% - |...|
% - \declareandlabel{...}
% - \verbpdfref{...}
% If "MakeTikzPictures" isn't running through check one of the externalized
% LOG files what is the cause of that.
% -----------------------------------------------------------------------------


%%%%%%%%%%%%%%%%%%%%%%%%%%%%%%%%%%%%%%%%%%%%%%%%%%%%%%%%%%%%%%%%%%%%%%%%%%%%%
%
% Package pgfplots.sty documentation.
%
% Copyright 2007/2008 by Christian Feuersaenger.
%
% This program is free software: you can redistribute it and/or modify
% it under the terms of the GNU General Public License as published by
% the Free Software Foundation, either version 3 of the License, or
% (at your option) any later version.
%
% This program is distributed in the hope that it will be useful,
% but WITHOUT ANY WARRANTY; without even the implied warranty of
% MERCHANTABILITY or FITNESS FOR A PARTICULAR PURPOSE.  See the
% GNU General Public License for more details.
%
% You should have received a copy of the GNU General Public License
% along with this program.  If not, see <http://www.gnu.org/licenses/>.
%
%
%%%%%%%%%%%%%%%%%%%%%%%%%%%%%%%%%%%%%%%%%%%%%%%%%%%%%%%%%%%%%%%%%%%%%%%%%%%%%
% SEE pgfplots-macros.tex as well!
%\pdfminorversion=5 % to allow compression
%\pdfobjcompresslevel=2
\documentclass[a4paper,openany]{book}

\let\bookmaketitle=\maketitle

% -----------------------------------
% this here is from ltxdoc:
\usepackage{doc}[=v2]
\AtBeginDocument{\MakeShortVerb{\|}}
%\setlength{\textwidth}{355pt}
%\addtolength\marginparwidth{30pt}
%\addtolength\oddsidemargin{20pt}
%\addtolength\evensidemargin{20pt}
\def\cmd#1{\cs{\expandafter\cmd@to@cs\string#1}}
\def\cmd@to@cs#1#2{\char\number`#2\relax}
\DeclareRobustCommand\cs[1]{\texttt{\char`\\#1}}
\providecommand\marg[1]{%
  {\ttfamily\char`\{}\meta{#1}{\ttfamily\char`\}}}
\providecommand\oarg[1]{%
  {\ttfamily[}\meta{#1}{\ttfamily]}}
\providecommand\parg[1]{%
  {\ttfamily(}\meta{#1}{\ttfamily)}}
\raggedbottom

% -----------------------------------
\input{pgfplots.preamble.tex}

\makeatletter
% I want a two-column index just as in pgfmanual styles. This here
% was the best way to get one:
\def\index@prologue{\section*{Index}\addcontentsline{toc}{chapter}{Index}
}
\makeatother

%\RequirePackage[german,english,francais]{babel}

\def\matlabcolormaptext{This colormap is similar to one shipped with Matlab$^\text{\textregistered}$ under a similar name.}

\IfFileExists{tikzlibraryspy.code.tex}{%
\usetikzlibrary{spy}
}{%
    \message{ERROR: tikz SPY library NOT available. The manual will only compile partially.^^J}%
}%

\usepackage{xparse}% for colorbrewer manual

\usetikzlibrary{
    decorations.markings,
    decorations.footprints,
    shapes.arrows,
    matrix,
    positioning,
}

\usepgfplotslibrary{
    fillbetween,
    ternary,
    smithchart,
    patchplots,
    polar,
    colormaps,
    colorbrewer,
    colortol,           % docu not ready yet
}
\pgfqkeys{/codeexample}{%
    every codeexample/.append style={
        /pgfplots/every ternary axis/.append style={
            /pgfplots/legend style={fill=graphicbackground},
        }
    },
    tabsize=4,
}

\pgfplotsmanualenableexternalizationofexpensive

%\usetikzlibrary{external}
%\tikzexternalize[prefix=figures/]{pgfplots}

\title{%
    Manual for Package \PGFPlots{}\\
    {\small 2D/3D Plots in \LaTeX{}, Version \pgfplotsversion}\\
    {\small\url{https://github.com/pgf-tikz/pgfplots}}
    %\\{\small Attention: you are using an unstable development version.}
}

\makeatletter
\long\def\abstractsmuggle{%
    \centering
    \textbf{Abstract}\\[0.5cm]

    \begin{minipage}{12cm}
        \PGFPlots{} draws high-quality function plots in normal or logarithmic
        scaling with a user-friendly interface directly in \TeX{}. The user
        supplies axis labels, legend entries and the plot coordinates for one or
        more plots and \PGFPlots{} applies axis scaling, computes any logarithms
        and axis ticks and draws the plots. It supports line plots, scatter
        plots, piecewise constant plots, bar plots, area plots, mesh and surface
        plots, patch plots, contour plots, quiver plots, histogram plots, box
        plots, polar axes, ternary diagrams, smith charts and some more. It is
        based on Till Tantau's package \PGF{}/\Tikz{}.
    \end{minipage}

}%

\expandafter\date\expandafter{\@date\\[2cm]
    \abstractsmuggle
}%
\makeatother

%\includeonly{pgfplots.reference}


\begin{document}

\def\plotcoords{%
\addplot coordinates {
(5,8.312e-02)    (17,2.547e-02)   (49,7.407e-03)
(129,2.102e-03)  (321,5.874e-04)  (769,1.623e-04)
(1793,4.442e-05) (4097,1.207e-05) (9217,3.261e-06)
};

\addplot coordinates{
(7,8.472e-02)    (31,3.044e-02)    (111,1.022e-02)
(351,3.303e-03)  (1023,1.039e-03)  (2815,3.196e-04)
(7423,9.658e-05) (18943,2.873e-05) (47103,8.437e-06)
};

\addplot coordinates{
(9,7.881e-02)     (49,3.243e-02)    (209,1.232e-02)
(769,4.454e-03)   (2561,1.551e-03)  (7937,5.236e-04)
(23297,1.723e-04) (65537,5.545e-05) (178177,1.751e-05)
};

\addplot coordinates{
(11,6.887e-02)    (71,3.177e-02)     (351,1.341e-02)
(1471,5.334e-03)  (5503,2.027e-03)   (18943,7.415e-04)
(61183,2.628e-04) (187903,9.063e-05) (553983,3.053e-05)
};

\addplot coordinates{
(13,5.755e-02)     (97,2.925e-02)     (545,1.351e-02)
(2561,5.842e-03)   (10625,2.397e-03)  (40193,9.414e-04)
(141569,3.564e-04) (471041,1.308e-04)
(1496065,4.670e-05)
};
}%


\bookmaketitle
\tableofcontents
\include{pgfplots.title_abstract_intro}
\include{pgfplots.preliminaries}
\include{pgfplots.intro}
\include{pgfplots.reference}
\include{pgfplots.libs}
\include{pgfplots.resources}
\include{pgfplots.importexport}
\include{pgfplots.basic.reference}

\printindex

\bibliographystyle{abbrv} %gerapali} %gerabbrv} %gerunsrt.bst} %gerabbrv}% gerplain}
\nocite{pgfplotstable}
\nocite{programmingnotes}
\bibliography{pgfplots}
\end{document}

% -----------------------------------------------------------------------------
% For Stefan Pinnow as reminder on what to look for when editing the manual
% -----------------------------------------------------------------------------
% There should be no line breaks in the following environments
% - |...|
% - \declareandlabel{...}
% - \verbpdfref{...}
% If "MakeTikzPictures" isn't running through check one of the externalized
% LOG files what is the cause of that.
% -----------------------------------------------------------------------------


%%%%%%%%%%%%%%%%%%%%%%%%%%%%%%%%%%%%%%%%%%%%%%%%%%%%%%%%%%%%%%%%%%%%%%%%%%%%%
%
% Package pgfplots.sty documentation.
%
% Copyright 2007/2008 by Christian Feuersaenger.
%
% This program is free software: you can redistribute it and/or modify
% it under the terms of the GNU General Public License as published by
% the Free Software Foundation, either version 3 of the License, or
% (at your option) any later version.
%
% This program is distributed in the hope that it will be useful,
% but WITHOUT ANY WARRANTY; without even the implied warranty of
% MERCHANTABILITY or FITNESS FOR A PARTICULAR PURPOSE.  See the
% GNU General Public License for more details.
%
% You should have received a copy of the GNU General Public License
% along with this program.  If not, see <http://www.gnu.org/licenses/>.
%
%
%%%%%%%%%%%%%%%%%%%%%%%%%%%%%%%%%%%%%%%%%%%%%%%%%%%%%%%%%%%%%%%%%%%%%%%%%%%%%
% SEE pgfplots-macros.tex as well!
%\pdfminorversion=5 % to allow compression
%\pdfobjcompresslevel=2
\documentclass[a4paper,openany]{book}

\let\bookmaketitle=\maketitle

% -----------------------------------
% this here is from ltxdoc:
\usepackage{doc}[=v2]
\AtBeginDocument{\MakeShortVerb{\|}}
%\setlength{\textwidth}{355pt}
%\addtolength\marginparwidth{30pt}
%\addtolength\oddsidemargin{20pt}
%\addtolength\evensidemargin{20pt}
\def\cmd#1{\cs{\expandafter\cmd@to@cs\string#1}}
\def\cmd@to@cs#1#2{\char\number`#2\relax}
\DeclareRobustCommand\cs[1]{\texttt{\char`\\#1}}
\providecommand\marg[1]{%
  {\ttfamily\char`\{}\meta{#1}{\ttfamily\char`\}}}
\providecommand\oarg[1]{%
  {\ttfamily[}\meta{#1}{\ttfamily]}}
\providecommand\parg[1]{%
  {\ttfamily(}\meta{#1}{\ttfamily)}}
\raggedbottom

% -----------------------------------
\input{pgfplots.preamble.tex}

\makeatletter
% I want a two-column index just as in pgfmanual styles. This here
% was the best way to get one:
\def\index@prologue{\section*{Index}\addcontentsline{toc}{chapter}{Index}
}
\makeatother

%\RequirePackage[german,english,francais]{babel}

\def\matlabcolormaptext{This colormap is similar to one shipped with Matlab$^\text{\textregistered}$ under a similar name.}

\IfFileExists{tikzlibraryspy.code.tex}{%
\usetikzlibrary{spy}
}{%
    \message{ERROR: tikz SPY library NOT available. The manual will only compile partially.^^J}%
}%

\usepackage{xparse}% for colorbrewer manual

\usetikzlibrary{
    decorations.markings,
    decorations.footprints,
    shapes.arrows,
    matrix,
    positioning,
}

\usepgfplotslibrary{
    fillbetween,
    ternary,
    smithchart,
    patchplots,
    polar,
    colormaps,
    colorbrewer,
    colortol,           % docu not ready yet
}
\pgfqkeys{/codeexample}{%
    every codeexample/.append style={
        /pgfplots/every ternary axis/.append style={
            /pgfplots/legend style={fill=graphicbackground},
        }
    },
    tabsize=4,
}

\pgfplotsmanualenableexternalizationofexpensive

%\usetikzlibrary{external}
%\tikzexternalize[prefix=figures/]{pgfplots}

\title{%
    Manual for Package \PGFPlots{}\\
    {\small 2D/3D Plots in \LaTeX{}, Version \pgfplotsversion}\\
    {\small\url{https://github.com/pgf-tikz/pgfplots}}
    %\\{\small Attention: you are using an unstable development version.}
}

\makeatletter
\long\def\abstractsmuggle{%
    \centering
    \textbf{Abstract}\\[0.5cm]

    \begin{minipage}{12cm}
        \PGFPlots{} draws high-quality function plots in normal or logarithmic
        scaling with a user-friendly interface directly in \TeX{}. The user
        supplies axis labels, legend entries and the plot coordinates for one or
        more plots and \PGFPlots{} applies axis scaling, computes any logarithms
        and axis ticks and draws the plots. It supports line plots, scatter
        plots, piecewise constant plots, bar plots, area plots, mesh and surface
        plots, patch plots, contour plots, quiver plots, histogram plots, box
        plots, polar axes, ternary diagrams, smith charts and some more. It is
        based on Till Tantau's package \PGF{}/\Tikz{}.
    \end{minipage}

}%

\expandafter\date\expandafter{\@date\\[2cm]
    \abstractsmuggle
}%
\makeatother

%\includeonly{pgfplots.reference}


\begin{document}

\def\plotcoords{%
\addplot coordinates {
(5,8.312e-02)    (17,2.547e-02)   (49,7.407e-03)
(129,2.102e-03)  (321,5.874e-04)  (769,1.623e-04)
(1793,4.442e-05) (4097,1.207e-05) (9217,3.261e-06)
};

\addplot coordinates{
(7,8.472e-02)    (31,3.044e-02)    (111,1.022e-02)
(351,3.303e-03)  (1023,1.039e-03)  (2815,3.196e-04)
(7423,9.658e-05) (18943,2.873e-05) (47103,8.437e-06)
};

\addplot coordinates{
(9,7.881e-02)     (49,3.243e-02)    (209,1.232e-02)
(769,4.454e-03)   (2561,1.551e-03)  (7937,5.236e-04)
(23297,1.723e-04) (65537,5.545e-05) (178177,1.751e-05)
};

\addplot coordinates{
(11,6.887e-02)    (71,3.177e-02)     (351,1.341e-02)
(1471,5.334e-03)  (5503,2.027e-03)   (18943,7.415e-04)
(61183,2.628e-04) (187903,9.063e-05) (553983,3.053e-05)
};

\addplot coordinates{
(13,5.755e-02)     (97,2.925e-02)     (545,1.351e-02)
(2561,5.842e-03)   (10625,2.397e-03)  (40193,9.414e-04)
(141569,3.564e-04) (471041,1.308e-04)
(1496065,4.670e-05)
};
}%


\bookmaketitle
\tableofcontents
\include{pgfplots.title_abstract_intro}
\include{pgfplots.preliminaries}
\include{pgfplots.intro}
\include{pgfplots.reference}
\include{pgfplots.libs}
\include{pgfplots.resources}
\include{pgfplots.importexport}
\include{pgfplots.basic.reference}

\printindex

\bibliographystyle{abbrv} %gerapali} %gerabbrv} %gerunsrt.bst} %gerabbrv}% gerplain}
\nocite{pgfplotstable}
\nocite{programmingnotes}
\bibliography{pgfplots}
\end{document}

% -----------------------------------------------------------------------------
% For Stefan Pinnow as reminder on what to look for when editing the manual
% -----------------------------------------------------------------------------
% There should be no line breaks in the following environments
% - |...|
% - \declareandlabel{...}
% - \verbpdfref{...}
% If "MakeTikzPictures" isn't running through check one of the externalized
% LOG files what is the cause of that.
% -----------------------------------------------------------------------------


%%%%%%%%%%%%%%%%%%%%%%%%%%%%%%%%%%%%%%%%%%%%%%%%%%%%%%%%%%%%%%%%%%%%%%%%%%%%%
%
% Package pgfplots.sty documentation.
%
% Copyright 2007/2008 by Christian Feuersaenger.
%
% This program is free software: you can redistribute it and/or modify
% it under the terms of the GNU General Public License as published by
% the Free Software Foundation, either version 3 of the License, or
% (at your option) any later version.
%
% This program is distributed in the hope that it will be useful,
% but WITHOUT ANY WARRANTY; without even the implied warranty of
% MERCHANTABILITY or FITNESS FOR A PARTICULAR PURPOSE.  See the
% GNU General Public License for more details.
%
% You should have received a copy of the GNU General Public License
% along with this program.  If not, see <http://www.gnu.org/licenses/>.
%
%
%%%%%%%%%%%%%%%%%%%%%%%%%%%%%%%%%%%%%%%%%%%%%%%%%%%%%%%%%%%%%%%%%%%%%%%%%%%%%
% SEE pgfplots-macros.tex as well!
%\pdfminorversion=5 % to allow compression
%\pdfobjcompresslevel=2
\documentclass[a4paper,openany]{book}

\let\bookmaketitle=\maketitle

% -----------------------------------
% this here is from ltxdoc:
\usepackage{doc}[=v2]
\AtBeginDocument{\MakeShortVerb{\|}}
%\setlength{\textwidth}{355pt}
%\addtolength\marginparwidth{30pt}
%\addtolength\oddsidemargin{20pt}
%\addtolength\evensidemargin{20pt}
\def\cmd#1{\cs{\expandafter\cmd@to@cs\string#1}}
\def\cmd@to@cs#1#2{\char\number`#2\relax}
\DeclareRobustCommand\cs[1]{\texttt{\char`\\#1}}
\providecommand\marg[1]{%
  {\ttfamily\char`\{}\meta{#1}{\ttfamily\char`\}}}
\providecommand\oarg[1]{%
  {\ttfamily[}\meta{#1}{\ttfamily]}}
\providecommand\parg[1]{%
  {\ttfamily(}\meta{#1}{\ttfamily)}}
\raggedbottom

% -----------------------------------
\input{pgfplots.preamble.tex}

\makeatletter
% I want a two-column index just as in pgfmanual styles. This here
% was the best way to get one:
\def\index@prologue{\section*{Index}\addcontentsline{toc}{chapter}{Index}
}
\makeatother

%\RequirePackage[german,english,francais]{babel}

\def\matlabcolormaptext{This colormap is similar to one shipped with Matlab$^\text{\textregistered}$ under a similar name.}

\IfFileExists{tikzlibraryspy.code.tex}{%
\usetikzlibrary{spy}
}{%
    \message{ERROR: tikz SPY library NOT available. The manual will only compile partially.^^J}%
}%

\usepackage{xparse}% for colorbrewer manual

\usetikzlibrary{
    decorations.markings,
    decorations.footprints,
    shapes.arrows,
    matrix,
    positioning,
}

\usepgfplotslibrary{
    fillbetween,
    ternary,
    smithchart,
    patchplots,
    polar,
    colormaps,
    colorbrewer,
    colortol,           % docu not ready yet
}
\pgfqkeys{/codeexample}{%
    every codeexample/.append style={
        /pgfplots/every ternary axis/.append style={
            /pgfplots/legend style={fill=graphicbackground},
        }
    },
    tabsize=4,
}

\pgfplotsmanualenableexternalizationofexpensive

%\usetikzlibrary{external}
%\tikzexternalize[prefix=figures/]{pgfplots}

\title{%
    Manual for Package \PGFPlots{}\\
    {\small 2D/3D Plots in \LaTeX{}, Version \pgfplotsversion}\\
    {\small\url{https://github.com/pgf-tikz/pgfplots}}
    %\\{\small Attention: you are using an unstable development version.}
}

\makeatletter
\long\def\abstractsmuggle{%
    \centering
    \textbf{Abstract}\\[0.5cm]

    \begin{minipage}{12cm}
        \PGFPlots{} draws high-quality function plots in normal or logarithmic
        scaling with a user-friendly interface directly in \TeX{}. The user
        supplies axis labels, legend entries and the plot coordinates for one or
        more plots and \PGFPlots{} applies axis scaling, computes any logarithms
        and axis ticks and draws the plots. It supports line plots, scatter
        plots, piecewise constant plots, bar plots, area plots, mesh and surface
        plots, patch plots, contour plots, quiver plots, histogram plots, box
        plots, polar axes, ternary diagrams, smith charts and some more. It is
        based on Till Tantau's package \PGF{}/\Tikz{}.
    \end{minipage}

}%

\expandafter\date\expandafter{\@date\\[2cm]
    \abstractsmuggle
}%
\makeatother

%\includeonly{pgfplots.reference}


\begin{document}

\def\plotcoords{%
\addplot coordinates {
(5,8.312e-02)    (17,2.547e-02)   (49,7.407e-03)
(129,2.102e-03)  (321,5.874e-04)  (769,1.623e-04)
(1793,4.442e-05) (4097,1.207e-05) (9217,3.261e-06)
};

\addplot coordinates{
(7,8.472e-02)    (31,3.044e-02)    (111,1.022e-02)
(351,3.303e-03)  (1023,1.039e-03)  (2815,3.196e-04)
(7423,9.658e-05) (18943,2.873e-05) (47103,8.437e-06)
};

\addplot coordinates{
(9,7.881e-02)     (49,3.243e-02)    (209,1.232e-02)
(769,4.454e-03)   (2561,1.551e-03)  (7937,5.236e-04)
(23297,1.723e-04) (65537,5.545e-05) (178177,1.751e-05)
};

\addplot coordinates{
(11,6.887e-02)    (71,3.177e-02)     (351,1.341e-02)
(1471,5.334e-03)  (5503,2.027e-03)   (18943,7.415e-04)
(61183,2.628e-04) (187903,9.063e-05) (553983,3.053e-05)
};

\addplot coordinates{
(13,5.755e-02)     (97,2.925e-02)     (545,1.351e-02)
(2561,5.842e-03)   (10625,2.397e-03)  (40193,9.414e-04)
(141569,3.564e-04) (471041,1.308e-04)
(1496065,4.670e-05)
};
}%


\bookmaketitle
\tableofcontents
\include{pgfplots.title_abstract_intro}
\include{pgfplots.preliminaries}
\include{pgfplots.intro}
\include{pgfplots.reference}
\include{pgfplots.libs}
\include{pgfplots.resources}
\include{pgfplots.importexport}
\include{pgfplots.basic.reference}

\printindex

\bibliographystyle{abbrv} %gerapali} %gerabbrv} %gerunsrt.bst} %gerabbrv}% gerplain}
\nocite{pgfplotstable}
\nocite{programmingnotes}
\bibliography{pgfplots}
\end{document}

% -----------------------------------------------------------------------------
% For Stefan Pinnow as reminder on what to look for when editing the manual
% -----------------------------------------------------------------------------
% There should be no line breaks in the following environments
% - |...|
% - \declareandlabel{...}
% - \verbpdfref{...}
% If "MakeTikzPictures" isn't running through check one of the externalized
% LOG files what is the cause of that.
% -----------------------------------------------------------------------------


%%%%%%%%%%%%%%%%%%%%%%%%%%%%%%%%%%%%%%%%%%%%%%%%%%%%%%%%%%%%%%%%%%%%%%%%%%%%%
%
% Package pgfplots.sty documentation.
%
% Copyright 2007/2008 by Christian Feuersaenger.
%
% This program is free software: you can redistribute it and/or modify
% it under the terms of the GNU General Public License as published by
% the Free Software Foundation, either version 3 of the License, or
% (at your option) any later version.
%
% This program is distributed in the hope that it will be useful,
% but WITHOUT ANY WARRANTY; without even the implied warranty of
% MERCHANTABILITY or FITNESS FOR A PARTICULAR PURPOSE.  See the
% GNU General Public License for more details.
%
% You should have received a copy of the GNU General Public License
% along with this program.  If not, see <http://www.gnu.org/licenses/>.
%
%
%%%%%%%%%%%%%%%%%%%%%%%%%%%%%%%%%%%%%%%%%%%%%%%%%%%%%%%%%%%%%%%%%%%%%%%%%%%%%
% SEE pgfplots-macros.tex as well!
%\pdfminorversion=5 % to allow compression
%\pdfobjcompresslevel=2
\documentclass[a4paper,openany]{book}

\let\bookmaketitle=\maketitle

% -----------------------------------
% this here is from ltxdoc:
\usepackage{doc}[=v2]
\AtBeginDocument{\MakeShortVerb{\|}}
%\setlength{\textwidth}{355pt}
%\addtolength\marginparwidth{30pt}
%\addtolength\oddsidemargin{20pt}
%\addtolength\evensidemargin{20pt}
\def\cmd#1{\cs{\expandafter\cmd@to@cs\string#1}}
\def\cmd@to@cs#1#2{\char\number`#2\relax}
\DeclareRobustCommand\cs[1]{\texttt{\char`\\#1}}
\providecommand\marg[1]{%
  {\ttfamily\char`\{}\meta{#1}{\ttfamily\char`\}}}
\providecommand\oarg[1]{%
  {\ttfamily[}\meta{#1}{\ttfamily]}}
\providecommand\parg[1]{%
  {\ttfamily(}\meta{#1}{\ttfamily)}}
\raggedbottom

% -----------------------------------
\input{pgfplots.preamble.tex}

\makeatletter
% I want a two-column index just as in pgfmanual styles. This here
% was the best way to get one:
\def\index@prologue{\section*{Index}\addcontentsline{toc}{chapter}{Index}
}
\makeatother

%\RequirePackage[german,english,francais]{babel}

\def\matlabcolormaptext{This colormap is similar to one shipped with Matlab$^\text{\textregistered}$ under a similar name.}

\IfFileExists{tikzlibraryspy.code.tex}{%
\usetikzlibrary{spy}
}{%
    \message{ERROR: tikz SPY library NOT available. The manual will only compile partially.^^J}%
}%

\usepackage{xparse}% for colorbrewer manual

\usetikzlibrary{
    decorations.markings,
    decorations.footprints,
    shapes.arrows,
    matrix,
    positioning,
}

\usepgfplotslibrary{
    fillbetween,
    ternary,
    smithchart,
    patchplots,
    polar,
    colormaps,
    colorbrewer,
    colortol,           % docu not ready yet
}
\pgfqkeys{/codeexample}{%
    every codeexample/.append style={
        /pgfplots/every ternary axis/.append style={
            /pgfplots/legend style={fill=graphicbackground},
        }
    },
    tabsize=4,
}

\pgfplotsmanualenableexternalizationofexpensive

%\usetikzlibrary{external}
%\tikzexternalize[prefix=figures/]{pgfplots}

\title{%
    Manual for Package \PGFPlots{}\\
    {\small 2D/3D Plots in \LaTeX{}, Version \pgfplotsversion}\\
    {\small\url{https://github.com/pgf-tikz/pgfplots}}
    %\\{\small Attention: you are using an unstable development version.}
}

\makeatletter
\long\def\abstractsmuggle{%
    \centering
    \textbf{Abstract}\\[0.5cm]

    \begin{minipage}{12cm}
        \PGFPlots{} draws high-quality function plots in normal or logarithmic
        scaling with a user-friendly interface directly in \TeX{}. The user
        supplies axis labels, legend entries and the plot coordinates for one or
        more plots and \PGFPlots{} applies axis scaling, computes any logarithms
        and axis ticks and draws the plots. It supports line plots, scatter
        plots, piecewise constant plots, bar plots, area plots, mesh and surface
        plots, patch plots, contour plots, quiver plots, histogram plots, box
        plots, polar axes, ternary diagrams, smith charts and some more. It is
        based on Till Tantau's package \PGF{}/\Tikz{}.
    \end{minipage}

}%

\expandafter\date\expandafter{\@date\\[2cm]
    \abstractsmuggle
}%
\makeatother

%\includeonly{pgfplots.reference}


\begin{document}

\def\plotcoords{%
\addplot coordinates {
(5,8.312e-02)    (17,2.547e-02)   (49,7.407e-03)
(129,2.102e-03)  (321,5.874e-04)  (769,1.623e-04)
(1793,4.442e-05) (4097,1.207e-05) (9217,3.261e-06)
};

\addplot coordinates{
(7,8.472e-02)    (31,3.044e-02)    (111,1.022e-02)
(351,3.303e-03)  (1023,1.039e-03)  (2815,3.196e-04)
(7423,9.658e-05) (18943,2.873e-05) (47103,8.437e-06)
};

\addplot coordinates{
(9,7.881e-02)     (49,3.243e-02)    (209,1.232e-02)
(769,4.454e-03)   (2561,1.551e-03)  (7937,5.236e-04)
(23297,1.723e-04) (65537,5.545e-05) (178177,1.751e-05)
};

\addplot coordinates{
(11,6.887e-02)    (71,3.177e-02)     (351,1.341e-02)
(1471,5.334e-03)  (5503,2.027e-03)   (18943,7.415e-04)
(61183,2.628e-04) (187903,9.063e-05) (553983,3.053e-05)
};

\addplot coordinates{
(13,5.755e-02)     (97,2.925e-02)     (545,1.351e-02)
(2561,5.842e-03)   (10625,2.397e-03)  (40193,9.414e-04)
(141569,3.564e-04) (471041,1.308e-04)
(1496065,4.670e-05)
};
}%


\bookmaketitle
\tableofcontents
\include{pgfplots.title_abstract_intro}
\include{pgfplots.preliminaries}
\include{pgfplots.intro}
\include{pgfplots.reference}
\include{pgfplots.libs}
\include{pgfplots.resources}
\include{pgfplots.importexport}
\include{pgfplots.basic.reference}

\printindex

\bibliographystyle{abbrv} %gerapali} %gerabbrv} %gerunsrt.bst} %gerabbrv}% gerplain}
\nocite{pgfplotstable}
\nocite{programmingnotes}
\bibliography{pgfplots}
\end{document}

% -----------------------------------------------------------------------------
% For Stefan Pinnow as reminder on what to look for when editing the manual
% -----------------------------------------------------------------------------
% There should be no line breaks in the following environments
% - |...|
% - \declareandlabel{...}
% - \verbpdfref{...}
% If "MakeTikzPictures" isn't running through check one of the externalized
% LOG files what is the cause of that.
% -----------------------------------------------------------------------------


%%%%%%%%%%%%%%%%%%%%%%%%%%%%%%%%%%%%%%%%%%%%%%%%%%%%%%%%%%%%%%%%%%%%%%%%%%%%%
%
% Package pgfplots.sty documentation.
%
% Copyright 2007/2008 by Christian Feuersaenger.
%
% This program is free software: you can redistribute it and/or modify
% it under the terms of the GNU General Public License as published by
% the Free Software Foundation, either version 3 of the License, or
% (at your option) any later version.
%
% This program is distributed in the hope that it will be useful,
% but WITHOUT ANY WARRANTY; without even the implied warranty of
% MERCHANTABILITY or FITNESS FOR A PARTICULAR PURPOSE.  See the
% GNU General Public License for more details.
%
% You should have received a copy of the GNU General Public License
% along with this program.  If not, see <http://www.gnu.org/licenses/>.
%
%
%%%%%%%%%%%%%%%%%%%%%%%%%%%%%%%%%%%%%%%%%%%%%%%%%%%%%%%%%%%%%%%%%%%%%%%%%%%%%
% SEE pgfplots-macros.tex as well!
%\pdfminorversion=5 % to allow compression
%\pdfobjcompresslevel=2
\documentclass[a4paper,openany]{book}

\let\bookmaketitle=\maketitle

% -----------------------------------
% this here is from ltxdoc:
\usepackage{doc}[=v2]
\AtBeginDocument{\MakeShortVerb{\|}}
%\setlength{\textwidth}{355pt}
%\addtolength\marginparwidth{30pt}
%\addtolength\oddsidemargin{20pt}
%\addtolength\evensidemargin{20pt}
\def\cmd#1{\cs{\expandafter\cmd@to@cs\string#1}}
\def\cmd@to@cs#1#2{\char\number`#2\relax}
\DeclareRobustCommand\cs[1]{\texttt{\char`\\#1}}
\providecommand\marg[1]{%
  {\ttfamily\char`\{}\meta{#1}{\ttfamily\char`\}}}
\providecommand\oarg[1]{%
  {\ttfamily[}\meta{#1}{\ttfamily]}}
\providecommand\parg[1]{%
  {\ttfamily(}\meta{#1}{\ttfamily)}}
\raggedbottom

% -----------------------------------
\input{pgfplots.preamble.tex}

\makeatletter
% I want a two-column index just as in pgfmanual styles. This here
% was the best way to get one:
\def\index@prologue{\section*{Index}\addcontentsline{toc}{chapter}{Index}
}
\makeatother

%\RequirePackage[german,english,francais]{babel}

\def\matlabcolormaptext{This colormap is similar to one shipped with Matlab$^\text{\textregistered}$ under a similar name.}

\IfFileExists{tikzlibraryspy.code.tex}{%
\usetikzlibrary{spy}
}{%
    \message{ERROR: tikz SPY library NOT available. The manual will only compile partially.^^J}%
}%

\usepackage{xparse}% for colorbrewer manual

\usetikzlibrary{
    decorations.markings,
    decorations.footprints,
    shapes.arrows,
    matrix,
    positioning,
}

\usepgfplotslibrary{
    fillbetween,
    ternary,
    smithchart,
    patchplots,
    polar,
    colormaps,
    colorbrewer,
    colortol,           % docu not ready yet
}
\pgfqkeys{/codeexample}{%
    every codeexample/.append style={
        /pgfplots/every ternary axis/.append style={
            /pgfplots/legend style={fill=graphicbackground},
        }
    },
    tabsize=4,
}

\pgfplotsmanualenableexternalizationofexpensive

%\usetikzlibrary{external}
%\tikzexternalize[prefix=figures/]{pgfplots}

\title{%
    Manual for Package \PGFPlots{}\\
    {\small 2D/3D Plots in \LaTeX{}, Version \pgfplotsversion}\\
    {\small\url{https://github.com/pgf-tikz/pgfplots}}
    %\\{\small Attention: you are using an unstable development version.}
}

\makeatletter
\long\def\abstractsmuggle{%
    \centering
    \textbf{Abstract}\\[0.5cm]

    \begin{minipage}{12cm}
        \PGFPlots{} draws high-quality function plots in normal or logarithmic
        scaling with a user-friendly interface directly in \TeX{}. The user
        supplies axis labels, legend entries and the plot coordinates for one or
        more plots and \PGFPlots{} applies axis scaling, computes any logarithms
        and axis ticks and draws the plots. It supports line plots, scatter
        plots, piecewise constant plots, bar plots, area plots, mesh and surface
        plots, patch plots, contour plots, quiver plots, histogram plots, box
        plots, polar axes, ternary diagrams, smith charts and some more. It is
        based on Till Tantau's package \PGF{}/\Tikz{}.
    \end{minipage}

}%

\expandafter\date\expandafter{\@date\\[2cm]
    \abstractsmuggle
}%
\makeatother

%\includeonly{pgfplots.reference}


\begin{document}

\def\plotcoords{%
\addplot coordinates {
(5,8.312e-02)    (17,2.547e-02)   (49,7.407e-03)
(129,2.102e-03)  (321,5.874e-04)  (769,1.623e-04)
(1793,4.442e-05) (4097,1.207e-05) (9217,3.261e-06)
};

\addplot coordinates{
(7,8.472e-02)    (31,3.044e-02)    (111,1.022e-02)
(351,3.303e-03)  (1023,1.039e-03)  (2815,3.196e-04)
(7423,9.658e-05) (18943,2.873e-05) (47103,8.437e-06)
};

\addplot coordinates{
(9,7.881e-02)     (49,3.243e-02)    (209,1.232e-02)
(769,4.454e-03)   (2561,1.551e-03)  (7937,5.236e-04)
(23297,1.723e-04) (65537,5.545e-05) (178177,1.751e-05)
};

\addplot coordinates{
(11,6.887e-02)    (71,3.177e-02)     (351,1.341e-02)
(1471,5.334e-03)  (5503,2.027e-03)   (18943,7.415e-04)
(61183,2.628e-04) (187903,9.063e-05) (553983,3.053e-05)
};

\addplot coordinates{
(13,5.755e-02)     (97,2.925e-02)     (545,1.351e-02)
(2561,5.842e-03)   (10625,2.397e-03)  (40193,9.414e-04)
(141569,3.564e-04) (471041,1.308e-04)
(1496065,4.670e-05)
};
}%


\bookmaketitle
\tableofcontents
\include{pgfplots.title_abstract_intro}
\include{pgfplots.preliminaries}
\include{pgfplots.intro}
\include{pgfplots.reference}
\include{pgfplots.libs}
\include{pgfplots.resources}
\include{pgfplots.importexport}
\include{pgfplots.basic.reference}

\printindex

\bibliographystyle{abbrv} %gerapali} %gerabbrv} %gerunsrt.bst} %gerabbrv}% gerplain}
\nocite{pgfplotstable}
\nocite{programmingnotes}
\bibliography{pgfplots}
\end{document}

% -----------------------------------------------------------------------------
% For Stefan Pinnow as reminder on what to look for when editing the manual
% -----------------------------------------------------------------------------
% There should be no line breaks in the following environments
% - |...|
% - \declareandlabel{...}
% - \verbpdfref{...}
% If "MakeTikzPictures" isn't running through check one of the externalized
% LOG files what is the cause of that.
% -----------------------------------------------------------------------------


%%%%%%%%%%%%%%%%%%%%%%%%%%%%%%%%%%%%%%%%%%%%%%%%%%%%%%%%%%%%%%%%%%%%%%%%%%%%%
%
% Package pgfplots.sty documentation.
%
% Copyright 2007/2008 by Christian Feuersaenger.
%
% This program is free software: you can redistribute it and/or modify
% it under the terms of the GNU General Public License as published by
% the Free Software Foundation, either version 3 of the License, or
% (at your option) any later version.
%
% This program is distributed in the hope that it will be useful,
% but WITHOUT ANY WARRANTY; without even the implied warranty of
% MERCHANTABILITY or FITNESS FOR A PARTICULAR PURPOSE.  See the
% GNU General Public License for more details.
%
% You should have received a copy of the GNU General Public License
% along with this program.  If not, see <http://www.gnu.org/licenses/>.
%
%
%%%%%%%%%%%%%%%%%%%%%%%%%%%%%%%%%%%%%%%%%%%%%%%%%%%%%%%%%%%%%%%%%%%%%%%%%%%%%
% SEE pgfplots-macros.tex as well!
%\pdfminorversion=5 % to allow compression
%\pdfobjcompresslevel=2
\documentclass[a4paper,openany]{book}

\let\bookmaketitle=\maketitle

% -----------------------------------
% this here is from ltxdoc:
\usepackage{doc}[=v2]
\AtBeginDocument{\MakeShortVerb{\|}}
%\setlength{\textwidth}{355pt}
%\addtolength\marginparwidth{30pt}
%\addtolength\oddsidemargin{20pt}
%\addtolength\evensidemargin{20pt}
\def\cmd#1{\cs{\expandafter\cmd@to@cs\string#1}}
\def\cmd@to@cs#1#2{\char\number`#2\relax}
\DeclareRobustCommand\cs[1]{\texttt{\char`\\#1}}
\providecommand\marg[1]{%
  {\ttfamily\char`\{}\meta{#1}{\ttfamily\char`\}}}
\providecommand\oarg[1]{%
  {\ttfamily[}\meta{#1}{\ttfamily]}}
\providecommand\parg[1]{%
  {\ttfamily(}\meta{#1}{\ttfamily)}}
\raggedbottom

% -----------------------------------
\input{pgfplots.preamble.tex}

\makeatletter
% I want a two-column index just as in pgfmanual styles. This here
% was the best way to get one:
\def\index@prologue{\section*{Index}\addcontentsline{toc}{chapter}{Index}
}
\makeatother

%\RequirePackage[german,english,francais]{babel}

\def\matlabcolormaptext{This colormap is similar to one shipped with Matlab$^\text{\textregistered}$ under a similar name.}

\IfFileExists{tikzlibraryspy.code.tex}{%
\usetikzlibrary{spy}
}{%
    \message{ERROR: tikz SPY library NOT available. The manual will only compile partially.^^J}%
}%

\usepackage{xparse}% for colorbrewer manual

\usetikzlibrary{
    decorations.markings,
    decorations.footprints,
    shapes.arrows,
    matrix,
    positioning,
}

\usepgfplotslibrary{
    fillbetween,
    ternary,
    smithchart,
    patchplots,
    polar,
    colormaps,
    colorbrewer,
    colortol,           % docu not ready yet
}
\pgfqkeys{/codeexample}{%
    every codeexample/.append style={
        /pgfplots/every ternary axis/.append style={
            /pgfplots/legend style={fill=graphicbackground},
        }
    },
    tabsize=4,
}

\pgfplotsmanualenableexternalizationofexpensive

%\usetikzlibrary{external}
%\tikzexternalize[prefix=figures/]{pgfplots}

\title{%
    Manual for Package \PGFPlots{}\\
    {\small 2D/3D Plots in \LaTeX{}, Version \pgfplotsversion}\\
    {\small\url{https://github.com/pgf-tikz/pgfplots}}
    %\\{\small Attention: you are using an unstable development version.}
}

\makeatletter
\long\def\abstractsmuggle{%
    \centering
    \textbf{Abstract}\\[0.5cm]

    \begin{minipage}{12cm}
        \PGFPlots{} draws high-quality function plots in normal or logarithmic
        scaling with a user-friendly interface directly in \TeX{}. The user
        supplies axis labels, legend entries and the plot coordinates for one or
        more plots and \PGFPlots{} applies axis scaling, computes any logarithms
        and axis ticks and draws the plots. It supports line plots, scatter
        plots, piecewise constant plots, bar plots, area plots, mesh and surface
        plots, patch plots, contour plots, quiver plots, histogram plots, box
        plots, polar axes, ternary diagrams, smith charts and some more. It is
        based on Till Tantau's package \PGF{}/\Tikz{}.
    \end{minipage}

}%

\expandafter\date\expandafter{\@date\\[2cm]
    \abstractsmuggle
}%
\makeatother

%\includeonly{pgfplots.reference}


\begin{document}

\def\plotcoords{%
\addplot coordinates {
(5,8.312e-02)    (17,2.547e-02)   (49,7.407e-03)
(129,2.102e-03)  (321,5.874e-04)  (769,1.623e-04)
(1793,4.442e-05) (4097,1.207e-05) (9217,3.261e-06)
};

\addplot coordinates{
(7,8.472e-02)    (31,3.044e-02)    (111,1.022e-02)
(351,3.303e-03)  (1023,1.039e-03)  (2815,3.196e-04)
(7423,9.658e-05) (18943,2.873e-05) (47103,8.437e-06)
};

\addplot coordinates{
(9,7.881e-02)     (49,3.243e-02)    (209,1.232e-02)
(769,4.454e-03)   (2561,1.551e-03)  (7937,5.236e-04)
(23297,1.723e-04) (65537,5.545e-05) (178177,1.751e-05)
};

\addplot coordinates{
(11,6.887e-02)    (71,3.177e-02)     (351,1.341e-02)
(1471,5.334e-03)  (5503,2.027e-03)   (18943,7.415e-04)
(61183,2.628e-04) (187903,9.063e-05) (553983,3.053e-05)
};

\addplot coordinates{
(13,5.755e-02)     (97,2.925e-02)     (545,1.351e-02)
(2561,5.842e-03)   (10625,2.397e-03)  (40193,9.414e-04)
(141569,3.564e-04) (471041,1.308e-04)
(1496065,4.670e-05)
};
}%


\bookmaketitle
\tableofcontents
\include{pgfplots.title_abstract_intro}
\include{pgfplots.preliminaries}
\include{pgfplots.intro}
\include{pgfplots.reference}
\include{pgfplots.libs}
\include{pgfplots.resources}
\include{pgfplots.importexport}
\include{pgfplots.basic.reference}

\printindex

\bibliographystyle{abbrv} %gerapali} %gerabbrv} %gerunsrt.bst} %gerabbrv}% gerplain}
\nocite{pgfplotstable}
\nocite{programmingnotes}
\bibliography{pgfplots}
\end{document}

% -----------------------------------------------------------------------------
% For Stefan Pinnow as reminder on what to look for when editing the manual
% -----------------------------------------------------------------------------
% There should be no line breaks in the following environments
% - |...|
% - \declareandlabel{...}
% - \verbpdfref{...}
% If "MakeTikzPictures" isn't running through check one of the externalized
% LOG files what is the cause of that.
% -----------------------------------------------------------------------------

\include{pgfplots.basic.reference}

\printindex

\bibliographystyle{abbrv} %gerapali} %gerabbrv} %gerunsrt.bst} %gerabbrv}% gerplain}
\nocite{pgfplotstable}
\nocite{programmingnotes}
\bibliography{pgfplots}
\end{document}

% -----------------------------------------------------------------------------
% For Stefan Pinnow as reminder on what to look for when editing the manual
% -----------------------------------------------------------------------------
% There should be no line breaks in the following environments
% - |...|
% - \declareandlabel{...}
% - \verbpdfref{...}
% If "MakeTikzPictures" isn't running through check one of the externalized
% LOG files what is the cause of that.
% -----------------------------------------------------------------------------

\include{pgfplots.basic.reference}

\printindex

\bibliographystyle{abbrv} %gerapali} %gerabbrv} %gerunsrt.bst} %gerabbrv}% gerplain}
\nocite{pgfplotstable}
\nocite{programmingnotes}
\bibliography{pgfplots}
\end{document}

% -----------------------------------------------------------------------------
% For Stefan Pinnow as reminder on what to look for when editing the manual
% -----------------------------------------------------------------------------
% There should be no line breaks in the following environments
% - |...|
% - \declareandlabel{...}
% - \verbpdfref{...}
% If "MakeTikzPictures" isn't running through check one of the externalized
% LOG files what is the cause of that.
% -----------------------------------------------------------------------------


%%%%%%%%%%%%%%%%%%%%%%%%%%%%%%%%%%%%%%%%%%%%%%%%%%%%%%%%%%%%%%%%%%%%%%%%%%%%%
%
% Package pgfplots.sty documentation.
%
% Copyright 2007/2008 by Christian Feuersaenger.
%
% This program is free software: you can redistribute it and/or modify
% it under the terms of the GNU General Public License as published by
% the Free Software Foundation, either version 3 of the License, or
% (at your option) any later version.
%
% This program is distributed in the hope that it will be useful,
% but WITHOUT ANY WARRANTY; without even the implied warranty of
% MERCHANTABILITY or FITNESS FOR A PARTICULAR PURPOSE.  See the
% GNU General Public License for more details.
%
% You should have received a copy of the GNU General Public License
% along with this program.  If not, see <http://www.gnu.org/licenses/>.
%
%
%%%%%%%%%%%%%%%%%%%%%%%%%%%%%%%%%%%%%%%%%%%%%%%%%%%%%%%%%%%%%%%%%%%%%%%%%%%%%
% SEE pgfplots-macros.tex as well!
%\pdfminorversion=5 % to allow compression
%\pdfobjcompresslevel=2
\documentclass[a4paper,openany]{book}

\let\bookmaketitle=\maketitle

% -----------------------------------
% this here is from ltxdoc:
\usepackage{doc}[=v2]
\AtBeginDocument{\MakeShortVerb{\|}}
%\setlength{\textwidth}{355pt}
%\addtolength\marginparwidth{30pt}
%\addtolength\oddsidemargin{20pt}
%\addtolength\evensidemargin{20pt}
\def\cmd#1{\cs{\expandafter\cmd@to@cs\string#1}}
\def\cmd@to@cs#1#2{\char\number`#2\relax}
\DeclareRobustCommand\cs[1]{\texttt{\char`\\#1}}
\providecommand\marg[1]{%
  {\ttfamily\char`\{}\meta{#1}{\ttfamily\char`\}}}
\providecommand\oarg[1]{%
  {\ttfamily[}\meta{#1}{\ttfamily]}}
\providecommand\parg[1]{%
  {\ttfamily(}\meta{#1}{\ttfamily)}}
\raggedbottom

% -----------------------------------
\input{pgfplots.preamble.tex}

\makeatletter
% I want a two-column index just as in pgfmanual styles. This here
% was the best way to get one:
\def\index@prologue{\section*{Index}\addcontentsline{toc}{chapter}{Index}
}
\makeatother

%\RequirePackage[german,english,francais]{babel}

\def\matlabcolormaptext{This colormap is similar to one shipped with Matlab$^\text{\textregistered}$ under a similar name.}

\IfFileExists{tikzlibraryspy.code.tex}{%
\usetikzlibrary{spy}
}{%
    \message{ERROR: tikz SPY library NOT available. The manual will only compile partially.^^J}%
}%

\usepackage{xparse}% for colorbrewer manual

\usetikzlibrary{
    decorations.markings,
    decorations.footprints,
    shapes.arrows,
    matrix,
    positioning,
}

\usepgfplotslibrary{
    fillbetween,
    ternary,
    smithchart,
    patchplots,
    polar,
    colormaps,
    colorbrewer,
    colortol,           % docu not ready yet
}
\pgfqkeys{/codeexample}{%
    every codeexample/.append style={
        /pgfplots/every ternary axis/.append style={
            /pgfplots/legend style={fill=graphicbackground},
        }
    },
    tabsize=4,
}

\pgfplotsmanualenableexternalizationofexpensive

%\usetikzlibrary{external}
%\tikzexternalize[prefix=figures/]{pgfplots}

\title{%
    Manual for Package \PGFPlots{}\\
    {\small 2D/3D Plots in \LaTeX{}, Version \pgfplotsversion}\\
    {\small\url{https://github.com/pgf-tikz/pgfplots}}
    %\\{\small Attention: you are using an unstable development version.}
}

\makeatletter
\long\def\abstractsmuggle{%
    \centering
    \textbf{Abstract}\\[0.5cm]

    \begin{minipage}{12cm}
        \PGFPlots{} draws high-quality function plots in normal or logarithmic
        scaling with a user-friendly interface directly in \TeX{}. The user
        supplies axis labels, legend entries and the plot coordinates for one or
        more plots and \PGFPlots{} applies axis scaling, computes any logarithms
        and axis ticks and draws the plots. It supports line plots, scatter
        plots, piecewise constant plots, bar plots, area plots, mesh and surface
        plots, patch plots, contour plots, quiver plots, histogram plots, box
        plots, polar axes, ternary diagrams, smith charts and some more. It is
        based on Till Tantau's package \PGF{}/\Tikz{}.
    \end{minipage}

}%

\expandafter\date\expandafter{\@date\\[2cm]
    \abstractsmuggle
}%
\makeatother

%\includeonly{pgfplots.reference}


\begin{document}

\def\plotcoords{%
\addplot coordinates {
(5,8.312e-02)    (17,2.547e-02)   (49,7.407e-03)
(129,2.102e-03)  (321,5.874e-04)  (769,1.623e-04)
(1793,4.442e-05) (4097,1.207e-05) (9217,3.261e-06)
};

\addplot coordinates{
(7,8.472e-02)    (31,3.044e-02)    (111,1.022e-02)
(351,3.303e-03)  (1023,1.039e-03)  (2815,3.196e-04)
(7423,9.658e-05) (18943,2.873e-05) (47103,8.437e-06)
};

\addplot coordinates{
(9,7.881e-02)     (49,3.243e-02)    (209,1.232e-02)
(769,4.454e-03)   (2561,1.551e-03)  (7937,5.236e-04)
(23297,1.723e-04) (65537,5.545e-05) (178177,1.751e-05)
};

\addplot coordinates{
(11,6.887e-02)    (71,3.177e-02)     (351,1.341e-02)
(1471,5.334e-03)  (5503,2.027e-03)   (18943,7.415e-04)
(61183,2.628e-04) (187903,9.063e-05) (553983,3.053e-05)
};

\addplot coordinates{
(13,5.755e-02)     (97,2.925e-02)     (545,1.351e-02)
(2561,5.842e-03)   (10625,2.397e-03)  (40193,9.414e-04)
(141569,3.564e-04) (471041,1.308e-04)
(1496065,4.670e-05)
};
}%


\bookmaketitle
\tableofcontents

%%%%%%%%%%%%%%%%%%%%%%%%%%%%%%%%%%%%%%%%%%%%%%%%%%%%%%%%%%%%%%%%%%%%%%%%%%%%%
%
% Package pgfplots.sty documentation.
%
% Copyright 2007/2008 by Christian Feuersaenger.
%
% This program is free software: you can redistribute it and/or modify
% it under the terms of the GNU General Public License as published by
% the Free Software Foundation, either version 3 of the License, or
% (at your option) any later version.
%
% This program is distributed in the hope that it will be useful,
% but WITHOUT ANY WARRANTY; without even the implied warranty of
% MERCHANTABILITY or FITNESS FOR A PARTICULAR PURPOSE.  See the
% GNU General Public License for more details.
%
% You should have received a copy of the GNU General Public License
% along with this program.  If not, see <http://www.gnu.org/licenses/>.
%
%
%%%%%%%%%%%%%%%%%%%%%%%%%%%%%%%%%%%%%%%%%%%%%%%%%%%%%%%%%%%%%%%%%%%%%%%%%%%%%
% SEE pgfplots-macros.tex as well!
%\pdfminorversion=5 % to allow compression
%\pdfobjcompresslevel=2
\documentclass[a4paper,openany]{book}

\let\bookmaketitle=\maketitle

% -----------------------------------
% this here is from ltxdoc:
\usepackage{doc}[=v2]
\AtBeginDocument{\MakeShortVerb{\|}}
%\setlength{\textwidth}{355pt}
%\addtolength\marginparwidth{30pt}
%\addtolength\oddsidemargin{20pt}
%\addtolength\evensidemargin{20pt}
\def\cmd#1{\cs{\expandafter\cmd@to@cs\string#1}}
\def\cmd@to@cs#1#2{\char\number`#2\relax}
\DeclareRobustCommand\cs[1]{\texttt{\char`\\#1}}
\providecommand\marg[1]{%
  {\ttfamily\char`\{}\meta{#1}{\ttfamily\char`\}}}
\providecommand\oarg[1]{%
  {\ttfamily[}\meta{#1}{\ttfamily]}}
\providecommand\parg[1]{%
  {\ttfamily(}\meta{#1}{\ttfamily)}}
\raggedbottom

% -----------------------------------
\input{pgfplots.preamble.tex}

\makeatletter
% I want a two-column index just as in pgfmanual styles. This here
% was the best way to get one:
\def\index@prologue{\section*{Index}\addcontentsline{toc}{chapter}{Index}
}
\makeatother

%\RequirePackage[german,english,francais]{babel}

\def\matlabcolormaptext{This colormap is similar to one shipped with Matlab$^\text{\textregistered}$ under a similar name.}

\IfFileExists{tikzlibraryspy.code.tex}{%
\usetikzlibrary{spy}
}{%
    \message{ERROR: tikz SPY library NOT available. The manual will only compile partially.^^J}%
}%

\usepackage{xparse}% for colorbrewer manual

\usetikzlibrary{
    decorations.markings,
    decorations.footprints,
    shapes.arrows,
    matrix,
    positioning,
}

\usepgfplotslibrary{
    fillbetween,
    ternary,
    smithchart,
    patchplots,
    polar,
    colormaps,
    colorbrewer,
    colortol,           % docu not ready yet
}
\pgfqkeys{/codeexample}{%
    every codeexample/.append style={
        /pgfplots/every ternary axis/.append style={
            /pgfplots/legend style={fill=graphicbackground},
        }
    },
    tabsize=4,
}

\pgfplotsmanualenableexternalizationofexpensive

%\usetikzlibrary{external}
%\tikzexternalize[prefix=figures/]{pgfplots}

\title{%
    Manual for Package \PGFPlots{}\\
    {\small 2D/3D Plots in \LaTeX{}, Version \pgfplotsversion}\\
    {\small\url{https://github.com/pgf-tikz/pgfplots}}
    %\\{\small Attention: you are using an unstable development version.}
}

\makeatletter
\long\def\abstractsmuggle{%
    \centering
    \textbf{Abstract}\\[0.5cm]

    \begin{minipage}{12cm}
        \PGFPlots{} draws high-quality function plots in normal or logarithmic
        scaling with a user-friendly interface directly in \TeX{}. The user
        supplies axis labels, legend entries and the plot coordinates for one or
        more plots and \PGFPlots{} applies axis scaling, computes any logarithms
        and axis ticks and draws the plots. It supports line plots, scatter
        plots, piecewise constant plots, bar plots, area plots, mesh and surface
        plots, patch plots, contour plots, quiver plots, histogram plots, box
        plots, polar axes, ternary diagrams, smith charts and some more. It is
        based on Till Tantau's package \PGF{}/\Tikz{}.
    \end{minipage}

}%

\expandafter\date\expandafter{\@date\\[2cm]
    \abstractsmuggle
}%
\makeatother

%\includeonly{pgfplots.reference}


\begin{document}

\def\plotcoords{%
\addplot coordinates {
(5,8.312e-02)    (17,2.547e-02)   (49,7.407e-03)
(129,2.102e-03)  (321,5.874e-04)  (769,1.623e-04)
(1793,4.442e-05) (4097,1.207e-05) (9217,3.261e-06)
};

\addplot coordinates{
(7,8.472e-02)    (31,3.044e-02)    (111,1.022e-02)
(351,3.303e-03)  (1023,1.039e-03)  (2815,3.196e-04)
(7423,9.658e-05) (18943,2.873e-05) (47103,8.437e-06)
};

\addplot coordinates{
(9,7.881e-02)     (49,3.243e-02)    (209,1.232e-02)
(769,4.454e-03)   (2561,1.551e-03)  (7937,5.236e-04)
(23297,1.723e-04) (65537,5.545e-05) (178177,1.751e-05)
};

\addplot coordinates{
(11,6.887e-02)    (71,3.177e-02)     (351,1.341e-02)
(1471,5.334e-03)  (5503,2.027e-03)   (18943,7.415e-04)
(61183,2.628e-04) (187903,9.063e-05) (553983,3.053e-05)
};

\addplot coordinates{
(13,5.755e-02)     (97,2.925e-02)     (545,1.351e-02)
(2561,5.842e-03)   (10625,2.397e-03)  (40193,9.414e-04)
(141569,3.564e-04) (471041,1.308e-04)
(1496065,4.670e-05)
};
}%


\bookmaketitle
\tableofcontents

%%%%%%%%%%%%%%%%%%%%%%%%%%%%%%%%%%%%%%%%%%%%%%%%%%%%%%%%%%%%%%%%%%%%%%%%%%%%%
%
% Package pgfplots.sty documentation.
%
% Copyright 2007/2008 by Christian Feuersaenger.
%
% This program is free software: you can redistribute it and/or modify
% it under the terms of the GNU General Public License as published by
% the Free Software Foundation, either version 3 of the License, or
% (at your option) any later version.
%
% This program is distributed in the hope that it will be useful,
% but WITHOUT ANY WARRANTY; without even the implied warranty of
% MERCHANTABILITY or FITNESS FOR A PARTICULAR PURPOSE.  See the
% GNU General Public License for more details.
%
% You should have received a copy of the GNU General Public License
% along with this program.  If not, see <http://www.gnu.org/licenses/>.
%
%
%%%%%%%%%%%%%%%%%%%%%%%%%%%%%%%%%%%%%%%%%%%%%%%%%%%%%%%%%%%%%%%%%%%%%%%%%%%%%
% SEE pgfplots-macros.tex as well!
%\pdfminorversion=5 % to allow compression
%\pdfobjcompresslevel=2
\documentclass[a4paper,openany]{book}

\let\bookmaketitle=\maketitle

% -----------------------------------
% this here is from ltxdoc:
\usepackage{doc}[=v2]
\AtBeginDocument{\MakeShortVerb{\|}}
%\setlength{\textwidth}{355pt}
%\addtolength\marginparwidth{30pt}
%\addtolength\oddsidemargin{20pt}
%\addtolength\evensidemargin{20pt}
\def\cmd#1{\cs{\expandafter\cmd@to@cs\string#1}}
\def\cmd@to@cs#1#2{\char\number`#2\relax}
\DeclareRobustCommand\cs[1]{\texttt{\char`\\#1}}
\providecommand\marg[1]{%
  {\ttfamily\char`\{}\meta{#1}{\ttfamily\char`\}}}
\providecommand\oarg[1]{%
  {\ttfamily[}\meta{#1}{\ttfamily]}}
\providecommand\parg[1]{%
  {\ttfamily(}\meta{#1}{\ttfamily)}}
\raggedbottom

% -----------------------------------
\input{pgfplots.preamble.tex}

\makeatletter
% I want a two-column index just as in pgfmanual styles. This here
% was the best way to get one:
\def\index@prologue{\section*{Index}\addcontentsline{toc}{chapter}{Index}
}
\makeatother

%\RequirePackage[german,english,francais]{babel}

\def\matlabcolormaptext{This colormap is similar to one shipped with Matlab$^\text{\textregistered}$ under a similar name.}

\IfFileExists{tikzlibraryspy.code.tex}{%
\usetikzlibrary{spy}
}{%
    \message{ERROR: tikz SPY library NOT available. The manual will only compile partially.^^J}%
}%

\usepackage{xparse}% for colorbrewer manual

\usetikzlibrary{
    decorations.markings,
    decorations.footprints,
    shapes.arrows,
    matrix,
    positioning,
}

\usepgfplotslibrary{
    fillbetween,
    ternary,
    smithchart,
    patchplots,
    polar,
    colormaps,
    colorbrewer,
    colortol,           % docu not ready yet
}
\pgfqkeys{/codeexample}{%
    every codeexample/.append style={
        /pgfplots/every ternary axis/.append style={
            /pgfplots/legend style={fill=graphicbackground},
        }
    },
    tabsize=4,
}

\pgfplotsmanualenableexternalizationofexpensive

%\usetikzlibrary{external}
%\tikzexternalize[prefix=figures/]{pgfplots}

\title{%
    Manual for Package \PGFPlots{}\\
    {\small 2D/3D Plots in \LaTeX{}, Version \pgfplotsversion}\\
    {\small\url{https://github.com/pgf-tikz/pgfplots}}
    %\\{\small Attention: you are using an unstable development version.}
}

\makeatletter
\long\def\abstractsmuggle{%
    \centering
    \textbf{Abstract}\\[0.5cm]

    \begin{minipage}{12cm}
        \PGFPlots{} draws high-quality function plots in normal or logarithmic
        scaling with a user-friendly interface directly in \TeX{}. The user
        supplies axis labels, legend entries and the plot coordinates for one or
        more plots and \PGFPlots{} applies axis scaling, computes any logarithms
        and axis ticks and draws the plots. It supports line plots, scatter
        plots, piecewise constant plots, bar plots, area plots, mesh and surface
        plots, patch plots, contour plots, quiver plots, histogram plots, box
        plots, polar axes, ternary diagrams, smith charts and some more. It is
        based on Till Tantau's package \PGF{}/\Tikz{}.
    \end{minipage}

}%

\expandafter\date\expandafter{\@date\\[2cm]
    \abstractsmuggle
}%
\makeatother

%\includeonly{pgfplots.reference}


\begin{document}

\def\plotcoords{%
\addplot coordinates {
(5,8.312e-02)    (17,2.547e-02)   (49,7.407e-03)
(129,2.102e-03)  (321,5.874e-04)  (769,1.623e-04)
(1793,4.442e-05) (4097,1.207e-05) (9217,3.261e-06)
};

\addplot coordinates{
(7,8.472e-02)    (31,3.044e-02)    (111,1.022e-02)
(351,3.303e-03)  (1023,1.039e-03)  (2815,3.196e-04)
(7423,9.658e-05) (18943,2.873e-05) (47103,8.437e-06)
};

\addplot coordinates{
(9,7.881e-02)     (49,3.243e-02)    (209,1.232e-02)
(769,4.454e-03)   (2561,1.551e-03)  (7937,5.236e-04)
(23297,1.723e-04) (65537,5.545e-05) (178177,1.751e-05)
};

\addplot coordinates{
(11,6.887e-02)    (71,3.177e-02)     (351,1.341e-02)
(1471,5.334e-03)  (5503,2.027e-03)   (18943,7.415e-04)
(61183,2.628e-04) (187903,9.063e-05) (553983,3.053e-05)
};

\addplot coordinates{
(13,5.755e-02)     (97,2.925e-02)     (545,1.351e-02)
(2561,5.842e-03)   (10625,2.397e-03)  (40193,9.414e-04)
(141569,3.564e-04) (471041,1.308e-04)
(1496065,4.670e-05)
};
}%


\bookmaketitle
\tableofcontents
\include{pgfplots.title_abstract_intro}
\include{pgfplots.preliminaries}
\include{pgfplots.intro}
\include{pgfplots.reference}
\include{pgfplots.libs}
\include{pgfplots.resources}
\include{pgfplots.importexport}
\include{pgfplots.basic.reference}

\printindex

\bibliographystyle{abbrv} %gerapali} %gerabbrv} %gerunsrt.bst} %gerabbrv}% gerplain}
\nocite{pgfplotstable}
\nocite{programmingnotes}
\bibliography{pgfplots}
\end{document}

% -----------------------------------------------------------------------------
% For Stefan Pinnow as reminder on what to look for when editing the manual
% -----------------------------------------------------------------------------
% There should be no line breaks in the following environments
% - |...|
% - \declareandlabel{...}
% - \verbpdfref{...}
% If "MakeTikzPictures" isn't running through check one of the externalized
% LOG files what is the cause of that.
% -----------------------------------------------------------------------------


%%%%%%%%%%%%%%%%%%%%%%%%%%%%%%%%%%%%%%%%%%%%%%%%%%%%%%%%%%%%%%%%%%%%%%%%%%%%%
%
% Package pgfplots.sty documentation.
%
% Copyright 2007/2008 by Christian Feuersaenger.
%
% This program is free software: you can redistribute it and/or modify
% it under the terms of the GNU General Public License as published by
% the Free Software Foundation, either version 3 of the License, or
% (at your option) any later version.
%
% This program is distributed in the hope that it will be useful,
% but WITHOUT ANY WARRANTY; without even the implied warranty of
% MERCHANTABILITY or FITNESS FOR A PARTICULAR PURPOSE.  See the
% GNU General Public License for more details.
%
% You should have received a copy of the GNU General Public License
% along with this program.  If not, see <http://www.gnu.org/licenses/>.
%
%
%%%%%%%%%%%%%%%%%%%%%%%%%%%%%%%%%%%%%%%%%%%%%%%%%%%%%%%%%%%%%%%%%%%%%%%%%%%%%
% SEE pgfplots-macros.tex as well!
%\pdfminorversion=5 % to allow compression
%\pdfobjcompresslevel=2
\documentclass[a4paper,openany]{book}

\let\bookmaketitle=\maketitle

% -----------------------------------
% this here is from ltxdoc:
\usepackage{doc}[=v2]
\AtBeginDocument{\MakeShortVerb{\|}}
%\setlength{\textwidth}{355pt}
%\addtolength\marginparwidth{30pt}
%\addtolength\oddsidemargin{20pt}
%\addtolength\evensidemargin{20pt}
\def\cmd#1{\cs{\expandafter\cmd@to@cs\string#1}}
\def\cmd@to@cs#1#2{\char\number`#2\relax}
\DeclareRobustCommand\cs[1]{\texttt{\char`\\#1}}
\providecommand\marg[1]{%
  {\ttfamily\char`\{}\meta{#1}{\ttfamily\char`\}}}
\providecommand\oarg[1]{%
  {\ttfamily[}\meta{#1}{\ttfamily]}}
\providecommand\parg[1]{%
  {\ttfamily(}\meta{#1}{\ttfamily)}}
\raggedbottom

% -----------------------------------
\input{pgfplots.preamble.tex}

\makeatletter
% I want a two-column index just as in pgfmanual styles. This here
% was the best way to get one:
\def\index@prologue{\section*{Index}\addcontentsline{toc}{chapter}{Index}
}
\makeatother

%\RequirePackage[german,english,francais]{babel}

\def\matlabcolormaptext{This colormap is similar to one shipped with Matlab$^\text{\textregistered}$ under a similar name.}

\IfFileExists{tikzlibraryspy.code.tex}{%
\usetikzlibrary{spy}
}{%
    \message{ERROR: tikz SPY library NOT available. The manual will only compile partially.^^J}%
}%

\usepackage{xparse}% for colorbrewer manual

\usetikzlibrary{
    decorations.markings,
    decorations.footprints,
    shapes.arrows,
    matrix,
    positioning,
}

\usepgfplotslibrary{
    fillbetween,
    ternary,
    smithchart,
    patchplots,
    polar,
    colormaps,
    colorbrewer,
    colortol,           % docu not ready yet
}
\pgfqkeys{/codeexample}{%
    every codeexample/.append style={
        /pgfplots/every ternary axis/.append style={
            /pgfplots/legend style={fill=graphicbackground},
        }
    },
    tabsize=4,
}

\pgfplotsmanualenableexternalizationofexpensive

%\usetikzlibrary{external}
%\tikzexternalize[prefix=figures/]{pgfplots}

\title{%
    Manual for Package \PGFPlots{}\\
    {\small 2D/3D Plots in \LaTeX{}, Version \pgfplotsversion}\\
    {\small\url{https://github.com/pgf-tikz/pgfplots}}
    %\\{\small Attention: you are using an unstable development version.}
}

\makeatletter
\long\def\abstractsmuggle{%
    \centering
    \textbf{Abstract}\\[0.5cm]

    \begin{minipage}{12cm}
        \PGFPlots{} draws high-quality function plots in normal or logarithmic
        scaling with a user-friendly interface directly in \TeX{}. The user
        supplies axis labels, legend entries and the plot coordinates for one or
        more plots and \PGFPlots{} applies axis scaling, computes any logarithms
        and axis ticks and draws the plots. It supports line plots, scatter
        plots, piecewise constant plots, bar plots, area plots, mesh and surface
        plots, patch plots, contour plots, quiver plots, histogram plots, box
        plots, polar axes, ternary diagrams, smith charts and some more. It is
        based on Till Tantau's package \PGF{}/\Tikz{}.
    \end{minipage}

}%

\expandafter\date\expandafter{\@date\\[2cm]
    \abstractsmuggle
}%
\makeatother

%\includeonly{pgfplots.reference}


\begin{document}

\def\plotcoords{%
\addplot coordinates {
(5,8.312e-02)    (17,2.547e-02)   (49,7.407e-03)
(129,2.102e-03)  (321,5.874e-04)  (769,1.623e-04)
(1793,4.442e-05) (4097,1.207e-05) (9217,3.261e-06)
};

\addplot coordinates{
(7,8.472e-02)    (31,3.044e-02)    (111,1.022e-02)
(351,3.303e-03)  (1023,1.039e-03)  (2815,3.196e-04)
(7423,9.658e-05) (18943,2.873e-05) (47103,8.437e-06)
};

\addplot coordinates{
(9,7.881e-02)     (49,3.243e-02)    (209,1.232e-02)
(769,4.454e-03)   (2561,1.551e-03)  (7937,5.236e-04)
(23297,1.723e-04) (65537,5.545e-05) (178177,1.751e-05)
};

\addplot coordinates{
(11,6.887e-02)    (71,3.177e-02)     (351,1.341e-02)
(1471,5.334e-03)  (5503,2.027e-03)   (18943,7.415e-04)
(61183,2.628e-04) (187903,9.063e-05) (553983,3.053e-05)
};

\addplot coordinates{
(13,5.755e-02)     (97,2.925e-02)     (545,1.351e-02)
(2561,5.842e-03)   (10625,2.397e-03)  (40193,9.414e-04)
(141569,3.564e-04) (471041,1.308e-04)
(1496065,4.670e-05)
};
}%


\bookmaketitle
\tableofcontents
\include{pgfplots.title_abstract_intro}
\include{pgfplots.preliminaries}
\include{pgfplots.intro}
\include{pgfplots.reference}
\include{pgfplots.libs}
\include{pgfplots.resources}
\include{pgfplots.importexport}
\include{pgfplots.basic.reference}

\printindex

\bibliographystyle{abbrv} %gerapali} %gerabbrv} %gerunsrt.bst} %gerabbrv}% gerplain}
\nocite{pgfplotstable}
\nocite{programmingnotes}
\bibliography{pgfplots}
\end{document}

% -----------------------------------------------------------------------------
% For Stefan Pinnow as reminder on what to look for when editing the manual
% -----------------------------------------------------------------------------
% There should be no line breaks in the following environments
% - |...|
% - \declareandlabel{...}
% - \verbpdfref{...}
% If "MakeTikzPictures" isn't running through check one of the externalized
% LOG files what is the cause of that.
% -----------------------------------------------------------------------------


%%%%%%%%%%%%%%%%%%%%%%%%%%%%%%%%%%%%%%%%%%%%%%%%%%%%%%%%%%%%%%%%%%%%%%%%%%%%%
%
% Package pgfplots.sty documentation.
%
% Copyright 2007/2008 by Christian Feuersaenger.
%
% This program is free software: you can redistribute it and/or modify
% it under the terms of the GNU General Public License as published by
% the Free Software Foundation, either version 3 of the License, or
% (at your option) any later version.
%
% This program is distributed in the hope that it will be useful,
% but WITHOUT ANY WARRANTY; without even the implied warranty of
% MERCHANTABILITY or FITNESS FOR A PARTICULAR PURPOSE.  See the
% GNU General Public License for more details.
%
% You should have received a copy of the GNU General Public License
% along with this program.  If not, see <http://www.gnu.org/licenses/>.
%
%
%%%%%%%%%%%%%%%%%%%%%%%%%%%%%%%%%%%%%%%%%%%%%%%%%%%%%%%%%%%%%%%%%%%%%%%%%%%%%
% SEE pgfplots-macros.tex as well!
%\pdfminorversion=5 % to allow compression
%\pdfobjcompresslevel=2
\documentclass[a4paper,openany]{book}

\let\bookmaketitle=\maketitle

% -----------------------------------
% this here is from ltxdoc:
\usepackage{doc}[=v2]
\AtBeginDocument{\MakeShortVerb{\|}}
%\setlength{\textwidth}{355pt}
%\addtolength\marginparwidth{30pt}
%\addtolength\oddsidemargin{20pt}
%\addtolength\evensidemargin{20pt}
\def\cmd#1{\cs{\expandafter\cmd@to@cs\string#1}}
\def\cmd@to@cs#1#2{\char\number`#2\relax}
\DeclareRobustCommand\cs[1]{\texttt{\char`\\#1}}
\providecommand\marg[1]{%
  {\ttfamily\char`\{}\meta{#1}{\ttfamily\char`\}}}
\providecommand\oarg[1]{%
  {\ttfamily[}\meta{#1}{\ttfamily]}}
\providecommand\parg[1]{%
  {\ttfamily(}\meta{#1}{\ttfamily)}}
\raggedbottom

% -----------------------------------
\input{pgfplots.preamble.tex}

\makeatletter
% I want a two-column index just as in pgfmanual styles. This here
% was the best way to get one:
\def\index@prologue{\section*{Index}\addcontentsline{toc}{chapter}{Index}
}
\makeatother

%\RequirePackage[german,english,francais]{babel}

\def\matlabcolormaptext{This colormap is similar to one shipped with Matlab$^\text{\textregistered}$ under a similar name.}

\IfFileExists{tikzlibraryspy.code.tex}{%
\usetikzlibrary{spy}
}{%
    \message{ERROR: tikz SPY library NOT available. The manual will only compile partially.^^J}%
}%

\usepackage{xparse}% for colorbrewer manual

\usetikzlibrary{
    decorations.markings,
    decorations.footprints,
    shapes.arrows,
    matrix,
    positioning,
}

\usepgfplotslibrary{
    fillbetween,
    ternary,
    smithchart,
    patchplots,
    polar,
    colormaps,
    colorbrewer,
    colortol,           % docu not ready yet
}
\pgfqkeys{/codeexample}{%
    every codeexample/.append style={
        /pgfplots/every ternary axis/.append style={
            /pgfplots/legend style={fill=graphicbackground},
        }
    },
    tabsize=4,
}

\pgfplotsmanualenableexternalizationofexpensive

%\usetikzlibrary{external}
%\tikzexternalize[prefix=figures/]{pgfplots}

\title{%
    Manual for Package \PGFPlots{}\\
    {\small 2D/3D Plots in \LaTeX{}, Version \pgfplotsversion}\\
    {\small\url{https://github.com/pgf-tikz/pgfplots}}
    %\\{\small Attention: you are using an unstable development version.}
}

\makeatletter
\long\def\abstractsmuggle{%
    \centering
    \textbf{Abstract}\\[0.5cm]

    \begin{minipage}{12cm}
        \PGFPlots{} draws high-quality function plots in normal or logarithmic
        scaling with a user-friendly interface directly in \TeX{}. The user
        supplies axis labels, legend entries and the plot coordinates for one or
        more plots and \PGFPlots{} applies axis scaling, computes any logarithms
        and axis ticks and draws the plots. It supports line plots, scatter
        plots, piecewise constant plots, bar plots, area plots, mesh and surface
        plots, patch plots, contour plots, quiver plots, histogram plots, box
        plots, polar axes, ternary diagrams, smith charts and some more. It is
        based on Till Tantau's package \PGF{}/\Tikz{}.
    \end{minipage}

}%

\expandafter\date\expandafter{\@date\\[2cm]
    \abstractsmuggle
}%
\makeatother

%\includeonly{pgfplots.reference}


\begin{document}

\def\plotcoords{%
\addplot coordinates {
(5,8.312e-02)    (17,2.547e-02)   (49,7.407e-03)
(129,2.102e-03)  (321,5.874e-04)  (769,1.623e-04)
(1793,4.442e-05) (4097,1.207e-05) (9217,3.261e-06)
};

\addplot coordinates{
(7,8.472e-02)    (31,3.044e-02)    (111,1.022e-02)
(351,3.303e-03)  (1023,1.039e-03)  (2815,3.196e-04)
(7423,9.658e-05) (18943,2.873e-05) (47103,8.437e-06)
};

\addplot coordinates{
(9,7.881e-02)     (49,3.243e-02)    (209,1.232e-02)
(769,4.454e-03)   (2561,1.551e-03)  (7937,5.236e-04)
(23297,1.723e-04) (65537,5.545e-05) (178177,1.751e-05)
};

\addplot coordinates{
(11,6.887e-02)    (71,3.177e-02)     (351,1.341e-02)
(1471,5.334e-03)  (5503,2.027e-03)   (18943,7.415e-04)
(61183,2.628e-04) (187903,9.063e-05) (553983,3.053e-05)
};

\addplot coordinates{
(13,5.755e-02)     (97,2.925e-02)     (545,1.351e-02)
(2561,5.842e-03)   (10625,2.397e-03)  (40193,9.414e-04)
(141569,3.564e-04) (471041,1.308e-04)
(1496065,4.670e-05)
};
}%


\bookmaketitle
\tableofcontents
\include{pgfplots.title_abstract_intro}
\include{pgfplots.preliminaries}
\include{pgfplots.intro}
\include{pgfplots.reference}
\include{pgfplots.libs}
\include{pgfplots.resources}
\include{pgfplots.importexport}
\include{pgfplots.basic.reference}

\printindex

\bibliographystyle{abbrv} %gerapali} %gerabbrv} %gerunsrt.bst} %gerabbrv}% gerplain}
\nocite{pgfplotstable}
\nocite{programmingnotes}
\bibliography{pgfplots}
\end{document}

% -----------------------------------------------------------------------------
% For Stefan Pinnow as reminder on what to look for when editing the manual
% -----------------------------------------------------------------------------
% There should be no line breaks in the following environments
% - |...|
% - \declareandlabel{...}
% - \verbpdfref{...}
% If "MakeTikzPictures" isn't running through check one of the externalized
% LOG files what is the cause of that.
% -----------------------------------------------------------------------------


%%%%%%%%%%%%%%%%%%%%%%%%%%%%%%%%%%%%%%%%%%%%%%%%%%%%%%%%%%%%%%%%%%%%%%%%%%%%%
%
% Package pgfplots.sty documentation.
%
% Copyright 2007/2008 by Christian Feuersaenger.
%
% This program is free software: you can redistribute it and/or modify
% it under the terms of the GNU General Public License as published by
% the Free Software Foundation, either version 3 of the License, or
% (at your option) any later version.
%
% This program is distributed in the hope that it will be useful,
% but WITHOUT ANY WARRANTY; without even the implied warranty of
% MERCHANTABILITY or FITNESS FOR A PARTICULAR PURPOSE.  See the
% GNU General Public License for more details.
%
% You should have received a copy of the GNU General Public License
% along with this program.  If not, see <http://www.gnu.org/licenses/>.
%
%
%%%%%%%%%%%%%%%%%%%%%%%%%%%%%%%%%%%%%%%%%%%%%%%%%%%%%%%%%%%%%%%%%%%%%%%%%%%%%
% SEE pgfplots-macros.tex as well!
%\pdfminorversion=5 % to allow compression
%\pdfobjcompresslevel=2
\documentclass[a4paper,openany]{book}

\let\bookmaketitle=\maketitle

% -----------------------------------
% this here is from ltxdoc:
\usepackage{doc}[=v2]
\AtBeginDocument{\MakeShortVerb{\|}}
%\setlength{\textwidth}{355pt}
%\addtolength\marginparwidth{30pt}
%\addtolength\oddsidemargin{20pt}
%\addtolength\evensidemargin{20pt}
\def\cmd#1{\cs{\expandafter\cmd@to@cs\string#1}}
\def\cmd@to@cs#1#2{\char\number`#2\relax}
\DeclareRobustCommand\cs[1]{\texttt{\char`\\#1}}
\providecommand\marg[1]{%
  {\ttfamily\char`\{}\meta{#1}{\ttfamily\char`\}}}
\providecommand\oarg[1]{%
  {\ttfamily[}\meta{#1}{\ttfamily]}}
\providecommand\parg[1]{%
  {\ttfamily(}\meta{#1}{\ttfamily)}}
\raggedbottom

% -----------------------------------
\input{pgfplots.preamble.tex}

\makeatletter
% I want a two-column index just as in pgfmanual styles. This here
% was the best way to get one:
\def\index@prologue{\section*{Index}\addcontentsline{toc}{chapter}{Index}
}
\makeatother

%\RequirePackage[german,english,francais]{babel}

\def\matlabcolormaptext{This colormap is similar to one shipped with Matlab$^\text{\textregistered}$ under a similar name.}

\IfFileExists{tikzlibraryspy.code.tex}{%
\usetikzlibrary{spy}
}{%
    \message{ERROR: tikz SPY library NOT available. The manual will only compile partially.^^J}%
}%

\usepackage{xparse}% for colorbrewer manual

\usetikzlibrary{
    decorations.markings,
    decorations.footprints,
    shapes.arrows,
    matrix,
    positioning,
}

\usepgfplotslibrary{
    fillbetween,
    ternary,
    smithchart,
    patchplots,
    polar,
    colormaps,
    colorbrewer,
    colortol,           % docu not ready yet
}
\pgfqkeys{/codeexample}{%
    every codeexample/.append style={
        /pgfplots/every ternary axis/.append style={
            /pgfplots/legend style={fill=graphicbackground},
        }
    },
    tabsize=4,
}

\pgfplotsmanualenableexternalizationofexpensive

%\usetikzlibrary{external}
%\tikzexternalize[prefix=figures/]{pgfplots}

\title{%
    Manual for Package \PGFPlots{}\\
    {\small 2D/3D Plots in \LaTeX{}, Version \pgfplotsversion}\\
    {\small\url{https://github.com/pgf-tikz/pgfplots}}
    %\\{\small Attention: you are using an unstable development version.}
}

\makeatletter
\long\def\abstractsmuggle{%
    \centering
    \textbf{Abstract}\\[0.5cm]

    \begin{minipage}{12cm}
        \PGFPlots{} draws high-quality function plots in normal or logarithmic
        scaling with a user-friendly interface directly in \TeX{}. The user
        supplies axis labels, legend entries and the plot coordinates for one or
        more plots and \PGFPlots{} applies axis scaling, computes any logarithms
        and axis ticks and draws the plots. It supports line plots, scatter
        plots, piecewise constant plots, bar plots, area plots, mesh and surface
        plots, patch plots, contour plots, quiver plots, histogram plots, box
        plots, polar axes, ternary diagrams, smith charts and some more. It is
        based on Till Tantau's package \PGF{}/\Tikz{}.
    \end{minipage}

}%

\expandafter\date\expandafter{\@date\\[2cm]
    \abstractsmuggle
}%
\makeatother

%\includeonly{pgfplots.reference}


\begin{document}

\def\plotcoords{%
\addplot coordinates {
(5,8.312e-02)    (17,2.547e-02)   (49,7.407e-03)
(129,2.102e-03)  (321,5.874e-04)  (769,1.623e-04)
(1793,4.442e-05) (4097,1.207e-05) (9217,3.261e-06)
};

\addplot coordinates{
(7,8.472e-02)    (31,3.044e-02)    (111,1.022e-02)
(351,3.303e-03)  (1023,1.039e-03)  (2815,3.196e-04)
(7423,9.658e-05) (18943,2.873e-05) (47103,8.437e-06)
};

\addplot coordinates{
(9,7.881e-02)     (49,3.243e-02)    (209,1.232e-02)
(769,4.454e-03)   (2561,1.551e-03)  (7937,5.236e-04)
(23297,1.723e-04) (65537,5.545e-05) (178177,1.751e-05)
};

\addplot coordinates{
(11,6.887e-02)    (71,3.177e-02)     (351,1.341e-02)
(1471,5.334e-03)  (5503,2.027e-03)   (18943,7.415e-04)
(61183,2.628e-04) (187903,9.063e-05) (553983,3.053e-05)
};

\addplot coordinates{
(13,5.755e-02)     (97,2.925e-02)     (545,1.351e-02)
(2561,5.842e-03)   (10625,2.397e-03)  (40193,9.414e-04)
(141569,3.564e-04) (471041,1.308e-04)
(1496065,4.670e-05)
};
}%


\bookmaketitle
\tableofcontents
\include{pgfplots.title_abstract_intro}
\include{pgfplots.preliminaries}
\include{pgfplots.intro}
\include{pgfplots.reference}
\include{pgfplots.libs}
\include{pgfplots.resources}
\include{pgfplots.importexport}
\include{pgfplots.basic.reference}

\printindex

\bibliographystyle{abbrv} %gerapali} %gerabbrv} %gerunsrt.bst} %gerabbrv}% gerplain}
\nocite{pgfplotstable}
\nocite{programmingnotes}
\bibliography{pgfplots}
\end{document}

% -----------------------------------------------------------------------------
% For Stefan Pinnow as reminder on what to look for when editing the manual
% -----------------------------------------------------------------------------
% There should be no line breaks in the following environments
% - |...|
% - \declareandlabel{...}
% - \verbpdfref{...}
% If "MakeTikzPictures" isn't running through check one of the externalized
% LOG files what is the cause of that.
% -----------------------------------------------------------------------------


%%%%%%%%%%%%%%%%%%%%%%%%%%%%%%%%%%%%%%%%%%%%%%%%%%%%%%%%%%%%%%%%%%%%%%%%%%%%%
%
% Package pgfplots.sty documentation.
%
% Copyright 2007/2008 by Christian Feuersaenger.
%
% This program is free software: you can redistribute it and/or modify
% it under the terms of the GNU General Public License as published by
% the Free Software Foundation, either version 3 of the License, or
% (at your option) any later version.
%
% This program is distributed in the hope that it will be useful,
% but WITHOUT ANY WARRANTY; without even the implied warranty of
% MERCHANTABILITY or FITNESS FOR A PARTICULAR PURPOSE.  See the
% GNU General Public License for more details.
%
% You should have received a copy of the GNU General Public License
% along with this program.  If not, see <http://www.gnu.org/licenses/>.
%
%
%%%%%%%%%%%%%%%%%%%%%%%%%%%%%%%%%%%%%%%%%%%%%%%%%%%%%%%%%%%%%%%%%%%%%%%%%%%%%
% SEE pgfplots-macros.tex as well!
%\pdfminorversion=5 % to allow compression
%\pdfobjcompresslevel=2
\documentclass[a4paper,openany]{book}

\let\bookmaketitle=\maketitle

% -----------------------------------
% this here is from ltxdoc:
\usepackage{doc}[=v2]
\AtBeginDocument{\MakeShortVerb{\|}}
%\setlength{\textwidth}{355pt}
%\addtolength\marginparwidth{30pt}
%\addtolength\oddsidemargin{20pt}
%\addtolength\evensidemargin{20pt}
\def\cmd#1{\cs{\expandafter\cmd@to@cs\string#1}}
\def\cmd@to@cs#1#2{\char\number`#2\relax}
\DeclareRobustCommand\cs[1]{\texttt{\char`\\#1}}
\providecommand\marg[1]{%
  {\ttfamily\char`\{}\meta{#1}{\ttfamily\char`\}}}
\providecommand\oarg[1]{%
  {\ttfamily[}\meta{#1}{\ttfamily]}}
\providecommand\parg[1]{%
  {\ttfamily(}\meta{#1}{\ttfamily)}}
\raggedbottom

% -----------------------------------
\input{pgfplots.preamble.tex}

\makeatletter
% I want a two-column index just as in pgfmanual styles. This here
% was the best way to get one:
\def\index@prologue{\section*{Index}\addcontentsline{toc}{chapter}{Index}
}
\makeatother

%\RequirePackage[german,english,francais]{babel}

\def\matlabcolormaptext{This colormap is similar to one shipped with Matlab$^\text{\textregistered}$ under a similar name.}

\IfFileExists{tikzlibraryspy.code.tex}{%
\usetikzlibrary{spy}
}{%
    \message{ERROR: tikz SPY library NOT available. The manual will only compile partially.^^J}%
}%

\usepackage{xparse}% for colorbrewer manual

\usetikzlibrary{
    decorations.markings,
    decorations.footprints,
    shapes.arrows,
    matrix,
    positioning,
}

\usepgfplotslibrary{
    fillbetween,
    ternary,
    smithchart,
    patchplots,
    polar,
    colormaps,
    colorbrewer,
    colortol,           % docu not ready yet
}
\pgfqkeys{/codeexample}{%
    every codeexample/.append style={
        /pgfplots/every ternary axis/.append style={
            /pgfplots/legend style={fill=graphicbackground},
        }
    },
    tabsize=4,
}

\pgfplotsmanualenableexternalizationofexpensive

%\usetikzlibrary{external}
%\tikzexternalize[prefix=figures/]{pgfplots}

\title{%
    Manual for Package \PGFPlots{}\\
    {\small 2D/3D Plots in \LaTeX{}, Version \pgfplotsversion}\\
    {\small\url{https://github.com/pgf-tikz/pgfplots}}
    %\\{\small Attention: you are using an unstable development version.}
}

\makeatletter
\long\def\abstractsmuggle{%
    \centering
    \textbf{Abstract}\\[0.5cm]

    \begin{minipage}{12cm}
        \PGFPlots{} draws high-quality function plots in normal or logarithmic
        scaling with a user-friendly interface directly in \TeX{}. The user
        supplies axis labels, legend entries and the plot coordinates for one or
        more plots and \PGFPlots{} applies axis scaling, computes any logarithms
        and axis ticks and draws the plots. It supports line plots, scatter
        plots, piecewise constant plots, bar plots, area plots, mesh and surface
        plots, patch plots, contour plots, quiver plots, histogram plots, box
        plots, polar axes, ternary diagrams, smith charts and some more. It is
        based on Till Tantau's package \PGF{}/\Tikz{}.
    \end{minipage}

}%

\expandafter\date\expandafter{\@date\\[2cm]
    \abstractsmuggle
}%
\makeatother

%\includeonly{pgfplots.reference}


\begin{document}

\def\plotcoords{%
\addplot coordinates {
(5,8.312e-02)    (17,2.547e-02)   (49,7.407e-03)
(129,2.102e-03)  (321,5.874e-04)  (769,1.623e-04)
(1793,4.442e-05) (4097,1.207e-05) (9217,3.261e-06)
};

\addplot coordinates{
(7,8.472e-02)    (31,3.044e-02)    (111,1.022e-02)
(351,3.303e-03)  (1023,1.039e-03)  (2815,3.196e-04)
(7423,9.658e-05) (18943,2.873e-05) (47103,8.437e-06)
};

\addplot coordinates{
(9,7.881e-02)     (49,3.243e-02)    (209,1.232e-02)
(769,4.454e-03)   (2561,1.551e-03)  (7937,5.236e-04)
(23297,1.723e-04) (65537,5.545e-05) (178177,1.751e-05)
};

\addplot coordinates{
(11,6.887e-02)    (71,3.177e-02)     (351,1.341e-02)
(1471,5.334e-03)  (5503,2.027e-03)   (18943,7.415e-04)
(61183,2.628e-04) (187903,9.063e-05) (553983,3.053e-05)
};

\addplot coordinates{
(13,5.755e-02)     (97,2.925e-02)     (545,1.351e-02)
(2561,5.842e-03)   (10625,2.397e-03)  (40193,9.414e-04)
(141569,3.564e-04) (471041,1.308e-04)
(1496065,4.670e-05)
};
}%


\bookmaketitle
\tableofcontents
\include{pgfplots.title_abstract_intro}
\include{pgfplots.preliminaries}
\include{pgfplots.intro}
\include{pgfplots.reference}
\include{pgfplots.libs}
\include{pgfplots.resources}
\include{pgfplots.importexport}
\include{pgfplots.basic.reference}

\printindex

\bibliographystyle{abbrv} %gerapali} %gerabbrv} %gerunsrt.bst} %gerabbrv}% gerplain}
\nocite{pgfplotstable}
\nocite{programmingnotes}
\bibliography{pgfplots}
\end{document}

% -----------------------------------------------------------------------------
% For Stefan Pinnow as reminder on what to look for when editing the manual
% -----------------------------------------------------------------------------
% There should be no line breaks in the following environments
% - |...|
% - \declareandlabel{...}
% - \verbpdfref{...}
% If "MakeTikzPictures" isn't running through check one of the externalized
% LOG files what is the cause of that.
% -----------------------------------------------------------------------------


%%%%%%%%%%%%%%%%%%%%%%%%%%%%%%%%%%%%%%%%%%%%%%%%%%%%%%%%%%%%%%%%%%%%%%%%%%%%%
%
% Package pgfplots.sty documentation.
%
% Copyright 2007/2008 by Christian Feuersaenger.
%
% This program is free software: you can redistribute it and/or modify
% it under the terms of the GNU General Public License as published by
% the Free Software Foundation, either version 3 of the License, or
% (at your option) any later version.
%
% This program is distributed in the hope that it will be useful,
% but WITHOUT ANY WARRANTY; without even the implied warranty of
% MERCHANTABILITY or FITNESS FOR A PARTICULAR PURPOSE.  See the
% GNU General Public License for more details.
%
% You should have received a copy of the GNU General Public License
% along with this program.  If not, see <http://www.gnu.org/licenses/>.
%
%
%%%%%%%%%%%%%%%%%%%%%%%%%%%%%%%%%%%%%%%%%%%%%%%%%%%%%%%%%%%%%%%%%%%%%%%%%%%%%
% SEE pgfplots-macros.tex as well!
%\pdfminorversion=5 % to allow compression
%\pdfobjcompresslevel=2
\documentclass[a4paper,openany]{book}

\let\bookmaketitle=\maketitle

% -----------------------------------
% this here is from ltxdoc:
\usepackage{doc}[=v2]
\AtBeginDocument{\MakeShortVerb{\|}}
%\setlength{\textwidth}{355pt}
%\addtolength\marginparwidth{30pt}
%\addtolength\oddsidemargin{20pt}
%\addtolength\evensidemargin{20pt}
\def\cmd#1{\cs{\expandafter\cmd@to@cs\string#1}}
\def\cmd@to@cs#1#2{\char\number`#2\relax}
\DeclareRobustCommand\cs[1]{\texttt{\char`\\#1}}
\providecommand\marg[1]{%
  {\ttfamily\char`\{}\meta{#1}{\ttfamily\char`\}}}
\providecommand\oarg[1]{%
  {\ttfamily[}\meta{#1}{\ttfamily]}}
\providecommand\parg[1]{%
  {\ttfamily(}\meta{#1}{\ttfamily)}}
\raggedbottom

% -----------------------------------
\input{pgfplots.preamble.tex}

\makeatletter
% I want a two-column index just as in pgfmanual styles. This here
% was the best way to get one:
\def\index@prologue{\section*{Index}\addcontentsline{toc}{chapter}{Index}
}
\makeatother

%\RequirePackage[german,english,francais]{babel}

\def\matlabcolormaptext{This colormap is similar to one shipped with Matlab$^\text{\textregistered}$ under a similar name.}

\IfFileExists{tikzlibraryspy.code.tex}{%
\usetikzlibrary{spy}
}{%
    \message{ERROR: tikz SPY library NOT available. The manual will only compile partially.^^J}%
}%

\usepackage{xparse}% for colorbrewer manual

\usetikzlibrary{
    decorations.markings,
    decorations.footprints,
    shapes.arrows,
    matrix,
    positioning,
}

\usepgfplotslibrary{
    fillbetween,
    ternary,
    smithchart,
    patchplots,
    polar,
    colormaps,
    colorbrewer,
    colortol,           % docu not ready yet
}
\pgfqkeys{/codeexample}{%
    every codeexample/.append style={
        /pgfplots/every ternary axis/.append style={
            /pgfplots/legend style={fill=graphicbackground},
        }
    },
    tabsize=4,
}

\pgfplotsmanualenableexternalizationofexpensive

%\usetikzlibrary{external}
%\tikzexternalize[prefix=figures/]{pgfplots}

\title{%
    Manual for Package \PGFPlots{}\\
    {\small 2D/3D Plots in \LaTeX{}, Version \pgfplotsversion}\\
    {\small\url{https://github.com/pgf-tikz/pgfplots}}
    %\\{\small Attention: you are using an unstable development version.}
}

\makeatletter
\long\def\abstractsmuggle{%
    \centering
    \textbf{Abstract}\\[0.5cm]

    \begin{minipage}{12cm}
        \PGFPlots{} draws high-quality function plots in normal or logarithmic
        scaling with a user-friendly interface directly in \TeX{}. The user
        supplies axis labels, legend entries and the plot coordinates for one or
        more plots and \PGFPlots{} applies axis scaling, computes any logarithms
        and axis ticks and draws the plots. It supports line plots, scatter
        plots, piecewise constant plots, bar plots, area plots, mesh and surface
        plots, patch plots, contour plots, quiver plots, histogram plots, box
        plots, polar axes, ternary diagrams, smith charts and some more. It is
        based on Till Tantau's package \PGF{}/\Tikz{}.
    \end{minipage}

}%

\expandafter\date\expandafter{\@date\\[2cm]
    \abstractsmuggle
}%
\makeatother

%\includeonly{pgfplots.reference}


\begin{document}

\def\plotcoords{%
\addplot coordinates {
(5,8.312e-02)    (17,2.547e-02)   (49,7.407e-03)
(129,2.102e-03)  (321,5.874e-04)  (769,1.623e-04)
(1793,4.442e-05) (4097,1.207e-05) (9217,3.261e-06)
};

\addplot coordinates{
(7,8.472e-02)    (31,3.044e-02)    (111,1.022e-02)
(351,3.303e-03)  (1023,1.039e-03)  (2815,3.196e-04)
(7423,9.658e-05) (18943,2.873e-05) (47103,8.437e-06)
};

\addplot coordinates{
(9,7.881e-02)     (49,3.243e-02)    (209,1.232e-02)
(769,4.454e-03)   (2561,1.551e-03)  (7937,5.236e-04)
(23297,1.723e-04) (65537,5.545e-05) (178177,1.751e-05)
};

\addplot coordinates{
(11,6.887e-02)    (71,3.177e-02)     (351,1.341e-02)
(1471,5.334e-03)  (5503,2.027e-03)   (18943,7.415e-04)
(61183,2.628e-04) (187903,9.063e-05) (553983,3.053e-05)
};

\addplot coordinates{
(13,5.755e-02)     (97,2.925e-02)     (545,1.351e-02)
(2561,5.842e-03)   (10625,2.397e-03)  (40193,9.414e-04)
(141569,3.564e-04) (471041,1.308e-04)
(1496065,4.670e-05)
};
}%


\bookmaketitle
\tableofcontents
\include{pgfplots.title_abstract_intro}
\include{pgfplots.preliminaries}
\include{pgfplots.intro}
\include{pgfplots.reference}
\include{pgfplots.libs}
\include{pgfplots.resources}
\include{pgfplots.importexport}
\include{pgfplots.basic.reference}

\printindex

\bibliographystyle{abbrv} %gerapali} %gerabbrv} %gerunsrt.bst} %gerabbrv}% gerplain}
\nocite{pgfplotstable}
\nocite{programmingnotes}
\bibliography{pgfplots}
\end{document}

% -----------------------------------------------------------------------------
% For Stefan Pinnow as reminder on what to look for when editing the manual
% -----------------------------------------------------------------------------
% There should be no line breaks in the following environments
% - |...|
% - \declareandlabel{...}
% - \verbpdfref{...}
% If "MakeTikzPictures" isn't running through check one of the externalized
% LOG files what is the cause of that.
% -----------------------------------------------------------------------------


%%%%%%%%%%%%%%%%%%%%%%%%%%%%%%%%%%%%%%%%%%%%%%%%%%%%%%%%%%%%%%%%%%%%%%%%%%%%%
%
% Package pgfplots.sty documentation.
%
% Copyright 2007/2008 by Christian Feuersaenger.
%
% This program is free software: you can redistribute it and/or modify
% it under the terms of the GNU General Public License as published by
% the Free Software Foundation, either version 3 of the License, or
% (at your option) any later version.
%
% This program is distributed in the hope that it will be useful,
% but WITHOUT ANY WARRANTY; without even the implied warranty of
% MERCHANTABILITY or FITNESS FOR A PARTICULAR PURPOSE.  See the
% GNU General Public License for more details.
%
% You should have received a copy of the GNU General Public License
% along with this program.  If not, see <http://www.gnu.org/licenses/>.
%
%
%%%%%%%%%%%%%%%%%%%%%%%%%%%%%%%%%%%%%%%%%%%%%%%%%%%%%%%%%%%%%%%%%%%%%%%%%%%%%
% SEE pgfplots-macros.tex as well!
%\pdfminorversion=5 % to allow compression
%\pdfobjcompresslevel=2
\documentclass[a4paper,openany]{book}

\let\bookmaketitle=\maketitle

% -----------------------------------
% this here is from ltxdoc:
\usepackage{doc}[=v2]
\AtBeginDocument{\MakeShortVerb{\|}}
%\setlength{\textwidth}{355pt}
%\addtolength\marginparwidth{30pt}
%\addtolength\oddsidemargin{20pt}
%\addtolength\evensidemargin{20pt}
\def\cmd#1{\cs{\expandafter\cmd@to@cs\string#1}}
\def\cmd@to@cs#1#2{\char\number`#2\relax}
\DeclareRobustCommand\cs[1]{\texttt{\char`\\#1}}
\providecommand\marg[1]{%
  {\ttfamily\char`\{}\meta{#1}{\ttfamily\char`\}}}
\providecommand\oarg[1]{%
  {\ttfamily[}\meta{#1}{\ttfamily]}}
\providecommand\parg[1]{%
  {\ttfamily(}\meta{#1}{\ttfamily)}}
\raggedbottom

% -----------------------------------
\input{pgfplots.preamble.tex}

\makeatletter
% I want a two-column index just as in pgfmanual styles. This here
% was the best way to get one:
\def\index@prologue{\section*{Index}\addcontentsline{toc}{chapter}{Index}
}
\makeatother

%\RequirePackage[german,english,francais]{babel}

\def\matlabcolormaptext{This colormap is similar to one shipped with Matlab$^\text{\textregistered}$ under a similar name.}

\IfFileExists{tikzlibraryspy.code.tex}{%
\usetikzlibrary{spy}
}{%
    \message{ERROR: tikz SPY library NOT available. The manual will only compile partially.^^J}%
}%

\usepackage{xparse}% for colorbrewer manual

\usetikzlibrary{
    decorations.markings,
    decorations.footprints,
    shapes.arrows,
    matrix,
    positioning,
}

\usepgfplotslibrary{
    fillbetween,
    ternary,
    smithchart,
    patchplots,
    polar,
    colormaps,
    colorbrewer,
    colortol,           % docu not ready yet
}
\pgfqkeys{/codeexample}{%
    every codeexample/.append style={
        /pgfplots/every ternary axis/.append style={
            /pgfplots/legend style={fill=graphicbackground},
        }
    },
    tabsize=4,
}

\pgfplotsmanualenableexternalizationofexpensive

%\usetikzlibrary{external}
%\tikzexternalize[prefix=figures/]{pgfplots}

\title{%
    Manual for Package \PGFPlots{}\\
    {\small 2D/3D Plots in \LaTeX{}, Version \pgfplotsversion}\\
    {\small\url{https://github.com/pgf-tikz/pgfplots}}
    %\\{\small Attention: you are using an unstable development version.}
}

\makeatletter
\long\def\abstractsmuggle{%
    \centering
    \textbf{Abstract}\\[0.5cm]

    \begin{minipage}{12cm}
        \PGFPlots{} draws high-quality function plots in normal or logarithmic
        scaling with a user-friendly interface directly in \TeX{}. The user
        supplies axis labels, legend entries and the plot coordinates for one or
        more plots and \PGFPlots{} applies axis scaling, computes any logarithms
        and axis ticks and draws the plots. It supports line plots, scatter
        plots, piecewise constant plots, bar plots, area plots, mesh and surface
        plots, patch plots, contour plots, quiver plots, histogram plots, box
        plots, polar axes, ternary diagrams, smith charts and some more. It is
        based on Till Tantau's package \PGF{}/\Tikz{}.
    \end{minipage}

}%

\expandafter\date\expandafter{\@date\\[2cm]
    \abstractsmuggle
}%
\makeatother

%\includeonly{pgfplots.reference}


\begin{document}

\def\plotcoords{%
\addplot coordinates {
(5,8.312e-02)    (17,2.547e-02)   (49,7.407e-03)
(129,2.102e-03)  (321,5.874e-04)  (769,1.623e-04)
(1793,4.442e-05) (4097,1.207e-05) (9217,3.261e-06)
};

\addplot coordinates{
(7,8.472e-02)    (31,3.044e-02)    (111,1.022e-02)
(351,3.303e-03)  (1023,1.039e-03)  (2815,3.196e-04)
(7423,9.658e-05) (18943,2.873e-05) (47103,8.437e-06)
};

\addplot coordinates{
(9,7.881e-02)     (49,3.243e-02)    (209,1.232e-02)
(769,4.454e-03)   (2561,1.551e-03)  (7937,5.236e-04)
(23297,1.723e-04) (65537,5.545e-05) (178177,1.751e-05)
};

\addplot coordinates{
(11,6.887e-02)    (71,3.177e-02)     (351,1.341e-02)
(1471,5.334e-03)  (5503,2.027e-03)   (18943,7.415e-04)
(61183,2.628e-04) (187903,9.063e-05) (553983,3.053e-05)
};

\addplot coordinates{
(13,5.755e-02)     (97,2.925e-02)     (545,1.351e-02)
(2561,5.842e-03)   (10625,2.397e-03)  (40193,9.414e-04)
(141569,3.564e-04) (471041,1.308e-04)
(1496065,4.670e-05)
};
}%


\bookmaketitle
\tableofcontents
\include{pgfplots.title_abstract_intro}
\include{pgfplots.preliminaries}
\include{pgfplots.intro}
\include{pgfplots.reference}
\include{pgfplots.libs}
\include{pgfplots.resources}
\include{pgfplots.importexport}
\include{pgfplots.basic.reference}

\printindex

\bibliographystyle{abbrv} %gerapali} %gerabbrv} %gerunsrt.bst} %gerabbrv}% gerplain}
\nocite{pgfplotstable}
\nocite{programmingnotes}
\bibliography{pgfplots}
\end{document}

% -----------------------------------------------------------------------------
% For Stefan Pinnow as reminder on what to look for when editing the manual
% -----------------------------------------------------------------------------
% There should be no line breaks in the following environments
% - |...|
% - \declareandlabel{...}
% - \verbpdfref{...}
% If "MakeTikzPictures" isn't running through check one of the externalized
% LOG files what is the cause of that.
% -----------------------------------------------------------------------------

\include{pgfplots.basic.reference}

\printindex

\bibliographystyle{abbrv} %gerapali} %gerabbrv} %gerunsrt.bst} %gerabbrv}% gerplain}
\nocite{pgfplotstable}
\nocite{programmingnotes}
\bibliography{pgfplots}
\end{document}

% -----------------------------------------------------------------------------
% For Stefan Pinnow as reminder on what to look for when editing the manual
% -----------------------------------------------------------------------------
% There should be no line breaks in the following environments
% - |...|
% - \declareandlabel{...}
% - \verbpdfref{...}
% If "MakeTikzPictures" isn't running through check one of the externalized
% LOG files what is the cause of that.
% -----------------------------------------------------------------------------


%%%%%%%%%%%%%%%%%%%%%%%%%%%%%%%%%%%%%%%%%%%%%%%%%%%%%%%%%%%%%%%%%%%%%%%%%%%%%
%
% Package pgfplots.sty documentation.
%
% Copyright 2007/2008 by Christian Feuersaenger.
%
% This program is free software: you can redistribute it and/or modify
% it under the terms of the GNU General Public License as published by
% the Free Software Foundation, either version 3 of the License, or
% (at your option) any later version.
%
% This program is distributed in the hope that it will be useful,
% but WITHOUT ANY WARRANTY; without even the implied warranty of
% MERCHANTABILITY or FITNESS FOR A PARTICULAR PURPOSE.  See the
% GNU General Public License for more details.
%
% You should have received a copy of the GNU General Public License
% along with this program.  If not, see <http://www.gnu.org/licenses/>.
%
%
%%%%%%%%%%%%%%%%%%%%%%%%%%%%%%%%%%%%%%%%%%%%%%%%%%%%%%%%%%%%%%%%%%%%%%%%%%%%%
% SEE pgfplots-macros.tex as well!
%\pdfminorversion=5 % to allow compression
%\pdfobjcompresslevel=2
\documentclass[a4paper,openany]{book}

\let\bookmaketitle=\maketitle

% -----------------------------------
% this here is from ltxdoc:
\usepackage{doc}[=v2]
\AtBeginDocument{\MakeShortVerb{\|}}
%\setlength{\textwidth}{355pt}
%\addtolength\marginparwidth{30pt}
%\addtolength\oddsidemargin{20pt}
%\addtolength\evensidemargin{20pt}
\def\cmd#1{\cs{\expandafter\cmd@to@cs\string#1}}
\def\cmd@to@cs#1#2{\char\number`#2\relax}
\DeclareRobustCommand\cs[1]{\texttt{\char`\\#1}}
\providecommand\marg[1]{%
  {\ttfamily\char`\{}\meta{#1}{\ttfamily\char`\}}}
\providecommand\oarg[1]{%
  {\ttfamily[}\meta{#1}{\ttfamily]}}
\providecommand\parg[1]{%
  {\ttfamily(}\meta{#1}{\ttfamily)}}
\raggedbottom

% -----------------------------------
\input{pgfplots.preamble.tex}

\makeatletter
% I want a two-column index just as in pgfmanual styles. This here
% was the best way to get one:
\def\index@prologue{\section*{Index}\addcontentsline{toc}{chapter}{Index}
}
\makeatother

%\RequirePackage[german,english,francais]{babel}

\def\matlabcolormaptext{This colormap is similar to one shipped with Matlab$^\text{\textregistered}$ under a similar name.}

\IfFileExists{tikzlibraryspy.code.tex}{%
\usetikzlibrary{spy}
}{%
    \message{ERROR: tikz SPY library NOT available. The manual will only compile partially.^^J}%
}%

\usepackage{xparse}% for colorbrewer manual

\usetikzlibrary{
    decorations.markings,
    decorations.footprints,
    shapes.arrows,
    matrix,
    positioning,
}

\usepgfplotslibrary{
    fillbetween,
    ternary,
    smithchart,
    patchplots,
    polar,
    colormaps,
    colorbrewer,
    colortol,           % docu not ready yet
}
\pgfqkeys{/codeexample}{%
    every codeexample/.append style={
        /pgfplots/every ternary axis/.append style={
            /pgfplots/legend style={fill=graphicbackground},
        }
    },
    tabsize=4,
}

\pgfplotsmanualenableexternalizationofexpensive

%\usetikzlibrary{external}
%\tikzexternalize[prefix=figures/]{pgfplots}

\title{%
    Manual for Package \PGFPlots{}\\
    {\small 2D/3D Plots in \LaTeX{}, Version \pgfplotsversion}\\
    {\small\url{https://github.com/pgf-tikz/pgfplots}}
    %\\{\small Attention: you are using an unstable development version.}
}

\makeatletter
\long\def\abstractsmuggle{%
    \centering
    \textbf{Abstract}\\[0.5cm]

    \begin{minipage}{12cm}
        \PGFPlots{} draws high-quality function plots in normal or logarithmic
        scaling with a user-friendly interface directly in \TeX{}. The user
        supplies axis labels, legend entries and the plot coordinates for one or
        more plots and \PGFPlots{} applies axis scaling, computes any logarithms
        and axis ticks and draws the plots. It supports line plots, scatter
        plots, piecewise constant plots, bar plots, area plots, mesh and surface
        plots, patch plots, contour plots, quiver plots, histogram plots, box
        plots, polar axes, ternary diagrams, smith charts and some more. It is
        based on Till Tantau's package \PGF{}/\Tikz{}.
    \end{minipage}

}%

\expandafter\date\expandafter{\@date\\[2cm]
    \abstractsmuggle
}%
\makeatother

%\includeonly{pgfplots.reference}


\begin{document}

\def\plotcoords{%
\addplot coordinates {
(5,8.312e-02)    (17,2.547e-02)   (49,7.407e-03)
(129,2.102e-03)  (321,5.874e-04)  (769,1.623e-04)
(1793,4.442e-05) (4097,1.207e-05) (9217,3.261e-06)
};

\addplot coordinates{
(7,8.472e-02)    (31,3.044e-02)    (111,1.022e-02)
(351,3.303e-03)  (1023,1.039e-03)  (2815,3.196e-04)
(7423,9.658e-05) (18943,2.873e-05) (47103,8.437e-06)
};

\addplot coordinates{
(9,7.881e-02)     (49,3.243e-02)    (209,1.232e-02)
(769,4.454e-03)   (2561,1.551e-03)  (7937,5.236e-04)
(23297,1.723e-04) (65537,5.545e-05) (178177,1.751e-05)
};

\addplot coordinates{
(11,6.887e-02)    (71,3.177e-02)     (351,1.341e-02)
(1471,5.334e-03)  (5503,2.027e-03)   (18943,7.415e-04)
(61183,2.628e-04) (187903,9.063e-05) (553983,3.053e-05)
};

\addplot coordinates{
(13,5.755e-02)     (97,2.925e-02)     (545,1.351e-02)
(2561,5.842e-03)   (10625,2.397e-03)  (40193,9.414e-04)
(141569,3.564e-04) (471041,1.308e-04)
(1496065,4.670e-05)
};
}%


\bookmaketitle
\tableofcontents

%%%%%%%%%%%%%%%%%%%%%%%%%%%%%%%%%%%%%%%%%%%%%%%%%%%%%%%%%%%%%%%%%%%%%%%%%%%%%
%
% Package pgfplots.sty documentation.
%
% Copyright 2007/2008 by Christian Feuersaenger.
%
% This program is free software: you can redistribute it and/or modify
% it under the terms of the GNU General Public License as published by
% the Free Software Foundation, either version 3 of the License, or
% (at your option) any later version.
%
% This program is distributed in the hope that it will be useful,
% but WITHOUT ANY WARRANTY; without even the implied warranty of
% MERCHANTABILITY or FITNESS FOR A PARTICULAR PURPOSE.  See the
% GNU General Public License for more details.
%
% You should have received a copy of the GNU General Public License
% along with this program.  If not, see <http://www.gnu.org/licenses/>.
%
%
%%%%%%%%%%%%%%%%%%%%%%%%%%%%%%%%%%%%%%%%%%%%%%%%%%%%%%%%%%%%%%%%%%%%%%%%%%%%%
% SEE pgfplots-macros.tex as well!
%\pdfminorversion=5 % to allow compression
%\pdfobjcompresslevel=2
\documentclass[a4paper,openany]{book}

\let\bookmaketitle=\maketitle

% -----------------------------------
% this here is from ltxdoc:
\usepackage{doc}[=v2]
\AtBeginDocument{\MakeShortVerb{\|}}
%\setlength{\textwidth}{355pt}
%\addtolength\marginparwidth{30pt}
%\addtolength\oddsidemargin{20pt}
%\addtolength\evensidemargin{20pt}
\def\cmd#1{\cs{\expandafter\cmd@to@cs\string#1}}
\def\cmd@to@cs#1#2{\char\number`#2\relax}
\DeclareRobustCommand\cs[1]{\texttt{\char`\\#1}}
\providecommand\marg[1]{%
  {\ttfamily\char`\{}\meta{#1}{\ttfamily\char`\}}}
\providecommand\oarg[1]{%
  {\ttfamily[}\meta{#1}{\ttfamily]}}
\providecommand\parg[1]{%
  {\ttfamily(}\meta{#1}{\ttfamily)}}
\raggedbottom

% -----------------------------------
\input{pgfplots.preamble.tex}

\makeatletter
% I want a two-column index just as in pgfmanual styles. This here
% was the best way to get one:
\def\index@prologue{\section*{Index}\addcontentsline{toc}{chapter}{Index}
}
\makeatother

%\RequirePackage[german,english,francais]{babel}

\def\matlabcolormaptext{This colormap is similar to one shipped with Matlab$^\text{\textregistered}$ under a similar name.}

\IfFileExists{tikzlibraryspy.code.tex}{%
\usetikzlibrary{spy}
}{%
    \message{ERROR: tikz SPY library NOT available. The manual will only compile partially.^^J}%
}%

\usepackage{xparse}% for colorbrewer manual

\usetikzlibrary{
    decorations.markings,
    decorations.footprints,
    shapes.arrows,
    matrix,
    positioning,
}

\usepgfplotslibrary{
    fillbetween,
    ternary,
    smithchart,
    patchplots,
    polar,
    colormaps,
    colorbrewer,
    colortol,           % docu not ready yet
}
\pgfqkeys{/codeexample}{%
    every codeexample/.append style={
        /pgfplots/every ternary axis/.append style={
            /pgfplots/legend style={fill=graphicbackground},
        }
    },
    tabsize=4,
}

\pgfplotsmanualenableexternalizationofexpensive

%\usetikzlibrary{external}
%\tikzexternalize[prefix=figures/]{pgfplots}

\title{%
    Manual for Package \PGFPlots{}\\
    {\small 2D/3D Plots in \LaTeX{}, Version \pgfplotsversion}\\
    {\small\url{https://github.com/pgf-tikz/pgfplots}}
    %\\{\small Attention: you are using an unstable development version.}
}

\makeatletter
\long\def\abstractsmuggle{%
    \centering
    \textbf{Abstract}\\[0.5cm]

    \begin{minipage}{12cm}
        \PGFPlots{} draws high-quality function plots in normal or logarithmic
        scaling with a user-friendly interface directly in \TeX{}. The user
        supplies axis labels, legend entries and the plot coordinates for one or
        more plots and \PGFPlots{} applies axis scaling, computes any logarithms
        and axis ticks and draws the plots. It supports line plots, scatter
        plots, piecewise constant plots, bar plots, area plots, mesh and surface
        plots, patch plots, contour plots, quiver plots, histogram plots, box
        plots, polar axes, ternary diagrams, smith charts and some more. It is
        based on Till Tantau's package \PGF{}/\Tikz{}.
    \end{minipage}

}%

\expandafter\date\expandafter{\@date\\[2cm]
    \abstractsmuggle
}%
\makeatother

%\includeonly{pgfplots.reference}


\begin{document}

\def\plotcoords{%
\addplot coordinates {
(5,8.312e-02)    (17,2.547e-02)   (49,7.407e-03)
(129,2.102e-03)  (321,5.874e-04)  (769,1.623e-04)
(1793,4.442e-05) (4097,1.207e-05) (9217,3.261e-06)
};

\addplot coordinates{
(7,8.472e-02)    (31,3.044e-02)    (111,1.022e-02)
(351,3.303e-03)  (1023,1.039e-03)  (2815,3.196e-04)
(7423,9.658e-05) (18943,2.873e-05) (47103,8.437e-06)
};

\addplot coordinates{
(9,7.881e-02)     (49,3.243e-02)    (209,1.232e-02)
(769,4.454e-03)   (2561,1.551e-03)  (7937,5.236e-04)
(23297,1.723e-04) (65537,5.545e-05) (178177,1.751e-05)
};

\addplot coordinates{
(11,6.887e-02)    (71,3.177e-02)     (351,1.341e-02)
(1471,5.334e-03)  (5503,2.027e-03)   (18943,7.415e-04)
(61183,2.628e-04) (187903,9.063e-05) (553983,3.053e-05)
};

\addplot coordinates{
(13,5.755e-02)     (97,2.925e-02)     (545,1.351e-02)
(2561,5.842e-03)   (10625,2.397e-03)  (40193,9.414e-04)
(141569,3.564e-04) (471041,1.308e-04)
(1496065,4.670e-05)
};
}%


\bookmaketitle
\tableofcontents
\include{pgfplots.title_abstract_intro}
\include{pgfplots.preliminaries}
\include{pgfplots.intro}
\include{pgfplots.reference}
\include{pgfplots.libs}
\include{pgfplots.resources}
\include{pgfplots.importexport}
\include{pgfplots.basic.reference}

\printindex

\bibliographystyle{abbrv} %gerapali} %gerabbrv} %gerunsrt.bst} %gerabbrv}% gerplain}
\nocite{pgfplotstable}
\nocite{programmingnotes}
\bibliography{pgfplots}
\end{document}

% -----------------------------------------------------------------------------
% For Stefan Pinnow as reminder on what to look for when editing the manual
% -----------------------------------------------------------------------------
% There should be no line breaks in the following environments
% - |...|
% - \declareandlabel{...}
% - \verbpdfref{...}
% If "MakeTikzPictures" isn't running through check one of the externalized
% LOG files what is the cause of that.
% -----------------------------------------------------------------------------


%%%%%%%%%%%%%%%%%%%%%%%%%%%%%%%%%%%%%%%%%%%%%%%%%%%%%%%%%%%%%%%%%%%%%%%%%%%%%
%
% Package pgfplots.sty documentation.
%
% Copyright 2007/2008 by Christian Feuersaenger.
%
% This program is free software: you can redistribute it and/or modify
% it under the terms of the GNU General Public License as published by
% the Free Software Foundation, either version 3 of the License, or
% (at your option) any later version.
%
% This program is distributed in the hope that it will be useful,
% but WITHOUT ANY WARRANTY; without even the implied warranty of
% MERCHANTABILITY or FITNESS FOR A PARTICULAR PURPOSE.  See the
% GNU General Public License for more details.
%
% You should have received a copy of the GNU General Public License
% along with this program.  If not, see <http://www.gnu.org/licenses/>.
%
%
%%%%%%%%%%%%%%%%%%%%%%%%%%%%%%%%%%%%%%%%%%%%%%%%%%%%%%%%%%%%%%%%%%%%%%%%%%%%%
% SEE pgfplots-macros.tex as well!
%\pdfminorversion=5 % to allow compression
%\pdfobjcompresslevel=2
\documentclass[a4paper,openany]{book}

\let\bookmaketitle=\maketitle

% -----------------------------------
% this here is from ltxdoc:
\usepackage{doc}[=v2]
\AtBeginDocument{\MakeShortVerb{\|}}
%\setlength{\textwidth}{355pt}
%\addtolength\marginparwidth{30pt}
%\addtolength\oddsidemargin{20pt}
%\addtolength\evensidemargin{20pt}
\def\cmd#1{\cs{\expandafter\cmd@to@cs\string#1}}
\def\cmd@to@cs#1#2{\char\number`#2\relax}
\DeclareRobustCommand\cs[1]{\texttt{\char`\\#1}}
\providecommand\marg[1]{%
  {\ttfamily\char`\{}\meta{#1}{\ttfamily\char`\}}}
\providecommand\oarg[1]{%
  {\ttfamily[}\meta{#1}{\ttfamily]}}
\providecommand\parg[1]{%
  {\ttfamily(}\meta{#1}{\ttfamily)}}
\raggedbottom

% -----------------------------------
\input{pgfplots.preamble.tex}

\makeatletter
% I want a two-column index just as in pgfmanual styles. This here
% was the best way to get one:
\def\index@prologue{\section*{Index}\addcontentsline{toc}{chapter}{Index}
}
\makeatother

%\RequirePackage[german,english,francais]{babel}

\def\matlabcolormaptext{This colormap is similar to one shipped with Matlab$^\text{\textregistered}$ under a similar name.}

\IfFileExists{tikzlibraryspy.code.tex}{%
\usetikzlibrary{spy}
}{%
    \message{ERROR: tikz SPY library NOT available. The manual will only compile partially.^^J}%
}%

\usepackage{xparse}% for colorbrewer manual

\usetikzlibrary{
    decorations.markings,
    decorations.footprints,
    shapes.arrows,
    matrix,
    positioning,
}

\usepgfplotslibrary{
    fillbetween,
    ternary,
    smithchart,
    patchplots,
    polar,
    colormaps,
    colorbrewer,
    colortol,           % docu not ready yet
}
\pgfqkeys{/codeexample}{%
    every codeexample/.append style={
        /pgfplots/every ternary axis/.append style={
            /pgfplots/legend style={fill=graphicbackground},
        }
    },
    tabsize=4,
}

\pgfplotsmanualenableexternalizationofexpensive

%\usetikzlibrary{external}
%\tikzexternalize[prefix=figures/]{pgfplots}

\title{%
    Manual for Package \PGFPlots{}\\
    {\small 2D/3D Plots in \LaTeX{}, Version \pgfplotsversion}\\
    {\small\url{https://github.com/pgf-tikz/pgfplots}}
    %\\{\small Attention: you are using an unstable development version.}
}

\makeatletter
\long\def\abstractsmuggle{%
    \centering
    \textbf{Abstract}\\[0.5cm]

    \begin{minipage}{12cm}
        \PGFPlots{} draws high-quality function plots in normal or logarithmic
        scaling with a user-friendly interface directly in \TeX{}. The user
        supplies axis labels, legend entries and the plot coordinates for one or
        more plots and \PGFPlots{} applies axis scaling, computes any logarithms
        and axis ticks and draws the plots. It supports line plots, scatter
        plots, piecewise constant plots, bar plots, area plots, mesh and surface
        plots, patch plots, contour plots, quiver plots, histogram plots, box
        plots, polar axes, ternary diagrams, smith charts and some more. It is
        based on Till Tantau's package \PGF{}/\Tikz{}.
    \end{minipage}

}%

\expandafter\date\expandafter{\@date\\[2cm]
    \abstractsmuggle
}%
\makeatother

%\includeonly{pgfplots.reference}


\begin{document}

\def\plotcoords{%
\addplot coordinates {
(5,8.312e-02)    (17,2.547e-02)   (49,7.407e-03)
(129,2.102e-03)  (321,5.874e-04)  (769,1.623e-04)
(1793,4.442e-05) (4097,1.207e-05) (9217,3.261e-06)
};

\addplot coordinates{
(7,8.472e-02)    (31,3.044e-02)    (111,1.022e-02)
(351,3.303e-03)  (1023,1.039e-03)  (2815,3.196e-04)
(7423,9.658e-05) (18943,2.873e-05) (47103,8.437e-06)
};

\addplot coordinates{
(9,7.881e-02)     (49,3.243e-02)    (209,1.232e-02)
(769,4.454e-03)   (2561,1.551e-03)  (7937,5.236e-04)
(23297,1.723e-04) (65537,5.545e-05) (178177,1.751e-05)
};

\addplot coordinates{
(11,6.887e-02)    (71,3.177e-02)     (351,1.341e-02)
(1471,5.334e-03)  (5503,2.027e-03)   (18943,7.415e-04)
(61183,2.628e-04) (187903,9.063e-05) (553983,3.053e-05)
};

\addplot coordinates{
(13,5.755e-02)     (97,2.925e-02)     (545,1.351e-02)
(2561,5.842e-03)   (10625,2.397e-03)  (40193,9.414e-04)
(141569,3.564e-04) (471041,1.308e-04)
(1496065,4.670e-05)
};
}%


\bookmaketitle
\tableofcontents
\include{pgfplots.title_abstract_intro}
\include{pgfplots.preliminaries}
\include{pgfplots.intro}
\include{pgfplots.reference}
\include{pgfplots.libs}
\include{pgfplots.resources}
\include{pgfplots.importexport}
\include{pgfplots.basic.reference}

\printindex

\bibliographystyle{abbrv} %gerapali} %gerabbrv} %gerunsrt.bst} %gerabbrv}% gerplain}
\nocite{pgfplotstable}
\nocite{programmingnotes}
\bibliography{pgfplots}
\end{document}

% -----------------------------------------------------------------------------
% For Stefan Pinnow as reminder on what to look for when editing the manual
% -----------------------------------------------------------------------------
% There should be no line breaks in the following environments
% - |...|
% - \declareandlabel{...}
% - \verbpdfref{...}
% If "MakeTikzPictures" isn't running through check one of the externalized
% LOG files what is the cause of that.
% -----------------------------------------------------------------------------


%%%%%%%%%%%%%%%%%%%%%%%%%%%%%%%%%%%%%%%%%%%%%%%%%%%%%%%%%%%%%%%%%%%%%%%%%%%%%
%
% Package pgfplots.sty documentation.
%
% Copyright 2007/2008 by Christian Feuersaenger.
%
% This program is free software: you can redistribute it and/or modify
% it under the terms of the GNU General Public License as published by
% the Free Software Foundation, either version 3 of the License, or
% (at your option) any later version.
%
% This program is distributed in the hope that it will be useful,
% but WITHOUT ANY WARRANTY; without even the implied warranty of
% MERCHANTABILITY or FITNESS FOR A PARTICULAR PURPOSE.  See the
% GNU General Public License for more details.
%
% You should have received a copy of the GNU General Public License
% along with this program.  If not, see <http://www.gnu.org/licenses/>.
%
%
%%%%%%%%%%%%%%%%%%%%%%%%%%%%%%%%%%%%%%%%%%%%%%%%%%%%%%%%%%%%%%%%%%%%%%%%%%%%%
% SEE pgfplots-macros.tex as well!
%\pdfminorversion=5 % to allow compression
%\pdfobjcompresslevel=2
\documentclass[a4paper,openany]{book}

\let\bookmaketitle=\maketitle

% -----------------------------------
% this here is from ltxdoc:
\usepackage{doc}[=v2]
\AtBeginDocument{\MakeShortVerb{\|}}
%\setlength{\textwidth}{355pt}
%\addtolength\marginparwidth{30pt}
%\addtolength\oddsidemargin{20pt}
%\addtolength\evensidemargin{20pt}
\def\cmd#1{\cs{\expandafter\cmd@to@cs\string#1}}
\def\cmd@to@cs#1#2{\char\number`#2\relax}
\DeclareRobustCommand\cs[1]{\texttt{\char`\\#1}}
\providecommand\marg[1]{%
  {\ttfamily\char`\{}\meta{#1}{\ttfamily\char`\}}}
\providecommand\oarg[1]{%
  {\ttfamily[}\meta{#1}{\ttfamily]}}
\providecommand\parg[1]{%
  {\ttfamily(}\meta{#1}{\ttfamily)}}
\raggedbottom

% -----------------------------------
\input{pgfplots.preamble.tex}

\makeatletter
% I want a two-column index just as in pgfmanual styles. This here
% was the best way to get one:
\def\index@prologue{\section*{Index}\addcontentsline{toc}{chapter}{Index}
}
\makeatother

%\RequirePackage[german,english,francais]{babel}

\def\matlabcolormaptext{This colormap is similar to one shipped with Matlab$^\text{\textregistered}$ under a similar name.}

\IfFileExists{tikzlibraryspy.code.tex}{%
\usetikzlibrary{spy}
}{%
    \message{ERROR: tikz SPY library NOT available. The manual will only compile partially.^^J}%
}%

\usepackage{xparse}% for colorbrewer manual

\usetikzlibrary{
    decorations.markings,
    decorations.footprints,
    shapes.arrows,
    matrix,
    positioning,
}

\usepgfplotslibrary{
    fillbetween,
    ternary,
    smithchart,
    patchplots,
    polar,
    colormaps,
    colorbrewer,
    colortol,           % docu not ready yet
}
\pgfqkeys{/codeexample}{%
    every codeexample/.append style={
        /pgfplots/every ternary axis/.append style={
            /pgfplots/legend style={fill=graphicbackground},
        }
    },
    tabsize=4,
}

\pgfplotsmanualenableexternalizationofexpensive

%\usetikzlibrary{external}
%\tikzexternalize[prefix=figures/]{pgfplots}

\title{%
    Manual for Package \PGFPlots{}\\
    {\small 2D/3D Plots in \LaTeX{}, Version \pgfplotsversion}\\
    {\small\url{https://github.com/pgf-tikz/pgfplots}}
    %\\{\small Attention: you are using an unstable development version.}
}

\makeatletter
\long\def\abstractsmuggle{%
    \centering
    \textbf{Abstract}\\[0.5cm]

    \begin{minipage}{12cm}
        \PGFPlots{} draws high-quality function plots in normal or logarithmic
        scaling with a user-friendly interface directly in \TeX{}. The user
        supplies axis labels, legend entries and the plot coordinates for one or
        more plots and \PGFPlots{} applies axis scaling, computes any logarithms
        and axis ticks and draws the plots. It supports line plots, scatter
        plots, piecewise constant plots, bar plots, area plots, mesh and surface
        plots, patch plots, contour plots, quiver plots, histogram plots, box
        plots, polar axes, ternary diagrams, smith charts and some more. It is
        based on Till Tantau's package \PGF{}/\Tikz{}.
    \end{minipage}

}%

\expandafter\date\expandafter{\@date\\[2cm]
    \abstractsmuggle
}%
\makeatother

%\includeonly{pgfplots.reference}


\begin{document}

\def\plotcoords{%
\addplot coordinates {
(5,8.312e-02)    (17,2.547e-02)   (49,7.407e-03)
(129,2.102e-03)  (321,5.874e-04)  (769,1.623e-04)
(1793,4.442e-05) (4097,1.207e-05) (9217,3.261e-06)
};

\addplot coordinates{
(7,8.472e-02)    (31,3.044e-02)    (111,1.022e-02)
(351,3.303e-03)  (1023,1.039e-03)  (2815,3.196e-04)
(7423,9.658e-05) (18943,2.873e-05) (47103,8.437e-06)
};

\addplot coordinates{
(9,7.881e-02)     (49,3.243e-02)    (209,1.232e-02)
(769,4.454e-03)   (2561,1.551e-03)  (7937,5.236e-04)
(23297,1.723e-04) (65537,5.545e-05) (178177,1.751e-05)
};

\addplot coordinates{
(11,6.887e-02)    (71,3.177e-02)     (351,1.341e-02)
(1471,5.334e-03)  (5503,2.027e-03)   (18943,7.415e-04)
(61183,2.628e-04) (187903,9.063e-05) (553983,3.053e-05)
};

\addplot coordinates{
(13,5.755e-02)     (97,2.925e-02)     (545,1.351e-02)
(2561,5.842e-03)   (10625,2.397e-03)  (40193,9.414e-04)
(141569,3.564e-04) (471041,1.308e-04)
(1496065,4.670e-05)
};
}%


\bookmaketitle
\tableofcontents
\include{pgfplots.title_abstract_intro}
\include{pgfplots.preliminaries}
\include{pgfplots.intro}
\include{pgfplots.reference}
\include{pgfplots.libs}
\include{pgfplots.resources}
\include{pgfplots.importexport}
\include{pgfplots.basic.reference}

\printindex

\bibliographystyle{abbrv} %gerapali} %gerabbrv} %gerunsrt.bst} %gerabbrv}% gerplain}
\nocite{pgfplotstable}
\nocite{programmingnotes}
\bibliography{pgfplots}
\end{document}

% -----------------------------------------------------------------------------
% For Stefan Pinnow as reminder on what to look for when editing the manual
% -----------------------------------------------------------------------------
% There should be no line breaks in the following environments
% - |...|
% - \declareandlabel{...}
% - \verbpdfref{...}
% If "MakeTikzPictures" isn't running through check one of the externalized
% LOG files what is the cause of that.
% -----------------------------------------------------------------------------


%%%%%%%%%%%%%%%%%%%%%%%%%%%%%%%%%%%%%%%%%%%%%%%%%%%%%%%%%%%%%%%%%%%%%%%%%%%%%
%
% Package pgfplots.sty documentation.
%
% Copyright 2007/2008 by Christian Feuersaenger.
%
% This program is free software: you can redistribute it and/or modify
% it under the terms of the GNU General Public License as published by
% the Free Software Foundation, either version 3 of the License, or
% (at your option) any later version.
%
% This program is distributed in the hope that it will be useful,
% but WITHOUT ANY WARRANTY; without even the implied warranty of
% MERCHANTABILITY or FITNESS FOR A PARTICULAR PURPOSE.  See the
% GNU General Public License for more details.
%
% You should have received a copy of the GNU General Public License
% along with this program.  If not, see <http://www.gnu.org/licenses/>.
%
%
%%%%%%%%%%%%%%%%%%%%%%%%%%%%%%%%%%%%%%%%%%%%%%%%%%%%%%%%%%%%%%%%%%%%%%%%%%%%%
% SEE pgfplots-macros.tex as well!
%\pdfminorversion=5 % to allow compression
%\pdfobjcompresslevel=2
\documentclass[a4paper,openany]{book}

\let\bookmaketitle=\maketitle

% -----------------------------------
% this here is from ltxdoc:
\usepackage{doc}[=v2]
\AtBeginDocument{\MakeShortVerb{\|}}
%\setlength{\textwidth}{355pt}
%\addtolength\marginparwidth{30pt}
%\addtolength\oddsidemargin{20pt}
%\addtolength\evensidemargin{20pt}
\def\cmd#1{\cs{\expandafter\cmd@to@cs\string#1}}
\def\cmd@to@cs#1#2{\char\number`#2\relax}
\DeclareRobustCommand\cs[1]{\texttt{\char`\\#1}}
\providecommand\marg[1]{%
  {\ttfamily\char`\{}\meta{#1}{\ttfamily\char`\}}}
\providecommand\oarg[1]{%
  {\ttfamily[}\meta{#1}{\ttfamily]}}
\providecommand\parg[1]{%
  {\ttfamily(}\meta{#1}{\ttfamily)}}
\raggedbottom

% -----------------------------------
\input{pgfplots.preamble.tex}

\makeatletter
% I want a two-column index just as in pgfmanual styles. This here
% was the best way to get one:
\def\index@prologue{\section*{Index}\addcontentsline{toc}{chapter}{Index}
}
\makeatother

%\RequirePackage[german,english,francais]{babel}

\def\matlabcolormaptext{This colormap is similar to one shipped with Matlab$^\text{\textregistered}$ under a similar name.}

\IfFileExists{tikzlibraryspy.code.tex}{%
\usetikzlibrary{spy}
}{%
    \message{ERROR: tikz SPY library NOT available. The manual will only compile partially.^^J}%
}%

\usepackage{xparse}% for colorbrewer manual

\usetikzlibrary{
    decorations.markings,
    decorations.footprints,
    shapes.arrows,
    matrix,
    positioning,
}

\usepgfplotslibrary{
    fillbetween,
    ternary,
    smithchart,
    patchplots,
    polar,
    colormaps,
    colorbrewer,
    colortol,           % docu not ready yet
}
\pgfqkeys{/codeexample}{%
    every codeexample/.append style={
        /pgfplots/every ternary axis/.append style={
            /pgfplots/legend style={fill=graphicbackground},
        }
    },
    tabsize=4,
}

\pgfplotsmanualenableexternalizationofexpensive

%\usetikzlibrary{external}
%\tikzexternalize[prefix=figures/]{pgfplots}

\title{%
    Manual for Package \PGFPlots{}\\
    {\small 2D/3D Plots in \LaTeX{}, Version \pgfplotsversion}\\
    {\small\url{https://github.com/pgf-tikz/pgfplots}}
    %\\{\small Attention: you are using an unstable development version.}
}

\makeatletter
\long\def\abstractsmuggle{%
    \centering
    \textbf{Abstract}\\[0.5cm]

    \begin{minipage}{12cm}
        \PGFPlots{} draws high-quality function plots in normal or logarithmic
        scaling with a user-friendly interface directly in \TeX{}. The user
        supplies axis labels, legend entries and the plot coordinates for one or
        more plots and \PGFPlots{} applies axis scaling, computes any logarithms
        and axis ticks and draws the plots. It supports line plots, scatter
        plots, piecewise constant plots, bar plots, area plots, mesh and surface
        plots, patch plots, contour plots, quiver plots, histogram plots, box
        plots, polar axes, ternary diagrams, smith charts and some more. It is
        based on Till Tantau's package \PGF{}/\Tikz{}.
    \end{minipage}

}%

\expandafter\date\expandafter{\@date\\[2cm]
    \abstractsmuggle
}%
\makeatother

%\includeonly{pgfplots.reference}


\begin{document}

\def\plotcoords{%
\addplot coordinates {
(5,8.312e-02)    (17,2.547e-02)   (49,7.407e-03)
(129,2.102e-03)  (321,5.874e-04)  (769,1.623e-04)
(1793,4.442e-05) (4097,1.207e-05) (9217,3.261e-06)
};

\addplot coordinates{
(7,8.472e-02)    (31,3.044e-02)    (111,1.022e-02)
(351,3.303e-03)  (1023,1.039e-03)  (2815,3.196e-04)
(7423,9.658e-05) (18943,2.873e-05) (47103,8.437e-06)
};

\addplot coordinates{
(9,7.881e-02)     (49,3.243e-02)    (209,1.232e-02)
(769,4.454e-03)   (2561,1.551e-03)  (7937,5.236e-04)
(23297,1.723e-04) (65537,5.545e-05) (178177,1.751e-05)
};

\addplot coordinates{
(11,6.887e-02)    (71,3.177e-02)     (351,1.341e-02)
(1471,5.334e-03)  (5503,2.027e-03)   (18943,7.415e-04)
(61183,2.628e-04) (187903,9.063e-05) (553983,3.053e-05)
};

\addplot coordinates{
(13,5.755e-02)     (97,2.925e-02)     (545,1.351e-02)
(2561,5.842e-03)   (10625,2.397e-03)  (40193,9.414e-04)
(141569,3.564e-04) (471041,1.308e-04)
(1496065,4.670e-05)
};
}%


\bookmaketitle
\tableofcontents
\include{pgfplots.title_abstract_intro}
\include{pgfplots.preliminaries}
\include{pgfplots.intro}
\include{pgfplots.reference}
\include{pgfplots.libs}
\include{pgfplots.resources}
\include{pgfplots.importexport}
\include{pgfplots.basic.reference}

\printindex

\bibliographystyle{abbrv} %gerapali} %gerabbrv} %gerunsrt.bst} %gerabbrv}% gerplain}
\nocite{pgfplotstable}
\nocite{programmingnotes}
\bibliography{pgfplots}
\end{document}

% -----------------------------------------------------------------------------
% For Stefan Pinnow as reminder on what to look for when editing the manual
% -----------------------------------------------------------------------------
% There should be no line breaks in the following environments
% - |...|
% - \declareandlabel{...}
% - \verbpdfref{...}
% If "MakeTikzPictures" isn't running through check one of the externalized
% LOG files what is the cause of that.
% -----------------------------------------------------------------------------


%%%%%%%%%%%%%%%%%%%%%%%%%%%%%%%%%%%%%%%%%%%%%%%%%%%%%%%%%%%%%%%%%%%%%%%%%%%%%
%
% Package pgfplots.sty documentation.
%
% Copyright 2007/2008 by Christian Feuersaenger.
%
% This program is free software: you can redistribute it and/or modify
% it under the terms of the GNU General Public License as published by
% the Free Software Foundation, either version 3 of the License, or
% (at your option) any later version.
%
% This program is distributed in the hope that it will be useful,
% but WITHOUT ANY WARRANTY; without even the implied warranty of
% MERCHANTABILITY or FITNESS FOR A PARTICULAR PURPOSE.  See the
% GNU General Public License for more details.
%
% You should have received a copy of the GNU General Public License
% along with this program.  If not, see <http://www.gnu.org/licenses/>.
%
%
%%%%%%%%%%%%%%%%%%%%%%%%%%%%%%%%%%%%%%%%%%%%%%%%%%%%%%%%%%%%%%%%%%%%%%%%%%%%%
% SEE pgfplots-macros.tex as well!
%\pdfminorversion=5 % to allow compression
%\pdfobjcompresslevel=2
\documentclass[a4paper,openany]{book}

\let\bookmaketitle=\maketitle

% -----------------------------------
% this here is from ltxdoc:
\usepackage{doc}[=v2]
\AtBeginDocument{\MakeShortVerb{\|}}
%\setlength{\textwidth}{355pt}
%\addtolength\marginparwidth{30pt}
%\addtolength\oddsidemargin{20pt}
%\addtolength\evensidemargin{20pt}
\def\cmd#1{\cs{\expandafter\cmd@to@cs\string#1}}
\def\cmd@to@cs#1#2{\char\number`#2\relax}
\DeclareRobustCommand\cs[1]{\texttt{\char`\\#1}}
\providecommand\marg[1]{%
  {\ttfamily\char`\{}\meta{#1}{\ttfamily\char`\}}}
\providecommand\oarg[1]{%
  {\ttfamily[}\meta{#1}{\ttfamily]}}
\providecommand\parg[1]{%
  {\ttfamily(}\meta{#1}{\ttfamily)}}
\raggedbottom

% -----------------------------------
\input{pgfplots.preamble.tex}

\makeatletter
% I want a two-column index just as in pgfmanual styles. This here
% was the best way to get one:
\def\index@prologue{\section*{Index}\addcontentsline{toc}{chapter}{Index}
}
\makeatother

%\RequirePackage[german,english,francais]{babel}

\def\matlabcolormaptext{This colormap is similar to one shipped with Matlab$^\text{\textregistered}$ under a similar name.}

\IfFileExists{tikzlibraryspy.code.tex}{%
\usetikzlibrary{spy}
}{%
    \message{ERROR: tikz SPY library NOT available. The manual will only compile partially.^^J}%
}%

\usepackage{xparse}% for colorbrewer manual

\usetikzlibrary{
    decorations.markings,
    decorations.footprints,
    shapes.arrows,
    matrix,
    positioning,
}

\usepgfplotslibrary{
    fillbetween,
    ternary,
    smithchart,
    patchplots,
    polar,
    colormaps,
    colorbrewer,
    colortol,           % docu not ready yet
}
\pgfqkeys{/codeexample}{%
    every codeexample/.append style={
        /pgfplots/every ternary axis/.append style={
            /pgfplots/legend style={fill=graphicbackground},
        }
    },
    tabsize=4,
}

\pgfplotsmanualenableexternalizationofexpensive

%\usetikzlibrary{external}
%\tikzexternalize[prefix=figures/]{pgfplots}

\title{%
    Manual for Package \PGFPlots{}\\
    {\small 2D/3D Plots in \LaTeX{}, Version \pgfplotsversion}\\
    {\small\url{https://github.com/pgf-tikz/pgfplots}}
    %\\{\small Attention: you are using an unstable development version.}
}

\makeatletter
\long\def\abstractsmuggle{%
    \centering
    \textbf{Abstract}\\[0.5cm]

    \begin{minipage}{12cm}
        \PGFPlots{} draws high-quality function plots in normal or logarithmic
        scaling with a user-friendly interface directly in \TeX{}. The user
        supplies axis labels, legend entries and the plot coordinates for one or
        more plots and \PGFPlots{} applies axis scaling, computes any logarithms
        and axis ticks and draws the plots. It supports line plots, scatter
        plots, piecewise constant plots, bar plots, area plots, mesh and surface
        plots, patch plots, contour plots, quiver plots, histogram plots, box
        plots, polar axes, ternary diagrams, smith charts and some more. It is
        based on Till Tantau's package \PGF{}/\Tikz{}.
    \end{minipage}

}%

\expandafter\date\expandafter{\@date\\[2cm]
    \abstractsmuggle
}%
\makeatother

%\includeonly{pgfplots.reference}


\begin{document}

\def\plotcoords{%
\addplot coordinates {
(5,8.312e-02)    (17,2.547e-02)   (49,7.407e-03)
(129,2.102e-03)  (321,5.874e-04)  (769,1.623e-04)
(1793,4.442e-05) (4097,1.207e-05) (9217,3.261e-06)
};

\addplot coordinates{
(7,8.472e-02)    (31,3.044e-02)    (111,1.022e-02)
(351,3.303e-03)  (1023,1.039e-03)  (2815,3.196e-04)
(7423,9.658e-05) (18943,2.873e-05) (47103,8.437e-06)
};

\addplot coordinates{
(9,7.881e-02)     (49,3.243e-02)    (209,1.232e-02)
(769,4.454e-03)   (2561,1.551e-03)  (7937,5.236e-04)
(23297,1.723e-04) (65537,5.545e-05) (178177,1.751e-05)
};

\addplot coordinates{
(11,6.887e-02)    (71,3.177e-02)     (351,1.341e-02)
(1471,5.334e-03)  (5503,2.027e-03)   (18943,7.415e-04)
(61183,2.628e-04) (187903,9.063e-05) (553983,3.053e-05)
};

\addplot coordinates{
(13,5.755e-02)     (97,2.925e-02)     (545,1.351e-02)
(2561,5.842e-03)   (10625,2.397e-03)  (40193,9.414e-04)
(141569,3.564e-04) (471041,1.308e-04)
(1496065,4.670e-05)
};
}%


\bookmaketitle
\tableofcontents
\include{pgfplots.title_abstract_intro}
\include{pgfplots.preliminaries}
\include{pgfplots.intro}
\include{pgfplots.reference}
\include{pgfplots.libs}
\include{pgfplots.resources}
\include{pgfplots.importexport}
\include{pgfplots.basic.reference}

\printindex

\bibliographystyle{abbrv} %gerapali} %gerabbrv} %gerunsrt.bst} %gerabbrv}% gerplain}
\nocite{pgfplotstable}
\nocite{programmingnotes}
\bibliography{pgfplots}
\end{document}

% -----------------------------------------------------------------------------
% For Stefan Pinnow as reminder on what to look for when editing the manual
% -----------------------------------------------------------------------------
% There should be no line breaks in the following environments
% - |...|
% - \declareandlabel{...}
% - \verbpdfref{...}
% If "MakeTikzPictures" isn't running through check one of the externalized
% LOG files what is the cause of that.
% -----------------------------------------------------------------------------


%%%%%%%%%%%%%%%%%%%%%%%%%%%%%%%%%%%%%%%%%%%%%%%%%%%%%%%%%%%%%%%%%%%%%%%%%%%%%
%
% Package pgfplots.sty documentation.
%
% Copyright 2007/2008 by Christian Feuersaenger.
%
% This program is free software: you can redistribute it and/or modify
% it under the terms of the GNU General Public License as published by
% the Free Software Foundation, either version 3 of the License, or
% (at your option) any later version.
%
% This program is distributed in the hope that it will be useful,
% but WITHOUT ANY WARRANTY; without even the implied warranty of
% MERCHANTABILITY or FITNESS FOR A PARTICULAR PURPOSE.  See the
% GNU General Public License for more details.
%
% You should have received a copy of the GNU General Public License
% along with this program.  If not, see <http://www.gnu.org/licenses/>.
%
%
%%%%%%%%%%%%%%%%%%%%%%%%%%%%%%%%%%%%%%%%%%%%%%%%%%%%%%%%%%%%%%%%%%%%%%%%%%%%%
% SEE pgfplots-macros.tex as well!
%\pdfminorversion=5 % to allow compression
%\pdfobjcompresslevel=2
\documentclass[a4paper,openany]{book}

\let\bookmaketitle=\maketitle

% -----------------------------------
% this here is from ltxdoc:
\usepackage{doc}[=v2]
\AtBeginDocument{\MakeShortVerb{\|}}
%\setlength{\textwidth}{355pt}
%\addtolength\marginparwidth{30pt}
%\addtolength\oddsidemargin{20pt}
%\addtolength\evensidemargin{20pt}
\def\cmd#1{\cs{\expandafter\cmd@to@cs\string#1}}
\def\cmd@to@cs#1#2{\char\number`#2\relax}
\DeclareRobustCommand\cs[1]{\texttt{\char`\\#1}}
\providecommand\marg[1]{%
  {\ttfamily\char`\{}\meta{#1}{\ttfamily\char`\}}}
\providecommand\oarg[1]{%
  {\ttfamily[}\meta{#1}{\ttfamily]}}
\providecommand\parg[1]{%
  {\ttfamily(}\meta{#1}{\ttfamily)}}
\raggedbottom

% -----------------------------------
\input{pgfplots.preamble.tex}

\makeatletter
% I want a two-column index just as in pgfmanual styles. This here
% was the best way to get one:
\def\index@prologue{\section*{Index}\addcontentsline{toc}{chapter}{Index}
}
\makeatother

%\RequirePackage[german,english,francais]{babel}

\def\matlabcolormaptext{This colormap is similar to one shipped with Matlab$^\text{\textregistered}$ under a similar name.}

\IfFileExists{tikzlibraryspy.code.tex}{%
\usetikzlibrary{spy}
}{%
    \message{ERROR: tikz SPY library NOT available. The manual will only compile partially.^^J}%
}%

\usepackage{xparse}% for colorbrewer manual

\usetikzlibrary{
    decorations.markings,
    decorations.footprints,
    shapes.arrows,
    matrix,
    positioning,
}

\usepgfplotslibrary{
    fillbetween,
    ternary,
    smithchart,
    patchplots,
    polar,
    colormaps,
    colorbrewer,
    colortol,           % docu not ready yet
}
\pgfqkeys{/codeexample}{%
    every codeexample/.append style={
        /pgfplots/every ternary axis/.append style={
            /pgfplots/legend style={fill=graphicbackground},
        }
    },
    tabsize=4,
}

\pgfplotsmanualenableexternalizationofexpensive

%\usetikzlibrary{external}
%\tikzexternalize[prefix=figures/]{pgfplots}

\title{%
    Manual for Package \PGFPlots{}\\
    {\small 2D/3D Plots in \LaTeX{}, Version \pgfplotsversion}\\
    {\small\url{https://github.com/pgf-tikz/pgfplots}}
    %\\{\small Attention: you are using an unstable development version.}
}

\makeatletter
\long\def\abstractsmuggle{%
    \centering
    \textbf{Abstract}\\[0.5cm]

    \begin{minipage}{12cm}
        \PGFPlots{} draws high-quality function plots in normal or logarithmic
        scaling with a user-friendly interface directly in \TeX{}. The user
        supplies axis labels, legend entries and the plot coordinates for one or
        more plots and \PGFPlots{} applies axis scaling, computes any logarithms
        and axis ticks and draws the plots. It supports line plots, scatter
        plots, piecewise constant plots, bar plots, area plots, mesh and surface
        plots, patch plots, contour plots, quiver plots, histogram plots, box
        plots, polar axes, ternary diagrams, smith charts and some more. It is
        based on Till Tantau's package \PGF{}/\Tikz{}.
    \end{minipage}

}%

\expandafter\date\expandafter{\@date\\[2cm]
    \abstractsmuggle
}%
\makeatother

%\includeonly{pgfplots.reference}


\begin{document}

\def\plotcoords{%
\addplot coordinates {
(5,8.312e-02)    (17,2.547e-02)   (49,7.407e-03)
(129,2.102e-03)  (321,5.874e-04)  (769,1.623e-04)
(1793,4.442e-05) (4097,1.207e-05) (9217,3.261e-06)
};

\addplot coordinates{
(7,8.472e-02)    (31,3.044e-02)    (111,1.022e-02)
(351,3.303e-03)  (1023,1.039e-03)  (2815,3.196e-04)
(7423,9.658e-05) (18943,2.873e-05) (47103,8.437e-06)
};

\addplot coordinates{
(9,7.881e-02)     (49,3.243e-02)    (209,1.232e-02)
(769,4.454e-03)   (2561,1.551e-03)  (7937,5.236e-04)
(23297,1.723e-04) (65537,5.545e-05) (178177,1.751e-05)
};

\addplot coordinates{
(11,6.887e-02)    (71,3.177e-02)     (351,1.341e-02)
(1471,5.334e-03)  (5503,2.027e-03)   (18943,7.415e-04)
(61183,2.628e-04) (187903,9.063e-05) (553983,3.053e-05)
};

\addplot coordinates{
(13,5.755e-02)     (97,2.925e-02)     (545,1.351e-02)
(2561,5.842e-03)   (10625,2.397e-03)  (40193,9.414e-04)
(141569,3.564e-04) (471041,1.308e-04)
(1496065,4.670e-05)
};
}%


\bookmaketitle
\tableofcontents
\include{pgfplots.title_abstract_intro}
\include{pgfplots.preliminaries}
\include{pgfplots.intro}
\include{pgfplots.reference}
\include{pgfplots.libs}
\include{pgfplots.resources}
\include{pgfplots.importexport}
\include{pgfplots.basic.reference}

\printindex

\bibliographystyle{abbrv} %gerapali} %gerabbrv} %gerunsrt.bst} %gerabbrv}% gerplain}
\nocite{pgfplotstable}
\nocite{programmingnotes}
\bibliography{pgfplots}
\end{document}

% -----------------------------------------------------------------------------
% For Stefan Pinnow as reminder on what to look for when editing the manual
% -----------------------------------------------------------------------------
% There should be no line breaks in the following environments
% - |...|
% - \declareandlabel{...}
% - \verbpdfref{...}
% If "MakeTikzPictures" isn't running through check one of the externalized
% LOG files what is the cause of that.
% -----------------------------------------------------------------------------


%%%%%%%%%%%%%%%%%%%%%%%%%%%%%%%%%%%%%%%%%%%%%%%%%%%%%%%%%%%%%%%%%%%%%%%%%%%%%
%
% Package pgfplots.sty documentation.
%
% Copyright 2007/2008 by Christian Feuersaenger.
%
% This program is free software: you can redistribute it and/or modify
% it under the terms of the GNU General Public License as published by
% the Free Software Foundation, either version 3 of the License, or
% (at your option) any later version.
%
% This program is distributed in the hope that it will be useful,
% but WITHOUT ANY WARRANTY; without even the implied warranty of
% MERCHANTABILITY or FITNESS FOR A PARTICULAR PURPOSE.  See the
% GNU General Public License for more details.
%
% You should have received a copy of the GNU General Public License
% along with this program.  If not, see <http://www.gnu.org/licenses/>.
%
%
%%%%%%%%%%%%%%%%%%%%%%%%%%%%%%%%%%%%%%%%%%%%%%%%%%%%%%%%%%%%%%%%%%%%%%%%%%%%%
% SEE pgfplots-macros.tex as well!
%\pdfminorversion=5 % to allow compression
%\pdfobjcompresslevel=2
\documentclass[a4paper,openany]{book}

\let\bookmaketitle=\maketitle

% -----------------------------------
% this here is from ltxdoc:
\usepackage{doc}[=v2]
\AtBeginDocument{\MakeShortVerb{\|}}
%\setlength{\textwidth}{355pt}
%\addtolength\marginparwidth{30pt}
%\addtolength\oddsidemargin{20pt}
%\addtolength\evensidemargin{20pt}
\def\cmd#1{\cs{\expandafter\cmd@to@cs\string#1}}
\def\cmd@to@cs#1#2{\char\number`#2\relax}
\DeclareRobustCommand\cs[1]{\texttt{\char`\\#1}}
\providecommand\marg[1]{%
  {\ttfamily\char`\{}\meta{#1}{\ttfamily\char`\}}}
\providecommand\oarg[1]{%
  {\ttfamily[}\meta{#1}{\ttfamily]}}
\providecommand\parg[1]{%
  {\ttfamily(}\meta{#1}{\ttfamily)}}
\raggedbottom

% -----------------------------------
\input{pgfplots.preamble.tex}

\makeatletter
% I want a two-column index just as in pgfmanual styles. This here
% was the best way to get one:
\def\index@prologue{\section*{Index}\addcontentsline{toc}{chapter}{Index}
}
\makeatother

%\RequirePackage[german,english,francais]{babel}

\def\matlabcolormaptext{This colormap is similar to one shipped with Matlab$^\text{\textregistered}$ under a similar name.}

\IfFileExists{tikzlibraryspy.code.tex}{%
\usetikzlibrary{spy}
}{%
    \message{ERROR: tikz SPY library NOT available. The manual will only compile partially.^^J}%
}%

\usepackage{xparse}% for colorbrewer manual

\usetikzlibrary{
    decorations.markings,
    decorations.footprints,
    shapes.arrows,
    matrix,
    positioning,
}

\usepgfplotslibrary{
    fillbetween,
    ternary,
    smithchart,
    patchplots,
    polar,
    colormaps,
    colorbrewer,
    colortol,           % docu not ready yet
}
\pgfqkeys{/codeexample}{%
    every codeexample/.append style={
        /pgfplots/every ternary axis/.append style={
            /pgfplots/legend style={fill=graphicbackground},
        }
    },
    tabsize=4,
}

\pgfplotsmanualenableexternalizationofexpensive

%\usetikzlibrary{external}
%\tikzexternalize[prefix=figures/]{pgfplots}

\title{%
    Manual for Package \PGFPlots{}\\
    {\small 2D/3D Plots in \LaTeX{}, Version \pgfplotsversion}\\
    {\small\url{https://github.com/pgf-tikz/pgfplots}}
    %\\{\small Attention: you are using an unstable development version.}
}

\makeatletter
\long\def\abstractsmuggle{%
    \centering
    \textbf{Abstract}\\[0.5cm]

    \begin{minipage}{12cm}
        \PGFPlots{} draws high-quality function plots in normal or logarithmic
        scaling with a user-friendly interface directly in \TeX{}. The user
        supplies axis labels, legend entries and the plot coordinates for one or
        more plots and \PGFPlots{} applies axis scaling, computes any logarithms
        and axis ticks and draws the plots. It supports line plots, scatter
        plots, piecewise constant plots, bar plots, area plots, mesh and surface
        plots, patch plots, contour plots, quiver plots, histogram plots, box
        plots, polar axes, ternary diagrams, smith charts and some more. It is
        based on Till Tantau's package \PGF{}/\Tikz{}.
    \end{minipage}

}%

\expandafter\date\expandafter{\@date\\[2cm]
    \abstractsmuggle
}%
\makeatother

%\includeonly{pgfplots.reference}


\begin{document}

\def\plotcoords{%
\addplot coordinates {
(5,8.312e-02)    (17,2.547e-02)   (49,7.407e-03)
(129,2.102e-03)  (321,5.874e-04)  (769,1.623e-04)
(1793,4.442e-05) (4097,1.207e-05) (9217,3.261e-06)
};

\addplot coordinates{
(7,8.472e-02)    (31,3.044e-02)    (111,1.022e-02)
(351,3.303e-03)  (1023,1.039e-03)  (2815,3.196e-04)
(7423,9.658e-05) (18943,2.873e-05) (47103,8.437e-06)
};

\addplot coordinates{
(9,7.881e-02)     (49,3.243e-02)    (209,1.232e-02)
(769,4.454e-03)   (2561,1.551e-03)  (7937,5.236e-04)
(23297,1.723e-04) (65537,5.545e-05) (178177,1.751e-05)
};

\addplot coordinates{
(11,6.887e-02)    (71,3.177e-02)     (351,1.341e-02)
(1471,5.334e-03)  (5503,2.027e-03)   (18943,7.415e-04)
(61183,2.628e-04) (187903,9.063e-05) (553983,3.053e-05)
};

\addplot coordinates{
(13,5.755e-02)     (97,2.925e-02)     (545,1.351e-02)
(2561,5.842e-03)   (10625,2.397e-03)  (40193,9.414e-04)
(141569,3.564e-04) (471041,1.308e-04)
(1496065,4.670e-05)
};
}%


\bookmaketitle
\tableofcontents
\include{pgfplots.title_abstract_intro}
\include{pgfplots.preliminaries}
\include{pgfplots.intro}
\include{pgfplots.reference}
\include{pgfplots.libs}
\include{pgfplots.resources}
\include{pgfplots.importexport}
\include{pgfplots.basic.reference}

\printindex

\bibliographystyle{abbrv} %gerapali} %gerabbrv} %gerunsrt.bst} %gerabbrv}% gerplain}
\nocite{pgfplotstable}
\nocite{programmingnotes}
\bibliography{pgfplots}
\end{document}

% -----------------------------------------------------------------------------
% For Stefan Pinnow as reminder on what to look for when editing the manual
% -----------------------------------------------------------------------------
% There should be no line breaks in the following environments
% - |...|
% - \declareandlabel{...}
% - \verbpdfref{...}
% If "MakeTikzPictures" isn't running through check one of the externalized
% LOG files what is the cause of that.
% -----------------------------------------------------------------------------

\include{pgfplots.basic.reference}

\printindex

\bibliographystyle{abbrv} %gerapali} %gerabbrv} %gerunsrt.bst} %gerabbrv}% gerplain}
\nocite{pgfplotstable}
\nocite{programmingnotes}
\bibliography{pgfplots}
\end{document}

% -----------------------------------------------------------------------------
% For Stefan Pinnow as reminder on what to look for when editing the manual
% -----------------------------------------------------------------------------
% There should be no line breaks in the following environments
% - |...|
% - \declareandlabel{...}
% - \verbpdfref{...}
% If "MakeTikzPictures" isn't running through check one of the externalized
% LOG files what is the cause of that.
% -----------------------------------------------------------------------------


%%%%%%%%%%%%%%%%%%%%%%%%%%%%%%%%%%%%%%%%%%%%%%%%%%%%%%%%%%%%%%%%%%%%%%%%%%%%%
%
% Package pgfplots.sty documentation.
%
% Copyright 2007/2008 by Christian Feuersaenger.
%
% This program is free software: you can redistribute it and/or modify
% it under the terms of the GNU General Public License as published by
% the Free Software Foundation, either version 3 of the License, or
% (at your option) any later version.
%
% This program is distributed in the hope that it will be useful,
% but WITHOUT ANY WARRANTY; without even the implied warranty of
% MERCHANTABILITY or FITNESS FOR A PARTICULAR PURPOSE.  See the
% GNU General Public License for more details.
%
% You should have received a copy of the GNU General Public License
% along with this program.  If not, see <http://www.gnu.org/licenses/>.
%
%
%%%%%%%%%%%%%%%%%%%%%%%%%%%%%%%%%%%%%%%%%%%%%%%%%%%%%%%%%%%%%%%%%%%%%%%%%%%%%
% SEE pgfplots-macros.tex as well!
%\pdfminorversion=5 % to allow compression
%\pdfobjcompresslevel=2
\documentclass[a4paper,openany]{book}

\let\bookmaketitle=\maketitle

% -----------------------------------
% this here is from ltxdoc:
\usepackage{doc}[=v2]
\AtBeginDocument{\MakeShortVerb{\|}}
%\setlength{\textwidth}{355pt}
%\addtolength\marginparwidth{30pt}
%\addtolength\oddsidemargin{20pt}
%\addtolength\evensidemargin{20pt}
\def\cmd#1{\cs{\expandafter\cmd@to@cs\string#1}}
\def\cmd@to@cs#1#2{\char\number`#2\relax}
\DeclareRobustCommand\cs[1]{\texttt{\char`\\#1}}
\providecommand\marg[1]{%
  {\ttfamily\char`\{}\meta{#1}{\ttfamily\char`\}}}
\providecommand\oarg[1]{%
  {\ttfamily[}\meta{#1}{\ttfamily]}}
\providecommand\parg[1]{%
  {\ttfamily(}\meta{#1}{\ttfamily)}}
\raggedbottom

% -----------------------------------
\input{pgfplots.preamble.tex}

\makeatletter
% I want a two-column index just as in pgfmanual styles. This here
% was the best way to get one:
\def\index@prologue{\section*{Index}\addcontentsline{toc}{chapter}{Index}
}
\makeatother

%\RequirePackage[german,english,francais]{babel}

\def\matlabcolormaptext{This colormap is similar to one shipped with Matlab$^\text{\textregistered}$ under a similar name.}

\IfFileExists{tikzlibraryspy.code.tex}{%
\usetikzlibrary{spy}
}{%
    \message{ERROR: tikz SPY library NOT available. The manual will only compile partially.^^J}%
}%

\usepackage{xparse}% for colorbrewer manual

\usetikzlibrary{
    decorations.markings,
    decorations.footprints,
    shapes.arrows,
    matrix,
    positioning,
}

\usepgfplotslibrary{
    fillbetween,
    ternary,
    smithchart,
    patchplots,
    polar,
    colormaps,
    colorbrewer,
    colortol,           % docu not ready yet
}
\pgfqkeys{/codeexample}{%
    every codeexample/.append style={
        /pgfplots/every ternary axis/.append style={
            /pgfplots/legend style={fill=graphicbackground},
        }
    },
    tabsize=4,
}

\pgfplotsmanualenableexternalizationofexpensive

%\usetikzlibrary{external}
%\tikzexternalize[prefix=figures/]{pgfplots}

\title{%
    Manual for Package \PGFPlots{}\\
    {\small 2D/3D Plots in \LaTeX{}, Version \pgfplotsversion}\\
    {\small\url{https://github.com/pgf-tikz/pgfplots}}
    %\\{\small Attention: you are using an unstable development version.}
}

\makeatletter
\long\def\abstractsmuggle{%
    \centering
    \textbf{Abstract}\\[0.5cm]

    \begin{minipage}{12cm}
        \PGFPlots{} draws high-quality function plots in normal or logarithmic
        scaling with a user-friendly interface directly in \TeX{}. The user
        supplies axis labels, legend entries and the plot coordinates for one or
        more plots and \PGFPlots{} applies axis scaling, computes any logarithms
        and axis ticks and draws the plots. It supports line plots, scatter
        plots, piecewise constant plots, bar plots, area plots, mesh and surface
        plots, patch plots, contour plots, quiver plots, histogram plots, box
        plots, polar axes, ternary diagrams, smith charts and some more. It is
        based on Till Tantau's package \PGF{}/\Tikz{}.
    \end{minipage}

}%

\expandafter\date\expandafter{\@date\\[2cm]
    \abstractsmuggle
}%
\makeatother

%\includeonly{pgfplots.reference}


\begin{document}

\def\plotcoords{%
\addplot coordinates {
(5,8.312e-02)    (17,2.547e-02)   (49,7.407e-03)
(129,2.102e-03)  (321,5.874e-04)  (769,1.623e-04)
(1793,4.442e-05) (4097,1.207e-05) (9217,3.261e-06)
};

\addplot coordinates{
(7,8.472e-02)    (31,3.044e-02)    (111,1.022e-02)
(351,3.303e-03)  (1023,1.039e-03)  (2815,3.196e-04)
(7423,9.658e-05) (18943,2.873e-05) (47103,8.437e-06)
};

\addplot coordinates{
(9,7.881e-02)     (49,3.243e-02)    (209,1.232e-02)
(769,4.454e-03)   (2561,1.551e-03)  (7937,5.236e-04)
(23297,1.723e-04) (65537,5.545e-05) (178177,1.751e-05)
};

\addplot coordinates{
(11,6.887e-02)    (71,3.177e-02)     (351,1.341e-02)
(1471,5.334e-03)  (5503,2.027e-03)   (18943,7.415e-04)
(61183,2.628e-04) (187903,9.063e-05) (553983,3.053e-05)
};

\addplot coordinates{
(13,5.755e-02)     (97,2.925e-02)     (545,1.351e-02)
(2561,5.842e-03)   (10625,2.397e-03)  (40193,9.414e-04)
(141569,3.564e-04) (471041,1.308e-04)
(1496065,4.670e-05)
};
}%


\bookmaketitle
\tableofcontents

%%%%%%%%%%%%%%%%%%%%%%%%%%%%%%%%%%%%%%%%%%%%%%%%%%%%%%%%%%%%%%%%%%%%%%%%%%%%%
%
% Package pgfplots.sty documentation.
%
% Copyright 2007/2008 by Christian Feuersaenger.
%
% This program is free software: you can redistribute it and/or modify
% it under the terms of the GNU General Public License as published by
% the Free Software Foundation, either version 3 of the License, or
% (at your option) any later version.
%
% This program is distributed in the hope that it will be useful,
% but WITHOUT ANY WARRANTY; without even the implied warranty of
% MERCHANTABILITY or FITNESS FOR A PARTICULAR PURPOSE.  See the
% GNU General Public License for more details.
%
% You should have received a copy of the GNU General Public License
% along with this program.  If not, see <http://www.gnu.org/licenses/>.
%
%
%%%%%%%%%%%%%%%%%%%%%%%%%%%%%%%%%%%%%%%%%%%%%%%%%%%%%%%%%%%%%%%%%%%%%%%%%%%%%
% SEE pgfplots-macros.tex as well!
%\pdfminorversion=5 % to allow compression
%\pdfobjcompresslevel=2
\documentclass[a4paper,openany]{book}

\let\bookmaketitle=\maketitle

% -----------------------------------
% this here is from ltxdoc:
\usepackage{doc}[=v2]
\AtBeginDocument{\MakeShortVerb{\|}}
%\setlength{\textwidth}{355pt}
%\addtolength\marginparwidth{30pt}
%\addtolength\oddsidemargin{20pt}
%\addtolength\evensidemargin{20pt}
\def\cmd#1{\cs{\expandafter\cmd@to@cs\string#1}}
\def\cmd@to@cs#1#2{\char\number`#2\relax}
\DeclareRobustCommand\cs[1]{\texttt{\char`\\#1}}
\providecommand\marg[1]{%
  {\ttfamily\char`\{}\meta{#1}{\ttfamily\char`\}}}
\providecommand\oarg[1]{%
  {\ttfamily[}\meta{#1}{\ttfamily]}}
\providecommand\parg[1]{%
  {\ttfamily(}\meta{#1}{\ttfamily)}}
\raggedbottom

% -----------------------------------
\input{pgfplots.preamble.tex}

\makeatletter
% I want a two-column index just as in pgfmanual styles. This here
% was the best way to get one:
\def\index@prologue{\section*{Index}\addcontentsline{toc}{chapter}{Index}
}
\makeatother

%\RequirePackage[german,english,francais]{babel}

\def\matlabcolormaptext{This colormap is similar to one shipped with Matlab$^\text{\textregistered}$ under a similar name.}

\IfFileExists{tikzlibraryspy.code.tex}{%
\usetikzlibrary{spy}
}{%
    \message{ERROR: tikz SPY library NOT available. The manual will only compile partially.^^J}%
}%

\usepackage{xparse}% for colorbrewer manual

\usetikzlibrary{
    decorations.markings,
    decorations.footprints,
    shapes.arrows,
    matrix,
    positioning,
}

\usepgfplotslibrary{
    fillbetween,
    ternary,
    smithchart,
    patchplots,
    polar,
    colormaps,
    colorbrewer,
    colortol,           % docu not ready yet
}
\pgfqkeys{/codeexample}{%
    every codeexample/.append style={
        /pgfplots/every ternary axis/.append style={
            /pgfplots/legend style={fill=graphicbackground},
        }
    },
    tabsize=4,
}

\pgfplotsmanualenableexternalizationofexpensive

%\usetikzlibrary{external}
%\tikzexternalize[prefix=figures/]{pgfplots}

\title{%
    Manual for Package \PGFPlots{}\\
    {\small 2D/3D Plots in \LaTeX{}, Version \pgfplotsversion}\\
    {\small\url{https://github.com/pgf-tikz/pgfplots}}
    %\\{\small Attention: you are using an unstable development version.}
}

\makeatletter
\long\def\abstractsmuggle{%
    \centering
    \textbf{Abstract}\\[0.5cm]

    \begin{minipage}{12cm}
        \PGFPlots{} draws high-quality function plots in normal or logarithmic
        scaling with a user-friendly interface directly in \TeX{}. The user
        supplies axis labels, legend entries and the plot coordinates for one or
        more plots and \PGFPlots{} applies axis scaling, computes any logarithms
        and axis ticks and draws the plots. It supports line plots, scatter
        plots, piecewise constant plots, bar plots, area plots, mesh and surface
        plots, patch plots, contour plots, quiver plots, histogram plots, box
        plots, polar axes, ternary diagrams, smith charts and some more. It is
        based on Till Tantau's package \PGF{}/\Tikz{}.
    \end{minipage}

}%

\expandafter\date\expandafter{\@date\\[2cm]
    \abstractsmuggle
}%
\makeatother

%\includeonly{pgfplots.reference}


\begin{document}

\def\plotcoords{%
\addplot coordinates {
(5,8.312e-02)    (17,2.547e-02)   (49,7.407e-03)
(129,2.102e-03)  (321,5.874e-04)  (769,1.623e-04)
(1793,4.442e-05) (4097,1.207e-05) (9217,3.261e-06)
};

\addplot coordinates{
(7,8.472e-02)    (31,3.044e-02)    (111,1.022e-02)
(351,3.303e-03)  (1023,1.039e-03)  (2815,3.196e-04)
(7423,9.658e-05) (18943,2.873e-05) (47103,8.437e-06)
};

\addplot coordinates{
(9,7.881e-02)     (49,3.243e-02)    (209,1.232e-02)
(769,4.454e-03)   (2561,1.551e-03)  (7937,5.236e-04)
(23297,1.723e-04) (65537,5.545e-05) (178177,1.751e-05)
};

\addplot coordinates{
(11,6.887e-02)    (71,3.177e-02)     (351,1.341e-02)
(1471,5.334e-03)  (5503,2.027e-03)   (18943,7.415e-04)
(61183,2.628e-04) (187903,9.063e-05) (553983,3.053e-05)
};

\addplot coordinates{
(13,5.755e-02)     (97,2.925e-02)     (545,1.351e-02)
(2561,5.842e-03)   (10625,2.397e-03)  (40193,9.414e-04)
(141569,3.564e-04) (471041,1.308e-04)
(1496065,4.670e-05)
};
}%


\bookmaketitle
\tableofcontents
\include{pgfplots.title_abstract_intro}
\include{pgfplots.preliminaries}
\include{pgfplots.intro}
\include{pgfplots.reference}
\include{pgfplots.libs}
\include{pgfplots.resources}
\include{pgfplots.importexport}
\include{pgfplots.basic.reference}

\printindex

\bibliographystyle{abbrv} %gerapali} %gerabbrv} %gerunsrt.bst} %gerabbrv}% gerplain}
\nocite{pgfplotstable}
\nocite{programmingnotes}
\bibliography{pgfplots}
\end{document}

% -----------------------------------------------------------------------------
% For Stefan Pinnow as reminder on what to look for when editing the manual
% -----------------------------------------------------------------------------
% There should be no line breaks in the following environments
% - |...|
% - \declareandlabel{...}
% - \verbpdfref{...}
% If "MakeTikzPictures" isn't running through check one of the externalized
% LOG files what is the cause of that.
% -----------------------------------------------------------------------------


%%%%%%%%%%%%%%%%%%%%%%%%%%%%%%%%%%%%%%%%%%%%%%%%%%%%%%%%%%%%%%%%%%%%%%%%%%%%%
%
% Package pgfplots.sty documentation.
%
% Copyright 2007/2008 by Christian Feuersaenger.
%
% This program is free software: you can redistribute it and/or modify
% it under the terms of the GNU General Public License as published by
% the Free Software Foundation, either version 3 of the License, or
% (at your option) any later version.
%
% This program is distributed in the hope that it will be useful,
% but WITHOUT ANY WARRANTY; without even the implied warranty of
% MERCHANTABILITY or FITNESS FOR A PARTICULAR PURPOSE.  See the
% GNU General Public License for more details.
%
% You should have received a copy of the GNU General Public License
% along with this program.  If not, see <http://www.gnu.org/licenses/>.
%
%
%%%%%%%%%%%%%%%%%%%%%%%%%%%%%%%%%%%%%%%%%%%%%%%%%%%%%%%%%%%%%%%%%%%%%%%%%%%%%
% SEE pgfplots-macros.tex as well!
%\pdfminorversion=5 % to allow compression
%\pdfobjcompresslevel=2
\documentclass[a4paper,openany]{book}

\let\bookmaketitle=\maketitle

% -----------------------------------
% this here is from ltxdoc:
\usepackage{doc}[=v2]
\AtBeginDocument{\MakeShortVerb{\|}}
%\setlength{\textwidth}{355pt}
%\addtolength\marginparwidth{30pt}
%\addtolength\oddsidemargin{20pt}
%\addtolength\evensidemargin{20pt}
\def\cmd#1{\cs{\expandafter\cmd@to@cs\string#1}}
\def\cmd@to@cs#1#2{\char\number`#2\relax}
\DeclareRobustCommand\cs[1]{\texttt{\char`\\#1}}
\providecommand\marg[1]{%
  {\ttfamily\char`\{}\meta{#1}{\ttfamily\char`\}}}
\providecommand\oarg[1]{%
  {\ttfamily[}\meta{#1}{\ttfamily]}}
\providecommand\parg[1]{%
  {\ttfamily(}\meta{#1}{\ttfamily)}}
\raggedbottom

% -----------------------------------
\input{pgfplots.preamble.tex}

\makeatletter
% I want a two-column index just as in pgfmanual styles. This here
% was the best way to get one:
\def\index@prologue{\section*{Index}\addcontentsline{toc}{chapter}{Index}
}
\makeatother

%\RequirePackage[german,english,francais]{babel}

\def\matlabcolormaptext{This colormap is similar to one shipped with Matlab$^\text{\textregistered}$ under a similar name.}

\IfFileExists{tikzlibraryspy.code.tex}{%
\usetikzlibrary{spy}
}{%
    \message{ERROR: tikz SPY library NOT available. The manual will only compile partially.^^J}%
}%

\usepackage{xparse}% for colorbrewer manual

\usetikzlibrary{
    decorations.markings,
    decorations.footprints,
    shapes.arrows,
    matrix,
    positioning,
}

\usepgfplotslibrary{
    fillbetween,
    ternary,
    smithchart,
    patchplots,
    polar,
    colormaps,
    colorbrewer,
    colortol,           % docu not ready yet
}
\pgfqkeys{/codeexample}{%
    every codeexample/.append style={
        /pgfplots/every ternary axis/.append style={
            /pgfplots/legend style={fill=graphicbackground},
        }
    },
    tabsize=4,
}

\pgfplotsmanualenableexternalizationofexpensive

%\usetikzlibrary{external}
%\tikzexternalize[prefix=figures/]{pgfplots}

\title{%
    Manual for Package \PGFPlots{}\\
    {\small 2D/3D Plots in \LaTeX{}, Version \pgfplotsversion}\\
    {\small\url{https://github.com/pgf-tikz/pgfplots}}
    %\\{\small Attention: you are using an unstable development version.}
}

\makeatletter
\long\def\abstractsmuggle{%
    \centering
    \textbf{Abstract}\\[0.5cm]

    \begin{minipage}{12cm}
        \PGFPlots{} draws high-quality function plots in normal or logarithmic
        scaling with a user-friendly interface directly in \TeX{}. The user
        supplies axis labels, legend entries and the plot coordinates for one or
        more plots and \PGFPlots{} applies axis scaling, computes any logarithms
        and axis ticks and draws the plots. It supports line plots, scatter
        plots, piecewise constant plots, bar plots, area plots, mesh and surface
        plots, patch plots, contour plots, quiver plots, histogram plots, box
        plots, polar axes, ternary diagrams, smith charts and some more. It is
        based on Till Tantau's package \PGF{}/\Tikz{}.
    \end{minipage}

}%

\expandafter\date\expandafter{\@date\\[2cm]
    \abstractsmuggle
}%
\makeatother

%\includeonly{pgfplots.reference}


\begin{document}

\def\plotcoords{%
\addplot coordinates {
(5,8.312e-02)    (17,2.547e-02)   (49,7.407e-03)
(129,2.102e-03)  (321,5.874e-04)  (769,1.623e-04)
(1793,4.442e-05) (4097,1.207e-05) (9217,3.261e-06)
};

\addplot coordinates{
(7,8.472e-02)    (31,3.044e-02)    (111,1.022e-02)
(351,3.303e-03)  (1023,1.039e-03)  (2815,3.196e-04)
(7423,9.658e-05) (18943,2.873e-05) (47103,8.437e-06)
};

\addplot coordinates{
(9,7.881e-02)     (49,3.243e-02)    (209,1.232e-02)
(769,4.454e-03)   (2561,1.551e-03)  (7937,5.236e-04)
(23297,1.723e-04) (65537,5.545e-05) (178177,1.751e-05)
};

\addplot coordinates{
(11,6.887e-02)    (71,3.177e-02)     (351,1.341e-02)
(1471,5.334e-03)  (5503,2.027e-03)   (18943,7.415e-04)
(61183,2.628e-04) (187903,9.063e-05) (553983,3.053e-05)
};

\addplot coordinates{
(13,5.755e-02)     (97,2.925e-02)     (545,1.351e-02)
(2561,5.842e-03)   (10625,2.397e-03)  (40193,9.414e-04)
(141569,3.564e-04) (471041,1.308e-04)
(1496065,4.670e-05)
};
}%


\bookmaketitle
\tableofcontents
\include{pgfplots.title_abstract_intro}
\include{pgfplots.preliminaries}
\include{pgfplots.intro}
\include{pgfplots.reference}
\include{pgfplots.libs}
\include{pgfplots.resources}
\include{pgfplots.importexport}
\include{pgfplots.basic.reference}

\printindex

\bibliographystyle{abbrv} %gerapali} %gerabbrv} %gerunsrt.bst} %gerabbrv}% gerplain}
\nocite{pgfplotstable}
\nocite{programmingnotes}
\bibliography{pgfplots}
\end{document}

% -----------------------------------------------------------------------------
% For Stefan Pinnow as reminder on what to look for when editing the manual
% -----------------------------------------------------------------------------
% There should be no line breaks in the following environments
% - |...|
% - \declareandlabel{...}
% - \verbpdfref{...}
% If "MakeTikzPictures" isn't running through check one of the externalized
% LOG files what is the cause of that.
% -----------------------------------------------------------------------------


%%%%%%%%%%%%%%%%%%%%%%%%%%%%%%%%%%%%%%%%%%%%%%%%%%%%%%%%%%%%%%%%%%%%%%%%%%%%%
%
% Package pgfplots.sty documentation.
%
% Copyright 2007/2008 by Christian Feuersaenger.
%
% This program is free software: you can redistribute it and/or modify
% it under the terms of the GNU General Public License as published by
% the Free Software Foundation, either version 3 of the License, or
% (at your option) any later version.
%
% This program is distributed in the hope that it will be useful,
% but WITHOUT ANY WARRANTY; without even the implied warranty of
% MERCHANTABILITY or FITNESS FOR A PARTICULAR PURPOSE.  See the
% GNU General Public License for more details.
%
% You should have received a copy of the GNU General Public License
% along with this program.  If not, see <http://www.gnu.org/licenses/>.
%
%
%%%%%%%%%%%%%%%%%%%%%%%%%%%%%%%%%%%%%%%%%%%%%%%%%%%%%%%%%%%%%%%%%%%%%%%%%%%%%
% SEE pgfplots-macros.tex as well!
%\pdfminorversion=5 % to allow compression
%\pdfobjcompresslevel=2
\documentclass[a4paper,openany]{book}

\let\bookmaketitle=\maketitle

% -----------------------------------
% this here is from ltxdoc:
\usepackage{doc}[=v2]
\AtBeginDocument{\MakeShortVerb{\|}}
%\setlength{\textwidth}{355pt}
%\addtolength\marginparwidth{30pt}
%\addtolength\oddsidemargin{20pt}
%\addtolength\evensidemargin{20pt}
\def\cmd#1{\cs{\expandafter\cmd@to@cs\string#1}}
\def\cmd@to@cs#1#2{\char\number`#2\relax}
\DeclareRobustCommand\cs[1]{\texttt{\char`\\#1}}
\providecommand\marg[1]{%
  {\ttfamily\char`\{}\meta{#1}{\ttfamily\char`\}}}
\providecommand\oarg[1]{%
  {\ttfamily[}\meta{#1}{\ttfamily]}}
\providecommand\parg[1]{%
  {\ttfamily(}\meta{#1}{\ttfamily)}}
\raggedbottom

% -----------------------------------
\input{pgfplots.preamble.tex}

\makeatletter
% I want a two-column index just as in pgfmanual styles. This here
% was the best way to get one:
\def\index@prologue{\section*{Index}\addcontentsline{toc}{chapter}{Index}
}
\makeatother

%\RequirePackage[german,english,francais]{babel}

\def\matlabcolormaptext{This colormap is similar to one shipped with Matlab$^\text{\textregistered}$ under a similar name.}

\IfFileExists{tikzlibraryspy.code.tex}{%
\usetikzlibrary{spy}
}{%
    \message{ERROR: tikz SPY library NOT available. The manual will only compile partially.^^J}%
}%

\usepackage{xparse}% for colorbrewer manual

\usetikzlibrary{
    decorations.markings,
    decorations.footprints,
    shapes.arrows,
    matrix,
    positioning,
}

\usepgfplotslibrary{
    fillbetween,
    ternary,
    smithchart,
    patchplots,
    polar,
    colormaps,
    colorbrewer,
    colortol,           % docu not ready yet
}
\pgfqkeys{/codeexample}{%
    every codeexample/.append style={
        /pgfplots/every ternary axis/.append style={
            /pgfplots/legend style={fill=graphicbackground},
        }
    },
    tabsize=4,
}

\pgfplotsmanualenableexternalizationofexpensive

%\usetikzlibrary{external}
%\tikzexternalize[prefix=figures/]{pgfplots}

\title{%
    Manual for Package \PGFPlots{}\\
    {\small 2D/3D Plots in \LaTeX{}, Version \pgfplotsversion}\\
    {\small\url{https://github.com/pgf-tikz/pgfplots}}
    %\\{\small Attention: you are using an unstable development version.}
}

\makeatletter
\long\def\abstractsmuggle{%
    \centering
    \textbf{Abstract}\\[0.5cm]

    \begin{minipage}{12cm}
        \PGFPlots{} draws high-quality function plots in normal or logarithmic
        scaling with a user-friendly interface directly in \TeX{}. The user
        supplies axis labels, legend entries and the plot coordinates for one or
        more plots and \PGFPlots{} applies axis scaling, computes any logarithms
        and axis ticks and draws the plots. It supports line plots, scatter
        plots, piecewise constant plots, bar plots, area plots, mesh and surface
        plots, patch plots, contour plots, quiver plots, histogram plots, box
        plots, polar axes, ternary diagrams, smith charts and some more. It is
        based on Till Tantau's package \PGF{}/\Tikz{}.
    \end{minipage}

}%

\expandafter\date\expandafter{\@date\\[2cm]
    \abstractsmuggle
}%
\makeatother

%\includeonly{pgfplots.reference}


\begin{document}

\def\plotcoords{%
\addplot coordinates {
(5,8.312e-02)    (17,2.547e-02)   (49,7.407e-03)
(129,2.102e-03)  (321,5.874e-04)  (769,1.623e-04)
(1793,4.442e-05) (4097,1.207e-05) (9217,3.261e-06)
};

\addplot coordinates{
(7,8.472e-02)    (31,3.044e-02)    (111,1.022e-02)
(351,3.303e-03)  (1023,1.039e-03)  (2815,3.196e-04)
(7423,9.658e-05) (18943,2.873e-05) (47103,8.437e-06)
};

\addplot coordinates{
(9,7.881e-02)     (49,3.243e-02)    (209,1.232e-02)
(769,4.454e-03)   (2561,1.551e-03)  (7937,5.236e-04)
(23297,1.723e-04) (65537,5.545e-05) (178177,1.751e-05)
};

\addplot coordinates{
(11,6.887e-02)    (71,3.177e-02)     (351,1.341e-02)
(1471,5.334e-03)  (5503,2.027e-03)   (18943,7.415e-04)
(61183,2.628e-04) (187903,9.063e-05) (553983,3.053e-05)
};

\addplot coordinates{
(13,5.755e-02)     (97,2.925e-02)     (545,1.351e-02)
(2561,5.842e-03)   (10625,2.397e-03)  (40193,9.414e-04)
(141569,3.564e-04) (471041,1.308e-04)
(1496065,4.670e-05)
};
}%


\bookmaketitle
\tableofcontents
\include{pgfplots.title_abstract_intro}
\include{pgfplots.preliminaries}
\include{pgfplots.intro}
\include{pgfplots.reference}
\include{pgfplots.libs}
\include{pgfplots.resources}
\include{pgfplots.importexport}
\include{pgfplots.basic.reference}

\printindex

\bibliographystyle{abbrv} %gerapali} %gerabbrv} %gerunsrt.bst} %gerabbrv}% gerplain}
\nocite{pgfplotstable}
\nocite{programmingnotes}
\bibliography{pgfplots}
\end{document}

% -----------------------------------------------------------------------------
% For Stefan Pinnow as reminder on what to look for when editing the manual
% -----------------------------------------------------------------------------
% There should be no line breaks in the following environments
% - |...|
% - \declareandlabel{...}
% - \verbpdfref{...}
% If "MakeTikzPictures" isn't running through check one of the externalized
% LOG files what is the cause of that.
% -----------------------------------------------------------------------------


%%%%%%%%%%%%%%%%%%%%%%%%%%%%%%%%%%%%%%%%%%%%%%%%%%%%%%%%%%%%%%%%%%%%%%%%%%%%%
%
% Package pgfplots.sty documentation.
%
% Copyright 2007/2008 by Christian Feuersaenger.
%
% This program is free software: you can redistribute it and/or modify
% it under the terms of the GNU General Public License as published by
% the Free Software Foundation, either version 3 of the License, or
% (at your option) any later version.
%
% This program is distributed in the hope that it will be useful,
% but WITHOUT ANY WARRANTY; without even the implied warranty of
% MERCHANTABILITY or FITNESS FOR A PARTICULAR PURPOSE.  See the
% GNU General Public License for more details.
%
% You should have received a copy of the GNU General Public License
% along with this program.  If not, see <http://www.gnu.org/licenses/>.
%
%
%%%%%%%%%%%%%%%%%%%%%%%%%%%%%%%%%%%%%%%%%%%%%%%%%%%%%%%%%%%%%%%%%%%%%%%%%%%%%
% SEE pgfplots-macros.tex as well!
%\pdfminorversion=5 % to allow compression
%\pdfobjcompresslevel=2
\documentclass[a4paper,openany]{book}

\let\bookmaketitle=\maketitle

% -----------------------------------
% this here is from ltxdoc:
\usepackage{doc}[=v2]
\AtBeginDocument{\MakeShortVerb{\|}}
%\setlength{\textwidth}{355pt}
%\addtolength\marginparwidth{30pt}
%\addtolength\oddsidemargin{20pt}
%\addtolength\evensidemargin{20pt}
\def\cmd#1{\cs{\expandafter\cmd@to@cs\string#1}}
\def\cmd@to@cs#1#2{\char\number`#2\relax}
\DeclareRobustCommand\cs[1]{\texttt{\char`\\#1}}
\providecommand\marg[1]{%
  {\ttfamily\char`\{}\meta{#1}{\ttfamily\char`\}}}
\providecommand\oarg[1]{%
  {\ttfamily[}\meta{#1}{\ttfamily]}}
\providecommand\parg[1]{%
  {\ttfamily(}\meta{#1}{\ttfamily)}}
\raggedbottom

% -----------------------------------
\input{pgfplots.preamble.tex}

\makeatletter
% I want a two-column index just as in pgfmanual styles. This here
% was the best way to get one:
\def\index@prologue{\section*{Index}\addcontentsline{toc}{chapter}{Index}
}
\makeatother

%\RequirePackage[german,english,francais]{babel}

\def\matlabcolormaptext{This colormap is similar to one shipped with Matlab$^\text{\textregistered}$ under a similar name.}

\IfFileExists{tikzlibraryspy.code.tex}{%
\usetikzlibrary{spy}
}{%
    \message{ERROR: tikz SPY library NOT available. The manual will only compile partially.^^J}%
}%

\usepackage{xparse}% for colorbrewer manual

\usetikzlibrary{
    decorations.markings,
    decorations.footprints,
    shapes.arrows,
    matrix,
    positioning,
}

\usepgfplotslibrary{
    fillbetween,
    ternary,
    smithchart,
    patchplots,
    polar,
    colormaps,
    colorbrewer,
    colortol,           % docu not ready yet
}
\pgfqkeys{/codeexample}{%
    every codeexample/.append style={
        /pgfplots/every ternary axis/.append style={
            /pgfplots/legend style={fill=graphicbackground},
        }
    },
    tabsize=4,
}

\pgfplotsmanualenableexternalizationofexpensive

%\usetikzlibrary{external}
%\tikzexternalize[prefix=figures/]{pgfplots}

\title{%
    Manual for Package \PGFPlots{}\\
    {\small 2D/3D Plots in \LaTeX{}, Version \pgfplotsversion}\\
    {\small\url{https://github.com/pgf-tikz/pgfplots}}
    %\\{\small Attention: you are using an unstable development version.}
}

\makeatletter
\long\def\abstractsmuggle{%
    \centering
    \textbf{Abstract}\\[0.5cm]

    \begin{minipage}{12cm}
        \PGFPlots{} draws high-quality function plots in normal or logarithmic
        scaling with a user-friendly interface directly in \TeX{}. The user
        supplies axis labels, legend entries and the plot coordinates for one or
        more plots and \PGFPlots{} applies axis scaling, computes any logarithms
        and axis ticks and draws the plots. It supports line plots, scatter
        plots, piecewise constant plots, bar plots, area plots, mesh and surface
        plots, patch plots, contour plots, quiver plots, histogram plots, box
        plots, polar axes, ternary diagrams, smith charts and some more. It is
        based on Till Tantau's package \PGF{}/\Tikz{}.
    \end{minipage}

}%

\expandafter\date\expandafter{\@date\\[2cm]
    \abstractsmuggle
}%
\makeatother

%\includeonly{pgfplots.reference}


\begin{document}

\def\plotcoords{%
\addplot coordinates {
(5,8.312e-02)    (17,2.547e-02)   (49,7.407e-03)
(129,2.102e-03)  (321,5.874e-04)  (769,1.623e-04)
(1793,4.442e-05) (4097,1.207e-05) (9217,3.261e-06)
};

\addplot coordinates{
(7,8.472e-02)    (31,3.044e-02)    (111,1.022e-02)
(351,3.303e-03)  (1023,1.039e-03)  (2815,3.196e-04)
(7423,9.658e-05) (18943,2.873e-05) (47103,8.437e-06)
};

\addplot coordinates{
(9,7.881e-02)     (49,3.243e-02)    (209,1.232e-02)
(769,4.454e-03)   (2561,1.551e-03)  (7937,5.236e-04)
(23297,1.723e-04) (65537,5.545e-05) (178177,1.751e-05)
};

\addplot coordinates{
(11,6.887e-02)    (71,3.177e-02)     (351,1.341e-02)
(1471,5.334e-03)  (5503,2.027e-03)   (18943,7.415e-04)
(61183,2.628e-04) (187903,9.063e-05) (553983,3.053e-05)
};

\addplot coordinates{
(13,5.755e-02)     (97,2.925e-02)     (545,1.351e-02)
(2561,5.842e-03)   (10625,2.397e-03)  (40193,9.414e-04)
(141569,3.564e-04) (471041,1.308e-04)
(1496065,4.670e-05)
};
}%


\bookmaketitle
\tableofcontents
\include{pgfplots.title_abstract_intro}
\include{pgfplots.preliminaries}
\include{pgfplots.intro}
\include{pgfplots.reference}
\include{pgfplots.libs}
\include{pgfplots.resources}
\include{pgfplots.importexport}
\include{pgfplots.basic.reference}

\printindex

\bibliographystyle{abbrv} %gerapali} %gerabbrv} %gerunsrt.bst} %gerabbrv}% gerplain}
\nocite{pgfplotstable}
\nocite{programmingnotes}
\bibliography{pgfplots}
\end{document}

% -----------------------------------------------------------------------------
% For Stefan Pinnow as reminder on what to look for when editing the manual
% -----------------------------------------------------------------------------
% There should be no line breaks in the following environments
% - |...|
% - \declareandlabel{...}
% - \verbpdfref{...}
% If "MakeTikzPictures" isn't running through check one of the externalized
% LOG files what is the cause of that.
% -----------------------------------------------------------------------------


%%%%%%%%%%%%%%%%%%%%%%%%%%%%%%%%%%%%%%%%%%%%%%%%%%%%%%%%%%%%%%%%%%%%%%%%%%%%%
%
% Package pgfplots.sty documentation.
%
% Copyright 2007/2008 by Christian Feuersaenger.
%
% This program is free software: you can redistribute it and/or modify
% it under the terms of the GNU General Public License as published by
% the Free Software Foundation, either version 3 of the License, or
% (at your option) any later version.
%
% This program is distributed in the hope that it will be useful,
% but WITHOUT ANY WARRANTY; without even the implied warranty of
% MERCHANTABILITY or FITNESS FOR A PARTICULAR PURPOSE.  See the
% GNU General Public License for more details.
%
% You should have received a copy of the GNU General Public License
% along with this program.  If not, see <http://www.gnu.org/licenses/>.
%
%
%%%%%%%%%%%%%%%%%%%%%%%%%%%%%%%%%%%%%%%%%%%%%%%%%%%%%%%%%%%%%%%%%%%%%%%%%%%%%
% SEE pgfplots-macros.tex as well!
%\pdfminorversion=5 % to allow compression
%\pdfobjcompresslevel=2
\documentclass[a4paper,openany]{book}

\let\bookmaketitle=\maketitle

% -----------------------------------
% this here is from ltxdoc:
\usepackage{doc}[=v2]
\AtBeginDocument{\MakeShortVerb{\|}}
%\setlength{\textwidth}{355pt}
%\addtolength\marginparwidth{30pt}
%\addtolength\oddsidemargin{20pt}
%\addtolength\evensidemargin{20pt}
\def\cmd#1{\cs{\expandafter\cmd@to@cs\string#1}}
\def\cmd@to@cs#1#2{\char\number`#2\relax}
\DeclareRobustCommand\cs[1]{\texttt{\char`\\#1}}
\providecommand\marg[1]{%
  {\ttfamily\char`\{}\meta{#1}{\ttfamily\char`\}}}
\providecommand\oarg[1]{%
  {\ttfamily[}\meta{#1}{\ttfamily]}}
\providecommand\parg[1]{%
  {\ttfamily(}\meta{#1}{\ttfamily)}}
\raggedbottom

% -----------------------------------
\input{pgfplots.preamble.tex}

\makeatletter
% I want a two-column index just as in pgfmanual styles. This here
% was the best way to get one:
\def\index@prologue{\section*{Index}\addcontentsline{toc}{chapter}{Index}
}
\makeatother

%\RequirePackage[german,english,francais]{babel}

\def\matlabcolormaptext{This colormap is similar to one shipped with Matlab$^\text{\textregistered}$ under a similar name.}

\IfFileExists{tikzlibraryspy.code.tex}{%
\usetikzlibrary{spy}
}{%
    \message{ERROR: tikz SPY library NOT available. The manual will only compile partially.^^J}%
}%

\usepackage{xparse}% for colorbrewer manual

\usetikzlibrary{
    decorations.markings,
    decorations.footprints,
    shapes.arrows,
    matrix,
    positioning,
}

\usepgfplotslibrary{
    fillbetween,
    ternary,
    smithchart,
    patchplots,
    polar,
    colormaps,
    colorbrewer,
    colortol,           % docu not ready yet
}
\pgfqkeys{/codeexample}{%
    every codeexample/.append style={
        /pgfplots/every ternary axis/.append style={
            /pgfplots/legend style={fill=graphicbackground},
        }
    },
    tabsize=4,
}

\pgfplotsmanualenableexternalizationofexpensive

%\usetikzlibrary{external}
%\tikzexternalize[prefix=figures/]{pgfplots}

\title{%
    Manual for Package \PGFPlots{}\\
    {\small 2D/3D Plots in \LaTeX{}, Version \pgfplotsversion}\\
    {\small\url{https://github.com/pgf-tikz/pgfplots}}
    %\\{\small Attention: you are using an unstable development version.}
}

\makeatletter
\long\def\abstractsmuggle{%
    \centering
    \textbf{Abstract}\\[0.5cm]

    \begin{minipage}{12cm}
        \PGFPlots{} draws high-quality function plots in normal or logarithmic
        scaling with a user-friendly interface directly in \TeX{}. The user
        supplies axis labels, legend entries and the plot coordinates for one or
        more plots and \PGFPlots{} applies axis scaling, computes any logarithms
        and axis ticks and draws the plots. It supports line plots, scatter
        plots, piecewise constant plots, bar plots, area plots, mesh and surface
        plots, patch plots, contour plots, quiver plots, histogram plots, box
        plots, polar axes, ternary diagrams, smith charts and some more. It is
        based on Till Tantau's package \PGF{}/\Tikz{}.
    \end{minipage}

}%

\expandafter\date\expandafter{\@date\\[2cm]
    \abstractsmuggle
}%
\makeatother

%\includeonly{pgfplots.reference}


\begin{document}

\def\plotcoords{%
\addplot coordinates {
(5,8.312e-02)    (17,2.547e-02)   (49,7.407e-03)
(129,2.102e-03)  (321,5.874e-04)  (769,1.623e-04)
(1793,4.442e-05) (4097,1.207e-05) (9217,3.261e-06)
};

\addplot coordinates{
(7,8.472e-02)    (31,3.044e-02)    (111,1.022e-02)
(351,3.303e-03)  (1023,1.039e-03)  (2815,3.196e-04)
(7423,9.658e-05) (18943,2.873e-05) (47103,8.437e-06)
};

\addplot coordinates{
(9,7.881e-02)     (49,3.243e-02)    (209,1.232e-02)
(769,4.454e-03)   (2561,1.551e-03)  (7937,5.236e-04)
(23297,1.723e-04) (65537,5.545e-05) (178177,1.751e-05)
};

\addplot coordinates{
(11,6.887e-02)    (71,3.177e-02)     (351,1.341e-02)
(1471,5.334e-03)  (5503,2.027e-03)   (18943,7.415e-04)
(61183,2.628e-04) (187903,9.063e-05) (553983,3.053e-05)
};

\addplot coordinates{
(13,5.755e-02)     (97,2.925e-02)     (545,1.351e-02)
(2561,5.842e-03)   (10625,2.397e-03)  (40193,9.414e-04)
(141569,3.564e-04) (471041,1.308e-04)
(1496065,4.670e-05)
};
}%


\bookmaketitle
\tableofcontents
\include{pgfplots.title_abstract_intro}
\include{pgfplots.preliminaries}
\include{pgfplots.intro}
\include{pgfplots.reference}
\include{pgfplots.libs}
\include{pgfplots.resources}
\include{pgfplots.importexport}
\include{pgfplots.basic.reference}

\printindex

\bibliographystyle{abbrv} %gerapali} %gerabbrv} %gerunsrt.bst} %gerabbrv}% gerplain}
\nocite{pgfplotstable}
\nocite{programmingnotes}
\bibliography{pgfplots}
\end{document}

% -----------------------------------------------------------------------------
% For Stefan Pinnow as reminder on what to look for when editing the manual
% -----------------------------------------------------------------------------
% There should be no line breaks in the following environments
% - |...|
% - \declareandlabel{...}
% - \verbpdfref{...}
% If "MakeTikzPictures" isn't running through check one of the externalized
% LOG files what is the cause of that.
% -----------------------------------------------------------------------------


%%%%%%%%%%%%%%%%%%%%%%%%%%%%%%%%%%%%%%%%%%%%%%%%%%%%%%%%%%%%%%%%%%%%%%%%%%%%%
%
% Package pgfplots.sty documentation.
%
% Copyright 2007/2008 by Christian Feuersaenger.
%
% This program is free software: you can redistribute it and/or modify
% it under the terms of the GNU General Public License as published by
% the Free Software Foundation, either version 3 of the License, or
% (at your option) any later version.
%
% This program is distributed in the hope that it will be useful,
% but WITHOUT ANY WARRANTY; without even the implied warranty of
% MERCHANTABILITY or FITNESS FOR A PARTICULAR PURPOSE.  See the
% GNU General Public License for more details.
%
% You should have received a copy of the GNU General Public License
% along with this program.  If not, see <http://www.gnu.org/licenses/>.
%
%
%%%%%%%%%%%%%%%%%%%%%%%%%%%%%%%%%%%%%%%%%%%%%%%%%%%%%%%%%%%%%%%%%%%%%%%%%%%%%
% SEE pgfplots-macros.tex as well!
%\pdfminorversion=5 % to allow compression
%\pdfobjcompresslevel=2
\documentclass[a4paper,openany]{book}

\let\bookmaketitle=\maketitle

% -----------------------------------
% this here is from ltxdoc:
\usepackage{doc}[=v2]
\AtBeginDocument{\MakeShortVerb{\|}}
%\setlength{\textwidth}{355pt}
%\addtolength\marginparwidth{30pt}
%\addtolength\oddsidemargin{20pt}
%\addtolength\evensidemargin{20pt}
\def\cmd#1{\cs{\expandafter\cmd@to@cs\string#1}}
\def\cmd@to@cs#1#2{\char\number`#2\relax}
\DeclareRobustCommand\cs[1]{\texttt{\char`\\#1}}
\providecommand\marg[1]{%
  {\ttfamily\char`\{}\meta{#1}{\ttfamily\char`\}}}
\providecommand\oarg[1]{%
  {\ttfamily[}\meta{#1}{\ttfamily]}}
\providecommand\parg[1]{%
  {\ttfamily(}\meta{#1}{\ttfamily)}}
\raggedbottom

% -----------------------------------
\input{pgfplots.preamble.tex}

\makeatletter
% I want a two-column index just as in pgfmanual styles. This here
% was the best way to get one:
\def\index@prologue{\section*{Index}\addcontentsline{toc}{chapter}{Index}
}
\makeatother

%\RequirePackage[german,english,francais]{babel}

\def\matlabcolormaptext{This colormap is similar to one shipped with Matlab$^\text{\textregistered}$ under a similar name.}

\IfFileExists{tikzlibraryspy.code.tex}{%
\usetikzlibrary{spy}
}{%
    \message{ERROR: tikz SPY library NOT available. The manual will only compile partially.^^J}%
}%

\usepackage{xparse}% for colorbrewer manual

\usetikzlibrary{
    decorations.markings,
    decorations.footprints,
    shapes.arrows,
    matrix,
    positioning,
}

\usepgfplotslibrary{
    fillbetween,
    ternary,
    smithchart,
    patchplots,
    polar,
    colormaps,
    colorbrewer,
    colortol,           % docu not ready yet
}
\pgfqkeys{/codeexample}{%
    every codeexample/.append style={
        /pgfplots/every ternary axis/.append style={
            /pgfplots/legend style={fill=graphicbackground},
        }
    },
    tabsize=4,
}

\pgfplotsmanualenableexternalizationofexpensive

%\usetikzlibrary{external}
%\tikzexternalize[prefix=figures/]{pgfplots}

\title{%
    Manual for Package \PGFPlots{}\\
    {\small 2D/3D Plots in \LaTeX{}, Version \pgfplotsversion}\\
    {\small\url{https://github.com/pgf-tikz/pgfplots}}
    %\\{\small Attention: you are using an unstable development version.}
}

\makeatletter
\long\def\abstractsmuggle{%
    \centering
    \textbf{Abstract}\\[0.5cm]

    \begin{minipage}{12cm}
        \PGFPlots{} draws high-quality function plots in normal or logarithmic
        scaling with a user-friendly interface directly in \TeX{}. The user
        supplies axis labels, legend entries and the plot coordinates for one or
        more plots and \PGFPlots{} applies axis scaling, computes any logarithms
        and axis ticks and draws the plots. It supports line plots, scatter
        plots, piecewise constant plots, bar plots, area plots, mesh and surface
        plots, patch plots, contour plots, quiver plots, histogram plots, box
        plots, polar axes, ternary diagrams, smith charts and some more. It is
        based on Till Tantau's package \PGF{}/\Tikz{}.
    \end{minipage}

}%

\expandafter\date\expandafter{\@date\\[2cm]
    \abstractsmuggle
}%
\makeatother

%\includeonly{pgfplots.reference}


\begin{document}

\def\plotcoords{%
\addplot coordinates {
(5,8.312e-02)    (17,2.547e-02)   (49,7.407e-03)
(129,2.102e-03)  (321,5.874e-04)  (769,1.623e-04)
(1793,4.442e-05) (4097,1.207e-05) (9217,3.261e-06)
};

\addplot coordinates{
(7,8.472e-02)    (31,3.044e-02)    (111,1.022e-02)
(351,3.303e-03)  (1023,1.039e-03)  (2815,3.196e-04)
(7423,9.658e-05) (18943,2.873e-05) (47103,8.437e-06)
};

\addplot coordinates{
(9,7.881e-02)     (49,3.243e-02)    (209,1.232e-02)
(769,4.454e-03)   (2561,1.551e-03)  (7937,5.236e-04)
(23297,1.723e-04) (65537,5.545e-05) (178177,1.751e-05)
};

\addplot coordinates{
(11,6.887e-02)    (71,3.177e-02)     (351,1.341e-02)
(1471,5.334e-03)  (5503,2.027e-03)   (18943,7.415e-04)
(61183,2.628e-04) (187903,9.063e-05) (553983,3.053e-05)
};

\addplot coordinates{
(13,5.755e-02)     (97,2.925e-02)     (545,1.351e-02)
(2561,5.842e-03)   (10625,2.397e-03)  (40193,9.414e-04)
(141569,3.564e-04) (471041,1.308e-04)
(1496065,4.670e-05)
};
}%


\bookmaketitle
\tableofcontents
\include{pgfplots.title_abstract_intro}
\include{pgfplots.preliminaries}
\include{pgfplots.intro}
\include{pgfplots.reference}
\include{pgfplots.libs}
\include{pgfplots.resources}
\include{pgfplots.importexport}
\include{pgfplots.basic.reference}

\printindex

\bibliographystyle{abbrv} %gerapali} %gerabbrv} %gerunsrt.bst} %gerabbrv}% gerplain}
\nocite{pgfplotstable}
\nocite{programmingnotes}
\bibliography{pgfplots}
\end{document}

% -----------------------------------------------------------------------------
% For Stefan Pinnow as reminder on what to look for when editing the manual
% -----------------------------------------------------------------------------
% There should be no line breaks in the following environments
% - |...|
% - \declareandlabel{...}
% - \verbpdfref{...}
% If "MakeTikzPictures" isn't running through check one of the externalized
% LOG files what is the cause of that.
% -----------------------------------------------------------------------------


%%%%%%%%%%%%%%%%%%%%%%%%%%%%%%%%%%%%%%%%%%%%%%%%%%%%%%%%%%%%%%%%%%%%%%%%%%%%%
%
% Package pgfplots.sty documentation.
%
% Copyright 2007/2008 by Christian Feuersaenger.
%
% This program is free software: you can redistribute it and/or modify
% it under the terms of the GNU General Public License as published by
% the Free Software Foundation, either version 3 of the License, or
% (at your option) any later version.
%
% This program is distributed in the hope that it will be useful,
% but WITHOUT ANY WARRANTY; without even the implied warranty of
% MERCHANTABILITY or FITNESS FOR A PARTICULAR PURPOSE.  See the
% GNU General Public License for more details.
%
% You should have received a copy of the GNU General Public License
% along with this program.  If not, see <http://www.gnu.org/licenses/>.
%
%
%%%%%%%%%%%%%%%%%%%%%%%%%%%%%%%%%%%%%%%%%%%%%%%%%%%%%%%%%%%%%%%%%%%%%%%%%%%%%
% SEE pgfplots-macros.tex as well!
%\pdfminorversion=5 % to allow compression
%\pdfobjcompresslevel=2
\documentclass[a4paper,openany]{book}

\let\bookmaketitle=\maketitle

% -----------------------------------
% this here is from ltxdoc:
\usepackage{doc}[=v2]
\AtBeginDocument{\MakeShortVerb{\|}}
%\setlength{\textwidth}{355pt}
%\addtolength\marginparwidth{30pt}
%\addtolength\oddsidemargin{20pt}
%\addtolength\evensidemargin{20pt}
\def\cmd#1{\cs{\expandafter\cmd@to@cs\string#1}}
\def\cmd@to@cs#1#2{\char\number`#2\relax}
\DeclareRobustCommand\cs[1]{\texttt{\char`\\#1}}
\providecommand\marg[1]{%
  {\ttfamily\char`\{}\meta{#1}{\ttfamily\char`\}}}
\providecommand\oarg[1]{%
  {\ttfamily[}\meta{#1}{\ttfamily]}}
\providecommand\parg[1]{%
  {\ttfamily(}\meta{#1}{\ttfamily)}}
\raggedbottom

% -----------------------------------
\input{pgfplots.preamble.tex}

\makeatletter
% I want a two-column index just as in pgfmanual styles. This here
% was the best way to get one:
\def\index@prologue{\section*{Index}\addcontentsline{toc}{chapter}{Index}
}
\makeatother

%\RequirePackage[german,english,francais]{babel}

\def\matlabcolormaptext{This colormap is similar to one shipped with Matlab$^\text{\textregistered}$ under a similar name.}

\IfFileExists{tikzlibraryspy.code.tex}{%
\usetikzlibrary{spy}
}{%
    \message{ERROR: tikz SPY library NOT available. The manual will only compile partially.^^J}%
}%

\usepackage{xparse}% for colorbrewer manual

\usetikzlibrary{
    decorations.markings,
    decorations.footprints,
    shapes.arrows,
    matrix,
    positioning,
}

\usepgfplotslibrary{
    fillbetween,
    ternary,
    smithchart,
    patchplots,
    polar,
    colormaps,
    colorbrewer,
    colortol,           % docu not ready yet
}
\pgfqkeys{/codeexample}{%
    every codeexample/.append style={
        /pgfplots/every ternary axis/.append style={
            /pgfplots/legend style={fill=graphicbackground},
        }
    },
    tabsize=4,
}

\pgfplotsmanualenableexternalizationofexpensive

%\usetikzlibrary{external}
%\tikzexternalize[prefix=figures/]{pgfplots}

\title{%
    Manual for Package \PGFPlots{}\\
    {\small 2D/3D Plots in \LaTeX{}, Version \pgfplotsversion}\\
    {\small\url{https://github.com/pgf-tikz/pgfplots}}
    %\\{\small Attention: you are using an unstable development version.}
}

\makeatletter
\long\def\abstractsmuggle{%
    \centering
    \textbf{Abstract}\\[0.5cm]

    \begin{minipage}{12cm}
        \PGFPlots{} draws high-quality function plots in normal or logarithmic
        scaling with a user-friendly interface directly in \TeX{}. The user
        supplies axis labels, legend entries and the plot coordinates for one or
        more plots and \PGFPlots{} applies axis scaling, computes any logarithms
        and axis ticks and draws the plots. It supports line plots, scatter
        plots, piecewise constant plots, bar plots, area plots, mesh and surface
        plots, patch plots, contour plots, quiver plots, histogram plots, box
        plots, polar axes, ternary diagrams, smith charts and some more. It is
        based on Till Tantau's package \PGF{}/\Tikz{}.
    \end{minipage}

}%

\expandafter\date\expandafter{\@date\\[2cm]
    \abstractsmuggle
}%
\makeatother

%\includeonly{pgfplots.reference}


\begin{document}

\def\plotcoords{%
\addplot coordinates {
(5,8.312e-02)    (17,2.547e-02)   (49,7.407e-03)
(129,2.102e-03)  (321,5.874e-04)  (769,1.623e-04)
(1793,4.442e-05) (4097,1.207e-05) (9217,3.261e-06)
};

\addplot coordinates{
(7,8.472e-02)    (31,3.044e-02)    (111,1.022e-02)
(351,3.303e-03)  (1023,1.039e-03)  (2815,3.196e-04)
(7423,9.658e-05) (18943,2.873e-05) (47103,8.437e-06)
};

\addplot coordinates{
(9,7.881e-02)     (49,3.243e-02)    (209,1.232e-02)
(769,4.454e-03)   (2561,1.551e-03)  (7937,5.236e-04)
(23297,1.723e-04) (65537,5.545e-05) (178177,1.751e-05)
};

\addplot coordinates{
(11,6.887e-02)    (71,3.177e-02)     (351,1.341e-02)
(1471,5.334e-03)  (5503,2.027e-03)   (18943,7.415e-04)
(61183,2.628e-04) (187903,9.063e-05) (553983,3.053e-05)
};

\addplot coordinates{
(13,5.755e-02)     (97,2.925e-02)     (545,1.351e-02)
(2561,5.842e-03)   (10625,2.397e-03)  (40193,9.414e-04)
(141569,3.564e-04) (471041,1.308e-04)
(1496065,4.670e-05)
};
}%


\bookmaketitle
\tableofcontents
\include{pgfplots.title_abstract_intro}
\include{pgfplots.preliminaries}
\include{pgfplots.intro}
\include{pgfplots.reference}
\include{pgfplots.libs}
\include{pgfplots.resources}
\include{pgfplots.importexport}
\include{pgfplots.basic.reference}

\printindex

\bibliographystyle{abbrv} %gerapali} %gerabbrv} %gerunsrt.bst} %gerabbrv}% gerplain}
\nocite{pgfplotstable}
\nocite{programmingnotes}
\bibliography{pgfplots}
\end{document}

% -----------------------------------------------------------------------------
% For Stefan Pinnow as reminder on what to look for when editing the manual
% -----------------------------------------------------------------------------
% There should be no line breaks in the following environments
% - |...|
% - \declareandlabel{...}
% - \verbpdfref{...}
% If "MakeTikzPictures" isn't running through check one of the externalized
% LOG files what is the cause of that.
% -----------------------------------------------------------------------------

\include{pgfplots.basic.reference}

\printindex

\bibliographystyle{abbrv} %gerapali} %gerabbrv} %gerunsrt.bst} %gerabbrv}% gerplain}
\nocite{pgfplotstable}
\nocite{programmingnotes}
\bibliography{pgfplots}
\end{document}

% -----------------------------------------------------------------------------
% For Stefan Pinnow as reminder on what to look for when editing the manual
% -----------------------------------------------------------------------------
% There should be no line breaks in the following environments
% - |...|
% - \declareandlabel{...}
% - \verbpdfref{...}
% If "MakeTikzPictures" isn't running through check one of the externalized
% LOG files what is the cause of that.
% -----------------------------------------------------------------------------


%%%%%%%%%%%%%%%%%%%%%%%%%%%%%%%%%%%%%%%%%%%%%%%%%%%%%%%%%%%%%%%%%%%%%%%%%%%%%
%
% Package pgfplots.sty documentation.
%
% Copyright 2007/2008 by Christian Feuersaenger.
%
% This program is free software: you can redistribute it and/or modify
% it under the terms of the GNU General Public License as published by
% the Free Software Foundation, either version 3 of the License, or
% (at your option) any later version.
%
% This program is distributed in the hope that it will be useful,
% but WITHOUT ANY WARRANTY; without even the implied warranty of
% MERCHANTABILITY or FITNESS FOR A PARTICULAR PURPOSE.  See the
% GNU General Public License for more details.
%
% You should have received a copy of the GNU General Public License
% along with this program.  If not, see <http://www.gnu.org/licenses/>.
%
%
%%%%%%%%%%%%%%%%%%%%%%%%%%%%%%%%%%%%%%%%%%%%%%%%%%%%%%%%%%%%%%%%%%%%%%%%%%%%%
% SEE pgfplots-macros.tex as well!
%\pdfminorversion=5 % to allow compression
%\pdfobjcompresslevel=2
\documentclass[a4paper,openany]{book}

\let\bookmaketitle=\maketitle

% -----------------------------------
% this here is from ltxdoc:
\usepackage{doc}[=v2]
\AtBeginDocument{\MakeShortVerb{\|}}
%\setlength{\textwidth}{355pt}
%\addtolength\marginparwidth{30pt}
%\addtolength\oddsidemargin{20pt}
%\addtolength\evensidemargin{20pt}
\def\cmd#1{\cs{\expandafter\cmd@to@cs\string#1}}
\def\cmd@to@cs#1#2{\char\number`#2\relax}
\DeclareRobustCommand\cs[1]{\texttt{\char`\\#1}}
\providecommand\marg[1]{%
  {\ttfamily\char`\{}\meta{#1}{\ttfamily\char`\}}}
\providecommand\oarg[1]{%
  {\ttfamily[}\meta{#1}{\ttfamily]}}
\providecommand\parg[1]{%
  {\ttfamily(}\meta{#1}{\ttfamily)}}
\raggedbottom

% -----------------------------------
\input{pgfplots.preamble.tex}

\makeatletter
% I want a two-column index just as in pgfmanual styles. This here
% was the best way to get one:
\def\index@prologue{\section*{Index}\addcontentsline{toc}{chapter}{Index}
}
\makeatother

%\RequirePackage[german,english,francais]{babel}

\def\matlabcolormaptext{This colormap is similar to one shipped with Matlab$^\text{\textregistered}$ under a similar name.}

\IfFileExists{tikzlibraryspy.code.tex}{%
\usetikzlibrary{spy}
}{%
    \message{ERROR: tikz SPY library NOT available. The manual will only compile partially.^^J}%
}%

\usepackage{xparse}% for colorbrewer manual

\usetikzlibrary{
    decorations.markings,
    decorations.footprints,
    shapes.arrows,
    matrix,
    positioning,
}

\usepgfplotslibrary{
    fillbetween,
    ternary,
    smithchart,
    patchplots,
    polar,
    colormaps,
    colorbrewer,
    colortol,           % docu not ready yet
}
\pgfqkeys{/codeexample}{%
    every codeexample/.append style={
        /pgfplots/every ternary axis/.append style={
            /pgfplots/legend style={fill=graphicbackground},
        }
    },
    tabsize=4,
}

\pgfplotsmanualenableexternalizationofexpensive

%\usetikzlibrary{external}
%\tikzexternalize[prefix=figures/]{pgfplots}

\title{%
    Manual for Package \PGFPlots{}\\
    {\small 2D/3D Plots in \LaTeX{}, Version \pgfplotsversion}\\
    {\small\url{https://github.com/pgf-tikz/pgfplots}}
    %\\{\small Attention: you are using an unstable development version.}
}

\makeatletter
\long\def\abstractsmuggle{%
    \centering
    \textbf{Abstract}\\[0.5cm]

    \begin{minipage}{12cm}
        \PGFPlots{} draws high-quality function plots in normal or logarithmic
        scaling with a user-friendly interface directly in \TeX{}. The user
        supplies axis labels, legend entries and the plot coordinates for one or
        more plots and \PGFPlots{} applies axis scaling, computes any logarithms
        and axis ticks and draws the plots. It supports line plots, scatter
        plots, piecewise constant plots, bar plots, area plots, mesh and surface
        plots, patch plots, contour plots, quiver plots, histogram plots, box
        plots, polar axes, ternary diagrams, smith charts and some more. It is
        based on Till Tantau's package \PGF{}/\Tikz{}.
    \end{minipage}

}%

\expandafter\date\expandafter{\@date\\[2cm]
    \abstractsmuggle
}%
\makeatother

%\includeonly{pgfplots.reference}


\begin{document}

\def\plotcoords{%
\addplot coordinates {
(5,8.312e-02)    (17,2.547e-02)   (49,7.407e-03)
(129,2.102e-03)  (321,5.874e-04)  (769,1.623e-04)
(1793,4.442e-05) (4097,1.207e-05) (9217,3.261e-06)
};

\addplot coordinates{
(7,8.472e-02)    (31,3.044e-02)    (111,1.022e-02)
(351,3.303e-03)  (1023,1.039e-03)  (2815,3.196e-04)
(7423,9.658e-05) (18943,2.873e-05) (47103,8.437e-06)
};

\addplot coordinates{
(9,7.881e-02)     (49,3.243e-02)    (209,1.232e-02)
(769,4.454e-03)   (2561,1.551e-03)  (7937,5.236e-04)
(23297,1.723e-04) (65537,5.545e-05) (178177,1.751e-05)
};

\addplot coordinates{
(11,6.887e-02)    (71,3.177e-02)     (351,1.341e-02)
(1471,5.334e-03)  (5503,2.027e-03)   (18943,7.415e-04)
(61183,2.628e-04) (187903,9.063e-05) (553983,3.053e-05)
};

\addplot coordinates{
(13,5.755e-02)     (97,2.925e-02)     (545,1.351e-02)
(2561,5.842e-03)   (10625,2.397e-03)  (40193,9.414e-04)
(141569,3.564e-04) (471041,1.308e-04)
(1496065,4.670e-05)
};
}%


\bookmaketitle
\tableofcontents

%%%%%%%%%%%%%%%%%%%%%%%%%%%%%%%%%%%%%%%%%%%%%%%%%%%%%%%%%%%%%%%%%%%%%%%%%%%%%
%
% Package pgfplots.sty documentation.
%
% Copyright 2007/2008 by Christian Feuersaenger.
%
% This program is free software: you can redistribute it and/or modify
% it under the terms of the GNU General Public License as published by
% the Free Software Foundation, either version 3 of the License, or
% (at your option) any later version.
%
% This program is distributed in the hope that it will be useful,
% but WITHOUT ANY WARRANTY; without even the implied warranty of
% MERCHANTABILITY or FITNESS FOR A PARTICULAR PURPOSE.  See the
% GNU General Public License for more details.
%
% You should have received a copy of the GNU General Public License
% along with this program.  If not, see <http://www.gnu.org/licenses/>.
%
%
%%%%%%%%%%%%%%%%%%%%%%%%%%%%%%%%%%%%%%%%%%%%%%%%%%%%%%%%%%%%%%%%%%%%%%%%%%%%%
% SEE pgfplots-macros.tex as well!
%\pdfminorversion=5 % to allow compression
%\pdfobjcompresslevel=2
\documentclass[a4paper,openany]{book}

\let\bookmaketitle=\maketitle

% -----------------------------------
% this here is from ltxdoc:
\usepackage{doc}[=v2]
\AtBeginDocument{\MakeShortVerb{\|}}
%\setlength{\textwidth}{355pt}
%\addtolength\marginparwidth{30pt}
%\addtolength\oddsidemargin{20pt}
%\addtolength\evensidemargin{20pt}
\def\cmd#1{\cs{\expandafter\cmd@to@cs\string#1}}
\def\cmd@to@cs#1#2{\char\number`#2\relax}
\DeclareRobustCommand\cs[1]{\texttt{\char`\\#1}}
\providecommand\marg[1]{%
  {\ttfamily\char`\{}\meta{#1}{\ttfamily\char`\}}}
\providecommand\oarg[1]{%
  {\ttfamily[}\meta{#1}{\ttfamily]}}
\providecommand\parg[1]{%
  {\ttfamily(}\meta{#1}{\ttfamily)}}
\raggedbottom

% -----------------------------------
\input{pgfplots.preamble.tex}

\makeatletter
% I want a two-column index just as in pgfmanual styles. This here
% was the best way to get one:
\def\index@prologue{\section*{Index}\addcontentsline{toc}{chapter}{Index}
}
\makeatother

%\RequirePackage[german,english,francais]{babel}

\def\matlabcolormaptext{This colormap is similar to one shipped with Matlab$^\text{\textregistered}$ under a similar name.}

\IfFileExists{tikzlibraryspy.code.tex}{%
\usetikzlibrary{spy}
}{%
    \message{ERROR: tikz SPY library NOT available. The manual will only compile partially.^^J}%
}%

\usepackage{xparse}% for colorbrewer manual

\usetikzlibrary{
    decorations.markings,
    decorations.footprints,
    shapes.arrows,
    matrix,
    positioning,
}

\usepgfplotslibrary{
    fillbetween,
    ternary,
    smithchart,
    patchplots,
    polar,
    colormaps,
    colorbrewer,
    colortol,           % docu not ready yet
}
\pgfqkeys{/codeexample}{%
    every codeexample/.append style={
        /pgfplots/every ternary axis/.append style={
            /pgfplots/legend style={fill=graphicbackground},
        }
    },
    tabsize=4,
}

\pgfplotsmanualenableexternalizationofexpensive

%\usetikzlibrary{external}
%\tikzexternalize[prefix=figures/]{pgfplots}

\title{%
    Manual for Package \PGFPlots{}\\
    {\small 2D/3D Plots in \LaTeX{}, Version \pgfplotsversion}\\
    {\small\url{https://github.com/pgf-tikz/pgfplots}}
    %\\{\small Attention: you are using an unstable development version.}
}

\makeatletter
\long\def\abstractsmuggle{%
    \centering
    \textbf{Abstract}\\[0.5cm]

    \begin{minipage}{12cm}
        \PGFPlots{} draws high-quality function plots in normal or logarithmic
        scaling with a user-friendly interface directly in \TeX{}. The user
        supplies axis labels, legend entries and the plot coordinates for one or
        more plots and \PGFPlots{} applies axis scaling, computes any logarithms
        and axis ticks and draws the plots. It supports line plots, scatter
        plots, piecewise constant plots, bar plots, area plots, mesh and surface
        plots, patch plots, contour plots, quiver plots, histogram plots, box
        plots, polar axes, ternary diagrams, smith charts and some more. It is
        based on Till Tantau's package \PGF{}/\Tikz{}.
    \end{minipage}

}%

\expandafter\date\expandafter{\@date\\[2cm]
    \abstractsmuggle
}%
\makeatother

%\includeonly{pgfplots.reference}


\begin{document}

\def\plotcoords{%
\addplot coordinates {
(5,8.312e-02)    (17,2.547e-02)   (49,7.407e-03)
(129,2.102e-03)  (321,5.874e-04)  (769,1.623e-04)
(1793,4.442e-05) (4097,1.207e-05) (9217,3.261e-06)
};

\addplot coordinates{
(7,8.472e-02)    (31,3.044e-02)    (111,1.022e-02)
(351,3.303e-03)  (1023,1.039e-03)  (2815,3.196e-04)
(7423,9.658e-05) (18943,2.873e-05) (47103,8.437e-06)
};

\addplot coordinates{
(9,7.881e-02)     (49,3.243e-02)    (209,1.232e-02)
(769,4.454e-03)   (2561,1.551e-03)  (7937,5.236e-04)
(23297,1.723e-04) (65537,5.545e-05) (178177,1.751e-05)
};

\addplot coordinates{
(11,6.887e-02)    (71,3.177e-02)     (351,1.341e-02)
(1471,5.334e-03)  (5503,2.027e-03)   (18943,7.415e-04)
(61183,2.628e-04) (187903,9.063e-05) (553983,3.053e-05)
};

\addplot coordinates{
(13,5.755e-02)     (97,2.925e-02)     (545,1.351e-02)
(2561,5.842e-03)   (10625,2.397e-03)  (40193,9.414e-04)
(141569,3.564e-04) (471041,1.308e-04)
(1496065,4.670e-05)
};
}%


\bookmaketitle
\tableofcontents
\include{pgfplots.title_abstract_intro}
\include{pgfplots.preliminaries}
\include{pgfplots.intro}
\include{pgfplots.reference}
\include{pgfplots.libs}
\include{pgfplots.resources}
\include{pgfplots.importexport}
\include{pgfplots.basic.reference}

\printindex

\bibliographystyle{abbrv} %gerapali} %gerabbrv} %gerunsrt.bst} %gerabbrv}% gerplain}
\nocite{pgfplotstable}
\nocite{programmingnotes}
\bibliography{pgfplots}
\end{document}

% -----------------------------------------------------------------------------
% For Stefan Pinnow as reminder on what to look for when editing the manual
% -----------------------------------------------------------------------------
% There should be no line breaks in the following environments
% - |...|
% - \declareandlabel{...}
% - \verbpdfref{...}
% If "MakeTikzPictures" isn't running through check one of the externalized
% LOG files what is the cause of that.
% -----------------------------------------------------------------------------


%%%%%%%%%%%%%%%%%%%%%%%%%%%%%%%%%%%%%%%%%%%%%%%%%%%%%%%%%%%%%%%%%%%%%%%%%%%%%
%
% Package pgfplots.sty documentation.
%
% Copyright 2007/2008 by Christian Feuersaenger.
%
% This program is free software: you can redistribute it and/or modify
% it under the terms of the GNU General Public License as published by
% the Free Software Foundation, either version 3 of the License, or
% (at your option) any later version.
%
% This program is distributed in the hope that it will be useful,
% but WITHOUT ANY WARRANTY; without even the implied warranty of
% MERCHANTABILITY or FITNESS FOR A PARTICULAR PURPOSE.  See the
% GNU General Public License for more details.
%
% You should have received a copy of the GNU General Public License
% along with this program.  If not, see <http://www.gnu.org/licenses/>.
%
%
%%%%%%%%%%%%%%%%%%%%%%%%%%%%%%%%%%%%%%%%%%%%%%%%%%%%%%%%%%%%%%%%%%%%%%%%%%%%%
% SEE pgfplots-macros.tex as well!
%\pdfminorversion=5 % to allow compression
%\pdfobjcompresslevel=2
\documentclass[a4paper,openany]{book}

\let\bookmaketitle=\maketitle

% -----------------------------------
% this here is from ltxdoc:
\usepackage{doc}[=v2]
\AtBeginDocument{\MakeShortVerb{\|}}
%\setlength{\textwidth}{355pt}
%\addtolength\marginparwidth{30pt}
%\addtolength\oddsidemargin{20pt}
%\addtolength\evensidemargin{20pt}
\def\cmd#1{\cs{\expandafter\cmd@to@cs\string#1}}
\def\cmd@to@cs#1#2{\char\number`#2\relax}
\DeclareRobustCommand\cs[1]{\texttt{\char`\\#1}}
\providecommand\marg[1]{%
  {\ttfamily\char`\{}\meta{#1}{\ttfamily\char`\}}}
\providecommand\oarg[1]{%
  {\ttfamily[}\meta{#1}{\ttfamily]}}
\providecommand\parg[1]{%
  {\ttfamily(}\meta{#1}{\ttfamily)}}
\raggedbottom

% -----------------------------------
\input{pgfplots.preamble.tex}

\makeatletter
% I want a two-column index just as in pgfmanual styles. This here
% was the best way to get one:
\def\index@prologue{\section*{Index}\addcontentsline{toc}{chapter}{Index}
}
\makeatother

%\RequirePackage[german,english,francais]{babel}

\def\matlabcolormaptext{This colormap is similar to one shipped with Matlab$^\text{\textregistered}$ under a similar name.}

\IfFileExists{tikzlibraryspy.code.tex}{%
\usetikzlibrary{spy}
}{%
    \message{ERROR: tikz SPY library NOT available. The manual will only compile partially.^^J}%
}%

\usepackage{xparse}% for colorbrewer manual

\usetikzlibrary{
    decorations.markings,
    decorations.footprints,
    shapes.arrows,
    matrix,
    positioning,
}

\usepgfplotslibrary{
    fillbetween,
    ternary,
    smithchart,
    patchplots,
    polar,
    colormaps,
    colorbrewer,
    colortol,           % docu not ready yet
}
\pgfqkeys{/codeexample}{%
    every codeexample/.append style={
        /pgfplots/every ternary axis/.append style={
            /pgfplots/legend style={fill=graphicbackground},
        }
    },
    tabsize=4,
}

\pgfplotsmanualenableexternalizationofexpensive

%\usetikzlibrary{external}
%\tikzexternalize[prefix=figures/]{pgfplots}

\title{%
    Manual for Package \PGFPlots{}\\
    {\small 2D/3D Plots in \LaTeX{}, Version \pgfplotsversion}\\
    {\small\url{https://github.com/pgf-tikz/pgfplots}}
    %\\{\small Attention: you are using an unstable development version.}
}

\makeatletter
\long\def\abstractsmuggle{%
    \centering
    \textbf{Abstract}\\[0.5cm]

    \begin{minipage}{12cm}
        \PGFPlots{} draws high-quality function plots in normal or logarithmic
        scaling with a user-friendly interface directly in \TeX{}. The user
        supplies axis labels, legend entries and the plot coordinates for one or
        more plots and \PGFPlots{} applies axis scaling, computes any logarithms
        and axis ticks and draws the plots. It supports line plots, scatter
        plots, piecewise constant plots, bar plots, area plots, mesh and surface
        plots, patch plots, contour plots, quiver plots, histogram plots, box
        plots, polar axes, ternary diagrams, smith charts and some more. It is
        based on Till Tantau's package \PGF{}/\Tikz{}.
    \end{minipage}

}%

\expandafter\date\expandafter{\@date\\[2cm]
    \abstractsmuggle
}%
\makeatother

%\includeonly{pgfplots.reference}


\begin{document}

\def\plotcoords{%
\addplot coordinates {
(5,8.312e-02)    (17,2.547e-02)   (49,7.407e-03)
(129,2.102e-03)  (321,5.874e-04)  (769,1.623e-04)
(1793,4.442e-05) (4097,1.207e-05) (9217,3.261e-06)
};

\addplot coordinates{
(7,8.472e-02)    (31,3.044e-02)    (111,1.022e-02)
(351,3.303e-03)  (1023,1.039e-03)  (2815,3.196e-04)
(7423,9.658e-05) (18943,2.873e-05) (47103,8.437e-06)
};

\addplot coordinates{
(9,7.881e-02)     (49,3.243e-02)    (209,1.232e-02)
(769,4.454e-03)   (2561,1.551e-03)  (7937,5.236e-04)
(23297,1.723e-04) (65537,5.545e-05) (178177,1.751e-05)
};

\addplot coordinates{
(11,6.887e-02)    (71,3.177e-02)     (351,1.341e-02)
(1471,5.334e-03)  (5503,2.027e-03)   (18943,7.415e-04)
(61183,2.628e-04) (187903,9.063e-05) (553983,3.053e-05)
};

\addplot coordinates{
(13,5.755e-02)     (97,2.925e-02)     (545,1.351e-02)
(2561,5.842e-03)   (10625,2.397e-03)  (40193,9.414e-04)
(141569,3.564e-04) (471041,1.308e-04)
(1496065,4.670e-05)
};
}%


\bookmaketitle
\tableofcontents
\include{pgfplots.title_abstract_intro}
\include{pgfplots.preliminaries}
\include{pgfplots.intro}
\include{pgfplots.reference}
\include{pgfplots.libs}
\include{pgfplots.resources}
\include{pgfplots.importexport}
\include{pgfplots.basic.reference}

\printindex

\bibliographystyle{abbrv} %gerapali} %gerabbrv} %gerunsrt.bst} %gerabbrv}% gerplain}
\nocite{pgfplotstable}
\nocite{programmingnotes}
\bibliography{pgfplots}
\end{document}

% -----------------------------------------------------------------------------
% For Stefan Pinnow as reminder on what to look for when editing the manual
% -----------------------------------------------------------------------------
% There should be no line breaks in the following environments
% - |...|
% - \declareandlabel{...}
% - \verbpdfref{...}
% If "MakeTikzPictures" isn't running through check one of the externalized
% LOG files what is the cause of that.
% -----------------------------------------------------------------------------


%%%%%%%%%%%%%%%%%%%%%%%%%%%%%%%%%%%%%%%%%%%%%%%%%%%%%%%%%%%%%%%%%%%%%%%%%%%%%
%
% Package pgfplots.sty documentation.
%
% Copyright 2007/2008 by Christian Feuersaenger.
%
% This program is free software: you can redistribute it and/or modify
% it under the terms of the GNU General Public License as published by
% the Free Software Foundation, either version 3 of the License, or
% (at your option) any later version.
%
% This program is distributed in the hope that it will be useful,
% but WITHOUT ANY WARRANTY; without even the implied warranty of
% MERCHANTABILITY or FITNESS FOR A PARTICULAR PURPOSE.  See the
% GNU General Public License for more details.
%
% You should have received a copy of the GNU General Public License
% along with this program.  If not, see <http://www.gnu.org/licenses/>.
%
%
%%%%%%%%%%%%%%%%%%%%%%%%%%%%%%%%%%%%%%%%%%%%%%%%%%%%%%%%%%%%%%%%%%%%%%%%%%%%%
% SEE pgfplots-macros.tex as well!
%\pdfminorversion=5 % to allow compression
%\pdfobjcompresslevel=2
\documentclass[a4paper,openany]{book}

\let\bookmaketitle=\maketitle

% -----------------------------------
% this here is from ltxdoc:
\usepackage{doc}[=v2]
\AtBeginDocument{\MakeShortVerb{\|}}
%\setlength{\textwidth}{355pt}
%\addtolength\marginparwidth{30pt}
%\addtolength\oddsidemargin{20pt}
%\addtolength\evensidemargin{20pt}
\def\cmd#1{\cs{\expandafter\cmd@to@cs\string#1}}
\def\cmd@to@cs#1#2{\char\number`#2\relax}
\DeclareRobustCommand\cs[1]{\texttt{\char`\\#1}}
\providecommand\marg[1]{%
  {\ttfamily\char`\{}\meta{#1}{\ttfamily\char`\}}}
\providecommand\oarg[1]{%
  {\ttfamily[}\meta{#1}{\ttfamily]}}
\providecommand\parg[1]{%
  {\ttfamily(}\meta{#1}{\ttfamily)}}
\raggedbottom

% -----------------------------------
\input{pgfplots.preamble.tex}

\makeatletter
% I want a two-column index just as in pgfmanual styles. This here
% was the best way to get one:
\def\index@prologue{\section*{Index}\addcontentsline{toc}{chapter}{Index}
}
\makeatother

%\RequirePackage[german,english,francais]{babel}

\def\matlabcolormaptext{This colormap is similar to one shipped with Matlab$^\text{\textregistered}$ under a similar name.}

\IfFileExists{tikzlibraryspy.code.tex}{%
\usetikzlibrary{spy}
}{%
    \message{ERROR: tikz SPY library NOT available. The manual will only compile partially.^^J}%
}%

\usepackage{xparse}% for colorbrewer manual

\usetikzlibrary{
    decorations.markings,
    decorations.footprints,
    shapes.arrows,
    matrix,
    positioning,
}

\usepgfplotslibrary{
    fillbetween,
    ternary,
    smithchart,
    patchplots,
    polar,
    colormaps,
    colorbrewer,
    colortol,           % docu not ready yet
}
\pgfqkeys{/codeexample}{%
    every codeexample/.append style={
        /pgfplots/every ternary axis/.append style={
            /pgfplots/legend style={fill=graphicbackground},
        }
    },
    tabsize=4,
}

\pgfplotsmanualenableexternalizationofexpensive

%\usetikzlibrary{external}
%\tikzexternalize[prefix=figures/]{pgfplots}

\title{%
    Manual for Package \PGFPlots{}\\
    {\small 2D/3D Plots in \LaTeX{}, Version \pgfplotsversion}\\
    {\small\url{https://github.com/pgf-tikz/pgfplots}}
    %\\{\small Attention: you are using an unstable development version.}
}

\makeatletter
\long\def\abstractsmuggle{%
    \centering
    \textbf{Abstract}\\[0.5cm]

    \begin{minipage}{12cm}
        \PGFPlots{} draws high-quality function plots in normal or logarithmic
        scaling with a user-friendly interface directly in \TeX{}. The user
        supplies axis labels, legend entries and the plot coordinates for one or
        more plots and \PGFPlots{} applies axis scaling, computes any logarithms
        and axis ticks and draws the plots. It supports line plots, scatter
        plots, piecewise constant plots, bar plots, area plots, mesh and surface
        plots, patch plots, contour plots, quiver plots, histogram plots, box
        plots, polar axes, ternary diagrams, smith charts and some more. It is
        based on Till Tantau's package \PGF{}/\Tikz{}.
    \end{minipage}

}%

\expandafter\date\expandafter{\@date\\[2cm]
    \abstractsmuggle
}%
\makeatother

%\includeonly{pgfplots.reference}


\begin{document}

\def\plotcoords{%
\addplot coordinates {
(5,8.312e-02)    (17,2.547e-02)   (49,7.407e-03)
(129,2.102e-03)  (321,5.874e-04)  (769,1.623e-04)
(1793,4.442e-05) (4097,1.207e-05) (9217,3.261e-06)
};

\addplot coordinates{
(7,8.472e-02)    (31,3.044e-02)    (111,1.022e-02)
(351,3.303e-03)  (1023,1.039e-03)  (2815,3.196e-04)
(7423,9.658e-05) (18943,2.873e-05) (47103,8.437e-06)
};

\addplot coordinates{
(9,7.881e-02)     (49,3.243e-02)    (209,1.232e-02)
(769,4.454e-03)   (2561,1.551e-03)  (7937,5.236e-04)
(23297,1.723e-04) (65537,5.545e-05) (178177,1.751e-05)
};

\addplot coordinates{
(11,6.887e-02)    (71,3.177e-02)     (351,1.341e-02)
(1471,5.334e-03)  (5503,2.027e-03)   (18943,7.415e-04)
(61183,2.628e-04) (187903,9.063e-05) (553983,3.053e-05)
};

\addplot coordinates{
(13,5.755e-02)     (97,2.925e-02)     (545,1.351e-02)
(2561,5.842e-03)   (10625,2.397e-03)  (40193,9.414e-04)
(141569,3.564e-04) (471041,1.308e-04)
(1496065,4.670e-05)
};
}%


\bookmaketitle
\tableofcontents
\include{pgfplots.title_abstract_intro}
\include{pgfplots.preliminaries}
\include{pgfplots.intro}
\include{pgfplots.reference}
\include{pgfplots.libs}
\include{pgfplots.resources}
\include{pgfplots.importexport}
\include{pgfplots.basic.reference}

\printindex

\bibliographystyle{abbrv} %gerapali} %gerabbrv} %gerunsrt.bst} %gerabbrv}% gerplain}
\nocite{pgfplotstable}
\nocite{programmingnotes}
\bibliography{pgfplots}
\end{document}

% -----------------------------------------------------------------------------
% For Stefan Pinnow as reminder on what to look for when editing the manual
% -----------------------------------------------------------------------------
% There should be no line breaks in the following environments
% - |...|
% - \declareandlabel{...}
% - \verbpdfref{...}
% If "MakeTikzPictures" isn't running through check one of the externalized
% LOG files what is the cause of that.
% -----------------------------------------------------------------------------


%%%%%%%%%%%%%%%%%%%%%%%%%%%%%%%%%%%%%%%%%%%%%%%%%%%%%%%%%%%%%%%%%%%%%%%%%%%%%
%
% Package pgfplots.sty documentation.
%
% Copyright 2007/2008 by Christian Feuersaenger.
%
% This program is free software: you can redistribute it and/or modify
% it under the terms of the GNU General Public License as published by
% the Free Software Foundation, either version 3 of the License, or
% (at your option) any later version.
%
% This program is distributed in the hope that it will be useful,
% but WITHOUT ANY WARRANTY; without even the implied warranty of
% MERCHANTABILITY or FITNESS FOR A PARTICULAR PURPOSE.  See the
% GNU General Public License for more details.
%
% You should have received a copy of the GNU General Public License
% along with this program.  If not, see <http://www.gnu.org/licenses/>.
%
%
%%%%%%%%%%%%%%%%%%%%%%%%%%%%%%%%%%%%%%%%%%%%%%%%%%%%%%%%%%%%%%%%%%%%%%%%%%%%%
% SEE pgfplots-macros.tex as well!
%\pdfminorversion=5 % to allow compression
%\pdfobjcompresslevel=2
\documentclass[a4paper,openany]{book}

\let\bookmaketitle=\maketitle

% -----------------------------------
% this here is from ltxdoc:
\usepackage{doc}[=v2]
\AtBeginDocument{\MakeShortVerb{\|}}
%\setlength{\textwidth}{355pt}
%\addtolength\marginparwidth{30pt}
%\addtolength\oddsidemargin{20pt}
%\addtolength\evensidemargin{20pt}
\def\cmd#1{\cs{\expandafter\cmd@to@cs\string#1}}
\def\cmd@to@cs#1#2{\char\number`#2\relax}
\DeclareRobustCommand\cs[1]{\texttt{\char`\\#1}}
\providecommand\marg[1]{%
  {\ttfamily\char`\{}\meta{#1}{\ttfamily\char`\}}}
\providecommand\oarg[1]{%
  {\ttfamily[}\meta{#1}{\ttfamily]}}
\providecommand\parg[1]{%
  {\ttfamily(}\meta{#1}{\ttfamily)}}
\raggedbottom

% -----------------------------------
\input{pgfplots.preamble.tex}

\makeatletter
% I want a two-column index just as in pgfmanual styles. This here
% was the best way to get one:
\def\index@prologue{\section*{Index}\addcontentsline{toc}{chapter}{Index}
}
\makeatother

%\RequirePackage[german,english,francais]{babel}

\def\matlabcolormaptext{This colormap is similar to one shipped with Matlab$^\text{\textregistered}$ under a similar name.}

\IfFileExists{tikzlibraryspy.code.tex}{%
\usetikzlibrary{spy}
}{%
    \message{ERROR: tikz SPY library NOT available. The manual will only compile partially.^^J}%
}%

\usepackage{xparse}% for colorbrewer manual

\usetikzlibrary{
    decorations.markings,
    decorations.footprints,
    shapes.arrows,
    matrix,
    positioning,
}

\usepgfplotslibrary{
    fillbetween,
    ternary,
    smithchart,
    patchplots,
    polar,
    colormaps,
    colorbrewer,
    colortol,           % docu not ready yet
}
\pgfqkeys{/codeexample}{%
    every codeexample/.append style={
        /pgfplots/every ternary axis/.append style={
            /pgfplots/legend style={fill=graphicbackground},
        }
    },
    tabsize=4,
}

\pgfplotsmanualenableexternalizationofexpensive

%\usetikzlibrary{external}
%\tikzexternalize[prefix=figures/]{pgfplots}

\title{%
    Manual for Package \PGFPlots{}\\
    {\small 2D/3D Plots in \LaTeX{}, Version \pgfplotsversion}\\
    {\small\url{https://github.com/pgf-tikz/pgfplots}}
    %\\{\small Attention: you are using an unstable development version.}
}

\makeatletter
\long\def\abstractsmuggle{%
    \centering
    \textbf{Abstract}\\[0.5cm]

    \begin{minipage}{12cm}
        \PGFPlots{} draws high-quality function plots in normal or logarithmic
        scaling with a user-friendly interface directly in \TeX{}. The user
        supplies axis labels, legend entries and the plot coordinates for one or
        more plots and \PGFPlots{} applies axis scaling, computes any logarithms
        and axis ticks and draws the plots. It supports line plots, scatter
        plots, piecewise constant plots, bar plots, area plots, mesh and surface
        plots, patch plots, contour plots, quiver plots, histogram plots, box
        plots, polar axes, ternary diagrams, smith charts and some more. It is
        based on Till Tantau's package \PGF{}/\Tikz{}.
    \end{minipage}

}%

\expandafter\date\expandafter{\@date\\[2cm]
    \abstractsmuggle
}%
\makeatother

%\includeonly{pgfplots.reference}


\begin{document}

\def\plotcoords{%
\addplot coordinates {
(5,8.312e-02)    (17,2.547e-02)   (49,7.407e-03)
(129,2.102e-03)  (321,5.874e-04)  (769,1.623e-04)
(1793,4.442e-05) (4097,1.207e-05) (9217,3.261e-06)
};

\addplot coordinates{
(7,8.472e-02)    (31,3.044e-02)    (111,1.022e-02)
(351,3.303e-03)  (1023,1.039e-03)  (2815,3.196e-04)
(7423,9.658e-05) (18943,2.873e-05) (47103,8.437e-06)
};

\addplot coordinates{
(9,7.881e-02)     (49,3.243e-02)    (209,1.232e-02)
(769,4.454e-03)   (2561,1.551e-03)  (7937,5.236e-04)
(23297,1.723e-04) (65537,5.545e-05) (178177,1.751e-05)
};

\addplot coordinates{
(11,6.887e-02)    (71,3.177e-02)     (351,1.341e-02)
(1471,5.334e-03)  (5503,2.027e-03)   (18943,7.415e-04)
(61183,2.628e-04) (187903,9.063e-05) (553983,3.053e-05)
};

\addplot coordinates{
(13,5.755e-02)     (97,2.925e-02)     (545,1.351e-02)
(2561,5.842e-03)   (10625,2.397e-03)  (40193,9.414e-04)
(141569,3.564e-04) (471041,1.308e-04)
(1496065,4.670e-05)
};
}%


\bookmaketitle
\tableofcontents
\include{pgfplots.title_abstract_intro}
\include{pgfplots.preliminaries}
\include{pgfplots.intro}
\include{pgfplots.reference}
\include{pgfplots.libs}
\include{pgfplots.resources}
\include{pgfplots.importexport}
\include{pgfplots.basic.reference}

\printindex

\bibliographystyle{abbrv} %gerapali} %gerabbrv} %gerunsrt.bst} %gerabbrv}% gerplain}
\nocite{pgfplotstable}
\nocite{programmingnotes}
\bibliography{pgfplots}
\end{document}

% -----------------------------------------------------------------------------
% For Stefan Pinnow as reminder on what to look for when editing the manual
% -----------------------------------------------------------------------------
% There should be no line breaks in the following environments
% - |...|
% - \declareandlabel{...}
% - \verbpdfref{...}
% If "MakeTikzPictures" isn't running through check one of the externalized
% LOG files what is the cause of that.
% -----------------------------------------------------------------------------


%%%%%%%%%%%%%%%%%%%%%%%%%%%%%%%%%%%%%%%%%%%%%%%%%%%%%%%%%%%%%%%%%%%%%%%%%%%%%
%
% Package pgfplots.sty documentation.
%
% Copyright 2007/2008 by Christian Feuersaenger.
%
% This program is free software: you can redistribute it and/or modify
% it under the terms of the GNU General Public License as published by
% the Free Software Foundation, either version 3 of the License, or
% (at your option) any later version.
%
% This program is distributed in the hope that it will be useful,
% but WITHOUT ANY WARRANTY; without even the implied warranty of
% MERCHANTABILITY or FITNESS FOR A PARTICULAR PURPOSE.  See the
% GNU General Public License for more details.
%
% You should have received a copy of the GNU General Public License
% along with this program.  If not, see <http://www.gnu.org/licenses/>.
%
%
%%%%%%%%%%%%%%%%%%%%%%%%%%%%%%%%%%%%%%%%%%%%%%%%%%%%%%%%%%%%%%%%%%%%%%%%%%%%%
% SEE pgfplots-macros.tex as well!
%\pdfminorversion=5 % to allow compression
%\pdfobjcompresslevel=2
\documentclass[a4paper,openany]{book}

\let\bookmaketitle=\maketitle

% -----------------------------------
% this here is from ltxdoc:
\usepackage{doc}[=v2]
\AtBeginDocument{\MakeShortVerb{\|}}
%\setlength{\textwidth}{355pt}
%\addtolength\marginparwidth{30pt}
%\addtolength\oddsidemargin{20pt}
%\addtolength\evensidemargin{20pt}
\def\cmd#1{\cs{\expandafter\cmd@to@cs\string#1}}
\def\cmd@to@cs#1#2{\char\number`#2\relax}
\DeclareRobustCommand\cs[1]{\texttt{\char`\\#1}}
\providecommand\marg[1]{%
  {\ttfamily\char`\{}\meta{#1}{\ttfamily\char`\}}}
\providecommand\oarg[1]{%
  {\ttfamily[}\meta{#1}{\ttfamily]}}
\providecommand\parg[1]{%
  {\ttfamily(}\meta{#1}{\ttfamily)}}
\raggedbottom

% -----------------------------------
\input{pgfplots.preamble.tex}

\makeatletter
% I want a two-column index just as in pgfmanual styles. This here
% was the best way to get one:
\def\index@prologue{\section*{Index}\addcontentsline{toc}{chapter}{Index}
}
\makeatother

%\RequirePackage[german,english,francais]{babel}

\def\matlabcolormaptext{This colormap is similar to one shipped with Matlab$^\text{\textregistered}$ under a similar name.}

\IfFileExists{tikzlibraryspy.code.tex}{%
\usetikzlibrary{spy}
}{%
    \message{ERROR: tikz SPY library NOT available. The manual will only compile partially.^^J}%
}%

\usepackage{xparse}% for colorbrewer manual

\usetikzlibrary{
    decorations.markings,
    decorations.footprints,
    shapes.arrows,
    matrix,
    positioning,
}

\usepgfplotslibrary{
    fillbetween,
    ternary,
    smithchart,
    patchplots,
    polar,
    colormaps,
    colorbrewer,
    colortol,           % docu not ready yet
}
\pgfqkeys{/codeexample}{%
    every codeexample/.append style={
        /pgfplots/every ternary axis/.append style={
            /pgfplots/legend style={fill=graphicbackground},
        }
    },
    tabsize=4,
}

\pgfplotsmanualenableexternalizationofexpensive

%\usetikzlibrary{external}
%\tikzexternalize[prefix=figures/]{pgfplots}

\title{%
    Manual for Package \PGFPlots{}\\
    {\small 2D/3D Plots in \LaTeX{}, Version \pgfplotsversion}\\
    {\small\url{https://github.com/pgf-tikz/pgfplots}}
    %\\{\small Attention: you are using an unstable development version.}
}

\makeatletter
\long\def\abstractsmuggle{%
    \centering
    \textbf{Abstract}\\[0.5cm]

    \begin{minipage}{12cm}
        \PGFPlots{} draws high-quality function plots in normal or logarithmic
        scaling with a user-friendly interface directly in \TeX{}. The user
        supplies axis labels, legend entries and the plot coordinates for one or
        more plots and \PGFPlots{} applies axis scaling, computes any logarithms
        and axis ticks and draws the plots. It supports line plots, scatter
        plots, piecewise constant plots, bar plots, area plots, mesh and surface
        plots, patch plots, contour plots, quiver plots, histogram plots, box
        plots, polar axes, ternary diagrams, smith charts and some more. It is
        based on Till Tantau's package \PGF{}/\Tikz{}.
    \end{minipage}

}%

\expandafter\date\expandafter{\@date\\[2cm]
    \abstractsmuggle
}%
\makeatother

%\includeonly{pgfplots.reference}


\begin{document}

\def\plotcoords{%
\addplot coordinates {
(5,8.312e-02)    (17,2.547e-02)   (49,7.407e-03)
(129,2.102e-03)  (321,5.874e-04)  (769,1.623e-04)
(1793,4.442e-05) (4097,1.207e-05) (9217,3.261e-06)
};

\addplot coordinates{
(7,8.472e-02)    (31,3.044e-02)    (111,1.022e-02)
(351,3.303e-03)  (1023,1.039e-03)  (2815,3.196e-04)
(7423,9.658e-05) (18943,2.873e-05) (47103,8.437e-06)
};

\addplot coordinates{
(9,7.881e-02)     (49,3.243e-02)    (209,1.232e-02)
(769,4.454e-03)   (2561,1.551e-03)  (7937,5.236e-04)
(23297,1.723e-04) (65537,5.545e-05) (178177,1.751e-05)
};

\addplot coordinates{
(11,6.887e-02)    (71,3.177e-02)     (351,1.341e-02)
(1471,5.334e-03)  (5503,2.027e-03)   (18943,7.415e-04)
(61183,2.628e-04) (187903,9.063e-05) (553983,3.053e-05)
};

\addplot coordinates{
(13,5.755e-02)     (97,2.925e-02)     (545,1.351e-02)
(2561,5.842e-03)   (10625,2.397e-03)  (40193,9.414e-04)
(141569,3.564e-04) (471041,1.308e-04)
(1496065,4.670e-05)
};
}%


\bookmaketitle
\tableofcontents
\include{pgfplots.title_abstract_intro}
\include{pgfplots.preliminaries}
\include{pgfplots.intro}
\include{pgfplots.reference}
\include{pgfplots.libs}
\include{pgfplots.resources}
\include{pgfplots.importexport}
\include{pgfplots.basic.reference}

\printindex

\bibliographystyle{abbrv} %gerapali} %gerabbrv} %gerunsrt.bst} %gerabbrv}% gerplain}
\nocite{pgfplotstable}
\nocite{programmingnotes}
\bibliography{pgfplots}
\end{document}

% -----------------------------------------------------------------------------
% For Stefan Pinnow as reminder on what to look for when editing the manual
% -----------------------------------------------------------------------------
% There should be no line breaks in the following environments
% - |...|
% - \declareandlabel{...}
% - \verbpdfref{...}
% If "MakeTikzPictures" isn't running through check one of the externalized
% LOG files what is the cause of that.
% -----------------------------------------------------------------------------


%%%%%%%%%%%%%%%%%%%%%%%%%%%%%%%%%%%%%%%%%%%%%%%%%%%%%%%%%%%%%%%%%%%%%%%%%%%%%
%
% Package pgfplots.sty documentation.
%
% Copyright 2007/2008 by Christian Feuersaenger.
%
% This program is free software: you can redistribute it and/or modify
% it under the terms of the GNU General Public License as published by
% the Free Software Foundation, either version 3 of the License, or
% (at your option) any later version.
%
% This program is distributed in the hope that it will be useful,
% but WITHOUT ANY WARRANTY; without even the implied warranty of
% MERCHANTABILITY or FITNESS FOR A PARTICULAR PURPOSE.  See the
% GNU General Public License for more details.
%
% You should have received a copy of the GNU General Public License
% along with this program.  If not, see <http://www.gnu.org/licenses/>.
%
%
%%%%%%%%%%%%%%%%%%%%%%%%%%%%%%%%%%%%%%%%%%%%%%%%%%%%%%%%%%%%%%%%%%%%%%%%%%%%%
% SEE pgfplots-macros.tex as well!
%\pdfminorversion=5 % to allow compression
%\pdfobjcompresslevel=2
\documentclass[a4paper,openany]{book}

\let\bookmaketitle=\maketitle

% -----------------------------------
% this here is from ltxdoc:
\usepackage{doc}[=v2]
\AtBeginDocument{\MakeShortVerb{\|}}
%\setlength{\textwidth}{355pt}
%\addtolength\marginparwidth{30pt}
%\addtolength\oddsidemargin{20pt}
%\addtolength\evensidemargin{20pt}
\def\cmd#1{\cs{\expandafter\cmd@to@cs\string#1}}
\def\cmd@to@cs#1#2{\char\number`#2\relax}
\DeclareRobustCommand\cs[1]{\texttt{\char`\\#1}}
\providecommand\marg[1]{%
  {\ttfamily\char`\{}\meta{#1}{\ttfamily\char`\}}}
\providecommand\oarg[1]{%
  {\ttfamily[}\meta{#1}{\ttfamily]}}
\providecommand\parg[1]{%
  {\ttfamily(}\meta{#1}{\ttfamily)}}
\raggedbottom

% -----------------------------------
\input{pgfplots.preamble.tex}

\makeatletter
% I want a two-column index just as in pgfmanual styles. This here
% was the best way to get one:
\def\index@prologue{\section*{Index}\addcontentsline{toc}{chapter}{Index}
}
\makeatother

%\RequirePackage[german,english,francais]{babel}

\def\matlabcolormaptext{This colormap is similar to one shipped with Matlab$^\text{\textregistered}$ under a similar name.}

\IfFileExists{tikzlibraryspy.code.tex}{%
\usetikzlibrary{spy}
}{%
    \message{ERROR: tikz SPY library NOT available. The manual will only compile partially.^^J}%
}%

\usepackage{xparse}% for colorbrewer manual

\usetikzlibrary{
    decorations.markings,
    decorations.footprints,
    shapes.arrows,
    matrix,
    positioning,
}

\usepgfplotslibrary{
    fillbetween,
    ternary,
    smithchart,
    patchplots,
    polar,
    colormaps,
    colorbrewer,
    colortol,           % docu not ready yet
}
\pgfqkeys{/codeexample}{%
    every codeexample/.append style={
        /pgfplots/every ternary axis/.append style={
            /pgfplots/legend style={fill=graphicbackground},
        }
    },
    tabsize=4,
}

\pgfplotsmanualenableexternalizationofexpensive

%\usetikzlibrary{external}
%\tikzexternalize[prefix=figures/]{pgfplots}

\title{%
    Manual for Package \PGFPlots{}\\
    {\small 2D/3D Plots in \LaTeX{}, Version \pgfplotsversion}\\
    {\small\url{https://github.com/pgf-tikz/pgfplots}}
    %\\{\small Attention: you are using an unstable development version.}
}

\makeatletter
\long\def\abstractsmuggle{%
    \centering
    \textbf{Abstract}\\[0.5cm]

    \begin{minipage}{12cm}
        \PGFPlots{} draws high-quality function plots in normal or logarithmic
        scaling with a user-friendly interface directly in \TeX{}. The user
        supplies axis labels, legend entries and the plot coordinates for one or
        more plots and \PGFPlots{} applies axis scaling, computes any logarithms
        and axis ticks and draws the plots. It supports line plots, scatter
        plots, piecewise constant plots, bar plots, area plots, mesh and surface
        plots, patch plots, contour plots, quiver plots, histogram plots, box
        plots, polar axes, ternary diagrams, smith charts and some more. It is
        based on Till Tantau's package \PGF{}/\Tikz{}.
    \end{minipage}

}%

\expandafter\date\expandafter{\@date\\[2cm]
    \abstractsmuggle
}%
\makeatother

%\includeonly{pgfplots.reference}


\begin{document}

\def\plotcoords{%
\addplot coordinates {
(5,8.312e-02)    (17,2.547e-02)   (49,7.407e-03)
(129,2.102e-03)  (321,5.874e-04)  (769,1.623e-04)
(1793,4.442e-05) (4097,1.207e-05) (9217,3.261e-06)
};

\addplot coordinates{
(7,8.472e-02)    (31,3.044e-02)    (111,1.022e-02)
(351,3.303e-03)  (1023,1.039e-03)  (2815,3.196e-04)
(7423,9.658e-05) (18943,2.873e-05) (47103,8.437e-06)
};

\addplot coordinates{
(9,7.881e-02)     (49,3.243e-02)    (209,1.232e-02)
(769,4.454e-03)   (2561,1.551e-03)  (7937,5.236e-04)
(23297,1.723e-04) (65537,5.545e-05) (178177,1.751e-05)
};

\addplot coordinates{
(11,6.887e-02)    (71,3.177e-02)     (351,1.341e-02)
(1471,5.334e-03)  (5503,2.027e-03)   (18943,7.415e-04)
(61183,2.628e-04) (187903,9.063e-05) (553983,3.053e-05)
};

\addplot coordinates{
(13,5.755e-02)     (97,2.925e-02)     (545,1.351e-02)
(2561,5.842e-03)   (10625,2.397e-03)  (40193,9.414e-04)
(141569,3.564e-04) (471041,1.308e-04)
(1496065,4.670e-05)
};
}%


\bookmaketitle
\tableofcontents
\include{pgfplots.title_abstract_intro}
\include{pgfplots.preliminaries}
\include{pgfplots.intro}
\include{pgfplots.reference}
\include{pgfplots.libs}
\include{pgfplots.resources}
\include{pgfplots.importexport}
\include{pgfplots.basic.reference}

\printindex

\bibliographystyle{abbrv} %gerapali} %gerabbrv} %gerunsrt.bst} %gerabbrv}% gerplain}
\nocite{pgfplotstable}
\nocite{programmingnotes}
\bibliography{pgfplots}
\end{document}

% -----------------------------------------------------------------------------
% For Stefan Pinnow as reminder on what to look for when editing the manual
% -----------------------------------------------------------------------------
% There should be no line breaks in the following environments
% - |...|
% - \declareandlabel{...}
% - \verbpdfref{...}
% If "MakeTikzPictures" isn't running through check one of the externalized
% LOG files what is the cause of that.
% -----------------------------------------------------------------------------


%%%%%%%%%%%%%%%%%%%%%%%%%%%%%%%%%%%%%%%%%%%%%%%%%%%%%%%%%%%%%%%%%%%%%%%%%%%%%
%
% Package pgfplots.sty documentation.
%
% Copyright 2007/2008 by Christian Feuersaenger.
%
% This program is free software: you can redistribute it and/or modify
% it under the terms of the GNU General Public License as published by
% the Free Software Foundation, either version 3 of the License, or
% (at your option) any later version.
%
% This program is distributed in the hope that it will be useful,
% but WITHOUT ANY WARRANTY; without even the implied warranty of
% MERCHANTABILITY or FITNESS FOR A PARTICULAR PURPOSE.  See the
% GNU General Public License for more details.
%
% You should have received a copy of the GNU General Public License
% along with this program.  If not, see <http://www.gnu.org/licenses/>.
%
%
%%%%%%%%%%%%%%%%%%%%%%%%%%%%%%%%%%%%%%%%%%%%%%%%%%%%%%%%%%%%%%%%%%%%%%%%%%%%%
% SEE pgfplots-macros.tex as well!
%\pdfminorversion=5 % to allow compression
%\pdfobjcompresslevel=2
\documentclass[a4paper,openany]{book}

\let\bookmaketitle=\maketitle

% -----------------------------------
% this here is from ltxdoc:
\usepackage{doc}[=v2]
\AtBeginDocument{\MakeShortVerb{\|}}
%\setlength{\textwidth}{355pt}
%\addtolength\marginparwidth{30pt}
%\addtolength\oddsidemargin{20pt}
%\addtolength\evensidemargin{20pt}
\def\cmd#1{\cs{\expandafter\cmd@to@cs\string#1}}
\def\cmd@to@cs#1#2{\char\number`#2\relax}
\DeclareRobustCommand\cs[1]{\texttt{\char`\\#1}}
\providecommand\marg[1]{%
  {\ttfamily\char`\{}\meta{#1}{\ttfamily\char`\}}}
\providecommand\oarg[1]{%
  {\ttfamily[}\meta{#1}{\ttfamily]}}
\providecommand\parg[1]{%
  {\ttfamily(}\meta{#1}{\ttfamily)}}
\raggedbottom

% -----------------------------------
\input{pgfplots.preamble.tex}

\makeatletter
% I want a two-column index just as in pgfmanual styles. This here
% was the best way to get one:
\def\index@prologue{\section*{Index}\addcontentsline{toc}{chapter}{Index}
}
\makeatother

%\RequirePackage[german,english,francais]{babel}

\def\matlabcolormaptext{This colormap is similar to one shipped with Matlab$^\text{\textregistered}$ under a similar name.}

\IfFileExists{tikzlibraryspy.code.tex}{%
\usetikzlibrary{spy}
}{%
    \message{ERROR: tikz SPY library NOT available. The manual will only compile partially.^^J}%
}%

\usepackage{xparse}% for colorbrewer manual

\usetikzlibrary{
    decorations.markings,
    decorations.footprints,
    shapes.arrows,
    matrix,
    positioning,
}

\usepgfplotslibrary{
    fillbetween,
    ternary,
    smithchart,
    patchplots,
    polar,
    colormaps,
    colorbrewer,
    colortol,           % docu not ready yet
}
\pgfqkeys{/codeexample}{%
    every codeexample/.append style={
        /pgfplots/every ternary axis/.append style={
            /pgfplots/legend style={fill=graphicbackground},
        }
    },
    tabsize=4,
}

\pgfplotsmanualenableexternalizationofexpensive

%\usetikzlibrary{external}
%\tikzexternalize[prefix=figures/]{pgfplots}

\title{%
    Manual for Package \PGFPlots{}\\
    {\small 2D/3D Plots in \LaTeX{}, Version \pgfplotsversion}\\
    {\small\url{https://github.com/pgf-tikz/pgfplots}}
    %\\{\small Attention: you are using an unstable development version.}
}

\makeatletter
\long\def\abstractsmuggle{%
    \centering
    \textbf{Abstract}\\[0.5cm]

    \begin{minipage}{12cm}
        \PGFPlots{} draws high-quality function plots in normal or logarithmic
        scaling with a user-friendly interface directly in \TeX{}. The user
        supplies axis labels, legend entries and the plot coordinates for one or
        more plots and \PGFPlots{} applies axis scaling, computes any logarithms
        and axis ticks and draws the plots. It supports line plots, scatter
        plots, piecewise constant plots, bar plots, area plots, mesh and surface
        plots, patch plots, contour plots, quiver plots, histogram plots, box
        plots, polar axes, ternary diagrams, smith charts and some more. It is
        based on Till Tantau's package \PGF{}/\Tikz{}.
    \end{minipage}

}%

\expandafter\date\expandafter{\@date\\[2cm]
    \abstractsmuggle
}%
\makeatother

%\includeonly{pgfplots.reference}


\begin{document}

\def\plotcoords{%
\addplot coordinates {
(5,8.312e-02)    (17,2.547e-02)   (49,7.407e-03)
(129,2.102e-03)  (321,5.874e-04)  (769,1.623e-04)
(1793,4.442e-05) (4097,1.207e-05) (9217,3.261e-06)
};

\addplot coordinates{
(7,8.472e-02)    (31,3.044e-02)    (111,1.022e-02)
(351,3.303e-03)  (1023,1.039e-03)  (2815,3.196e-04)
(7423,9.658e-05) (18943,2.873e-05) (47103,8.437e-06)
};

\addplot coordinates{
(9,7.881e-02)     (49,3.243e-02)    (209,1.232e-02)
(769,4.454e-03)   (2561,1.551e-03)  (7937,5.236e-04)
(23297,1.723e-04) (65537,5.545e-05) (178177,1.751e-05)
};

\addplot coordinates{
(11,6.887e-02)    (71,3.177e-02)     (351,1.341e-02)
(1471,5.334e-03)  (5503,2.027e-03)   (18943,7.415e-04)
(61183,2.628e-04) (187903,9.063e-05) (553983,3.053e-05)
};

\addplot coordinates{
(13,5.755e-02)     (97,2.925e-02)     (545,1.351e-02)
(2561,5.842e-03)   (10625,2.397e-03)  (40193,9.414e-04)
(141569,3.564e-04) (471041,1.308e-04)
(1496065,4.670e-05)
};
}%


\bookmaketitle
\tableofcontents
\include{pgfplots.title_abstract_intro}
\include{pgfplots.preliminaries}
\include{pgfplots.intro}
\include{pgfplots.reference}
\include{pgfplots.libs}
\include{pgfplots.resources}
\include{pgfplots.importexport}
\include{pgfplots.basic.reference}

\printindex

\bibliographystyle{abbrv} %gerapali} %gerabbrv} %gerunsrt.bst} %gerabbrv}% gerplain}
\nocite{pgfplotstable}
\nocite{programmingnotes}
\bibliography{pgfplots}
\end{document}

% -----------------------------------------------------------------------------
% For Stefan Pinnow as reminder on what to look for when editing the manual
% -----------------------------------------------------------------------------
% There should be no line breaks in the following environments
% - |...|
% - \declareandlabel{...}
% - \verbpdfref{...}
% If "MakeTikzPictures" isn't running through check one of the externalized
% LOG files what is the cause of that.
% -----------------------------------------------------------------------------

\include{pgfplots.basic.reference}

\printindex

\bibliographystyle{abbrv} %gerapali} %gerabbrv} %gerunsrt.bst} %gerabbrv}% gerplain}
\nocite{pgfplotstable}
\nocite{programmingnotes}
\bibliography{pgfplots}
\end{document}

% -----------------------------------------------------------------------------
% For Stefan Pinnow as reminder on what to look for when editing the manual
% -----------------------------------------------------------------------------
% There should be no line breaks in the following environments
% - |...|
% - \declareandlabel{...}
% - \verbpdfref{...}
% If "MakeTikzPictures" isn't running through check one of the externalized
% LOG files what is the cause of that.
% -----------------------------------------------------------------------------


%%%%%%%%%%%%%%%%%%%%%%%%%%%%%%%%%%%%%%%%%%%%%%%%%%%%%%%%%%%%%%%%%%%%%%%%%%%%%
%
% Package pgfplots.sty documentation.
%
% Copyright 2007/2008 by Christian Feuersaenger.
%
% This program is free software: you can redistribute it and/or modify
% it under the terms of the GNU General Public License as published by
% the Free Software Foundation, either version 3 of the License, or
% (at your option) any later version.
%
% This program is distributed in the hope that it will be useful,
% but WITHOUT ANY WARRANTY; without even the implied warranty of
% MERCHANTABILITY or FITNESS FOR A PARTICULAR PURPOSE.  See the
% GNU General Public License for more details.
%
% You should have received a copy of the GNU General Public License
% along with this program.  If not, see <http://www.gnu.org/licenses/>.
%
%
%%%%%%%%%%%%%%%%%%%%%%%%%%%%%%%%%%%%%%%%%%%%%%%%%%%%%%%%%%%%%%%%%%%%%%%%%%%%%
% SEE pgfplots-macros.tex as well!
%\pdfminorversion=5 % to allow compression
%\pdfobjcompresslevel=2
\documentclass[a4paper,openany]{book}

\let\bookmaketitle=\maketitle

% -----------------------------------
% this here is from ltxdoc:
\usepackage{doc}[=v2]
\AtBeginDocument{\MakeShortVerb{\|}}
%\setlength{\textwidth}{355pt}
%\addtolength\marginparwidth{30pt}
%\addtolength\oddsidemargin{20pt}
%\addtolength\evensidemargin{20pt}
\def\cmd#1{\cs{\expandafter\cmd@to@cs\string#1}}
\def\cmd@to@cs#1#2{\char\number`#2\relax}
\DeclareRobustCommand\cs[1]{\texttt{\char`\\#1}}
\providecommand\marg[1]{%
  {\ttfamily\char`\{}\meta{#1}{\ttfamily\char`\}}}
\providecommand\oarg[1]{%
  {\ttfamily[}\meta{#1}{\ttfamily]}}
\providecommand\parg[1]{%
  {\ttfamily(}\meta{#1}{\ttfamily)}}
\raggedbottom

% -----------------------------------
\input{pgfplots.preamble.tex}

\makeatletter
% I want a two-column index just as in pgfmanual styles. This here
% was the best way to get one:
\def\index@prologue{\section*{Index}\addcontentsline{toc}{chapter}{Index}
}
\makeatother

%\RequirePackage[german,english,francais]{babel}

\def\matlabcolormaptext{This colormap is similar to one shipped with Matlab$^\text{\textregistered}$ under a similar name.}

\IfFileExists{tikzlibraryspy.code.tex}{%
\usetikzlibrary{spy}
}{%
    \message{ERROR: tikz SPY library NOT available. The manual will only compile partially.^^J}%
}%

\usepackage{xparse}% for colorbrewer manual

\usetikzlibrary{
    decorations.markings,
    decorations.footprints,
    shapes.arrows,
    matrix,
    positioning,
}

\usepgfplotslibrary{
    fillbetween,
    ternary,
    smithchart,
    patchplots,
    polar,
    colormaps,
    colorbrewer,
    colortol,           % docu not ready yet
}
\pgfqkeys{/codeexample}{%
    every codeexample/.append style={
        /pgfplots/every ternary axis/.append style={
            /pgfplots/legend style={fill=graphicbackground},
        }
    },
    tabsize=4,
}

\pgfplotsmanualenableexternalizationofexpensive

%\usetikzlibrary{external}
%\tikzexternalize[prefix=figures/]{pgfplots}

\title{%
    Manual for Package \PGFPlots{}\\
    {\small 2D/3D Plots in \LaTeX{}, Version \pgfplotsversion}\\
    {\small\url{https://github.com/pgf-tikz/pgfplots}}
    %\\{\small Attention: you are using an unstable development version.}
}

\makeatletter
\long\def\abstractsmuggle{%
    \centering
    \textbf{Abstract}\\[0.5cm]

    \begin{minipage}{12cm}
        \PGFPlots{} draws high-quality function plots in normal or logarithmic
        scaling with a user-friendly interface directly in \TeX{}. The user
        supplies axis labels, legend entries and the plot coordinates for one or
        more plots and \PGFPlots{} applies axis scaling, computes any logarithms
        and axis ticks and draws the plots. It supports line plots, scatter
        plots, piecewise constant plots, bar plots, area plots, mesh and surface
        plots, patch plots, contour plots, quiver plots, histogram plots, box
        plots, polar axes, ternary diagrams, smith charts and some more. It is
        based on Till Tantau's package \PGF{}/\Tikz{}.
    \end{minipage}

}%

\expandafter\date\expandafter{\@date\\[2cm]
    \abstractsmuggle
}%
\makeatother

%\includeonly{pgfplots.reference}


\begin{document}

\def\plotcoords{%
\addplot coordinates {
(5,8.312e-02)    (17,2.547e-02)   (49,7.407e-03)
(129,2.102e-03)  (321,5.874e-04)  (769,1.623e-04)
(1793,4.442e-05) (4097,1.207e-05) (9217,3.261e-06)
};

\addplot coordinates{
(7,8.472e-02)    (31,3.044e-02)    (111,1.022e-02)
(351,3.303e-03)  (1023,1.039e-03)  (2815,3.196e-04)
(7423,9.658e-05) (18943,2.873e-05) (47103,8.437e-06)
};

\addplot coordinates{
(9,7.881e-02)     (49,3.243e-02)    (209,1.232e-02)
(769,4.454e-03)   (2561,1.551e-03)  (7937,5.236e-04)
(23297,1.723e-04) (65537,5.545e-05) (178177,1.751e-05)
};

\addplot coordinates{
(11,6.887e-02)    (71,3.177e-02)     (351,1.341e-02)
(1471,5.334e-03)  (5503,2.027e-03)   (18943,7.415e-04)
(61183,2.628e-04) (187903,9.063e-05) (553983,3.053e-05)
};

\addplot coordinates{
(13,5.755e-02)     (97,2.925e-02)     (545,1.351e-02)
(2561,5.842e-03)   (10625,2.397e-03)  (40193,9.414e-04)
(141569,3.564e-04) (471041,1.308e-04)
(1496065,4.670e-05)
};
}%


\bookmaketitle
\tableofcontents

%%%%%%%%%%%%%%%%%%%%%%%%%%%%%%%%%%%%%%%%%%%%%%%%%%%%%%%%%%%%%%%%%%%%%%%%%%%%%
%
% Package pgfplots.sty documentation.
%
% Copyright 2007/2008 by Christian Feuersaenger.
%
% This program is free software: you can redistribute it and/or modify
% it under the terms of the GNU General Public License as published by
% the Free Software Foundation, either version 3 of the License, or
% (at your option) any later version.
%
% This program is distributed in the hope that it will be useful,
% but WITHOUT ANY WARRANTY; without even the implied warranty of
% MERCHANTABILITY or FITNESS FOR A PARTICULAR PURPOSE.  See the
% GNU General Public License for more details.
%
% You should have received a copy of the GNU General Public License
% along with this program.  If not, see <http://www.gnu.org/licenses/>.
%
%
%%%%%%%%%%%%%%%%%%%%%%%%%%%%%%%%%%%%%%%%%%%%%%%%%%%%%%%%%%%%%%%%%%%%%%%%%%%%%
% SEE pgfplots-macros.tex as well!
%\pdfminorversion=5 % to allow compression
%\pdfobjcompresslevel=2
\documentclass[a4paper,openany]{book}

\let\bookmaketitle=\maketitle

% -----------------------------------
% this here is from ltxdoc:
\usepackage{doc}[=v2]
\AtBeginDocument{\MakeShortVerb{\|}}
%\setlength{\textwidth}{355pt}
%\addtolength\marginparwidth{30pt}
%\addtolength\oddsidemargin{20pt}
%\addtolength\evensidemargin{20pt}
\def\cmd#1{\cs{\expandafter\cmd@to@cs\string#1}}
\def\cmd@to@cs#1#2{\char\number`#2\relax}
\DeclareRobustCommand\cs[1]{\texttt{\char`\\#1}}
\providecommand\marg[1]{%
  {\ttfamily\char`\{}\meta{#1}{\ttfamily\char`\}}}
\providecommand\oarg[1]{%
  {\ttfamily[}\meta{#1}{\ttfamily]}}
\providecommand\parg[1]{%
  {\ttfamily(}\meta{#1}{\ttfamily)}}
\raggedbottom

% -----------------------------------
\input{pgfplots.preamble.tex}

\makeatletter
% I want a two-column index just as in pgfmanual styles. This here
% was the best way to get one:
\def\index@prologue{\section*{Index}\addcontentsline{toc}{chapter}{Index}
}
\makeatother

%\RequirePackage[german,english,francais]{babel}

\def\matlabcolormaptext{This colormap is similar to one shipped with Matlab$^\text{\textregistered}$ under a similar name.}

\IfFileExists{tikzlibraryspy.code.tex}{%
\usetikzlibrary{spy}
}{%
    \message{ERROR: tikz SPY library NOT available. The manual will only compile partially.^^J}%
}%

\usepackage{xparse}% for colorbrewer manual

\usetikzlibrary{
    decorations.markings,
    decorations.footprints,
    shapes.arrows,
    matrix,
    positioning,
}

\usepgfplotslibrary{
    fillbetween,
    ternary,
    smithchart,
    patchplots,
    polar,
    colormaps,
    colorbrewer,
    colortol,           % docu not ready yet
}
\pgfqkeys{/codeexample}{%
    every codeexample/.append style={
        /pgfplots/every ternary axis/.append style={
            /pgfplots/legend style={fill=graphicbackground},
        }
    },
    tabsize=4,
}

\pgfplotsmanualenableexternalizationofexpensive

%\usetikzlibrary{external}
%\tikzexternalize[prefix=figures/]{pgfplots}

\title{%
    Manual for Package \PGFPlots{}\\
    {\small 2D/3D Plots in \LaTeX{}, Version \pgfplotsversion}\\
    {\small\url{https://github.com/pgf-tikz/pgfplots}}
    %\\{\small Attention: you are using an unstable development version.}
}

\makeatletter
\long\def\abstractsmuggle{%
    \centering
    \textbf{Abstract}\\[0.5cm]

    \begin{minipage}{12cm}
        \PGFPlots{} draws high-quality function plots in normal or logarithmic
        scaling with a user-friendly interface directly in \TeX{}. The user
        supplies axis labels, legend entries and the plot coordinates for one or
        more plots and \PGFPlots{} applies axis scaling, computes any logarithms
        and axis ticks and draws the plots. It supports line plots, scatter
        plots, piecewise constant plots, bar plots, area plots, mesh and surface
        plots, patch plots, contour plots, quiver plots, histogram plots, box
        plots, polar axes, ternary diagrams, smith charts and some more. It is
        based on Till Tantau's package \PGF{}/\Tikz{}.
    \end{minipage}

}%

\expandafter\date\expandafter{\@date\\[2cm]
    \abstractsmuggle
}%
\makeatother

%\includeonly{pgfplots.reference}


\begin{document}

\def\plotcoords{%
\addplot coordinates {
(5,8.312e-02)    (17,2.547e-02)   (49,7.407e-03)
(129,2.102e-03)  (321,5.874e-04)  (769,1.623e-04)
(1793,4.442e-05) (4097,1.207e-05) (9217,3.261e-06)
};

\addplot coordinates{
(7,8.472e-02)    (31,3.044e-02)    (111,1.022e-02)
(351,3.303e-03)  (1023,1.039e-03)  (2815,3.196e-04)
(7423,9.658e-05) (18943,2.873e-05) (47103,8.437e-06)
};

\addplot coordinates{
(9,7.881e-02)     (49,3.243e-02)    (209,1.232e-02)
(769,4.454e-03)   (2561,1.551e-03)  (7937,5.236e-04)
(23297,1.723e-04) (65537,5.545e-05) (178177,1.751e-05)
};

\addplot coordinates{
(11,6.887e-02)    (71,3.177e-02)     (351,1.341e-02)
(1471,5.334e-03)  (5503,2.027e-03)   (18943,7.415e-04)
(61183,2.628e-04) (187903,9.063e-05) (553983,3.053e-05)
};

\addplot coordinates{
(13,5.755e-02)     (97,2.925e-02)     (545,1.351e-02)
(2561,5.842e-03)   (10625,2.397e-03)  (40193,9.414e-04)
(141569,3.564e-04) (471041,1.308e-04)
(1496065,4.670e-05)
};
}%


\bookmaketitle
\tableofcontents
\include{pgfplots.title_abstract_intro}
\include{pgfplots.preliminaries}
\include{pgfplots.intro}
\include{pgfplots.reference}
\include{pgfplots.libs}
\include{pgfplots.resources}
\include{pgfplots.importexport}
\include{pgfplots.basic.reference}

\printindex

\bibliographystyle{abbrv} %gerapali} %gerabbrv} %gerunsrt.bst} %gerabbrv}% gerplain}
\nocite{pgfplotstable}
\nocite{programmingnotes}
\bibliography{pgfplots}
\end{document}

% -----------------------------------------------------------------------------
% For Stefan Pinnow as reminder on what to look for when editing the manual
% -----------------------------------------------------------------------------
% There should be no line breaks in the following environments
% - |...|
% - \declareandlabel{...}
% - \verbpdfref{...}
% If "MakeTikzPictures" isn't running through check one of the externalized
% LOG files what is the cause of that.
% -----------------------------------------------------------------------------


%%%%%%%%%%%%%%%%%%%%%%%%%%%%%%%%%%%%%%%%%%%%%%%%%%%%%%%%%%%%%%%%%%%%%%%%%%%%%
%
% Package pgfplots.sty documentation.
%
% Copyright 2007/2008 by Christian Feuersaenger.
%
% This program is free software: you can redistribute it and/or modify
% it under the terms of the GNU General Public License as published by
% the Free Software Foundation, either version 3 of the License, or
% (at your option) any later version.
%
% This program is distributed in the hope that it will be useful,
% but WITHOUT ANY WARRANTY; without even the implied warranty of
% MERCHANTABILITY or FITNESS FOR A PARTICULAR PURPOSE.  See the
% GNU General Public License for more details.
%
% You should have received a copy of the GNU General Public License
% along with this program.  If not, see <http://www.gnu.org/licenses/>.
%
%
%%%%%%%%%%%%%%%%%%%%%%%%%%%%%%%%%%%%%%%%%%%%%%%%%%%%%%%%%%%%%%%%%%%%%%%%%%%%%
% SEE pgfplots-macros.tex as well!
%\pdfminorversion=5 % to allow compression
%\pdfobjcompresslevel=2
\documentclass[a4paper,openany]{book}

\let\bookmaketitle=\maketitle

% -----------------------------------
% this here is from ltxdoc:
\usepackage{doc}[=v2]
\AtBeginDocument{\MakeShortVerb{\|}}
%\setlength{\textwidth}{355pt}
%\addtolength\marginparwidth{30pt}
%\addtolength\oddsidemargin{20pt}
%\addtolength\evensidemargin{20pt}
\def\cmd#1{\cs{\expandafter\cmd@to@cs\string#1}}
\def\cmd@to@cs#1#2{\char\number`#2\relax}
\DeclareRobustCommand\cs[1]{\texttt{\char`\\#1}}
\providecommand\marg[1]{%
  {\ttfamily\char`\{}\meta{#1}{\ttfamily\char`\}}}
\providecommand\oarg[1]{%
  {\ttfamily[}\meta{#1}{\ttfamily]}}
\providecommand\parg[1]{%
  {\ttfamily(}\meta{#1}{\ttfamily)}}
\raggedbottom

% -----------------------------------
\input{pgfplots.preamble.tex}

\makeatletter
% I want a two-column index just as in pgfmanual styles. This here
% was the best way to get one:
\def\index@prologue{\section*{Index}\addcontentsline{toc}{chapter}{Index}
}
\makeatother

%\RequirePackage[german,english,francais]{babel}

\def\matlabcolormaptext{This colormap is similar to one shipped with Matlab$^\text{\textregistered}$ under a similar name.}

\IfFileExists{tikzlibraryspy.code.tex}{%
\usetikzlibrary{spy}
}{%
    \message{ERROR: tikz SPY library NOT available. The manual will only compile partially.^^J}%
}%

\usepackage{xparse}% for colorbrewer manual

\usetikzlibrary{
    decorations.markings,
    decorations.footprints,
    shapes.arrows,
    matrix,
    positioning,
}

\usepgfplotslibrary{
    fillbetween,
    ternary,
    smithchart,
    patchplots,
    polar,
    colormaps,
    colorbrewer,
    colortol,           % docu not ready yet
}
\pgfqkeys{/codeexample}{%
    every codeexample/.append style={
        /pgfplots/every ternary axis/.append style={
            /pgfplots/legend style={fill=graphicbackground},
        }
    },
    tabsize=4,
}

\pgfplotsmanualenableexternalizationofexpensive

%\usetikzlibrary{external}
%\tikzexternalize[prefix=figures/]{pgfplots}

\title{%
    Manual for Package \PGFPlots{}\\
    {\small 2D/3D Plots in \LaTeX{}, Version \pgfplotsversion}\\
    {\small\url{https://github.com/pgf-tikz/pgfplots}}
    %\\{\small Attention: you are using an unstable development version.}
}

\makeatletter
\long\def\abstractsmuggle{%
    \centering
    \textbf{Abstract}\\[0.5cm]

    \begin{minipage}{12cm}
        \PGFPlots{} draws high-quality function plots in normal or logarithmic
        scaling with a user-friendly interface directly in \TeX{}. The user
        supplies axis labels, legend entries and the plot coordinates for one or
        more plots and \PGFPlots{} applies axis scaling, computes any logarithms
        and axis ticks and draws the plots. It supports line plots, scatter
        plots, piecewise constant plots, bar plots, area plots, mesh and surface
        plots, patch plots, contour plots, quiver plots, histogram plots, box
        plots, polar axes, ternary diagrams, smith charts and some more. It is
        based on Till Tantau's package \PGF{}/\Tikz{}.
    \end{minipage}

}%

\expandafter\date\expandafter{\@date\\[2cm]
    \abstractsmuggle
}%
\makeatother

%\includeonly{pgfplots.reference}


\begin{document}

\def\plotcoords{%
\addplot coordinates {
(5,8.312e-02)    (17,2.547e-02)   (49,7.407e-03)
(129,2.102e-03)  (321,5.874e-04)  (769,1.623e-04)
(1793,4.442e-05) (4097,1.207e-05) (9217,3.261e-06)
};

\addplot coordinates{
(7,8.472e-02)    (31,3.044e-02)    (111,1.022e-02)
(351,3.303e-03)  (1023,1.039e-03)  (2815,3.196e-04)
(7423,9.658e-05) (18943,2.873e-05) (47103,8.437e-06)
};

\addplot coordinates{
(9,7.881e-02)     (49,3.243e-02)    (209,1.232e-02)
(769,4.454e-03)   (2561,1.551e-03)  (7937,5.236e-04)
(23297,1.723e-04) (65537,5.545e-05) (178177,1.751e-05)
};

\addplot coordinates{
(11,6.887e-02)    (71,3.177e-02)     (351,1.341e-02)
(1471,5.334e-03)  (5503,2.027e-03)   (18943,7.415e-04)
(61183,2.628e-04) (187903,9.063e-05) (553983,3.053e-05)
};

\addplot coordinates{
(13,5.755e-02)     (97,2.925e-02)     (545,1.351e-02)
(2561,5.842e-03)   (10625,2.397e-03)  (40193,9.414e-04)
(141569,3.564e-04) (471041,1.308e-04)
(1496065,4.670e-05)
};
}%


\bookmaketitle
\tableofcontents
\include{pgfplots.title_abstract_intro}
\include{pgfplots.preliminaries}
\include{pgfplots.intro}
\include{pgfplots.reference}
\include{pgfplots.libs}
\include{pgfplots.resources}
\include{pgfplots.importexport}
\include{pgfplots.basic.reference}

\printindex

\bibliographystyle{abbrv} %gerapali} %gerabbrv} %gerunsrt.bst} %gerabbrv}% gerplain}
\nocite{pgfplotstable}
\nocite{programmingnotes}
\bibliography{pgfplots}
\end{document}

% -----------------------------------------------------------------------------
% For Stefan Pinnow as reminder on what to look for when editing the manual
% -----------------------------------------------------------------------------
% There should be no line breaks in the following environments
% - |...|
% - \declareandlabel{...}
% - \verbpdfref{...}
% If "MakeTikzPictures" isn't running through check one of the externalized
% LOG files what is the cause of that.
% -----------------------------------------------------------------------------


%%%%%%%%%%%%%%%%%%%%%%%%%%%%%%%%%%%%%%%%%%%%%%%%%%%%%%%%%%%%%%%%%%%%%%%%%%%%%
%
% Package pgfplots.sty documentation.
%
% Copyright 2007/2008 by Christian Feuersaenger.
%
% This program is free software: you can redistribute it and/or modify
% it under the terms of the GNU General Public License as published by
% the Free Software Foundation, either version 3 of the License, or
% (at your option) any later version.
%
% This program is distributed in the hope that it will be useful,
% but WITHOUT ANY WARRANTY; without even the implied warranty of
% MERCHANTABILITY or FITNESS FOR A PARTICULAR PURPOSE.  See the
% GNU General Public License for more details.
%
% You should have received a copy of the GNU General Public License
% along with this program.  If not, see <http://www.gnu.org/licenses/>.
%
%
%%%%%%%%%%%%%%%%%%%%%%%%%%%%%%%%%%%%%%%%%%%%%%%%%%%%%%%%%%%%%%%%%%%%%%%%%%%%%
% SEE pgfplots-macros.tex as well!
%\pdfminorversion=5 % to allow compression
%\pdfobjcompresslevel=2
\documentclass[a4paper,openany]{book}

\let\bookmaketitle=\maketitle

% -----------------------------------
% this here is from ltxdoc:
\usepackage{doc}[=v2]
\AtBeginDocument{\MakeShortVerb{\|}}
%\setlength{\textwidth}{355pt}
%\addtolength\marginparwidth{30pt}
%\addtolength\oddsidemargin{20pt}
%\addtolength\evensidemargin{20pt}
\def\cmd#1{\cs{\expandafter\cmd@to@cs\string#1}}
\def\cmd@to@cs#1#2{\char\number`#2\relax}
\DeclareRobustCommand\cs[1]{\texttt{\char`\\#1}}
\providecommand\marg[1]{%
  {\ttfamily\char`\{}\meta{#1}{\ttfamily\char`\}}}
\providecommand\oarg[1]{%
  {\ttfamily[}\meta{#1}{\ttfamily]}}
\providecommand\parg[1]{%
  {\ttfamily(}\meta{#1}{\ttfamily)}}
\raggedbottom

% -----------------------------------
\input{pgfplots.preamble.tex}

\makeatletter
% I want a two-column index just as in pgfmanual styles. This here
% was the best way to get one:
\def\index@prologue{\section*{Index}\addcontentsline{toc}{chapter}{Index}
}
\makeatother

%\RequirePackage[german,english,francais]{babel}

\def\matlabcolormaptext{This colormap is similar to one shipped with Matlab$^\text{\textregistered}$ under a similar name.}

\IfFileExists{tikzlibraryspy.code.tex}{%
\usetikzlibrary{spy}
}{%
    \message{ERROR: tikz SPY library NOT available. The manual will only compile partially.^^J}%
}%

\usepackage{xparse}% for colorbrewer manual

\usetikzlibrary{
    decorations.markings,
    decorations.footprints,
    shapes.arrows,
    matrix,
    positioning,
}

\usepgfplotslibrary{
    fillbetween,
    ternary,
    smithchart,
    patchplots,
    polar,
    colormaps,
    colorbrewer,
    colortol,           % docu not ready yet
}
\pgfqkeys{/codeexample}{%
    every codeexample/.append style={
        /pgfplots/every ternary axis/.append style={
            /pgfplots/legend style={fill=graphicbackground},
        }
    },
    tabsize=4,
}

\pgfplotsmanualenableexternalizationofexpensive

%\usetikzlibrary{external}
%\tikzexternalize[prefix=figures/]{pgfplots}

\title{%
    Manual for Package \PGFPlots{}\\
    {\small 2D/3D Plots in \LaTeX{}, Version \pgfplotsversion}\\
    {\small\url{https://github.com/pgf-tikz/pgfplots}}
    %\\{\small Attention: you are using an unstable development version.}
}

\makeatletter
\long\def\abstractsmuggle{%
    \centering
    \textbf{Abstract}\\[0.5cm]

    \begin{minipage}{12cm}
        \PGFPlots{} draws high-quality function plots in normal or logarithmic
        scaling with a user-friendly interface directly in \TeX{}. The user
        supplies axis labels, legend entries and the plot coordinates for one or
        more plots and \PGFPlots{} applies axis scaling, computes any logarithms
        and axis ticks and draws the plots. It supports line plots, scatter
        plots, piecewise constant plots, bar plots, area plots, mesh and surface
        plots, patch plots, contour plots, quiver plots, histogram plots, box
        plots, polar axes, ternary diagrams, smith charts and some more. It is
        based on Till Tantau's package \PGF{}/\Tikz{}.
    \end{minipage}

}%

\expandafter\date\expandafter{\@date\\[2cm]
    \abstractsmuggle
}%
\makeatother

%\includeonly{pgfplots.reference}


\begin{document}

\def\plotcoords{%
\addplot coordinates {
(5,8.312e-02)    (17,2.547e-02)   (49,7.407e-03)
(129,2.102e-03)  (321,5.874e-04)  (769,1.623e-04)
(1793,4.442e-05) (4097,1.207e-05) (9217,3.261e-06)
};

\addplot coordinates{
(7,8.472e-02)    (31,3.044e-02)    (111,1.022e-02)
(351,3.303e-03)  (1023,1.039e-03)  (2815,3.196e-04)
(7423,9.658e-05) (18943,2.873e-05) (47103,8.437e-06)
};

\addplot coordinates{
(9,7.881e-02)     (49,3.243e-02)    (209,1.232e-02)
(769,4.454e-03)   (2561,1.551e-03)  (7937,5.236e-04)
(23297,1.723e-04) (65537,5.545e-05) (178177,1.751e-05)
};

\addplot coordinates{
(11,6.887e-02)    (71,3.177e-02)     (351,1.341e-02)
(1471,5.334e-03)  (5503,2.027e-03)   (18943,7.415e-04)
(61183,2.628e-04) (187903,9.063e-05) (553983,3.053e-05)
};

\addplot coordinates{
(13,5.755e-02)     (97,2.925e-02)     (545,1.351e-02)
(2561,5.842e-03)   (10625,2.397e-03)  (40193,9.414e-04)
(141569,3.564e-04) (471041,1.308e-04)
(1496065,4.670e-05)
};
}%


\bookmaketitle
\tableofcontents
\include{pgfplots.title_abstract_intro}
\include{pgfplots.preliminaries}
\include{pgfplots.intro}
\include{pgfplots.reference}
\include{pgfplots.libs}
\include{pgfplots.resources}
\include{pgfplots.importexport}
\include{pgfplots.basic.reference}

\printindex

\bibliographystyle{abbrv} %gerapali} %gerabbrv} %gerunsrt.bst} %gerabbrv}% gerplain}
\nocite{pgfplotstable}
\nocite{programmingnotes}
\bibliography{pgfplots}
\end{document}

% -----------------------------------------------------------------------------
% For Stefan Pinnow as reminder on what to look for when editing the manual
% -----------------------------------------------------------------------------
% There should be no line breaks in the following environments
% - |...|
% - \declareandlabel{...}
% - \verbpdfref{...}
% If "MakeTikzPictures" isn't running through check one of the externalized
% LOG files what is the cause of that.
% -----------------------------------------------------------------------------


%%%%%%%%%%%%%%%%%%%%%%%%%%%%%%%%%%%%%%%%%%%%%%%%%%%%%%%%%%%%%%%%%%%%%%%%%%%%%
%
% Package pgfplots.sty documentation.
%
% Copyright 2007/2008 by Christian Feuersaenger.
%
% This program is free software: you can redistribute it and/or modify
% it under the terms of the GNU General Public License as published by
% the Free Software Foundation, either version 3 of the License, or
% (at your option) any later version.
%
% This program is distributed in the hope that it will be useful,
% but WITHOUT ANY WARRANTY; without even the implied warranty of
% MERCHANTABILITY or FITNESS FOR A PARTICULAR PURPOSE.  See the
% GNU General Public License for more details.
%
% You should have received a copy of the GNU General Public License
% along with this program.  If not, see <http://www.gnu.org/licenses/>.
%
%
%%%%%%%%%%%%%%%%%%%%%%%%%%%%%%%%%%%%%%%%%%%%%%%%%%%%%%%%%%%%%%%%%%%%%%%%%%%%%
% SEE pgfplots-macros.tex as well!
%\pdfminorversion=5 % to allow compression
%\pdfobjcompresslevel=2
\documentclass[a4paper,openany]{book}

\let\bookmaketitle=\maketitle

% -----------------------------------
% this here is from ltxdoc:
\usepackage{doc}[=v2]
\AtBeginDocument{\MakeShortVerb{\|}}
%\setlength{\textwidth}{355pt}
%\addtolength\marginparwidth{30pt}
%\addtolength\oddsidemargin{20pt}
%\addtolength\evensidemargin{20pt}
\def\cmd#1{\cs{\expandafter\cmd@to@cs\string#1}}
\def\cmd@to@cs#1#2{\char\number`#2\relax}
\DeclareRobustCommand\cs[1]{\texttt{\char`\\#1}}
\providecommand\marg[1]{%
  {\ttfamily\char`\{}\meta{#1}{\ttfamily\char`\}}}
\providecommand\oarg[1]{%
  {\ttfamily[}\meta{#1}{\ttfamily]}}
\providecommand\parg[1]{%
  {\ttfamily(}\meta{#1}{\ttfamily)}}
\raggedbottom

% -----------------------------------
\input{pgfplots.preamble.tex}

\makeatletter
% I want a two-column index just as in pgfmanual styles. This here
% was the best way to get one:
\def\index@prologue{\section*{Index}\addcontentsline{toc}{chapter}{Index}
}
\makeatother

%\RequirePackage[german,english,francais]{babel}

\def\matlabcolormaptext{This colormap is similar to one shipped with Matlab$^\text{\textregistered}$ under a similar name.}

\IfFileExists{tikzlibraryspy.code.tex}{%
\usetikzlibrary{spy}
}{%
    \message{ERROR: tikz SPY library NOT available. The manual will only compile partially.^^J}%
}%

\usepackage{xparse}% for colorbrewer manual

\usetikzlibrary{
    decorations.markings,
    decorations.footprints,
    shapes.arrows,
    matrix,
    positioning,
}

\usepgfplotslibrary{
    fillbetween,
    ternary,
    smithchart,
    patchplots,
    polar,
    colormaps,
    colorbrewer,
    colortol,           % docu not ready yet
}
\pgfqkeys{/codeexample}{%
    every codeexample/.append style={
        /pgfplots/every ternary axis/.append style={
            /pgfplots/legend style={fill=graphicbackground},
        }
    },
    tabsize=4,
}

\pgfplotsmanualenableexternalizationofexpensive

%\usetikzlibrary{external}
%\tikzexternalize[prefix=figures/]{pgfplots}

\title{%
    Manual for Package \PGFPlots{}\\
    {\small 2D/3D Plots in \LaTeX{}, Version \pgfplotsversion}\\
    {\small\url{https://github.com/pgf-tikz/pgfplots}}
    %\\{\small Attention: you are using an unstable development version.}
}

\makeatletter
\long\def\abstractsmuggle{%
    \centering
    \textbf{Abstract}\\[0.5cm]

    \begin{minipage}{12cm}
        \PGFPlots{} draws high-quality function plots in normal or logarithmic
        scaling with a user-friendly interface directly in \TeX{}. The user
        supplies axis labels, legend entries and the plot coordinates for one or
        more plots and \PGFPlots{} applies axis scaling, computes any logarithms
        and axis ticks and draws the plots. It supports line plots, scatter
        plots, piecewise constant plots, bar plots, area plots, mesh and surface
        plots, patch plots, contour plots, quiver plots, histogram plots, box
        plots, polar axes, ternary diagrams, smith charts and some more. It is
        based on Till Tantau's package \PGF{}/\Tikz{}.
    \end{minipage}

}%

\expandafter\date\expandafter{\@date\\[2cm]
    \abstractsmuggle
}%
\makeatother

%\includeonly{pgfplots.reference}


\begin{document}

\def\plotcoords{%
\addplot coordinates {
(5,8.312e-02)    (17,2.547e-02)   (49,7.407e-03)
(129,2.102e-03)  (321,5.874e-04)  (769,1.623e-04)
(1793,4.442e-05) (4097,1.207e-05) (9217,3.261e-06)
};

\addplot coordinates{
(7,8.472e-02)    (31,3.044e-02)    (111,1.022e-02)
(351,3.303e-03)  (1023,1.039e-03)  (2815,3.196e-04)
(7423,9.658e-05) (18943,2.873e-05) (47103,8.437e-06)
};

\addplot coordinates{
(9,7.881e-02)     (49,3.243e-02)    (209,1.232e-02)
(769,4.454e-03)   (2561,1.551e-03)  (7937,5.236e-04)
(23297,1.723e-04) (65537,5.545e-05) (178177,1.751e-05)
};

\addplot coordinates{
(11,6.887e-02)    (71,3.177e-02)     (351,1.341e-02)
(1471,5.334e-03)  (5503,2.027e-03)   (18943,7.415e-04)
(61183,2.628e-04) (187903,9.063e-05) (553983,3.053e-05)
};

\addplot coordinates{
(13,5.755e-02)     (97,2.925e-02)     (545,1.351e-02)
(2561,5.842e-03)   (10625,2.397e-03)  (40193,9.414e-04)
(141569,3.564e-04) (471041,1.308e-04)
(1496065,4.670e-05)
};
}%


\bookmaketitle
\tableofcontents
\include{pgfplots.title_abstract_intro}
\include{pgfplots.preliminaries}
\include{pgfplots.intro}
\include{pgfplots.reference}
\include{pgfplots.libs}
\include{pgfplots.resources}
\include{pgfplots.importexport}
\include{pgfplots.basic.reference}

\printindex

\bibliographystyle{abbrv} %gerapali} %gerabbrv} %gerunsrt.bst} %gerabbrv}% gerplain}
\nocite{pgfplotstable}
\nocite{programmingnotes}
\bibliography{pgfplots}
\end{document}

% -----------------------------------------------------------------------------
% For Stefan Pinnow as reminder on what to look for when editing the manual
% -----------------------------------------------------------------------------
% There should be no line breaks in the following environments
% - |...|
% - \declareandlabel{...}
% - \verbpdfref{...}
% If "MakeTikzPictures" isn't running through check one of the externalized
% LOG files what is the cause of that.
% -----------------------------------------------------------------------------


%%%%%%%%%%%%%%%%%%%%%%%%%%%%%%%%%%%%%%%%%%%%%%%%%%%%%%%%%%%%%%%%%%%%%%%%%%%%%
%
% Package pgfplots.sty documentation.
%
% Copyright 2007/2008 by Christian Feuersaenger.
%
% This program is free software: you can redistribute it and/or modify
% it under the terms of the GNU General Public License as published by
% the Free Software Foundation, either version 3 of the License, or
% (at your option) any later version.
%
% This program is distributed in the hope that it will be useful,
% but WITHOUT ANY WARRANTY; without even the implied warranty of
% MERCHANTABILITY or FITNESS FOR A PARTICULAR PURPOSE.  See the
% GNU General Public License for more details.
%
% You should have received a copy of the GNU General Public License
% along with this program.  If not, see <http://www.gnu.org/licenses/>.
%
%
%%%%%%%%%%%%%%%%%%%%%%%%%%%%%%%%%%%%%%%%%%%%%%%%%%%%%%%%%%%%%%%%%%%%%%%%%%%%%
% SEE pgfplots-macros.tex as well!
%\pdfminorversion=5 % to allow compression
%\pdfobjcompresslevel=2
\documentclass[a4paper,openany]{book}

\let\bookmaketitle=\maketitle

% -----------------------------------
% this here is from ltxdoc:
\usepackage{doc}[=v2]
\AtBeginDocument{\MakeShortVerb{\|}}
%\setlength{\textwidth}{355pt}
%\addtolength\marginparwidth{30pt}
%\addtolength\oddsidemargin{20pt}
%\addtolength\evensidemargin{20pt}
\def\cmd#1{\cs{\expandafter\cmd@to@cs\string#1}}
\def\cmd@to@cs#1#2{\char\number`#2\relax}
\DeclareRobustCommand\cs[1]{\texttt{\char`\\#1}}
\providecommand\marg[1]{%
  {\ttfamily\char`\{}\meta{#1}{\ttfamily\char`\}}}
\providecommand\oarg[1]{%
  {\ttfamily[}\meta{#1}{\ttfamily]}}
\providecommand\parg[1]{%
  {\ttfamily(}\meta{#1}{\ttfamily)}}
\raggedbottom

% -----------------------------------
\input{pgfplots.preamble.tex}

\makeatletter
% I want a two-column index just as in pgfmanual styles. This here
% was the best way to get one:
\def\index@prologue{\section*{Index}\addcontentsline{toc}{chapter}{Index}
}
\makeatother

%\RequirePackage[german,english,francais]{babel}

\def\matlabcolormaptext{This colormap is similar to one shipped with Matlab$^\text{\textregistered}$ under a similar name.}

\IfFileExists{tikzlibraryspy.code.tex}{%
\usetikzlibrary{spy}
}{%
    \message{ERROR: tikz SPY library NOT available. The manual will only compile partially.^^J}%
}%

\usepackage{xparse}% for colorbrewer manual

\usetikzlibrary{
    decorations.markings,
    decorations.footprints,
    shapes.arrows,
    matrix,
    positioning,
}

\usepgfplotslibrary{
    fillbetween,
    ternary,
    smithchart,
    patchplots,
    polar,
    colormaps,
    colorbrewer,
    colortol,           % docu not ready yet
}
\pgfqkeys{/codeexample}{%
    every codeexample/.append style={
        /pgfplots/every ternary axis/.append style={
            /pgfplots/legend style={fill=graphicbackground},
        }
    },
    tabsize=4,
}

\pgfplotsmanualenableexternalizationofexpensive

%\usetikzlibrary{external}
%\tikzexternalize[prefix=figures/]{pgfplots}

\title{%
    Manual for Package \PGFPlots{}\\
    {\small 2D/3D Plots in \LaTeX{}, Version \pgfplotsversion}\\
    {\small\url{https://github.com/pgf-tikz/pgfplots}}
    %\\{\small Attention: you are using an unstable development version.}
}

\makeatletter
\long\def\abstractsmuggle{%
    \centering
    \textbf{Abstract}\\[0.5cm]

    \begin{minipage}{12cm}
        \PGFPlots{} draws high-quality function plots in normal or logarithmic
        scaling with a user-friendly interface directly in \TeX{}. The user
        supplies axis labels, legend entries and the plot coordinates for one or
        more plots and \PGFPlots{} applies axis scaling, computes any logarithms
        and axis ticks and draws the plots. It supports line plots, scatter
        plots, piecewise constant plots, bar plots, area plots, mesh and surface
        plots, patch plots, contour plots, quiver plots, histogram plots, box
        plots, polar axes, ternary diagrams, smith charts and some more. It is
        based on Till Tantau's package \PGF{}/\Tikz{}.
    \end{minipage}

}%

\expandafter\date\expandafter{\@date\\[2cm]
    \abstractsmuggle
}%
\makeatother

%\includeonly{pgfplots.reference}


\begin{document}

\def\plotcoords{%
\addplot coordinates {
(5,8.312e-02)    (17,2.547e-02)   (49,7.407e-03)
(129,2.102e-03)  (321,5.874e-04)  (769,1.623e-04)
(1793,4.442e-05) (4097,1.207e-05) (9217,3.261e-06)
};

\addplot coordinates{
(7,8.472e-02)    (31,3.044e-02)    (111,1.022e-02)
(351,3.303e-03)  (1023,1.039e-03)  (2815,3.196e-04)
(7423,9.658e-05) (18943,2.873e-05) (47103,8.437e-06)
};

\addplot coordinates{
(9,7.881e-02)     (49,3.243e-02)    (209,1.232e-02)
(769,4.454e-03)   (2561,1.551e-03)  (7937,5.236e-04)
(23297,1.723e-04) (65537,5.545e-05) (178177,1.751e-05)
};

\addplot coordinates{
(11,6.887e-02)    (71,3.177e-02)     (351,1.341e-02)
(1471,5.334e-03)  (5503,2.027e-03)   (18943,7.415e-04)
(61183,2.628e-04) (187903,9.063e-05) (553983,3.053e-05)
};

\addplot coordinates{
(13,5.755e-02)     (97,2.925e-02)     (545,1.351e-02)
(2561,5.842e-03)   (10625,2.397e-03)  (40193,9.414e-04)
(141569,3.564e-04) (471041,1.308e-04)
(1496065,4.670e-05)
};
}%


\bookmaketitle
\tableofcontents
\include{pgfplots.title_abstract_intro}
\include{pgfplots.preliminaries}
\include{pgfplots.intro}
\include{pgfplots.reference}
\include{pgfplots.libs}
\include{pgfplots.resources}
\include{pgfplots.importexport}
\include{pgfplots.basic.reference}

\printindex

\bibliographystyle{abbrv} %gerapali} %gerabbrv} %gerunsrt.bst} %gerabbrv}% gerplain}
\nocite{pgfplotstable}
\nocite{programmingnotes}
\bibliography{pgfplots}
\end{document}

% -----------------------------------------------------------------------------
% For Stefan Pinnow as reminder on what to look for when editing the manual
% -----------------------------------------------------------------------------
% There should be no line breaks in the following environments
% - |...|
% - \declareandlabel{...}
% - \verbpdfref{...}
% If "MakeTikzPictures" isn't running through check one of the externalized
% LOG files what is the cause of that.
% -----------------------------------------------------------------------------


%%%%%%%%%%%%%%%%%%%%%%%%%%%%%%%%%%%%%%%%%%%%%%%%%%%%%%%%%%%%%%%%%%%%%%%%%%%%%
%
% Package pgfplots.sty documentation.
%
% Copyright 2007/2008 by Christian Feuersaenger.
%
% This program is free software: you can redistribute it and/or modify
% it under the terms of the GNU General Public License as published by
% the Free Software Foundation, either version 3 of the License, or
% (at your option) any later version.
%
% This program is distributed in the hope that it will be useful,
% but WITHOUT ANY WARRANTY; without even the implied warranty of
% MERCHANTABILITY or FITNESS FOR A PARTICULAR PURPOSE.  See the
% GNU General Public License for more details.
%
% You should have received a copy of the GNU General Public License
% along with this program.  If not, see <http://www.gnu.org/licenses/>.
%
%
%%%%%%%%%%%%%%%%%%%%%%%%%%%%%%%%%%%%%%%%%%%%%%%%%%%%%%%%%%%%%%%%%%%%%%%%%%%%%
% SEE pgfplots-macros.tex as well!
%\pdfminorversion=5 % to allow compression
%\pdfobjcompresslevel=2
\documentclass[a4paper,openany]{book}

\let\bookmaketitle=\maketitle

% -----------------------------------
% this here is from ltxdoc:
\usepackage{doc}[=v2]
\AtBeginDocument{\MakeShortVerb{\|}}
%\setlength{\textwidth}{355pt}
%\addtolength\marginparwidth{30pt}
%\addtolength\oddsidemargin{20pt}
%\addtolength\evensidemargin{20pt}
\def\cmd#1{\cs{\expandafter\cmd@to@cs\string#1}}
\def\cmd@to@cs#1#2{\char\number`#2\relax}
\DeclareRobustCommand\cs[1]{\texttt{\char`\\#1}}
\providecommand\marg[1]{%
  {\ttfamily\char`\{}\meta{#1}{\ttfamily\char`\}}}
\providecommand\oarg[1]{%
  {\ttfamily[}\meta{#1}{\ttfamily]}}
\providecommand\parg[1]{%
  {\ttfamily(}\meta{#1}{\ttfamily)}}
\raggedbottom

% -----------------------------------
\input{pgfplots.preamble.tex}

\makeatletter
% I want a two-column index just as in pgfmanual styles. This here
% was the best way to get one:
\def\index@prologue{\section*{Index}\addcontentsline{toc}{chapter}{Index}
}
\makeatother

%\RequirePackage[german,english,francais]{babel}

\def\matlabcolormaptext{This colormap is similar to one shipped with Matlab$^\text{\textregistered}$ under a similar name.}

\IfFileExists{tikzlibraryspy.code.tex}{%
\usetikzlibrary{spy}
}{%
    \message{ERROR: tikz SPY library NOT available. The manual will only compile partially.^^J}%
}%

\usepackage{xparse}% for colorbrewer manual

\usetikzlibrary{
    decorations.markings,
    decorations.footprints,
    shapes.arrows,
    matrix,
    positioning,
}

\usepgfplotslibrary{
    fillbetween,
    ternary,
    smithchart,
    patchplots,
    polar,
    colormaps,
    colorbrewer,
    colortol,           % docu not ready yet
}
\pgfqkeys{/codeexample}{%
    every codeexample/.append style={
        /pgfplots/every ternary axis/.append style={
            /pgfplots/legend style={fill=graphicbackground},
        }
    },
    tabsize=4,
}

\pgfplotsmanualenableexternalizationofexpensive

%\usetikzlibrary{external}
%\tikzexternalize[prefix=figures/]{pgfplots}

\title{%
    Manual for Package \PGFPlots{}\\
    {\small 2D/3D Plots in \LaTeX{}, Version \pgfplotsversion}\\
    {\small\url{https://github.com/pgf-tikz/pgfplots}}
    %\\{\small Attention: you are using an unstable development version.}
}

\makeatletter
\long\def\abstractsmuggle{%
    \centering
    \textbf{Abstract}\\[0.5cm]

    \begin{minipage}{12cm}
        \PGFPlots{} draws high-quality function plots in normal or logarithmic
        scaling with a user-friendly interface directly in \TeX{}. The user
        supplies axis labels, legend entries and the plot coordinates for one or
        more plots and \PGFPlots{} applies axis scaling, computes any logarithms
        and axis ticks and draws the plots. It supports line plots, scatter
        plots, piecewise constant plots, bar plots, area plots, mesh and surface
        plots, patch plots, contour plots, quiver plots, histogram plots, box
        plots, polar axes, ternary diagrams, smith charts and some more. It is
        based on Till Tantau's package \PGF{}/\Tikz{}.
    \end{minipage}

}%

\expandafter\date\expandafter{\@date\\[2cm]
    \abstractsmuggle
}%
\makeatother

%\includeonly{pgfplots.reference}


\begin{document}

\def\plotcoords{%
\addplot coordinates {
(5,8.312e-02)    (17,2.547e-02)   (49,7.407e-03)
(129,2.102e-03)  (321,5.874e-04)  (769,1.623e-04)
(1793,4.442e-05) (4097,1.207e-05) (9217,3.261e-06)
};

\addplot coordinates{
(7,8.472e-02)    (31,3.044e-02)    (111,1.022e-02)
(351,3.303e-03)  (1023,1.039e-03)  (2815,3.196e-04)
(7423,9.658e-05) (18943,2.873e-05) (47103,8.437e-06)
};

\addplot coordinates{
(9,7.881e-02)     (49,3.243e-02)    (209,1.232e-02)
(769,4.454e-03)   (2561,1.551e-03)  (7937,5.236e-04)
(23297,1.723e-04) (65537,5.545e-05) (178177,1.751e-05)
};

\addplot coordinates{
(11,6.887e-02)    (71,3.177e-02)     (351,1.341e-02)
(1471,5.334e-03)  (5503,2.027e-03)   (18943,7.415e-04)
(61183,2.628e-04) (187903,9.063e-05) (553983,3.053e-05)
};

\addplot coordinates{
(13,5.755e-02)     (97,2.925e-02)     (545,1.351e-02)
(2561,5.842e-03)   (10625,2.397e-03)  (40193,9.414e-04)
(141569,3.564e-04) (471041,1.308e-04)
(1496065,4.670e-05)
};
}%


\bookmaketitle
\tableofcontents
\include{pgfplots.title_abstract_intro}
\include{pgfplots.preliminaries}
\include{pgfplots.intro}
\include{pgfplots.reference}
\include{pgfplots.libs}
\include{pgfplots.resources}
\include{pgfplots.importexport}
\include{pgfplots.basic.reference}

\printindex

\bibliographystyle{abbrv} %gerapali} %gerabbrv} %gerunsrt.bst} %gerabbrv}% gerplain}
\nocite{pgfplotstable}
\nocite{programmingnotes}
\bibliography{pgfplots}
\end{document}

% -----------------------------------------------------------------------------
% For Stefan Pinnow as reminder on what to look for when editing the manual
% -----------------------------------------------------------------------------
% There should be no line breaks in the following environments
% - |...|
% - \declareandlabel{...}
% - \verbpdfref{...}
% If "MakeTikzPictures" isn't running through check one of the externalized
% LOG files what is the cause of that.
% -----------------------------------------------------------------------------


%%%%%%%%%%%%%%%%%%%%%%%%%%%%%%%%%%%%%%%%%%%%%%%%%%%%%%%%%%%%%%%%%%%%%%%%%%%%%
%
% Package pgfplots.sty documentation.
%
% Copyright 2007/2008 by Christian Feuersaenger.
%
% This program is free software: you can redistribute it and/or modify
% it under the terms of the GNU General Public License as published by
% the Free Software Foundation, either version 3 of the License, or
% (at your option) any later version.
%
% This program is distributed in the hope that it will be useful,
% but WITHOUT ANY WARRANTY; without even the implied warranty of
% MERCHANTABILITY or FITNESS FOR A PARTICULAR PURPOSE.  See the
% GNU General Public License for more details.
%
% You should have received a copy of the GNU General Public License
% along with this program.  If not, see <http://www.gnu.org/licenses/>.
%
%
%%%%%%%%%%%%%%%%%%%%%%%%%%%%%%%%%%%%%%%%%%%%%%%%%%%%%%%%%%%%%%%%%%%%%%%%%%%%%
% SEE pgfplots-macros.tex as well!
%\pdfminorversion=5 % to allow compression
%\pdfobjcompresslevel=2
\documentclass[a4paper,openany]{book}

\let\bookmaketitle=\maketitle

% -----------------------------------
% this here is from ltxdoc:
\usepackage{doc}[=v2]
\AtBeginDocument{\MakeShortVerb{\|}}
%\setlength{\textwidth}{355pt}
%\addtolength\marginparwidth{30pt}
%\addtolength\oddsidemargin{20pt}
%\addtolength\evensidemargin{20pt}
\def\cmd#1{\cs{\expandafter\cmd@to@cs\string#1}}
\def\cmd@to@cs#1#2{\char\number`#2\relax}
\DeclareRobustCommand\cs[1]{\texttt{\char`\\#1}}
\providecommand\marg[1]{%
  {\ttfamily\char`\{}\meta{#1}{\ttfamily\char`\}}}
\providecommand\oarg[1]{%
  {\ttfamily[}\meta{#1}{\ttfamily]}}
\providecommand\parg[1]{%
  {\ttfamily(}\meta{#1}{\ttfamily)}}
\raggedbottom

% -----------------------------------
\input{pgfplots.preamble.tex}

\makeatletter
% I want a two-column index just as in pgfmanual styles. This here
% was the best way to get one:
\def\index@prologue{\section*{Index}\addcontentsline{toc}{chapter}{Index}
}
\makeatother

%\RequirePackage[german,english,francais]{babel}

\def\matlabcolormaptext{This colormap is similar to one shipped with Matlab$^\text{\textregistered}$ under a similar name.}

\IfFileExists{tikzlibraryspy.code.tex}{%
\usetikzlibrary{spy}
}{%
    \message{ERROR: tikz SPY library NOT available. The manual will only compile partially.^^J}%
}%

\usepackage{xparse}% for colorbrewer manual

\usetikzlibrary{
    decorations.markings,
    decorations.footprints,
    shapes.arrows,
    matrix,
    positioning,
}

\usepgfplotslibrary{
    fillbetween,
    ternary,
    smithchart,
    patchplots,
    polar,
    colormaps,
    colorbrewer,
    colortol,           % docu not ready yet
}
\pgfqkeys{/codeexample}{%
    every codeexample/.append style={
        /pgfplots/every ternary axis/.append style={
            /pgfplots/legend style={fill=graphicbackground},
        }
    },
    tabsize=4,
}

\pgfplotsmanualenableexternalizationofexpensive

%\usetikzlibrary{external}
%\tikzexternalize[prefix=figures/]{pgfplots}

\title{%
    Manual for Package \PGFPlots{}\\
    {\small 2D/3D Plots in \LaTeX{}, Version \pgfplotsversion}\\
    {\small\url{https://github.com/pgf-tikz/pgfplots}}
    %\\{\small Attention: you are using an unstable development version.}
}

\makeatletter
\long\def\abstractsmuggle{%
    \centering
    \textbf{Abstract}\\[0.5cm]

    \begin{minipage}{12cm}
        \PGFPlots{} draws high-quality function plots in normal or logarithmic
        scaling with a user-friendly interface directly in \TeX{}. The user
        supplies axis labels, legend entries and the plot coordinates for one or
        more plots and \PGFPlots{} applies axis scaling, computes any logarithms
        and axis ticks and draws the plots. It supports line plots, scatter
        plots, piecewise constant plots, bar plots, area plots, mesh and surface
        plots, patch plots, contour plots, quiver plots, histogram plots, box
        plots, polar axes, ternary diagrams, smith charts and some more. It is
        based on Till Tantau's package \PGF{}/\Tikz{}.
    \end{minipage}

}%

\expandafter\date\expandafter{\@date\\[2cm]
    \abstractsmuggle
}%
\makeatother

%\includeonly{pgfplots.reference}


\begin{document}

\def\plotcoords{%
\addplot coordinates {
(5,8.312e-02)    (17,2.547e-02)   (49,7.407e-03)
(129,2.102e-03)  (321,5.874e-04)  (769,1.623e-04)
(1793,4.442e-05) (4097,1.207e-05) (9217,3.261e-06)
};

\addplot coordinates{
(7,8.472e-02)    (31,3.044e-02)    (111,1.022e-02)
(351,3.303e-03)  (1023,1.039e-03)  (2815,3.196e-04)
(7423,9.658e-05) (18943,2.873e-05) (47103,8.437e-06)
};

\addplot coordinates{
(9,7.881e-02)     (49,3.243e-02)    (209,1.232e-02)
(769,4.454e-03)   (2561,1.551e-03)  (7937,5.236e-04)
(23297,1.723e-04) (65537,5.545e-05) (178177,1.751e-05)
};

\addplot coordinates{
(11,6.887e-02)    (71,3.177e-02)     (351,1.341e-02)
(1471,5.334e-03)  (5503,2.027e-03)   (18943,7.415e-04)
(61183,2.628e-04) (187903,9.063e-05) (553983,3.053e-05)
};

\addplot coordinates{
(13,5.755e-02)     (97,2.925e-02)     (545,1.351e-02)
(2561,5.842e-03)   (10625,2.397e-03)  (40193,9.414e-04)
(141569,3.564e-04) (471041,1.308e-04)
(1496065,4.670e-05)
};
}%


\bookmaketitle
\tableofcontents
\include{pgfplots.title_abstract_intro}
\include{pgfplots.preliminaries}
\include{pgfplots.intro}
\include{pgfplots.reference}
\include{pgfplots.libs}
\include{pgfplots.resources}
\include{pgfplots.importexport}
\include{pgfplots.basic.reference}

\printindex

\bibliographystyle{abbrv} %gerapali} %gerabbrv} %gerunsrt.bst} %gerabbrv}% gerplain}
\nocite{pgfplotstable}
\nocite{programmingnotes}
\bibliography{pgfplots}
\end{document}

% -----------------------------------------------------------------------------
% For Stefan Pinnow as reminder on what to look for when editing the manual
% -----------------------------------------------------------------------------
% There should be no line breaks in the following environments
% - |...|
% - \declareandlabel{...}
% - \verbpdfref{...}
% If "MakeTikzPictures" isn't running through check one of the externalized
% LOG files what is the cause of that.
% -----------------------------------------------------------------------------

\include{pgfplots.basic.reference}

\printindex

\bibliographystyle{abbrv} %gerapali} %gerabbrv} %gerunsrt.bst} %gerabbrv}% gerplain}
\nocite{pgfplotstable}
\nocite{programmingnotes}
\bibliography{pgfplots}
\end{document}

% -----------------------------------------------------------------------------
% For Stefan Pinnow as reminder on what to look for when editing the manual
% -----------------------------------------------------------------------------
% There should be no line breaks in the following environments
% - |...|
% - \declareandlabel{...}
% - \verbpdfref{...}
% If "MakeTikzPictures" isn't running through check one of the externalized
% LOG files what is the cause of that.
% -----------------------------------------------------------------------------


%%%%%%%%%%%%%%%%%%%%%%%%%%%%%%%%%%%%%%%%%%%%%%%%%%%%%%%%%%%%%%%%%%%%%%%%%%%%%
%
% Package pgfplots.sty documentation.
%
% Copyright 2007/2008 by Christian Feuersaenger.
%
% This program is free software: you can redistribute it and/or modify
% it under the terms of the GNU General Public License as published by
% the Free Software Foundation, either version 3 of the License, or
% (at your option) any later version.
%
% This program is distributed in the hope that it will be useful,
% but WITHOUT ANY WARRANTY; without even the implied warranty of
% MERCHANTABILITY or FITNESS FOR A PARTICULAR PURPOSE.  See the
% GNU General Public License for more details.
%
% You should have received a copy of the GNU General Public License
% along with this program.  If not, see <http://www.gnu.org/licenses/>.
%
%
%%%%%%%%%%%%%%%%%%%%%%%%%%%%%%%%%%%%%%%%%%%%%%%%%%%%%%%%%%%%%%%%%%%%%%%%%%%%%
% SEE pgfplots-macros.tex as well!
%\pdfminorversion=5 % to allow compression
%\pdfobjcompresslevel=2
\documentclass[a4paper,openany]{book}

\let\bookmaketitle=\maketitle

% -----------------------------------
% this here is from ltxdoc:
\usepackage{doc}[=v2]
\AtBeginDocument{\MakeShortVerb{\|}}
%\setlength{\textwidth}{355pt}
%\addtolength\marginparwidth{30pt}
%\addtolength\oddsidemargin{20pt}
%\addtolength\evensidemargin{20pt}
\def\cmd#1{\cs{\expandafter\cmd@to@cs\string#1}}
\def\cmd@to@cs#1#2{\char\number`#2\relax}
\DeclareRobustCommand\cs[1]{\texttt{\char`\\#1}}
\providecommand\marg[1]{%
  {\ttfamily\char`\{}\meta{#1}{\ttfamily\char`\}}}
\providecommand\oarg[1]{%
  {\ttfamily[}\meta{#1}{\ttfamily]}}
\providecommand\parg[1]{%
  {\ttfamily(}\meta{#1}{\ttfamily)}}
\raggedbottom

% -----------------------------------
\input{pgfplots.preamble.tex}

\makeatletter
% I want a two-column index just as in pgfmanual styles. This here
% was the best way to get one:
\def\index@prologue{\section*{Index}\addcontentsline{toc}{chapter}{Index}
}
\makeatother

%\RequirePackage[german,english,francais]{babel}

\def\matlabcolormaptext{This colormap is similar to one shipped with Matlab$^\text{\textregistered}$ under a similar name.}

\IfFileExists{tikzlibraryspy.code.tex}{%
\usetikzlibrary{spy}
}{%
    \message{ERROR: tikz SPY library NOT available. The manual will only compile partially.^^J}%
}%

\usepackage{xparse}% for colorbrewer manual

\usetikzlibrary{
    decorations.markings,
    decorations.footprints,
    shapes.arrows,
    matrix,
    positioning,
}

\usepgfplotslibrary{
    fillbetween,
    ternary,
    smithchart,
    patchplots,
    polar,
    colormaps,
    colorbrewer,
    colortol,           % docu not ready yet
}
\pgfqkeys{/codeexample}{%
    every codeexample/.append style={
        /pgfplots/every ternary axis/.append style={
            /pgfplots/legend style={fill=graphicbackground},
        }
    },
    tabsize=4,
}

\pgfplotsmanualenableexternalizationofexpensive

%\usetikzlibrary{external}
%\tikzexternalize[prefix=figures/]{pgfplots}

\title{%
    Manual for Package \PGFPlots{}\\
    {\small 2D/3D Plots in \LaTeX{}, Version \pgfplotsversion}\\
    {\small\url{https://github.com/pgf-tikz/pgfplots}}
    %\\{\small Attention: you are using an unstable development version.}
}

\makeatletter
\long\def\abstractsmuggle{%
    \centering
    \textbf{Abstract}\\[0.5cm]

    \begin{minipage}{12cm}
        \PGFPlots{} draws high-quality function plots in normal or logarithmic
        scaling with a user-friendly interface directly in \TeX{}. The user
        supplies axis labels, legend entries and the plot coordinates for one or
        more plots and \PGFPlots{} applies axis scaling, computes any logarithms
        and axis ticks and draws the plots. It supports line plots, scatter
        plots, piecewise constant plots, bar plots, area plots, mesh and surface
        plots, patch plots, contour plots, quiver plots, histogram plots, box
        plots, polar axes, ternary diagrams, smith charts and some more. It is
        based on Till Tantau's package \PGF{}/\Tikz{}.
    \end{minipage}

}%

\expandafter\date\expandafter{\@date\\[2cm]
    \abstractsmuggle
}%
\makeatother

%\includeonly{pgfplots.reference}


\begin{document}

\def\plotcoords{%
\addplot coordinates {
(5,8.312e-02)    (17,2.547e-02)   (49,7.407e-03)
(129,2.102e-03)  (321,5.874e-04)  (769,1.623e-04)
(1793,4.442e-05) (4097,1.207e-05) (9217,3.261e-06)
};

\addplot coordinates{
(7,8.472e-02)    (31,3.044e-02)    (111,1.022e-02)
(351,3.303e-03)  (1023,1.039e-03)  (2815,3.196e-04)
(7423,9.658e-05) (18943,2.873e-05) (47103,8.437e-06)
};

\addplot coordinates{
(9,7.881e-02)     (49,3.243e-02)    (209,1.232e-02)
(769,4.454e-03)   (2561,1.551e-03)  (7937,5.236e-04)
(23297,1.723e-04) (65537,5.545e-05) (178177,1.751e-05)
};

\addplot coordinates{
(11,6.887e-02)    (71,3.177e-02)     (351,1.341e-02)
(1471,5.334e-03)  (5503,2.027e-03)   (18943,7.415e-04)
(61183,2.628e-04) (187903,9.063e-05) (553983,3.053e-05)
};

\addplot coordinates{
(13,5.755e-02)     (97,2.925e-02)     (545,1.351e-02)
(2561,5.842e-03)   (10625,2.397e-03)  (40193,9.414e-04)
(141569,3.564e-04) (471041,1.308e-04)
(1496065,4.670e-05)
};
}%


\bookmaketitle
\tableofcontents

%%%%%%%%%%%%%%%%%%%%%%%%%%%%%%%%%%%%%%%%%%%%%%%%%%%%%%%%%%%%%%%%%%%%%%%%%%%%%
%
% Package pgfplots.sty documentation.
%
% Copyright 2007/2008 by Christian Feuersaenger.
%
% This program is free software: you can redistribute it and/or modify
% it under the terms of the GNU General Public License as published by
% the Free Software Foundation, either version 3 of the License, or
% (at your option) any later version.
%
% This program is distributed in the hope that it will be useful,
% but WITHOUT ANY WARRANTY; without even the implied warranty of
% MERCHANTABILITY or FITNESS FOR A PARTICULAR PURPOSE.  See the
% GNU General Public License for more details.
%
% You should have received a copy of the GNU General Public License
% along with this program.  If not, see <http://www.gnu.org/licenses/>.
%
%
%%%%%%%%%%%%%%%%%%%%%%%%%%%%%%%%%%%%%%%%%%%%%%%%%%%%%%%%%%%%%%%%%%%%%%%%%%%%%
% SEE pgfplots-macros.tex as well!
%\pdfminorversion=5 % to allow compression
%\pdfobjcompresslevel=2
\documentclass[a4paper,openany]{book}

\let\bookmaketitle=\maketitle

% -----------------------------------
% this here is from ltxdoc:
\usepackage{doc}[=v2]
\AtBeginDocument{\MakeShortVerb{\|}}
%\setlength{\textwidth}{355pt}
%\addtolength\marginparwidth{30pt}
%\addtolength\oddsidemargin{20pt}
%\addtolength\evensidemargin{20pt}
\def\cmd#1{\cs{\expandafter\cmd@to@cs\string#1}}
\def\cmd@to@cs#1#2{\char\number`#2\relax}
\DeclareRobustCommand\cs[1]{\texttt{\char`\\#1}}
\providecommand\marg[1]{%
  {\ttfamily\char`\{}\meta{#1}{\ttfamily\char`\}}}
\providecommand\oarg[1]{%
  {\ttfamily[}\meta{#1}{\ttfamily]}}
\providecommand\parg[1]{%
  {\ttfamily(}\meta{#1}{\ttfamily)}}
\raggedbottom

% -----------------------------------
\input{pgfplots.preamble.tex}

\makeatletter
% I want a two-column index just as in pgfmanual styles. This here
% was the best way to get one:
\def\index@prologue{\section*{Index}\addcontentsline{toc}{chapter}{Index}
}
\makeatother

%\RequirePackage[german,english,francais]{babel}

\def\matlabcolormaptext{This colormap is similar to one shipped with Matlab$^\text{\textregistered}$ under a similar name.}

\IfFileExists{tikzlibraryspy.code.tex}{%
\usetikzlibrary{spy}
}{%
    \message{ERROR: tikz SPY library NOT available. The manual will only compile partially.^^J}%
}%

\usepackage{xparse}% for colorbrewer manual

\usetikzlibrary{
    decorations.markings,
    decorations.footprints,
    shapes.arrows,
    matrix,
    positioning,
}

\usepgfplotslibrary{
    fillbetween,
    ternary,
    smithchart,
    patchplots,
    polar,
    colormaps,
    colorbrewer,
    colortol,           % docu not ready yet
}
\pgfqkeys{/codeexample}{%
    every codeexample/.append style={
        /pgfplots/every ternary axis/.append style={
            /pgfplots/legend style={fill=graphicbackground},
        }
    },
    tabsize=4,
}

\pgfplotsmanualenableexternalizationofexpensive

%\usetikzlibrary{external}
%\tikzexternalize[prefix=figures/]{pgfplots}

\title{%
    Manual for Package \PGFPlots{}\\
    {\small 2D/3D Plots in \LaTeX{}, Version \pgfplotsversion}\\
    {\small\url{https://github.com/pgf-tikz/pgfplots}}
    %\\{\small Attention: you are using an unstable development version.}
}

\makeatletter
\long\def\abstractsmuggle{%
    \centering
    \textbf{Abstract}\\[0.5cm]

    \begin{minipage}{12cm}
        \PGFPlots{} draws high-quality function plots in normal or logarithmic
        scaling with a user-friendly interface directly in \TeX{}. The user
        supplies axis labels, legend entries and the plot coordinates for one or
        more plots and \PGFPlots{} applies axis scaling, computes any logarithms
        and axis ticks and draws the plots. It supports line plots, scatter
        plots, piecewise constant plots, bar plots, area plots, mesh and surface
        plots, patch plots, contour plots, quiver plots, histogram plots, box
        plots, polar axes, ternary diagrams, smith charts and some more. It is
        based on Till Tantau's package \PGF{}/\Tikz{}.
    \end{minipage}

}%

\expandafter\date\expandafter{\@date\\[2cm]
    \abstractsmuggle
}%
\makeatother

%\includeonly{pgfplots.reference}


\begin{document}

\def\plotcoords{%
\addplot coordinates {
(5,8.312e-02)    (17,2.547e-02)   (49,7.407e-03)
(129,2.102e-03)  (321,5.874e-04)  (769,1.623e-04)
(1793,4.442e-05) (4097,1.207e-05) (9217,3.261e-06)
};

\addplot coordinates{
(7,8.472e-02)    (31,3.044e-02)    (111,1.022e-02)
(351,3.303e-03)  (1023,1.039e-03)  (2815,3.196e-04)
(7423,9.658e-05) (18943,2.873e-05) (47103,8.437e-06)
};

\addplot coordinates{
(9,7.881e-02)     (49,3.243e-02)    (209,1.232e-02)
(769,4.454e-03)   (2561,1.551e-03)  (7937,5.236e-04)
(23297,1.723e-04) (65537,5.545e-05) (178177,1.751e-05)
};

\addplot coordinates{
(11,6.887e-02)    (71,3.177e-02)     (351,1.341e-02)
(1471,5.334e-03)  (5503,2.027e-03)   (18943,7.415e-04)
(61183,2.628e-04) (187903,9.063e-05) (553983,3.053e-05)
};

\addplot coordinates{
(13,5.755e-02)     (97,2.925e-02)     (545,1.351e-02)
(2561,5.842e-03)   (10625,2.397e-03)  (40193,9.414e-04)
(141569,3.564e-04) (471041,1.308e-04)
(1496065,4.670e-05)
};
}%


\bookmaketitle
\tableofcontents
\include{pgfplots.title_abstract_intro}
\include{pgfplots.preliminaries}
\include{pgfplots.intro}
\include{pgfplots.reference}
\include{pgfplots.libs}
\include{pgfplots.resources}
\include{pgfplots.importexport}
\include{pgfplots.basic.reference}

\printindex

\bibliographystyle{abbrv} %gerapali} %gerabbrv} %gerunsrt.bst} %gerabbrv}% gerplain}
\nocite{pgfplotstable}
\nocite{programmingnotes}
\bibliography{pgfplots}
\end{document}

% -----------------------------------------------------------------------------
% For Stefan Pinnow as reminder on what to look for when editing the manual
% -----------------------------------------------------------------------------
% There should be no line breaks in the following environments
% - |...|
% - \declareandlabel{...}
% - \verbpdfref{...}
% If "MakeTikzPictures" isn't running through check one of the externalized
% LOG files what is the cause of that.
% -----------------------------------------------------------------------------


%%%%%%%%%%%%%%%%%%%%%%%%%%%%%%%%%%%%%%%%%%%%%%%%%%%%%%%%%%%%%%%%%%%%%%%%%%%%%
%
% Package pgfplots.sty documentation.
%
% Copyright 2007/2008 by Christian Feuersaenger.
%
% This program is free software: you can redistribute it and/or modify
% it under the terms of the GNU General Public License as published by
% the Free Software Foundation, either version 3 of the License, or
% (at your option) any later version.
%
% This program is distributed in the hope that it will be useful,
% but WITHOUT ANY WARRANTY; without even the implied warranty of
% MERCHANTABILITY or FITNESS FOR A PARTICULAR PURPOSE.  See the
% GNU General Public License for more details.
%
% You should have received a copy of the GNU General Public License
% along with this program.  If not, see <http://www.gnu.org/licenses/>.
%
%
%%%%%%%%%%%%%%%%%%%%%%%%%%%%%%%%%%%%%%%%%%%%%%%%%%%%%%%%%%%%%%%%%%%%%%%%%%%%%
% SEE pgfplots-macros.tex as well!
%\pdfminorversion=5 % to allow compression
%\pdfobjcompresslevel=2
\documentclass[a4paper,openany]{book}

\let\bookmaketitle=\maketitle

% -----------------------------------
% this here is from ltxdoc:
\usepackage{doc}[=v2]
\AtBeginDocument{\MakeShortVerb{\|}}
%\setlength{\textwidth}{355pt}
%\addtolength\marginparwidth{30pt}
%\addtolength\oddsidemargin{20pt}
%\addtolength\evensidemargin{20pt}
\def\cmd#1{\cs{\expandafter\cmd@to@cs\string#1}}
\def\cmd@to@cs#1#2{\char\number`#2\relax}
\DeclareRobustCommand\cs[1]{\texttt{\char`\\#1}}
\providecommand\marg[1]{%
  {\ttfamily\char`\{}\meta{#1}{\ttfamily\char`\}}}
\providecommand\oarg[1]{%
  {\ttfamily[}\meta{#1}{\ttfamily]}}
\providecommand\parg[1]{%
  {\ttfamily(}\meta{#1}{\ttfamily)}}
\raggedbottom

% -----------------------------------
\input{pgfplots.preamble.tex}

\makeatletter
% I want a two-column index just as in pgfmanual styles. This here
% was the best way to get one:
\def\index@prologue{\section*{Index}\addcontentsline{toc}{chapter}{Index}
}
\makeatother

%\RequirePackage[german,english,francais]{babel}

\def\matlabcolormaptext{This colormap is similar to one shipped with Matlab$^\text{\textregistered}$ under a similar name.}

\IfFileExists{tikzlibraryspy.code.tex}{%
\usetikzlibrary{spy}
}{%
    \message{ERROR: tikz SPY library NOT available. The manual will only compile partially.^^J}%
}%

\usepackage{xparse}% for colorbrewer manual

\usetikzlibrary{
    decorations.markings,
    decorations.footprints,
    shapes.arrows,
    matrix,
    positioning,
}

\usepgfplotslibrary{
    fillbetween,
    ternary,
    smithchart,
    patchplots,
    polar,
    colormaps,
    colorbrewer,
    colortol,           % docu not ready yet
}
\pgfqkeys{/codeexample}{%
    every codeexample/.append style={
        /pgfplots/every ternary axis/.append style={
            /pgfplots/legend style={fill=graphicbackground},
        }
    },
    tabsize=4,
}

\pgfplotsmanualenableexternalizationofexpensive

%\usetikzlibrary{external}
%\tikzexternalize[prefix=figures/]{pgfplots}

\title{%
    Manual for Package \PGFPlots{}\\
    {\small 2D/3D Plots in \LaTeX{}, Version \pgfplotsversion}\\
    {\small\url{https://github.com/pgf-tikz/pgfplots}}
    %\\{\small Attention: you are using an unstable development version.}
}

\makeatletter
\long\def\abstractsmuggle{%
    \centering
    \textbf{Abstract}\\[0.5cm]

    \begin{minipage}{12cm}
        \PGFPlots{} draws high-quality function plots in normal or logarithmic
        scaling with a user-friendly interface directly in \TeX{}. The user
        supplies axis labels, legend entries and the plot coordinates for one or
        more plots and \PGFPlots{} applies axis scaling, computes any logarithms
        and axis ticks and draws the plots. It supports line plots, scatter
        plots, piecewise constant plots, bar plots, area plots, mesh and surface
        plots, patch plots, contour plots, quiver plots, histogram plots, box
        plots, polar axes, ternary diagrams, smith charts and some more. It is
        based on Till Tantau's package \PGF{}/\Tikz{}.
    \end{minipage}

}%

\expandafter\date\expandafter{\@date\\[2cm]
    \abstractsmuggle
}%
\makeatother

%\includeonly{pgfplots.reference}


\begin{document}

\def\plotcoords{%
\addplot coordinates {
(5,8.312e-02)    (17,2.547e-02)   (49,7.407e-03)
(129,2.102e-03)  (321,5.874e-04)  (769,1.623e-04)
(1793,4.442e-05) (4097,1.207e-05) (9217,3.261e-06)
};

\addplot coordinates{
(7,8.472e-02)    (31,3.044e-02)    (111,1.022e-02)
(351,3.303e-03)  (1023,1.039e-03)  (2815,3.196e-04)
(7423,9.658e-05) (18943,2.873e-05) (47103,8.437e-06)
};

\addplot coordinates{
(9,7.881e-02)     (49,3.243e-02)    (209,1.232e-02)
(769,4.454e-03)   (2561,1.551e-03)  (7937,5.236e-04)
(23297,1.723e-04) (65537,5.545e-05) (178177,1.751e-05)
};

\addplot coordinates{
(11,6.887e-02)    (71,3.177e-02)     (351,1.341e-02)
(1471,5.334e-03)  (5503,2.027e-03)   (18943,7.415e-04)
(61183,2.628e-04) (187903,9.063e-05) (553983,3.053e-05)
};

\addplot coordinates{
(13,5.755e-02)     (97,2.925e-02)     (545,1.351e-02)
(2561,5.842e-03)   (10625,2.397e-03)  (40193,9.414e-04)
(141569,3.564e-04) (471041,1.308e-04)
(1496065,4.670e-05)
};
}%


\bookmaketitle
\tableofcontents
\include{pgfplots.title_abstract_intro}
\include{pgfplots.preliminaries}
\include{pgfplots.intro}
\include{pgfplots.reference}
\include{pgfplots.libs}
\include{pgfplots.resources}
\include{pgfplots.importexport}
\include{pgfplots.basic.reference}

\printindex

\bibliographystyle{abbrv} %gerapali} %gerabbrv} %gerunsrt.bst} %gerabbrv}% gerplain}
\nocite{pgfplotstable}
\nocite{programmingnotes}
\bibliography{pgfplots}
\end{document}

% -----------------------------------------------------------------------------
% For Stefan Pinnow as reminder on what to look for when editing the manual
% -----------------------------------------------------------------------------
% There should be no line breaks in the following environments
% - |...|
% - \declareandlabel{...}
% - \verbpdfref{...}
% If "MakeTikzPictures" isn't running through check one of the externalized
% LOG files what is the cause of that.
% -----------------------------------------------------------------------------


%%%%%%%%%%%%%%%%%%%%%%%%%%%%%%%%%%%%%%%%%%%%%%%%%%%%%%%%%%%%%%%%%%%%%%%%%%%%%
%
% Package pgfplots.sty documentation.
%
% Copyright 2007/2008 by Christian Feuersaenger.
%
% This program is free software: you can redistribute it and/or modify
% it under the terms of the GNU General Public License as published by
% the Free Software Foundation, either version 3 of the License, or
% (at your option) any later version.
%
% This program is distributed in the hope that it will be useful,
% but WITHOUT ANY WARRANTY; without even the implied warranty of
% MERCHANTABILITY or FITNESS FOR A PARTICULAR PURPOSE.  See the
% GNU General Public License for more details.
%
% You should have received a copy of the GNU General Public License
% along with this program.  If not, see <http://www.gnu.org/licenses/>.
%
%
%%%%%%%%%%%%%%%%%%%%%%%%%%%%%%%%%%%%%%%%%%%%%%%%%%%%%%%%%%%%%%%%%%%%%%%%%%%%%
% SEE pgfplots-macros.tex as well!
%\pdfminorversion=5 % to allow compression
%\pdfobjcompresslevel=2
\documentclass[a4paper,openany]{book}

\let\bookmaketitle=\maketitle

% -----------------------------------
% this here is from ltxdoc:
\usepackage{doc}[=v2]
\AtBeginDocument{\MakeShortVerb{\|}}
%\setlength{\textwidth}{355pt}
%\addtolength\marginparwidth{30pt}
%\addtolength\oddsidemargin{20pt}
%\addtolength\evensidemargin{20pt}
\def\cmd#1{\cs{\expandafter\cmd@to@cs\string#1}}
\def\cmd@to@cs#1#2{\char\number`#2\relax}
\DeclareRobustCommand\cs[1]{\texttt{\char`\\#1}}
\providecommand\marg[1]{%
  {\ttfamily\char`\{}\meta{#1}{\ttfamily\char`\}}}
\providecommand\oarg[1]{%
  {\ttfamily[}\meta{#1}{\ttfamily]}}
\providecommand\parg[1]{%
  {\ttfamily(}\meta{#1}{\ttfamily)}}
\raggedbottom

% -----------------------------------
\input{pgfplots.preamble.tex}

\makeatletter
% I want a two-column index just as in pgfmanual styles. This here
% was the best way to get one:
\def\index@prologue{\section*{Index}\addcontentsline{toc}{chapter}{Index}
}
\makeatother

%\RequirePackage[german,english,francais]{babel}

\def\matlabcolormaptext{This colormap is similar to one shipped with Matlab$^\text{\textregistered}$ under a similar name.}

\IfFileExists{tikzlibraryspy.code.tex}{%
\usetikzlibrary{spy}
}{%
    \message{ERROR: tikz SPY library NOT available. The manual will only compile partially.^^J}%
}%

\usepackage{xparse}% for colorbrewer manual

\usetikzlibrary{
    decorations.markings,
    decorations.footprints,
    shapes.arrows,
    matrix,
    positioning,
}

\usepgfplotslibrary{
    fillbetween,
    ternary,
    smithchart,
    patchplots,
    polar,
    colormaps,
    colorbrewer,
    colortol,           % docu not ready yet
}
\pgfqkeys{/codeexample}{%
    every codeexample/.append style={
        /pgfplots/every ternary axis/.append style={
            /pgfplots/legend style={fill=graphicbackground},
        }
    },
    tabsize=4,
}

\pgfplotsmanualenableexternalizationofexpensive

%\usetikzlibrary{external}
%\tikzexternalize[prefix=figures/]{pgfplots}

\title{%
    Manual for Package \PGFPlots{}\\
    {\small 2D/3D Plots in \LaTeX{}, Version \pgfplotsversion}\\
    {\small\url{https://github.com/pgf-tikz/pgfplots}}
    %\\{\small Attention: you are using an unstable development version.}
}

\makeatletter
\long\def\abstractsmuggle{%
    \centering
    \textbf{Abstract}\\[0.5cm]

    \begin{minipage}{12cm}
        \PGFPlots{} draws high-quality function plots in normal or logarithmic
        scaling with a user-friendly interface directly in \TeX{}. The user
        supplies axis labels, legend entries and the plot coordinates for one or
        more plots and \PGFPlots{} applies axis scaling, computes any logarithms
        and axis ticks and draws the plots. It supports line plots, scatter
        plots, piecewise constant plots, bar plots, area plots, mesh and surface
        plots, patch plots, contour plots, quiver plots, histogram plots, box
        plots, polar axes, ternary diagrams, smith charts and some more. It is
        based on Till Tantau's package \PGF{}/\Tikz{}.
    \end{minipage}

}%

\expandafter\date\expandafter{\@date\\[2cm]
    \abstractsmuggle
}%
\makeatother

%\includeonly{pgfplots.reference}


\begin{document}

\def\plotcoords{%
\addplot coordinates {
(5,8.312e-02)    (17,2.547e-02)   (49,7.407e-03)
(129,2.102e-03)  (321,5.874e-04)  (769,1.623e-04)
(1793,4.442e-05) (4097,1.207e-05) (9217,3.261e-06)
};

\addplot coordinates{
(7,8.472e-02)    (31,3.044e-02)    (111,1.022e-02)
(351,3.303e-03)  (1023,1.039e-03)  (2815,3.196e-04)
(7423,9.658e-05) (18943,2.873e-05) (47103,8.437e-06)
};

\addplot coordinates{
(9,7.881e-02)     (49,3.243e-02)    (209,1.232e-02)
(769,4.454e-03)   (2561,1.551e-03)  (7937,5.236e-04)
(23297,1.723e-04) (65537,5.545e-05) (178177,1.751e-05)
};

\addplot coordinates{
(11,6.887e-02)    (71,3.177e-02)     (351,1.341e-02)
(1471,5.334e-03)  (5503,2.027e-03)   (18943,7.415e-04)
(61183,2.628e-04) (187903,9.063e-05) (553983,3.053e-05)
};

\addplot coordinates{
(13,5.755e-02)     (97,2.925e-02)     (545,1.351e-02)
(2561,5.842e-03)   (10625,2.397e-03)  (40193,9.414e-04)
(141569,3.564e-04) (471041,1.308e-04)
(1496065,4.670e-05)
};
}%


\bookmaketitle
\tableofcontents
\include{pgfplots.title_abstract_intro}
\include{pgfplots.preliminaries}
\include{pgfplots.intro}
\include{pgfplots.reference}
\include{pgfplots.libs}
\include{pgfplots.resources}
\include{pgfplots.importexport}
\include{pgfplots.basic.reference}

\printindex

\bibliographystyle{abbrv} %gerapali} %gerabbrv} %gerunsrt.bst} %gerabbrv}% gerplain}
\nocite{pgfplotstable}
\nocite{programmingnotes}
\bibliography{pgfplots}
\end{document}

% -----------------------------------------------------------------------------
% For Stefan Pinnow as reminder on what to look for when editing the manual
% -----------------------------------------------------------------------------
% There should be no line breaks in the following environments
% - |...|
% - \declareandlabel{...}
% - \verbpdfref{...}
% If "MakeTikzPictures" isn't running through check one of the externalized
% LOG files what is the cause of that.
% -----------------------------------------------------------------------------


%%%%%%%%%%%%%%%%%%%%%%%%%%%%%%%%%%%%%%%%%%%%%%%%%%%%%%%%%%%%%%%%%%%%%%%%%%%%%
%
% Package pgfplots.sty documentation.
%
% Copyright 2007/2008 by Christian Feuersaenger.
%
% This program is free software: you can redistribute it and/or modify
% it under the terms of the GNU General Public License as published by
% the Free Software Foundation, either version 3 of the License, or
% (at your option) any later version.
%
% This program is distributed in the hope that it will be useful,
% but WITHOUT ANY WARRANTY; without even the implied warranty of
% MERCHANTABILITY or FITNESS FOR A PARTICULAR PURPOSE.  See the
% GNU General Public License for more details.
%
% You should have received a copy of the GNU General Public License
% along with this program.  If not, see <http://www.gnu.org/licenses/>.
%
%
%%%%%%%%%%%%%%%%%%%%%%%%%%%%%%%%%%%%%%%%%%%%%%%%%%%%%%%%%%%%%%%%%%%%%%%%%%%%%
% SEE pgfplots-macros.tex as well!
%\pdfminorversion=5 % to allow compression
%\pdfobjcompresslevel=2
\documentclass[a4paper,openany]{book}

\let\bookmaketitle=\maketitle

% -----------------------------------
% this here is from ltxdoc:
\usepackage{doc}[=v2]
\AtBeginDocument{\MakeShortVerb{\|}}
%\setlength{\textwidth}{355pt}
%\addtolength\marginparwidth{30pt}
%\addtolength\oddsidemargin{20pt}
%\addtolength\evensidemargin{20pt}
\def\cmd#1{\cs{\expandafter\cmd@to@cs\string#1}}
\def\cmd@to@cs#1#2{\char\number`#2\relax}
\DeclareRobustCommand\cs[1]{\texttt{\char`\\#1}}
\providecommand\marg[1]{%
  {\ttfamily\char`\{}\meta{#1}{\ttfamily\char`\}}}
\providecommand\oarg[1]{%
  {\ttfamily[}\meta{#1}{\ttfamily]}}
\providecommand\parg[1]{%
  {\ttfamily(}\meta{#1}{\ttfamily)}}
\raggedbottom

% -----------------------------------
\input{pgfplots.preamble.tex}

\makeatletter
% I want a two-column index just as in pgfmanual styles. This here
% was the best way to get one:
\def\index@prologue{\section*{Index}\addcontentsline{toc}{chapter}{Index}
}
\makeatother

%\RequirePackage[german,english,francais]{babel}

\def\matlabcolormaptext{This colormap is similar to one shipped with Matlab$^\text{\textregistered}$ under a similar name.}

\IfFileExists{tikzlibraryspy.code.tex}{%
\usetikzlibrary{spy}
}{%
    \message{ERROR: tikz SPY library NOT available. The manual will only compile partially.^^J}%
}%

\usepackage{xparse}% for colorbrewer manual

\usetikzlibrary{
    decorations.markings,
    decorations.footprints,
    shapes.arrows,
    matrix,
    positioning,
}

\usepgfplotslibrary{
    fillbetween,
    ternary,
    smithchart,
    patchplots,
    polar,
    colormaps,
    colorbrewer,
    colortol,           % docu not ready yet
}
\pgfqkeys{/codeexample}{%
    every codeexample/.append style={
        /pgfplots/every ternary axis/.append style={
            /pgfplots/legend style={fill=graphicbackground},
        }
    },
    tabsize=4,
}

\pgfplotsmanualenableexternalizationofexpensive

%\usetikzlibrary{external}
%\tikzexternalize[prefix=figures/]{pgfplots}

\title{%
    Manual for Package \PGFPlots{}\\
    {\small 2D/3D Plots in \LaTeX{}, Version \pgfplotsversion}\\
    {\small\url{https://github.com/pgf-tikz/pgfplots}}
    %\\{\small Attention: you are using an unstable development version.}
}

\makeatletter
\long\def\abstractsmuggle{%
    \centering
    \textbf{Abstract}\\[0.5cm]

    \begin{minipage}{12cm}
        \PGFPlots{} draws high-quality function plots in normal or logarithmic
        scaling with a user-friendly interface directly in \TeX{}. The user
        supplies axis labels, legend entries and the plot coordinates for one or
        more plots and \PGFPlots{} applies axis scaling, computes any logarithms
        and axis ticks and draws the plots. It supports line plots, scatter
        plots, piecewise constant plots, bar plots, area plots, mesh and surface
        plots, patch plots, contour plots, quiver plots, histogram plots, box
        plots, polar axes, ternary diagrams, smith charts and some more. It is
        based on Till Tantau's package \PGF{}/\Tikz{}.
    \end{minipage}

}%

\expandafter\date\expandafter{\@date\\[2cm]
    \abstractsmuggle
}%
\makeatother

%\includeonly{pgfplots.reference}


\begin{document}

\def\plotcoords{%
\addplot coordinates {
(5,8.312e-02)    (17,2.547e-02)   (49,7.407e-03)
(129,2.102e-03)  (321,5.874e-04)  (769,1.623e-04)
(1793,4.442e-05) (4097,1.207e-05) (9217,3.261e-06)
};

\addplot coordinates{
(7,8.472e-02)    (31,3.044e-02)    (111,1.022e-02)
(351,3.303e-03)  (1023,1.039e-03)  (2815,3.196e-04)
(7423,9.658e-05) (18943,2.873e-05) (47103,8.437e-06)
};

\addplot coordinates{
(9,7.881e-02)     (49,3.243e-02)    (209,1.232e-02)
(769,4.454e-03)   (2561,1.551e-03)  (7937,5.236e-04)
(23297,1.723e-04) (65537,5.545e-05) (178177,1.751e-05)
};

\addplot coordinates{
(11,6.887e-02)    (71,3.177e-02)     (351,1.341e-02)
(1471,5.334e-03)  (5503,2.027e-03)   (18943,7.415e-04)
(61183,2.628e-04) (187903,9.063e-05) (553983,3.053e-05)
};

\addplot coordinates{
(13,5.755e-02)     (97,2.925e-02)     (545,1.351e-02)
(2561,5.842e-03)   (10625,2.397e-03)  (40193,9.414e-04)
(141569,3.564e-04) (471041,1.308e-04)
(1496065,4.670e-05)
};
}%


\bookmaketitle
\tableofcontents
\include{pgfplots.title_abstract_intro}
\include{pgfplots.preliminaries}
\include{pgfplots.intro}
\include{pgfplots.reference}
\include{pgfplots.libs}
\include{pgfplots.resources}
\include{pgfplots.importexport}
\include{pgfplots.basic.reference}

\printindex

\bibliographystyle{abbrv} %gerapali} %gerabbrv} %gerunsrt.bst} %gerabbrv}% gerplain}
\nocite{pgfplotstable}
\nocite{programmingnotes}
\bibliography{pgfplots}
\end{document}

% -----------------------------------------------------------------------------
% For Stefan Pinnow as reminder on what to look for when editing the manual
% -----------------------------------------------------------------------------
% There should be no line breaks in the following environments
% - |...|
% - \declareandlabel{...}
% - \verbpdfref{...}
% If "MakeTikzPictures" isn't running through check one of the externalized
% LOG files what is the cause of that.
% -----------------------------------------------------------------------------


%%%%%%%%%%%%%%%%%%%%%%%%%%%%%%%%%%%%%%%%%%%%%%%%%%%%%%%%%%%%%%%%%%%%%%%%%%%%%
%
% Package pgfplots.sty documentation.
%
% Copyright 2007/2008 by Christian Feuersaenger.
%
% This program is free software: you can redistribute it and/or modify
% it under the terms of the GNU General Public License as published by
% the Free Software Foundation, either version 3 of the License, or
% (at your option) any later version.
%
% This program is distributed in the hope that it will be useful,
% but WITHOUT ANY WARRANTY; without even the implied warranty of
% MERCHANTABILITY or FITNESS FOR A PARTICULAR PURPOSE.  See the
% GNU General Public License for more details.
%
% You should have received a copy of the GNU General Public License
% along with this program.  If not, see <http://www.gnu.org/licenses/>.
%
%
%%%%%%%%%%%%%%%%%%%%%%%%%%%%%%%%%%%%%%%%%%%%%%%%%%%%%%%%%%%%%%%%%%%%%%%%%%%%%
% SEE pgfplots-macros.tex as well!
%\pdfminorversion=5 % to allow compression
%\pdfobjcompresslevel=2
\documentclass[a4paper,openany]{book}

\let\bookmaketitle=\maketitle

% -----------------------------------
% this here is from ltxdoc:
\usepackage{doc}[=v2]
\AtBeginDocument{\MakeShortVerb{\|}}
%\setlength{\textwidth}{355pt}
%\addtolength\marginparwidth{30pt}
%\addtolength\oddsidemargin{20pt}
%\addtolength\evensidemargin{20pt}
\def\cmd#1{\cs{\expandafter\cmd@to@cs\string#1}}
\def\cmd@to@cs#1#2{\char\number`#2\relax}
\DeclareRobustCommand\cs[1]{\texttt{\char`\\#1}}
\providecommand\marg[1]{%
  {\ttfamily\char`\{}\meta{#1}{\ttfamily\char`\}}}
\providecommand\oarg[1]{%
  {\ttfamily[}\meta{#1}{\ttfamily]}}
\providecommand\parg[1]{%
  {\ttfamily(}\meta{#1}{\ttfamily)}}
\raggedbottom

% -----------------------------------
\input{pgfplots.preamble.tex}

\makeatletter
% I want a two-column index just as in pgfmanual styles. This here
% was the best way to get one:
\def\index@prologue{\section*{Index}\addcontentsline{toc}{chapter}{Index}
}
\makeatother

%\RequirePackage[german,english,francais]{babel}

\def\matlabcolormaptext{This colormap is similar to one shipped with Matlab$^\text{\textregistered}$ under a similar name.}

\IfFileExists{tikzlibraryspy.code.tex}{%
\usetikzlibrary{spy}
}{%
    \message{ERROR: tikz SPY library NOT available. The manual will only compile partially.^^J}%
}%

\usepackage{xparse}% for colorbrewer manual

\usetikzlibrary{
    decorations.markings,
    decorations.footprints,
    shapes.arrows,
    matrix,
    positioning,
}

\usepgfplotslibrary{
    fillbetween,
    ternary,
    smithchart,
    patchplots,
    polar,
    colormaps,
    colorbrewer,
    colortol,           % docu not ready yet
}
\pgfqkeys{/codeexample}{%
    every codeexample/.append style={
        /pgfplots/every ternary axis/.append style={
            /pgfplots/legend style={fill=graphicbackground},
        }
    },
    tabsize=4,
}

\pgfplotsmanualenableexternalizationofexpensive

%\usetikzlibrary{external}
%\tikzexternalize[prefix=figures/]{pgfplots}

\title{%
    Manual for Package \PGFPlots{}\\
    {\small 2D/3D Plots in \LaTeX{}, Version \pgfplotsversion}\\
    {\small\url{https://github.com/pgf-tikz/pgfplots}}
    %\\{\small Attention: you are using an unstable development version.}
}

\makeatletter
\long\def\abstractsmuggle{%
    \centering
    \textbf{Abstract}\\[0.5cm]

    \begin{minipage}{12cm}
        \PGFPlots{} draws high-quality function plots in normal or logarithmic
        scaling with a user-friendly interface directly in \TeX{}. The user
        supplies axis labels, legend entries and the plot coordinates for one or
        more plots and \PGFPlots{} applies axis scaling, computes any logarithms
        and axis ticks and draws the plots. It supports line plots, scatter
        plots, piecewise constant plots, bar plots, area plots, mesh and surface
        plots, patch plots, contour plots, quiver plots, histogram plots, box
        plots, polar axes, ternary diagrams, smith charts and some more. It is
        based on Till Tantau's package \PGF{}/\Tikz{}.
    \end{minipage}

}%

\expandafter\date\expandafter{\@date\\[2cm]
    \abstractsmuggle
}%
\makeatother

%\includeonly{pgfplots.reference}


\begin{document}

\def\plotcoords{%
\addplot coordinates {
(5,8.312e-02)    (17,2.547e-02)   (49,7.407e-03)
(129,2.102e-03)  (321,5.874e-04)  (769,1.623e-04)
(1793,4.442e-05) (4097,1.207e-05) (9217,3.261e-06)
};

\addplot coordinates{
(7,8.472e-02)    (31,3.044e-02)    (111,1.022e-02)
(351,3.303e-03)  (1023,1.039e-03)  (2815,3.196e-04)
(7423,9.658e-05) (18943,2.873e-05) (47103,8.437e-06)
};

\addplot coordinates{
(9,7.881e-02)     (49,3.243e-02)    (209,1.232e-02)
(769,4.454e-03)   (2561,1.551e-03)  (7937,5.236e-04)
(23297,1.723e-04) (65537,5.545e-05) (178177,1.751e-05)
};

\addplot coordinates{
(11,6.887e-02)    (71,3.177e-02)     (351,1.341e-02)
(1471,5.334e-03)  (5503,2.027e-03)   (18943,7.415e-04)
(61183,2.628e-04) (187903,9.063e-05) (553983,3.053e-05)
};

\addplot coordinates{
(13,5.755e-02)     (97,2.925e-02)     (545,1.351e-02)
(2561,5.842e-03)   (10625,2.397e-03)  (40193,9.414e-04)
(141569,3.564e-04) (471041,1.308e-04)
(1496065,4.670e-05)
};
}%


\bookmaketitle
\tableofcontents
\include{pgfplots.title_abstract_intro}
\include{pgfplots.preliminaries}
\include{pgfplots.intro}
\include{pgfplots.reference}
\include{pgfplots.libs}
\include{pgfplots.resources}
\include{pgfplots.importexport}
\include{pgfplots.basic.reference}

\printindex

\bibliographystyle{abbrv} %gerapali} %gerabbrv} %gerunsrt.bst} %gerabbrv}% gerplain}
\nocite{pgfplotstable}
\nocite{programmingnotes}
\bibliography{pgfplots}
\end{document}

% -----------------------------------------------------------------------------
% For Stefan Pinnow as reminder on what to look for when editing the manual
% -----------------------------------------------------------------------------
% There should be no line breaks in the following environments
% - |...|
% - \declareandlabel{...}
% - \verbpdfref{...}
% If "MakeTikzPictures" isn't running through check one of the externalized
% LOG files what is the cause of that.
% -----------------------------------------------------------------------------


%%%%%%%%%%%%%%%%%%%%%%%%%%%%%%%%%%%%%%%%%%%%%%%%%%%%%%%%%%%%%%%%%%%%%%%%%%%%%
%
% Package pgfplots.sty documentation.
%
% Copyright 2007/2008 by Christian Feuersaenger.
%
% This program is free software: you can redistribute it and/or modify
% it under the terms of the GNU General Public License as published by
% the Free Software Foundation, either version 3 of the License, or
% (at your option) any later version.
%
% This program is distributed in the hope that it will be useful,
% but WITHOUT ANY WARRANTY; without even the implied warranty of
% MERCHANTABILITY or FITNESS FOR A PARTICULAR PURPOSE.  See the
% GNU General Public License for more details.
%
% You should have received a copy of the GNU General Public License
% along with this program.  If not, see <http://www.gnu.org/licenses/>.
%
%
%%%%%%%%%%%%%%%%%%%%%%%%%%%%%%%%%%%%%%%%%%%%%%%%%%%%%%%%%%%%%%%%%%%%%%%%%%%%%
% SEE pgfplots-macros.tex as well!
%\pdfminorversion=5 % to allow compression
%\pdfobjcompresslevel=2
\documentclass[a4paper,openany]{book}

\let\bookmaketitle=\maketitle

% -----------------------------------
% this here is from ltxdoc:
\usepackage{doc}[=v2]
\AtBeginDocument{\MakeShortVerb{\|}}
%\setlength{\textwidth}{355pt}
%\addtolength\marginparwidth{30pt}
%\addtolength\oddsidemargin{20pt}
%\addtolength\evensidemargin{20pt}
\def\cmd#1{\cs{\expandafter\cmd@to@cs\string#1}}
\def\cmd@to@cs#1#2{\char\number`#2\relax}
\DeclareRobustCommand\cs[1]{\texttt{\char`\\#1}}
\providecommand\marg[1]{%
  {\ttfamily\char`\{}\meta{#1}{\ttfamily\char`\}}}
\providecommand\oarg[1]{%
  {\ttfamily[}\meta{#1}{\ttfamily]}}
\providecommand\parg[1]{%
  {\ttfamily(}\meta{#1}{\ttfamily)}}
\raggedbottom

% -----------------------------------
\input{pgfplots.preamble.tex}

\makeatletter
% I want a two-column index just as in pgfmanual styles. This here
% was the best way to get one:
\def\index@prologue{\section*{Index}\addcontentsline{toc}{chapter}{Index}
}
\makeatother

%\RequirePackage[german,english,francais]{babel}

\def\matlabcolormaptext{This colormap is similar to one shipped with Matlab$^\text{\textregistered}$ under a similar name.}

\IfFileExists{tikzlibraryspy.code.tex}{%
\usetikzlibrary{spy}
}{%
    \message{ERROR: tikz SPY library NOT available. The manual will only compile partially.^^J}%
}%

\usepackage{xparse}% for colorbrewer manual

\usetikzlibrary{
    decorations.markings,
    decorations.footprints,
    shapes.arrows,
    matrix,
    positioning,
}

\usepgfplotslibrary{
    fillbetween,
    ternary,
    smithchart,
    patchplots,
    polar,
    colormaps,
    colorbrewer,
    colortol,           % docu not ready yet
}
\pgfqkeys{/codeexample}{%
    every codeexample/.append style={
        /pgfplots/every ternary axis/.append style={
            /pgfplots/legend style={fill=graphicbackground},
        }
    },
    tabsize=4,
}

\pgfplotsmanualenableexternalizationofexpensive

%\usetikzlibrary{external}
%\tikzexternalize[prefix=figures/]{pgfplots}

\title{%
    Manual for Package \PGFPlots{}\\
    {\small 2D/3D Plots in \LaTeX{}, Version \pgfplotsversion}\\
    {\small\url{https://github.com/pgf-tikz/pgfplots}}
    %\\{\small Attention: you are using an unstable development version.}
}

\makeatletter
\long\def\abstractsmuggle{%
    \centering
    \textbf{Abstract}\\[0.5cm]

    \begin{minipage}{12cm}
        \PGFPlots{} draws high-quality function plots in normal or logarithmic
        scaling with a user-friendly interface directly in \TeX{}. The user
        supplies axis labels, legend entries and the plot coordinates for one or
        more plots and \PGFPlots{} applies axis scaling, computes any logarithms
        and axis ticks and draws the plots. It supports line plots, scatter
        plots, piecewise constant plots, bar plots, area plots, mesh and surface
        plots, patch plots, contour plots, quiver plots, histogram plots, box
        plots, polar axes, ternary diagrams, smith charts and some more. It is
        based on Till Tantau's package \PGF{}/\Tikz{}.
    \end{minipage}

}%

\expandafter\date\expandafter{\@date\\[2cm]
    \abstractsmuggle
}%
\makeatother

%\includeonly{pgfplots.reference}


\begin{document}

\def\plotcoords{%
\addplot coordinates {
(5,8.312e-02)    (17,2.547e-02)   (49,7.407e-03)
(129,2.102e-03)  (321,5.874e-04)  (769,1.623e-04)
(1793,4.442e-05) (4097,1.207e-05) (9217,3.261e-06)
};

\addplot coordinates{
(7,8.472e-02)    (31,3.044e-02)    (111,1.022e-02)
(351,3.303e-03)  (1023,1.039e-03)  (2815,3.196e-04)
(7423,9.658e-05) (18943,2.873e-05) (47103,8.437e-06)
};

\addplot coordinates{
(9,7.881e-02)     (49,3.243e-02)    (209,1.232e-02)
(769,4.454e-03)   (2561,1.551e-03)  (7937,5.236e-04)
(23297,1.723e-04) (65537,5.545e-05) (178177,1.751e-05)
};

\addplot coordinates{
(11,6.887e-02)    (71,3.177e-02)     (351,1.341e-02)
(1471,5.334e-03)  (5503,2.027e-03)   (18943,7.415e-04)
(61183,2.628e-04) (187903,9.063e-05) (553983,3.053e-05)
};

\addplot coordinates{
(13,5.755e-02)     (97,2.925e-02)     (545,1.351e-02)
(2561,5.842e-03)   (10625,2.397e-03)  (40193,9.414e-04)
(141569,3.564e-04) (471041,1.308e-04)
(1496065,4.670e-05)
};
}%


\bookmaketitle
\tableofcontents
\include{pgfplots.title_abstract_intro}
\include{pgfplots.preliminaries}
\include{pgfplots.intro}
\include{pgfplots.reference}
\include{pgfplots.libs}
\include{pgfplots.resources}
\include{pgfplots.importexport}
\include{pgfplots.basic.reference}

\printindex

\bibliographystyle{abbrv} %gerapali} %gerabbrv} %gerunsrt.bst} %gerabbrv}% gerplain}
\nocite{pgfplotstable}
\nocite{programmingnotes}
\bibliography{pgfplots}
\end{document}

% -----------------------------------------------------------------------------
% For Stefan Pinnow as reminder on what to look for when editing the manual
% -----------------------------------------------------------------------------
% There should be no line breaks in the following environments
% - |...|
% - \declareandlabel{...}
% - \verbpdfref{...}
% If "MakeTikzPictures" isn't running through check one of the externalized
% LOG files what is the cause of that.
% -----------------------------------------------------------------------------


%%%%%%%%%%%%%%%%%%%%%%%%%%%%%%%%%%%%%%%%%%%%%%%%%%%%%%%%%%%%%%%%%%%%%%%%%%%%%
%
% Package pgfplots.sty documentation.
%
% Copyright 2007/2008 by Christian Feuersaenger.
%
% This program is free software: you can redistribute it and/or modify
% it under the terms of the GNU General Public License as published by
% the Free Software Foundation, either version 3 of the License, or
% (at your option) any later version.
%
% This program is distributed in the hope that it will be useful,
% but WITHOUT ANY WARRANTY; without even the implied warranty of
% MERCHANTABILITY or FITNESS FOR A PARTICULAR PURPOSE.  See the
% GNU General Public License for more details.
%
% You should have received a copy of the GNU General Public License
% along with this program.  If not, see <http://www.gnu.org/licenses/>.
%
%
%%%%%%%%%%%%%%%%%%%%%%%%%%%%%%%%%%%%%%%%%%%%%%%%%%%%%%%%%%%%%%%%%%%%%%%%%%%%%
% SEE pgfplots-macros.tex as well!
%\pdfminorversion=5 % to allow compression
%\pdfobjcompresslevel=2
\documentclass[a4paper,openany]{book}

\let\bookmaketitle=\maketitle

% -----------------------------------
% this here is from ltxdoc:
\usepackage{doc}[=v2]
\AtBeginDocument{\MakeShortVerb{\|}}
%\setlength{\textwidth}{355pt}
%\addtolength\marginparwidth{30pt}
%\addtolength\oddsidemargin{20pt}
%\addtolength\evensidemargin{20pt}
\def\cmd#1{\cs{\expandafter\cmd@to@cs\string#1}}
\def\cmd@to@cs#1#2{\char\number`#2\relax}
\DeclareRobustCommand\cs[1]{\texttt{\char`\\#1}}
\providecommand\marg[1]{%
  {\ttfamily\char`\{}\meta{#1}{\ttfamily\char`\}}}
\providecommand\oarg[1]{%
  {\ttfamily[}\meta{#1}{\ttfamily]}}
\providecommand\parg[1]{%
  {\ttfamily(}\meta{#1}{\ttfamily)}}
\raggedbottom

% -----------------------------------
\input{pgfplots.preamble.tex}

\makeatletter
% I want a two-column index just as in pgfmanual styles. This here
% was the best way to get one:
\def\index@prologue{\section*{Index}\addcontentsline{toc}{chapter}{Index}
}
\makeatother

%\RequirePackage[german,english,francais]{babel}

\def\matlabcolormaptext{This colormap is similar to one shipped with Matlab$^\text{\textregistered}$ under a similar name.}

\IfFileExists{tikzlibraryspy.code.tex}{%
\usetikzlibrary{spy}
}{%
    \message{ERROR: tikz SPY library NOT available. The manual will only compile partially.^^J}%
}%

\usepackage{xparse}% for colorbrewer manual

\usetikzlibrary{
    decorations.markings,
    decorations.footprints,
    shapes.arrows,
    matrix,
    positioning,
}

\usepgfplotslibrary{
    fillbetween,
    ternary,
    smithchart,
    patchplots,
    polar,
    colormaps,
    colorbrewer,
    colortol,           % docu not ready yet
}
\pgfqkeys{/codeexample}{%
    every codeexample/.append style={
        /pgfplots/every ternary axis/.append style={
            /pgfplots/legend style={fill=graphicbackground},
        }
    },
    tabsize=4,
}

\pgfplotsmanualenableexternalizationofexpensive

%\usetikzlibrary{external}
%\tikzexternalize[prefix=figures/]{pgfplots}

\title{%
    Manual for Package \PGFPlots{}\\
    {\small 2D/3D Plots in \LaTeX{}, Version \pgfplotsversion}\\
    {\small\url{https://github.com/pgf-tikz/pgfplots}}
    %\\{\small Attention: you are using an unstable development version.}
}

\makeatletter
\long\def\abstractsmuggle{%
    \centering
    \textbf{Abstract}\\[0.5cm]

    \begin{minipage}{12cm}
        \PGFPlots{} draws high-quality function plots in normal or logarithmic
        scaling with a user-friendly interface directly in \TeX{}. The user
        supplies axis labels, legend entries and the plot coordinates for one or
        more plots and \PGFPlots{} applies axis scaling, computes any logarithms
        and axis ticks and draws the plots. It supports line plots, scatter
        plots, piecewise constant plots, bar plots, area plots, mesh and surface
        plots, patch plots, contour plots, quiver plots, histogram plots, box
        plots, polar axes, ternary diagrams, smith charts and some more. It is
        based on Till Tantau's package \PGF{}/\Tikz{}.
    \end{minipage}

}%

\expandafter\date\expandafter{\@date\\[2cm]
    \abstractsmuggle
}%
\makeatother

%\includeonly{pgfplots.reference}


\begin{document}

\def\plotcoords{%
\addplot coordinates {
(5,8.312e-02)    (17,2.547e-02)   (49,7.407e-03)
(129,2.102e-03)  (321,5.874e-04)  (769,1.623e-04)
(1793,4.442e-05) (4097,1.207e-05) (9217,3.261e-06)
};

\addplot coordinates{
(7,8.472e-02)    (31,3.044e-02)    (111,1.022e-02)
(351,3.303e-03)  (1023,1.039e-03)  (2815,3.196e-04)
(7423,9.658e-05) (18943,2.873e-05) (47103,8.437e-06)
};

\addplot coordinates{
(9,7.881e-02)     (49,3.243e-02)    (209,1.232e-02)
(769,4.454e-03)   (2561,1.551e-03)  (7937,5.236e-04)
(23297,1.723e-04) (65537,5.545e-05) (178177,1.751e-05)
};

\addplot coordinates{
(11,6.887e-02)    (71,3.177e-02)     (351,1.341e-02)
(1471,5.334e-03)  (5503,2.027e-03)   (18943,7.415e-04)
(61183,2.628e-04) (187903,9.063e-05) (553983,3.053e-05)
};

\addplot coordinates{
(13,5.755e-02)     (97,2.925e-02)     (545,1.351e-02)
(2561,5.842e-03)   (10625,2.397e-03)  (40193,9.414e-04)
(141569,3.564e-04) (471041,1.308e-04)
(1496065,4.670e-05)
};
}%


\bookmaketitle
\tableofcontents
\include{pgfplots.title_abstract_intro}
\include{pgfplots.preliminaries}
\include{pgfplots.intro}
\include{pgfplots.reference}
\include{pgfplots.libs}
\include{pgfplots.resources}
\include{pgfplots.importexport}
\include{pgfplots.basic.reference}

\printindex

\bibliographystyle{abbrv} %gerapali} %gerabbrv} %gerunsrt.bst} %gerabbrv}% gerplain}
\nocite{pgfplotstable}
\nocite{programmingnotes}
\bibliography{pgfplots}
\end{document}

% -----------------------------------------------------------------------------
% For Stefan Pinnow as reminder on what to look for when editing the manual
% -----------------------------------------------------------------------------
% There should be no line breaks in the following environments
% - |...|
% - \declareandlabel{...}
% - \verbpdfref{...}
% If "MakeTikzPictures" isn't running through check one of the externalized
% LOG files what is the cause of that.
% -----------------------------------------------------------------------------

\include{pgfplots.basic.reference}

\printindex

\bibliographystyle{abbrv} %gerapali} %gerabbrv} %gerunsrt.bst} %gerabbrv}% gerplain}
\nocite{pgfplotstable}
\nocite{programmingnotes}
\bibliography{pgfplots}
\end{document}

% -----------------------------------------------------------------------------
% For Stefan Pinnow as reminder on what to look for when editing the manual
% -----------------------------------------------------------------------------
% There should be no line breaks in the following environments
% - |...|
% - \declareandlabel{...}
% - \verbpdfref{...}
% If "MakeTikzPictures" isn't running through check one of the externalized
% LOG files what is the cause of that.
% -----------------------------------------------------------------------------


%%%%%%%%%%%%%%%%%%%%%%%%%%%%%%%%%%%%%%%%%%%%%%%%%%%%%%%%%%%%%%%%%%%%%%%%%%%%%
%
% Package pgfplots.sty documentation.
%
% Copyright 2007/2008 by Christian Feuersaenger.
%
% This program is free software: you can redistribute it and/or modify
% it under the terms of the GNU General Public License as published by
% the Free Software Foundation, either version 3 of the License, or
% (at your option) any later version.
%
% This program is distributed in the hope that it will be useful,
% but WITHOUT ANY WARRANTY; without even the implied warranty of
% MERCHANTABILITY or FITNESS FOR A PARTICULAR PURPOSE.  See the
% GNU General Public License for more details.
%
% You should have received a copy of the GNU General Public License
% along with this program.  If not, see <http://www.gnu.org/licenses/>.
%
%
%%%%%%%%%%%%%%%%%%%%%%%%%%%%%%%%%%%%%%%%%%%%%%%%%%%%%%%%%%%%%%%%%%%%%%%%%%%%%
% SEE pgfplots-macros.tex as well!
%\pdfminorversion=5 % to allow compression
%\pdfobjcompresslevel=2
\documentclass[a4paper,openany]{book}

\let\bookmaketitle=\maketitle

% -----------------------------------
% this here is from ltxdoc:
\usepackage{doc}[=v2]
\AtBeginDocument{\MakeShortVerb{\|}}
%\setlength{\textwidth}{355pt}
%\addtolength\marginparwidth{30pt}
%\addtolength\oddsidemargin{20pt}
%\addtolength\evensidemargin{20pt}
\def\cmd#1{\cs{\expandafter\cmd@to@cs\string#1}}
\def\cmd@to@cs#1#2{\char\number`#2\relax}
\DeclareRobustCommand\cs[1]{\texttt{\char`\\#1}}
\providecommand\marg[1]{%
  {\ttfamily\char`\{}\meta{#1}{\ttfamily\char`\}}}
\providecommand\oarg[1]{%
  {\ttfamily[}\meta{#1}{\ttfamily]}}
\providecommand\parg[1]{%
  {\ttfamily(}\meta{#1}{\ttfamily)}}
\raggedbottom

% -----------------------------------
\input{pgfplots.preamble.tex}

\makeatletter
% I want a two-column index just as in pgfmanual styles. This here
% was the best way to get one:
\def\index@prologue{\section*{Index}\addcontentsline{toc}{chapter}{Index}
}
\makeatother

%\RequirePackage[german,english,francais]{babel}

\def\matlabcolormaptext{This colormap is similar to one shipped with Matlab$^\text{\textregistered}$ under a similar name.}

\IfFileExists{tikzlibraryspy.code.tex}{%
\usetikzlibrary{spy}
}{%
    \message{ERROR: tikz SPY library NOT available. The manual will only compile partially.^^J}%
}%

\usepackage{xparse}% for colorbrewer manual

\usetikzlibrary{
    decorations.markings,
    decorations.footprints,
    shapes.arrows,
    matrix,
    positioning,
}

\usepgfplotslibrary{
    fillbetween,
    ternary,
    smithchart,
    patchplots,
    polar,
    colormaps,
    colorbrewer,
    colortol,           % docu not ready yet
}
\pgfqkeys{/codeexample}{%
    every codeexample/.append style={
        /pgfplots/every ternary axis/.append style={
            /pgfplots/legend style={fill=graphicbackground},
        }
    },
    tabsize=4,
}

\pgfplotsmanualenableexternalizationofexpensive

%\usetikzlibrary{external}
%\tikzexternalize[prefix=figures/]{pgfplots}

\title{%
    Manual for Package \PGFPlots{}\\
    {\small 2D/3D Plots in \LaTeX{}, Version \pgfplotsversion}\\
    {\small\url{https://github.com/pgf-tikz/pgfplots}}
    %\\{\small Attention: you are using an unstable development version.}
}

\makeatletter
\long\def\abstractsmuggle{%
    \centering
    \textbf{Abstract}\\[0.5cm]

    \begin{minipage}{12cm}
        \PGFPlots{} draws high-quality function plots in normal or logarithmic
        scaling with a user-friendly interface directly in \TeX{}. The user
        supplies axis labels, legend entries and the plot coordinates for one or
        more plots and \PGFPlots{} applies axis scaling, computes any logarithms
        and axis ticks and draws the plots. It supports line plots, scatter
        plots, piecewise constant plots, bar plots, area plots, mesh and surface
        plots, patch plots, contour plots, quiver plots, histogram plots, box
        plots, polar axes, ternary diagrams, smith charts and some more. It is
        based on Till Tantau's package \PGF{}/\Tikz{}.
    \end{minipage}

}%

\expandafter\date\expandafter{\@date\\[2cm]
    \abstractsmuggle
}%
\makeatother

%\includeonly{pgfplots.reference}


\begin{document}

\def\plotcoords{%
\addplot coordinates {
(5,8.312e-02)    (17,2.547e-02)   (49,7.407e-03)
(129,2.102e-03)  (321,5.874e-04)  (769,1.623e-04)
(1793,4.442e-05) (4097,1.207e-05) (9217,3.261e-06)
};

\addplot coordinates{
(7,8.472e-02)    (31,3.044e-02)    (111,1.022e-02)
(351,3.303e-03)  (1023,1.039e-03)  (2815,3.196e-04)
(7423,9.658e-05) (18943,2.873e-05) (47103,8.437e-06)
};

\addplot coordinates{
(9,7.881e-02)     (49,3.243e-02)    (209,1.232e-02)
(769,4.454e-03)   (2561,1.551e-03)  (7937,5.236e-04)
(23297,1.723e-04) (65537,5.545e-05) (178177,1.751e-05)
};

\addplot coordinates{
(11,6.887e-02)    (71,3.177e-02)     (351,1.341e-02)
(1471,5.334e-03)  (5503,2.027e-03)   (18943,7.415e-04)
(61183,2.628e-04) (187903,9.063e-05) (553983,3.053e-05)
};

\addplot coordinates{
(13,5.755e-02)     (97,2.925e-02)     (545,1.351e-02)
(2561,5.842e-03)   (10625,2.397e-03)  (40193,9.414e-04)
(141569,3.564e-04) (471041,1.308e-04)
(1496065,4.670e-05)
};
}%


\bookmaketitle
\tableofcontents

%%%%%%%%%%%%%%%%%%%%%%%%%%%%%%%%%%%%%%%%%%%%%%%%%%%%%%%%%%%%%%%%%%%%%%%%%%%%%
%
% Package pgfplots.sty documentation.
%
% Copyright 2007/2008 by Christian Feuersaenger.
%
% This program is free software: you can redistribute it and/or modify
% it under the terms of the GNU General Public License as published by
% the Free Software Foundation, either version 3 of the License, or
% (at your option) any later version.
%
% This program is distributed in the hope that it will be useful,
% but WITHOUT ANY WARRANTY; without even the implied warranty of
% MERCHANTABILITY or FITNESS FOR A PARTICULAR PURPOSE.  See the
% GNU General Public License for more details.
%
% You should have received a copy of the GNU General Public License
% along with this program.  If not, see <http://www.gnu.org/licenses/>.
%
%
%%%%%%%%%%%%%%%%%%%%%%%%%%%%%%%%%%%%%%%%%%%%%%%%%%%%%%%%%%%%%%%%%%%%%%%%%%%%%
% SEE pgfplots-macros.tex as well!
%\pdfminorversion=5 % to allow compression
%\pdfobjcompresslevel=2
\documentclass[a4paper,openany]{book}

\let\bookmaketitle=\maketitle

% -----------------------------------
% this here is from ltxdoc:
\usepackage{doc}[=v2]
\AtBeginDocument{\MakeShortVerb{\|}}
%\setlength{\textwidth}{355pt}
%\addtolength\marginparwidth{30pt}
%\addtolength\oddsidemargin{20pt}
%\addtolength\evensidemargin{20pt}
\def\cmd#1{\cs{\expandafter\cmd@to@cs\string#1}}
\def\cmd@to@cs#1#2{\char\number`#2\relax}
\DeclareRobustCommand\cs[1]{\texttt{\char`\\#1}}
\providecommand\marg[1]{%
  {\ttfamily\char`\{}\meta{#1}{\ttfamily\char`\}}}
\providecommand\oarg[1]{%
  {\ttfamily[}\meta{#1}{\ttfamily]}}
\providecommand\parg[1]{%
  {\ttfamily(}\meta{#1}{\ttfamily)}}
\raggedbottom

% -----------------------------------
\input{pgfplots.preamble.tex}

\makeatletter
% I want a two-column index just as in pgfmanual styles. This here
% was the best way to get one:
\def\index@prologue{\section*{Index}\addcontentsline{toc}{chapter}{Index}
}
\makeatother

%\RequirePackage[german,english,francais]{babel}

\def\matlabcolormaptext{This colormap is similar to one shipped with Matlab$^\text{\textregistered}$ under a similar name.}

\IfFileExists{tikzlibraryspy.code.tex}{%
\usetikzlibrary{spy}
}{%
    \message{ERROR: tikz SPY library NOT available. The manual will only compile partially.^^J}%
}%

\usepackage{xparse}% for colorbrewer manual

\usetikzlibrary{
    decorations.markings,
    decorations.footprints,
    shapes.arrows,
    matrix,
    positioning,
}

\usepgfplotslibrary{
    fillbetween,
    ternary,
    smithchart,
    patchplots,
    polar,
    colormaps,
    colorbrewer,
    colortol,           % docu not ready yet
}
\pgfqkeys{/codeexample}{%
    every codeexample/.append style={
        /pgfplots/every ternary axis/.append style={
            /pgfplots/legend style={fill=graphicbackground},
        }
    },
    tabsize=4,
}

\pgfplotsmanualenableexternalizationofexpensive

%\usetikzlibrary{external}
%\tikzexternalize[prefix=figures/]{pgfplots}

\title{%
    Manual for Package \PGFPlots{}\\
    {\small 2D/3D Plots in \LaTeX{}, Version \pgfplotsversion}\\
    {\small\url{https://github.com/pgf-tikz/pgfplots}}
    %\\{\small Attention: you are using an unstable development version.}
}

\makeatletter
\long\def\abstractsmuggle{%
    \centering
    \textbf{Abstract}\\[0.5cm]

    \begin{minipage}{12cm}
        \PGFPlots{} draws high-quality function plots in normal or logarithmic
        scaling with a user-friendly interface directly in \TeX{}. The user
        supplies axis labels, legend entries and the plot coordinates for one or
        more plots and \PGFPlots{} applies axis scaling, computes any logarithms
        and axis ticks and draws the plots. It supports line plots, scatter
        plots, piecewise constant plots, bar plots, area plots, mesh and surface
        plots, patch plots, contour plots, quiver plots, histogram plots, box
        plots, polar axes, ternary diagrams, smith charts and some more. It is
        based on Till Tantau's package \PGF{}/\Tikz{}.
    \end{minipage}

}%

\expandafter\date\expandafter{\@date\\[2cm]
    \abstractsmuggle
}%
\makeatother

%\includeonly{pgfplots.reference}


\begin{document}

\def\plotcoords{%
\addplot coordinates {
(5,8.312e-02)    (17,2.547e-02)   (49,7.407e-03)
(129,2.102e-03)  (321,5.874e-04)  (769,1.623e-04)
(1793,4.442e-05) (4097,1.207e-05) (9217,3.261e-06)
};

\addplot coordinates{
(7,8.472e-02)    (31,3.044e-02)    (111,1.022e-02)
(351,3.303e-03)  (1023,1.039e-03)  (2815,3.196e-04)
(7423,9.658e-05) (18943,2.873e-05) (47103,8.437e-06)
};

\addplot coordinates{
(9,7.881e-02)     (49,3.243e-02)    (209,1.232e-02)
(769,4.454e-03)   (2561,1.551e-03)  (7937,5.236e-04)
(23297,1.723e-04) (65537,5.545e-05) (178177,1.751e-05)
};

\addplot coordinates{
(11,6.887e-02)    (71,3.177e-02)     (351,1.341e-02)
(1471,5.334e-03)  (5503,2.027e-03)   (18943,7.415e-04)
(61183,2.628e-04) (187903,9.063e-05) (553983,3.053e-05)
};

\addplot coordinates{
(13,5.755e-02)     (97,2.925e-02)     (545,1.351e-02)
(2561,5.842e-03)   (10625,2.397e-03)  (40193,9.414e-04)
(141569,3.564e-04) (471041,1.308e-04)
(1496065,4.670e-05)
};
}%


\bookmaketitle
\tableofcontents
\include{pgfplots.title_abstract_intro}
\include{pgfplots.preliminaries}
\include{pgfplots.intro}
\include{pgfplots.reference}
\include{pgfplots.libs}
\include{pgfplots.resources}
\include{pgfplots.importexport}
\include{pgfplots.basic.reference}

\printindex

\bibliographystyle{abbrv} %gerapali} %gerabbrv} %gerunsrt.bst} %gerabbrv}% gerplain}
\nocite{pgfplotstable}
\nocite{programmingnotes}
\bibliography{pgfplots}
\end{document}

% -----------------------------------------------------------------------------
% For Stefan Pinnow as reminder on what to look for when editing the manual
% -----------------------------------------------------------------------------
% There should be no line breaks in the following environments
% - |...|
% - \declareandlabel{...}
% - \verbpdfref{...}
% If "MakeTikzPictures" isn't running through check one of the externalized
% LOG files what is the cause of that.
% -----------------------------------------------------------------------------


%%%%%%%%%%%%%%%%%%%%%%%%%%%%%%%%%%%%%%%%%%%%%%%%%%%%%%%%%%%%%%%%%%%%%%%%%%%%%
%
% Package pgfplots.sty documentation.
%
% Copyright 2007/2008 by Christian Feuersaenger.
%
% This program is free software: you can redistribute it and/or modify
% it under the terms of the GNU General Public License as published by
% the Free Software Foundation, either version 3 of the License, or
% (at your option) any later version.
%
% This program is distributed in the hope that it will be useful,
% but WITHOUT ANY WARRANTY; without even the implied warranty of
% MERCHANTABILITY or FITNESS FOR A PARTICULAR PURPOSE.  See the
% GNU General Public License for more details.
%
% You should have received a copy of the GNU General Public License
% along with this program.  If not, see <http://www.gnu.org/licenses/>.
%
%
%%%%%%%%%%%%%%%%%%%%%%%%%%%%%%%%%%%%%%%%%%%%%%%%%%%%%%%%%%%%%%%%%%%%%%%%%%%%%
% SEE pgfplots-macros.tex as well!
%\pdfminorversion=5 % to allow compression
%\pdfobjcompresslevel=2
\documentclass[a4paper,openany]{book}

\let\bookmaketitle=\maketitle

% -----------------------------------
% this here is from ltxdoc:
\usepackage{doc}[=v2]
\AtBeginDocument{\MakeShortVerb{\|}}
%\setlength{\textwidth}{355pt}
%\addtolength\marginparwidth{30pt}
%\addtolength\oddsidemargin{20pt}
%\addtolength\evensidemargin{20pt}
\def\cmd#1{\cs{\expandafter\cmd@to@cs\string#1}}
\def\cmd@to@cs#1#2{\char\number`#2\relax}
\DeclareRobustCommand\cs[1]{\texttt{\char`\\#1}}
\providecommand\marg[1]{%
  {\ttfamily\char`\{}\meta{#1}{\ttfamily\char`\}}}
\providecommand\oarg[1]{%
  {\ttfamily[}\meta{#1}{\ttfamily]}}
\providecommand\parg[1]{%
  {\ttfamily(}\meta{#1}{\ttfamily)}}
\raggedbottom

% -----------------------------------
\input{pgfplots.preamble.tex}

\makeatletter
% I want a two-column index just as in pgfmanual styles. This here
% was the best way to get one:
\def\index@prologue{\section*{Index}\addcontentsline{toc}{chapter}{Index}
}
\makeatother

%\RequirePackage[german,english,francais]{babel}

\def\matlabcolormaptext{This colormap is similar to one shipped with Matlab$^\text{\textregistered}$ under a similar name.}

\IfFileExists{tikzlibraryspy.code.tex}{%
\usetikzlibrary{spy}
}{%
    \message{ERROR: tikz SPY library NOT available. The manual will only compile partially.^^J}%
}%

\usepackage{xparse}% for colorbrewer manual

\usetikzlibrary{
    decorations.markings,
    decorations.footprints,
    shapes.arrows,
    matrix,
    positioning,
}

\usepgfplotslibrary{
    fillbetween,
    ternary,
    smithchart,
    patchplots,
    polar,
    colormaps,
    colorbrewer,
    colortol,           % docu not ready yet
}
\pgfqkeys{/codeexample}{%
    every codeexample/.append style={
        /pgfplots/every ternary axis/.append style={
            /pgfplots/legend style={fill=graphicbackground},
        }
    },
    tabsize=4,
}

\pgfplotsmanualenableexternalizationofexpensive

%\usetikzlibrary{external}
%\tikzexternalize[prefix=figures/]{pgfplots}

\title{%
    Manual for Package \PGFPlots{}\\
    {\small 2D/3D Plots in \LaTeX{}, Version \pgfplotsversion}\\
    {\small\url{https://github.com/pgf-tikz/pgfplots}}
    %\\{\small Attention: you are using an unstable development version.}
}

\makeatletter
\long\def\abstractsmuggle{%
    \centering
    \textbf{Abstract}\\[0.5cm]

    \begin{minipage}{12cm}
        \PGFPlots{} draws high-quality function plots in normal or logarithmic
        scaling with a user-friendly interface directly in \TeX{}. The user
        supplies axis labels, legend entries and the plot coordinates for one or
        more plots and \PGFPlots{} applies axis scaling, computes any logarithms
        and axis ticks and draws the plots. It supports line plots, scatter
        plots, piecewise constant plots, bar plots, area plots, mesh and surface
        plots, patch plots, contour plots, quiver plots, histogram plots, box
        plots, polar axes, ternary diagrams, smith charts and some more. It is
        based on Till Tantau's package \PGF{}/\Tikz{}.
    \end{minipage}

}%

\expandafter\date\expandafter{\@date\\[2cm]
    \abstractsmuggle
}%
\makeatother

%\includeonly{pgfplots.reference}


\begin{document}

\def\plotcoords{%
\addplot coordinates {
(5,8.312e-02)    (17,2.547e-02)   (49,7.407e-03)
(129,2.102e-03)  (321,5.874e-04)  (769,1.623e-04)
(1793,4.442e-05) (4097,1.207e-05) (9217,3.261e-06)
};

\addplot coordinates{
(7,8.472e-02)    (31,3.044e-02)    (111,1.022e-02)
(351,3.303e-03)  (1023,1.039e-03)  (2815,3.196e-04)
(7423,9.658e-05) (18943,2.873e-05) (47103,8.437e-06)
};

\addplot coordinates{
(9,7.881e-02)     (49,3.243e-02)    (209,1.232e-02)
(769,4.454e-03)   (2561,1.551e-03)  (7937,5.236e-04)
(23297,1.723e-04) (65537,5.545e-05) (178177,1.751e-05)
};

\addplot coordinates{
(11,6.887e-02)    (71,3.177e-02)     (351,1.341e-02)
(1471,5.334e-03)  (5503,2.027e-03)   (18943,7.415e-04)
(61183,2.628e-04) (187903,9.063e-05) (553983,3.053e-05)
};

\addplot coordinates{
(13,5.755e-02)     (97,2.925e-02)     (545,1.351e-02)
(2561,5.842e-03)   (10625,2.397e-03)  (40193,9.414e-04)
(141569,3.564e-04) (471041,1.308e-04)
(1496065,4.670e-05)
};
}%


\bookmaketitle
\tableofcontents
\include{pgfplots.title_abstract_intro}
\include{pgfplots.preliminaries}
\include{pgfplots.intro}
\include{pgfplots.reference}
\include{pgfplots.libs}
\include{pgfplots.resources}
\include{pgfplots.importexport}
\include{pgfplots.basic.reference}

\printindex

\bibliographystyle{abbrv} %gerapali} %gerabbrv} %gerunsrt.bst} %gerabbrv}% gerplain}
\nocite{pgfplotstable}
\nocite{programmingnotes}
\bibliography{pgfplots}
\end{document}

% -----------------------------------------------------------------------------
% For Stefan Pinnow as reminder on what to look for when editing the manual
% -----------------------------------------------------------------------------
% There should be no line breaks in the following environments
% - |...|
% - \declareandlabel{...}
% - \verbpdfref{...}
% If "MakeTikzPictures" isn't running through check one of the externalized
% LOG files what is the cause of that.
% -----------------------------------------------------------------------------


%%%%%%%%%%%%%%%%%%%%%%%%%%%%%%%%%%%%%%%%%%%%%%%%%%%%%%%%%%%%%%%%%%%%%%%%%%%%%
%
% Package pgfplots.sty documentation.
%
% Copyright 2007/2008 by Christian Feuersaenger.
%
% This program is free software: you can redistribute it and/or modify
% it under the terms of the GNU General Public License as published by
% the Free Software Foundation, either version 3 of the License, or
% (at your option) any later version.
%
% This program is distributed in the hope that it will be useful,
% but WITHOUT ANY WARRANTY; without even the implied warranty of
% MERCHANTABILITY or FITNESS FOR A PARTICULAR PURPOSE.  See the
% GNU General Public License for more details.
%
% You should have received a copy of the GNU General Public License
% along with this program.  If not, see <http://www.gnu.org/licenses/>.
%
%
%%%%%%%%%%%%%%%%%%%%%%%%%%%%%%%%%%%%%%%%%%%%%%%%%%%%%%%%%%%%%%%%%%%%%%%%%%%%%
% SEE pgfplots-macros.tex as well!
%\pdfminorversion=5 % to allow compression
%\pdfobjcompresslevel=2
\documentclass[a4paper,openany]{book}

\let\bookmaketitle=\maketitle

% -----------------------------------
% this here is from ltxdoc:
\usepackage{doc}[=v2]
\AtBeginDocument{\MakeShortVerb{\|}}
%\setlength{\textwidth}{355pt}
%\addtolength\marginparwidth{30pt}
%\addtolength\oddsidemargin{20pt}
%\addtolength\evensidemargin{20pt}
\def\cmd#1{\cs{\expandafter\cmd@to@cs\string#1}}
\def\cmd@to@cs#1#2{\char\number`#2\relax}
\DeclareRobustCommand\cs[1]{\texttt{\char`\\#1}}
\providecommand\marg[1]{%
  {\ttfamily\char`\{}\meta{#1}{\ttfamily\char`\}}}
\providecommand\oarg[1]{%
  {\ttfamily[}\meta{#1}{\ttfamily]}}
\providecommand\parg[1]{%
  {\ttfamily(}\meta{#1}{\ttfamily)}}
\raggedbottom

% -----------------------------------
\input{pgfplots.preamble.tex}

\makeatletter
% I want a two-column index just as in pgfmanual styles. This here
% was the best way to get one:
\def\index@prologue{\section*{Index}\addcontentsline{toc}{chapter}{Index}
}
\makeatother

%\RequirePackage[german,english,francais]{babel}

\def\matlabcolormaptext{This colormap is similar to one shipped with Matlab$^\text{\textregistered}$ under a similar name.}

\IfFileExists{tikzlibraryspy.code.tex}{%
\usetikzlibrary{spy}
}{%
    \message{ERROR: tikz SPY library NOT available. The manual will only compile partially.^^J}%
}%

\usepackage{xparse}% for colorbrewer manual

\usetikzlibrary{
    decorations.markings,
    decorations.footprints,
    shapes.arrows,
    matrix,
    positioning,
}

\usepgfplotslibrary{
    fillbetween,
    ternary,
    smithchart,
    patchplots,
    polar,
    colormaps,
    colorbrewer,
    colortol,           % docu not ready yet
}
\pgfqkeys{/codeexample}{%
    every codeexample/.append style={
        /pgfplots/every ternary axis/.append style={
            /pgfplots/legend style={fill=graphicbackground},
        }
    },
    tabsize=4,
}

\pgfplotsmanualenableexternalizationofexpensive

%\usetikzlibrary{external}
%\tikzexternalize[prefix=figures/]{pgfplots}

\title{%
    Manual for Package \PGFPlots{}\\
    {\small 2D/3D Plots in \LaTeX{}, Version \pgfplotsversion}\\
    {\small\url{https://github.com/pgf-tikz/pgfplots}}
    %\\{\small Attention: you are using an unstable development version.}
}

\makeatletter
\long\def\abstractsmuggle{%
    \centering
    \textbf{Abstract}\\[0.5cm]

    \begin{minipage}{12cm}
        \PGFPlots{} draws high-quality function plots in normal or logarithmic
        scaling with a user-friendly interface directly in \TeX{}. The user
        supplies axis labels, legend entries and the plot coordinates for one or
        more plots and \PGFPlots{} applies axis scaling, computes any logarithms
        and axis ticks and draws the plots. It supports line plots, scatter
        plots, piecewise constant plots, bar plots, area plots, mesh and surface
        plots, patch plots, contour plots, quiver plots, histogram plots, box
        plots, polar axes, ternary diagrams, smith charts and some more. It is
        based on Till Tantau's package \PGF{}/\Tikz{}.
    \end{minipage}

}%

\expandafter\date\expandafter{\@date\\[2cm]
    \abstractsmuggle
}%
\makeatother

%\includeonly{pgfplots.reference}


\begin{document}

\def\plotcoords{%
\addplot coordinates {
(5,8.312e-02)    (17,2.547e-02)   (49,7.407e-03)
(129,2.102e-03)  (321,5.874e-04)  (769,1.623e-04)
(1793,4.442e-05) (4097,1.207e-05) (9217,3.261e-06)
};

\addplot coordinates{
(7,8.472e-02)    (31,3.044e-02)    (111,1.022e-02)
(351,3.303e-03)  (1023,1.039e-03)  (2815,3.196e-04)
(7423,9.658e-05) (18943,2.873e-05) (47103,8.437e-06)
};

\addplot coordinates{
(9,7.881e-02)     (49,3.243e-02)    (209,1.232e-02)
(769,4.454e-03)   (2561,1.551e-03)  (7937,5.236e-04)
(23297,1.723e-04) (65537,5.545e-05) (178177,1.751e-05)
};

\addplot coordinates{
(11,6.887e-02)    (71,3.177e-02)     (351,1.341e-02)
(1471,5.334e-03)  (5503,2.027e-03)   (18943,7.415e-04)
(61183,2.628e-04) (187903,9.063e-05) (553983,3.053e-05)
};

\addplot coordinates{
(13,5.755e-02)     (97,2.925e-02)     (545,1.351e-02)
(2561,5.842e-03)   (10625,2.397e-03)  (40193,9.414e-04)
(141569,3.564e-04) (471041,1.308e-04)
(1496065,4.670e-05)
};
}%


\bookmaketitle
\tableofcontents
\include{pgfplots.title_abstract_intro}
\include{pgfplots.preliminaries}
\include{pgfplots.intro}
\include{pgfplots.reference}
\include{pgfplots.libs}
\include{pgfplots.resources}
\include{pgfplots.importexport}
\include{pgfplots.basic.reference}

\printindex

\bibliographystyle{abbrv} %gerapali} %gerabbrv} %gerunsrt.bst} %gerabbrv}% gerplain}
\nocite{pgfplotstable}
\nocite{programmingnotes}
\bibliography{pgfplots}
\end{document}

% -----------------------------------------------------------------------------
% For Stefan Pinnow as reminder on what to look for when editing the manual
% -----------------------------------------------------------------------------
% There should be no line breaks in the following environments
% - |...|
% - \declareandlabel{...}
% - \verbpdfref{...}
% If "MakeTikzPictures" isn't running through check one of the externalized
% LOG files what is the cause of that.
% -----------------------------------------------------------------------------


%%%%%%%%%%%%%%%%%%%%%%%%%%%%%%%%%%%%%%%%%%%%%%%%%%%%%%%%%%%%%%%%%%%%%%%%%%%%%
%
% Package pgfplots.sty documentation.
%
% Copyright 2007/2008 by Christian Feuersaenger.
%
% This program is free software: you can redistribute it and/or modify
% it under the terms of the GNU General Public License as published by
% the Free Software Foundation, either version 3 of the License, or
% (at your option) any later version.
%
% This program is distributed in the hope that it will be useful,
% but WITHOUT ANY WARRANTY; without even the implied warranty of
% MERCHANTABILITY or FITNESS FOR A PARTICULAR PURPOSE.  See the
% GNU General Public License for more details.
%
% You should have received a copy of the GNU General Public License
% along with this program.  If not, see <http://www.gnu.org/licenses/>.
%
%
%%%%%%%%%%%%%%%%%%%%%%%%%%%%%%%%%%%%%%%%%%%%%%%%%%%%%%%%%%%%%%%%%%%%%%%%%%%%%
% SEE pgfplots-macros.tex as well!
%\pdfminorversion=5 % to allow compression
%\pdfobjcompresslevel=2
\documentclass[a4paper,openany]{book}

\let\bookmaketitle=\maketitle

% -----------------------------------
% this here is from ltxdoc:
\usepackage{doc}[=v2]
\AtBeginDocument{\MakeShortVerb{\|}}
%\setlength{\textwidth}{355pt}
%\addtolength\marginparwidth{30pt}
%\addtolength\oddsidemargin{20pt}
%\addtolength\evensidemargin{20pt}
\def\cmd#1{\cs{\expandafter\cmd@to@cs\string#1}}
\def\cmd@to@cs#1#2{\char\number`#2\relax}
\DeclareRobustCommand\cs[1]{\texttt{\char`\\#1}}
\providecommand\marg[1]{%
  {\ttfamily\char`\{}\meta{#1}{\ttfamily\char`\}}}
\providecommand\oarg[1]{%
  {\ttfamily[}\meta{#1}{\ttfamily]}}
\providecommand\parg[1]{%
  {\ttfamily(}\meta{#1}{\ttfamily)}}
\raggedbottom

% -----------------------------------
\input{pgfplots.preamble.tex}

\makeatletter
% I want a two-column index just as in pgfmanual styles. This here
% was the best way to get one:
\def\index@prologue{\section*{Index}\addcontentsline{toc}{chapter}{Index}
}
\makeatother

%\RequirePackage[german,english,francais]{babel}

\def\matlabcolormaptext{This colormap is similar to one shipped with Matlab$^\text{\textregistered}$ under a similar name.}

\IfFileExists{tikzlibraryspy.code.tex}{%
\usetikzlibrary{spy}
}{%
    \message{ERROR: tikz SPY library NOT available. The manual will only compile partially.^^J}%
}%

\usepackage{xparse}% for colorbrewer manual

\usetikzlibrary{
    decorations.markings,
    decorations.footprints,
    shapes.arrows,
    matrix,
    positioning,
}

\usepgfplotslibrary{
    fillbetween,
    ternary,
    smithchart,
    patchplots,
    polar,
    colormaps,
    colorbrewer,
    colortol,           % docu not ready yet
}
\pgfqkeys{/codeexample}{%
    every codeexample/.append style={
        /pgfplots/every ternary axis/.append style={
            /pgfplots/legend style={fill=graphicbackground},
        }
    },
    tabsize=4,
}

\pgfplotsmanualenableexternalizationofexpensive

%\usetikzlibrary{external}
%\tikzexternalize[prefix=figures/]{pgfplots}

\title{%
    Manual for Package \PGFPlots{}\\
    {\small 2D/3D Plots in \LaTeX{}, Version \pgfplotsversion}\\
    {\small\url{https://github.com/pgf-tikz/pgfplots}}
    %\\{\small Attention: you are using an unstable development version.}
}

\makeatletter
\long\def\abstractsmuggle{%
    \centering
    \textbf{Abstract}\\[0.5cm]

    \begin{minipage}{12cm}
        \PGFPlots{} draws high-quality function plots in normal or logarithmic
        scaling with a user-friendly interface directly in \TeX{}. The user
        supplies axis labels, legend entries and the plot coordinates for one or
        more plots and \PGFPlots{} applies axis scaling, computes any logarithms
        and axis ticks and draws the plots. It supports line plots, scatter
        plots, piecewise constant plots, bar plots, area plots, mesh and surface
        plots, patch plots, contour plots, quiver plots, histogram plots, box
        plots, polar axes, ternary diagrams, smith charts and some more. It is
        based on Till Tantau's package \PGF{}/\Tikz{}.
    \end{minipage}

}%

\expandafter\date\expandafter{\@date\\[2cm]
    \abstractsmuggle
}%
\makeatother

%\includeonly{pgfplots.reference}


\begin{document}

\def\plotcoords{%
\addplot coordinates {
(5,8.312e-02)    (17,2.547e-02)   (49,7.407e-03)
(129,2.102e-03)  (321,5.874e-04)  (769,1.623e-04)
(1793,4.442e-05) (4097,1.207e-05) (9217,3.261e-06)
};

\addplot coordinates{
(7,8.472e-02)    (31,3.044e-02)    (111,1.022e-02)
(351,3.303e-03)  (1023,1.039e-03)  (2815,3.196e-04)
(7423,9.658e-05) (18943,2.873e-05) (47103,8.437e-06)
};

\addplot coordinates{
(9,7.881e-02)     (49,3.243e-02)    (209,1.232e-02)
(769,4.454e-03)   (2561,1.551e-03)  (7937,5.236e-04)
(23297,1.723e-04) (65537,5.545e-05) (178177,1.751e-05)
};

\addplot coordinates{
(11,6.887e-02)    (71,3.177e-02)     (351,1.341e-02)
(1471,5.334e-03)  (5503,2.027e-03)   (18943,7.415e-04)
(61183,2.628e-04) (187903,9.063e-05) (553983,3.053e-05)
};

\addplot coordinates{
(13,5.755e-02)     (97,2.925e-02)     (545,1.351e-02)
(2561,5.842e-03)   (10625,2.397e-03)  (40193,9.414e-04)
(141569,3.564e-04) (471041,1.308e-04)
(1496065,4.670e-05)
};
}%


\bookmaketitle
\tableofcontents
\include{pgfplots.title_abstract_intro}
\include{pgfplots.preliminaries}
\include{pgfplots.intro}
\include{pgfplots.reference}
\include{pgfplots.libs}
\include{pgfplots.resources}
\include{pgfplots.importexport}
\include{pgfplots.basic.reference}

\printindex

\bibliographystyle{abbrv} %gerapali} %gerabbrv} %gerunsrt.bst} %gerabbrv}% gerplain}
\nocite{pgfplotstable}
\nocite{programmingnotes}
\bibliography{pgfplots}
\end{document}

% -----------------------------------------------------------------------------
% For Stefan Pinnow as reminder on what to look for when editing the manual
% -----------------------------------------------------------------------------
% There should be no line breaks in the following environments
% - |...|
% - \declareandlabel{...}
% - \verbpdfref{...}
% If "MakeTikzPictures" isn't running through check one of the externalized
% LOG files what is the cause of that.
% -----------------------------------------------------------------------------


%%%%%%%%%%%%%%%%%%%%%%%%%%%%%%%%%%%%%%%%%%%%%%%%%%%%%%%%%%%%%%%%%%%%%%%%%%%%%
%
% Package pgfplots.sty documentation.
%
% Copyright 2007/2008 by Christian Feuersaenger.
%
% This program is free software: you can redistribute it and/or modify
% it under the terms of the GNU General Public License as published by
% the Free Software Foundation, either version 3 of the License, or
% (at your option) any later version.
%
% This program is distributed in the hope that it will be useful,
% but WITHOUT ANY WARRANTY; without even the implied warranty of
% MERCHANTABILITY or FITNESS FOR A PARTICULAR PURPOSE.  See the
% GNU General Public License for more details.
%
% You should have received a copy of the GNU General Public License
% along with this program.  If not, see <http://www.gnu.org/licenses/>.
%
%
%%%%%%%%%%%%%%%%%%%%%%%%%%%%%%%%%%%%%%%%%%%%%%%%%%%%%%%%%%%%%%%%%%%%%%%%%%%%%
% SEE pgfplots-macros.tex as well!
%\pdfminorversion=5 % to allow compression
%\pdfobjcompresslevel=2
\documentclass[a4paper,openany]{book}

\let\bookmaketitle=\maketitle

% -----------------------------------
% this here is from ltxdoc:
\usepackage{doc}[=v2]
\AtBeginDocument{\MakeShortVerb{\|}}
%\setlength{\textwidth}{355pt}
%\addtolength\marginparwidth{30pt}
%\addtolength\oddsidemargin{20pt}
%\addtolength\evensidemargin{20pt}
\def\cmd#1{\cs{\expandafter\cmd@to@cs\string#1}}
\def\cmd@to@cs#1#2{\char\number`#2\relax}
\DeclareRobustCommand\cs[1]{\texttt{\char`\\#1}}
\providecommand\marg[1]{%
  {\ttfamily\char`\{}\meta{#1}{\ttfamily\char`\}}}
\providecommand\oarg[1]{%
  {\ttfamily[}\meta{#1}{\ttfamily]}}
\providecommand\parg[1]{%
  {\ttfamily(}\meta{#1}{\ttfamily)}}
\raggedbottom

% -----------------------------------
\input{pgfplots.preamble.tex}

\makeatletter
% I want a two-column index just as in pgfmanual styles. This here
% was the best way to get one:
\def\index@prologue{\section*{Index}\addcontentsline{toc}{chapter}{Index}
}
\makeatother

%\RequirePackage[german,english,francais]{babel}

\def\matlabcolormaptext{This colormap is similar to one shipped with Matlab$^\text{\textregistered}$ under a similar name.}

\IfFileExists{tikzlibraryspy.code.tex}{%
\usetikzlibrary{spy}
}{%
    \message{ERROR: tikz SPY library NOT available. The manual will only compile partially.^^J}%
}%

\usepackage{xparse}% for colorbrewer manual

\usetikzlibrary{
    decorations.markings,
    decorations.footprints,
    shapes.arrows,
    matrix,
    positioning,
}

\usepgfplotslibrary{
    fillbetween,
    ternary,
    smithchart,
    patchplots,
    polar,
    colormaps,
    colorbrewer,
    colortol,           % docu not ready yet
}
\pgfqkeys{/codeexample}{%
    every codeexample/.append style={
        /pgfplots/every ternary axis/.append style={
            /pgfplots/legend style={fill=graphicbackground},
        }
    },
    tabsize=4,
}

\pgfplotsmanualenableexternalizationofexpensive

%\usetikzlibrary{external}
%\tikzexternalize[prefix=figures/]{pgfplots}

\title{%
    Manual for Package \PGFPlots{}\\
    {\small 2D/3D Plots in \LaTeX{}, Version \pgfplotsversion}\\
    {\small\url{https://github.com/pgf-tikz/pgfplots}}
    %\\{\small Attention: you are using an unstable development version.}
}

\makeatletter
\long\def\abstractsmuggle{%
    \centering
    \textbf{Abstract}\\[0.5cm]

    \begin{minipage}{12cm}
        \PGFPlots{} draws high-quality function plots in normal or logarithmic
        scaling with a user-friendly interface directly in \TeX{}. The user
        supplies axis labels, legend entries and the plot coordinates for one or
        more plots and \PGFPlots{} applies axis scaling, computes any logarithms
        and axis ticks and draws the plots. It supports line plots, scatter
        plots, piecewise constant plots, bar plots, area plots, mesh and surface
        plots, patch plots, contour plots, quiver plots, histogram plots, box
        plots, polar axes, ternary diagrams, smith charts and some more. It is
        based on Till Tantau's package \PGF{}/\Tikz{}.
    \end{minipage}

}%

\expandafter\date\expandafter{\@date\\[2cm]
    \abstractsmuggle
}%
\makeatother

%\includeonly{pgfplots.reference}


\begin{document}

\def\plotcoords{%
\addplot coordinates {
(5,8.312e-02)    (17,2.547e-02)   (49,7.407e-03)
(129,2.102e-03)  (321,5.874e-04)  (769,1.623e-04)
(1793,4.442e-05) (4097,1.207e-05) (9217,3.261e-06)
};

\addplot coordinates{
(7,8.472e-02)    (31,3.044e-02)    (111,1.022e-02)
(351,3.303e-03)  (1023,1.039e-03)  (2815,3.196e-04)
(7423,9.658e-05) (18943,2.873e-05) (47103,8.437e-06)
};

\addplot coordinates{
(9,7.881e-02)     (49,3.243e-02)    (209,1.232e-02)
(769,4.454e-03)   (2561,1.551e-03)  (7937,5.236e-04)
(23297,1.723e-04) (65537,5.545e-05) (178177,1.751e-05)
};

\addplot coordinates{
(11,6.887e-02)    (71,3.177e-02)     (351,1.341e-02)
(1471,5.334e-03)  (5503,2.027e-03)   (18943,7.415e-04)
(61183,2.628e-04) (187903,9.063e-05) (553983,3.053e-05)
};

\addplot coordinates{
(13,5.755e-02)     (97,2.925e-02)     (545,1.351e-02)
(2561,5.842e-03)   (10625,2.397e-03)  (40193,9.414e-04)
(141569,3.564e-04) (471041,1.308e-04)
(1496065,4.670e-05)
};
}%


\bookmaketitle
\tableofcontents
\include{pgfplots.title_abstract_intro}
\include{pgfplots.preliminaries}
\include{pgfplots.intro}
\include{pgfplots.reference}
\include{pgfplots.libs}
\include{pgfplots.resources}
\include{pgfplots.importexport}
\include{pgfplots.basic.reference}

\printindex

\bibliographystyle{abbrv} %gerapali} %gerabbrv} %gerunsrt.bst} %gerabbrv}% gerplain}
\nocite{pgfplotstable}
\nocite{programmingnotes}
\bibliography{pgfplots}
\end{document}

% -----------------------------------------------------------------------------
% For Stefan Pinnow as reminder on what to look for when editing the manual
% -----------------------------------------------------------------------------
% There should be no line breaks in the following environments
% - |...|
% - \declareandlabel{...}
% - \verbpdfref{...}
% If "MakeTikzPictures" isn't running through check one of the externalized
% LOG files what is the cause of that.
% -----------------------------------------------------------------------------


%%%%%%%%%%%%%%%%%%%%%%%%%%%%%%%%%%%%%%%%%%%%%%%%%%%%%%%%%%%%%%%%%%%%%%%%%%%%%
%
% Package pgfplots.sty documentation.
%
% Copyright 2007/2008 by Christian Feuersaenger.
%
% This program is free software: you can redistribute it and/or modify
% it under the terms of the GNU General Public License as published by
% the Free Software Foundation, either version 3 of the License, or
% (at your option) any later version.
%
% This program is distributed in the hope that it will be useful,
% but WITHOUT ANY WARRANTY; without even the implied warranty of
% MERCHANTABILITY or FITNESS FOR A PARTICULAR PURPOSE.  See the
% GNU General Public License for more details.
%
% You should have received a copy of the GNU General Public License
% along with this program.  If not, see <http://www.gnu.org/licenses/>.
%
%
%%%%%%%%%%%%%%%%%%%%%%%%%%%%%%%%%%%%%%%%%%%%%%%%%%%%%%%%%%%%%%%%%%%%%%%%%%%%%
% SEE pgfplots-macros.tex as well!
%\pdfminorversion=5 % to allow compression
%\pdfobjcompresslevel=2
\documentclass[a4paper,openany]{book}

\let\bookmaketitle=\maketitle

% -----------------------------------
% this here is from ltxdoc:
\usepackage{doc}[=v2]
\AtBeginDocument{\MakeShortVerb{\|}}
%\setlength{\textwidth}{355pt}
%\addtolength\marginparwidth{30pt}
%\addtolength\oddsidemargin{20pt}
%\addtolength\evensidemargin{20pt}
\def\cmd#1{\cs{\expandafter\cmd@to@cs\string#1}}
\def\cmd@to@cs#1#2{\char\number`#2\relax}
\DeclareRobustCommand\cs[1]{\texttt{\char`\\#1}}
\providecommand\marg[1]{%
  {\ttfamily\char`\{}\meta{#1}{\ttfamily\char`\}}}
\providecommand\oarg[1]{%
  {\ttfamily[}\meta{#1}{\ttfamily]}}
\providecommand\parg[1]{%
  {\ttfamily(}\meta{#1}{\ttfamily)}}
\raggedbottom

% -----------------------------------
\input{pgfplots.preamble.tex}

\makeatletter
% I want a two-column index just as in pgfmanual styles. This here
% was the best way to get one:
\def\index@prologue{\section*{Index}\addcontentsline{toc}{chapter}{Index}
}
\makeatother

%\RequirePackage[german,english,francais]{babel}

\def\matlabcolormaptext{This colormap is similar to one shipped with Matlab$^\text{\textregistered}$ under a similar name.}

\IfFileExists{tikzlibraryspy.code.tex}{%
\usetikzlibrary{spy}
}{%
    \message{ERROR: tikz SPY library NOT available. The manual will only compile partially.^^J}%
}%

\usepackage{xparse}% for colorbrewer manual

\usetikzlibrary{
    decorations.markings,
    decorations.footprints,
    shapes.arrows,
    matrix,
    positioning,
}

\usepgfplotslibrary{
    fillbetween,
    ternary,
    smithchart,
    patchplots,
    polar,
    colormaps,
    colorbrewer,
    colortol,           % docu not ready yet
}
\pgfqkeys{/codeexample}{%
    every codeexample/.append style={
        /pgfplots/every ternary axis/.append style={
            /pgfplots/legend style={fill=graphicbackground},
        }
    },
    tabsize=4,
}

\pgfplotsmanualenableexternalizationofexpensive

%\usetikzlibrary{external}
%\tikzexternalize[prefix=figures/]{pgfplots}

\title{%
    Manual for Package \PGFPlots{}\\
    {\small 2D/3D Plots in \LaTeX{}, Version \pgfplotsversion}\\
    {\small\url{https://github.com/pgf-tikz/pgfplots}}
    %\\{\small Attention: you are using an unstable development version.}
}

\makeatletter
\long\def\abstractsmuggle{%
    \centering
    \textbf{Abstract}\\[0.5cm]

    \begin{minipage}{12cm}
        \PGFPlots{} draws high-quality function plots in normal or logarithmic
        scaling with a user-friendly interface directly in \TeX{}. The user
        supplies axis labels, legend entries and the plot coordinates for one or
        more plots and \PGFPlots{} applies axis scaling, computes any logarithms
        and axis ticks and draws the plots. It supports line plots, scatter
        plots, piecewise constant plots, bar plots, area plots, mesh and surface
        plots, patch plots, contour plots, quiver plots, histogram plots, box
        plots, polar axes, ternary diagrams, smith charts and some more. It is
        based on Till Tantau's package \PGF{}/\Tikz{}.
    \end{minipage}

}%

\expandafter\date\expandafter{\@date\\[2cm]
    \abstractsmuggle
}%
\makeatother

%\includeonly{pgfplots.reference}


\begin{document}

\def\plotcoords{%
\addplot coordinates {
(5,8.312e-02)    (17,2.547e-02)   (49,7.407e-03)
(129,2.102e-03)  (321,5.874e-04)  (769,1.623e-04)
(1793,4.442e-05) (4097,1.207e-05) (9217,3.261e-06)
};

\addplot coordinates{
(7,8.472e-02)    (31,3.044e-02)    (111,1.022e-02)
(351,3.303e-03)  (1023,1.039e-03)  (2815,3.196e-04)
(7423,9.658e-05) (18943,2.873e-05) (47103,8.437e-06)
};

\addplot coordinates{
(9,7.881e-02)     (49,3.243e-02)    (209,1.232e-02)
(769,4.454e-03)   (2561,1.551e-03)  (7937,5.236e-04)
(23297,1.723e-04) (65537,5.545e-05) (178177,1.751e-05)
};

\addplot coordinates{
(11,6.887e-02)    (71,3.177e-02)     (351,1.341e-02)
(1471,5.334e-03)  (5503,2.027e-03)   (18943,7.415e-04)
(61183,2.628e-04) (187903,9.063e-05) (553983,3.053e-05)
};

\addplot coordinates{
(13,5.755e-02)     (97,2.925e-02)     (545,1.351e-02)
(2561,5.842e-03)   (10625,2.397e-03)  (40193,9.414e-04)
(141569,3.564e-04) (471041,1.308e-04)
(1496065,4.670e-05)
};
}%


\bookmaketitle
\tableofcontents
\include{pgfplots.title_abstract_intro}
\include{pgfplots.preliminaries}
\include{pgfplots.intro}
\include{pgfplots.reference}
\include{pgfplots.libs}
\include{pgfplots.resources}
\include{pgfplots.importexport}
\include{pgfplots.basic.reference}

\printindex

\bibliographystyle{abbrv} %gerapali} %gerabbrv} %gerunsrt.bst} %gerabbrv}% gerplain}
\nocite{pgfplotstable}
\nocite{programmingnotes}
\bibliography{pgfplots}
\end{document}

% -----------------------------------------------------------------------------
% For Stefan Pinnow as reminder on what to look for when editing the manual
% -----------------------------------------------------------------------------
% There should be no line breaks in the following environments
% - |...|
% - \declareandlabel{...}
% - \verbpdfref{...}
% If "MakeTikzPictures" isn't running through check one of the externalized
% LOG files what is the cause of that.
% -----------------------------------------------------------------------------


%%%%%%%%%%%%%%%%%%%%%%%%%%%%%%%%%%%%%%%%%%%%%%%%%%%%%%%%%%%%%%%%%%%%%%%%%%%%%
%
% Package pgfplots.sty documentation.
%
% Copyright 2007/2008 by Christian Feuersaenger.
%
% This program is free software: you can redistribute it and/or modify
% it under the terms of the GNU General Public License as published by
% the Free Software Foundation, either version 3 of the License, or
% (at your option) any later version.
%
% This program is distributed in the hope that it will be useful,
% but WITHOUT ANY WARRANTY; without even the implied warranty of
% MERCHANTABILITY or FITNESS FOR A PARTICULAR PURPOSE.  See the
% GNU General Public License for more details.
%
% You should have received a copy of the GNU General Public License
% along with this program.  If not, see <http://www.gnu.org/licenses/>.
%
%
%%%%%%%%%%%%%%%%%%%%%%%%%%%%%%%%%%%%%%%%%%%%%%%%%%%%%%%%%%%%%%%%%%%%%%%%%%%%%
% SEE pgfplots-macros.tex as well!
%\pdfminorversion=5 % to allow compression
%\pdfobjcompresslevel=2
\documentclass[a4paper,openany]{book}

\let\bookmaketitle=\maketitle

% -----------------------------------
% this here is from ltxdoc:
\usepackage{doc}[=v2]
\AtBeginDocument{\MakeShortVerb{\|}}
%\setlength{\textwidth}{355pt}
%\addtolength\marginparwidth{30pt}
%\addtolength\oddsidemargin{20pt}
%\addtolength\evensidemargin{20pt}
\def\cmd#1{\cs{\expandafter\cmd@to@cs\string#1}}
\def\cmd@to@cs#1#2{\char\number`#2\relax}
\DeclareRobustCommand\cs[1]{\texttt{\char`\\#1}}
\providecommand\marg[1]{%
  {\ttfamily\char`\{}\meta{#1}{\ttfamily\char`\}}}
\providecommand\oarg[1]{%
  {\ttfamily[}\meta{#1}{\ttfamily]}}
\providecommand\parg[1]{%
  {\ttfamily(}\meta{#1}{\ttfamily)}}
\raggedbottom

% -----------------------------------
\input{pgfplots.preamble.tex}

\makeatletter
% I want a two-column index just as in pgfmanual styles. This here
% was the best way to get one:
\def\index@prologue{\section*{Index}\addcontentsline{toc}{chapter}{Index}
}
\makeatother

%\RequirePackage[german,english,francais]{babel}

\def\matlabcolormaptext{This colormap is similar to one shipped with Matlab$^\text{\textregistered}$ under a similar name.}

\IfFileExists{tikzlibraryspy.code.tex}{%
\usetikzlibrary{spy}
}{%
    \message{ERROR: tikz SPY library NOT available. The manual will only compile partially.^^J}%
}%

\usepackage{xparse}% for colorbrewer manual

\usetikzlibrary{
    decorations.markings,
    decorations.footprints,
    shapes.arrows,
    matrix,
    positioning,
}

\usepgfplotslibrary{
    fillbetween,
    ternary,
    smithchart,
    patchplots,
    polar,
    colormaps,
    colorbrewer,
    colortol,           % docu not ready yet
}
\pgfqkeys{/codeexample}{%
    every codeexample/.append style={
        /pgfplots/every ternary axis/.append style={
            /pgfplots/legend style={fill=graphicbackground},
        }
    },
    tabsize=4,
}

\pgfplotsmanualenableexternalizationofexpensive

%\usetikzlibrary{external}
%\tikzexternalize[prefix=figures/]{pgfplots}

\title{%
    Manual for Package \PGFPlots{}\\
    {\small 2D/3D Plots in \LaTeX{}, Version \pgfplotsversion}\\
    {\small\url{https://github.com/pgf-tikz/pgfplots}}
    %\\{\small Attention: you are using an unstable development version.}
}

\makeatletter
\long\def\abstractsmuggle{%
    \centering
    \textbf{Abstract}\\[0.5cm]

    \begin{minipage}{12cm}
        \PGFPlots{} draws high-quality function plots in normal or logarithmic
        scaling with a user-friendly interface directly in \TeX{}. The user
        supplies axis labels, legend entries and the plot coordinates for one or
        more plots and \PGFPlots{} applies axis scaling, computes any logarithms
        and axis ticks and draws the plots. It supports line plots, scatter
        plots, piecewise constant plots, bar plots, area plots, mesh and surface
        plots, patch plots, contour plots, quiver plots, histogram plots, box
        plots, polar axes, ternary diagrams, smith charts and some more. It is
        based on Till Tantau's package \PGF{}/\Tikz{}.
    \end{minipage}

}%

\expandafter\date\expandafter{\@date\\[2cm]
    \abstractsmuggle
}%
\makeatother

%\includeonly{pgfplots.reference}


\begin{document}

\def\plotcoords{%
\addplot coordinates {
(5,8.312e-02)    (17,2.547e-02)   (49,7.407e-03)
(129,2.102e-03)  (321,5.874e-04)  (769,1.623e-04)
(1793,4.442e-05) (4097,1.207e-05) (9217,3.261e-06)
};

\addplot coordinates{
(7,8.472e-02)    (31,3.044e-02)    (111,1.022e-02)
(351,3.303e-03)  (1023,1.039e-03)  (2815,3.196e-04)
(7423,9.658e-05) (18943,2.873e-05) (47103,8.437e-06)
};

\addplot coordinates{
(9,7.881e-02)     (49,3.243e-02)    (209,1.232e-02)
(769,4.454e-03)   (2561,1.551e-03)  (7937,5.236e-04)
(23297,1.723e-04) (65537,5.545e-05) (178177,1.751e-05)
};

\addplot coordinates{
(11,6.887e-02)    (71,3.177e-02)     (351,1.341e-02)
(1471,5.334e-03)  (5503,2.027e-03)   (18943,7.415e-04)
(61183,2.628e-04) (187903,9.063e-05) (553983,3.053e-05)
};

\addplot coordinates{
(13,5.755e-02)     (97,2.925e-02)     (545,1.351e-02)
(2561,5.842e-03)   (10625,2.397e-03)  (40193,9.414e-04)
(141569,3.564e-04) (471041,1.308e-04)
(1496065,4.670e-05)
};
}%


\bookmaketitle
\tableofcontents
\include{pgfplots.title_abstract_intro}
\include{pgfplots.preliminaries}
\include{pgfplots.intro}
\include{pgfplots.reference}
\include{pgfplots.libs}
\include{pgfplots.resources}
\include{pgfplots.importexport}
\include{pgfplots.basic.reference}

\printindex

\bibliographystyle{abbrv} %gerapali} %gerabbrv} %gerunsrt.bst} %gerabbrv}% gerplain}
\nocite{pgfplotstable}
\nocite{programmingnotes}
\bibliography{pgfplots}
\end{document}

% -----------------------------------------------------------------------------
% For Stefan Pinnow as reminder on what to look for when editing the manual
% -----------------------------------------------------------------------------
% There should be no line breaks in the following environments
% - |...|
% - \declareandlabel{...}
% - \verbpdfref{...}
% If "MakeTikzPictures" isn't running through check one of the externalized
% LOG files what is the cause of that.
% -----------------------------------------------------------------------------

\include{pgfplots.basic.reference}

\printindex

\bibliographystyle{abbrv} %gerapali} %gerabbrv} %gerunsrt.bst} %gerabbrv}% gerplain}
\nocite{pgfplotstable}
\nocite{programmingnotes}
\bibliography{pgfplots}
\end{document}

% -----------------------------------------------------------------------------
% For Stefan Pinnow as reminder on what to look for when editing the manual
% -----------------------------------------------------------------------------
% There should be no line breaks in the following environments
% - |...|
% - \declareandlabel{...}
% - \verbpdfref{...}
% If "MakeTikzPictures" isn't running through check one of the externalized
% LOG files what is the cause of that.
% -----------------------------------------------------------------------------

\include{pgfplots.basic.reference}

\printindex

\bibliographystyle{abbrv} %gerapali} %gerabbrv} %gerunsrt.bst} %gerabbrv}% gerplain}
\nocite{pgfplotstable}
\nocite{programmingnotes}
\bibliography{pgfplots}
\end{document}

% -----------------------------------------------------------------------------
% For Stefan Pinnow as reminder on what to look for when editing the manual
% -----------------------------------------------------------------------------
% There should be no line breaks in the following environments
% - |...|
% - \declareandlabel{...}
% - \verbpdfref{...}
% If "MakeTikzPictures" isn't running through check one of the externalized
% LOG files what is the cause of that.
% -----------------------------------------------------------------------------


%%%%%%%%%%%%%%%%%%%%%%%%%%%%%%%%%%%%%%%%%%%%%%%%%%%%%%%%%%%%%%%%%%%%%%%%%%%%%
%
% Package pgfplots.sty documentation.
%
% Copyright 2007/2008 by Christian Feuersaenger.
%
% This program is free software: you can redistribute it and/or modify
% it under the terms of the GNU General Public License as published by
% the Free Software Foundation, either version 3 of the License, or
% (at your option) any later version.
%
% This program is distributed in the hope that it will be useful,
% but WITHOUT ANY WARRANTY; without even the implied warranty of
% MERCHANTABILITY or FITNESS FOR A PARTICULAR PURPOSE.  See the
% GNU General Public License for more details.
%
% You should have received a copy of the GNU General Public License
% along with this program.  If not, see <http://www.gnu.org/licenses/>.
%
%
%%%%%%%%%%%%%%%%%%%%%%%%%%%%%%%%%%%%%%%%%%%%%%%%%%%%%%%%%%%%%%%%%%%%%%%%%%%%%
% SEE pgfplots-macros.tex as well!
%\pdfminorversion=5 % to allow compression
%\pdfobjcompresslevel=2
\documentclass[a4paper,openany]{book}

\let\bookmaketitle=\maketitle

% -----------------------------------
% this here is from ltxdoc:
\usepackage{doc}[=v2]
\AtBeginDocument{\MakeShortVerb{\|}}
%\setlength{\textwidth}{355pt}
%\addtolength\marginparwidth{30pt}
%\addtolength\oddsidemargin{20pt}
%\addtolength\evensidemargin{20pt}
\def\cmd#1{\cs{\expandafter\cmd@to@cs\string#1}}
\def\cmd@to@cs#1#2{\char\number`#2\relax}
\DeclareRobustCommand\cs[1]{\texttt{\char`\\#1}}
\providecommand\marg[1]{%
  {\ttfamily\char`\{}\meta{#1}{\ttfamily\char`\}}}
\providecommand\oarg[1]{%
  {\ttfamily[}\meta{#1}{\ttfamily]}}
\providecommand\parg[1]{%
  {\ttfamily(}\meta{#1}{\ttfamily)}}
\raggedbottom

% -----------------------------------
\input{pgfplots.preamble.tex}

\makeatletter
% I want a two-column index just as in pgfmanual styles. This here
% was the best way to get one:
\def\index@prologue{\section*{Index}\addcontentsline{toc}{chapter}{Index}
}
\makeatother

%\RequirePackage[german,english,francais]{babel}

\def\matlabcolormaptext{This colormap is similar to one shipped with Matlab$^\text{\textregistered}$ under a similar name.}

\IfFileExists{tikzlibraryspy.code.tex}{%
\usetikzlibrary{spy}
}{%
    \message{ERROR: tikz SPY library NOT available. The manual will only compile partially.^^J}%
}%

\usepackage{xparse}% for colorbrewer manual

\usetikzlibrary{
    decorations.markings,
    decorations.footprints,
    shapes.arrows,
    matrix,
    positioning,
}

\usepgfplotslibrary{
    fillbetween,
    ternary,
    smithchart,
    patchplots,
    polar,
    colormaps,
    colorbrewer,
    colortol,           % docu not ready yet
}
\pgfqkeys{/codeexample}{%
    every codeexample/.append style={
        /pgfplots/every ternary axis/.append style={
            /pgfplots/legend style={fill=graphicbackground},
        }
    },
    tabsize=4,
}

\pgfplotsmanualenableexternalizationofexpensive

%\usetikzlibrary{external}
%\tikzexternalize[prefix=figures/]{pgfplots}

\title{%
    Manual for Package \PGFPlots{}\\
    {\small 2D/3D Plots in \LaTeX{}, Version \pgfplotsversion}\\
    {\small\url{https://github.com/pgf-tikz/pgfplots}}
    %\\{\small Attention: you are using an unstable development version.}
}

\makeatletter
\long\def\abstractsmuggle{%
    \centering
    \textbf{Abstract}\\[0.5cm]

    \begin{minipage}{12cm}
        \PGFPlots{} draws high-quality function plots in normal or logarithmic
        scaling with a user-friendly interface directly in \TeX{}. The user
        supplies axis labels, legend entries and the plot coordinates for one or
        more plots and \PGFPlots{} applies axis scaling, computes any logarithms
        and axis ticks and draws the plots. It supports line plots, scatter
        plots, piecewise constant plots, bar plots, area plots, mesh and surface
        plots, patch plots, contour plots, quiver plots, histogram plots, box
        plots, polar axes, ternary diagrams, smith charts and some more. It is
        based on Till Tantau's package \PGF{}/\Tikz{}.
    \end{minipage}

}%

\expandafter\date\expandafter{\@date\\[2cm]
    \abstractsmuggle
}%
\makeatother

%\includeonly{pgfplots.reference}


\begin{document}

\def\plotcoords{%
\addplot coordinates {
(5,8.312e-02)    (17,2.547e-02)   (49,7.407e-03)
(129,2.102e-03)  (321,5.874e-04)  (769,1.623e-04)
(1793,4.442e-05) (4097,1.207e-05) (9217,3.261e-06)
};

\addplot coordinates{
(7,8.472e-02)    (31,3.044e-02)    (111,1.022e-02)
(351,3.303e-03)  (1023,1.039e-03)  (2815,3.196e-04)
(7423,9.658e-05) (18943,2.873e-05) (47103,8.437e-06)
};

\addplot coordinates{
(9,7.881e-02)     (49,3.243e-02)    (209,1.232e-02)
(769,4.454e-03)   (2561,1.551e-03)  (7937,5.236e-04)
(23297,1.723e-04) (65537,5.545e-05) (178177,1.751e-05)
};

\addplot coordinates{
(11,6.887e-02)    (71,3.177e-02)     (351,1.341e-02)
(1471,5.334e-03)  (5503,2.027e-03)   (18943,7.415e-04)
(61183,2.628e-04) (187903,9.063e-05) (553983,3.053e-05)
};

\addplot coordinates{
(13,5.755e-02)     (97,2.925e-02)     (545,1.351e-02)
(2561,5.842e-03)   (10625,2.397e-03)  (40193,9.414e-04)
(141569,3.564e-04) (471041,1.308e-04)
(1496065,4.670e-05)
};
}%


\bookmaketitle
\tableofcontents

%%%%%%%%%%%%%%%%%%%%%%%%%%%%%%%%%%%%%%%%%%%%%%%%%%%%%%%%%%%%%%%%%%%%%%%%%%%%%
%
% Package pgfplots.sty documentation.
%
% Copyright 2007/2008 by Christian Feuersaenger.
%
% This program is free software: you can redistribute it and/or modify
% it under the terms of the GNU General Public License as published by
% the Free Software Foundation, either version 3 of the License, or
% (at your option) any later version.
%
% This program is distributed in the hope that it will be useful,
% but WITHOUT ANY WARRANTY; without even the implied warranty of
% MERCHANTABILITY or FITNESS FOR A PARTICULAR PURPOSE.  See the
% GNU General Public License for more details.
%
% You should have received a copy of the GNU General Public License
% along with this program.  If not, see <http://www.gnu.org/licenses/>.
%
%
%%%%%%%%%%%%%%%%%%%%%%%%%%%%%%%%%%%%%%%%%%%%%%%%%%%%%%%%%%%%%%%%%%%%%%%%%%%%%
% SEE pgfplots-macros.tex as well!
%\pdfminorversion=5 % to allow compression
%\pdfobjcompresslevel=2
\documentclass[a4paper,openany]{book}

\let\bookmaketitle=\maketitle

% -----------------------------------
% this here is from ltxdoc:
\usepackage{doc}[=v2]
\AtBeginDocument{\MakeShortVerb{\|}}
%\setlength{\textwidth}{355pt}
%\addtolength\marginparwidth{30pt}
%\addtolength\oddsidemargin{20pt}
%\addtolength\evensidemargin{20pt}
\def\cmd#1{\cs{\expandafter\cmd@to@cs\string#1}}
\def\cmd@to@cs#1#2{\char\number`#2\relax}
\DeclareRobustCommand\cs[1]{\texttt{\char`\\#1}}
\providecommand\marg[1]{%
  {\ttfamily\char`\{}\meta{#1}{\ttfamily\char`\}}}
\providecommand\oarg[1]{%
  {\ttfamily[}\meta{#1}{\ttfamily]}}
\providecommand\parg[1]{%
  {\ttfamily(}\meta{#1}{\ttfamily)}}
\raggedbottom

% -----------------------------------
\input{pgfplots.preamble.tex}

\makeatletter
% I want a two-column index just as in pgfmanual styles. This here
% was the best way to get one:
\def\index@prologue{\section*{Index}\addcontentsline{toc}{chapter}{Index}
}
\makeatother

%\RequirePackage[german,english,francais]{babel}

\def\matlabcolormaptext{This colormap is similar to one shipped with Matlab$^\text{\textregistered}$ under a similar name.}

\IfFileExists{tikzlibraryspy.code.tex}{%
\usetikzlibrary{spy}
}{%
    \message{ERROR: tikz SPY library NOT available. The manual will only compile partially.^^J}%
}%

\usepackage{xparse}% for colorbrewer manual

\usetikzlibrary{
    decorations.markings,
    decorations.footprints,
    shapes.arrows,
    matrix,
    positioning,
}

\usepgfplotslibrary{
    fillbetween,
    ternary,
    smithchart,
    patchplots,
    polar,
    colormaps,
    colorbrewer,
    colortol,           % docu not ready yet
}
\pgfqkeys{/codeexample}{%
    every codeexample/.append style={
        /pgfplots/every ternary axis/.append style={
            /pgfplots/legend style={fill=graphicbackground},
        }
    },
    tabsize=4,
}

\pgfplotsmanualenableexternalizationofexpensive

%\usetikzlibrary{external}
%\tikzexternalize[prefix=figures/]{pgfplots}

\title{%
    Manual for Package \PGFPlots{}\\
    {\small 2D/3D Plots in \LaTeX{}, Version \pgfplotsversion}\\
    {\small\url{https://github.com/pgf-tikz/pgfplots}}
    %\\{\small Attention: you are using an unstable development version.}
}

\makeatletter
\long\def\abstractsmuggle{%
    \centering
    \textbf{Abstract}\\[0.5cm]

    \begin{minipage}{12cm}
        \PGFPlots{} draws high-quality function plots in normal or logarithmic
        scaling with a user-friendly interface directly in \TeX{}. The user
        supplies axis labels, legend entries and the plot coordinates for one or
        more plots and \PGFPlots{} applies axis scaling, computes any logarithms
        and axis ticks and draws the plots. It supports line plots, scatter
        plots, piecewise constant plots, bar plots, area plots, mesh and surface
        plots, patch plots, contour plots, quiver plots, histogram plots, box
        plots, polar axes, ternary diagrams, smith charts and some more. It is
        based on Till Tantau's package \PGF{}/\Tikz{}.
    \end{minipage}

}%

\expandafter\date\expandafter{\@date\\[2cm]
    \abstractsmuggle
}%
\makeatother

%\includeonly{pgfplots.reference}


\begin{document}

\def\plotcoords{%
\addplot coordinates {
(5,8.312e-02)    (17,2.547e-02)   (49,7.407e-03)
(129,2.102e-03)  (321,5.874e-04)  (769,1.623e-04)
(1793,4.442e-05) (4097,1.207e-05) (9217,3.261e-06)
};

\addplot coordinates{
(7,8.472e-02)    (31,3.044e-02)    (111,1.022e-02)
(351,3.303e-03)  (1023,1.039e-03)  (2815,3.196e-04)
(7423,9.658e-05) (18943,2.873e-05) (47103,8.437e-06)
};

\addplot coordinates{
(9,7.881e-02)     (49,3.243e-02)    (209,1.232e-02)
(769,4.454e-03)   (2561,1.551e-03)  (7937,5.236e-04)
(23297,1.723e-04) (65537,5.545e-05) (178177,1.751e-05)
};

\addplot coordinates{
(11,6.887e-02)    (71,3.177e-02)     (351,1.341e-02)
(1471,5.334e-03)  (5503,2.027e-03)   (18943,7.415e-04)
(61183,2.628e-04) (187903,9.063e-05) (553983,3.053e-05)
};

\addplot coordinates{
(13,5.755e-02)     (97,2.925e-02)     (545,1.351e-02)
(2561,5.842e-03)   (10625,2.397e-03)  (40193,9.414e-04)
(141569,3.564e-04) (471041,1.308e-04)
(1496065,4.670e-05)
};
}%


\bookmaketitle
\tableofcontents

%%%%%%%%%%%%%%%%%%%%%%%%%%%%%%%%%%%%%%%%%%%%%%%%%%%%%%%%%%%%%%%%%%%%%%%%%%%%%
%
% Package pgfplots.sty documentation.
%
% Copyright 2007/2008 by Christian Feuersaenger.
%
% This program is free software: you can redistribute it and/or modify
% it under the terms of the GNU General Public License as published by
% the Free Software Foundation, either version 3 of the License, or
% (at your option) any later version.
%
% This program is distributed in the hope that it will be useful,
% but WITHOUT ANY WARRANTY; without even the implied warranty of
% MERCHANTABILITY or FITNESS FOR A PARTICULAR PURPOSE.  See the
% GNU General Public License for more details.
%
% You should have received a copy of the GNU General Public License
% along with this program.  If not, see <http://www.gnu.org/licenses/>.
%
%
%%%%%%%%%%%%%%%%%%%%%%%%%%%%%%%%%%%%%%%%%%%%%%%%%%%%%%%%%%%%%%%%%%%%%%%%%%%%%
% SEE pgfplots-macros.tex as well!
%\pdfminorversion=5 % to allow compression
%\pdfobjcompresslevel=2
\documentclass[a4paper,openany]{book}

\let\bookmaketitle=\maketitle

% -----------------------------------
% this here is from ltxdoc:
\usepackage{doc}[=v2]
\AtBeginDocument{\MakeShortVerb{\|}}
%\setlength{\textwidth}{355pt}
%\addtolength\marginparwidth{30pt}
%\addtolength\oddsidemargin{20pt}
%\addtolength\evensidemargin{20pt}
\def\cmd#1{\cs{\expandafter\cmd@to@cs\string#1}}
\def\cmd@to@cs#1#2{\char\number`#2\relax}
\DeclareRobustCommand\cs[1]{\texttt{\char`\\#1}}
\providecommand\marg[1]{%
  {\ttfamily\char`\{}\meta{#1}{\ttfamily\char`\}}}
\providecommand\oarg[1]{%
  {\ttfamily[}\meta{#1}{\ttfamily]}}
\providecommand\parg[1]{%
  {\ttfamily(}\meta{#1}{\ttfamily)}}
\raggedbottom

% -----------------------------------
\input{pgfplots.preamble.tex}

\makeatletter
% I want a two-column index just as in pgfmanual styles. This here
% was the best way to get one:
\def\index@prologue{\section*{Index}\addcontentsline{toc}{chapter}{Index}
}
\makeatother

%\RequirePackage[german,english,francais]{babel}

\def\matlabcolormaptext{This colormap is similar to one shipped with Matlab$^\text{\textregistered}$ under a similar name.}

\IfFileExists{tikzlibraryspy.code.tex}{%
\usetikzlibrary{spy}
}{%
    \message{ERROR: tikz SPY library NOT available. The manual will only compile partially.^^J}%
}%

\usepackage{xparse}% for colorbrewer manual

\usetikzlibrary{
    decorations.markings,
    decorations.footprints,
    shapes.arrows,
    matrix,
    positioning,
}

\usepgfplotslibrary{
    fillbetween,
    ternary,
    smithchart,
    patchplots,
    polar,
    colormaps,
    colorbrewer,
    colortol,           % docu not ready yet
}
\pgfqkeys{/codeexample}{%
    every codeexample/.append style={
        /pgfplots/every ternary axis/.append style={
            /pgfplots/legend style={fill=graphicbackground},
        }
    },
    tabsize=4,
}

\pgfplotsmanualenableexternalizationofexpensive

%\usetikzlibrary{external}
%\tikzexternalize[prefix=figures/]{pgfplots}

\title{%
    Manual for Package \PGFPlots{}\\
    {\small 2D/3D Plots in \LaTeX{}, Version \pgfplotsversion}\\
    {\small\url{https://github.com/pgf-tikz/pgfplots}}
    %\\{\small Attention: you are using an unstable development version.}
}

\makeatletter
\long\def\abstractsmuggle{%
    \centering
    \textbf{Abstract}\\[0.5cm]

    \begin{minipage}{12cm}
        \PGFPlots{} draws high-quality function plots in normal or logarithmic
        scaling with a user-friendly interface directly in \TeX{}. The user
        supplies axis labels, legend entries and the plot coordinates for one or
        more plots and \PGFPlots{} applies axis scaling, computes any logarithms
        and axis ticks and draws the plots. It supports line plots, scatter
        plots, piecewise constant plots, bar plots, area plots, mesh and surface
        plots, patch plots, contour plots, quiver plots, histogram plots, box
        plots, polar axes, ternary diagrams, smith charts and some more. It is
        based on Till Tantau's package \PGF{}/\Tikz{}.
    \end{minipage}

}%

\expandafter\date\expandafter{\@date\\[2cm]
    \abstractsmuggle
}%
\makeatother

%\includeonly{pgfplots.reference}


\begin{document}

\def\plotcoords{%
\addplot coordinates {
(5,8.312e-02)    (17,2.547e-02)   (49,7.407e-03)
(129,2.102e-03)  (321,5.874e-04)  (769,1.623e-04)
(1793,4.442e-05) (4097,1.207e-05) (9217,3.261e-06)
};

\addplot coordinates{
(7,8.472e-02)    (31,3.044e-02)    (111,1.022e-02)
(351,3.303e-03)  (1023,1.039e-03)  (2815,3.196e-04)
(7423,9.658e-05) (18943,2.873e-05) (47103,8.437e-06)
};

\addplot coordinates{
(9,7.881e-02)     (49,3.243e-02)    (209,1.232e-02)
(769,4.454e-03)   (2561,1.551e-03)  (7937,5.236e-04)
(23297,1.723e-04) (65537,5.545e-05) (178177,1.751e-05)
};

\addplot coordinates{
(11,6.887e-02)    (71,3.177e-02)     (351,1.341e-02)
(1471,5.334e-03)  (5503,2.027e-03)   (18943,7.415e-04)
(61183,2.628e-04) (187903,9.063e-05) (553983,3.053e-05)
};

\addplot coordinates{
(13,5.755e-02)     (97,2.925e-02)     (545,1.351e-02)
(2561,5.842e-03)   (10625,2.397e-03)  (40193,9.414e-04)
(141569,3.564e-04) (471041,1.308e-04)
(1496065,4.670e-05)
};
}%


\bookmaketitle
\tableofcontents
\include{pgfplots.title_abstract_intro}
\include{pgfplots.preliminaries}
\include{pgfplots.intro}
\include{pgfplots.reference}
\include{pgfplots.libs}
\include{pgfplots.resources}
\include{pgfplots.importexport}
\include{pgfplots.basic.reference}

\printindex

\bibliographystyle{abbrv} %gerapali} %gerabbrv} %gerunsrt.bst} %gerabbrv}% gerplain}
\nocite{pgfplotstable}
\nocite{programmingnotes}
\bibliography{pgfplots}
\end{document}

% -----------------------------------------------------------------------------
% For Stefan Pinnow as reminder on what to look for when editing the manual
% -----------------------------------------------------------------------------
% There should be no line breaks in the following environments
% - |...|
% - \declareandlabel{...}
% - \verbpdfref{...}
% If "MakeTikzPictures" isn't running through check one of the externalized
% LOG files what is the cause of that.
% -----------------------------------------------------------------------------


%%%%%%%%%%%%%%%%%%%%%%%%%%%%%%%%%%%%%%%%%%%%%%%%%%%%%%%%%%%%%%%%%%%%%%%%%%%%%
%
% Package pgfplots.sty documentation.
%
% Copyright 2007/2008 by Christian Feuersaenger.
%
% This program is free software: you can redistribute it and/or modify
% it under the terms of the GNU General Public License as published by
% the Free Software Foundation, either version 3 of the License, or
% (at your option) any later version.
%
% This program is distributed in the hope that it will be useful,
% but WITHOUT ANY WARRANTY; without even the implied warranty of
% MERCHANTABILITY or FITNESS FOR A PARTICULAR PURPOSE.  See the
% GNU General Public License for more details.
%
% You should have received a copy of the GNU General Public License
% along with this program.  If not, see <http://www.gnu.org/licenses/>.
%
%
%%%%%%%%%%%%%%%%%%%%%%%%%%%%%%%%%%%%%%%%%%%%%%%%%%%%%%%%%%%%%%%%%%%%%%%%%%%%%
% SEE pgfplots-macros.tex as well!
%\pdfminorversion=5 % to allow compression
%\pdfobjcompresslevel=2
\documentclass[a4paper,openany]{book}

\let\bookmaketitle=\maketitle

% -----------------------------------
% this here is from ltxdoc:
\usepackage{doc}[=v2]
\AtBeginDocument{\MakeShortVerb{\|}}
%\setlength{\textwidth}{355pt}
%\addtolength\marginparwidth{30pt}
%\addtolength\oddsidemargin{20pt}
%\addtolength\evensidemargin{20pt}
\def\cmd#1{\cs{\expandafter\cmd@to@cs\string#1}}
\def\cmd@to@cs#1#2{\char\number`#2\relax}
\DeclareRobustCommand\cs[1]{\texttt{\char`\\#1}}
\providecommand\marg[1]{%
  {\ttfamily\char`\{}\meta{#1}{\ttfamily\char`\}}}
\providecommand\oarg[1]{%
  {\ttfamily[}\meta{#1}{\ttfamily]}}
\providecommand\parg[1]{%
  {\ttfamily(}\meta{#1}{\ttfamily)}}
\raggedbottom

% -----------------------------------
\input{pgfplots.preamble.tex}

\makeatletter
% I want a two-column index just as in pgfmanual styles. This here
% was the best way to get one:
\def\index@prologue{\section*{Index}\addcontentsline{toc}{chapter}{Index}
}
\makeatother

%\RequirePackage[german,english,francais]{babel}

\def\matlabcolormaptext{This colormap is similar to one shipped with Matlab$^\text{\textregistered}$ under a similar name.}

\IfFileExists{tikzlibraryspy.code.tex}{%
\usetikzlibrary{spy}
}{%
    \message{ERROR: tikz SPY library NOT available. The manual will only compile partially.^^J}%
}%

\usepackage{xparse}% for colorbrewer manual

\usetikzlibrary{
    decorations.markings,
    decorations.footprints,
    shapes.arrows,
    matrix,
    positioning,
}

\usepgfplotslibrary{
    fillbetween,
    ternary,
    smithchart,
    patchplots,
    polar,
    colormaps,
    colorbrewer,
    colortol,           % docu not ready yet
}
\pgfqkeys{/codeexample}{%
    every codeexample/.append style={
        /pgfplots/every ternary axis/.append style={
            /pgfplots/legend style={fill=graphicbackground},
        }
    },
    tabsize=4,
}

\pgfplotsmanualenableexternalizationofexpensive

%\usetikzlibrary{external}
%\tikzexternalize[prefix=figures/]{pgfplots}

\title{%
    Manual for Package \PGFPlots{}\\
    {\small 2D/3D Plots in \LaTeX{}, Version \pgfplotsversion}\\
    {\small\url{https://github.com/pgf-tikz/pgfplots}}
    %\\{\small Attention: you are using an unstable development version.}
}

\makeatletter
\long\def\abstractsmuggle{%
    \centering
    \textbf{Abstract}\\[0.5cm]

    \begin{minipage}{12cm}
        \PGFPlots{} draws high-quality function plots in normal or logarithmic
        scaling with a user-friendly interface directly in \TeX{}. The user
        supplies axis labels, legend entries and the plot coordinates for one or
        more plots and \PGFPlots{} applies axis scaling, computes any logarithms
        and axis ticks and draws the plots. It supports line plots, scatter
        plots, piecewise constant plots, bar plots, area plots, mesh and surface
        plots, patch plots, contour plots, quiver plots, histogram plots, box
        plots, polar axes, ternary diagrams, smith charts and some more. It is
        based on Till Tantau's package \PGF{}/\Tikz{}.
    \end{minipage}

}%

\expandafter\date\expandafter{\@date\\[2cm]
    \abstractsmuggle
}%
\makeatother

%\includeonly{pgfplots.reference}


\begin{document}

\def\plotcoords{%
\addplot coordinates {
(5,8.312e-02)    (17,2.547e-02)   (49,7.407e-03)
(129,2.102e-03)  (321,5.874e-04)  (769,1.623e-04)
(1793,4.442e-05) (4097,1.207e-05) (9217,3.261e-06)
};

\addplot coordinates{
(7,8.472e-02)    (31,3.044e-02)    (111,1.022e-02)
(351,3.303e-03)  (1023,1.039e-03)  (2815,3.196e-04)
(7423,9.658e-05) (18943,2.873e-05) (47103,8.437e-06)
};

\addplot coordinates{
(9,7.881e-02)     (49,3.243e-02)    (209,1.232e-02)
(769,4.454e-03)   (2561,1.551e-03)  (7937,5.236e-04)
(23297,1.723e-04) (65537,5.545e-05) (178177,1.751e-05)
};

\addplot coordinates{
(11,6.887e-02)    (71,3.177e-02)     (351,1.341e-02)
(1471,5.334e-03)  (5503,2.027e-03)   (18943,7.415e-04)
(61183,2.628e-04) (187903,9.063e-05) (553983,3.053e-05)
};

\addplot coordinates{
(13,5.755e-02)     (97,2.925e-02)     (545,1.351e-02)
(2561,5.842e-03)   (10625,2.397e-03)  (40193,9.414e-04)
(141569,3.564e-04) (471041,1.308e-04)
(1496065,4.670e-05)
};
}%


\bookmaketitle
\tableofcontents
\include{pgfplots.title_abstract_intro}
\include{pgfplots.preliminaries}
\include{pgfplots.intro}
\include{pgfplots.reference}
\include{pgfplots.libs}
\include{pgfplots.resources}
\include{pgfplots.importexport}
\include{pgfplots.basic.reference}

\printindex

\bibliographystyle{abbrv} %gerapali} %gerabbrv} %gerunsrt.bst} %gerabbrv}% gerplain}
\nocite{pgfplotstable}
\nocite{programmingnotes}
\bibliography{pgfplots}
\end{document}

% -----------------------------------------------------------------------------
% For Stefan Pinnow as reminder on what to look for when editing the manual
% -----------------------------------------------------------------------------
% There should be no line breaks in the following environments
% - |...|
% - \declareandlabel{...}
% - \verbpdfref{...}
% If "MakeTikzPictures" isn't running through check one of the externalized
% LOG files what is the cause of that.
% -----------------------------------------------------------------------------


%%%%%%%%%%%%%%%%%%%%%%%%%%%%%%%%%%%%%%%%%%%%%%%%%%%%%%%%%%%%%%%%%%%%%%%%%%%%%
%
% Package pgfplots.sty documentation.
%
% Copyright 2007/2008 by Christian Feuersaenger.
%
% This program is free software: you can redistribute it and/or modify
% it under the terms of the GNU General Public License as published by
% the Free Software Foundation, either version 3 of the License, or
% (at your option) any later version.
%
% This program is distributed in the hope that it will be useful,
% but WITHOUT ANY WARRANTY; without even the implied warranty of
% MERCHANTABILITY or FITNESS FOR A PARTICULAR PURPOSE.  See the
% GNU General Public License for more details.
%
% You should have received a copy of the GNU General Public License
% along with this program.  If not, see <http://www.gnu.org/licenses/>.
%
%
%%%%%%%%%%%%%%%%%%%%%%%%%%%%%%%%%%%%%%%%%%%%%%%%%%%%%%%%%%%%%%%%%%%%%%%%%%%%%
% SEE pgfplots-macros.tex as well!
%\pdfminorversion=5 % to allow compression
%\pdfobjcompresslevel=2
\documentclass[a4paper,openany]{book}

\let\bookmaketitle=\maketitle

% -----------------------------------
% this here is from ltxdoc:
\usepackage{doc}[=v2]
\AtBeginDocument{\MakeShortVerb{\|}}
%\setlength{\textwidth}{355pt}
%\addtolength\marginparwidth{30pt}
%\addtolength\oddsidemargin{20pt}
%\addtolength\evensidemargin{20pt}
\def\cmd#1{\cs{\expandafter\cmd@to@cs\string#1}}
\def\cmd@to@cs#1#2{\char\number`#2\relax}
\DeclareRobustCommand\cs[1]{\texttt{\char`\\#1}}
\providecommand\marg[1]{%
  {\ttfamily\char`\{}\meta{#1}{\ttfamily\char`\}}}
\providecommand\oarg[1]{%
  {\ttfamily[}\meta{#1}{\ttfamily]}}
\providecommand\parg[1]{%
  {\ttfamily(}\meta{#1}{\ttfamily)}}
\raggedbottom

% -----------------------------------
\input{pgfplots.preamble.tex}

\makeatletter
% I want a two-column index just as in pgfmanual styles. This here
% was the best way to get one:
\def\index@prologue{\section*{Index}\addcontentsline{toc}{chapter}{Index}
}
\makeatother

%\RequirePackage[german,english,francais]{babel}

\def\matlabcolormaptext{This colormap is similar to one shipped with Matlab$^\text{\textregistered}$ under a similar name.}

\IfFileExists{tikzlibraryspy.code.tex}{%
\usetikzlibrary{spy}
}{%
    \message{ERROR: tikz SPY library NOT available. The manual will only compile partially.^^J}%
}%

\usepackage{xparse}% for colorbrewer manual

\usetikzlibrary{
    decorations.markings,
    decorations.footprints,
    shapes.arrows,
    matrix,
    positioning,
}

\usepgfplotslibrary{
    fillbetween,
    ternary,
    smithchart,
    patchplots,
    polar,
    colormaps,
    colorbrewer,
    colortol,           % docu not ready yet
}
\pgfqkeys{/codeexample}{%
    every codeexample/.append style={
        /pgfplots/every ternary axis/.append style={
            /pgfplots/legend style={fill=graphicbackground},
        }
    },
    tabsize=4,
}

\pgfplotsmanualenableexternalizationofexpensive

%\usetikzlibrary{external}
%\tikzexternalize[prefix=figures/]{pgfplots}

\title{%
    Manual for Package \PGFPlots{}\\
    {\small 2D/3D Plots in \LaTeX{}, Version \pgfplotsversion}\\
    {\small\url{https://github.com/pgf-tikz/pgfplots}}
    %\\{\small Attention: you are using an unstable development version.}
}

\makeatletter
\long\def\abstractsmuggle{%
    \centering
    \textbf{Abstract}\\[0.5cm]

    \begin{minipage}{12cm}
        \PGFPlots{} draws high-quality function plots in normal or logarithmic
        scaling with a user-friendly interface directly in \TeX{}. The user
        supplies axis labels, legend entries and the plot coordinates for one or
        more plots and \PGFPlots{} applies axis scaling, computes any logarithms
        and axis ticks and draws the plots. It supports line plots, scatter
        plots, piecewise constant plots, bar plots, area plots, mesh and surface
        plots, patch plots, contour plots, quiver plots, histogram plots, box
        plots, polar axes, ternary diagrams, smith charts and some more. It is
        based on Till Tantau's package \PGF{}/\Tikz{}.
    \end{minipage}

}%

\expandafter\date\expandafter{\@date\\[2cm]
    \abstractsmuggle
}%
\makeatother

%\includeonly{pgfplots.reference}


\begin{document}

\def\plotcoords{%
\addplot coordinates {
(5,8.312e-02)    (17,2.547e-02)   (49,7.407e-03)
(129,2.102e-03)  (321,5.874e-04)  (769,1.623e-04)
(1793,4.442e-05) (4097,1.207e-05) (9217,3.261e-06)
};

\addplot coordinates{
(7,8.472e-02)    (31,3.044e-02)    (111,1.022e-02)
(351,3.303e-03)  (1023,1.039e-03)  (2815,3.196e-04)
(7423,9.658e-05) (18943,2.873e-05) (47103,8.437e-06)
};

\addplot coordinates{
(9,7.881e-02)     (49,3.243e-02)    (209,1.232e-02)
(769,4.454e-03)   (2561,1.551e-03)  (7937,5.236e-04)
(23297,1.723e-04) (65537,5.545e-05) (178177,1.751e-05)
};

\addplot coordinates{
(11,6.887e-02)    (71,3.177e-02)     (351,1.341e-02)
(1471,5.334e-03)  (5503,2.027e-03)   (18943,7.415e-04)
(61183,2.628e-04) (187903,9.063e-05) (553983,3.053e-05)
};

\addplot coordinates{
(13,5.755e-02)     (97,2.925e-02)     (545,1.351e-02)
(2561,5.842e-03)   (10625,2.397e-03)  (40193,9.414e-04)
(141569,3.564e-04) (471041,1.308e-04)
(1496065,4.670e-05)
};
}%


\bookmaketitle
\tableofcontents
\include{pgfplots.title_abstract_intro}
\include{pgfplots.preliminaries}
\include{pgfplots.intro}
\include{pgfplots.reference}
\include{pgfplots.libs}
\include{pgfplots.resources}
\include{pgfplots.importexport}
\include{pgfplots.basic.reference}

\printindex

\bibliographystyle{abbrv} %gerapali} %gerabbrv} %gerunsrt.bst} %gerabbrv}% gerplain}
\nocite{pgfplotstable}
\nocite{programmingnotes}
\bibliography{pgfplots}
\end{document}

% -----------------------------------------------------------------------------
% For Stefan Pinnow as reminder on what to look for when editing the manual
% -----------------------------------------------------------------------------
% There should be no line breaks in the following environments
% - |...|
% - \declareandlabel{...}
% - \verbpdfref{...}
% If "MakeTikzPictures" isn't running through check one of the externalized
% LOG files what is the cause of that.
% -----------------------------------------------------------------------------


%%%%%%%%%%%%%%%%%%%%%%%%%%%%%%%%%%%%%%%%%%%%%%%%%%%%%%%%%%%%%%%%%%%%%%%%%%%%%
%
% Package pgfplots.sty documentation.
%
% Copyright 2007/2008 by Christian Feuersaenger.
%
% This program is free software: you can redistribute it and/or modify
% it under the terms of the GNU General Public License as published by
% the Free Software Foundation, either version 3 of the License, or
% (at your option) any later version.
%
% This program is distributed in the hope that it will be useful,
% but WITHOUT ANY WARRANTY; without even the implied warranty of
% MERCHANTABILITY or FITNESS FOR A PARTICULAR PURPOSE.  See the
% GNU General Public License for more details.
%
% You should have received a copy of the GNU General Public License
% along with this program.  If not, see <http://www.gnu.org/licenses/>.
%
%
%%%%%%%%%%%%%%%%%%%%%%%%%%%%%%%%%%%%%%%%%%%%%%%%%%%%%%%%%%%%%%%%%%%%%%%%%%%%%
% SEE pgfplots-macros.tex as well!
%\pdfminorversion=5 % to allow compression
%\pdfobjcompresslevel=2
\documentclass[a4paper,openany]{book}

\let\bookmaketitle=\maketitle

% -----------------------------------
% this here is from ltxdoc:
\usepackage{doc}[=v2]
\AtBeginDocument{\MakeShortVerb{\|}}
%\setlength{\textwidth}{355pt}
%\addtolength\marginparwidth{30pt}
%\addtolength\oddsidemargin{20pt}
%\addtolength\evensidemargin{20pt}
\def\cmd#1{\cs{\expandafter\cmd@to@cs\string#1}}
\def\cmd@to@cs#1#2{\char\number`#2\relax}
\DeclareRobustCommand\cs[1]{\texttt{\char`\\#1}}
\providecommand\marg[1]{%
  {\ttfamily\char`\{}\meta{#1}{\ttfamily\char`\}}}
\providecommand\oarg[1]{%
  {\ttfamily[}\meta{#1}{\ttfamily]}}
\providecommand\parg[1]{%
  {\ttfamily(}\meta{#1}{\ttfamily)}}
\raggedbottom

% -----------------------------------
\input{pgfplots.preamble.tex}

\makeatletter
% I want a two-column index just as in pgfmanual styles. This here
% was the best way to get one:
\def\index@prologue{\section*{Index}\addcontentsline{toc}{chapter}{Index}
}
\makeatother

%\RequirePackage[german,english,francais]{babel}

\def\matlabcolormaptext{This colormap is similar to one shipped with Matlab$^\text{\textregistered}$ under a similar name.}

\IfFileExists{tikzlibraryspy.code.tex}{%
\usetikzlibrary{spy}
}{%
    \message{ERROR: tikz SPY library NOT available. The manual will only compile partially.^^J}%
}%

\usepackage{xparse}% for colorbrewer manual

\usetikzlibrary{
    decorations.markings,
    decorations.footprints,
    shapes.arrows,
    matrix,
    positioning,
}

\usepgfplotslibrary{
    fillbetween,
    ternary,
    smithchart,
    patchplots,
    polar,
    colormaps,
    colorbrewer,
    colortol,           % docu not ready yet
}
\pgfqkeys{/codeexample}{%
    every codeexample/.append style={
        /pgfplots/every ternary axis/.append style={
            /pgfplots/legend style={fill=graphicbackground},
        }
    },
    tabsize=4,
}

\pgfplotsmanualenableexternalizationofexpensive

%\usetikzlibrary{external}
%\tikzexternalize[prefix=figures/]{pgfplots}

\title{%
    Manual for Package \PGFPlots{}\\
    {\small 2D/3D Plots in \LaTeX{}, Version \pgfplotsversion}\\
    {\small\url{https://github.com/pgf-tikz/pgfplots}}
    %\\{\small Attention: you are using an unstable development version.}
}

\makeatletter
\long\def\abstractsmuggle{%
    \centering
    \textbf{Abstract}\\[0.5cm]

    \begin{minipage}{12cm}
        \PGFPlots{} draws high-quality function plots in normal or logarithmic
        scaling with a user-friendly interface directly in \TeX{}. The user
        supplies axis labels, legend entries and the plot coordinates for one or
        more plots and \PGFPlots{} applies axis scaling, computes any logarithms
        and axis ticks and draws the plots. It supports line plots, scatter
        plots, piecewise constant plots, bar plots, area plots, mesh and surface
        plots, patch plots, contour plots, quiver plots, histogram plots, box
        plots, polar axes, ternary diagrams, smith charts and some more. It is
        based on Till Tantau's package \PGF{}/\Tikz{}.
    \end{minipage}

}%

\expandafter\date\expandafter{\@date\\[2cm]
    \abstractsmuggle
}%
\makeatother

%\includeonly{pgfplots.reference}


\begin{document}

\def\plotcoords{%
\addplot coordinates {
(5,8.312e-02)    (17,2.547e-02)   (49,7.407e-03)
(129,2.102e-03)  (321,5.874e-04)  (769,1.623e-04)
(1793,4.442e-05) (4097,1.207e-05) (9217,3.261e-06)
};

\addplot coordinates{
(7,8.472e-02)    (31,3.044e-02)    (111,1.022e-02)
(351,3.303e-03)  (1023,1.039e-03)  (2815,3.196e-04)
(7423,9.658e-05) (18943,2.873e-05) (47103,8.437e-06)
};

\addplot coordinates{
(9,7.881e-02)     (49,3.243e-02)    (209,1.232e-02)
(769,4.454e-03)   (2561,1.551e-03)  (7937,5.236e-04)
(23297,1.723e-04) (65537,5.545e-05) (178177,1.751e-05)
};

\addplot coordinates{
(11,6.887e-02)    (71,3.177e-02)     (351,1.341e-02)
(1471,5.334e-03)  (5503,2.027e-03)   (18943,7.415e-04)
(61183,2.628e-04) (187903,9.063e-05) (553983,3.053e-05)
};

\addplot coordinates{
(13,5.755e-02)     (97,2.925e-02)     (545,1.351e-02)
(2561,5.842e-03)   (10625,2.397e-03)  (40193,9.414e-04)
(141569,3.564e-04) (471041,1.308e-04)
(1496065,4.670e-05)
};
}%


\bookmaketitle
\tableofcontents
\include{pgfplots.title_abstract_intro}
\include{pgfplots.preliminaries}
\include{pgfplots.intro}
\include{pgfplots.reference}
\include{pgfplots.libs}
\include{pgfplots.resources}
\include{pgfplots.importexport}
\include{pgfplots.basic.reference}

\printindex

\bibliographystyle{abbrv} %gerapali} %gerabbrv} %gerunsrt.bst} %gerabbrv}% gerplain}
\nocite{pgfplotstable}
\nocite{programmingnotes}
\bibliography{pgfplots}
\end{document}

% -----------------------------------------------------------------------------
% For Stefan Pinnow as reminder on what to look for when editing the manual
% -----------------------------------------------------------------------------
% There should be no line breaks in the following environments
% - |...|
% - \declareandlabel{...}
% - \verbpdfref{...}
% If "MakeTikzPictures" isn't running through check one of the externalized
% LOG files what is the cause of that.
% -----------------------------------------------------------------------------


%%%%%%%%%%%%%%%%%%%%%%%%%%%%%%%%%%%%%%%%%%%%%%%%%%%%%%%%%%%%%%%%%%%%%%%%%%%%%
%
% Package pgfplots.sty documentation.
%
% Copyright 2007/2008 by Christian Feuersaenger.
%
% This program is free software: you can redistribute it and/or modify
% it under the terms of the GNU General Public License as published by
% the Free Software Foundation, either version 3 of the License, or
% (at your option) any later version.
%
% This program is distributed in the hope that it will be useful,
% but WITHOUT ANY WARRANTY; without even the implied warranty of
% MERCHANTABILITY or FITNESS FOR A PARTICULAR PURPOSE.  See the
% GNU General Public License for more details.
%
% You should have received a copy of the GNU General Public License
% along with this program.  If not, see <http://www.gnu.org/licenses/>.
%
%
%%%%%%%%%%%%%%%%%%%%%%%%%%%%%%%%%%%%%%%%%%%%%%%%%%%%%%%%%%%%%%%%%%%%%%%%%%%%%
% SEE pgfplots-macros.tex as well!
%\pdfminorversion=5 % to allow compression
%\pdfobjcompresslevel=2
\documentclass[a4paper,openany]{book}

\let\bookmaketitle=\maketitle

% -----------------------------------
% this here is from ltxdoc:
\usepackage{doc}[=v2]
\AtBeginDocument{\MakeShortVerb{\|}}
%\setlength{\textwidth}{355pt}
%\addtolength\marginparwidth{30pt}
%\addtolength\oddsidemargin{20pt}
%\addtolength\evensidemargin{20pt}
\def\cmd#1{\cs{\expandafter\cmd@to@cs\string#1}}
\def\cmd@to@cs#1#2{\char\number`#2\relax}
\DeclareRobustCommand\cs[1]{\texttt{\char`\\#1}}
\providecommand\marg[1]{%
  {\ttfamily\char`\{}\meta{#1}{\ttfamily\char`\}}}
\providecommand\oarg[1]{%
  {\ttfamily[}\meta{#1}{\ttfamily]}}
\providecommand\parg[1]{%
  {\ttfamily(}\meta{#1}{\ttfamily)}}
\raggedbottom

% -----------------------------------
\input{pgfplots.preamble.tex}

\makeatletter
% I want a two-column index just as in pgfmanual styles. This here
% was the best way to get one:
\def\index@prologue{\section*{Index}\addcontentsline{toc}{chapter}{Index}
}
\makeatother

%\RequirePackage[german,english,francais]{babel}

\def\matlabcolormaptext{This colormap is similar to one shipped with Matlab$^\text{\textregistered}$ under a similar name.}

\IfFileExists{tikzlibraryspy.code.tex}{%
\usetikzlibrary{spy}
}{%
    \message{ERROR: tikz SPY library NOT available. The manual will only compile partially.^^J}%
}%

\usepackage{xparse}% for colorbrewer manual

\usetikzlibrary{
    decorations.markings,
    decorations.footprints,
    shapes.arrows,
    matrix,
    positioning,
}

\usepgfplotslibrary{
    fillbetween,
    ternary,
    smithchart,
    patchplots,
    polar,
    colormaps,
    colorbrewer,
    colortol,           % docu not ready yet
}
\pgfqkeys{/codeexample}{%
    every codeexample/.append style={
        /pgfplots/every ternary axis/.append style={
            /pgfplots/legend style={fill=graphicbackground},
        }
    },
    tabsize=4,
}

\pgfplotsmanualenableexternalizationofexpensive

%\usetikzlibrary{external}
%\tikzexternalize[prefix=figures/]{pgfplots}

\title{%
    Manual for Package \PGFPlots{}\\
    {\small 2D/3D Plots in \LaTeX{}, Version \pgfplotsversion}\\
    {\small\url{https://github.com/pgf-tikz/pgfplots}}
    %\\{\small Attention: you are using an unstable development version.}
}

\makeatletter
\long\def\abstractsmuggle{%
    \centering
    \textbf{Abstract}\\[0.5cm]

    \begin{minipage}{12cm}
        \PGFPlots{} draws high-quality function plots in normal or logarithmic
        scaling with a user-friendly interface directly in \TeX{}. The user
        supplies axis labels, legend entries and the plot coordinates for one or
        more plots and \PGFPlots{} applies axis scaling, computes any logarithms
        and axis ticks and draws the plots. It supports line plots, scatter
        plots, piecewise constant plots, bar plots, area plots, mesh and surface
        plots, patch plots, contour plots, quiver plots, histogram plots, box
        plots, polar axes, ternary diagrams, smith charts and some more. It is
        based on Till Tantau's package \PGF{}/\Tikz{}.
    \end{minipage}

}%

\expandafter\date\expandafter{\@date\\[2cm]
    \abstractsmuggle
}%
\makeatother

%\includeonly{pgfplots.reference}


\begin{document}

\def\plotcoords{%
\addplot coordinates {
(5,8.312e-02)    (17,2.547e-02)   (49,7.407e-03)
(129,2.102e-03)  (321,5.874e-04)  (769,1.623e-04)
(1793,4.442e-05) (4097,1.207e-05) (9217,3.261e-06)
};

\addplot coordinates{
(7,8.472e-02)    (31,3.044e-02)    (111,1.022e-02)
(351,3.303e-03)  (1023,1.039e-03)  (2815,3.196e-04)
(7423,9.658e-05) (18943,2.873e-05) (47103,8.437e-06)
};

\addplot coordinates{
(9,7.881e-02)     (49,3.243e-02)    (209,1.232e-02)
(769,4.454e-03)   (2561,1.551e-03)  (7937,5.236e-04)
(23297,1.723e-04) (65537,5.545e-05) (178177,1.751e-05)
};

\addplot coordinates{
(11,6.887e-02)    (71,3.177e-02)     (351,1.341e-02)
(1471,5.334e-03)  (5503,2.027e-03)   (18943,7.415e-04)
(61183,2.628e-04) (187903,9.063e-05) (553983,3.053e-05)
};

\addplot coordinates{
(13,5.755e-02)     (97,2.925e-02)     (545,1.351e-02)
(2561,5.842e-03)   (10625,2.397e-03)  (40193,9.414e-04)
(141569,3.564e-04) (471041,1.308e-04)
(1496065,4.670e-05)
};
}%


\bookmaketitle
\tableofcontents
\include{pgfplots.title_abstract_intro}
\include{pgfplots.preliminaries}
\include{pgfplots.intro}
\include{pgfplots.reference}
\include{pgfplots.libs}
\include{pgfplots.resources}
\include{pgfplots.importexport}
\include{pgfplots.basic.reference}

\printindex

\bibliographystyle{abbrv} %gerapali} %gerabbrv} %gerunsrt.bst} %gerabbrv}% gerplain}
\nocite{pgfplotstable}
\nocite{programmingnotes}
\bibliography{pgfplots}
\end{document}

% -----------------------------------------------------------------------------
% For Stefan Pinnow as reminder on what to look for when editing the manual
% -----------------------------------------------------------------------------
% There should be no line breaks in the following environments
% - |...|
% - \declareandlabel{...}
% - \verbpdfref{...}
% If "MakeTikzPictures" isn't running through check one of the externalized
% LOG files what is the cause of that.
% -----------------------------------------------------------------------------


%%%%%%%%%%%%%%%%%%%%%%%%%%%%%%%%%%%%%%%%%%%%%%%%%%%%%%%%%%%%%%%%%%%%%%%%%%%%%
%
% Package pgfplots.sty documentation.
%
% Copyright 2007/2008 by Christian Feuersaenger.
%
% This program is free software: you can redistribute it and/or modify
% it under the terms of the GNU General Public License as published by
% the Free Software Foundation, either version 3 of the License, or
% (at your option) any later version.
%
% This program is distributed in the hope that it will be useful,
% but WITHOUT ANY WARRANTY; without even the implied warranty of
% MERCHANTABILITY or FITNESS FOR A PARTICULAR PURPOSE.  See the
% GNU General Public License for more details.
%
% You should have received a copy of the GNU General Public License
% along with this program.  If not, see <http://www.gnu.org/licenses/>.
%
%
%%%%%%%%%%%%%%%%%%%%%%%%%%%%%%%%%%%%%%%%%%%%%%%%%%%%%%%%%%%%%%%%%%%%%%%%%%%%%
% SEE pgfplots-macros.tex as well!
%\pdfminorversion=5 % to allow compression
%\pdfobjcompresslevel=2
\documentclass[a4paper,openany]{book}

\let\bookmaketitle=\maketitle

% -----------------------------------
% this here is from ltxdoc:
\usepackage{doc}[=v2]
\AtBeginDocument{\MakeShortVerb{\|}}
%\setlength{\textwidth}{355pt}
%\addtolength\marginparwidth{30pt}
%\addtolength\oddsidemargin{20pt}
%\addtolength\evensidemargin{20pt}
\def\cmd#1{\cs{\expandafter\cmd@to@cs\string#1}}
\def\cmd@to@cs#1#2{\char\number`#2\relax}
\DeclareRobustCommand\cs[1]{\texttt{\char`\\#1}}
\providecommand\marg[1]{%
  {\ttfamily\char`\{}\meta{#1}{\ttfamily\char`\}}}
\providecommand\oarg[1]{%
  {\ttfamily[}\meta{#1}{\ttfamily]}}
\providecommand\parg[1]{%
  {\ttfamily(}\meta{#1}{\ttfamily)}}
\raggedbottom

% -----------------------------------
\input{pgfplots.preamble.tex}

\makeatletter
% I want a two-column index just as in pgfmanual styles. This here
% was the best way to get one:
\def\index@prologue{\section*{Index}\addcontentsline{toc}{chapter}{Index}
}
\makeatother

%\RequirePackage[german,english,francais]{babel}

\def\matlabcolormaptext{This colormap is similar to one shipped with Matlab$^\text{\textregistered}$ under a similar name.}

\IfFileExists{tikzlibraryspy.code.tex}{%
\usetikzlibrary{spy}
}{%
    \message{ERROR: tikz SPY library NOT available. The manual will only compile partially.^^J}%
}%

\usepackage{xparse}% for colorbrewer manual

\usetikzlibrary{
    decorations.markings,
    decorations.footprints,
    shapes.arrows,
    matrix,
    positioning,
}

\usepgfplotslibrary{
    fillbetween,
    ternary,
    smithchart,
    patchplots,
    polar,
    colormaps,
    colorbrewer,
    colortol,           % docu not ready yet
}
\pgfqkeys{/codeexample}{%
    every codeexample/.append style={
        /pgfplots/every ternary axis/.append style={
            /pgfplots/legend style={fill=graphicbackground},
        }
    },
    tabsize=4,
}

\pgfplotsmanualenableexternalizationofexpensive

%\usetikzlibrary{external}
%\tikzexternalize[prefix=figures/]{pgfplots}

\title{%
    Manual for Package \PGFPlots{}\\
    {\small 2D/3D Plots in \LaTeX{}, Version \pgfplotsversion}\\
    {\small\url{https://github.com/pgf-tikz/pgfplots}}
    %\\{\small Attention: you are using an unstable development version.}
}

\makeatletter
\long\def\abstractsmuggle{%
    \centering
    \textbf{Abstract}\\[0.5cm]

    \begin{minipage}{12cm}
        \PGFPlots{} draws high-quality function plots in normal or logarithmic
        scaling with a user-friendly interface directly in \TeX{}. The user
        supplies axis labels, legend entries and the plot coordinates for one or
        more plots and \PGFPlots{} applies axis scaling, computes any logarithms
        and axis ticks and draws the plots. It supports line plots, scatter
        plots, piecewise constant plots, bar plots, area plots, mesh and surface
        plots, patch plots, contour plots, quiver plots, histogram plots, box
        plots, polar axes, ternary diagrams, smith charts and some more. It is
        based on Till Tantau's package \PGF{}/\Tikz{}.
    \end{minipage}

}%

\expandafter\date\expandafter{\@date\\[2cm]
    \abstractsmuggle
}%
\makeatother

%\includeonly{pgfplots.reference}


\begin{document}

\def\plotcoords{%
\addplot coordinates {
(5,8.312e-02)    (17,2.547e-02)   (49,7.407e-03)
(129,2.102e-03)  (321,5.874e-04)  (769,1.623e-04)
(1793,4.442e-05) (4097,1.207e-05) (9217,3.261e-06)
};

\addplot coordinates{
(7,8.472e-02)    (31,3.044e-02)    (111,1.022e-02)
(351,3.303e-03)  (1023,1.039e-03)  (2815,3.196e-04)
(7423,9.658e-05) (18943,2.873e-05) (47103,8.437e-06)
};

\addplot coordinates{
(9,7.881e-02)     (49,3.243e-02)    (209,1.232e-02)
(769,4.454e-03)   (2561,1.551e-03)  (7937,5.236e-04)
(23297,1.723e-04) (65537,5.545e-05) (178177,1.751e-05)
};

\addplot coordinates{
(11,6.887e-02)    (71,3.177e-02)     (351,1.341e-02)
(1471,5.334e-03)  (5503,2.027e-03)   (18943,7.415e-04)
(61183,2.628e-04) (187903,9.063e-05) (553983,3.053e-05)
};

\addplot coordinates{
(13,5.755e-02)     (97,2.925e-02)     (545,1.351e-02)
(2561,5.842e-03)   (10625,2.397e-03)  (40193,9.414e-04)
(141569,3.564e-04) (471041,1.308e-04)
(1496065,4.670e-05)
};
}%


\bookmaketitle
\tableofcontents
\include{pgfplots.title_abstract_intro}
\include{pgfplots.preliminaries}
\include{pgfplots.intro}
\include{pgfplots.reference}
\include{pgfplots.libs}
\include{pgfplots.resources}
\include{pgfplots.importexport}
\include{pgfplots.basic.reference}

\printindex

\bibliographystyle{abbrv} %gerapali} %gerabbrv} %gerunsrt.bst} %gerabbrv}% gerplain}
\nocite{pgfplotstable}
\nocite{programmingnotes}
\bibliography{pgfplots}
\end{document}

% -----------------------------------------------------------------------------
% For Stefan Pinnow as reminder on what to look for when editing the manual
% -----------------------------------------------------------------------------
% There should be no line breaks in the following environments
% - |...|
% - \declareandlabel{...}
% - \verbpdfref{...}
% If "MakeTikzPictures" isn't running through check one of the externalized
% LOG files what is the cause of that.
% -----------------------------------------------------------------------------


%%%%%%%%%%%%%%%%%%%%%%%%%%%%%%%%%%%%%%%%%%%%%%%%%%%%%%%%%%%%%%%%%%%%%%%%%%%%%
%
% Package pgfplots.sty documentation.
%
% Copyright 2007/2008 by Christian Feuersaenger.
%
% This program is free software: you can redistribute it and/or modify
% it under the terms of the GNU General Public License as published by
% the Free Software Foundation, either version 3 of the License, or
% (at your option) any later version.
%
% This program is distributed in the hope that it will be useful,
% but WITHOUT ANY WARRANTY; without even the implied warranty of
% MERCHANTABILITY or FITNESS FOR A PARTICULAR PURPOSE.  See the
% GNU General Public License for more details.
%
% You should have received a copy of the GNU General Public License
% along with this program.  If not, see <http://www.gnu.org/licenses/>.
%
%
%%%%%%%%%%%%%%%%%%%%%%%%%%%%%%%%%%%%%%%%%%%%%%%%%%%%%%%%%%%%%%%%%%%%%%%%%%%%%
% SEE pgfplots-macros.tex as well!
%\pdfminorversion=5 % to allow compression
%\pdfobjcompresslevel=2
\documentclass[a4paper,openany]{book}

\let\bookmaketitle=\maketitle

% -----------------------------------
% this here is from ltxdoc:
\usepackage{doc}[=v2]
\AtBeginDocument{\MakeShortVerb{\|}}
%\setlength{\textwidth}{355pt}
%\addtolength\marginparwidth{30pt}
%\addtolength\oddsidemargin{20pt}
%\addtolength\evensidemargin{20pt}
\def\cmd#1{\cs{\expandafter\cmd@to@cs\string#1}}
\def\cmd@to@cs#1#2{\char\number`#2\relax}
\DeclareRobustCommand\cs[1]{\texttt{\char`\\#1}}
\providecommand\marg[1]{%
  {\ttfamily\char`\{}\meta{#1}{\ttfamily\char`\}}}
\providecommand\oarg[1]{%
  {\ttfamily[}\meta{#1}{\ttfamily]}}
\providecommand\parg[1]{%
  {\ttfamily(}\meta{#1}{\ttfamily)}}
\raggedbottom

% -----------------------------------
\input{pgfplots.preamble.tex}

\makeatletter
% I want a two-column index just as in pgfmanual styles. This here
% was the best way to get one:
\def\index@prologue{\section*{Index}\addcontentsline{toc}{chapter}{Index}
}
\makeatother

%\RequirePackage[german,english,francais]{babel}

\def\matlabcolormaptext{This colormap is similar to one shipped with Matlab$^\text{\textregistered}$ under a similar name.}

\IfFileExists{tikzlibraryspy.code.tex}{%
\usetikzlibrary{spy}
}{%
    \message{ERROR: tikz SPY library NOT available. The manual will only compile partially.^^J}%
}%

\usepackage{xparse}% for colorbrewer manual

\usetikzlibrary{
    decorations.markings,
    decorations.footprints,
    shapes.arrows,
    matrix,
    positioning,
}

\usepgfplotslibrary{
    fillbetween,
    ternary,
    smithchart,
    patchplots,
    polar,
    colormaps,
    colorbrewer,
    colortol,           % docu not ready yet
}
\pgfqkeys{/codeexample}{%
    every codeexample/.append style={
        /pgfplots/every ternary axis/.append style={
            /pgfplots/legend style={fill=graphicbackground},
        }
    },
    tabsize=4,
}

\pgfplotsmanualenableexternalizationofexpensive

%\usetikzlibrary{external}
%\tikzexternalize[prefix=figures/]{pgfplots}

\title{%
    Manual for Package \PGFPlots{}\\
    {\small 2D/3D Plots in \LaTeX{}, Version \pgfplotsversion}\\
    {\small\url{https://github.com/pgf-tikz/pgfplots}}
    %\\{\small Attention: you are using an unstable development version.}
}

\makeatletter
\long\def\abstractsmuggle{%
    \centering
    \textbf{Abstract}\\[0.5cm]

    \begin{minipage}{12cm}
        \PGFPlots{} draws high-quality function plots in normal or logarithmic
        scaling with a user-friendly interface directly in \TeX{}. The user
        supplies axis labels, legend entries and the plot coordinates for one or
        more plots and \PGFPlots{} applies axis scaling, computes any logarithms
        and axis ticks and draws the plots. It supports line plots, scatter
        plots, piecewise constant plots, bar plots, area plots, mesh and surface
        plots, patch plots, contour plots, quiver plots, histogram plots, box
        plots, polar axes, ternary diagrams, smith charts and some more. It is
        based on Till Tantau's package \PGF{}/\Tikz{}.
    \end{minipage}

}%

\expandafter\date\expandafter{\@date\\[2cm]
    \abstractsmuggle
}%
\makeatother

%\includeonly{pgfplots.reference}


\begin{document}

\def\plotcoords{%
\addplot coordinates {
(5,8.312e-02)    (17,2.547e-02)   (49,7.407e-03)
(129,2.102e-03)  (321,5.874e-04)  (769,1.623e-04)
(1793,4.442e-05) (4097,1.207e-05) (9217,3.261e-06)
};

\addplot coordinates{
(7,8.472e-02)    (31,3.044e-02)    (111,1.022e-02)
(351,3.303e-03)  (1023,1.039e-03)  (2815,3.196e-04)
(7423,9.658e-05) (18943,2.873e-05) (47103,8.437e-06)
};

\addplot coordinates{
(9,7.881e-02)     (49,3.243e-02)    (209,1.232e-02)
(769,4.454e-03)   (2561,1.551e-03)  (7937,5.236e-04)
(23297,1.723e-04) (65537,5.545e-05) (178177,1.751e-05)
};

\addplot coordinates{
(11,6.887e-02)    (71,3.177e-02)     (351,1.341e-02)
(1471,5.334e-03)  (5503,2.027e-03)   (18943,7.415e-04)
(61183,2.628e-04) (187903,9.063e-05) (553983,3.053e-05)
};

\addplot coordinates{
(13,5.755e-02)     (97,2.925e-02)     (545,1.351e-02)
(2561,5.842e-03)   (10625,2.397e-03)  (40193,9.414e-04)
(141569,3.564e-04) (471041,1.308e-04)
(1496065,4.670e-05)
};
}%


\bookmaketitle
\tableofcontents
\include{pgfplots.title_abstract_intro}
\include{pgfplots.preliminaries}
\include{pgfplots.intro}
\include{pgfplots.reference}
\include{pgfplots.libs}
\include{pgfplots.resources}
\include{pgfplots.importexport}
\include{pgfplots.basic.reference}

\printindex

\bibliographystyle{abbrv} %gerapali} %gerabbrv} %gerunsrt.bst} %gerabbrv}% gerplain}
\nocite{pgfplotstable}
\nocite{programmingnotes}
\bibliography{pgfplots}
\end{document}

% -----------------------------------------------------------------------------
% For Stefan Pinnow as reminder on what to look for when editing the manual
% -----------------------------------------------------------------------------
% There should be no line breaks in the following environments
% - |...|
% - \declareandlabel{...}
% - \verbpdfref{...}
% If "MakeTikzPictures" isn't running through check one of the externalized
% LOG files what is the cause of that.
% -----------------------------------------------------------------------------

\include{pgfplots.basic.reference}

\printindex

\bibliographystyle{abbrv} %gerapali} %gerabbrv} %gerunsrt.bst} %gerabbrv}% gerplain}
\nocite{pgfplotstable}
\nocite{programmingnotes}
\bibliography{pgfplots}
\end{document}

% -----------------------------------------------------------------------------
% For Stefan Pinnow as reminder on what to look for when editing the manual
% -----------------------------------------------------------------------------
% There should be no line breaks in the following environments
% - |...|
% - \declareandlabel{...}
% - \verbpdfref{...}
% If "MakeTikzPictures" isn't running through check one of the externalized
% LOG files what is the cause of that.
% -----------------------------------------------------------------------------


%%%%%%%%%%%%%%%%%%%%%%%%%%%%%%%%%%%%%%%%%%%%%%%%%%%%%%%%%%%%%%%%%%%%%%%%%%%%%
%
% Package pgfplots.sty documentation.
%
% Copyright 2007/2008 by Christian Feuersaenger.
%
% This program is free software: you can redistribute it and/or modify
% it under the terms of the GNU General Public License as published by
% the Free Software Foundation, either version 3 of the License, or
% (at your option) any later version.
%
% This program is distributed in the hope that it will be useful,
% but WITHOUT ANY WARRANTY; without even the implied warranty of
% MERCHANTABILITY or FITNESS FOR A PARTICULAR PURPOSE.  See the
% GNU General Public License for more details.
%
% You should have received a copy of the GNU General Public License
% along with this program.  If not, see <http://www.gnu.org/licenses/>.
%
%
%%%%%%%%%%%%%%%%%%%%%%%%%%%%%%%%%%%%%%%%%%%%%%%%%%%%%%%%%%%%%%%%%%%%%%%%%%%%%
% SEE pgfplots-macros.tex as well!
%\pdfminorversion=5 % to allow compression
%\pdfobjcompresslevel=2
\documentclass[a4paper,openany]{book}

\let\bookmaketitle=\maketitle

% -----------------------------------
% this here is from ltxdoc:
\usepackage{doc}[=v2]
\AtBeginDocument{\MakeShortVerb{\|}}
%\setlength{\textwidth}{355pt}
%\addtolength\marginparwidth{30pt}
%\addtolength\oddsidemargin{20pt}
%\addtolength\evensidemargin{20pt}
\def\cmd#1{\cs{\expandafter\cmd@to@cs\string#1}}
\def\cmd@to@cs#1#2{\char\number`#2\relax}
\DeclareRobustCommand\cs[1]{\texttt{\char`\\#1}}
\providecommand\marg[1]{%
  {\ttfamily\char`\{}\meta{#1}{\ttfamily\char`\}}}
\providecommand\oarg[1]{%
  {\ttfamily[}\meta{#1}{\ttfamily]}}
\providecommand\parg[1]{%
  {\ttfamily(}\meta{#1}{\ttfamily)}}
\raggedbottom

% -----------------------------------
\input{pgfplots.preamble.tex}

\makeatletter
% I want a two-column index just as in pgfmanual styles. This here
% was the best way to get one:
\def\index@prologue{\section*{Index}\addcontentsline{toc}{chapter}{Index}
}
\makeatother

%\RequirePackage[german,english,francais]{babel}

\def\matlabcolormaptext{This colormap is similar to one shipped with Matlab$^\text{\textregistered}$ under a similar name.}

\IfFileExists{tikzlibraryspy.code.tex}{%
\usetikzlibrary{spy}
}{%
    \message{ERROR: tikz SPY library NOT available. The manual will only compile partially.^^J}%
}%

\usepackage{xparse}% for colorbrewer manual

\usetikzlibrary{
    decorations.markings,
    decorations.footprints,
    shapes.arrows,
    matrix,
    positioning,
}

\usepgfplotslibrary{
    fillbetween,
    ternary,
    smithchart,
    patchplots,
    polar,
    colormaps,
    colorbrewer,
    colortol,           % docu not ready yet
}
\pgfqkeys{/codeexample}{%
    every codeexample/.append style={
        /pgfplots/every ternary axis/.append style={
            /pgfplots/legend style={fill=graphicbackground},
        }
    },
    tabsize=4,
}

\pgfplotsmanualenableexternalizationofexpensive

%\usetikzlibrary{external}
%\tikzexternalize[prefix=figures/]{pgfplots}

\title{%
    Manual for Package \PGFPlots{}\\
    {\small 2D/3D Plots in \LaTeX{}, Version \pgfplotsversion}\\
    {\small\url{https://github.com/pgf-tikz/pgfplots}}
    %\\{\small Attention: you are using an unstable development version.}
}

\makeatletter
\long\def\abstractsmuggle{%
    \centering
    \textbf{Abstract}\\[0.5cm]

    \begin{minipage}{12cm}
        \PGFPlots{} draws high-quality function plots in normal or logarithmic
        scaling with a user-friendly interface directly in \TeX{}. The user
        supplies axis labels, legend entries and the plot coordinates for one or
        more plots and \PGFPlots{} applies axis scaling, computes any logarithms
        and axis ticks and draws the plots. It supports line plots, scatter
        plots, piecewise constant plots, bar plots, area plots, mesh and surface
        plots, patch plots, contour plots, quiver plots, histogram plots, box
        plots, polar axes, ternary diagrams, smith charts and some more. It is
        based on Till Tantau's package \PGF{}/\Tikz{}.
    \end{minipage}

}%

\expandafter\date\expandafter{\@date\\[2cm]
    \abstractsmuggle
}%
\makeatother

%\includeonly{pgfplots.reference}


\begin{document}

\def\plotcoords{%
\addplot coordinates {
(5,8.312e-02)    (17,2.547e-02)   (49,7.407e-03)
(129,2.102e-03)  (321,5.874e-04)  (769,1.623e-04)
(1793,4.442e-05) (4097,1.207e-05) (9217,3.261e-06)
};

\addplot coordinates{
(7,8.472e-02)    (31,3.044e-02)    (111,1.022e-02)
(351,3.303e-03)  (1023,1.039e-03)  (2815,3.196e-04)
(7423,9.658e-05) (18943,2.873e-05) (47103,8.437e-06)
};

\addplot coordinates{
(9,7.881e-02)     (49,3.243e-02)    (209,1.232e-02)
(769,4.454e-03)   (2561,1.551e-03)  (7937,5.236e-04)
(23297,1.723e-04) (65537,5.545e-05) (178177,1.751e-05)
};

\addplot coordinates{
(11,6.887e-02)    (71,3.177e-02)     (351,1.341e-02)
(1471,5.334e-03)  (5503,2.027e-03)   (18943,7.415e-04)
(61183,2.628e-04) (187903,9.063e-05) (553983,3.053e-05)
};

\addplot coordinates{
(13,5.755e-02)     (97,2.925e-02)     (545,1.351e-02)
(2561,5.842e-03)   (10625,2.397e-03)  (40193,9.414e-04)
(141569,3.564e-04) (471041,1.308e-04)
(1496065,4.670e-05)
};
}%


\bookmaketitle
\tableofcontents

%%%%%%%%%%%%%%%%%%%%%%%%%%%%%%%%%%%%%%%%%%%%%%%%%%%%%%%%%%%%%%%%%%%%%%%%%%%%%
%
% Package pgfplots.sty documentation.
%
% Copyright 2007/2008 by Christian Feuersaenger.
%
% This program is free software: you can redistribute it and/or modify
% it under the terms of the GNU General Public License as published by
% the Free Software Foundation, either version 3 of the License, or
% (at your option) any later version.
%
% This program is distributed in the hope that it will be useful,
% but WITHOUT ANY WARRANTY; without even the implied warranty of
% MERCHANTABILITY or FITNESS FOR A PARTICULAR PURPOSE.  See the
% GNU General Public License for more details.
%
% You should have received a copy of the GNU General Public License
% along with this program.  If not, see <http://www.gnu.org/licenses/>.
%
%
%%%%%%%%%%%%%%%%%%%%%%%%%%%%%%%%%%%%%%%%%%%%%%%%%%%%%%%%%%%%%%%%%%%%%%%%%%%%%
% SEE pgfplots-macros.tex as well!
%\pdfminorversion=5 % to allow compression
%\pdfobjcompresslevel=2
\documentclass[a4paper,openany]{book}

\let\bookmaketitle=\maketitle

% -----------------------------------
% this here is from ltxdoc:
\usepackage{doc}[=v2]
\AtBeginDocument{\MakeShortVerb{\|}}
%\setlength{\textwidth}{355pt}
%\addtolength\marginparwidth{30pt}
%\addtolength\oddsidemargin{20pt}
%\addtolength\evensidemargin{20pt}
\def\cmd#1{\cs{\expandafter\cmd@to@cs\string#1}}
\def\cmd@to@cs#1#2{\char\number`#2\relax}
\DeclareRobustCommand\cs[1]{\texttt{\char`\\#1}}
\providecommand\marg[1]{%
  {\ttfamily\char`\{}\meta{#1}{\ttfamily\char`\}}}
\providecommand\oarg[1]{%
  {\ttfamily[}\meta{#1}{\ttfamily]}}
\providecommand\parg[1]{%
  {\ttfamily(}\meta{#1}{\ttfamily)}}
\raggedbottom

% -----------------------------------
\input{pgfplots.preamble.tex}

\makeatletter
% I want a two-column index just as in pgfmanual styles. This here
% was the best way to get one:
\def\index@prologue{\section*{Index}\addcontentsline{toc}{chapter}{Index}
}
\makeatother

%\RequirePackage[german,english,francais]{babel}

\def\matlabcolormaptext{This colormap is similar to one shipped with Matlab$^\text{\textregistered}$ under a similar name.}

\IfFileExists{tikzlibraryspy.code.tex}{%
\usetikzlibrary{spy}
}{%
    \message{ERROR: tikz SPY library NOT available. The manual will only compile partially.^^J}%
}%

\usepackage{xparse}% for colorbrewer manual

\usetikzlibrary{
    decorations.markings,
    decorations.footprints,
    shapes.arrows,
    matrix,
    positioning,
}

\usepgfplotslibrary{
    fillbetween,
    ternary,
    smithchart,
    patchplots,
    polar,
    colormaps,
    colorbrewer,
    colortol,           % docu not ready yet
}
\pgfqkeys{/codeexample}{%
    every codeexample/.append style={
        /pgfplots/every ternary axis/.append style={
            /pgfplots/legend style={fill=graphicbackground},
        }
    },
    tabsize=4,
}

\pgfplotsmanualenableexternalizationofexpensive

%\usetikzlibrary{external}
%\tikzexternalize[prefix=figures/]{pgfplots}

\title{%
    Manual for Package \PGFPlots{}\\
    {\small 2D/3D Plots in \LaTeX{}, Version \pgfplotsversion}\\
    {\small\url{https://github.com/pgf-tikz/pgfplots}}
    %\\{\small Attention: you are using an unstable development version.}
}

\makeatletter
\long\def\abstractsmuggle{%
    \centering
    \textbf{Abstract}\\[0.5cm]

    \begin{minipage}{12cm}
        \PGFPlots{} draws high-quality function plots in normal or logarithmic
        scaling with a user-friendly interface directly in \TeX{}. The user
        supplies axis labels, legend entries and the plot coordinates for one or
        more plots and \PGFPlots{} applies axis scaling, computes any logarithms
        and axis ticks and draws the plots. It supports line plots, scatter
        plots, piecewise constant plots, bar plots, area plots, mesh and surface
        plots, patch plots, contour plots, quiver plots, histogram plots, box
        plots, polar axes, ternary diagrams, smith charts and some more. It is
        based on Till Tantau's package \PGF{}/\Tikz{}.
    \end{minipage}

}%

\expandafter\date\expandafter{\@date\\[2cm]
    \abstractsmuggle
}%
\makeatother

%\includeonly{pgfplots.reference}


\begin{document}

\def\plotcoords{%
\addplot coordinates {
(5,8.312e-02)    (17,2.547e-02)   (49,7.407e-03)
(129,2.102e-03)  (321,5.874e-04)  (769,1.623e-04)
(1793,4.442e-05) (4097,1.207e-05) (9217,3.261e-06)
};

\addplot coordinates{
(7,8.472e-02)    (31,3.044e-02)    (111,1.022e-02)
(351,3.303e-03)  (1023,1.039e-03)  (2815,3.196e-04)
(7423,9.658e-05) (18943,2.873e-05) (47103,8.437e-06)
};

\addplot coordinates{
(9,7.881e-02)     (49,3.243e-02)    (209,1.232e-02)
(769,4.454e-03)   (2561,1.551e-03)  (7937,5.236e-04)
(23297,1.723e-04) (65537,5.545e-05) (178177,1.751e-05)
};

\addplot coordinates{
(11,6.887e-02)    (71,3.177e-02)     (351,1.341e-02)
(1471,5.334e-03)  (5503,2.027e-03)   (18943,7.415e-04)
(61183,2.628e-04) (187903,9.063e-05) (553983,3.053e-05)
};

\addplot coordinates{
(13,5.755e-02)     (97,2.925e-02)     (545,1.351e-02)
(2561,5.842e-03)   (10625,2.397e-03)  (40193,9.414e-04)
(141569,3.564e-04) (471041,1.308e-04)
(1496065,4.670e-05)
};
}%


\bookmaketitle
\tableofcontents
\include{pgfplots.title_abstract_intro}
\include{pgfplots.preliminaries}
\include{pgfplots.intro}
\include{pgfplots.reference}
\include{pgfplots.libs}
\include{pgfplots.resources}
\include{pgfplots.importexport}
\include{pgfplots.basic.reference}

\printindex

\bibliographystyle{abbrv} %gerapali} %gerabbrv} %gerunsrt.bst} %gerabbrv}% gerplain}
\nocite{pgfplotstable}
\nocite{programmingnotes}
\bibliography{pgfplots}
\end{document}

% -----------------------------------------------------------------------------
% For Stefan Pinnow as reminder on what to look for when editing the manual
% -----------------------------------------------------------------------------
% There should be no line breaks in the following environments
% - |...|
% - \declareandlabel{...}
% - \verbpdfref{...}
% If "MakeTikzPictures" isn't running through check one of the externalized
% LOG files what is the cause of that.
% -----------------------------------------------------------------------------


%%%%%%%%%%%%%%%%%%%%%%%%%%%%%%%%%%%%%%%%%%%%%%%%%%%%%%%%%%%%%%%%%%%%%%%%%%%%%
%
% Package pgfplots.sty documentation.
%
% Copyright 2007/2008 by Christian Feuersaenger.
%
% This program is free software: you can redistribute it and/or modify
% it under the terms of the GNU General Public License as published by
% the Free Software Foundation, either version 3 of the License, or
% (at your option) any later version.
%
% This program is distributed in the hope that it will be useful,
% but WITHOUT ANY WARRANTY; without even the implied warranty of
% MERCHANTABILITY or FITNESS FOR A PARTICULAR PURPOSE.  See the
% GNU General Public License for more details.
%
% You should have received a copy of the GNU General Public License
% along with this program.  If not, see <http://www.gnu.org/licenses/>.
%
%
%%%%%%%%%%%%%%%%%%%%%%%%%%%%%%%%%%%%%%%%%%%%%%%%%%%%%%%%%%%%%%%%%%%%%%%%%%%%%
% SEE pgfplots-macros.tex as well!
%\pdfminorversion=5 % to allow compression
%\pdfobjcompresslevel=2
\documentclass[a4paper,openany]{book}

\let\bookmaketitle=\maketitle

% -----------------------------------
% this here is from ltxdoc:
\usepackage{doc}[=v2]
\AtBeginDocument{\MakeShortVerb{\|}}
%\setlength{\textwidth}{355pt}
%\addtolength\marginparwidth{30pt}
%\addtolength\oddsidemargin{20pt}
%\addtolength\evensidemargin{20pt}
\def\cmd#1{\cs{\expandafter\cmd@to@cs\string#1}}
\def\cmd@to@cs#1#2{\char\number`#2\relax}
\DeclareRobustCommand\cs[1]{\texttt{\char`\\#1}}
\providecommand\marg[1]{%
  {\ttfamily\char`\{}\meta{#1}{\ttfamily\char`\}}}
\providecommand\oarg[1]{%
  {\ttfamily[}\meta{#1}{\ttfamily]}}
\providecommand\parg[1]{%
  {\ttfamily(}\meta{#1}{\ttfamily)}}
\raggedbottom

% -----------------------------------
\input{pgfplots.preamble.tex}

\makeatletter
% I want a two-column index just as in pgfmanual styles. This here
% was the best way to get one:
\def\index@prologue{\section*{Index}\addcontentsline{toc}{chapter}{Index}
}
\makeatother

%\RequirePackage[german,english,francais]{babel}

\def\matlabcolormaptext{This colormap is similar to one shipped with Matlab$^\text{\textregistered}$ under a similar name.}

\IfFileExists{tikzlibraryspy.code.tex}{%
\usetikzlibrary{spy}
}{%
    \message{ERROR: tikz SPY library NOT available. The manual will only compile partially.^^J}%
}%

\usepackage{xparse}% for colorbrewer manual

\usetikzlibrary{
    decorations.markings,
    decorations.footprints,
    shapes.arrows,
    matrix,
    positioning,
}

\usepgfplotslibrary{
    fillbetween,
    ternary,
    smithchart,
    patchplots,
    polar,
    colormaps,
    colorbrewer,
    colortol,           % docu not ready yet
}
\pgfqkeys{/codeexample}{%
    every codeexample/.append style={
        /pgfplots/every ternary axis/.append style={
            /pgfplots/legend style={fill=graphicbackground},
        }
    },
    tabsize=4,
}

\pgfplotsmanualenableexternalizationofexpensive

%\usetikzlibrary{external}
%\tikzexternalize[prefix=figures/]{pgfplots}

\title{%
    Manual for Package \PGFPlots{}\\
    {\small 2D/3D Plots in \LaTeX{}, Version \pgfplotsversion}\\
    {\small\url{https://github.com/pgf-tikz/pgfplots}}
    %\\{\small Attention: you are using an unstable development version.}
}

\makeatletter
\long\def\abstractsmuggle{%
    \centering
    \textbf{Abstract}\\[0.5cm]

    \begin{minipage}{12cm}
        \PGFPlots{} draws high-quality function plots in normal or logarithmic
        scaling with a user-friendly interface directly in \TeX{}. The user
        supplies axis labels, legend entries and the plot coordinates for one or
        more plots and \PGFPlots{} applies axis scaling, computes any logarithms
        and axis ticks and draws the plots. It supports line plots, scatter
        plots, piecewise constant plots, bar plots, area plots, mesh and surface
        plots, patch plots, contour plots, quiver plots, histogram plots, box
        plots, polar axes, ternary diagrams, smith charts and some more. It is
        based on Till Tantau's package \PGF{}/\Tikz{}.
    \end{minipage}

}%

\expandafter\date\expandafter{\@date\\[2cm]
    \abstractsmuggle
}%
\makeatother

%\includeonly{pgfplots.reference}


\begin{document}

\def\plotcoords{%
\addplot coordinates {
(5,8.312e-02)    (17,2.547e-02)   (49,7.407e-03)
(129,2.102e-03)  (321,5.874e-04)  (769,1.623e-04)
(1793,4.442e-05) (4097,1.207e-05) (9217,3.261e-06)
};

\addplot coordinates{
(7,8.472e-02)    (31,3.044e-02)    (111,1.022e-02)
(351,3.303e-03)  (1023,1.039e-03)  (2815,3.196e-04)
(7423,9.658e-05) (18943,2.873e-05) (47103,8.437e-06)
};

\addplot coordinates{
(9,7.881e-02)     (49,3.243e-02)    (209,1.232e-02)
(769,4.454e-03)   (2561,1.551e-03)  (7937,5.236e-04)
(23297,1.723e-04) (65537,5.545e-05) (178177,1.751e-05)
};

\addplot coordinates{
(11,6.887e-02)    (71,3.177e-02)     (351,1.341e-02)
(1471,5.334e-03)  (5503,2.027e-03)   (18943,7.415e-04)
(61183,2.628e-04) (187903,9.063e-05) (553983,3.053e-05)
};

\addplot coordinates{
(13,5.755e-02)     (97,2.925e-02)     (545,1.351e-02)
(2561,5.842e-03)   (10625,2.397e-03)  (40193,9.414e-04)
(141569,3.564e-04) (471041,1.308e-04)
(1496065,4.670e-05)
};
}%


\bookmaketitle
\tableofcontents
\include{pgfplots.title_abstract_intro}
\include{pgfplots.preliminaries}
\include{pgfplots.intro}
\include{pgfplots.reference}
\include{pgfplots.libs}
\include{pgfplots.resources}
\include{pgfplots.importexport}
\include{pgfplots.basic.reference}

\printindex

\bibliographystyle{abbrv} %gerapali} %gerabbrv} %gerunsrt.bst} %gerabbrv}% gerplain}
\nocite{pgfplotstable}
\nocite{programmingnotes}
\bibliography{pgfplots}
\end{document}

% -----------------------------------------------------------------------------
% For Stefan Pinnow as reminder on what to look for when editing the manual
% -----------------------------------------------------------------------------
% There should be no line breaks in the following environments
% - |...|
% - \declareandlabel{...}
% - \verbpdfref{...}
% If "MakeTikzPictures" isn't running through check one of the externalized
% LOG files what is the cause of that.
% -----------------------------------------------------------------------------


%%%%%%%%%%%%%%%%%%%%%%%%%%%%%%%%%%%%%%%%%%%%%%%%%%%%%%%%%%%%%%%%%%%%%%%%%%%%%
%
% Package pgfplots.sty documentation.
%
% Copyright 2007/2008 by Christian Feuersaenger.
%
% This program is free software: you can redistribute it and/or modify
% it under the terms of the GNU General Public License as published by
% the Free Software Foundation, either version 3 of the License, or
% (at your option) any later version.
%
% This program is distributed in the hope that it will be useful,
% but WITHOUT ANY WARRANTY; without even the implied warranty of
% MERCHANTABILITY or FITNESS FOR A PARTICULAR PURPOSE.  See the
% GNU General Public License for more details.
%
% You should have received a copy of the GNU General Public License
% along with this program.  If not, see <http://www.gnu.org/licenses/>.
%
%
%%%%%%%%%%%%%%%%%%%%%%%%%%%%%%%%%%%%%%%%%%%%%%%%%%%%%%%%%%%%%%%%%%%%%%%%%%%%%
% SEE pgfplots-macros.tex as well!
%\pdfminorversion=5 % to allow compression
%\pdfobjcompresslevel=2
\documentclass[a4paper,openany]{book}

\let\bookmaketitle=\maketitle

% -----------------------------------
% this here is from ltxdoc:
\usepackage{doc}[=v2]
\AtBeginDocument{\MakeShortVerb{\|}}
%\setlength{\textwidth}{355pt}
%\addtolength\marginparwidth{30pt}
%\addtolength\oddsidemargin{20pt}
%\addtolength\evensidemargin{20pt}
\def\cmd#1{\cs{\expandafter\cmd@to@cs\string#1}}
\def\cmd@to@cs#1#2{\char\number`#2\relax}
\DeclareRobustCommand\cs[1]{\texttt{\char`\\#1}}
\providecommand\marg[1]{%
  {\ttfamily\char`\{}\meta{#1}{\ttfamily\char`\}}}
\providecommand\oarg[1]{%
  {\ttfamily[}\meta{#1}{\ttfamily]}}
\providecommand\parg[1]{%
  {\ttfamily(}\meta{#1}{\ttfamily)}}
\raggedbottom

% -----------------------------------
\input{pgfplots.preamble.tex}

\makeatletter
% I want a two-column index just as in pgfmanual styles. This here
% was the best way to get one:
\def\index@prologue{\section*{Index}\addcontentsline{toc}{chapter}{Index}
}
\makeatother

%\RequirePackage[german,english,francais]{babel}

\def\matlabcolormaptext{This colormap is similar to one shipped with Matlab$^\text{\textregistered}$ under a similar name.}

\IfFileExists{tikzlibraryspy.code.tex}{%
\usetikzlibrary{spy}
}{%
    \message{ERROR: tikz SPY library NOT available. The manual will only compile partially.^^J}%
}%

\usepackage{xparse}% for colorbrewer manual

\usetikzlibrary{
    decorations.markings,
    decorations.footprints,
    shapes.arrows,
    matrix,
    positioning,
}

\usepgfplotslibrary{
    fillbetween,
    ternary,
    smithchart,
    patchplots,
    polar,
    colormaps,
    colorbrewer,
    colortol,           % docu not ready yet
}
\pgfqkeys{/codeexample}{%
    every codeexample/.append style={
        /pgfplots/every ternary axis/.append style={
            /pgfplots/legend style={fill=graphicbackground},
        }
    },
    tabsize=4,
}

\pgfplotsmanualenableexternalizationofexpensive

%\usetikzlibrary{external}
%\tikzexternalize[prefix=figures/]{pgfplots}

\title{%
    Manual for Package \PGFPlots{}\\
    {\small 2D/3D Plots in \LaTeX{}, Version \pgfplotsversion}\\
    {\small\url{https://github.com/pgf-tikz/pgfplots}}
    %\\{\small Attention: you are using an unstable development version.}
}

\makeatletter
\long\def\abstractsmuggle{%
    \centering
    \textbf{Abstract}\\[0.5cm]

    \begin{minipage}{12cm}
        \PGFPlots{} draws high-quality function plots in normal or logarithmic
        scaling with a user-friendly interface directly in \TeX{}. The user
        supplies axis labels, legend entries and the plot coordinates for one or
        more plots and \PGFPlots{} applies axis scaling, computes any logarithms
        and axis ticks and draws the plots. It supports line plots, scatter
        plots, piecewise constant plots, bar plots, area plots, mesh and surface
        plots, patch plots, contour plots, quiver plots, histogram plots, box
        plots, polar axes, ternary diagrams, smith charts and some more. It is
        based on Till Tantau's package \PGF{}/\Tikz{}.
    \end{minipage}

}%

\expandafter\date\expandafter{\@date\\[2cm]
    \abstractsmuggle
}%
\makeatother

%\includeonly{pgfplots.reference}


\begin{document}

\def\plotcoords{%
\addplot coordinates {
(5,8.312e-02)    (17,2.547e-02)   (49,7.407e-03)
(129,2.102e-03)  (321,5.874e-04)  (769,1.623e-04)
(1793,4.442e-05) (4097,1.207e-05) (9217,3.261e-06)
};

\addplot coordinates{
(7,8.472e-02)    (31,3.044e-02)    (111,1.022e-02)
(351,3.303e-03)  (1023,1.039e-03)  (2815,3.196e-04)
(7423,9.658e-05) (18943,2.873e-05) (47103,8.437e-06)
};

\addplot coordinates{
(9,7.881e-02)     (49,3.243e-02)    (209,1.232e-02)
(769,4.454e-03)   (2561,1.551e-03)  (7937,5.236e-04)
(23297,1.723e-04) (65537,5.545e-05) (178177,1.751e-05)
};

\addplot coordinates{
(11,6.887e-02)    (71,3.177e-02)     (351,1.341e-02)
(1471,5.334e-03)  (5503,2.027e-03)   (18943,7.415e-04)
(61183,2.628e-04) (187903,9.063e-05) (553983,3.053e-05)
};

\addplot coordinates{
(13,5.755e-02)     (97,2.925e-02)     (545,1.351e-02)
(2561,5.842e-03)   (10625,2.397e-03)  (40193,9.414e-04)
(141569,3.564e-04) (471041,1.308e-04)
(1496065,4.670e-05)
};
}%


\bookmaketitle
\tableofcontents
\include{pgfplots.title_abstract_intro}
\include{pgfplots.preliminaries}
\include{pgfplots.intro}
\include{pgfplots.reference}
\include{pgfplots.libs}
\include{pgfplots.resources}
\include{pgfplots.importexport}
\include{pgfplots.basic.reference}

\printindex

\bibliographystyle{abbrv} %gerapali} %gerabbrv} %gerunsrt.bst} %gerabbrv}% gerplain}
\nocite{pgfplotstable}
\nocite{programmingnotes}
\bibliography{pgfplots}
\end{document}

% -----------------------------------------------------------------------------
% For Stefan Pinnow as reminder on what to look for when editing the manual
% -----------------------------------------------------------------------------
% There should be no line breaks in the following environments
% - |...|
% - \declareandlabel{...}
% - \verbpdfref{...}
% If "MakeTikzPictures" isn't running through check one of the externalized
% LOG files what is the cause of that.
% -----------------------------------------------------------------------------


%%%%%%%%%%%%%%%%%%%%%%%%%%%%%%%%%%%%%%%%%%%%%%%%%%%%%%%%%%%%%%%%%%%%%%%%%%%%%
%
% Package pgfplots.sty documentation.
%
% Copyright 2007/2008 by Christian Feuersaenger.
%
% This program is free software: you can redistribute it and/or modify
% it under the terms of the GNU General Public License as published by
% the Free Software Foundation, either version 3 of the License, or
% (at your option) any later version.
%
% This program is distributed in the hope that it will be useful,
% but WITHOUT ANY WARRANTY; without even the implied warranty of
% MERCHANTABILITY or FITNESS FOR A PARTICULAR PURPOSE.  See the
% GNU General Public License for more details.
%
% You should have received a copy of the GNU General Public License
% along with this program.  If not, see <http://www.gnu.org/licenses/>.
%
%
%%%%%%%%%%%%%%%%%%%%%%%%%%%%%%%%%%%%%%%%%%%%%%%%%%%%%%%%%%%%%%%%%%%%%%%%%%%%%
% SEE pgfplots-macros.tex as well!
%\pdfminorversion=5 % to allow compression
%\pdfobjcompresslevel=2
\documentclass[a4paper,openany]{book}

\let\bookmaketitle=\maketitle

% -----------------------------------
% this here is from ltxdoc:
\usepackage{doc}[=v2]
\AtBeginDocument{\MakeShortVerb{\|}}
%\setlength{\textwidth}{355pt}
%\addtolength\marginparwidth{30pt}
%\addtolength\oddsidemargin{20pt}
%\addtolength\evensidemargin{20pt}
\def\cmd#1{\cs{\expandafter\cmd@to@cs\string#1}}
\def\cmd@to@cs#1#2{\char\number`#2\relax}
\DeclareRobustCommand\cs[1]{\texttt{\char`\\#1}}
\providecommand\marg[1]{%
  {\ttfamily\char`\{}\meta{#1}{\ttfamily\char`\}}}
\providecommand\oarg[1]{%
  {\ttfamily[}\meta{#1}{\ttfamily]}}
\providecommand\parg[1]{%
  {\ttfamily(}\meta{#1}{\ttfamily)}}
\raggedbottom

% -----------------------------------
\input{pgfplots.preamble.tex}

\makeatletter
% I want a two-column index just as in pgfmanual styles. This here
% was the best way to get one:
\def\index@prologue{\section*{Index}\addcontentsline{toc}{chapter}{Index}
}
\makeatother

%\RequirePackage[german,english,francais]{babel}

\def\matlabcolormaptext{This colormap is similar to one shipped with Matlab$^\text{\textregistered}$ under a similar name.}

\IfFileExists{tikzlibraryspy.code.tex}{%
\usetikzlibrary{spy}
}{%
    \message{ERROR: tikz SPY library NOT available. The manual will only compile partially.^^J}%
}%

\usepackage{xparse}% for colorbrewer manual

\usetikzlibrary{
    decorations.markings,
    decorations.footprints,
    shapes.arrows,
    matrix,
    positioning,
}

\usepgfplotslibrary{
    fillbetween,
    ternary,
    smithchart,
    patchplots,
    polar,
    colormaps,
    colorbrewer,
    colortol,           % docu not ready yet
}
\pgfqkeys{/codeexample}{%
    every codeexample/.append style={
        /pgfplots/every ternary axis/.append style={
            /pgfplots/legend style={fill=graphicbackground},
        }
    },
    tabsize=4,
}

\pgfplotsmanualenableexternalizationofexpensive

%\usetikzlibrary{external}
%\tikzexternalize[prefix=figures/]{pgfplots}

\title{%
    Manual for Package \PGFPlots{}\\
    {\small 2D/3D Plots in \LaTeX{}, Version \pgfplotsversion}\\
    {\small\url{https://github.com/pgf-tikz/pgfplots}}
    %\\{\small Attention: you are using an unstable development version.}
}

\makeatletter
\long\def\abstractsmuggle{%
    \centering
    \textbf{Abstract}\\[0.5cm]

    \begin{minipage}{12cm}
        \PGFPlots{} draws high-quality function plots in normal or logarithmic
        scaling with a user-friendly interface directly in \TeX{}. The user
        supplies axis labels, legend entries and the plot coordinates for one or
        more plots and \PGFPlots{} applies axis scaling, computes any logarithms
        and axis ticks and draws the plots. It supports line plots, scatter
        plots, piecewise constant plots, bar plots, area plots, mesh and surface
        plots, patch plots, contour plots, quiver plots, histogram plots, box
        plots, polar axes, ternary diagrams, smith charts and some more. It is
        based on Till Tantau's package \PGF{}/\Tikz{}.
    \end{minipage}

}%

\expandafter\date\expandafter{\@date\\[2cm]
    \abstractsmuggle
}%
\makeatother

%\includeonly{pgfplots.reference}


\begin{document}

\def\plotcoords{%
\addplot coordinates {
(5,8.312e-02)    (17,2.547e-02)   (49,7.407e-03)
(129,2.102e-03)  (321,5.874e-04)  (769,1.623e-04)
(1793,4.442e-05) (4097,1.207e-05) (9217,3.261e-06)
};

\addplot coordinates{
(7,8.472e-02)    (31,3.044e-02)    (111,1.022e-02)
(351,3.303e-03)  (1023,1.039e-03)  (2815,3.196e-04)
(7423,9.658e-05) (18943,2.873e-05) (47103,8.437e-06)
};

\addplot coordinates{
(9,7.881e-02)     (49,3.243e-02)    (209,1.232e-02)
(769,4.454e-03)   (2561,1.551e-03)  (7937,5.236e-04)
(23297,1.723e-04) (65537,5.545e-05) (178177,1.751e-05)
};

\addplot coordinates{
(11,6.887e-02)    (71,3.177e-02)     (351,1.341e-02)
(1471,5.334e-03)  (5503,2.027e-03)   (18943,7.415e-04)
(61183,2.628e-04) (187903,9.063e-05) (553983,3.053e-05)
};

\addplot coordinates{
(13,5.755e-02)     (97,2.925e-02)     (545,1.351e-02)
(2561,5.842e-03)   (10625,2.397e-03)  (40193,9.414e-04)
(141569,3.564e-04) (471041,1.308e-04)
(1496065,4.670e-05)
};
}%


\bookmaketitle
\tableofcontents
\include{pgfplots.title_abstract_intro}
\include{pgfplots.preliminaries}
\include{pgfplots.intro}
\include{pgfplots.reference}
\include{pgfplots.libs}
\include{pgfplots.resources}
\include{pgfplots.importexport}
\include{pgfplots.basic.reference}

\printindex

\bibliographystyle{abbrv} %gerapali} %gerabbrv} %gerunsrt.bst} %gerabbrv}% gerplain}
\nocite{pgfplotstable}
\nocite{programmingnotes}
\bibliography{pgfplots}
\end{document}

% -----------------------------------------------------------------------------
% For Stefan Pinnow as reminder on what to look for when editing the manual
% -----------------------------------------------------------------------------
% There should be no line breaks in the following environments
% - |...|
% - \declareandlabel{...}
% - \verbpdfref{...}
% If "MakeTikzPictures" isn't running through check one of the externalized
% LOG files what is the cause of that.
% -----------------------------------------------------------------------------


%%%%%%%%%%%%%%%%%%%%%%%%%%%%%%%%%%%%%%%%%%%%%%%%%%%%%%%%%%%%%%%%%%%%%%%%%%%%%
%
% Package pgfplots.sty documentation.
%
% Copyright 2007/2008 by Christian Feuersaenger.
%
% This program is free software: you can redistribute it and/or modify
% it under the terms of the GNU General Public License as published by
% the Free Software Foundation, either version 3 of the License, or
% (at your option) any later version.
%
% This program is distributed in the hope that it will be useful,
% but WITHOUT ANY WARRANTY; without even the implied warranty of
% MERCHANTABILITY or FITNESS FOR A PARTICULAR PURPOSE.  See the
% GNU General Public License for more details.
%
% You should have received a copy of the GNU General Public License
% along with this program.  If not, see <http://www.gnu.org/licenses/>.
%
%
%%%%%%%%%%%%%%%%%%%%%%%%%%%%%%%%%%%%%%%%%%%%%%%%%%%%%%%%%%%%%%%%%%%%%%%%%%%%%
% SEE pgfplots-macros.tex as well!
%\pdfminorversion=5 % to allow compression
%\pdfobjcompresslevel=2
\documentclass[a4paper,openany]{book}

\let\bookmaketitle=\maketitle

% -----------------------------------
% this here is from ltxdoc:
\usepackage{doc}[=v2]
\AtBeginDocument{\MakeShortVerb{\|}}
%\setlength{\textwidth}{355pt}
%\addtolength\marginparwidth{30pt}
%\addtolength\oddsidemargin{20pt}
%\addtolength\evensidemargin{20pt}
\def\cmd#1{\cs{\expandafter\cmd@to@cs\string#1}}
\def\cmd@to@cs#1#2{\char\number`#2\relax}
\DeclareRobustCommand\cs[1]{\texttt{\char`\\#1}}
\providecommand\marg[1]{%
  {\ttfamily\char`\{}\meta{#1}{\ttfamily\char`\}}}
\providecommand\oarg[1]{%
  {\ttfamily[}\meta{#1}{\ttfamily]}}
\providecommand\parg[1]{%
  {\ttfamily(}\meta{#1}{\ttfamily)}}
\raggedbottom

% -----------------------------------
\input{pgfplots.preamble.tex}

\makeatletter
% I want a two-column index just as in pgfmanual styles. This here
% was the best way to get one:
\def\index@prologue{\section*{Index}\addcontentsline{toc}{chapter}{Index}
}
\makeatother

%\RequirePackage[german,english,francais]{babel}

\def\matlabcolormaptext{This colormap is similar to one shipped with Matlab$^\text{\textregistered}$ under a similar name.}

\IfFileExists{tikzlibraryspy.code.tex}{%
\usetikzlibrary{spy}
}{%
    \message{ERROR: tikz SPY library NOT available. The manual will only compile partially.^^J}%
}%

\usepackage{xparse}% for colorbrewer manual

\usetikzlibrary{
    decorations.markings,
    decorations.footprints,
    shapes.arrows,
    matrix,
    positioning,
}

\usepgfplotslibrary{
    fillbetween,
    ternary,
    smithchart,
    patchplots,
    polar,
    colormaps,
    colorbrewer,
    colortol,           % docu not ready yet
}
\pgfqkeys{/codeexample}{%
    every codeexample/.append style={
        /pgfplots/every ternary axis/.append style={
            /pgfplots/legend style={fill=graphicbackground},
        }
    },
    tabsize=4,
}

\pgfplotsmanualenableexternalizationofexpensive

%\usetikzlibrary{external}
%\tikzexternalize[prefix=figures/]{pgfplots}

\title{%
    Manual for Package \PGFPlots{}\\
    {\small 2D/3D Plots in \LaTeX{}, Version \pgfplotsversion}\\
    {\small\url{https://github.com/pgf-tikz/pgfplots}}
    %\\{\small Attention: you are using an unstable development version.}
}

\makeatletter
\long\def\abstractsmuggle{%
    \centering
    \textbf{Abstract}\\[0.5cm]

    \begin{minipage}{12cm}
        \PGFPlots{} draws high-quality function plots in normal or logarithmic
        scaling with a user-friendly interface directly in \TeX{}. The user
        supplies axis labels, legend entries and the plot coordinates for one or
        more plots and \PGFPlots{} applies axis scaling, computes any logarithms
        and axis ticks and draws the plots. It supports line plots, scatter
        plots, piecewise constant plots, bar plots, area plots, mesh and surface
        plots, patch plots, contour plots, quiver plots, histogram plots, box
        plots, polar axes, ternary diagrams, smith charts and some more. It is
        based on Till Tantau's package \PGF{}/\Tikz{}.
    \end{minipage}

}%

\expandafter\date\expandafter{\@date\\[2cm]
    \abstractsmuggle
}%
\makeatother

%\includeonly{pgfplots.reference}


\begin{document}

\def\plotcoords{%
\addplot coordinates {
(5,8.312e-02)    (17,2.547e-02)   (49,7.407e-03)
(129,2.102e-03)  (321,5.874e-04)  (769,1.623e-04)
(1793,4.442e-05) (4097,1.207e-05) (9217,3.261e-06)
};

\addplot coordinates{
(7,8.472e-02)    (31,3.044e-02)    (111,1.022e-02)
(351,3.303e-03)  (1023,1.039e-03)  (2815,3.196e-04)
(7423,9.658e-05) (18943,2.873e-05) (47103,8.437e-06)
};

\addplot coordinates{
(9,7.881e-02)     (49,3.243e-02)    (209,1.232e-02)
(769,4.454e-03)   (2561,1.551e-03)  (7937,5.236e-04)
(23297,1.723e-04) (65537,5.545e-05) (178177,1.751e-05)
};

\addplot coordinates{
(11,6.887e-02)    (71,3.177e-02)     (351,1.341e-02)
(1471,5.334e-03)  (5503,2.027e-03)   (18943,7.415e-04)
(61183,2.628e-04) (187903,9.063e-05) (553983,3.053e-05)
};

\addplot coordinates{
(13,5.755e-02)     (97,2.925e-02)     (545,1.351e-02)
(2561,5.842e-03)   (10625,2.397e-03)  (40193,9.414e-04)
(141569,3.564e-04) (471041,1.308e-04)
(1496065,4.670e-05)
};
}%


\bookmaketitle
\tableofcontents
\include{pgfplots.title_abstract_intro}
\include{pgfplots.preliminaries}
\include{pgfplots.intro}
\include{pgfplots.reference}
\include{pgfplots.libs}
\include{pgfplots.resources}
\include{pgfplots.importexport}
\include{pgfplots.basic.reference}

\printindex

\bibliographystyle{abbrv} %gerapali} %gerabbrv} %gerunsrt.bst} %gerabbrv}% gerplain}
\nocite{pgfplotstable}
\nocite{programmingnotes}
\bibliography{pgfplots}
\end{document}

% -----------------------------------------------------------------------------
% For Stefan Pinnow as reminder on what to look for when editing the manual
% -----------------------------------------------------------------------------
% There should be no line breaks in the following environments
% - |...|
% - \declareandlabel{...}
% - \verbpdfref{...}
% If "MakeTikzPictures" isn't running through check one of the externalized
% LOG files what is the cause of that.
% -----------------------------------------------------------------------------


%%%%%%%%%%%%%%%%%%%%%%%%%%%%%%%%%%%%%%%%%%%%%%%%%%%%%%%%%%%%%%%%%%%%%%%%%%%%%
%
% Package pgfplots.sty documentation.
%
% Copyright 2007/2008 by Christian Feuersaenger.
%
% This program is free software: you can redistribute it and/or modify
% it under the terms of the GNU General Public License as published by
% the Free Software Foundation, either version 3 of the License, or
% (at your option) any later version.
%
% This program is distributed in the hope that it will be useful,
% but WITHOUT ANY WARRANTY; without even the implied warranty of
% MERCHANTABILITY or FITNESS FOR A PARTICULAR PURPOSE.  See the
% GNU General Public License for more details.
%
% You should have received a copy of the GNU General Public License
% along with this program.  If not, see <http://www.gnu.org/licenses/>.
%
%
%%%%%%%%%%%%%%%%%%%%%%%%%%%%%%%%%%%%%%%%%%%%%%%%%%%%%%%%%%%%%%%%%%%%%%%%%%%%%
% SEE pgfplots-macros.tex as well!
%\pdfminorversion=5 % to allow compression
%\pdfobjcompresslevel=2
\documentclass[a4paper,openany]{book}

\let\bookmaketitle=\maketitle

% -----------------------------------
% this here is from ltxdoc:
\usepackage{doc}[=v2]
\AtBeginDocument{\MakeShortVerb{\|}}
%\setlength{\textwidth}{355pt}
%\addtolength\marginparwidth{30pt}
%\addtolength\oddsidemargin{20pt}
%\addtolength\evensidemargin{20pt}
\def\cmd#1{\cs{\expandafter\cmd@to@cs\string#1}}
\def\cmd@to@cs#1#2{\char\number`#2\relax}
\DeclareRobustCommand\cs[1]{\texttt{\char`\\#1}}
\providecommand\marg[1]{%
  {\ttfamily\char`\{}\meta{#1}{\ttfamily\char`\}}}
\providecommand\oarg[1]{%
  {\ttfamily[}\meta{#1}{\ttfamily]}}
\providecommand\parg[1]{%
  {\ttfamily(}\meta{#1}{\ttfamily)}}
\raggedbottom

% -----------------------------------
\input{pgfplots.preamble.tex}

\makeatletter
% I want a two-column index just as in pgfmanual styles. This here
% was the best way to get one:
\def\index@prologue{\section*{Index}\addcontentsline{toc}{chapter}{Index}
}
\makeatother

%\RequirePackage[german,english,francais]{babel}

\def\matlabcolormaptext{This colormap is similar to one shipped with Matlab$^\text{\textregistered}$ under a similar name.}

\IfFileExists{tikzlibraryspy.code.tex}{%
\usetikzlibrary{spy}
}{%
    \message{ERROR: tikz SPY library NOT available. The manual will only compile partially.^^J}%
}%

\usepackage{xparse}% for colorbrewer manual

\usetikzlibrary{
    decorations.markings,
    decorations.footprints,
    shapes.arrows,
    matrix,
    positioning,
}

\usepgfplotslibrary{
    fillbetween,
    ternary,
    smithchart,
    patchplots,
    polar,
    colormaps,
    colorbrewer,
    colortol,           % docu not ready yet
}
\pgfqkeys{/codeexample}{%
    every codeexample/.append style={
        /pgfplots/every ternary axis/.append style={
            /pgfplots/legend style={fill=graphicbackground},
        }
    },
    tabsize=4,
}

\pgfplotsmanualenableexternalizationofexpensive

%\usetikzlibrary{external}
%\tikzexternalize[prefix=figures/]{pgfplots}

\title{%
    Manual for Package \PGFPlots{}\\
    {\small 2D/3D Plots in \LaTeX{}, Version \pgfplotsversion}\\
    {\small\url{https://github.com/pgf-tikz/pgfplots}}
    %\\{\small Attention: you are using an unstable development version.}
}

\makeatletter
\long\def\abstractsmuggle{%
    \centering
    \textbf{Abstract}\\[0.5cm]

    \begin{minipage}{12cm}
        \PGFPlots{} draws high-quality function plots in normal or logarithmic
        scaling with a user-friendly interface directly in \TeX{}. The user
        supplies axis labels, legend entries and the plot coordinates for one or
        more plots and \PGFPlots{} applies axis scaling, computes any logarithms
        and axis ticks and draws the plots. It supports line plots, scatter
        plots, piecewise constant plots, bar plots, area plots, mesh and surface
        plots, patch plots, contour plots, quiver plots, histogram plots, box
        plots, polar axes, ternary diagrams, smith charts and some more. It is
        based on Till Tantau's package \PGF{}/\Tikz{}.
    \end{minipage}

}%

\expandafter\date\expandafter{\@date\\[2cm]
    \abstractsmuggle
}%
\makeatother

%\includeonly{pgfplots.reference}


\begin{document}

\def\plotcoords{%
\addplot coordinates {
(5,8.312e-02)    (17,2.547e-02)   (49,7.407e-03)
(129,2.102e-03)  (321,5.874e-04)  (769,1.623e-04)
(1793,4.442e-05) (4097,1.207e-05) (9217,3.261e-06)
};

\addplot coordinates{
(7,8.472e-02)    (31,3.044e-02)    (111,1.022e-02)
(351,3.303e-03)  (1023,1.039e-03)  (2815,3.196e-04)
(7423,9.658e-05) (18943,2.873e-05) (47103,8.437e-06)
};

\addplot coordinates{
(9,7.881e-02)     (49,3.243e-02)    (209,1.232e-02)
(769,4.454e-03)   (2561,1.551e-03)  (7937,5.236e-04)
(23297,1.723e-04) (65537,5.545e-05) (178177,1.751e-05)
};

\addplot coordinates{
(11,6.887e-02)    (71,3.177e-02)     (351,1.341e-02)
(1471,5.334e-03)  (5503,2.027e-03)   (18943,7.415e-04)
(61183,2.628e-04) (187903,9.063e-05) (553983,3.053e-05)
};

\addplot coordinates{
(13,5.755e-02)     (97,2.925e-02)     (545,1.351e-02)
(2561,5.842e-03)   (10625,2.397e-03)  (40193,9.414e-04)
(141569,3.564e-04) (471041,1.308e-04)
(1496065,4.670e-05)
};
}%


\bookmaketitle
\tableofcontents
\include{pgfplots.title_abstract_intro}
\include{pgfplots.preliminaries}
\include{pgfplots.intro}
\include{pgfplots.reference}
\include{pgfplots.libs}
\include{pgfplots.resources}
\include{pgfplots.importexport}
\include{pgfplots.basic.reference}

\printindex

\bibliographystyle{abbrv} %gerapali} %gerabbrv} %gerunsrt.bst} %gerabbrv}% gerplain}
\nocite{pgfplotstable}
\nocite{programmingnotes}
\bibliography{pgfplots}
\end{document}

% -----------------------------------------------------------------------------
% For Stefan Pinnow as reminder on what to look for when editing the manual
% -----------------------------------------------------------------------------
% There should be no line breaks in the following environments
% - |...|
% - \declareandlabel{...}
% - \verbpdfref{...}
% If "MakeTikzPictures" isn't running through check one of the externalized
% LOG files what is the cause of that.
% -----------------------------------------------------------------------------


%%%%%%%%%%%%%%%%%%%%%%%%%%%%%%%%%%%%%%%%%%%%%%%%%%%%%%%%%%%%%%%%%%%%%%%%%%%%%
%
% Package pgfplots.sty documentation.
%
% Copyright 2007/2008 by Christian Feuersaenger.
%
% This program is free software: you can redistribute it and/or modify
% it under the terms of the GNU General Public License as published by
% the Free Software Foundation, either version 3 of the License, or
% (at your option) any later version.
%
% This program is distributed in the hope that it will be useful,
% but WITHOUT ANY WARRANTY; without even the implied warranty of
% MERCHANTABILITY or FITNESS FOR A PARTICULAR PURPOSE.  See the
% GNU General Public License for more details.
%
% You should have received a copy of the GNU General Public License
% along with this program.  If not, see <http://www.gnu.org/licenses/>.
%
%
%%%%%%%%%%%%%%%%%%%%%%%%%%%%%%%%%%%%%%%%%%%%%%%%%%%%%%%%%%%%%%%%%%%%%%%%%%%%%
% SEE pgfplots-macros.tex as well!
%\pdfminorversion=5 % to allow compression
%\pdfobjcompresslevel=2
\documentclass[a4paper,openany]{book}

\let\bookmaketitle=\maketitle

% -----------------------------------
% this here is from ltxdoc:
\usepackage{doc}[=v2]
\AtBeginDocument{\MakeShortVerb{\|}}
%\setlength{\textwidth}{355pt}
%\addtolength\marginparwidth{30pt}
%\addtolength\oddsidemargin{20pt}
%\addtolength\evensidemargin{20pt}
\def\cmd#1{\cs{\expandafter\cmd@to@cs\string#1}}
\def\cmd@to@cs#1#2{\char\number`#2\relax}
\DeclareRobustCommand\cs[1]{\texttt{\char`\\#1}}
\providecommand\marg[1]{%
  {\ttfamily\char`\{}\meta{#1}{\ttfamily\char`\}}}
\providecommand\oarg[1]{%
  {\ttfamily[}\meta{#1}{\ttfamily]}}
\providecommand\parg[1]{%
  {\ttfamily(}\meta{#1}{\ttfamily)}}
\raggedbottom

% -----------------------------------
\input{pgfplots.preamble.tex}

\makeatletter
% I want a two-column index just as in pgfmanual styles. This here
% was the best way to get one:
\def\index@prologue{\section*{Index}\addcontentsline{toc}{chapter}{Index}
}
\makeatother

%\RequirePackage[german,english,francais]{babel}

\def\matlabcolormaptext{This colormap is similar to one shipped with Matlab$^\text{\textregistered}$ under a similar name.}

\IfFileExists{tikzlibraryspy.code.tex}{%
\usetikzlibrary{spy}
}{%
    \message{ERROR: tikz SPY library NOT available. The manual will only compile partially.^^J}%
}%

\usepackage{xparse}% for colorbrewer manual

\usetikzlibrary{
    decorations.markings,
    decorations.footprints,
    shapes.arrows,
    matrix,
    positioning,
}

\usepgfplotslibrary{
    fillbetween,
    ternary,
    smithchart,
    patchplots,
    polar,
    colormaps,
    colorbrewer,
    colortol,           % docu not ready yet
}
\pgfqkeys{/codeexample}{%
    every codeexample/.append style={
        /pgfplots/every ternary axis/.append style={
            /pgfplots/legend style={fill=graphicbackground},
        }
    },
    tabsize=4,
}

\pgfplotsmanualenableexternalizationofexpensive

%\usetikzlibrary{external}
%\tikzexternalize[prefix=figures/]{pgfplots}

\title{%
    Manual for Package \PGFPlots{}\\
    {\small 2D/3D Plots in \LaTeX{}, Version \pgfplotsversion}\\
    {\small\url{https://github.com/pgf-tikz/pgfplots}}
    %\\{\small Attention: you are using an unstable development version.}
}

\makeatletter
\long\def\abstractsmuggle{%
    \centering
    \textbf{Abstract}\\[0.5cm]

    \begin{minipage}{12cm}
        \PGFPlots{} draws high-quality function plots in normal or logarithmic
        scaling with a user-friendly interface directly in \TeX{}. The user
        supplies axis labels, legend entries and the plot coordinates for one or
        more plots and \PGFPlots{} applies axis scaling, computes any logarithms
        and axis ticks and draws the plots. It supports line plots, scatter
        plots, piecewise constant plots, bar plots, area plots, mesh and surface
        plots, patch plots, contour plots, quiver plots, histogram plots, box
        plots, polar axes, ternary diagrams, smith charts and some more. It is
        based on Till Tantau's package \PGF{}/\Tikz{}.
    \end{minipage}

}%

\expandafter\date\expandafter{\@date\\[2cm]
    \abstractsmuggle
}%
\makeatother

%\includeonly{pgfplots.reference}


\begin{document}

\def\plotcoords{%
\addplot coordinates {
(5,8.312e-02)    (17,2.547e-02)   (49,7.407e-03)
(129,2.102e-03)  (321,5.874e-04)  (769,1.623e-04)
(1793,4.442e-05) (4097,1.207e-05) (9217,3.261e-06)
};

\addplot coordinates{
(7,8.472e-02)    (31,3.044e-02)    (111,1.022e-02)
(351,3.303e-03)  (1023,1.039e-03)  (2815,3.196e-04)
(7423,9.658e-05) (18943,2.873e-05) (47103,8.437e-06)
};

\addplot coordinates{
(9,7.881e-02)     (49,3.243e-02)    (209,1.232e-02)
(769,4.454e-03)   (2561,1.551e-03)  (7937,5.236e-04)
(23297,1.723e-04) (65537,5.545e-05) (178177,1.751e-05)
};

\addplot coordinates{
(11,6.887e-02)    (71,3.177e-02)     (351,1.341e-02)
(1471,5.334e-03)  (5503,2.027e-03)   (18943,7.415e-04)
(61183,2.628e-04) (187903,9.063e-05) (553983,3.053e-05)
};

\addplot coordinates{
(13,5.755e-02)     (97,2.925e-02)     (545,1.351e-02)
(2561,5.842e-03)   (10625,2.397e-03)  (40193,9.414e-04)
(141569,3.564e-04) (471041,1.308e-04)
(1496065,4.670e-05)
};
}%


\bookmaketitle
\tableofcontents
\include{pgfplots.title_abstract_intro}
\include{pgfplots.preliminaries}
\include{pgfplots.intro}
\include{pgfplots.reference}
\include{pgfplots.libs}
\include{pgfplots.resources}
\include{pgfplots.importexport}
\include{pgfplots.basic.reference}

\printindex

\bibliographystyle{abbrv} %gerapali} %gerabbrv} %gerunsrt.bst} %gerabbrv}% gerplain}
\nocite{pgfplotstable}
\nocite{programmingnotes}
\bibliography{pgfplots}
\end{document}

% -----------------------------------------------------------------------------
% For Stefan Pinnow as reminder on what to look for when editing the manual
% -----------------------------------------------------------------------------
% There should be no line breaks in the following environments
% - |...|
% - \declareandlabel{...}
% - \verbpdfref{...}
% If "MakeTikzPictures" isn't running through check one of the externalized
% LOG files what is the cause of that.
% -----------------------------------------------------------------------------

\include{pgfplots.basic.reference}

\printindex

\bibliographystyle{abbrv} %gerapali} %gerabbrv} %gerunsrt.bst} %gerabbrv}% gerplain}
\nocite{pgfplotstable}
\nocite{programmingnotes}
\bibliography{pgfplots}
\end{document}

% -----------------------------------------------------------------------------
% For Stefan Pinnow as reminder on what to look for when editing the manual
% -----------------------------------------------------------------------------
% There should be no line breaks in the following environments
% - |...|
% - \declareandlabel{...}
% - \verbpdfref{...}
% If "MakeTikzPictures" isn't running through check one of the externalized
% LOG files what is the cause of that.
% -----------------------------------------------------------------------------


%%%%%%%%%%%%%%%%%%%%%%%%%%%%%%%%%%%%%%%%%%%%%%%%%%%%%%%%%%%%%%%%%%%%%%%%%%%%%
%
% Package pgfplots.sty documentation.
%
% Copyright 2007/2008 by Christian Feuersaenger.
%
% This program is free software: you can redistribute it and/or modify
% it under the terms of the GNU General Public License as published by
% the Free Software Foundation, either version 3 of the License, or
% (at your option) any later version.
%
% This program is distributed in the hope that it will be useful,
% but WITHOUT ANY WARRANTY; without even the implied warranty of
% MERCHANTABILITY or FITNESS FOR A PARTICULAR PURPOSE.  See the
% GNU General Public License for more details.
%
% You should have received a copy of the GNU General Public License
% along with this program.  If not, see <http://www.gnu.org/licenses/>.
%
%
%%%%%%%%%%%%%%%%%%%%%%%%%%%%%%%%%%%%%%%%%%%%%%%%%%%%%%%%%%%%%%%%%%%%%%%%%%%%%
% SEE pgfplots-macros.tex as well!
%\pdfminorversion=5 % to allow compression
%\pdfobjcompresslevel=2
\documentclass[a4paper,openany]{book}

\let\bookmaketitle=\maketitle

% -----------------------------------
% this here is from ltxdoc:
\usepackage{doc}[=v2]
\AtBeginDocument{\MakeShortVerb{\|}}
%\setlength{\textwidth}{355pt}
%\addtolength\marginparwidth{30pt}
%\addtolength\oddsidemargin{20pt}
%\addtolength\evensidemargin{20pt}
\def\cmd#1{\cs{\expandafter\cmd@to@cs\string#1}}
\def\cmd@to@cs#1#2{\char\number`#2\relax}
\DeclareRobustCommand\cs[1]{\texttt{\char`\\#1}}
\providecommand\marg[1]{%
  {\ttfamily\char`\{}\meta{#1}{\ttfamily\char`\}}}
\providecommand\oarg[1]{%
  {\ttfamily[}\meta{#1}{\ttfamily]}}
\providecommand\parg[1]{%
  {\ttfamily(}\meta{#1}{\ttfamily)}}
\raggedbottom

% -----------------------------------
\input{pgfplots.preamble.tex}

\makeatletter
% I want a two-column index just as in pgfmanual styles. This here
% was the best way to get one:
\def\index@prologue{\section*{Index}\addcontentsline{toc}{chapter}{Index}
}
\makeatother

%\RequirePackage[german,english,francais]{babel}

\def\matlabcolormaptext{This colormap is similar to one shipped with Matlab$^\text{\textregistered}$ under a similar name.}

\IfFileExists{tikzlibraryspy.code.tex}{%
\usetikzlibrary{spy}
}{%
    \message{ERROR: tikz SPY library NOT available. The manual will only compile partially.^^J}%
}%

\usepackage{xparse}% for colorbrewer manual

\usetikzlibrary{
    decorations.markings,
    decorations.footprints,
    shapes.arrows,
    matrix,
    positioning,
}

\usepgfplotslibrary{
    fillbetween,
    ternary,
    smithchart,
    patchplots,
    polar,
    colormaps,
    colorbrewer,
    colortol,           % docu not ready yet
}
\pgfqkeys{/codeexample}{%
    every codeexample/.append style={
        /pgfplots/every ternary axis/.append style={
            /pgfplots/legend style={fill=graphicbackground},
        }
    },
    tabsize=4,
}

\pgfplotsmanualenableexternalizationofexpensive

%\usetikzlibrary{external}
%\tikzexternalize[prefix=figures/]{pgfplots}

\title{%
    Manual for Package \PGFPlots{}\\
    {\small 2D/3D Plots in \LaTeX{}, Version \pgfplotsversion}\\
    {\small\url{https://github.com/pgf-tikz/pgfplots}}
    %\\{\small Attention: you are using an unstable development version.}
}

\makeatletter
\long\def\abstractsmuggle{%
    \centering
    \textbf{Abstract}\\[0.5cm]

    \begin{minipage}{12cm}
        \PGFPlots{} draws high-quality function plots in normal or logarithmic
        scaling with a user-friendly interface directly in \TeX{}. The user
        supplies axis labels, legend entries and the plot coordinates for one or
        more plots and \PGFPlots{} applies axis scaling, computes any logarithms
        and axis ticks and draws the plots. It supports line plots, scatter
        plots, piecewise constant plots, bar plots, area plots, mesh and surface
        plots, patch plots, contour plots, quiver plots, histogram plots, box
        plots, polar axes, ternary diagrams, smith charts and some more. It is
        based on Till Tantau's package \PGF{}/\Tikz{}.
    \end{minipage}

}%

\expandafter\date\expandafter{\@date\\[2cm]
    \abstractsmuggle
}%
\makeatother

%\includeonly{pgfplots.reference}


\begin{document}

\def\plotcoords{%
\addplot coordinates {
(5,8.312e-02)    (17,2.547e-02)   (49,7.407e-03)
(129,2.102e-03)  (321,5.874e-04)  (769,1.623e-04)
(1793,4.442e-05) (4097,1.207e-05) (9217,3.261e-06)
};

\addplot coordinates{
(7,8.472e-02)    (31,3.044e-02)    (111,1.022e-02)
(351,3.303e-03)  (1023,1.039e-03)  (2815,3.196e-04)
(7423,9.658e-05) (18943,2.873e-05) (47103,8.437e-06)
};

\addplot coordinates{
(9,7.881e-02)     (49,3.243e-02)    (209,1.232e-02)
(769,4.454e-03)   (2561,1.551e-03)  (7937,5.236e-04)
(23297,1.723e-04) (65537,5.545e-05) (178177,1.751e-05)
};

\addplot coordinates{
(11,6.887e-02)    (71,3.177e-02)     (351,1.341e-02)
(1471,5.334e-03)  (5503,2.027e-03)   (18943,7.415e-04)
(61183,2.628e-04) (187903,9.063e-05) (553983,3.053e-05)
};

\addplot coordinates{
(13,5.755e-02)     (97,2.925e-02)     (545,1.351e-02)
(2561,5.842e-03)   (10625,2.397e-03)  (40193,9.414e-04)
(141569,3.564e-04) (471041,1.308e-04)
(1496065,4.670e-05)
};
}%


\bookmaketitle
\tableofcontents

%%%%%%%%%%%%%%%%%%%%%%%%%%%%%%%%%%%%%%%%%%%%%%%%%%%%%%%%%%%%%%%%%%%%%%%%%%%%%
%
% Package pgfplots.sty documentation.
%
% Copyright 2007/2008 by Christian Feuersaenger.
%
% This program is free software: you can redistribute it and/or modify
% it under the terms of the GNU General Public License as published by
% the Free Software Foundation, either version 3 of the License, or
% (at your option) any later version.
%
% This program is distributed in the hope that it will be useful,
% but WITHOUT ANY WARRANTY; without even the implied warranty of
% MERCHANTABILITY or FITNESS FOR A PARTICULAR PURPOSE.  See the
% GNU General Public License for more details.
%
% You should have received a copy of the GNU General Public License
% along with this program.  If not, see <http://www.gnu.org/licenses/>.
%
%
%%%%%%%%%%%%%%%%%%%%%%%%%%%%%%%%%%%%%%%%%%%%%%%%%%%%%%%%%%%%%%%%%%%%%%%%%%%%%
% SEE pgfplots-macros.tex as well!
%\pdfminorversion=5 % to allow compression
%\pdfobjcompresslevel=2
\documentclass[a4paper,openany]{book}

\let\bookmaketitle=\maketitle

% -----------------------------------
% this here is from ltxdoc:
\usepackage{doc}[=v2]
\AtBeginDocument{\MakeShortVerb{\|}}
%\setlength{\textwidth}{355pt}
%\addtolength\marginparwidth{30pt}
%\addtolength\oddsidemargin{20pt}
%\addtolength\evensidemargin{20pt}
\def\cmd#1{\cs{\expandafter\cmd@to@cs\string#1}}
\def\cmd@to@cs#1#2{\char\number`#2\relax}
\DeclareRobustCommand\cs[1]{\texttt{\char`\\#1}}
\providecommand\marg[1]{%
  {\ttfamily\char`\{}\meta{#1}{\ttfamily\char`\}}}
\providecommand\oarg[1]{%
  {\ttfamily[}\meta{#1}{\ttfamily]}}
\providecommand\parg[1]{%
  {\ttfamily(}\meta{#1}{\ttfamily)}}
\raggedbottom

% -----------------------------------
\input{pgfplots.preamble.tex}

\makeatletter
% I want a two-column index just as in pgfmanual styles. This here
% was the best way to get one:
\def\index@prologue{\section*{Index}\addcontentsline{toc}{chapter}{Index}
}
\makeatother

%\RequirePackage[german,english,francais]{babel}

\def\matlabcolormaptext{This colormap is similar to one shipped with Matlab$^\text{\textregistered}$ under a similar name.}

\IfFileExists{tikzlibraryspy.code.tex}{%
\usetikzlibrary{spy}
}{%
    \message{ERROR: tikz SPY library NOT available. The manual will only compile partially.^^J}%
}%

\usepackage{xparse}% for colorbrewer manual

\usetikzlibrary{
    decorations.markings,
    decorations.footprints,
    shapes.arrows,
    matrix,
    positioning,
}

\usepgfplotslibrary{
    fillbetween,
    ternary,
    smithchart,
    patchplots,
    polar,
    colormaps,
    colorbrewer,
    colortol,           % docu not ready yet
}
\pgfqkeys{/codeexample}{%
    every codeexample/.append style={
        /pgfplots/every ternary axis/.append style={
            /pgfplots/legend style={fill=graphicbackground},
        }
    },
    tabsize=4,
}

\pgfplotsmanualenableexternalizationofexpensive

%\usetikzlibrary{external}
%\tikzexternalize[prefix=figures/]{pgfplots}

\title{%
    Manual for Package \PGFPlots{}\\
    {\small 2D/3D Plots in \LaTeX{}, Version \pgfplotsversion}\\
    {\small\url{https://github.com/pgf-tikz/pgfplots}}
    %\\{\small Attention: you are using an unstable development version.}
}

\makeatletter
\long\def\abstractsmuggle{%
    \centering
    \textbf{Abstract}\\[0.5cm]

    \begin{minipage}{12cm}
        \PGFPlots{} draws high-quality function plots in normal or logarithmic
        scaling with a user-friendly interface directly in \TeX{}. The user
        supplies axis labels, legend entries and the plot coordinates for one or
        more plots and \PGFPlots{} applies axis scaling, computes any logarithms
        and axis ticks and draws the plots. It supports line plots, scatter
        plots, piecewise constant plots, bar plots, area plots, mesh and surface
        plots, patch plots, contour plots, quiver plots, histogram plots, box
        plots, polar axes, ternary diagrams, smith charts and some more. It is
        based on Till Tantau's package \PGF{}/\Tikz{}.
    \end{minipage}

}%

\expandafter\date\expandafter{\@date\\[2cm]
    \abstractsmuggle
}%
\makeatother

%\includeonly{pgfplots.reference}


\begin{document}

\def\plotcoords{%
\addplot coordinates {
(5,8.312e-02)    (17,2.547e-02)   (49,7.407e-03)
(129,2.102e-03)  (321,5.874e-04)  (769,1.623e-04)
(1793,4.442e-05) (4097,1.207e-05) (9217,3.261e-06)
};

\addplot coordinates{
(7,8.472e-02)    (31,3.044e-02)    (111,1.022e-02)
(351,3.303e-03)  (1023,1.039e-03)  (2815,3.196e-04)
(7423,9.658e-05) (18943,2.873e-05) (47103,8.437e-06)
};

\addplot coordinates{
(9,7.881e-02)     (49,3.243e-02)    (209,1.232e-02)
(769,4.454e-03)   (2561,1.551e-03)  (7937,5.236e-04)
(23297,1.723e-04) (65537,5.545e-05) (178177,1.751e-05)
};

\addplot coordinates{
(11,6.887e-02)    (71,3.177e-02)     (351,1.341e-02)
(1471,5.334e-03)  (5503,2.027e-03)   (18943,7.415e-04)
(61183,2.628e-04) (187903,9.063e-05) (553983,3.053e-05)
};

\addplot coordinates{
(13,5.755e-02)     (97,2.925e-02)     (545,1.351e-02)
(2561,5.842e-03)   (10625,2.397e-03)  (40193,9.414e-04)
(141569,3.564e-04) (471041,1.308e-04)
(1496065,4.670e-05)
};
}%


\bookmaketitle
\tableofcontents
\include{pgfplots.title_abstract_intro}
\include{pgfplots.preliminaries}
\include{pgfplots.intro}
\include{pgfplots.reference}
\include{pgfplots.libs}
\include{pgfplots.resources}
\include{pgfplots.importexport}
\include{pgfplots.basic.reference}

\printindex

\bibliographystyle{abbrv} %gerapali} %gerabbrv} %gerunsrt.bst} %gerabbrv}% gerplain}
\nocite{pgfplotstable}
\nocite{programmingnotes}
\bibliography{pgfplots}
\end{document}

% -----------------------------------------------------------------------------
% For Stefan Pinnow as reminder on what to look for when editing the manual
% -----------------------------------------------------------------------------
% There should be no line breaks in the following environments
% - |...|
% - \declareandlabel{...}
% - \verbpdfref{...}
% If "MakeTikzPictures" isn't running through check one of the externalized
% LOG files what is the cause of that.
% -----------------------------------------------------------------------------


%%%%%%%%%%%%%%%%%%%%%%%%%%%%%%%%%%%%%%%%%%%%%%%%%%%%%%%%%%%%%%%%%%%%%%%%%%%%%
%
% Package pgfplots.sty documentation.
%
% Copyright 2007/2008 by Christian Feuersaenger.
%
% This program is free software: you can redistribute it and/or modify
% it under the terms of the GNU General Public License as published by
% the Free Software Foundation, either version 3 of the License, or
% (at your option) any later version.
%
% This program is distributed in the hope that it will be useful,
% but WITHOUT ANY WARRANTY; without even the implied warranty of
% MERCHANTABILITY or FITNESS FOR A PARTICULAR PURPOSE.  See the
% GNU General Public License for more details.
%
% You should have received a copy of the GNU General Public License
% along with this program.  If not, see <http://www.gnu.org/licenses/>.
%
%
%%%%%%%%%%%%%%%%%%%%%%%%%%%%%%%%%%%%%%%%%%%%%%%%%%%%%%%%%%%%%%%%%%%%%%%%%%%%%
% SEE pgfplots-macros.tex as well!
%\pdfminorversion=5 % to allow compression
%\pdfobjcompresslevel=2
\documentclass[a4paper,openany]{book}

\let\bookmaketitle=\maketitle

% -----------------------------------
% this here is from ltxdoc:
\usepackage{doc}[=v2]
\AtBeginDocument{\MakeShortVerb{\|}}
%\setlength{\textwidth}{355pt}
%\addtolength\marginparwidth{30pt}
%\addtolength\oddsidemargin{20pt}
%\addtolength\evensidemargin{20pt}
\def\cmd#1{\cs{\expandafter\cmd@to@cs\string#1}}
\def\cmd@to@cs#1#2{\char\number`#2\relax}
\DeclareRobustCommand\cs[1]{\texttt{\char`\\#1}}
\providecommand\marg[1]{%
  {\ttfamily\char`\{}\meta{#1}{\ttfamily\char`\}}}
\providecommand\oarg[1]{%
  {\ttfamily[}\meta{#1}{\ttfamily]}}
\providecommand\parg[1]{%
  {\ttfamily(}\meta{#1}{\ttfamily)}}
\raggedbottom

% -----------------------------------
\input{pgfplots.preamble.tex}

\makeatletter
% I want a two-column index just as in pgfmanual styles. This here
% was the best way to get one:
\def\index@prologue{\section*{Index}\addcontentsline{toc}{chapter}{Index}
}
\makeatother

%\RequirePackage[german,english,francais]{babel}

\def\matlabcolormaptext{This colormap is similar to one shipped with Matlab$^\text{\textregistered}$ under a similar name.}

\IfFileExists{tikzlibraryspy.code.tex}{%
\usetikzlibrary{spy}
}{%
    \message{ERROR: tikz SPY library NOT available. The manual will only compile partially.^^J}%
}%

\usepackage{xparse}% for colorbrewer manual

\usetikzlibrary{
    decorations.markings,
    decorations.footprints,
    shapes.arrows,
    matrix,
    positioning,
}

\usepgfplotslibrary{
    fillbetween,
    ternary,
    smithchart,
    patchplots,
    polar,
    colormaps,
    colorbrewer,
    colortol,           % docu not ready yet
}
\pgfqkeys{/codeexample}{%
    every codeexample/.append style={
        /pgfplots/every ternary axis/.append style={
            /pgfplots/legend style={fill=graphicbackground},
        }
    },
    tabsize=4,
}

\pgfplotsmanualenableexternalizationofexpensive

%\usetikzlibrary{external}
%\tikzexternalize[prefix=figures/]{pgfplots}

\title{%
    Manual for Package \PGFPlots{}\\
    {\small 2D/3D Plots in \LaTeX{}, Version \pgfplotsversion}\\
    {\small\url{https://github.com/pgf-tikz/pgfplots}}
    %\\{\small Attention: you are using an unstable development version.}
}

\makeatletter
\long\def\abstractsmuggle{%
    \centering
    \textbf{Abstract}\\[0.5cm]

    \begin{minipage}{12cm}
        \PGFPlots{} draws high-quality function plots in normal or logarithmic
        scaling with a user-friendly interface directly in \TeX{}. The user
        supplies axis labels, legend entries and the plot coordinates for one or
        more plots and \PGFPlots{} applies axis scaling, computes any logarithms
        and axis ticks and draws the plots. It supports line plots, scatter
        plots, piecewise constant plots, bar plots, area plots, mesh and surface
        plots, patch plots, contour plots, quiver plots, histogram plots, box
        plots, polar axes, ternary diagrams, smith charts and some more. It is
        based on Till Tantau's package \PGF{}/\Tikz{}.
    \end{minipage}

}%

\expandafter\date\expandafter{\@date\\[2cm]
    \abstractsmuggle
}%
\makeatother

%\includeonly{pgfplots.reference}


\begin{document}

\def\plotcoords{%
\addplot coordinates {
(5,8.312e-02)    (17,2.547e-02)   (49,7.407e-03)
(129,2.102e-03)  (321,5.874e-04)  (769,1.623e-04)
(1793,4.442e-05) (4097,1.207e-05) (9217,3.261e-06)
};

\addplot coordinates{
(7,8.472e-02)    (31,3.044e-02)    (111,1.022e-02)
(351,3.303e-03)  (1023,1.039e-03)  (2815,3.196e-04)
(7423,9.658e-05) (18943,2.873e-05) (47103,8.437e-06)
};

\addplot coordinates{
(9,7.881e-02)     (49,3.243e-02)    (209,1.232e-02)
(769,4.454e-03)   (2561,1.551e-03)  (7937,5.236e-04)
(23297,1.723e-04) (65537,5.545e-05) (178177,1.751e-05)
};

\addplot coordinates{
(11,6.887e-02)    (71,3.177e-02)     (351,1.341e-02)
(1471,5.334e-03)  (5503,2.027e-03)   (18943,7.415e-04)
(61183,2.628e-04) (187903,9.063e-05) (553983,3.053e-05)
};

\addplot coordinates{
(13,5.755e-02)     (97,2.925e-02)     (545,1.351e-02)
(2561,5.842e-03)   (10625,2.397e-03)  (40193,9.414e-04)
(141569,3.564e-04) (471041,1.308e-04)
(1496065,4.670e-05)
};
}%


\bookmaketitle
\tableofcontents
\include{pgfplots.title_abstract_intro}
\include{pgfplots.preliminaries}
\include{pgfplots.intro}
\include{pgfplots.reference}
\include{pgfplots.libs}
\include{pgfplots.resources}
\include{pgfplots.importexport}
\include{pgfplots.basic.reference}

\printindex

\bibliographystyle{abbrv} %gerapali} %gerabbrv} %gerunsrt.bst} %gerabbrv}% gerplain}
\nocite{pgfplotstable}
\nocite{programmingnotes}
\bibliography{pgfplots}
\end{document}

% -----------------------------------------------------------------------------
% For Stefan Pinnow as reminder on what to look for when editing the manual
% -----------------------------------------------------------------------------
% There should be no line breaks in the following environments
% - |...|
% - \declareandlabel{...}
% - \verbpdfref{...}
% If "MakeTikzPictures" isn't running through check one of the externalized
% LOG files what is the cause of that.
% -----------------------------------------------------------------------------


%%%%%%%%%%%%%%%%%%%%%%%%%%%%%%%%%%%%%%%%%%%%%%%%%%%%%%%%%%%%%%%%%%%%%%%%%%%%%
%
% Package pgfplots.sty documentation.
%
% Copyright 2007/2008 by Christian Feuersaenger.
%
% This program is free software: you can redistribute it and/or modify
% it under the terms of the GNU General Public License as published by
% the Free Software Foundation, either version 3 of the License, or
% (at your option) any later version.
%
% This program is distributed in the hope that it will be useful,
% but WITHOUT ANY WARRANTY; without even the implied warranty of
% MERCHANTABILITY or FITNESS FOR A PARTICULAR PURPOSE.  See the
% GNU General Public License for more details.
%
% You should have received a copy of the GNU General Public License
% along with this program.  If not, see <http://www.gnu.org/licenses/>.
%
%
%%%%%%%%%%%%%%%%%%%%%%%%%%%%%%%%%%%%%%%%%%%%%%%%%%%%%%%%%%%%%%%%%%%%%%%%%%%%%
% SEE pgfplots-macros.tex as well!
%\pdfminorversion=5 % to allow compression
%\pdfobjcompresslevel=2
\documentclass[a4paper,openany]{book}

\let\bookmaketitle=\maketitle

% -----------------------------------
% this here is from ltxdoc:
\usepackage{doc}[=v2]
\AtBeginDocument{\MakeShortVerb{\|}}
%\setlength{\textwidth}{355pt}
%\addtolength\marginparwidth{30pt}
%\addtolength\oddsidemargin{20pt}
%\addtolength\evensidemargin{20pt}
\def\cmd#1{\cs{\expandafter\cmd@to@cs\string#1}}
\def\cmd@to@cs#1#2{\char\number`#2\relax}
\DeclareRobustCommand\cs[1]{\texttt{\char`\\#1}}
\providecommand\marg[1]{%
  {\ttfamily\char`\{}\meta{#1}{\ttfamily\char`\}}}
\providecommand\oarg[1]{%
  {\ttfamily[}\meta{#1}{\ttfamily]}}
\providecommand\parg[1]{%
  {\ttfamily(}\meta{#1}{\ttfamily)}}
\raggedbottom

% -----------------------------------
\input{pgfplots.preamble.tex}

\makeatletter
% I want a two-column index just as in pgfmanual styles. This here
% was the best way to get one:
\def\index@prologue{\section*{Index}\addcontentsline{toc}{chapter}{Index}
}
\makeatother

%\RequirePackage[german,english,francais]{babel}

\def\matlabcolormaptext{This colormap is similar to one shipped with Matlab$^\text{\textregistered}$ under a similar name.}

\IfFileExists{tikzlibraryspy.code.tex}{%
\usetikzlibrary{spy}
}{%
    \message{ERROR: tikz SPY library NOT available. The manual will only compile partially.^^J}%
}%

\usepackage{xparse}% for colorbrewer manual

\usetikzlibrary{
    decorations.markings,
    decorations.footprints,
    shapes.arrows,
    matrix,
    positioning,
}

\usepgfplotslibrary{
    fillbetween,
    ternary,
    smithchart,
    patchplots,
    polar,
    colormaps,
    colorbrewer,
    colortol,           % docu not ready yet
}
\pgfqkeys{/codeexample}{%
    every codeexample/.append style={
        /pgfplots/every ternary axis/.append style={
            /pgfplots/legend style={fill=graphicbackground},
        }
    },
    tabsize=4,
}

\pgfplotsmanualenableexternalizationofexpensive

%\usetikzlibrary{external}
%\tikzexternalize[prefix=figures/]{pgfplots}

\title{%
    Manual for Package \PGFPlots{}\\
    {\small 2D/3D Plots in \LaTeX{}, Version \pgfplotsversion}\\
    {\small\url{https://github.com/pgf-tikz/pgfplots}}
    %\\{\small Attention: you are using an unstable development version.}
}

\makeatletter
\long\def\abstractsmuggle{%
    \centering
    \textbf{Abstract}\\[0.5cm]

    \begin{minipage}{12cm}
        \PGFPlots{} draws high-quality function plots in normal or logarithmic
        scaling with a user-friendly interface directly in \TeX{}. The user
        supplies axis labels, legend entries and the plot coordinates for one or
        more plots and \PGFPlots{} applies axis scaling, computes any logarithms
        and axis ticks and draws the plots. It supports line plots, scatter
        plots, piecewise constant plots, bar plots, area plots, mesh and surface
        plots, patch plots, contour plots, quiver plots, histogram plots, box
        plots, polar axes, ternary diagrams, smith charts and some more. It is
        based on Till Tantau's package \PGF{}/\Tikz{}.
    \end{minipage}

}%

\expandafter\date\expandafter{\@date\\[2cm]
    \abstractsmuggle
}%
\makeatother

%\includeonly{pgfplots.reference}


\begin{document}

\def\plotcoords{%
\addplot coordinates {
(5,8.312e-02)    (17,2.547e-02)   (49,7.407e-03)
(129,2.102e-03)  (321,5.874e-04)  (769,1.623e-04)
(1793,4.442e-05) (4097,1.207e-05) (9217,3.261e-06)
};

\addplot coordinates{
(7,8.472e-02)    (31,3.044e-02)    (111,1.022e-02)
(351,3.303e-03)  (1023,1.039e-03)  (2815,3.196e-04)
(7423,9.658e-05) (18943,2.873e-05) (47103,8.437e-06)
};

\addplot coordinates{
(9,7.881e-02)     (49,3.243e-02)    (209,1.232e-02)
(769,4.454e-03)   (2561,1.551e-03)  (7937,5.236e-04)
(23297,1.723e-04) (65537,5.545e-05) (178177,1.751e-05)
};

\addplot coordinates{
(11,6.887e-02)    (71,3.177e-02)     (351,1.341e-02)
(1471,5.334e-03)  (5503,2.027e-03)   (18943,7.415e-04)
(61183,2.628e-04) (187903,9.063e-05) (553983,3.053e-05)
};

\addplot coordinates{
(13,5.755e-02)     (97,2.925e-02)     (545,1.351e-02)
(2561,5.842e-03)   (10625,2.397e-03)  (40193,9.414e-04)
(141569,3.564e-04) (471041,1.308e-04)
(1496065,4.670e-05)
};
}%


\bookmaketitle
\tableofcontents
\include{pgfplots.title_abstract_intro}
\include{pgfplots.preliminaries}
\include{pgfplots.intro}
\include{pgfplots.reference}
\include{pgfplots.libs}
\include{pgfplots.resources}
\include{pgfplots.importexport}
\include{pgfplots.basic.reference}

\printindex

\bibliographystyle{abbrv} %gerapali} %gerabbrv} %gerunsrt.bst} %gerabbrv}% gerplain}
\nocite{pgfplotstable}
\nocite{programmingnotes}
\bibliography{pgfplots}
\end{document}

% -----------------------------------------------------------------------------
% For Stefan Pinnow as reminder on what to look for when editing the manual
% -----------------------------------------------------------------------------
% There should be no line breaks in the following environments
% - |...|
% - \declareandlabel{...}
% - \verbpdfref{...}
% If "MakeTikzPictures" isn't running through check one of the externalized
% LOG files what is the cause of that.
% -----------------------------------------------------------------------------


%%%%%%%%%%%%%%%%%%%%%%%%%%%%%%%%%%%%%%%%%%%%%%%%%%%%%%%%%%%%%%%%%%%%%%%%%%%%%
%
% Package pgfplots.sty documentation.
%
% Copyright 2007/2008 by Christian Feuersaenger.
%
% This program is free software: you can redistribute it and/or modify
% it under the terms of the GNU General Public License as published by
% the Free Software Foundation, either version 3 of the License, or
% (at your option) any later version.
%
% This program is distributed in the hope that it will be useful,
% but WITHOUT ANY WARRANTY; without even the implied warranty of
% MERCHANTABILITY or FITNESS FOR A PARTICULAR PURPOSE.  See the
% GNU General Public License for more details.
%
% You should have received a copy of the GNU General Public License
% along with this program.  If not, see <http://www.gnu.org/licenses/>.
%
%
%%%%%%%%%%%%%%%%%%%%%%%%%%%%%%%%%%%%%%%%%%%%%%%%%%%%%%%%%%%%%%%%%%%%%%%%%%%%%
% SEE pgfplots-macros.tex as well!
%\pdfminorversion=5 % to allow compression
%\pdfobjcompresslevel=2
\documentclass[a4paper,openany]{book}

\let\bookmaketitle=\maketitle

% -----------------------------------
% this here is from ltxdoc:
\usepackage{doc}[=v2]
\AtBeginDocument{\MakeShortVerb{\|}}
%\setlength{\textwidth}{355pt}
%\addtolength\marginparwidth{30pt}
%\addtolength\oddsidemargin{20pt}
%\addtolength\evensidemargin{20pt}
\def\cmd#1{\cs{\expandafter\cmd@to@cs\string#1}}
\def\cmd@to@cs#1#2{\char\number`#2\relax}
\DeclareRobustCommand\cs[1]{\texttt{\char`\\#1}}
\providecommand\marg[1]{%
  {\ttfamily\char`\{}\meta{#1}{\ttfamily\char`\}}}
\providecommand\oarg[1]{%
  {\ttfamily[}\meta{#1}{\ttfamily]}}
\providecommand\parg[1]{%
  {\ttfamily(}\meta{#1}{\ttfamily)}}
\raggedbottom

% -----------------------------------
\input{pgfplots.preamble.tex}

\makeatletter
% I want a two-column index just as in pgfmanual styles. This here
% was the best way to get one:
\def\index@prologue{\section*{Index}\addcontentsline{toc}{chapter}{Index}
}
\makeatother

%\RequirePackage[german,english,francais]{babel}

\def\matlabcolormaptext{This colormap is similar to one shipped with Matlab$^\text{\textregistered}$ under a similar name.}

\IfFileExists{tikzlibraryspy.code.tex}{%
\usetikzlibrary{spy}
}{%
    \message{ERROR: tikz SPY library NOT available. The manual will only compile partially.^^J}%
}%

\usepackage{xparse}% for colorbrewer manual

\usetikzlibrary{
    decorations.markings,
    decorations.footprints,
    shapes.arrows,
    matrix,
    positioning,
}

\usepgfplotslibrary{
    fillbetween,
    ternary,
    smithchart,
    patchplots,
    polar,
    colormaps,
    colorbrewer,
    colortol,           % docu not ready yet
}
\pgfqkeys{/codeexample}{%
    every codeexample/.append style={
        /pgfplots/every ternary axis/.append style={
            /pgfplots/legend style={fill=graphicbackground},
        }
    },
    tabsize=4,
}

\pgfplotsmanualenableexternalizationofexpensive

%\usetikzlibrary{external}
%\tikzexternalize[prefix=figures/]{pgfplots}

\title{%
    Manual for Package \PGFPlots{}\\
    {\small 2D/3D Plots in \LaTeX{}, Version \pgfplotsversion}\\
    {\small\url{https://github.com/pgf-tikz/pgfplots}}
    %\\{\small Attention: you are using an unstable development version.}
}

\makeatletter
\long\def\abstractsmuggle{%
    \centering
    \textbf{Abstract}\\[0.5cm]

    \begin{minipage}{12cm}
        \PGFPlots{} draws high-quality function plots in normal or logarithmic
        scaling with a user-friendly interface directly in \TeX{}. The user
        supplies axis labels, legend entries and the plot coordinates for one or
        more plots and \PGFPlots{} applies axis scaling, computes any logarithms
        and axis ticks and draws the plots. It supports line plots, scatter
        plots, piecewise constant plots, bar plots, area plots, mesh and surface
        plots, patch plots, contour plots, quiver plots, histogram plots, box
        plots, polar axes, ternary diagrams, smith charts and some more. It is
        based on Till Tantau's package \PGF{}/\Tikz{}.
    \end{minipage}

}%

\expandafter\date\expandafter{\@date\\[2cm]
    \abstractsmuggle
}%
\makeatother

%\includeonly{pgfplots.reference}


\begin{document}

\def\plotcoords{%
\addplot coordinates {
(5,8.312e-02)    (17,2.547e-02)   (49,7.407e-03)
(129,2.102e-03)  (321,5.874e-04)  (769,1.623e-04)
(1793,4.442e-05) (4097,1.207e-05) (9217,3.261e-06)
};

\addplot coordinates{
(7,8.472e-02)    (31,3.044e-02)    (111,1.022e-02)
(351,3.303e-03)  (1023,1.039e-03)  (2815,3.196e-04)
(7423,9.658e-05) (18943,2.873e-05) (47103,8.437e-06)
};

\addplot coordinates{
(9,7.881e-02)     (49,3.243e-02)    (209,1.232e-02)
(769,4.454e-03)   (2561,1.551e-03)  (7937,5.236e-04)
(23297,1.723e-04) (65537,5.545e-05) (178177,1.751e-05)
};

\addplot coordinates{
(11,6.887e-02)    (71,3.177e-02)     (351,1.341e-02)
(1471,5.334e-03)  (5503,2.027e-03)   (18943,7.415e-04)
(61183,2.628e-04) (187903,9.063e-05) (553983,3.053e-05)
};

\addplot coordinates{
(13,5.755e-02)     (97,2.925e-02)     (545,1.351e-02)
(2561,5.842e-03)   (10625,2.397e-03)  (40193,9.414e-04)
(141569,3.564e-04) (471041,1.308e-04)
(1496065,4.670e-05)
};
}%


\bookmaketitle
\tableofcontents
\include{pgfplots.title_abstract_intro}
\include{pgfplots.preliminaries}
\include{pgfplots.intro}
\include{pgfplots.reference}
\include{pgfplots.libs}
\include{pgfplots.resources}
\include{pgfplots.importexport}
\include{pgfplots.basic.reference}

\printindex

\bibliographystyle{abbrv} %gerapali} %gerabbrv} %gerunsrt.bst} %gerabbrv}% gerplain}
\nocite{pgfplotstable}
\nocite{programmingnotes}
\bibliography{pgfplots}
\end{document}

% -----------------------------------------------------------------------------
% For Stefan Pinnow as reminder on what to look for when editing the manual
% -----------------------------------------------------------------------------
% There should be no line breaks in the following environments
% - |...|
% - \declareandlabel{...}
% - \verbpdfref{...}
% If "MakeTikzPictures" isn't running through check one of the externalized
% LOG files what is the cause of that.
% -----------------------------------------------------------------------------


%%%%%%%%%%%%%%%%%%%%%%%%%%%%%%%%%%%%%%%%%%%%%%%%%%%%%%%%%%%%%%%%%%%%%%%%%%%%%
%
% Package pgfplots.sty documentation.
%
% Copyright 2007/2008 by Christian Feuersaenger.
%
% This program is free software: you can redistribute it and/or modify
% it under the terms of the GNU General Public License as published by
% the Free Software Foundation, either version 3 of the License, or
% (at your option) any later version.
%
% This program is distributed in the hope that it will be useful,
% but WITHOUT ANY WARRANTY; without even the implied warranty of
% MERCHANTABILITY or FITNESS FOR A PARTICULAR PURPOSE.  See the
% GNU General Public License for more details.
%
% You should have received a copy of the GNU General Public License
% along with this program.  If not, see <http://www.gnu.org/licenses/>.
%
%
%%%%%%%%%%%%%%%%%%%%%%%%%%%%%%%%%%%%%%%%%%%%%%%%%%%%%%%%%%%%%%%%%%%%%%%%%%%%%
% SEE pgfplots-macros.tex as well!
%\pdfminorversion=5 % to allow compression
%\pdfobjcompresslevel=2
\documentclass[a4paper,openany]{book}

\let\bookmaketitle=\maketitle

% -----------------------------------
% this here is from ltxdoc:
\usepackage{doc}[=v2]
\AtBeginDocument{\MakeShortVerb{\|}}
%\setlength{\textwidth}{355pt}
%\addtolength\marginparwidth{30pt}
%\addtolength\oddsidemargin{20pt}
%\addtolength\evensidemargin{20pt}
\def\cmd#1{\cs{\expandafter\cmd@to@cs\string#1}}
\def\cmd@to@cs#1#2{\char\number`#2\relax}
\DeclareRobustCommand\cs[1]{\texttt{\char`\\#1}}
\providecommand\marg[1]{%
  {\ttfamily\char`\{}\meta{#1}{\ttfamily\char`\}}}
\providecommand\oarg[1]{%
  {\ttfamily[}\meta{#1}{\ttfamily]}}
\providecommand\parg[1]{%
  {\ttfamily(}\meta{#1}{\ttfamily)}}
\raggedbottom

% -----------------------------------
\input{pgfplots.preamble.tex}

\makeatletter
% I want a two-column index just as in pgfmanual styles. This here
% was the best way to get one:
\def\index@prologue{\section*{Index}\addcontentsline{toc}{chapter}{Index}
}
\makeatother

%\RequirePackage[german,english,francais]{babel}

\def\matlabcolormaptext{This colormap is similar to one shipped with Matlab$^\text{\textregistered}$ under a similar name.}

\IfFileExists{tikzlibraryspy.code.tex}{%
\usetikzlibrary{spy}
}{%
    \message{ERROR: tikz SPY library NOT available. The manual will only compile partially.^^J}%
}%

\usepackage{xparse}% for colorbrewer manual

\usetikzlibrary{
    decorations.markings,
    decorations.footprints,
    shapes.arrows,
    matrix,
    positioning,
}

\usepgfplotslibrary{
    fillbetween,
    ternary,
    smithchart,
    patchplots,
    polar,
    colormaps,
    colorbrewer,
    colortol,           % docu not ready yet
}
\pgfqkeys{/codeexample}{%
    every codeexample/.append style={
        /pgfplots/every ternary axis/.append style={
            /pgfplots/legend style={fill=graphicbackground},
        }
    },
    tabsize=4,
}

\pgfplotsmanualenableexternalizationofexpensive

%\usetikzlibrary{external}
%\tikzexternalize[prefix=figures/]{pgfplots}

\title{%
    Manual for Package \PGFPlots{}\\
    {\small 2D/3D Plots in \LaTeX{}, Version \pgfplotsversion}\\
    {\small\url{https://github.com/pgf-tikz/pgfplots}}
    %\\{\small Attention: you are using an unstable development version.}
}

\makeatletter
\long\def\abstractsmuggle{%
    \centering
    \textbf{Abstract}\\[0.5cm]

    \begin{minipage}{12cm}
        \PGFPlots{} draws high-quality function plots in normal or logarithmic
        scaling with a user-friendly interface directly in \TeX{}. The user
        supplies axis labels, legend entries and the plot coordinates for one or
        more plots and \PGFPlots{} applies axis scaling, computes any logarithms
        and axis ticks and draws the plots. It supports line plots, scatter
        plots, piecewise constant plots, bar plots, area plots, mesh and surface
        plots, patch plots, contour plots, quiver plots, histogram plots, box
        plots, polar axes, ternary diagrams, smith charts and some more. It is
        based on Till Tantau's package \PGF{}/\Tikz{}.
    \end{minipage}

}%

\expandafter\date\expandafter{\@date\\[2cm]
    \abstractsmuggle
}%
\makeatother

%\includeonly{pgfplots.reference}


\begin{document}

\def\plotcoords{%
\addplot coordinates {
(5,8.312e-02)    (17,2.547e-02)   (49,7.407e-03)
(129,2.102e-03)  (321,5.874e-04)  (769,1.623e-04)
(1793,4.442e-05) (4097,1.207e-05) (9217,3.261e-06)
};

\addplot coordinates{
(7,8.472e-02)    (31,3.044e-02)    (111,1.022e-02)
(351,3.303e-03)  (1023,1.039e-03)  (2815,3.196e-04)
(7423,9.658e-05) (18943,2.873e-05) (47103,8.437e-06)
};

\addplot coordinates{
(9,7.881e-02)     (49,3.243e-02)    (209,1.232e-02)
(769,4.454e-03)   (2561,1.551e-03)  (7937,5.236e-04)
(23297,1.723e-04) (65537,5.545e-05) (178177,1.751e-05)
};

\addplot coordinates{
(11,6.887e-02)    (71,3.177e-02)     (351,1.341e-02)
(1471,5.334e-03)  (5503,2.027e-03)   (18943,7.415e-04)
(61183,2.628e-04) (187903,9.063e-05) (553983,3.053e-05)
};

\addplot coordinates{
(13,5.755e-02)     (97,2.925e-02)     (545,1.351e-02)
(2561,5.842e-03)   (10625,2.397e-03)  (40193,9.414e-04)
(141569,3.564e-04) (471041,1.308e-04)
(1496065,4.670e-05)
};
}%


\bookmaketitle
\tableofcontents
\include{pgfplots.title_abstract_intro}
\include{pgfplots.preliminaries}
\include{pgfplots.intro}
\include{pgfplots.reference}
\include{pgfplots.libs}
\include{pgfplots.resources}
\include{pgfplots.importexport}
\include{pgfplots.basic.reference}

\printindex

\bibliographystyle{abbrv} %gerapali} %gerabbrv} %gerunsrt.bst} %gerabbrv}% gerplain}
\nocite{pgfplotstable}
\nocite{programmingnotes}
\bibliography{pgfplots}
\end{document}

% -----------------------------------------------------------------------------
% For Stefan Pinnow as reminder on what to look for when editing the manual
% -----------------------------------------------------------------------------
% There should be no line breaks in the following environments
% - |...|
% - \declareandlabel{...}
% - \verbpdfref{...}
% If "MakeTikzPictures" isn't running through check one of the externalized
% LOG files what is the cause of that.
% -----------------------------------------------------------------------------


%%%%%%%%%%%%%%%%%%%%%%%%%%%%%%%%%%%%%%%%%%%%%%%%%%%%%%%%%%%%%%%%%%%%%%%%%%%%%
%
% Package pgfplots.sty documentation.
%
% Copyright 2007/2008 by Christian Feuersaenger.
%
% This program is free software: you can redistribute it and/or modify
% it under the terms of the GNU General Public License as published by
% the Free Software Foundation, either version 3 of the License, or
% (at your option) any later version.
%
% This program is distributed in the hope that it will be useful,
% but WITHOUT ANY WARRANTY; without even the implied warranty of
% MERCHANTABILITY or FITNESS FOR A PARTICULAR PURPOSE.  See the
% GNU General Public License for more details.
%
% You should have received a copy of the GNU General Public License
% along with this program.  If not, see <http://www.gnu.org/licenses/>.
%
%
%%%%%%%%%%%%%%%%%%%%%%%%%%%%%%%%%%%%%%%%%%%%%%%%%%%%%%%%%%%%%%%%%%%%%%%%%%%%%
% SEE pgfplots-macros.tex as well!
%\pdfminorversion=5 % to allow compression
%\pdfobjcompresslevel=2
\documentclass[a4paper,openany]{book}

\let\bookmaketitle=\maketitle

% -----------------------------------
% this here is from ltxdoc:
\usepackage{doc}[=v2]
\AtBeginDocument{\MakeShortVerb{\|}}
%\setlength{\textwidth}{355pt}
%\addtolength\marginparwidth{30pt}
%\addtolength\oddsidemargin{20pt}
%\addtolength\evensidemargin{20pt}
\def\cmd#1{\cs{\expandafter\cmd@to@cs\string#1}}
\def\cmd@to@cs#1#2{\char\number`#2\relax}
\DeclareRobustCommand\cs[1]{\texttt{\char`\\#1}}
\providecommand\marg[1]{%
  {\ttfamily\char`\{}\meta{#1}{\ttfamily\char`\}}}
\providecommand\oarg[1]{%
  {\ttfamily[}\meta{#1}{\ttfamily]}}
\providecommand\parg[1]{%
  {\ttfamily(}\meta{#1}{\ttfamily)}}
\raggedbottom

% -----------------------------------
\input{pgfplots.preamble.tex}

\makeatletter
% I want a two-column index just as in pgfmanual styles. This here
% was the best way to get one:
\def\index@prologue{\section*{Index}\addcontentsline{toc}{chapter}{Index}
}
\makeatother

%\RequirePackage[german,english,francais]{babel}

\def\matlabcolormaptext{This colormap is similar to one shipped with Matlab$^\text{\textregistered}$ under a similar name.}

\IfFileExists{tikzlibraryspy.code.tex}{%
\usetikzlibrary{spy}
}{%
    \message{ERROR: tikz SPY library NOT available. The manual will only compile partially.^^J}%
}%

\usepackage{xparse}% for colorbrewer manual

\usetikzlibrary{
    decorations.markings,
    decorations.footprints,
    shapes.arrows,
    matrix,
    positioning,
}

\usepgfplotslibrary{
    fillbetween,
    ternary,
    smithchart,
    patchplots,
    polar,
    colormaps,
    colorbrewer,
    colortol,           % docu not ready yet
}
\pgfqkeys{/codeexample}{%
    every codeexample/.append style={
        /pgfplots/every ternary axis/.append style={
            /pgfplots/legend style={fill=graphicbackground},
        }
    },
    tabsize=4,
}

\pgfplotsmanualenableexternalizationofexpensive

%\usetikzlibrary{external}
%\tikzexternalize[prefix=figures/]{pgfplots}

\title{%
    Manual for Package \PGFPlots{}\\
    {\small 2D/3D Plots in \LaTeX{}, Version \pgfplotsversion}\\
    {\small\url{https://github.com/pgf-tikz/pgfplots}}
    %\\{\small Attention: you are using an unstable development version.}
}

\makeatletter
\long\def\abstractsmuggle{%
    \centering
    \textbf{Abstract}\\[0.5cm]

    \begin{minipage}{12cm}
        \PGFPlots{} draws high-quality function plots in normal or logarithmic
        scaling with a user-friendly interface directly in \TeX{}. The user
        supplies axis labels, legend entries and the plot coordinates for one or
        more plots and \PGFPlots{} applies axis scaling, computes any logarithms
        and axis ticks and draws the plots. It supports line plots, scatter
        plots, piecewise constant plots, bar plots, area plots, mesh and surface
        plots, patch plots, contour plots, quiver plots, histogram plots, box
        plots, polar axes, ternary diagrams, smith charts and some more. It is
        based on Till Tantau's package \PGF{}/\Tikz{}.
    \end{minipage}

}%

\expandafter\date\expandafter{\@date\\[2cm]
    \abstractsmuggle
}%
\makeatother

%\includeonly{pgfplots.reference}


\begin{document}

\def\plotcoords{%
\addplot coordinates {
(5,8.312e-02)    (17,2.547e-02)   (49,7.407e-03)
(129,2.102e-03)  (321,5.874e-04)  (769,1.623e-04)
(1793,4.442e-05) (4097,1.207e-05) (9217,3.261e-06)
};

\addplot coordinates{
(7,8.472e-02)    (31,3.044e-02)    (111,1.022e-02)
(351,3.303e-03)  (1023,1.039e-03)  (2815,3.196e-04)
(7423,9.658e-05) (18943,2.873e-05) (47103,8.437e-06)
};

\addplot coordinates{
(9,7.881e-02)     (49,3.243e-02)    (209,1.232e-02)
(769,4.454e-03)   (2561,1.551e-03)  (7937,5.236e-04)
(23297,1.723e-04) (65537,5.545e-05) (178177,1.751e-05)
};

\addplot coordinates{
(11,6.887e-02)    (71,3.177e-02)     (351,1.341e-02)
(1471,5.334e-03)  (5503,2.027e-03)   (18943,7.415e-04)
(61183,2.628e-04) (187903,9.063e-05) (553983,3.053e-05)
};

\addplot coordinates{
(13,5.755e-02)     (97,2.925e-02)     (545,1.351e-02)
(2561,5.842e-03)   (10625,2.397e-03)  (40193,9.414e-04)
(141569,3.564e-04) (471041,1.308e-04)
(1496065,4.670e-05)
};
}%


\bookmaketitle
\tableofcontents
\include{pgfplots.title_abstract_intro}
\include{pgfplots.preliminaries}
\include{pgfplots.intro}
\include{pgfplots.reference}
\include{pgfplots.libs}
\include{pgfplots.resources}
\include{pgfplots.importexport}
\include{pgfplots.basic.reference}

\printindex

\bibliographystyle{abbrv} %gerapali} %gerabbrv} %gerunsrt.bst} %gerabbrv}% gerplain}
\nocite{pgfplotstable}
\nocite{programmingnotes}
\bibliography{pgfplots}
\end{document}

% -----------------------------------------------------------------------------
% For Stefan Pinnow as reminder on what to look for when editing the manual
% -----------------------------------------------------------------------------
% There should be no line breaks in the following environments
% - |...|
% - \declareandlabel{...}
% - \verbpdfref{...}
% If "MakeTikzPictures" isn't running through check one of the externalized
% LOG files what is the cause of that.
% -----------------------------------------------------------------------------


%%%%%%%%%%%%%%%%%%%%%%%%%%%%%%%%%%%%%%%%%%%%%%%%%%%%%%%%%%%%%%%%%%%%%%%%%%%%%
%
% Package pgfplots.sty documentation.
%
% Copyright 2007/2008 by Christian Feuersaenger.
%
% This program is free software: you can redistribute it and/or modify
% it under the terms of the GNU General Public License as published by
% the Free Software Foundation, either version 3 of the License, or
% (at your option) any later version.
%
% This program is distributed in the hope that it will be useful,
% but WITHOUT ANY WARRANTY; without even the implied warranty of
% MERCHANTABILITY or FITNESS FOR A PARTICULAR PURPOSE.  See the
% GNU General Public License for more details.
%
% You should have received a copy of the GNU General Public License
% along with this program.  If not, see <http://www.gnu.org/licenses/>.
%
%
%%%%%%%%%%%%%%%%%%%%%%%%%%%%%%%%%%%%%%%%%%%%%%%%%%%%%%%%%%%%%%%%%%%%%%%%%%%%%
% SEE pgfplots-macros.tex as well!
%\pdfminorversion=5 % to allow compression
%\pdfobjcompresslevel=2
\documentclass[a4paper,openany]{book}

\let\bookmaketitle=\maketitle

% -----------------------------------
% this here is from ltxdoc:
\usepackage{doc}[=v2]
\AtBeginDocument{\MakeShortVerb{\|}}
%\setlength{\textwidth}{355pt}
%\addtolength\marginparwidth{30pt}
%\addtolength\oddsidemargin{20pt}
%\addtolength\evensidemargin{20pt}
\def\cmd#1{\cs{\expandafter\cmd@to@cs\string#1}}
\def\cmd@to@cs#1#2{\char\number`#2\relax}
\DeclareRobustCommand\cs[1]{\texttt{\char`\\#1}}
\providecommand\marg[1]{%
  {\ttfamily\char`\{}\meta{#1}{\ttfamily\char`\}}}
\providecommand\oarg[1]{%
  {\ttfamily[}\meta{#1}{\ttfamily]}}
\providecommand\parg[1]{%
  {\ttfamily(}\meta{#1}{\ttfamily)}}
\raggedbottom

% -----------------------------------
\input{pgfplots.preamble.tex}

\makeatletter
% I want a two-column index just as in pgfmanual styles. This here
% was the best way to get one:
\def\index@prologue{\section*{Index}\addcontentsline{toc}{chapter}{Index}
}
\makeatother

%\RequirePackage[german,english,francais]{babel}

\def\matlabcolormaptext{This colormap is similar to one shipped with Matlab$^\text{\textregistered}$ under a similar name.}

\IfFileExists{tikzlibraryspy.code.tex}{%
\usetikzlibrary{spy}
}{%
    \message{ERROR: tikz SPY library NOT available. The manual will only compile partially.^^J}%
}%

\usepackage{xparse}% for colorbrewer manual

\usetikzlibrary{
    decorations.markings,
    decorations.footprints,
    shapes.arrows,
    matrix,
    positioning,
}

\usepgfplotslibrary{
    fillbetween,
    ternary,
    smithchart,
    patchplots,
    polar,
    colormaps,
    colorbrewer,
    colortol,           % docu not ready yet
}
\pgfqkeys{/codeexample}{%
    every codeexample/.append style={
        /pgfplots/every ternary axis/.append style={
            /pgfplots/legend style={fill=graphicbackground},
        }
    },
    tabsize=4,
}

\pgfplotsmanualenableexternalizationofexpensive

%\usetikzlibrary{external}
%\tikzexternalize[prefix=figures/]{pgfplots}

\title{%
    Manual for Package \PGFPlots{}\\
    {\small 2D/3D Plots in \LaTeX{}, Version \pgfplotsversion}\\
    {\small\url{https://github.com/pgf-tikz/pgfplots}}
    %\\{\small Attention: you are using an unstable development version.}
}

\makeatletter
\long\def\abstractsmuggle{%
    \centering
    \textbf{Abstract}\\[0.5cm]

    \begin{minipage}{12cm}
        \PGFPlots{} draws high-quality function plots in normal or logarithmic
        scaling with a user-friendly interface directly in \TeX{}. The user
        supplies axis labels, legend entries and the plot coordinates for one or
        more plots and \PGFPlots{} applies axis scaling, computes any logarithms
        and axis ticks and draws the plots. It supports line plots, scatter
        plots, piecewise constant plots, bar plots, area plots, mesh and surface
        plots, patch plots, contour plots, quiver plots, histogram plots, box
        plots, polar axes, ternary diagrams, smith charts and some more. It is
        based on Till Tantau's package \PGF{}/\Tikz{}.
    \end{minipage}

}%

\expandafter\date\expandafter{\@date\\[2cm]
    \abstractsmuggle
}%
\makeatother

%\includeonly{pgfplots.reference}


\begin{document}

\def\plotcoords{%
\addplot coordinates {
(5,8.312e-02)    (17,2.547e-02)   (49,7.407e-03)
(129,2.102e-03)  (321,5.874e-04)  (769,1.623e-04)
(1793,4.442e-05) (4097,1.207e-05) (9217,3.261e-06)
};

\addplot coordinates{
(7,8.472e-02)    (31,3.044e-02)    (111,1.022e-02)
(351,3.303e-03)  (1023,1.039e-03)  (2815,3.196e-04)
(7423,9.658e-05) (18943,2.873e-05) (47103,8.437e-06)
};

\addplot coordinates{
(9,7.881e-02)     (49,3.243e-02)    (209,1.232e-02)
(769,4.454e-03)   (2561,1.551e-03)  (7937,5.236e-04)
(23297,1.723e-04) (65537,5.545e-05) (178177,1.751e-05)
};

\addplot coordinates{
(11,6.887e-02)    (71,3.177e-02)     (351,1.341e-02)
(1471,5.334e-03)  (5503,2.027e-03)   (18943,7.415e-04)
(61183,2.628e-04) (187903,9.063e-05) (553983,3.053e-05)
};

\addplot coordinates{
(13,5.755e-02)     (97,2.925e-02)     (545,1.351e-02)
(2561,5.842e-03)   (10625,2.397e-03)  (40193,9.414e-04)
(141569,3.564e-04) (471041,1.308e-04)
(1496065,4.670e-05)
};
}%


\bookmaketitle
\tableofcontents
\include{pgfplots.title_abstract_intro}
\include{pgfplots.preliminaries}
\include{pgfplots.intro}
\include{pgfplots.reference}
\include{pgfplots.libs}
\include{pgfplots.resources}
\include{pgfplots.importexport}
\include{pgfplots.basic.reference}

\printindex

\bibliographystyle{abbrv} %gerapali} %gerabbrv} %gerunsrt.bst} %gerabbrv}% gerplain}
\nocite{pgfplotstable}
\nocite{programmingnotes}
\bibliography{pgfplots}
\end{document}

% -----------------------------------------------------------------------------
% For Stefan Pinnow as reminder on what to look for when editing the manual
% -----------------------------------------------------------------------------
% There should be no line breaks in the following environments
% - |...|
% - \declareandlabel{...}
% - \verbpdfref{...}
% If "MakeTikzPictures" isn't running through check one of the externalized
% LOG files what is the cause of that.
% -----------------------------------------------------------------------------

\include{pgfplots.basic.reference}

\printindex

\bibliographystyle{abbrv} %gerapali} %gerabbrv} %gerunsrt.bst} %gerabbrv}% gerplain}
\nocite{pgfplotstable}
\nocite{programmingnotes}
\bibliography{pgfplots}
\end{document}

% -----------------------------------------------------------------------------
% For Stefan Pinnow as reminder on what to look for when editing the manual
% -----------------------------------------------------------------------------
% There should be no line breaks in the following environments
% - |...|
% - \declareandlabel{...}
% - \verbpdfref{...}
% If "MakeTikzPictures" isn't running through check one of the externalized
% LOG files what is the cause of that.
% -----------------------------------------------------------------------------


%%%%%%%%%%%%%%%%%%%%%%%%%%%%%%%%%%%%%%%%%%%%%%%%%%%%%%%%%%%%%%%%%%%%%%%%%%%%%
%
% Package pgfplots.sty documentation.
%
% Copyright 2007/2008 by Christian Feuersaenger.
%
% This program is free software: you can redistribute it and/or modify
% it under the terms of the GNU General Public License as published by
% the Free Software Foundation, either version 3 of the License, or
% (at your option) any later version.
%
% This program is distributed in the hope that it will be useful,
% but WITHOUT ANY WARRANTY; without even the implied warranty of
% MERCHANTABILITY or FITNESS FOR A PARTICULAR PURPOSE.  See the
% GNU General Public License for more details.
%
% You should have received a copy of the GNU General Public License
% along with this program.  If not, see <http://www.gnu.org/licenses/>.
%
%
%%%%%%%%%%%%%%%%%%%%%%%%%%%%%%%%%%%%%%%%%%%%%%%%%%%%%%%%%%%%%%%%%%%%%%%%%%%%%
% SEE pgfplots-macros.tex as well!
%\pdfminorversion=5 % to allow compression
%\pdfobjcompresslevel=2
\documentclass[a4paper,openany]{book}

\let\bookmaketitle=\maketitle

% -----------------------------------
% this here is from ltxdoc:
\usepackage{doc}[=v2]
\AtBeginDocument{\MakeShortVerb{\|}}
%\setlength{\textwidth}{355pt}
%\addtolength\marginparwidth{30pt}
%\addtolength\oddsidemargin{20pt}
%\addtolength\evensidemargin{20pt}
\def\cmd#1{\cs{\expandafter\cmd@to@cs\string#1}}
\def\cmd@to@cs#1#2{\char\number`#2\relax}
\DeclareRobustCommand\cs[1]{\texttt{\char`\\#1}}
\providecommand\marg[1]{%
  {\ttfamily\char`\{}\meta{#1}{\ttfamily\char`\}}}
\providecommand\oarg[1]{%
  {\ttfamily[}\meta{#1}{\ttfamily]}}
\providecommand\parg[1]{%
  {\ttfamily(}\meta{#1}{\ttfamily)}}
\raggedbottom

% -----------------------------------
\input{pgfplots.preamble.tex}

\makeatletter
% I want a two-column index just as in pgfmanual styles. This here
% was the best way to get one:
\def\index@prologue{\section*{Index}\addcontentsline{toc}{chapter}{Index}
}
\makeatother

%\RequirePackage[german,english,francais]{babel}

\def\matlabcolormaptext{This colormap is similar to one shipped with Matlab$^\text{\textregistered}$ under a similar name.}

\IfFileExists{tikzlibraryspy.code.tex}{%
\usetikzlibrary{spy}
}{%
    \message{ERROR: tikz SPY library NOT available. The manual will only compile partially.^^J}%
}%

\usepackage{xparse}% for colorbrewer manual

\usetikzlibrary{
    decorations.markings,
    decorations.footprints,
    shapes.arrows,
    matrix,
    positioning,
}

\usepgfplotslibrary{
    fillbetween,
    ternary,
    smithchart,
    patchplots,
    polar,
    colormaps,
    colorbrewer,
    colortol,           % docu not ready yet
}
\pgfqkeys{/codeexample}{%
    every codeexample/.append style={
        /pgfplots/every ternary axis/.append style={
            /pgfplots/legend style={fill=graphicbackground},
        }
    },
    tabsize=4,
}

\pgfplotsmanualenableexternalizationofexpensive

%\usetikzlibrary{external}
%\tikzexternalize[prefix=figures/]{pgfplots}

\title{%
    Manual for Package \PGFPlots{}\\
    {\small 2D/3D Plots in \LaTeX{}, Version \pgfplotsversion}\\
    {\small\url{https://github.com/pgf-tikz/pgfplots}}
    %\\{\small Attention: you are using an unstable development version.}
}

\makeatletter
\long\def\abstractsmuggle{%
    \centering
    \textbf{Abstract}\\[0.5cm]

    \begin{minipage}{12cm}
        \PGFPlots{} draws high-quality function plots in normal or logarithmic
        scaling with a user-friendly interface directly in \TeX{}. The user
        supplies axis labels, legend entries and the plot coordinates for one or
        more plots and \PGFPlots{} applies axis scaling, computes any logarithms
        and axis ticks and draws the plots. It supports line plots, scatter
        plots, piecewise constant plots, bar plots, area plots, mesh and surface
        plots, patch plots, contour plots, quiver plots, histogram plots, box
        plots, polar axes, ternary diagrams, smith charts and some more. It is
        based on Till Tantau's package \PGF{}/\Tikz{}.
    \end{minipage}

}%

\expandafter\date\expandafter{\@date\\[2cm]
    \abstractsmuggle
}%
\makeatother

%\includeonly{pgfplots.reference}


\begin{document}

\def\plotcoords{%
\addplot coordinates {
(5,8.312e-02)    (17,2.547e-02)   (49,7.407e-03)
(129,2.102e-03)  (321,5.874e-04)  (769,1.623e-04)
(1793,4.442e-05) (4097,1.207e-05) (9217,3.261e-06)
};

\addplot coordinates{
(7,8.472e-02)    (31,3.044e-02)    (111,1.022e-02)
(351,3.303e-03)  (1023,1.039e-03)  (2815,3.196e-04)
(7423,9.658e-05) (18943,2.873e-05) (47103,8.437e-06)
};

\addplot coordinates{
(9,7.881e-02)     (49,3.243e-02)    (209,1.232e-02)
(769,4.454e-03)   (2561,1.551e-03)  (7937,5.236e-04)
(23297,1.723e-04) (65537,5.545e-05) (178177,1.751e-05)
};

\addplot coordinates{
(11,6.887e-02)    (71,3.177e-02)     (351,1.341e-02)
(1471,5.334e-03)  (5503,2.027e-03)   (18943,7.415e-04)
(61183,2.628e-04) (187903,9.063e-05) (553983,3.053e-05)
};

\addplot coordinates{
(13,5.755e-02)     (97,2.925e-02)     (545,1.351e-02)
(2561,5.842e-03)   (10625,2.397e-03)  (40193,9.414e-04)
(141569,3.564e-04) (471041,1.308e-04)
(1496065,4.670e-05)
};
}%


\bookmaketitle
\tableofcontents

%%%%%%%%%%%%%%%%%%%%%%%%%%%%%%%%%%%%%%%%%%%%%%%%%%%%%%%%%%%%%%%%%%%%%%%%%%%%%
%
% Package pgfplots.sty documentation.
%
% Copyright 2007/2008 by Christian Feuersaenger.
%
% This program is free software: you can redistribute it and/or modify
% it under the terms of the GNU General Public License as published by
% the Free Software Foundation, either version 3 of the License, or
% (at your option) any later version.
%
% This program is distributed in the hope that it will be useful,
% but WITHOUT ANY WARRANTY; without even the implied warranty of
% MERCHANTABILITY or FITNESS FOR A PARTICULAR PURPOSE.  See the
% GNU General Public License for more details.
%
% You should have received a copy of the GNU General Public License
% along with this program.  If not, see <http://www.gnu.org/licenses/>.
%
%
%%%%%%%%%%%%%%%%%%%%%%%%%%%%%%%%%%%%%%%%%%%%%%%%%%%%%%%%%%%%%%%%%%%%%%%%%%%%%
% SEE pgfplots-macros.tex as well!
%\pdfminorversion=5 % to allow compression
%\pdfobjcompresslevel=2
\documentclass[a4paper,openany]{book}

\let\bookmaketitle=\maketitle

% -----------------------------------
% this here is from ltxdoc:
\usepackage{doc}[=v2]
\AtBeginDocument{\MakeShortVerb{\|}}
%\setlength{\textwidth}{355pt}
%\addtolength\marginparwidth{30pt}
%\addtolength\oddsidemargin{20pt}
%\addtolength\evensidemargin{20pt}
\def\cmd#1{\cs{\expandafter\cmd@to@cs\string#1}}
\def\cmd@to@cs#1#2{\char\number`#2\relax}
\DeclareRobustCommand\cs[1]{\texttt{\char`\\#1}}
\providecommand\marg[1]{%
  {\ttfamily\char`\{}\meta{#1}{\ttfamily\char`\}}}
\providecommand\oarg[1]{%
  {\ttfamily[}\meta{#1}{\ttfamily]}}
\providecommand\parg[1]{%
  {\ttfamily(}\meta{#1}{\ttfamily)}}
\raggedbottom

% -----------------------------------
\input{pgfplots.preamble.tex}

\makeatletter
% I want a two-column index just as in pgfmanual styles. This here
% was the best way to get one:
\def\index@prologue{\section*{Index}\addcontentsline{toc}{chapter}{Index}
}
\makeatother

%\RequirePackage[german,english,francais]{babel}

\def\matlabcolormaptext{This colormap is similar to one shipped with Matlab$^\text{\textregistered}$ under a similar name.}

\IfFileExists{tikzlibraryspy.code.tex}{%
\usetikzlibrary{spy}
}{%
    \message{ERROR: tikz SPY library NOT available. The manual will only compile partially.^^J}%
}%

\usepackage{xparse}% for colorbrewer manual

\usetikzlibrary{
    decorations.markings,
    decorations.footprints,
    shapes.arrows,
    matrix,
    positioning,
}

\usepgfplotslibrary{
    fillbetween,
    ternary,
    smithchart,
    patchplots,
    polar,
    colormaps,
    colorbrewer,
    colortol,           % docu not ready yet
}
\pgfqkeys{/codeexample}{%
    every codeexample/.append style={
        /pgfplots/every ternary axis/.append style={
            /pgfplots/legend style={fill=graphicbackground},
        }
    },
    tabsize=4,
}

\pgfplotsmanualenableexternalizationofexpensive

%\usetikzlibrary{external}
%\tikzexternalize[prefix=figures/]{pgfplots}

\title{%
    Manual for Package \PGFPlots{}\\
    {\small 2D/3D Plots in \LaTeX{}, Version \pgfplotsversion}\\
    {\small\url{https://github.com/pgf-tikz/pgfplots}}
    %\\{\small Attention: you are using an unstable development version.}
}

\makeatletter
\long\def\abstractsmuggle{%
    \centering
    \textbf{Abstract}\\[0.5cm]

    \begin{minipage}{12cm}
        \PGFPlots{} draws high-quality function plots in normal or logarithmic
        scaling with a user-friendly interface directly in \TeX{}. The user
        supplies axis labels, legend entries and the plot coordinates for one or
        more plots and \PGFPlots{} applies axis scaling, computes any logarithms
        and axis ticks and draws the plots. It supports line plots, scatter
        plots, piecewise constant plots, bar plots, area plots, mesh and surface
        plots, patch plots, contour plots, quiver plots, histogram plots, box
        plots, polar axes, ternary diagrams, smith charts and some more. It is
        based on Till Tantau's package \PGF{}/\Tikz{}.
    \end{minipage}

}%

\expandafter\date\expandafter{\@date\\[2cm]
    \abstractsmuggle
}%
\makeatother

%\includeonly{pgfplots.reference}


\begin{document}

\def\plotcoords{%
\addplot coordinates {
(5,8.312e-02)    (17,2.547e-02)   (49,7.407e-03)
(129,2.102e-03)  (321,5.874e-04)  (769,1.623e-04)
(1793,4.442e-05) (4097,1.207e-05) (9217,3.261e-06)
};

\addplot coordinates{
(7,8.472e-02)    (31,3.044e-02)    (111,1.022e-02)
(351,3.303e-03)  (1023,1.039e-03)  (2815,3.196e-04)
(7423,9.658e-05) (18943,2.873e-05) (47103,8.437e-06)
};

\addplot coordinates{
(9,7.881e-02)     (49,3.243e-02)    (209,1.232e-02)
(769,4.454e-03)   (2561,1.551e-03)  (7937,5.236e-04)
(23297,1.723e-04) (65537,5.545e-05) (178177,1.751e-05)
};

\addplot coordinates{
(11,6.887e-02)    (71,3.177e-02)     (351,1.341e-02)
(1471,5.334e-03)  (5503,2.027e-03)   (18943,7.415e-04)
(61183,2.628e-04) (187903,9.063e-05) (553983,3.053e-05)
};

\addplot coordinates{
(13,5.755e-02)     (97,2.925e-02)     (545,1.351e-02)
(2561,5.842e-03)   (10625,2.397e-03)  (40193,9.414e-04)
(141569,3.564e-04) (471041,1.308e-04)
(1496065,4.670e-05)
};
}%


\bookmaketitle
\tableofcontents
\include{pgfplots.title_abstract_intro}
\include{pgfplots.preliminaries}
\include{pgfplots.intro}
\include{pgfplots.reference}
\include{pgfplots.libs}
\include{pgfplots.resources}
\include{pgfplots.importexport}
\include{pgfplots.basic.reference}

\printindex

\bibliographystyle{abbrv} %gerapali} %gerabbrv} %gerunsrt.bst} %gerabbrv}% gerplain}
\nocite{pgfplotstable}
\nocite{programmingnotes}
\bibliography{pgfplots}
\end{document}

% -----------------------------------------------------------------------------
% For Stefan Pinnow as reminder on what to look for when editing the manual
% -----------------------------------------------------------------------------
% There should be no line breaks in the following environments
% - |...|
% - \declareandlabel{...}
% - \verbpdfref{...}
% If "MakeTikzPictures" isn't running through check one of the externalized
% LOG files what is the cause of that.
% -----------------------------------------------------------------------------


%%%%%%%%%%%%%%%%%%%%%%%%%%%%%%%%%%%%%%%%%%%%%%%%%%%%%%%%%%%%%%%%%%%%%%%%%%%%%
%
% Package pgfplots.sty documentation.
%
% Copyright 2007/2008 by Christian Feuersaenger.
%
% This program is free software: you can redistribute it and/or modify
% it under the terms of the GNU General Public License as published by
% the Free Software Foundation, either version 3 of the License, or
% (at your option) any later version.
%
% This program is distributed in the hope that it will be useful,
% but WITHOUT ANY WARRANTY; without even the implied warranty of
% MERCHANTABILITY or FITNESS FOR A PARTICULAR PURPOSE.  See the
% GNU General Public License for more details.
%
% You should have received a copy of the GNU General Public License
% along with this program.  If not, see <http://www.gnu.org/licenses/>.
%
%
%%%%%%%%%%%%%%%%%%%%%%%%%%%%%%%%%%%%%%%%%%%%%%%%%%%%%%%%%%%%%%%%%%%%%%%%%%%%%
% SEE pgfplots-macros.tex as well!
%\pdfminorversion=5 % to allow compression
%\pdfobjcompresslevel=2
\documentclass[a4paper,openany]{book}

\let\bookmaketitle=\maketitle

% -----------------------------------
% this here is from ltxdoc:
\usepackage{doc}[=v2]
\AtBeginDocument{\MakeShortVerb{\|}}
%\setlength{\textwidth}{355pt}
%\addtolength\marginparwidth{30pt}
%\addtolength\oddsidemargin{20pt}
%\addtolength\evensidemargin{20pt}
\def\cmd#1{\cs{\expandafter\cmd@to@cs\string#1}}
\def\cmd@to@cs#1#2{\char\number`#2\relax}
\DeclareRobustCommand\cs[1]{\texttt{\char`\\#1}}
\providecommand\marg[1]{%
  {\ttfamily\char`\{}\meta{#1}{\ttfamily\char`\}}}
\providecommand\oarg[1]{%
  {\ttfamily[}\meta{#1}{\ttfamily]}}
\providecommand\parg[1]{%
  {\ttfamily(}\meta{#1}{\ttfamily)}}
\raggedbottom

% -----------------------------------
\input{pgfplots.preamble.tex}

\makeatletter
% I want a two-column index just as in pgfmanual styles. This here
% was the best way to get one:
\def\index@prologue{\section*{Index}\addcontentsline{toc}{chapter}{Index}
}
\makeatother

%\RequirePackage[german,english,francais]{babel}

\def\matlabcolormaptext{This colormap is similar to one shipped with Matlab$^\text{\textregistered}$ under a similar name.}

\IfFileExists{tikzlibraryspy.code.tex}{%
\usetikzlibrary{spy}
}{%
    \message{ERROR: tikz SPY library NOT available. The manual will only compile partially.^^J}%
}%

\usepackage{xparse}% for colorbrewer manual

\usetikzlibrary{
    decorations.markings,
    decorations.footprints,
    shapes.arrows,
    matrix,
    positioning,
}

\usepgfplotslibrary{
    fillbetween,
    ternary,
    smithchart,
    patchplots,
    polar,
    colormaps,
    colorbrewer,
    colortol,           % docu not ready yet
}
\pgfqkeys{/codeexample}{%
    every codeexample/.append style={
        /pgfplots/every ternary axis/.append style={
            /pgfplots/legend style={fill=graphicbackground},
        }
    },
    tabsize=4,
}

\pgfplotsmanualenableexternalizationofexpensive

%\usetikzlibrary{external}
%\tikzexternalize[prefix=figures/]{pgfplots}

\title{%
    Manual for Package \PGFPlots{}\\
    {\small 2D/3D Plots in \LaTeX{}, Version \pgfplotsversion}\\
    {\small\url{https://github.com/pgf-tikz/pgfplots}}
    %\\{\small Attention: you are using an unstable development version.}
}

\makeatletter
\long\def\abstractsmuggle{%
    \centering
    \textbf{Abstract}\\[0.5cm]

    \begin{minipage}{12cm}
        \PGFPlots{} draws high-quality function plots in normal or logarithmic
        scaling with a user-friendly interface directly in \TeX{}. The user
        supplies axis labels, legend entries and the plot coordinates for one or
        more plots and \PGFPlots{} applies axis scaling, computes any logarithms
        and axis ticks and draws the plots. It supports line plots, scatter
        plots, piecewise constant plots, bar plots, area plots, mesh and surface
        plots, patch plots, contour plots, quiver plots, histogram plots, box
        plots, polar axes, ternary diagrams, smith charts and some more. It is
        based on Till Tantau's package \PGF{}/\Tikz{}.
    \end{minipage}

}%

\expandafter\date\expandafter{\@date\\[2cm]
    \abstractsmuggle
}%
\makeatother

%\includeonly{pgfplots.reference}


\begin{document}

\def\plotcoords{%
\addplot coordinates {
(5,8.312e-02)    (17,2.547e-02)   (49,7.407e-03)
(129,2.102e-03)  (321,5.874e-04)  (769,1.623e-04)
(1793,4.442e-05) (4097,1.207e-05) (9217,3.261e-06)
};

\addplot coordinates{
(7,8.472e-02)    (31,3.044e-02)    (111,1.022e-02)
(351,3.303e-03)  (1023,1.039e-03)  (2815,3.196e-04)
(7423,9.658e-05) (18943,2.873e-05) (47103,8.437e-06)
};

\addplot coordinates{
(9,7.881e-02)     (49,3.243e-02)    (209,1.232e-02)
(769,4.454e-03)   (2561,1.551e-03)  (7937,5.236e-04)
(23297,1.723e-04) (65537,5.545e-05) (178177,1.751e-05)
};

\addplot coordinates{
(11,6.887e-02)    (71,3.177e-02)     (351,1.341e-02)
(1471,5.334e-03)  (5503,2.027e-03)   (18943,7.415e-04)
(61183,2.628e-04) (187903,9.063e-05) (553983,3.053e-05)
};

\addplot coordinates{
(13,5.755e-02)     (97,2.925e-02)     (545,1.351e-02)
(2561,5.842e-03)   (10625,2.397e-03)  (40193,9.414e-04)
(141569,3.564e-04) (471041,1.308e-04)
(1496065,4.670e-05)
};
}%


\bookmaketitle
\tableofcontents
\include{pgfplots.title_abstract_intro}
\include{pgfplots.preliminaries}
\include{pgfplots.intro}
\include{pgfplots.reference}
\include{pgfplots.libs}
\include{pgfplots.resources}
\include{pgfplots.importexport}
\include{pgfplots.basic.reference}

\printindex

\bibliographystyle{abbrv} %gerapali} %gerabbrv} %gerunsrt.bst} %gerabbrv}% gerplain}
\nocite{pgfplotstable}
\nocite{programmingnotes}
\bibliography{pgfplots}
\end{document}

% -----------------------------------------------------------------------------
% For Stefan Pinnow as reminder on what to look for when editing the manual
% -----------------------------------------------------------------------------
% There should be no line breaks in the following environments
% - |...|
% - \declareandlabel{...}
% - \verbpdfref{...}
% If "MakeTikzPictures" isn't running through check one of the externalized
% LOG files what is the cause of that.
% -----------------------------------------------------------------------------


%%%%%%%%%%%%%%%%%%%%%%%%%%%%%%%%%%%%%%%%%%%%%%%%%%%%%%%%%%%%%%%%%%%%%%%%%%%%%
%
% Package pgfplots.sty documentation.
%
% Copyright 2007/2008 by Christian Feuersaenger.
%
% This program is free software: you can redistribute it and/or modify
% it under the terms of the GNU General Public License as published by
% the Free Software Foundation, either version 3 of the License, or
% (at your option) any later version.
%
% This program is distributed in the hope that it will be useful,
% but WITHOUT ANY WARRANTY; without even the implied warranty of
% MERCHANTABILITY or FITNESS FOR A PARTICULAR PURPOSE.  See the
% GNU General Public License for more details.
%
% You should have received a copy of the GNU General Public License
% along with this program.  If not, see <http://www.gnu.org/licenses/>.
%
%
%%%%%%%%%%%%%%%%%%%%%%%%%%%%%%%%%%%%%%%%%%%%%%%%%%%%%%%%%%%%%%%%%%%%%%%%%%%%%
% SEE pgfplots-macros.tex as well!
%\pdfminorversion=5 % to allow compression
%\pdfobjcompresslevel=2
\documentclass[a4paper,openany]{book}

\let\bookmaketitle=\maketitle

% -----------------------------------
% this here is from ltxdoc:
\usepackage{doc}[=v2]
\AtBeginDocument{\MakeShortVerb{\|}}
%\setlength{\textwidth}{355pt}
%\addtolength\marginparwidth{30pt}
%\addtolength\oddsidemargin{20pt}
%\addtolength\evensidemargin{20pt}
\def\cmd#1{\cs{\expandafter\cmd@to@cs\string#1}}
\def\cmd@to@cs#1#2{\char\number`#2\relax}
\DeclareRobustCommand\cs[1]{\texttt{\char`\\#1}}
\providecommand\marg[1]{%
  {\ttfamily\char`\{}\meta{#1}{\ttfamily\char`\}}}
\providecommand\oarg[1]{%
  {\ttfamily[}\meta{#1}{\ttfamily]}}
\providecommand\parg[1]{%
  {\ttfamily(}\meta{#1}{\ttfamily)}}
\raggedbottom

% -----------------------------------
\input{pgfplots.preamble.tex}

\makeatletter
% I want a two-column index just as in pgfmanual styles. This here
% was the best way to get one:
\def\index@prologue{\section*{Index}\addcontentsline{toc}{chapter}{Index}
}
\makeatother

%\RequirePackage[german,english,francais]{babel}

\def\matlabcolormaptext{This colormap is similar to one shipped with Matlab$^\text{\textregistered}$ under a similar name.}

\IfFileExists{tikzlibraryspy.code.tex}{%
\usetikzlibrary{spy}
}{%
    \message{ERROR: tikz SPY library NOT available. The manual will only compile partially.^^J}%
}%

\usepackage{xparse}% for colorbrewer manual

\usetikzlibrary{
    decorations.markings,
    decorations.footprints,
    shapes.arrows,
    matrix,
    positioning,
}

\usepgfplotslibrary{
    fillbetween,
    ternary,
    smithchart,
    patchplots,
    polar,
    colormaps,
    colorbrewer,
    colortol,           % docu not ready yet
}
\pgfqkeys{/codeexample}{%
    every codeexample/.append style={
        /pgfplots/every ternary axis/.append style={
            /pgfplots/legend style={fill=graphicbackground},
        }
    },
    tabsize=4,
}

\pgfplotsmanualenableexternalizationofexpensive

%\usetikzlibrary{external}
%\tikzexternalize[prefix=figures/]{pgfplots}

\title{%
    Manual for Package \PGFPlots{}\\
    {\small 2D/3D Plots in \LaTeX{}, Version \pgfplotsversion}\\
    {\small\url{https://github.com/pgf-tikz/pgfplots}}
    %\\{\small Attention: you are using an unstable development version.}
}

\makeatletter
\long\def\abstractsmuggle{%
    \centering
    \textbf{Abstract}\\[0.5cm]

    \begin{minipage}{12cm}
        \PGFPlots{} draws high-quality function plots in normal or logarithmic
        scaling with a user-friendly interface directly in \TeX{}. The user
        supplies axis labels, legend entries and the plot coordinates for one or
        more plots and \PGFPlots{} applies axis scaling, computes any logarithms
        and axis ticks and draws the plots. It supports line plots, scatter
        plots, piecewise constant plots, bar plots, area plots, mesh and surface
        plots, patch plots, contour plots, quiver plots, histogram plots, box
        plots, polar axes, ternary diagrams, smith charts and some more. It is
        based on Till Tantau's package \PGF{}/\Tikz{}.
    \end{minipage}

}%

\expandafter\date\expandafter{\@date\\[2cm]
    \abstractsmuggle
}%
\makeatother

%\includeonly{pgfplots.reference}


\begin{document}

\def\plotcoords{%
\addplot coordinates {
(5,8.312e-02)    (17,2.547e-02)   (49,7.407e-03)
(129,2.102e-03)  (321,5.874e-04)  (769,1.623e-04)
(1793,4.442e-05) (4097,1.207e-05) (9217,3.261e-06)
};

\addplot coordinates{
(7,8.472e-02)    (31,3.044e-02)    (111,1.022e-02)
(351,3.303e-03)  (1023,1.039e-03)  (2815,3.196e-04)
(7423,9.658e-05) (18943,2.873e-05) (47103,8.437e-06)
};

\addplot coordinates{
(9,7.881e-02)     (49,3.243e-02)    (209,1.232e-02)
(769,4.454e-03)   (2561,1.551e-03)  (7937,5.236e-04)
(23297,1.723e-04) (65537,5.545e-05) (178177,1.751e-05)
};

\addplot coordinates{
(11,6.887e-02)    (71,3.177e-02)     (351,1.341e-02)
(1471,5.334e-03)  (5503,2.027e-03)   (18943,7.415e-04)
(61183,2.628e-04) (187903,9.063e-05) (553983,3.053e-05)
};

\addplot coordinates{
(13,5.755e-02)     (97,2.925e-02)     (545,1.351e-02)
(2561,5.842e-03)   (10625,2.397e-03)  (40193,9.414e-04)
(141569,3.564e-04) (471041,1.308e-04)
(1496065,4.670e-05)
};
}%


\bookmaketitle
\tableofcontents
\include{pgfplots.title_abstract_intro}
\include{pgfplots.preliminaries}
\include{pgfplots.intro}
\include{pgfplots.reference}
\include{pgfplots.libs}
\include{pgfplots.resources}
\include{pgfplots.importexport}
\include{pgfplots.basic.reference}

\printindex

\bibliographystyle{abbrv} %gerapali} %gerabbrv} %gerunsrt.bst} %gerabbrv}% gerplain}
\nocite{pgfplotstable}
\nocite{programmingnotes}
\bibliography{pgfplots}
\end{document}

% -----------------------------------------------------------------------------
% For Stefan Pinnow as reminder on what to look for when editing the manual
% -----------------------------------------------------------------------------
% There should be no line breaks in the following environments
% - |...|
% - \declareandlabel{...}
% - \verbpdfref{...}
% If "MakeTikzPictures" isn't running through check one of the externalized
% LOG files what is the cause of that.
% -----------------------------------------------------------------------------


%%%%%%%%%%%%%%%%%%%%%%%%%%%%%%%%%%%%%%%%%%%%%%%%%%%%%%%%%%%%%%%%%%%%%%%%%%%%%
%
% Package pgfplots.sty documentation.
%
% Copyright 2007/2008 by Christian Feuersaenger.
%
% This program is free software: you can redistribute it and/or modify
% it under the terms of the GNU General Public License as published by
% the Free Software Foundation, either version 3 of the License, or
% (at your option) any later version.
%
% This program is distributed in the hope that it will be useful,
% but WITHOUT ANY WARRANTY; without even the implied warranty of
% MERCHANTABILITY or FITNESS FOR A PARTICULAR PURPOSE.  See the
% GNU General Public License for more details.
%
% You should have received a copy of the GNU General Public License
% along with this program.  If not, see <http://www.gnu.org/licenses/>.
%
%
%%%%%%%%%%%%%%%%%%%%%%%%%%%%%%%%%%%%%%%%%%%%%%%%%%%%%%%%%%%%%%%%%%%%%%%%%%%%%
% SEE pgfplots-macros.tex as well!
%\pdfminorversion=5 % to allow compression
%\pdfobjcompresslevel=2
\documentclass[a4paper,openany]{book}

\let\bookmaketitle=\maketitle

% -----------------------------------
% this here is from ltxdoc:
\usepackage{doc}[=v2]
\AtBeginDocument{\MakeShortVerb{\|}}
%\setlength{\textwidth}{355pt}
%\addtolength\marginparwidth{30pt}
%\addtolength\oddsidemargin{20pt}
%\addtolength\evensidemargin{20pt}
\def\cmd#1{\cs{\expandafter\cmd@to@cs\string#1}}
\def\cmd@to@cs#1#2{\char\number`#2\relax}
\DeclareRobustCommand\cs[1]{\texttt{\char`\\#1}}
\providecommand\marg[1]{%
  {\ttfamily\char`\{}\meta{#1}{\ttfamily\char`\}}}
\providecommand\oarg[1]{%
  {\ttfamily[}\meta{#1}{\ttfamily]}}
\providecommand\parg[1]{%
  {\ttfamily(}\meta{#1}{\ttfamily)}}
\raggedbottom

% -----------------------------------
\input{pgfplots.preamble.tex}

\makeatletter
% I want a two-column index just as in pgfmanual styles. This here
% was the best way to get one:
\def\index@prologue{\section*{Index}\addcontentsline{toc}{chapter}{Index}
}
\makeatother

%\RequirePackage[german,english,francais]{babel}

\def\matlabcolormaptext{This colormap is similar to one shipped with Matlab$^\text{\textregistered}$ under a similar name.}

\IfFileExists{tikzlibraryspy.code.tex}{%
\usetikzlibrary{spy}
}{%
    \message{ERROR: tikz SPY library NOT available. The manual will only compile partially.^^J}%
}%

\usepackage{xparse}% for colorbrewer manual

\usetikzlibrary{
    decorations.markings,
    decorations.footprints,
    shapes.arrows,
    matrix,
    positioning,
}

\usepgfplotslibrary{
    fillbetween,
    ternary,
    smithchart,
    patchplots,
    polar,
    colormaps,
    colorbrewer,
    colortol,           % docu not ready yet
}
\pgfqkeys{/codeexample}{%
    every codeexample/.append style={
        /pgfplots/every ternary axis/.append style={
            /pgfplots/legend style={fill=graphicbackground},
        }
    },
    tabsize=4,
}

\pgfplotsmanualenableexternalizationofexpensive

%\usetikzlibrary{external}
%\tikzexternalize[prefix=figures/]{pgfplots}

\title{%
    Manual for Package \PGFPlots{}\\
    {\small 2D/3D Plots in \LaTeX{}, Version \pgfplotsversion}\\
    {\small\url{https://github.com/pgf-tikz/pgfplots}}
    %\\{\small Attention: you are using an unstable development version.}
}

\makeatletter
\long\def\abstractsmuggle{%
    \centering
    \textbf{Abstract}\\[0.5cm]

    \begin{minipage}{12cm}
        \PGFPlots{} draws high-quality function plots in normal or logarithmic
        scaling with a user-friendly interface directly in \TeX{}. The user
        supplies axis labels, legend entries and the plot coordinates for one or
        more plots and \PGFPlots{} applies axis scaling, computes any logarithms
        and axis ticks and draws the plots. It supports line plots, scatter
        plots, piecewise constant plots, bar plots, area plots, mesh and surface
        plots, patch plots, contour plots, quiver plots, histogram plots, box
        plots, polar axes, ternary diagrams, smith charts and some more. It is
        based on Till Tantau's package \PGF{}/\Tikz{}.
    \end{minipage}

}%

\expandafter\date\expandafter{\@date\\[2cm]
    \abstractsmuggle
}%
\makeatother

%\includeonly{pgfplots.reference}


\begin{document}

\def\plotcoords{%
\addplot coordinates {
(5,8.312e-02)    (17,2.547e-02)   (49,7.407e-03)
(129,2.102e-03)  (321,5.874e-04)  (769,1.623e-04)
(1793,4.442e-05) (4097,1.207e-05) (9217,3.261e-06)
};

\addplot coordinates{
(7,8.472e-02)    (31,3.044e-02)    (111,1.022e-02)
(351,3.303e-03)  (1023,1.039e-03)  (2815,3.196e-04)
(7423,9.658e-05) (18943,2.873e-05) (47103,8.437e-06)
};

\addplot coordinates{
(9,7.881e-02)     (49,3.243e-02)    (209,1.232e-02)
(769,4.454e-03)   (2561,1.551e-03)  (7937,5.236e-04)
(23297,1.723e-04) (65537,5.545e-05) (178177,1.751e-05)
};

\addplot coordinates{
(11,6.887e-02)    (71,3.177e-02)     (351,1.341e-02)
(1471,5.334e-03)  (5503,2.027e-03)   (18943,7.415e-04)
(61183,2.628e-04) (187903,9.063e-05) (553983,3.053e-05)
};

\addplot coordinates{
(13,5.755e-02)     (97,2.925e-02)     (545,1.351e-02)
(2561,5.842e-03)   (10625,2.397e-03)  (40193,9.414e-04)
(141569,3.564e-04) (471041,1.308e-04)
(1496065,4.670e-05)
};
}%


\bookmaketitle
\tableofcontents
\include{pgfplots.title_abstract_intro}
\include{pgfplots.preliminaries}
\include{pgfplots.intro}
\include{pgfplots.reference}
\include{pgfplots.libs}
\include{pgfplots.resources}
\include{pgfplots.importexport}
\include{pgfplots.basic.reference}

\printindex

\bibliographystyle{abbrv} %gerapali} %gerabbrv} %gerunsrt.bst} %gerabbrv}% gerplain}
\nocite{pgfplotstable}
\nocite{programmingnotes}
\bibliography{pgfplots}
\end{document}

% -----------------------------------------------------------------------------
% For Stefan Pinnow as reminder on what to look for when editing the manual
% -----------------------------------------------------------------------------
% There should be no line breaks in the following environments
% - |...|
% - \declareandlabel{...}
% - \verbpdfref{...}
% If "MakeTikzPictures" isn't running through check one of the externalized
% LOG files what is the cause of that.
% -----------------------------------------------------------------------------


%%%%%%%%%%%%%%%%%%%%%%%%%%%%%%%%%%%%%%%%%%%%%%%%%%%%%%%%%%%%%%%%%%%%%%%%%%%%%
%
% Package pgfplots.sty documentation.
%
% Copyright 2007/2008 by Christian Feuersaenger.
%
% This program is free software: you can redistribute it and/or modify
% it under the terms of the GNU General Public License as published by
% the Free Software Foundation, either version 3 of the License, or
% (at your option) any later version.
%
% This program is distributed in the hope that it will be useful,
% but WITHOUT ANY WARRANTY; without even the implied warranty of
% MERCHANTABILITY or FITNESS FOR A PARTICULAR PURPOSE.  See the
% GNU General Public License for more details.
%
% You should have received a copy of the GNU General Public License
% along with this program.  If not, see <http://www.gnu.org/licenses/>.
%
%
%%%%%%%%%%%%%%%%%%%%%%%%%%%%%%%%%%%%%%%%%%%%%%%%%%%%%%%%%%%%%%%%%%%%%%%%%%%%%
% SEE pgfplots-macros.tex as well!
%\pdfminorversion=5 % to allow compression
%\pdfobjcompresslevel=2
\documentclass[a4paper,openany]{book}

\let\bookmaketitle=\maketitle

% -----------------------------------
% this here is from ltxdoc:
\usepackage{doc}[=v2]
\AtBeginDocument{\MakeShortVerb{\|}}
%\setlength{\textwidth}{355pt}
%\addtolength\marginparwidth{30pt}
%\addtolength\oddsidemargin{20pt}
%\addtolength\evensidemargin{20pt}
\def\cmd#1{\cs{\expandafter\cmd@to@cs\string#1}}
\def\cmd@to@cs#1#2{\char\number`#2\relax}
\DeclareRobustCommand\cs[1]{\texttt{\char`\\#1}}
\providecommand\marg[1]{%
  {\ttfamily\char`\{}\meta{#1}{\ttfamily\char`\}}}
\providecommand\oarg[1]{%
  {\ttfamily[}\meta{#1}{\ttfamily]}}
\providecommand\parg[1]{%
  {\ttfamily(}\meta{#1}{\ttfamily)}}
\raggedbottom

% -----------------------------------
\input{pgfplots.preamble.tex}

\makeatletter
% I want a two-column index just as in pgfmanual styles. This here
% was the best way to get one:
\def\index@prologue{\section*{Index}\addcontentsline{toc}{chapter}{Index}
}
\makeatother

%\RequirePackage[german,english,francais]{babel}

\def\matlabcolormaptext{This colormap is similar to one shipped with Matlab$^\text{\textregistered}$ under a similar name.}

\IfFileExists{tikzlibraryspy.code.tex}{%
\usetikzlibrary{spy}
}{%
    \message{ERROR: tikz SPY library NOT available. The manual will only compile partially.^^J}%
}%

\usepackage{xparse}% for colorbrewer manual

\usetikzlibrary{
    decorations.markings,
    decorations.footprints,
    shapes.arrows,
    matrix,
    positioning,
}

\usepgfplotslibrary{
    fillbetween,
    ternary,
    smithchart,
    patchplots,
    polar,
    colormaps,
    colorbrewer,
    colortol,           % docu not ready yet
}
\pgfqkeys{/codeexample}{%
    every codeexample/.append style={
        /pgfplots/every ternary axis/.append style={
            /pgfplots/legend style={fill=graphicbackground},
        }
    },
    tabsize=4,
}

\pgfplotsmanualenableexternalizationofexpensive

%\usetikzlibrary{external}
%\tikzexternalize[prefix=figures/]{pgfplots}

\title{%
    Manual for Package \PGFPlots{}\\
    {\small 2D/3D Plots in \LaTeX{}, Version \pgfplotsversion}\\
    {\small\url{https://github.com/pgf-tikz/pgfplots}}
    %\\{\small Attention: you are using an unstable development version.}
}

\makeatletter
\long\def\abstractsmuggle{%
    \centering
    \textbf{Abstract}\\[0.5cm]

    \begin{minipage}{12cm}
        \PGFPlots{} draws high-quality function plots in normal or logarithmic
        scaling with a user-friendly interface directly in \TeX{}. The user
        supplies axis labels, legend entries and the plot coordinates for one or
        more plots and \PGFPlots{} applies axis scaling, computes any logarithms
        and axis ticks and draws the plots. It supports line plots, scatter
        plots, piecewise constant plots, bar plots, area plots, mesh and surface
        plots, patch plots, contour plots, quiver plots, histogram plots, box
        plots, polar axes, ternary diagrams, smith charts and some more. It is
        based on Till Tantau's package \PGF{}/\Tikz{}.
    \end{minipage}

}%

\expandafter\date\expandafter{\@date\\[2cm]
    \abstractsmuggle
}%
\makeatother

%\includeonly{pgfplots.reference}


\begin{document}

\def\plotcoords{%
\addplot coordinates {
(5,8.312e-02)    (17,2.547e-02)   (49,7.407e-03)
(129,2.102e-03)  (321,5.874e-04)  (769,1.623e-04)
(1793,4.442e-05) (4097,1.207e-05) (9217,3.261e-06)
};

\addplot coordinates{
(7,8.472e-02)    (31,3.044e-02)    (111,1.022e-02)
(351,3.303e-03)  (1023,1.039e-03)  (2815,3.196e-04)
(7423,9.658e-05) (18943,2.873e-05) (47103,8.437e-06)
};

\addplot coordinates{
(9,7.881e-02)     (49,3.243e-02)    (209,1.232e-02)
(769,4.454e-03)   (2561,1.551e-03)  (7937,5.236e-04)
(23297,1.723e-04) (65537,5.545e-05) (178177,1.751e-05)
};

\addplot coordinates{
(11,6.887e-02)    (71,3.177e-02)     (351,1.341e-02)
(1471,5.334e-03)  (5503,2.027e-03)   (18943,7.415e-04)
(61183,2.628e-04) (187903,9.063e-05) (553983,3.053e-05)
};

\addplot coordinates{
(13,5.755e-02)     (97,2.925e-02)     (545,1.351e-02)
(2561,5.842e-03)   (10625,2.397e-03)  (40193,9.414e-04)
(141569,3.564e-04) (471041,1.308e-04)
(1496065,4.670e-05)
};
}%


\bookmaketitle
\tableofcontents
\include{pgfplots.title_abstract_intro}
\include{pgfplots.preliminaries}
\include{pgfplots.intro}
\include{pgfplots.reference}
\include{pgfplots.libs}
\include{pgfplots.resources}
\include{pgfplots.importexport}
\include{pgfplots.basic.reference}

\printindex

\bibliographystyle{abbrv} %gerapali} %gerabbrv} %gerunsrt.bst} %gerabbrv}% gerplain}
\nocite{pgfplotstable}
\nocite{programmingnotes}
\bibliography{pgfplots}
\end{document}

% -----------------------------------------------------------------------------
% For Stefan Pinnow as reminder on what to look for when editing the manual
% -----------------------------------------------------------------------------
% There should be no line breaks in the following environments
% - |...|
% - \declareandlabel{...}
% - \verbpdfref{...}
% If "MakeTikzPictures" isn't running through check one of the externalized
% LOG files what is the cause of that.
% -----------------------------------------------------------------------------


%%%%%%%%%%%%%%%%%%%%%%%%%%%%%%%%%%%%%%%%%%%%%%%%%%%%%%%%%%%%%%%%%%%%%%%%%%%%%
%
% Package pgfplots.sty documentation.
%
% Copyright 2007/2008 by Christian Feuersaenger.
%
% This program is free software: you can redistribute it and/or modify
% it under the terms of the GNU General Public License as published by
% the Free Software Foundation, either version 3 of the License, or
% (at your option) any later version.
%
% This program is distributed in the hope that it will be useful,
% but WITHOUT ANY WARRANTY; without even the implied warranty of
% MERCHANTABILITY or FITNESS FOR A PARTICULAR PURPOSE.  See the
% GNU General Public License for more details.
%
% You should have received a copy of the GNU General Public License
% along with this program.  If not, see <http://www.gnu.org/licenses/>.
%
%
%%%%%%%%%%%%%%%%%%%%%%%%%%%%%%%%%%%%%%%%%%%%%%%%%%%%%%%%%%%%%%%%%%%%%%%%%%%%%
% SEE pgfplots-macros.tex as well!
%\pdfminorversion=5 % to allow compression
%\pdfobjcompresslevel=2
\documentclass[a4paper,openany]{book}

\let\bookmaketitle=\maketitle

% -----------------------------------
% this here is from ltxdoc:
\usepackage{doc}[=v2]
\AtBeginDocument{\MakeShortVerb{\|}}
%\setlength{\textwidth}{355pt}
%\addtolength\marginparwidth{30pt}
%\addtolength\oddsidemargin{20pt}
%\addtolength\evensidemargin{20pt}
\def\cmd#1{\cs{\expandafter\cmd@to@cs\string#1}}
\def\cmd@to@cs#1#2{\char\number`#2\relax}
\DeclareRobustCommand\cs[1]{\texttt{\char`\\#1}}
\providecommand\marg[1]{%
  {\ttfamily\char`\{}\meta{#1}{\ttfamily\char`\}}}
\providecommand\oarg[1]{%
  {\ttfamily[}\meta{#1}{\ttfamily]}}
\providecommand\parg[1]{%
  {\ttfamily(}\meta{#1}{\ttfamily)}}
\raggedbottom

% -----------------------------------
\input{pgfplots.preamble.tex}

\makeatletter
% I want a two-column index just as in pgfmanual styles. This here
% was the best way to get one:
\def\index@prologue{\section*{Index}\addcontentsline{toc}{chapter}{Index}
}
\makeatother

%\RequirePackage[german,english,francais]{babel}

\def\matlabcolormaptext{This colormap is similar to one shipped with Matlab$^\text{\textregistered}$ under a similar name.}

\IfFileExists{tikzlibraryspy.code.tex}{%
\usetikzlibrary{spy}
}{%
    \message{ERROR: tikz SPY library NOT available. The manual will only compile partially.^^J}%
}%

\usepackage{xparse}% for colorbrewer manual

\usetikzlibrary{
    decorations.markings,
    decorations.footprints,
    shapes.arrows,
    matrix,
    positioning,
}

\usepgfplotslibrary{
    fillbetween,
    ternary,
    smithchart,
    patchplots,
    polar,
    colormaps,
    colorbrewer,
    colortol,           % docu not ready yet
}
\pgfqkeys{/codeexample}{%
    every codeexample/.append style={
        /pgfplots/every ternary axis/.append style={
            /pgfplots/legend style={fill=graphicbackground},
        }
    },
    tabsize=4,
}

\pgfplotsmanualenableexternalizationofexpensive

%\usetikzlibrary{external}
%\tikzexternalize[prefix=figures/]{pgfplots}

\title{%
    Manual for Package \PGFPlots{}\\
    {\small 2D/3D Plots in \LaTeX{}, Version \pgfplotsversion}\\
    {\small\url{https://github.com/pgf-tikz/pgfplots}}
    %\\{\small Attention: you are using an unstable development version.}
}

\makeatletter
\long\def\abstractsmuggle{%
    \centering
    \textbf{Abstract}\\[0.5cm]

    \begin{minipage}{12cm}
        \PGFPlots{} draws high-quality function plots in normal or logarithmic
        scaling with a user-friendly interface directly in \TeX{}. The user
        supplies axis labels, legend entries and the plot coordinates for one or
        more plots and \PGFPlots{} applies axis scaling, computes any logarithms
        and axis ticks and draws the plots. It supports line plots, scatter
        plots, piecewise constant plots, bar plots, area plots, mesh and surface
        plots, patch plots, contour plots, quiver plots, histogram plots, box
        plots, polar axes, ternary diagrams, smith charts and some more. It is
        based on Till Tantau's package \PGF{}/\Tikz{}.
    \end{minipage}

}%

\expandafter\date\expandafter{\@date\\[2cm]
    \abstractsmuggle
}%
\makeatother

%\includeonly{pgfplots.reference}


\begin{document}

\def\plotcoords{%
\addplot coordinates {
(5,8.312e-02)    (17,2.547e-02)   (49,7.407e-03)
(129,2.102e-03)  (321,5.874e-04)  (769,1.623e-04)
(1793,4.442e-05) (4097,1.207e-05) (9217,3.261e-06)
};

\addplot coordinates{
(7,8.472e-02)    (31,3.044e-02)    (111,1.022e-02)
(351,3.303e-03)  (1023,1.039e-03)  (2815,3.196e-04)
(7423,9.658e-05) (18943,2.873e-05) (47103,8.437e-06)
};

\addplot coordinates{
(9,7.881e-02)     (49,3.243e-02)    (209,1.232e-02)
(769,4.454e-03)   (2561,1.551e-03)  (7937,5.236e-04)
(23297,1.723e-04) (65537,5.545e-05) (178177,1.751e-05)
};

\addplot coordinates{
(11,6.887e-02)    (71,3.177e-02)     (351,1.341e-02)
(1471,5.334e-03)  (5503,2.027e-03)   (18943,7.415e-04)
(61183,2.628e-04) (187903,9.063e-05) (553983,3.053e-05)
};

\addplot coordinates{
(13,5.755e-02)     (97,2.925e-02)     (545,1.351e-02)
(2561,5.842e-03)   (10625,2.397e-03)  (40193,9.414e-04)
(141569,3.564e-04) (471041,1.308e-04)
(1496065,4.670e-05)
};
}%


\bookmaketitle
\tableofcontents
\include{pgfplots.title_abstract_intro}
\include{pgfplots.preliminaries}
\include{pgfplots.intro}
\include{pgfplots.reference}
\include{pgfplots.libs}
\include{pgfplots.resources}
\include{pgfplots.importexport}
\include{pgfplots.basic.reference}

\printindex

\bibliographystyle{abbrv} %gerapali} %gerabbrv} %gerunsrt.bst} %gerabbrv}% gerplain}
\nocite{pgfplotstable}
\nocite{programmingnotes}
\bibliography{pgfplots}
\end{document}

% -----------------------------------------------------------------------------
% For Stefan Pinnow as reminder on what to look for when editing the manual
% -----------------------------------------------------------------------------
% There should be no line breaks in the following environments
% - |...|
% - \declareandlabel{...}
% - \verbpdfref{...}
% If "MakeTikzPictures" isn't running through check one of the externalized
% LOG files what is the cause of that.
% -----------------------------------------------------------------------------


%%%%%%%%%%%%%%%%%%%%%%%%%%%%%%%%%%%%%%%%%%%%%%%%%%%%%%%%%%%%%%%%%%%%%%%%%%%%%
%
% Package pgfplots.sty documentation.
%
% Copyright 2007/2008 by Christian Feuersaenger.
%
% This program is free software: you can redistribute it and/or modify
% it under the terms of the GNU General Public License as published by
% the Free Software Foundation, either version 3 of the License, or
% (at your option) any later version.
%
% This program is distributed in the hope that it will be useful,
% but WITHOUT ANY WARRANTY; without even the implied warranty of
% MERCHANTABILITY or FITNESS FOR A PARTICULAR PURPOSE.  See the
% GNU General Public License for more details.
%
% You should have received a copy of the GNU General Public License
% along with this program.  If not, see <http://www.gnu.org/licenses/>.
%
%
%%%%%%%%%%%%%%%%%%%%%%%%%%%%%%%%%%%%%%%%%%%%%%%%%%%%%%%%%%%%%%%%%%%%%%%%%%%%%
% SEE pgfplots-macros.tex as well!
%\pdfminorversion=5 % to allow compression
%\pdfobjcompresslevel=2
\documentclass[a4paper,openany]{book}

\let\bookmaketitle=\maketitle

% -----------------------------------
% this here is from ltxdoc:
\usepackage{doc}[=v2]
\AtBeginDocument{\MakeShortVerb{\|}}
%\setlength{\textwidth}{355pt}
%\addtolength\marginparwidth{30pt}
%\addtolength\oddsidemargin{20pt}
%\addtolength\evensidemargin{20pt}
\def\cmd#1{\cs{\expandafter\cmd@to@cs\string#1}}
\def\cmd@to@cs#1#2{\char\number`#2\relax}
\DeclareRobustCommand\cs[1]{\texttt{\char`\\#1}}
\providecommand\marg[1]{%
  {\ttfamily\char`\{}\meta{#1}{\ttfamily\char`\}}}
\providecommand\oarg[1]{%
  {\ttfamily[}\meta{#1}{\ttfamily]}}
\providecommand\parg[1]{%
  {\ttfamily(}\meta{#1}{\ttfamily)}}
\raggedbottom

% -----------------------------------
\input{pgfplots.preamble.tex}

\makeatletter
% I want a two-column index just as in pgfmanual styles. This here
% was the best way to get one:
\def\index@prologue{\section*{Index}\addcontentsline{toc}{chapter}{Index}
}
\makeatother

%\RequirePackage[german,english,francais]{babel}

\def\matlabcolormaptext{This colormap is similar to one shipped with Matlab$^\text{\textregistered}$ under a similar name.}

\IfFileExists{tikzlibraryspy.code.tex}{%
\usetikzlibrary{spy}
}{%
    \message{ERROR: tikz SPY library NOT available. The manual will only compile partially.^^J}%
}%

\usepackage{xparse}% for colorbrewer manual

\usetikzlibrary{
    decorations.markings,
    decorations.footprints,
    shapes.arrows,
    matrix,
    positioning,
}

\usepgfplotslibrary{
    fillbetween,
    ternary,
    smithchart,
    patchplots,
    polar,
    colormaps,
    colorbrewer,
    colortol,           % docu not ready yet
}
\pgfqkeys{/codeexample}{%
    every codeexample/.append style={
        /pgfplots/every ternary axis/.append style={
            /pgfplots/legend style={fill=graphicbackground},
        }
    },
    tabsize=4,
}

\pgfplotsmanualenableexternalizationofexpensive

%\usetikzlibrary{external}
%\tikzexternalize[prefix=figures/]{pgfplots}

\title{%
    Manual for Package \PGFPlots{}\\
    {\small 2D/3D Plots in \LaTeX{}, Version \pgfplotsversion}\\
    {\small\url{https://github.com/pgf-tikz/pgfplots}}
    %\\{\small Attention: you are using an unstable development version.}
}

\makeatletter
\long\def\abstractsmuggle{%
    \centering
    \textbf{Abstract}\\[0.5cm]

    \begin{minipage}{12cm}
        \PGFPlots{} draws high-quality function plots in normal or logarithmic
        scaling with a user-friendly interface directly in \TeX{}. The user
        supplies axis labels, legend entries and the plot coordinates for one or
        more plots and \PGFPlots{} applies axis scaling, computes any logarithms
        and axis ticks and draws the plots. It supports line plots, scatter
        plots, piecewise constant plots, bar plots, area plots, mesh and surface
        plots, patch plots, contour plots, quiver plots, histogram plots, box
        plots, polar axes, ternary diagrams, smith charts and some more. It is
        based on Till Tantau's package \PGF{}/\Tikz{}.
    \end{minipage}

}%

\expandafter\date\expandafter{\@date\\[2cm]
    \abstractsmuggle
}%
\makeatother

%\includeonly{pgfplots.reference}


\begin{document}

\def\plotcoords{%
\addplot coordinates {
(5,8.312e-02)    (17,2.547e-02)   (49,7.407e-03)
(129,2.102e-03)  (321,5.874e-04)  (769,1.623e-04)
(1793,4.442e-05) (4097,1.207e-05) (9217,3.261e-06)
};

\addplot coordinates{
(7,8.472e-02)    (31,3.044e-02)    (111,1.022e-02)
(351,3.303e-03)  (1023,1.039e-03)  (2815,3.196e-04)
(7423,9.658e-05) (18943,2.873e-05) (47103,8.437e-06)
};

\addplot coordinates{
(9,7.881e-02)     (49,3.243e-02)    (209,1.232e-02)
(769,4.454e-03)   (2561,1.551e-03)  (7937,5.236e-04)
(23297,1.723e-04) (65537,5.545e-05) (178177,1.751e-05)
};

\addplot coordinates{
(11,6.887e-02)    (71,3.177e-02)     (351,1.341e-02)
(1471,5.334e-03)  (5503,2.027e-03)   (18943,7.415e-04)
(61183,2.628e-04) (187903,9.063e-05) (553983,3.053e-05)
};

\addplot coordinates{
(13,5.755e-02)     (97,2.925e-02)     (545,1.351e-02)
(2561,5.842e-03)   (10625,2.397e-03)  (40193,9.414e-04)
(141569,3.564e-04) (471041,1.308e-04)
(1496065,4.670e-05)
};
}%


\bookmaketitle
\tableofcontents
\include{pgfplots.title_abstract_intro}
\include{pgfplots.preliminaries}
\include{pgfplots.intro}
\include{pgfplots.reference}
\include{pgfplots.libs}
\include{pgfplots.resources}
\include{pgfplots.importexport}
\include{pgfplots.basic.reference}

\printindex

\bibliographystyle{abbrv} %gerapali} %gerabbrv} %gerunsrt.bst} %gerabbrv}% gerplain}
\nocite{pgfplotstable}
\nocite{programmingnotes}
\bibliography{pgfplots}
\end{document}

% -----------------------------------------------------------------------------
% For Stefan Pinnow as reminder on what to look for when editing the manual
% -----------------------------------------------------------------------------
% There should be no line breaks in the following environments
% - |...|
% - \declareandlabel{...}
% - \verbpdfref{...}
% If "MakeTikzPictures" isn't running through check one of the externalized
% LOG files what is the cause of that.
% -----------------------------------------------------------------------------

\include{pgfplots.basic.reference}

\printindex

\bibliographystyle{abbrv} %gerapali} %gerabbrv} %gerunsrt.bst} %gerabbrv}% gerplain}
\nocite{pgfplotstable}
\nocite{programmingnotes}
\bibliography{pgfplots}
\end{document}

% -----------------------------------------------------------------------------
% For Stefan Pinnow as reminder on what to look for when editing the manual
% -----------------------------------------------------------------------------
% There should be no line breaks in the following environments
% - |...|
% - \declareandlabel{...}
% - \verbpdfref{...}
% If "MakeTikzPictures" isn't running through check one of the externalized
% LOG files what is the cause of that.
% -----------------------------------------------------------------------------


%%%%%%%%%%%%%%%%%%%%%%%%%%%%%%%%%%%%%%%%%%%%%%%%%%%%%%%%%%%%%%%%%%%%%%%%%%%%%
%
% Package pgfplots.sty documentation.
%
% Copyright 2007/2008 by Christian Feuersaenger.
%
% This program is free software: you can redistribute it and/or modify
% it under the terms of the GNU General Public License as published by
% the Free Software Foundation, either version 3 of the License, or
% (at your option) any later version.
%
% This program is distributed in the hope that it will be useful,
% but WITHOUT ANY WARRANTY; without even the implied warranty of
% MERCHANTABILITY or FITNESS FOR A PARTICULAR PURPOSE.  See the
% GNU General Public License for more details.
%
% You should have received a copy of the GNU General Public License
% along with this program.  If not, see <http://www.gnu.org/licenses/>.
%
%
%%%%%%%%%%%%%%%%%%%%%%%%%%%%%%%%%%%%%%%%%%%%%%%%%%%%%%%%%%%%%%%%%%%%%%%%%%%%%
% SEE pgfplots-macros.tex as well!
%\pdfminorversion=5 % to allow compression
%\pdfobjcompresslevel=2
\documentclass[a4paper,openany]{book}

\let\bookmaketitle=\maketitle

% -----------------------------------
% this here is from ltxdoc:
\usepackage{doc}[=v2]
\AtBeginDocument{\MakeShortVerb{\|}}
%\setlength{\textwidth}{355pt}
%\addtolength\marginparwidth{30pt}
%\addtolength\oddsidemargin{20pt}
%\addtolength\evensidemargin{20pt}
\def\cmd#1{\cs{\expandafter\cmd@to@cs\string#1}}
\def\cmd@to@cs#1#2{\char\number`#2\relax}
\DeclareRobustCommand\cs[1]{\texttt{\char`\\#1}}
\providecommand\marg[1]{%
  {\ttfamily\char`\{}\meta{#1}{\ttfamily\char`\}}}
\providecommand\oarg[1]{%
  {\ttfamily[}\meta{#1}{\ttfamily]}}
\providecommand\parg[1]{%
  {\ttfamily(}\meta{#1}{\ttfamily)}}
\raggedbottom

% -----------------------------------
\input{pgfplots.preamble.tex}

\makeatletter
% I want a two-column index just as in pgfmanual styles. This here
% was the best way to get one:
\def\index@prologue{\section*{Index}\addcontentsline{toc}{chapter}{Index}
}
\makeatother

%\RequirePackage[german,english,francais]{babel}

\def\matlabcolormaptext{This colormap is similar to one shipped with Matlab$^\text{\textregistered}$ under a similar name.}

\IfFileExists{tikzlibraryspy.code.tex}{%
\usetikzlibrary{spy}
}{%
    \message{ERROR: tikz SPY library NOT available. The manual will only compile partially.^^J}%
}%

\usepackage{xparse}% for colorbrewer manual

\usetikzlibrary{
    decorations.markings,
    decorations.footprints,
    shapes.arrows,
    matrix,
    positioning,
}

\usepgfplotslibrary{
    fillbetween,
    ternary,
    smithchart,
    patchplots,
    polar,
    colormaps,
    colorbrewer,
    colortol,           % docu not ready yet
}
\pgfqkeys{/codeexample}{%
    every codeexample/.append style={
        /pgfplots/every ternary axis/.append style={
            /pgfplots/legend style={fill=graphicbackground},
        }
    },
    tabsize=4,
}

\pgfplotsmanualenableexternalizationofexpensive

%\usetikzlibrary{external}
%\tikzexternalize[prefix=figures/]{pgfplots}

\title{%
    Manual for Package \PGFPlots{}\\
    {\small 2D/3D Plots in \LaTeX{}, Version \pgfplotsversion}\\
    {\small\url{https://github.com/pgf-tikz/pgfplots}}
    %\\{\small Attention: you are using an unstable development version.}
}

\makeatletter
\long\def\abstractsmuggle{%
    \centering
    \textbf{Abstract}\\[0.5cm]

    \begin{minipage}{12cm}
        \PGFPlots{} draws high-quality function plots in normal or logarithmic
        scaling with a user-friendly interface directly in \TeX{}. The user
        supplies axis labels, legend entries and the plot coordinates for one or
        more plots and \PGFPlots{} applies axis scaling, computes any logarithms
        and axis ticks and draws the plots. It supports line plots, scatter
        plots, piecewise constant plots, bar plots, area plots, mesh and surface
        plots, patch plots, contour plots, quiver plots, histogram plots, box
        plots, polar axes, ternary diagrams, smith charts and some more. It is
        based on Till Tantau's package \PGF{}/\Tikz{}.
    \end{minipage}

}%

\expandafter\date\expandafter{\@date\\[2cm]
    \abstractsmuggle
}%
\makeatother

%\includeonly{pgfplots.reference}


\begin{document}

\def\plotcoords{%
\addplot coordinates {
(5,8.312e-02)    (17,2.547e-02)   (49,7.407e-03)
(129,2.102e-03)  (321,5.874e-04)  (769,1.623e-04)
(1793,4.442e-05) (4097,1.207e-05) (9217,3.261e-06)
};

\addplot coordinates{
(7,8.472e-02)    (31,3.044e-02)    (111,1.022e-02)
(351,3.303e-03)  (1023,1.039e-03)  (2815,3.196e-04)
(7423,9.658e-05) (18943,2.873e-05) (47103,8.437e-06)
};

\addplot coordinates{
(9,7.881e-02)     (49,3.243e-02)    (209,1.232e-02)
(769,4.454e-03)   (2561,1.551e-03)  (7937,5.236e-04)
(23297,1.723e-04) (65537,5.545e-05) (178177,1.751e-05)
};

\addplot coordinates{
(11,6.887e-02)    (71,3.177e-02)     (351,1.341e-02)
(1471,5.334e-03)  (5503,2.027e-03)   (18943,7.415e-04)
(61183,2.628e-04) (187903,9.063e-05) (553983,3.053e-05)
};

\addplot coordinates{
(13,5.755e-02)     (97,2.925e-02)     (545,1.351e-02)
(2561,5.842e-03)   (10625,2.397e-03)  (40193,9.414e-04)
(141569,3.564e-04) (471041,1.308e-04)
(1496065,4.670e-05)
};
}%


\bookmaketitle
\tableofcontents

%%%%%%%%%%%%%%%%%%%%%%%%%%%%%%%%%%%%%%%%%%%%%%%%%%%%%%%%%%%%%%%%%%%%%%%%%%%%%
%
% Package pgfplots.sty documentation.
%
% Copyright 2007/2008 by Christian Feuersaenger.
%
% This program is free software: you can redistribute it and/or modify
% it under the terms of the GNU General Public License as published by
% the Free Software Foundation, either version 3 of the License, or
% (at your option) any later version.
%
% This program is distributed in the hope that it will be useful,
% but WITHOUT ANY WARRANTY; without even the implied warranty of
% MERCHANTABILITY or FITNESS FOR A PARTICULAR PURPOSE.  See the
% GNU General Public License for more details.
%
% You should have received a copy of the GNU General Public License
% along with this program.  If not, see <http://www.gnu.org/licenses/>.
%
%
%%%%%%%%%%%%%%%%%%%%%%%%%%%%%%%%%%%%%%%%%%%%%%%%%%%%%%%%%%%%%%%%%%%%%%%%%%%%%
% SEE pgfplots-macros.tex as well!
%\pdfminorversion=5 % to allow compression
%\pdfobjcompresslevel=2
\documentclass[a4paper,openany]{book}

\let\bookmaketitle=\maketitle

% -----------------------------------
% this here is from ltxdoc:
\usepackage{doc}[=v2]
\AtBeginDocument{\MakeShortVerb{\|}}
%\setlength{\textwidth}{355pt}
%\addtolength\marginparwidth{30pt}
%\addtolength\oddsidemargin{20pt}
%\addtolength\evensidemargin{20pt}
\def\cmd#1{\cs{\expandafter\cmd@to@cs\string#1}}
\def\cmd@to@cs#1#2{\char\number`#2\relax}
\DeclareRobustCommand\cs[1]{\texttt{\char`\\#1}}
\providecommand\marg[1]{%
  {\ttfamily\char`\{}\meta{#1}{\ttfamily\char`\}}}
\providecommand\oarg[1]{%
  {\ttfamily[}\meta{#1}{\ttfamily]}}
\providecommand\parg[1]{%
  {\ttfamily(}\meta{#1}{\ttfamily)}}
\raggedbottom

% -----------------------------------
\input{pgfplots.preamble.tex}

\makeatletter
% I want a two-column index just as in pgfmanual styles. This here
% was the best way to get one:
\def\index@prologue{\section*{Index}\addcontentsline{toc}{chapter}{Index}
}
\makeatother

%\RequirePackage[german,english,francais]{babel}

\def\matlabcolormaptext{This colormap is similar to one shipped with Matlab$^\text{\textregistered}$ under a similar name.}

\IfFileExists{tikzlibraryspy.code.tex}{%
\usetikzlibrary{spy}
}{%
    \message{ERROR: tikz SPY library NOT available. The manual will only compile partially.^^J}%
}%

\usepackage{xparse}% for colorbrewer manual

\usetikzlibrary{
    decorations.markings,
    decorations.footprints,
    shapes.arrows,
    matrix,
    positioning,
}

\usepgfplotslibrary{
    fillbetween,
    ternary,
    smithchart,
    patchplots,
    polar,
    colormaps,
    colorbrewer,
    colortol,           % docu not ready yet
}
\pgfqkeys{/codeexample}{%
    every codeexample/.append style={
        /pgfplots/every ternary axis/.append style={
            /pgfplots/legend style={fill=graphicbackground},
        }
    },
    tabsize=4,
}

\pgfplotsmanualenableexternalizationofexpensive

%\usetikzlibrary{external}
%\tikzexternalize[prefix=figures/]{pgfplots}

\title{%
    Manual for Package \PGFPlots{}\\
    {\small 2D/3D Plots in \LaTeX{}, Version \pgfplotsversion}\\
    {\small\url{https://github.com/pgf-tikz/pgfplots}}
    %\\{\small Attention: you are using an unstable development version.}
}

\makeatletter
\long\def\abstractsmuggle{%
    \centering
    \textbf{Abstract}\\[0.5cm]

    \begin{minipage}{12cm}
        \PGFPlots{} draws high-quality function plots in normal or logarithmic
        scaling with a user-friendly interface directly in \TeX{}. The user
        supplies axis labels, legend entries and the plot coordinates for one or
        more plots and \PGFPlots{} applies axis scaling, computes any logarithms
        and axis ticks and draws the plots. It supports line plots, scatter
        plots, piecewise constant plots, bar plots, area plots, mesh and surface
        plots, patch plots, contour plots, quiver plots, histogram plots, box
        plots, polar axes, ternary diagrams, smith charts and some more. It is
        based on Till Tantau's package \PGF{}/\Tikz{}.
    \end{minipage}

}%

\expandafter\date\expandafter{\@date\\[2cm]
    \abstractsmuggle
}%
\makeatother

%\includeonly{pgfplots.reference}


\begin{document}

\def\plotcoords{%
\addplot coordinates {
(5,8.312e-02)    (17,2.547e-02)   (49,7.407e-03)
(129,2.102e-03)  (321,5.874e-04)  (769,1.623e-04)
(1793,4.442e-05) (4097,1.207e-05) (9217,3.261e-06)
};

\addplot coordinates{
(7,8.472e-02)    (31,3.044e-02)    (111,1.022e-02)
(351,3.303e-03)  (1023,1.039e-03)  (2815,3.196e-04)
(7423,9.658e-05) (18943,2.873e-05) (47103,8.437e-06)
};

\addplot coordinates{
(9,7.881e-02)     (49,3.243e-02)    (209,1.232e-02)
(769,4.454e-03)   (2561,1.551e-03)  (7937,5.236e-04)
(23297,1.723e-04) (65537,5.545e-05) (178177,1.751e-05)
};

\addplot coordinates{
(11,6.887e-02)    (71,3.177e-02)     (351,1.341e-02)
(1471,5.334e-03)  (5503,2.027e-03)   (18943,7.415e-04)
(61183,2.628e-04) (187903,9.063e-05) (553983,3.053e-05)
};

\addplot coordinates{
(13,5.755e-02)     (97,2.925e-02)     (545,1.351e-02)
(2561,5.842e-03)   (10625,2.397e-03)  (40193,9.414e-04)
(141569,3.564e-04) (471041,1.308e-04)
(1496065,4.670e-05)
};
}%


\bookmaketitle
\tableofcontents
\include{pgfplots.title_abstract_intro}
\include{pgfplots.preliminaries}
\include{pgfplots.intro}
\include{pgfplots.reference}
\include{pgfplots.libs}
\include{pgfplots.resources}
\include{pgfplots.importexport}
\include{pgfplots.basic.reference}

\printindex

\bibliographystyle{abbrv} %gerapali} %gerabbrv} %gerunsrt.bst} %gerabbrv}% gerplain}
\nocite{pgfplotstable}
\nocite{programmingnotes}
\bibliography{pgfplots}
\end{document}

% -----------------------------------------------------------------------------
% For Stefan Pinnow as reminder on what to look for when editing the manual
% -----------------------------------------------------------------------------
% There should be no line breaks in the following environments
% - |...|
% - \declareandlabel{...}
% - \verbpdfref{...}
% If "MakeTikzPictures" isn't running through check one of the externalized
% LOG files what is the cause of that.
% -----------------------------------------------------------------------------


%%%%%%%%%%%%%%%%%%%%%%%%%%%%%%%%%%%%%%%%%%%%%%%%%%%%%%%%%%%%%%%%%%%%%%%%%%%%%
%
% Package pgfplots.sty documentation.
%
% Copyright 2007/2008 by Christian Feuersaenger.
%
% This program is free software: you can redistribute it and/or modify
% it under the terms of the GNU General Public License as published by
% the Free Software Foundation, either version 3 of the License, or
% (at your option) any later version.
%
% This program is distributed in the hope that it will be useful,
% but WITHOUT ANY WARRANTY; without even the implied warranty of
% MERCHANTABILITY or FITNESS FOR A PARTICULAR PURPOSE.  See the
% GNU General Public License for more details.
%
% You should have received a copy of the GNU General Public License
% along with this program.  If not, see <http://www.gnu.org/licenses/>.
%
%
%%%%%%%%%%%%%%%%%%%%%%%%%%%%%%%%%%%%%%%%%%%%%%%%%%%%%%%%%%%%%%%%%%%%%%%%%%%%%
% SEE pgfplots-macros.tex as well!
%\pdfminorversion=5 % to allow compression
%\pdfobjcompresslevel=2
\documentclass[a4paper,openany]{book}

\let\bookmaketitle=\maketitle

% -----------------------------------
% this here is from ltxdoc:
\usepackage{doc}[=v2]
\AtBeginDocument{\MakeShortVerb{\|}}
%\setlength{\textwidth}{355pt}
%\addtolength\marginparwidth{30pt}
%\addtolength\oddsidemargin{20pt}
%\addtolength\evensidemargin{20pt}
\def\cmd#1{\cs{\expandafter\cmd@to@cs\string#1}}
\def\cmd@to@cs#1#2{\char\number`#2\relax}
\DeclareRobustCommand\cs[1]{\texttt{\char`\\#1}}
\providecommand\marg[1]{%
  {\ttfamily\char`\{}\meta{#1}{\ttfamily\char`\}}}
\providecommand\oarg[1]{%
  {\ttfamily[}\meta{#1}{\ttfamily]}}
\providecommand\parg[1]{%
  {\ttfamily(}\meta{#1}{\ttfamily)}}
\raggedbottom

% -----------------------------------
\input{pgfplots.preamble.tex}

\makeatletter
% I want a two-column index just as in pgfmanual styles. This here
% was the best way to get one:
\def\index@prologue{\section*{Index}\addcontentsline{toc}{chapter}{Index}
}
\makeatother

%\RequirePackage[german,english,francais]{babel}

\def\matlabcolormaptext{This colormap is similar to one shipped with Matlab$^\text{\textregistered}$ under a similar name.}

\IfFileExists{tikzlibraryspy.code.tex}{%
\usetikzlibrary{spy}
}{%
    \message{ERROR: tikz SPY library NOT available. The manual will only compile partially.^^J}%
}%

\usepackage{xparse}% for colorbrewer manual

\usetikzlibrary{
    decorations.markings,
    decorations.footprints,
    shapes.arrows,
    matrix,
    positioning,
}

\usepgfplotslibrary{
    fillbetween,
    ternary,
    smithchart,
    patchplots,
    polar,
    colormaps,
    colorbrewer,
    colortol,           % docu not ready yet
}
\pgfqkeys{/codeexample}{%
    every codeexample/.append style={
        /pgfplots/every ternary axis/.append style={
            /pgfplots/legend style={fill=graphicbackground},
        }
    },
    tabsize=4,
}

\pgfplotsmanualenableexternalizationofexpensive

%\usetikzlibrary{external}
%\tikzexternalize[prefix=figures/]{pgfplots}

\title{%
    Manual for Package \PGFPlots{}\\
    {\small 2D/3D Plots in \LaTeX{}, Version \pgfplotsversion}\\
    {\small\url{https://github.com/pgf-tikz/pgfplots}}
    %\\{\small Attention: you are using an unstable development version.}
}

\makeatletter
\long\def\abstractsmuggle{%
    \centering
    \textbf{Abstract}\\[0.5cm]

    \begin{minipage}{12cm}
        \PGFPlots{} draws high-quality function plots in normal or logarithmic
        scaling with a user-friendly interface directly in \TeX{}. The user
        supplies axis labels, legend entries and the plot coordinates for one or
        more plots and \PGFPlots{} applies axis scaling, computes any logarithms
        and axis ticks and draws the plots. It supports line plots, scatter
        plots, piecewise constant plots, bar plots, area plots, mesh and surface
        plots, patch plots, contour plots, quiver plots, histogram plots, box
        plots, polar axes, ternary diagrams, smith charts and some more. It is
        based on Till Tantau's package \PGF{}/\Tikz{}.
    \end{minipage}

}%

\expandafter\date\expandafter{\@date\\[2cm]
    \abstractsmuggle
}%
\makeatother

%\includeonly{pgfplots.reference}


\begin{document}

\def\plotcoords{%
\addplot coordinates {
(5,8.312e-02)    (17,2.547e-02)   (49,7.407e-03)
(129,2.102e-03)  (321,5.874e-04)  (769,1.623e-04)
(1793,4.442e-05) (4097,1.207e-05) (9217,3.261e-06)
};

\addplot coordinates{
(7,8.472e-02)    (31,3.044e-02)    (111,1.022e-02)
(351,3.303e-03)  (1023,1.039e-03)  (2815,3.196e-04)
(7423,9.658e-05) (18943,2.873e-05) (47103,8.437e-06)
};

\addplot coordinates{
(9,7.881e-02)     (49,3.243e-02)    (209,1.232e-02)
(769,4.454e-03)   (2561,1.551e-03)  (7937,5.236e-04)
(23297,1.723e-04) (65537,5.545e-05) (178177,1.751e-05)
};

\addplot coordinates{
(11,6.887e-02)    (71,3.177e-02)     (351,1.341e-02)
(1471,5.334e-03)  (5503,2.027e-03)   (18943,7.415e-04)
(61183,2.628e-04) (187903,9.063e-05) (553983,3.053e-05)
};

\addplot coordinates{
(13,5.755e-02)     (97,2.925e-02)     (545,1.351e-02)
(2561,5.842e-03)   (10625,2.397e-03)  (40193,9.414e-04)
(141569,3.564e-04) (471041,1.308e-04)
(1496065,4.670e-05)
};
}%


\bookmaketitle
\tableofcontents
\include{pgfplots.title_abstract_intro}
\include{pgfplots.preliminaries}
\include{pgfplots.intro}
\include{pgfplots.reference}
\include{pgfplots.libs}
\include{pgfplots.resources}
\include{pgfplots.importexport}
\include{pgfplots.basic.reference}

\printindex

\bibliographystyle{abbrv} %gerapali} %gerabbrv} %gerunsrt.bst} %gerabbrv}% gerplain}
\nocite{pgfplotstable}
\nocite{programmingnotes}
\bibliography{pgfplots}
\end{document}

% -----------------------------------------------------------------------------
% For Stefan Pinnow as reminder on what to look for when editing the manual
% -----------------------------------------------------------------------------
% There should be no line breaks in the following environments
% - |...|
% - \declareandlabel{...}
% - \verbpdfref{...}
% If "MakeTikzPictures" isn't running through check one of the externalized
% LOG files what is the cause of that.
% -----------------------------------------------------------------------------


%%%%%%%%%%%%%%%%%%%%%%%%%%%%%%%%%%%%%%%%%%%%%%%%%%%%%%%%%%%%%%%%%%%%%%%%%%%%%
%
% Package pgfplots.sty documentation.
%
% Copyright 2007/2008 by Christian Feuersaenger.
%
% This program is free software: you can redistribute it and/or modify
% it under the terms of the GNU General Public License as published by
% the Free Software Foundation, either version 3 of the License, or
% (at your option) any later version.
%
% This program is distributed in the hope that it will be useful,
% but WITHOUT ANY WARRANTY; without even the implied warranty of
% MERCHANTABILITY or FITNESS FOR A PARTICULAR PURPOSE.  See the
% GNU General Public License for more details.
%
% You should have received a copy of the GNU General Public License
% along with this program.  If not, see <http://www.gnu.org/licenses/>.
%
%
%%%%%%%%%%%%%%%%%%%%%%%%%%%%%%%%%%%%%%%%%%%%%%%%%%%%%%%%%%%%%%%%%%%%%%%%%%%%%
% SEE pgfplots-macros.tex as well!
%\pdfminorversion=5 % to allow compression
%\pdfobjcompresslevel=2
\documentclass[a4paper,openany]{book}

\let\bookmaketitle=\maketitle

% -----------------------------------
% this here is from ltxdoc:
\usepackage{doc}[=v2]
\AtBeginDocument{\MakeShortVerb{\|}}
%\setlength{\textwidth}{355pt}
%\addtolength\marginparwidth{30pt}
%\addtolength\oddsidemargin{20pt}
%\addtolength\evensidemargin{20pt}
\def\cmd#1{\cs{\expandafter\cmd@to@cs\string#1}}
\def\cmd@to@cs#1#2{\char\number`#2\relax}
\DeclareRobustCommand\cs[1]{\texttt{\char`\\#1}}
\providecommand\marg[1]{%
  {\ttfamily\char`\{}\meta{#1}{\ttfamily\char`\}}}
\providecommand\oarg[1]{%
  {\ttfamily[}\meta{#1}{\ttfamily]}}
\providecommand\parg[1]{%
  {\ttfamily(}\meta{#1}{\ttfamily)}}
\raggedbottom

% -----------------------------------
\input{pgfplots.preamble.tex}

\makeatletter
% I want a two-column index just as in pgfmanual styles. This here
% was the best way to get one:
\def\index@prologue{\section*{Index}\addcontentsline{toc}{chapter}{Index}
}
\makeatother

%\RequirePackage[german,english,francais]{babel}

\def\matlabcolormaptext{This colormap is similar to one shipped with Matlab$^\text{\textregistered}$ under a similar name.}

\IfFileExists{tikzlibraryspy.code.tex}{%
\usetikzlibrary{spy}
}{%
    \message{ERROR: tikz SPY library NOT available. The manual will only compile partially.^^J}%
}%

\usepackage{xparse}% for colorbrewer manual

\usetikzlibrary{
    decorations.markings,
    decorations.footprints,
    shapes.arrows,
    matrix,
    positioning,
}

\usepgfplotslibrary{
    fillbetween,
    ternary,
    smithchart,
    patchplots,
    polar,
    colormaps,
    colorbrewer,
    colortol,           % docu not ready yet
}
\pgfqkeys{/codeexample}{%
    every codeexample/.append style={
        /pgfplots/every ternary axis/.append style={
            /pgfplots/legend style={fill=graphicbackground},
        }
    },
    tabsize=4,
}

\pgfplotsmanualenableexternalizationofexpensive

%\usetikzlibrary{external}
%\tikzexternalize[prefix=figures/]{pgfplots}

\title{%
    Manual for Package \PGFPlots{}\\
    {\small 2D/3D Plots in \LaTeX{}, Version \pgfplotsversion}\\
    {\small\url{https://github.com/pgf-tikz/pgfplots}}
    %\\{\small Attention: you are using an unstable development version.}
}

\makeatletter
\long\def\abstractsmuggle{%
    \centering
    \textbf{Abstract}\\[0.5cm]

    \begin{minipage}{12cm}
        \PGFPlots{} draws high-quality function plots in normal or logarithmic
        scaling with a user-friendly interface directly in \TeX{}. The user
        supplies axis labels, legend entries and the plot coordinates for one or
        more plots and \PGFPlots{} applies axis scaling, computes any logarithms
        and axis ticks and draws the plots. It supports line plots, scatter
        plots, piecewise constant plots, bar plots, area plots, mesh and surface
        plots, patch plots, contour plots, quiver plots, histogram plots, box
        plots, polar axes, ternary diagrams, smith charts and some more. It is
        based on Till Tantau's package \PGF{}/\Tikz{}.
    \end{minipage}

}%

\expandafter\date\expandafter{\@date\\[2cm]
    \abstractsmuggle
}%
\makeatother

%\includeonly{pgfplots.reference}


\begin{document}

\def\plotcoords{%
\addplot coordinates {
(5,8.312e-02)    (17,2.547e-02)   (49,7.407e-03)
(129,2.102e-03)  (321,5.874e-04)  (769,1.623e-04)
(1793,4.442e-05) (4097,1.207e-05) (9217,3.261e-06)
};

\addplot coordinates{
(7,8.472e-02)    (31,3.044e-02)    (111,1.022e-02)
(351,3.303e-03)  (1023,1.039e-03)  (2815,3.196e-04)
(7423,9.658e-05) (18943,2.873e-05) (47103,8.437e-06)
};

\addplot coordinates{
(9,7.881e-02)     (49,3.243e-02)    (209,1.232e-02)
(769,4.454e-03)   (2561,1.551e-03)  (7937,5.236e-04)
(23297,1.723e-04) (65537,5.545e-05) (178177,1.751e-05)
};

\addplot coordinates{
(11,6.887e-02)    (71,3.177e-02)     (351,1.341e-02)
(1471,5.334e-03)  (5503,2.027e-03)   (18943,7.415e-04)
(61183,2.628e-04) (187903,9.063e-05) (553983,3.053e-05)
};

\addplot coordinates{
(13,5.755e-02)     (97,2.925e-02)     (545,1.351e-02)
(2561,5.842e-03)   (10625,2.397e-03)  (40193,9.414e-04)
(141569,3.564e-04) (471041,1.308e-04)
(1496065,4.670e-05)
};
}%


\bookmaketitle
\tableofcontents
\include{pgfplots.title_abstract_intro}
\include{pgfplots.preliminaries}
\include{pgfplots.intro}
\include{pgfplots.reference}
\include{pgfplots.libs}
\include{pgfplots.resources}
\include{pgfplots.importexport}
\include{pgfplots.basic.reference}

\printindex

\bibliographystyle{abbrv} %gerapali} %gerabbrv} %gerunsrt.bst} %gerabbrv}% gerplain}
\nocite{pgfplotstable}
\nocite{programmingnotes}
\bibliography{pgfplots}
\end{document}

% -----------------------------------------------------------------------------
% For Stefan Pinnow as reminder on what to look for when editing the manual
% -----------------------------------------------------------------------------
% There should be no line breaks in the following environments
% - |...|
% - \declareandlabel{...}
% - \verbpdfref{...}
% If "MakeTikzPictures" isn't running through check one of the externalized
% LOG files what is the cause of that.
% -----------------------------------------------------------------------------


%%%%%%%%%%%%%%%%%%%%%%%%%%%%%%%%%%%%%%%%%%%%%%%%%%%%%%%%%%%%%%%%%%%%%%%%%%%%%
%
% Package pgfplots.sty documentation.
%
% Copyright 2007/2008 by Christian Feuersaenger.
%
% This program is free software: you can redistribute it and/or modify
% it under the terms of the GNU General Public License as published by
% the Free Software Foundation, either version 3 of the License, or
% (at your option) any later version.
%
% This program is distributed in the hope that it will be useful,
% but WITHOUT ANY WARRANTY; without even the implied warranty of
% MERCHANTABILITY or FITNESS FOR A PARTICULAR PURPOSE.  See the
% GNU General Public License for more details.
%
% You should have received a copy of the GNU General Public License
% along with this program.  If not, see <http://www.gnu.org/licenses/>.
%
%
%%%%%%%%%%%%%%%%%%%%%%%%%%%%%%%%%%%%%%%%%%%%%%%%%%%%%%%%%%%%%%%%%%%%%%%%%%%%%
% SEE pgfplots-macros.tex as well!
%\pdfminorversion=5 % to allow compression
%\pdfobjcompresslevel=2
\documentclass[a4paper,openany]{book}

\let\bookmaketitle=\maketitle

% -----------------------------------
% this here is from ltxdoc:
\usepackage{doc}[=v2]
\AtBeginDocument{\MakeShortVerb{\|}}
%\setlength{\textwidth}{355pt}
%\addtolength\marginparwidth{30pt}
%\addtolength\oddsidemargin{20pt}
%\addtolength\evensidemargin{20pt}
\def\cmd#1{\cs{\expandafter\cmd@to@cs\string#1}}
\def\cmd@to@cs#1#2{\char\number`#2\relax}
\DeclareRobustCommand\cs[1]{\texttt{\char`\\#1}}
\providecommand\marg[1]{%
  {\ttfamily\char`\{}\meta{#1}{\ttfamily\char`\}}}
\providecommand\oarg[1]{%
  {\ttfamily[}\meta{#1}{\ttfamily]}}
\providecommand\parg[1]{%
  {\ttfamily(}\meta{#1}{\ttfamily)}}
\raggedbottom

% -----------------------------------
\input{pgfplots.preamble.tex}

\makeatletter
% I want a two-column index just as in pgfmanual styles. This here
% was the best way to get one:
\def\index@prologue{\section*{Index}\addcontentsline{toc}{chapter}{Index}
}
\makeatother

%\RequirePackage[german,english,francais]{babel}

\def\matlabcolormaptext{This colormap is similar to one shipped with Matlab$^\text{\textregistered}$ under a similar name.}

\IfFileExists{tikzlibraryspy.code.tex}{%
\usetikzlibrary{spy}
}{%
    \message{ERROR: tikz SPY library NOT available. The manual will only compile partially.^^J}%
}%

\usepackage{xparse}% for colorbrewer manual

\usetikzlibrary{
    decorations.markings,
    decorations.footprints,
    shapes.arrows,
    matrix,
    positioning,
}

\usepgfplotslibrary{
    fillbetween,
    ternary,
    smithchart,
    patchplots,
    polar,
    colormaps,
    colorbrewer,
    colortol,           % docu not ready yet
}
\pgfqkeys{/codeexample}{%
    every codeexample/.append style={
        /pgfplots/every ternary axis/.append style={
            /pgfplots/legend style={fill=graphicbackground},
        }
    },
    tabsize=4,
}

\pgfplotsmanualenableexternalizationofexpensive

%\usetikzlibrary{external}
%\tikzexternalize[prefix=figures/]{pgfplots}

\title{%
    Manual for Package \PGFPlots{}\\
    {\small 2D/3D Plots in \LaTeX{}, Version \pgfplotsversion}\\
    {\small\url{https://github.com/pgf-tikz/pgfplots}}
    %\\{\small Attention: you are using an unstable development version.}
}

\makeatletter
\long\def\abstractsmuggle{%
    \centering
    \textbf{Abstract}\\[0.5cm]

    \begin{minipage}{12cm}
        \PGFPlots{} draws high-quality function plots in normal or logarithmic
        scaling with a user-friendly interface directly in \TeX{}. The user
        supplies axis labels, legend entries and the plot coordinates for one or
        more plots and \PGFPlots{} applies axis scaling, computes any logarithms
        and axis ticks and draws the plots. It supports line plots, scatter
        plots, piecewise constant plots, bar plots, area plots, mesh and surface
        plots, patch plots, contour plots, quiver plots, histogram plots, box
        plots, polar axes, ternary diagrams, smith charts and some more. It is
        based on Till Tantau's package \PGF{}/\Tikz{}.
    \end{minipage}

}%

\expandafter\date\expandafter{\@date\\[2cm]
    \abstractsmuggle
}%
\makeatother

%\includeonly{pgfplots.reference}


\begin{document}

\def\plotcoords{%
\addplot coordinates {
(5,8.312e-02)    (17,2.547e-02)   (49,7.407e-03)
(129,2.102e-03)  (321,5.874e-04)  (769,1.623e-04)
(1793,4.442e-05) (4097,1.207e-05) (9217,3.261e-06)
};

\addplot coordinates{
(7,8.472e-02)    (31,3.044e-02)    (111,1.022e-02)
(351,3.303e-03)  (1023,1.039e-03)  (2815,3.196e-04)
(7423,9.658e-05) (18943,2.873e-05) (47103,8.437e-06)
};

\addplot coordinates{
(9,7.881e-02)     (49,3.243e-02)    (209,1.232e-02)
(769,4.454e-03)   (2561,1.551e-03)  (7937,5.236e-04)
(23297,1.723e-04) (65537,5.545e-05) (178177,1.751e-05)
};

\addplot coordinates{
(11,6.887e-02)    (71,3.177e-02)     (351,1.341e-02)
(1471,5.334e-03)  (5503,2.027e-03)   (18943,7.415e-04)
(61183,2.628e-04) (187903,9.063e-05) (553983,3.053e-05)
};

\addplot coordinates{
(13,5.755e-02)     (97,2.925e-02)     (545,1.351e-02)
(2561,5.842e-03)   (10625,2.397e-03)  (40193,9.414e-04)
(141569,3.564e-04) (471041,1.308e-04)
(1496065,4.670e-05)
};
}%


\bookmaketitle
\tableofcontents
\include{pgfplots.title_abstract_intro}
\include{pgfplots.preliminaries}
\include{pgfplots.intro}
\include{pgfplots.reference}
\include{pgfplots.libs}
\include{pgfplots.resources}
\include{pgfplots.importexport}
\include{pgfplots.basic.reference}

\printindex

\bibliographystyle{abbrv} %gerapali} %gerabbrv} %gerunsrt.bst} %gerabbrv}% gerplain}
\nocite{pgfplotstable}
\nocite{programmingnotes}
\bibliography{pgfplots}
\end{document}

% -----------------------------------------------------------------------------
% For Stefan Pinnow as reminder on what to look for when editing the manual
% -----------------------------------------------------------------------------
% There should be no line breaks in the following environments
% - |...|
% - \declareandlabel{...}
% - \verbpdfref{...}
% If "MakeTikzPictures" isn't running through check one of the externalized
% LOG files what is the cause of that.
% -----------------------------------------------------------------------------


%%%%%%%%%%%%%%%%%%%%%%%%%%%%%%%%%%%%%%%%%%%%%%%%%%%%%%%%%%%%%%%%%%%%%%%%%%%%%
%
% Package pgfplots.sty documentation.
%
% Copyright 2007/2008 by Christian Feuersaenger.
%
% This program is free software: you can redistribute it and/or modify
% it under the terms of the GNU General Public License as published by
% the Free Software Foundation, either version 3 of the License, or
% (at your option) any later version.
%
% This program is distributed in the hope that it will be useful,
% but WITHOUT ANY WARRANTY; without even the implied warranty of
% MERCHANTABILITY or FITNESS FOR A PARTICULAR PURPOSE.  See the
% GNU General Public License for more details.
%
% You should have received a copy of the GNU General Public License
% along with this program.  If not, see <http://www.gnu.org/licenses/>.
%
%
%%%%%%%%%%%%%%%%%%%%%%%%%%%%%%%%%%%%%%%%%%%%%%%%%%%%%%%%%%%%%%%%%%%%%%%%%%%%%
% SEE pgfplots-macros.tex as well!
%\pdfminorversion=5 % to allow compression
%\pdfobjcompresslevel=2
\documentclass[a4paper,openany]{book}

\let\bookmaketitle=\maketitle

% -----------------------------------
% this here is from ltxdoc:
\usepackage{doc}[=v2]
\AtBeginDocument{\MakeShortVerb{\|}}
%\setlength{\textwidth}{355pt}
%\addtolength\marginparwidth{30pt}
%\addtolength\oddsidemargin{20pt}
%\addtolength\evensidemargin{20pt}
\def\cmd#1{\cs{\expandafter\cmd@to@cs\string#1}}
\def\cmd@to@cs#1#2{\char\number`#2\relax}
\DeclareRobustCommand\cs[1]{\texttt{\char`\\#1}}
\providecommand\marg[1]{%
  {\ttfamily\char`\{}\meta{#1}{\ttfamily\char`\}}}
\providecommand\oarg[1]{%
  {\ttfamily[}\meta{#1}{\ttfamily]}}
\providecommand\parg[1]{%
  {\ttfamily(}\meta{#1}{\ttfamily)}}
\raggedbottom

% -----------------------------------
\input{pgfplots.preamble.tex}

\makeatletter
% I want a two-column index just as in pgfmanual styles. This here
% was the best way to get one:
\def\index@prologue{\section*{Index}\addcontentsline{toc}{chapter}{Index}
}
\makeatother

%\RequirePackage[german,english,francais]{babel}

\def\matlabcolormaptext{This colormap is similar to one shipped with Matlab$^\text{\textregistered}$ under a similar name.}

\IfFileExists{tikzlibraryspy.code.tex}{%
\usetikzlibrary{spy}
}{%
    \message{ERROR: tikz SPY library NOT available. The manual will only compile partially.^^J}%
}%

\usepackage{xparse}% for colorbrewer manual

\usetikzlibrary{
    decorations.markings,
    decorations.footprints,
    shapes.arrows,
    matrix,
    positioning,
}

\usepgfplotslibrary{
    fillbetween,
    ternary,
    smithchart,
    patchplots,
    polar,
    colormaps,
    colorbrewer,
    colortol,           % docu not ready yet
}
\pgfqkeys{/codeexample}{%
    every codeexample/.append style={
        /pgfplots/every ternary axis/.append style={
            /pgfplots/legend style={fill=graphicbackground},
        }
    },
    tabsize=4,
}

\pgfplotsmanualenableexternalizationofexpensive

%\usetikzlibrary{external}
%\tikzexternalize[prefix=figures/]{pgfplots}

\title{%
    Manual for Package \PGFPlots{}\\
    {\small 2D/3D Plots in \LaTeX{}, Version \pgfplotsversion}\\
    {\small\url{https://github.com/pgf-tikz/pgfplots}}
    %\\{\small Attention: you are using an unstable development version.}
}

\makeatletter
\long\def\abstractsmuggle{%
    \centering
    \textbf{Abstract}\\[0.5cm]

    \begin{minipage}{12cm}
        \PGFPlots{} draws high-quality function plots in normal or logarithmic
        scaling with a user-friendly interface directly in \TeX{}. The user
        supplies axis labels, legend entries and the plot coordinates for one or
        more plots and \PGFPlots{} applies axis scaling, computes any logarithms
        and axis ticks and draws the plots. It supports line plots, scatter
        plots, piecewise constant plots, bar plots, area plots, mesh and surface
        plots, patch plots, contour plots, quiver plots, histogram plots, box
        plots, polar axes, ternary diagrams, smith charts and some more. It is
        based on Till Tantau's package \PGF{}/\Tikz{}.
    \end{minipage}

}%

\expandafter\date\expandafter{\@date\\[2cm]
    \abstractsmuggle
}%
\makeatother

%\includeonly{pgfplots.reference}


\begin{document}

\def\plotcoords{%
\addplot coordinates {
(5,8.312e-02)    (17,2.547e-02)   (49,7.407e-03)
(129,2.102e-03)  (321,5.874e-04)  (769,1.623e-04)
(1793,4.442e-05) (4097,1.207e-05) (9217,3.261e-06)
};

\addplot coordinates{
(7,8.472e-02)    (31,3.044e-02)    (111,1.022e-02)
(351,3.303e-03)  (1023,1.039e-03)  (2815,3.196e-04)
(7423,9.658e-05) (18943,2.873e-05) (47103,8.437e-06)
};

\addplot coordinates{
(9,7.881e-02)     (49,3.243e-02)    (209,1.232e-02)
(769,4.454e-03)   (2561,1.551e-03)  (7937,5.236e-04)
(23297,1.723e-04) (65537,5.545e-05) (178177,1.751e-05)
};

\addplot coordinates{
(11,6.887e-02)    (71,3.177e-02)     (351,1.341e-02)
(1471,5.334e-03)  (5503,2.027e-03)   (18943,7.415e-04)
(61183,2.628e-04) (187903,9.063e-05) (553983,3.053e-05)
};

\addplot coordinates{
(13,5.755e-02)     (97,2.925e-02)     (545,1.351e-02)
(2561,5.842e-03)   (10625,2.397e-03)  (40193,9.414e-04)
(141569,3.564e-04) (471041,1.308e-04)
(1496065,4.670e-05)
};
}%


\bookmaketitle
\tableofcontents
\include{pgfplots.title_abstract_intro}
\include{pgfplots.preliminaries}
\include{pgfplots.intro}
\include{pgfplots.reference}
\include{pgfplots.libs}
\include{pgfplots.resources}
\include{pgfplots.importexport}
\include{pgfplots.basic.reference}

\printindex

\bibliographystyle{abbrv} %gerapali} %gerabbrv} %gerunsrt.bst} %gerabbrv}% gerplain}
\nocite{pgfplotstable}
\nocite{programmingnotes}
\bibliography{pgfplots}
\end{document}

% -----------------------------------------------------------------------------
% For Stefan Pinnow as reminder on what to look for when editing the manual
% -----------------------------------------------------------------------------
% There should be no line breaks in the following environments
% - |...|
% - \declareandlabel{...}
% - \verbpdfref{...}
% If "MakeTikzPictures" isn't running through check one of the externalized
% LOG files what is the cause of that.
% -----------------------------------------------------------------------------


%%%%%%%%%%%%%%%%%%%%%%%%%%%%%%%%%%%%%%%%%%%%%%%%%%%%%%%%%%%%%%%%%%%%%%%%%%%%%
%
% Package pgfplots.sty documentation.
%
% Copyright 2007/2008 by Christian Feuersaenger.
%
% This program is free software: you can redistribute it and/or modify
% it under the terms of the GNU General Public License as published by
% the Free Software Foundation, either version 3 of the License, or
% (at your option) any later version.
%
% This program is distributed in the hope that it will be useful,
% but WITHOUT ANY WARRANTY; without even the implied warranty of
% MERCHANTABILITY or FITNESS FOR A PARTICULAR PURPOSE.  See the
% GNU General Public License for more details.
%
% You should have received a copy of the GNU General Public License
% along with this program.  If not, see <http://www.gnu.org/licenses/>.
%
%
%%%%%%%%%%%%%%%%%%%%%%%%%%%%%%%%%%%%%%%%%%%%%%%%%%%%%%%%%%%%%%%%%%%%%%%%%%%%%
% SEE pgfplots-macros.tex as well!
%\pdfminorversion=5 % to allow compression
%\pdfobjcompresslevel=2
\documentclass[a4paper,openany]{book}

\let\bookmaketitle=\maketitle

% -----------------------------------
% this here is from ltxdoc:
\usepackage{doc}[=v2]
\AtBeginDocument{\MakeShortVerb{\|}}
%\setlength{\textwidth}{355pt}
%\addtolength\marginparwidth{30pt}
%\addtolength\oddsidemargin{20pt}
%\addtolength\evensidemargin{20pt}
\def\cmd#1{\cs{\expandafter\cmd@to@cs\string#1}}
\def\cmd@to@cs#1#2{\char\number`#2\relax}
\DeclareRobustCommand\cs[1]{\texttt{\char`\\#1}}
\providecommand\marg[1]{%
  {\ttfamily\char`\{}\meta{#1}{\ttfamily\char`\}}}
\providecommand\oarg[1]{%
  {\ttfamily[}\meta{#1}{\ttfamily]}}
\providecommand\parg[1]{%
  {\ttfamily(}\meta{#1}{\ttfamily)}}
\raggedbottom

% -----------------------------------
\input{pgfplots.preamble.tex}

\makeatletter
% I want a two-column index just as in pgfmanual styles. This here
% was the best way to get one:
\def\index@prologue{\section*{Index}\addcontentsline{toc}{chapter}{Index}
}
\makeatother

%\RequirePackage[german,english,francais]{babel}

\def\matlabcolormaptext{This colormap is similar to one shipped with Matlab$^\text{\textregistered}$ under a similar name.}

\IfFileExists{tikzlibraryspy.code.tex}{%
\usetikzlibrary{spy}
}{%
    \message{ERROR: tikz SPY library NOT available. The manual will only compile partially.^^J}%
}%

\usepackage{xparse}% for colorbrewer manual

\usetikzlibrary{
    decorations.markings,
    decorations.footprints,
    shapes.arrows,
    matrix,
    positioning,
}

\usepgfplotslibrary{
    fillbetween,
    ternary,
    smithchart,
    patchplots,
    polar,
    colormaps,
    colorbrewer,
    colortol,           % docu not ready yet
}
\pgfqkeys{/codeexample}{%
    every codeexample/.append style={
        /pgfplots/every ternary axis/.append style={
            /pgfplots/legend style={fill=graphicbackground},
        }
    },
    tabsize=4,
}

\pgfplotsmanualenableexternalizationofexpensive

%\usetikzlibrary{external}
%\tikzexternalize[prefix=figures/]{pgfplots}

\title{%
    Manual for Package \PGFPlots{}\\
    {\small 2D/3D Plots in \LaTeX{}, Version \pgfplotsversion}\\
    {\small\url{https://github.com/pgf-tikz/pgfplots}}
    %\\{\small Attention: you are using an unstable development version.}
}

\makeatletter
\long\def\abstractsmuggle{%
    \centering
    \textbf{Abstract}\\[0.5cm]

    \begin{minipage}{12cm}
        \PGFPlots{} draws high-quality function plots in normal or logarithmic
        scaling with a user-friendly interface directly in \TeX{}. The user
        supplies axis labels, legend entries and the plot coordinates for one or
        more plots and \PGFPlots{} applies axis scaling, computes any logarithms
        and axis ticks and draws the plots. It supports line plots, scatter
        plots, piecewise constant plots, bar plots, area plots, mesh and surface
        plots, patch plots, contour plots, quiver plots, histogram plots, box
        plots, polar axes, ternary diagrams, smith charts and some more. It is
        based on Till Tantau's package \PGF{}/\Tikz{}.
    \end{minipage}

}%

\expandafter\date\expandafter{\@date\\[2cm]
    \abstractsmuggle
}%
\makeatother

%\includeonly{pgfplots.reference}


\begin{document}

\def\plotcoords{%
\addplot coordinates {
(5,8.312e-02)    (17,2.547e-02)   (49,7.407e-03)
(129,2.102e-03)  (321,5.874e-04)  (769,1.623e-04)
(1793,4.442e-05) (4097,1.207e-05) (9217,3.261e-06)
};

\addplot coordinates{
(7,8.472e-02)    (31,3.044e-02)    (111,1.022e-02)
(351,3.303e-03)  (1023,1.039e-03)  (2815,3.196e-04)
(7423,9.658e-05) (18943,2.873e-05) (47103,8.437e-06)
};

\addplot coordinates{
(9,7.881e-02)     (49,3.243e-02)    (209,1.232e-02)
(769,4.454e-03)   (2561,1.551e-03)  (7937,5.236e-04)
(23297,1.723e-04) (65537,5.545e-05) (178177,1.751e-05)
};

\addplot coordinates{
(11,6.887e-02)    (71,3.177e-02)     (351,1.341e-02)
(1471,5.334e-03)  (5503,2.027e-03)   (18943,7.415e-04)
(61183,2.628e-04) (187903,9.063e-05) (553983,3.053e-05)
};

\addplot coordinates{
(13,5.755e-02)     (97,2.925e-02)     (545,1.351e-02)
(2561,5.842e-03)   (10625,2.397e-03)  (40193,9.414e-04)
(141569,3.564e-04) (471041,1.308e-04)
(1496065,4.670e-05)
};
}%


\bookmaketitle
\tableofcontents
\include{pgfplots.title_abstract_intro}
\include{pgfplots.preliminaries}
\include{pgfplots.intro}
\include{pgfplots.reference}
\include{pgfplots.libs}
\include{pgfplots.resources}
\include{pgfplots.importexport}
\include{pgfplots.basic.reference}

\printindex

\bibliographystyle{abbrv} %gerapali} %gerabbrv} %gerunsrt.bst} %gerabbrv}% gerplain}
\nocite{pgfplotstable}
\nocite{programmingnotes}
\bibliography{pgfplots}
\end{document}

% -----------------------------------------------------------------------------
% For Stefan Pinnow as reminder on what to look for when editing the manual
% -----------------------------------------------------------------------------
% There should be no line breaks in the following environments
% - |...|
% - \declareandlabel{...}
% - \verbpdfref{...}
% If "MakeTikzPictures" isn't running through check one of the externalized
% LOG files what is the cause of that.
% -----------------------------------------------------------------------------


%%%%%%%%%%%%%%%%%%%%%%%%%%%%%%%%%%%%%%%%%%%%%%%%%%%%%%%%%%%%%%%%%%%%%%%%%%%%%
%
% Package pgfplots.sty documentation.
%
% Copyright 2007/2008 by Christian Feuersaenger.
%
% This program is free software: you can redistribute it and/or modify
% it under the terms of the GNU General Public License as published by
% the Free Software Foundation, either version 3 of the License, or
% (at your option) any later version.
%
% This program is distributed in the hope that it will be useful,
% but WITHOUT ANY WARRANTY; without even the implied warranty of
% MERCHANTABILITY or FITNESS FOR A PARTICULAR PURPOSE.  See the
% GNU General Public License for more details.
%
% You should have received a copy of the GNU General Public License
% along with this program.  If not, see <http://www.gnu.org/licenses/>.
%
%
%%%%%%%%%%%%%%%%%%%%%%%%%%%%%%%%%%%%%%%%%%%%%%%%%%%%%%%%%%%%%%%%%%%%%%%%%%%%%
% SEE pgfplots-macros.tex as well!
%\pdfminorversion=5 % to allow compression
%\pdfobjcompresslevel=2
\documentclass[a4paper,openany]{book}

\let\bookmaketitle=\maketitle

% -----------------------------------
% this here is from ltxdoc:
\usepackage{doc}[=v2]
\AtBeginDocument{\MakeShortVerb{\|}}
%\setlength{\textwidth}{355pt}
%\addtolength\marginparwidth{30pt}
%\addtolength\oddsidemargin{20pt}
%\addtolength\evensidemargin{20pt}
\def\cmd#1{\cs{\expandafter\cmd@to@cs\string#1}}
\def\cmd@to@cs#1#2{\char\number`#2\relax}
\DeclareRobustCommand\cs[1]{\texttt{\char`\\#1}}
\providecommand\marg[1]{%
  {\ttfamily\char`\{}\meta{#1}{\ttfamily\char`\}}}
\providecommand\oarg[1]{%
  {\ttfamily[}\meta{#1}{\ttfamily]}}
\providecommand\parg[1]{%
  {\ttfamily(}\meta{#1}{\ttfamily)}}
\raggedbottom

% -----------------------------------
\input{pgfplots.preamble.tex}

\makeatletter
% I want a two-column index just as in pgfmanual styles. This here
% was the best way to get one:
\def\index@prologue{\section*{Index}\addcontentsline{toc}{chapter}{Index}
}
\makeatother

%\RequirePackage[german,english,francais]{babel}

\def\matlabcolormaptext{This colormap is similar to one shipped with Matlab$^\text{\textregistered}$ under a similar name.}

\IfFileExists{tikzlibraryspy.code.tex}{%
\usetikzlibrary{spy}
}{%
    \message{ERROR: tikz SPY library NOT available. The manual will only compile partially.^^J}%
}%

\usepackage{xparse}% for colorbrewer manual

\usetikzlibrary{
    decorations.markings,
    decorations.footprints,
    shapes.arrows,
    matrix,
    positioning,
}

\usepgfplotslibrary{
    fillbetween,
    ternary,
    smithchart,
    patchplots,
    polar,
    colormaps,
    colorbrewer,
    colortol,           % docu not ready yet
}
\pgfqkeys{/codeexample}{%
    every codeexample/.append style={
        /pgfplots/every ternary axis/.append style={
            /pgfplots/legend style={fill=graphicbackground},
        }
    },
    tabsize=4,
}

\pgfplotsmanualenableexternalizationofexpensive

%\usetikzlibrary{external}
%\tikzexternalize[prefix=figures/]{pgfplots}

\title{%
    Manual for Package \PGFPlots{}\\
    {\small 2D/3D Plots in \LaTeX{}, Version \pgfplotsversion}\\
    {\small\url{https://github.com/pgf-tikz/pgfplots}}
    %\\{\small Attention: you are using an unstable development version.}
}

\makeatletter
\long\def\abstractsmuggle{%
    \centering
    \textbf{Abstract}\\[0.5cm]

    \begin{minipage}{12cm}
        \PGFPlots{} draws high-quality function plots in normal or logarithmic
        scaling with a user-friendly interface directly in \TeX{}. The user
        supplies axis labels, legend entries and the plot coordinates for one or
        more plots and \PGFPlots{} applies axis scaling, computes any logarithms
        and axis ticks and draws the plots. It supports line plots, scatter
        plots, piecewise constant plots, bar plots, area plots, mesh and surface
        plots, patch plots, contour plots, quiver plots, histogram plots, box
        plots, polar axes, ternary diagrams, smith charts and some more. It is
        based on Till Tantau's package \PGF{}/\Tikz{}.
    \end{minipage}

}%

\expandafter\date\expandafter{\@date\\[2cm]
    \abstractsmuggle
}%
\makeatother

%\includeonly{pgfplots.reference}


\begin{document}

\def\plotcoords{%
\addplot coordinates {
(5,8.312e-02)    (17,2.547e-02)   (49,7.407e-03)
(129,2.102e-03)  (321,5.874e-04)  (769,1.623e-04)
(1793,4.442e-05) (4097,1.207e-05) (9217,3.261e-06)
};

\addplot coordinates{
(7,8.472e-02)    (31,3.044e-02)    (111,1.022e-02)
(351,3.303e-03)  (1023,1.039e-03)  (2815,3.196e-04)
(7423,9.658e-05) (18943,2.873e-05) (47103,8.437e-06)
};

\addplot coordinates{
(9,7.881e-02)     (49,3.243e-02)    (209,1.232e-02)
(769,4.454e-03)   (2561,1.551e-03)  (7937,5.236e-04)
(23297,1.723e-04) (65537,5.545e-05) (178177,1.751e-05)
};

\addplot coordinates{
(11,6.887e-02)    (71,3.177e-02)     (351,1.341e-02)
(1471,5.334e-03)  (5503,2.027e-03)   (18943,7.415e-04)
(61183,2.628e-04) (187903,9.063e-05) (553983,3.053e-05)
};

\addplot coordinates{
(13,5.755e-02)     (97,2.925e-02)     (545,1.351e-02)
(2561,5.842e-03)   (10625,2.397e-03)  (40193,9.414e-04)
(141569,3.564e-04) (471041,1.308e-04)
(1496065,4.670e-05)
};
}%


\bookmaketitle
\tableofcontents
\include{pgfplots.title_abstract_intro}
\include{pgfplots.preliminaries}
\include{pgfplots.intro}
\include{pgfplots.reference}
\include{pgfplots.libs}
\include{pgfplots.resources}
\include{pgfplots.importexport}
\include{pgfplots.basic.reference}

\printindex

\bibliographystyle{abbrv} %gerapali} %gerabbrv} %gerunsrt.bst} %gerabbrv}% gerplain}
\nocite{pgfplotstable}
\nocite{programmingnotes}
\bibliography{pgfplots}
\end{document}

% -----------------------------------------------------------------------------
% For Stefan Pinnow as reminder on what to look for when editing the manual
% -----------------------------------------------------------------------------
% There should be no line breaks in the following environments
% - |...|
% - \declareandlabel{...}
% - \verbpdfref{...}
% If "MakeTikzPictures" isn't running through check one of the externalized
% LOG files what is the cause of that.
% -----------------------------------------------------------------------------

\include{pgfplots.basic.reference}

\printindex

\bibliographystyle{abbrv} %gerapali} %gerabbrv} %gerunsrt.bst} %gerabbrv}% gerplain}
\nocite{pgfplotstable}
\nocite{programmingnotes}
\bibliography{pgfplots}
\end{document}

% -----------------------------------------------------------------------------
% For Stefan Pinnow as reminder on what to look for when editing the manual
% -----------------------------------------------------------------------------
% There should be no line breaks in the following environments
% - |...|
% - \declareandlabel{...}
% - \verbpdfref{...}
% If "MakeTikzPictures" isn't running through check one of the externalized
% LOG files what is the cause of that.
% -----------------------------------------------------------------------------


%%%%%%%%%%%%%%%%%%%%%%%%%%%%%%%%%%%%%%%%%%%%%%%%%%%%%%%%%%%%%%%%%%%%%%%%%%%%%
%
% Package pgfplots.sty documentation.
%
% Copyright 2007/2008 by Christian Feuersaenger.
%
% This program is free software: you can redistribute it and/or modify
% it under the terms of the GNU General Public License as published by
% the Free Software Foundation, either version 3 of the License, or
% (at your option) any later version.
%
% This program is distributed in the hope that it will be useful,
% but WITHOUT ANY WARRANTY; without even the implied warranty of
% MERCHANTABILITY or FITNESS FOR A PARTICULAR PURPOSE.  See the
% GNU General Public License for more details.
%
% You should have received a copy of the GNU General Public License
% along with this program.  If not, see <http://www.gnu.org/licenses/>.
%
%
%%%%%%%%%%%%%%%%%%%%%%%%%%%%%%%%%%%%%%%%%%%%%%%%%%%%%%%%%%%%%%%%%%%%%%%%%%%%%
% SEE pgfplots-macros.tex as well!
%\pdfminorversion=5 % to allow compression
%\pdfobjcompresslevel=2
\documentclass[a4paper,openany]{book}

\let\bookmaketitle=\maketitle

% -----------------------------------
% this here is from ltxdoc:
\usepackage{doc}[=v2]
\AtBeginDocument{\MakeShortVerb{\|}}
%\setlength{\textwidth}{355pt}
%\addtolength\marginparwidth{30pt}
%\addtolength\oddsidemargin{20pt}
%\addtolength\evensidemargin{20pt}
\def\cmd#1{\cs{\expandafter\cmd@to@cs\string#1}}
\def\cmd@to@cs#1#2{\char\number`#2\relax}
\DeclareRobustCommand\cs[1]{\texttt{\char`\\#1}}
\providecommand\marg[1]{%
  {\ttfamily\char`\{}\meta{#1}{\ttfamily\char`\}}}
\providecommand\oarg[1]{%
  {\ttfamily[}\meta{#1}{\ttfamily]}}
\providecommand\parg[1]{%
  {\ttfamily(}\meta{#1}{\ttfamily)}}
\raggedbottom

% -----------------------------------
\input{pgfplots.preamble.tex}

\makeatletter
% I want a two-column index just as in pgfmanual styles. This here
% was the best way to get one:
\def\index@prologue{\section*{Index}\addcontentsline{toc}{chapter}{Index}
}
\makeatother

%\RequirePackage[german,english,francais]{babel}

\def\matlabcolormaptext{This colormap is similar to one shipped with Matlab$^\text{\textregistered}$ under a similar name.}

\IfFileExists{tikzlibraryspy.code.tex}{%
\usetikzlibrary{spy}
}{%
    \message{ERROR: tikz SPY library NOT available. The manual will only compile partially.^^J}%
}%

\usepackage{xparse}% for colorbrewer manual

\usetikzlibrary{
    decorations.markings,
    decorations.footprints,
    shapes.arrows,
    matrix,
    positioning,
}

\usepgfplotslibrary{
    fillbetween,
    ternary,
    smithchart,
    patchplots,
    polar,
    colormaps,
    colorbrewer,
    colortol,           % docu not ready yet
}
\pgfqkeys{/codeexample}{%
    every codeexample/.append style={
        /pgfplots/every ternary axis/.append style={
            /pgfplots/legend style={fill=graphicbackground},
        }
    },
    tabsize=4,
}

\pgfplotsmanualenableexternalizationofexpensive

%\usetikzlibrary{external}
%\tikzexternalize[prefix=figures/]{pgfplots}

\title{%
    Manual for Package \PGFPlots{}\\
    {\small 2D/3D Plots in \LaTeX{}, Version \pgfplotsversion}\\
    {\small\url{https://github.com/pgf-tikz/pgfplots}}
    %\\{\small Attention: you are using an unstable development version.}
}

\makeatletter
\long\def\abstractsmuggle{%
    \centering
    \textbf{Abstract}\\[0.5cm]

    \begin{minipage}{12cm}
        \PGFPlots{} draws high-quality function plots in normal or logarithmic
        scaling with a user-friendly interface directly in \TeX{}. The user
        supplies axis labels, legend entries and the plot coordinates for one or
        more plots and \PGFPlots{} applies axis scaling, computes any logarithms
        and axis ticks and draws the plots. It supports line plots, scatter
        plots, piecewise constant plots, bar plots, area plots, mesh and surface
        plots, patch plots, contour plots, quiver plots, histogram plots, box
        plots, polar axes, ternary diagrams, smith charts and some more. It is
        based on Till Tantau's package \PGF{}/\Tikz{}.
    \end{minipage}

}%

\expandafter\date\expandafter{\@date\\[2cm]
    \abstractsmuggle
}%
\makeatother

%\includeonly{pgfplots.reference}


\begin{document}

\def\plotcoords{%
\addplot coordinates {
(5,8.312e-02)    (17,2.547e-02)   (49,7.407e-03)
(129,2.102e-03)  (321,5.874e-04)  (769,1.623e-04)
(1793,4.442e-05) (4097,1.207e-05) (9217,3.261e-06)
};

\addplot coordinates{
(7,8.472e-02)    (31,3.044e-02)    (111,1.022e-02)
(351,3.303e-03)  (1023,1.039e-03)  (2815,3.196e-04)
(7423,9.658e-05) (18943,2.873e-05) (47103,8.437e-06)
};

\addplot coordinates{
(9,7.881e-02)     (49,3.243e-02)    (209,1.232e-02)
(769,4.454e-03)   (2561,1.551e-03)  (7937,5.236e-04)
(23297,1.723e-04) (65537,5.545e-05) (178177,1.751e-05)
};

\addplot coordinates{
(11,6.887e-02)    (71,3.177e-02)     (351,1.341e-02)
(1471,5.334e-03)  (5503,2.027e-03)   (18943,7.415e-04)
(61183,2.628e-04) (187903,9.063e-05) (553983,3.053e-05)
};

\addplot coordinates{
(13,5.755e-02)     (97,2.925e-02)     (545,1.351e-02)
(2561,5.842e-03)   (10625,2.397e-03)  (40193,9.414e-04)
(141569,3.564e-04) (471041,1.308e-04)
(1496065,4.670e-05)
};
}%


\bookmaketitle
\tableofcontents

%%%%%%%%%%%%%%%%%%%%%%%%%%%%%%%%%%%%%%%%%%%%%%%%%%%%%%%%%%%%%%%%%%%%%%%%%%%%%
%
% Package pgfplots.sty documentation.
%
% Copyright 2007/2008 by Christian Feuersaenger.
%
% This program is free software: you can redistribute it and/or modify
% it under the terms of the GNU General Public License as published by
% the Free Software Foundation, either version 3 of the License, or
% (at your option) any later version.
%
% This program is distributed in the hope that it will be useful,
% but WITHOUT ANY WARRANTY; without even the implied warranty of
% MERCHANTABILITY or FITNESS FOR A PARTICULAR PURPOSE.  See the
% GNU General Public License for more details.
%
% You should have received a copy of the GNU General Public License
% along with this program.  If not, see <http://www.gnu.org/licenses/>.
%
%
%%%%%%%%%%%%%%%%%%%%%%%%%%%%%%%%%%%%%%%%%%%%%%%%%%%%%%%%%%%%%%%%%%%%%%%%%%%%%
% SEE pgfplots-macros.tex as well!
%\pdfminorversion=5 % to allow compression
%\pdfobjcompresslevel=2
\documentclass[a4paper,openany]{book}

\let\bookmaketitle=\maketitle

% -----------------------------------
% this here is from ltxdoc:
\usepackage{doc}[=v2]
\AtBeginDocument{\MakeShortVerb{\|}}
%\setlength{\textwidth}{355pt}
%\addtolength\marginparwidth{30pt}
%\addtolength\oddsidemargin{20pt}
%\addtolength\evensidemargin{20pt}
\def\cmd#1{\cs{\expandafter\cmd@to@cs\string#1}}
\def\cmd@to@cs#1#2{\char\number`#2\relax}
\DeclareRobustCommand\cs[1]{\texttt{\char`\\#1}}
\providecommand\marg[1]{%
  {\ttfamily\char`\{}\meta{#1}{\ttfamily\char`\}}}
\providecommand\oarg[1]{%
  {\ttfamily[}\meta{#1}{\ttfamily]}}
\providecommand\parg[1]{%
  {\ttfamily(}\meta{#1}{\ttfamily)}}
\raggedbottom

% -----------------------------------
\input{pgfplots.preamble.tex}

\makeatletter
% I want a two-column index just as in pgfmanual styles. This here
% was the best way to get one:
\def\index@prologue{\section*{Index}\addcontentsline{toc}{chapter}{Index}
}
\makeatother

%\RequirePackage[german,english,francais]{babel}

\def\matlabcolormaptext{This colormap is similar to one shipped with Matlab$^\text{\textregistered}$ under a similar name.}

\IfFileExists{tikzlibraryspy.code.tex}{%
\usetikzlibrary{spy}
}{%
    \message{ERROR: tikz SPY library NOT available. The manual will only compile partially.^^J}%
}%

\usepackage{xparse}% for colorbrewer manual

\usetikzlibrary{
    decorations.markings,
    decorations.footprints,
    shapes.arrows,
    matrix,
    positioning,
}

\usepgfplotslibrary{
    fillbetween,
    ternary,
    smithchart,
    patchplots,
    polar,
    colormaps,
    colorbrewer,
    colortol,           % docu not ready yet
}
\pgfqkeys{/codeexample}{%
    every codeexample/.append style={
        /pgfplots/every ternary axis/.append style={
            /pgfplots/legend style={fill=graphicbackground},
        }
    },
    tabsize=4,
}

\pgfplotsmanualenableexternalizationofexpensive

%\usetikzlibrary{external}
%\tikzexternalize[prefix=figures/]{pgfplots}

\title{%
    Manual for Package \PGFPlots{}\\
    {\small 2D/3D Plots in \LaTeX{}, Version \pgfplotsversion}\\
    {\small\url{https://github.com/pgf-tikz/pgfplots}}
    %\\{\small Attention: you are using an unstable development version.}
}

\makeatletter
\long\def\abstractsmuggle{%
    \centering
    \textbf{Abstract}\\[0.5cm]

    \begin{minipage}{12cm}
        \PGFPlots{} draws high-quality function plots in normal or logarithmic
        scaling with a user-friendly interface directly in \TeX{}. The user
        supplies axis labels, legend entries and the plot coordinates for one or
        more plots and \PGFPlots{} applies axis scaling, computes any logarithms
        and axis ticks and draws the plots. It supports line plots, scatter
        plots, piecewise constant plots, bar plots, area plots, mesh and surface
        plots, patch plots, contour plots, quiver plots, histogram plots, box
        plots, polar axes, ternary diagrams, smith charts and some more. It is
        based on Till Tantau's package \PGF{}/\Tikz{}.
    \end{minipage}

}%

\expandafter\date\expandafter{\@date\\[2cm]
    \abstractsmuggle
}%
\makeatother

%\includeonly{pgfplots.reference}


\begin{document}

\def\plotcoords{%
\addplot coordinates {
(5,8.312e-02)    (17,2.547e-02)   (49,7.407e-03)
(129,2.102e-03)  (321,5.874e-04)  (769,1.623e-04)
(1793,4.442e-05) (4097,1.207e-05) (9217,3.261e-06)
};

\addplot coordinates{
(7,8.472e-02)    (31,3.044e-02)    (111,1.022e-02)
(351,3.303e-03)  (1023,1.039e-03)  (2815,3.196e-04)
(7423,9.658e-05) (18943,2.873e-05) (47103,8.437e-06)
};

\addplot coordinates{
(9,7.881e-02)     (49,3.243e-02)    (209,1.232e-02)
(769,4.454e-03)   (2561,1.551e-03)  (7937,5.236e-04)
(23297,1.723e-04) (65537,5.545e-05) (178177,1.751e-05)
};

\addplot coordinates{
(11,6.887e-02)    (71,3.177e-02)     (351,1.341e-02)
(1471,5.334e-03)  (5503,2.027e-03)   (18943,7.415e-04)
(61183,2.628e-04) (187903,9.063e-05) (553983,3.053e-05)
};

\addplot coordinates{
(13,5.755e-02)     (97,2.925e-02)     (545,1.351e-02)
(2561,5.842e-03)   (10625,2.397e-03)  (40193,9.414e-04)
(141569,3.564e-04) (471041,1.308e-04)
(1496065,4.670e-05)
};
}%


\bookmaketitle
\tableofcontents
\include{pgfplots.title_abstract_intro}
\include{pgfplots.preliminaries}
\include{pgfplots.intro}
\include{pgfplots.reference}
\include{pgfplots.libs}
\include{pgfplots.resources}
\include{pgfplots.importexport}
\include{pgfplots.basic.reference}

\printindex

\bibliographystyle{abbrv} %gerapali} %gerabbrv} %gerunsrt.bst} %gerabbrv}% gerplain}
\nocite{pgfplotstable}
\nocite{programmingnotes}
\bibliography{pgfplots}
\end{document}

% -----------------------------------------------------------------------------
% For Stefan Pinnow as reminder on what to look for when editing the manual
% -----------------------------------------------------------------------------
% There should be no line breaks in the following environments
% - |...|
% - \declareandlabel{...}
% - \verbpdfref{...}
% If "MakeTikzPictures" isn't running through check one of the externalized
% LOG files what is the cause of that.
% -----------------------------------------------------------------------------


%%%%%%%%%%%%%%%%%%%%%%%%%%%%%%%%%%%%%%%%%%%%%%%%%%%%%%%%%%%%%%%%%%%%%%%%%%%%%
%
% Package pgfplots.sty documentation.
%
% Copyright 2007/2008 by Christian Feuersaenger.
%
% This program is free software: you can redistribute it and/or modify
% it under the terms of the GNU General Public License as published by
% the Free Software Foundation, either version 3 of the License, or
% (at your option) any later version.
%
% This program is distributed in the hope that it will be useful,
% but WITHOUT ANY WARRANTY; without even the implied warranty of
% MERCHANTABILITY or FITNESS FOR A PARTICULAR PURPOSE.  See the
% GNU General Public License for more details.
%
% You should have received a copy of the GNU General Public License
% along with this program.  If not, see <http://www.gnu.org/licenses/>.
%
%
%%%%%%%%%%%%%%%%%%%%%%%%%%%%%%%%%%%%%%%%%%%%%%%%%%%%%%%%%%%%%%%%%%%%%%%%%%%%%
% SEE pgfplots-macros.tex as well!
%\pdfminorversion=5 % to allow compression
%\pdfobjcompresslevel=2
\documentclass[a4paper,openany]{book}

\let\bookmaketitle=\maketitle

% -----------------------------------
% this here is from ltxdoc:
\usepackage{doc}[=v2]
\AtBeginDocument{\MakeShortVerb{\|}}
%\setlength{\textwidth}{355pt}
%\addtolength\marginparwidth{30pt}
%\addtolength\oddsidemargin{20pt}
%\addtolength\evensidemargin{20pt}
\def\cmd#1{\cs{\expandafter\cmd@to@cs\string#1}}
\def\cmd@to@cs#1#2{\char\number`#2\relax}
\DeclareRobustCommand\cs[1]{\texttt{\char`\\#1}}
\providecommand\marg[1]{%
  {\ttfamily\char`\{}\meta{#1}{\ttfamily\char`\}}}
\providecommand\oarg[1]{%
  {\ttfamily[}\meta{#1}{\ttfamily]}}
\providecommand\parg[1]{%
  {\ttfamily(}\meta{#1}{\ttfamily)}}
\raggedbottom

% -----------------------------------
\input{pgfplots.preamble.tex}

\makeatletter
% I want a two-column index just as in pgfmanual styles. This here
% was the best way to get one:
\def\index@prologue{\section*{Index}\addcontentsline{toc}{chapter}{Index}
}
\makeatother

%\RequirePackage[german,english,francais]{babel}

\def\matlabcolormaptext{This colormap is similar to one shipped with Matlab$^\text{\textregistered}$ under a similar name.}

\IfFileExists{tikzlibraryspy.code.tex}{%
\usetikzlibrary{spy}
}{%
    \message{ERROR: tikz SPY library NOT available. The manual will only compile partially.^^J}%
}%

\usepackage{xparse}% for colorbrewer manual

\usetikzlibrary{
    decorations.markings,
    decorations.footprints,
    shapes.arrows,
    matrix,
    positioning,
}

\usepgfplotslibrary{
    fillbetween,
    ternary,
    smithchart,
    patchplots,
    polar,
    colormaps,
    colorbrewer,
    colortol,           % docu not ready yet
}
\pgfqkeys{/codeexample}{%
    every codeexample/.append style={
        /pgfplots/every ternary axis/.append style={
            /pgfplots/legend style={fill=graphicbackground},
        }
    },
    tabsize=4,
}

\pgfplotsmanualenableexternalizationofexpensive

%\usetikzlibrary{external}
%\tikzexternalize[prefix=figures/]{pgfplots}

\title{%
    Manual for Package \PGFPlots{}\\
    {\small 2D/3D Plots in \LaTeX{}, Version \pgfplotsversion}\\
    {\small\url{https://github.com/pgf-tikz/pgfplots}}
    %\\{\small Attention: you are using an unstable development version.}
}

\makeatletter
\long\def\abstractsmuggle{%
    \centering
    \textbf{Abstract}\\[0.5cm]

    \begin{minipage}{12cm}
        \PGFPlots{} draws high-quality function plots in normal or logarithmic
        scaling with a user-friendly interface directly in \TeX{}. The user
        supplies axis labels, legend entries and the plot coordinates for one or
        more plots and \PGFPlots{} applies axis scaling, computes any logarithms
        and axis ticks and draws the plots. It supports line plots, scatter
        plots, piecewise constant plots, bar plots, area plots, mesh and surface
        plots, patch plots, contour plots, quiver plots, histogram plots, box
        plots, polar axes, ternary diagrams, smith charts and some more. It is
        based on Till Tantau's package \PGF{}/\Tikz{}.
    \end{minipage}

}%

\expandafter\date\expandafter{\@date\\[2cm]
    \abstractsmuggle
}%
\makeatother

%\includeonly{pgfplots.reference}


\begin{document}

\def\plotcoords{%
\addplot coordinates {
(5,8.312e-02)    (17,2.547e-02)   (49,7.407e-03)
(129,2.102e-03)  (321,5.874e-04)  (769,1.623e-04)
(1793,4.442e-05) (4097,1.207e-05) (9217,3.261e-06)
};

\addplot coordinates{
(7,8.472e-02)    (31,3.044e-02)    (111,1.022e-02)
(351,3.303e-03)  (1023,1.039e-03)  (2815,3.196e-04)
(7423,9.658e-05) (18943,2.873e-05) (47103,8.437e-06)
};

\addplot coordinates{
(9,7.881e-02)     (49,3.243e-02)    (209,1.232e-02)
(769,4.454e-03)   (2561,1.551e-03)  (7937,5.236e-04)
(23297,1.723e-04) (65537,5.545e-05) (178177,1.751e-05)
};

\addplot coordinates{
(11,6.887e-02)    (71,3.177e-02)     (351,1.341e-02)
(1471,5.334e-03)  (5503,2.027e-03)   (18943,7.415e-04)
(61183,2.628e-04) (187903,9.063e-05) (553983,3.053e-05)
};

\addplot coordinates{
(13,5.755e-02)     (97,2.925e-02)     (545,1.351e-02)
(2561,5.842e-03)   (10625,2.397e-03)  (40193,9.414e-04)
(141569,3.564e-04) (471041,1.308e-04)
(1496065,4.670e-05)
};
}%


\bookmaketitle
\tableofcontents
\include{pgfplots.title_abstract_intro}
\include{pgfplots.preliminaries}
\include{pgfplots.intro}
\include{pgfplots.reference}
\include{pgfplots.libs}
\include{pgfplots.resources}
\include{pgfplots.importexport}
\include{pgfplots.basic.reference}

\printindex

\bibliographystyle{abbrv} %gerapali} %gerabbrv} %gerunsrt.bst} %gerabbrv}% gerplain}
\nocite{pgfplotstable}
\nocite{programmingnotes}
\bibliography{pgfplots}
\end{document}

% -----------------------------------------------------------------------------
% For Stefan Pinnow as reminder on what to look for when editing the manual
% -----------------------------------------------------------------------------
% There should be no line breaks in the following environments
% - |...|
% - \declareandlabel{...}
% - \verbpdfref{...}
% If "MakeTikzPictures" isn't running through check one of the externalized
% LOG files what is the cause of that.
% -----------------------------------------------------------------------------


%%%%%%%%%%%%%%%%%%%%%%%%%%%%%%%%%%%%%%%%%%%%%%%%%%%%%%%%%%%%%%%%%%%%%%%%%%%%%
%
% Package pgfplots.sty documentation.
%
% Copyright 2007/2008 by Christian Feuersaenger.
%
% This program is free software: you can redistribute it and/or modify
% it under the terms of the GNU General Public License as published by
% the Free Software Foundation, either version 3 of the License, or
% (at your option) any later version.
%
% This program is distributed in the hope that it will be useful,
% but WITHOUT ANY WARRANTY; without even the implied warranty of
% MERCHANTABILITY or FITNESS FOR A PARTICULAR PURPOSE.  See the
% GNU General Public License for more details.
%
% You should have received a copy of the GNU General Public License
% along with this program.  If not, see <http://www.gnu.org/licenses/>.
%
%
%%%%%%%%%%%%%%%%%%%%%%%%%%%%%%%%%%%%%%%%%%%%%%%%%%%%%%%%%%%%%%%%%%%%%%%%%%%%%
% SEE pgfplots-macros.tex as well!
%\pdfminorversion=5 % to allow compression
%\pdfobjcompresslevel=2
\documentclass[a4paper,openany]{book}

\let\bookmaketitle=\maketitle

% -----------------------------------
% this here is from ltxdoc:
\usepackage{doc}[=v2]
\AtBeginDocument{\MakeShortVerb{\|}}
%\setlength{\textwidth}{355pt}
%\addtolength\marginparwidth{30pt}
%\addtolength\oddsidemargin{20pt}
%\addtolength\evensidemargin{20pt}
\def\cmd#1{\cs{\expandafter\cmd@to@cs\string#1}}
\def\cmd@to@cs#1#2{\char\number`#2\relax}
\DeclareRobustCommand\cs[1]{\texttt{\char`\\#1}}
\providecommand\marg[1]{%
  {\ttfamily\char`\{}\meta{#1}{\ttfamily\char`\}}}
\providecommand\oarg[1]{%
  {\ttfamily[}\meta{#1}{\ttfamily]}}
\providecommand\parg[1]{%
  {\ttfamily(}\meta{#1}{\ttfamily)}}
\raggedbottom

% -----------------------------------
\input{pgfplots.preamble.tex}

\makeatletter
% I want a two-column index just as in pgfmanual styles. This here
% was the best way to get one:
\def\index@prologue{\section*{Index}\addcontentsline{toc}{chapter}{Index}
}
\makeatother

%\RequirePackage[german,english,francais]{babel}

\def\matlabcolormaptext{This colormap is similar to one shipped with Matlab$^\text{\textregistered}$ under a similar name.}

\IfFileExists{tikzlibraryspy.code.tex}{%
\usetikzlibrary{spy}
}{%
    \message{ERROR: tikz SPY library NOT available. The manual will only compile partially.^^J}%
}%

\usepackage{xparse}% for colorbrewer manual

\usetikzlibrary{
    decorations.markings,
    decorations.footprints,
    shapes.arrows,
    matrix,
    positioning,
}

\usepgfplotslibrary{
    fillbetween,
    ternary,
    smithchart,
    patchplots,
    polar,
    colormaps,
    colorbrewer,
    colortol,           % docu not ready yet
}
\pgfqkeys{/codeexample}{%
    every codeexample/.append style={
        /pgfplots/every ternary axis/.append style={
            /pgfplots/legend style={fill=graphicbackground},
        }
    },
    tabsize=4,
}

\pgfplotsmanualenableexternalizationofexpensive

%\usetikzlibrary{external}
%\tikzexternalize[prefix=figures/]{pgfplots}

\title{%
    Manual for Package \PGFPlots{}\\
    {\small 2D/3D Plots in \LaTeX{}, Version \pgfplotsversion}\\
    {\small\url{https://github.com/pgf-tikz/pgfplots}}
    %\\{\small Attention: you are using an unstable development version.}
}

\makeatletter
\long\def\abstractsmuggle{%
    \centering
    \textbf{Abstract}\\[0.5cm]

    \begin{minipage}{12cm}
        \PGFPlots{} draws high-quality function plots in normal or logarithmic
        scaling with a user-friendly interface directly in \TeX{}. The user
        supplies axis labels, legend entries and the plot coordinates for one or
        more plots and \PGFPlots{} applies axis scaling, computes any logarithms
        and axis ticks and draws the plots. It supports line plots, scatter
        plots, piecewise constant plots, bar plots, area plots, mesh and surface
        plots, patch plots, contour plots, quiver plots, histogram plots, box
        plots, polar axes, ternary diagrams, smith charts and some more. It is
        based on Till Tantau's package \PGF{}/\Tikz{}.
    \end{minipage}

}%

\expandafter\date\expandafter{\@date\\[2cm]
    \abstractsmuggle
}%
\makeatother

%\includeonly{pgfplots.reference}


\begin{document}

\def\plotcoords{%
\addplot coordinates {
(5,8.312e-02)    (17,2.547e-02)   (49,7.407e-03)
(129,2.102e-03)  (321,5.874e-04)  (769,1.623e-04)
(1793,4.442e-05) (4097,1.207e-05) (9217,3.261e-06)
};

\addplot coordinates{
(7,8.472e-02)    (31,3.044e-02)    (111,1.022e-02)
(351,3.303e-03)  (1023,1.039e-03)  (2815,3.196e-04)
(7423,9.658e-05) (18943,2.873e-05) (47103,8.437e-06)
};

\addplot coordinates{
(9,7.881e-02)     (49,3.243e-02)    (209,1.232e-02)
(769,4.454e-03)   (2561,1.551e-03)  (7937,5.236e-04)
(23297,1.723e-04) (65537,5.545e-05) (178177,1.751e-05)
};

\addplot coordinates{
(11,6.887e-02)    (71,3.177e-02)     (351,1.341e-02)
(1471,5.334e-03)  (5503,2.027e-03)   (18943,7.415e-04)
(61183,2.628e-04) (187903,9.063e-05) (553983,3.053e-05)
};

\addplot coordinates{
(13,5.755e-02)     (97,2.925e-02)     (545,1.351e-02)
(2561,5.842e-03)   (10625,2.397e-03)  (40193,9.414e-04)
(141569,3.564e-04) (471041,1.308e-04)
(1496065,4.670e-05)
};
}%


\bookmaketitle
\tableofcontents
\include{pgfplots.title_abstract_intro}
\include{pgfplots.preliminaries}
\include{pgfplots.intro}
\include{pgfplots.reference}
\include{pgfplots.libs}
\include{pgfplots.resources}
\include{pgfplots.importexport}
\include{pgfplots.basic.reference}

\printindex

\bibliographystyle{abbrv} %gerapali} %gerabbrv} %gerunsrt.bst} %gerabbrv}% gerplain}
\nocite{pgfplotstable}
\nocite{programmingnotes}
\bibliography{pgfplots}
\end{document}

% -----------------------------------------------------------------------------
% For Stefan Pinnow as reminder on what to look for when editing the manual
% -----------------------------------------------------------------------------
% There should be no line breaks in the following environments
% - |...|
% - \declareandlabel{...}
% - \verbpdfref{...}
% If "MakeTikzPictures" isn't running through check one of the externalized
% LOG files what is the cause of that.
% -----------------------------------------------------------------------------


%%%%%%%%%%%%%%%%%%%%%%%%%%%%%%%%%%%%%%%%%%%%%%%%%%%%%%%%%%%%%%%%%%%%%%%%%%%%%
%
% Package pgfplots.sty documentation.
%
% Copyright 2007/2008 by Christian Feuersaenger.
%
% This program is free software: you can redistribute it and/or modify
% it under the terms of the GNU General Public License as published by
% the Free Software Foundation, either version 3 of the License, or
% (at your option) any later version.
%
% This program is distributed in the hope that it will be useful,
% but WITHOUT ANY WARRANTY; without even the implied warranty of
% MERCHANTABILITY or FITNESS FOR A PARTICULAR PURPOSE.  See the
% GNU General Public License for more details.
%
% You should have received a copy of the GNU General Public License
% along with this program.  If not, see <http://www.gnu.org/licenses/>.
%
%
%%%%%%%%%%%%%%%%%%%%%%%%%%%%%%%%%%%%%%%%%%%%%%%%%%%%%%%%%%%%%%%%%%%%%%%%%%%%%
% SEE pgfplots-macros.tex as well!
%\pdfminorversion=5 % to allow compression
%\pdfobjcompresslevel=2
\documentclass[a4paper,openany]{book}

\let\bookmaketitle=\maketitle

% -----------------------------------
% this here is from ltxdoc:
\usepackage{doc}[=v2]
\AtBeginDocument{\MakeShortVerb{\|}}
%\setlength{\textwidth}{355pt}
%\addtolength\marginparwidth{30pt}
%\addtolength\oddsidemargin{20pt}
%\addtolength\evensidemargin{20pt}
\def\cmd#1{\cs{\expandafter\cmd@to@cs\string#1}}
\def\cmd@to@cs#1#2{\char\number`#2\relax}
\DeclareRobustCommand\cs[1]{\texttt{\char`\\#1}}
\providecommand\marg[1]{%
  {\ttfamily\char`\{}\meta{#1}{\ttfamily\char`\}}}
\providecommand\oarg[1]{%
  {\ttfamily[}\meta{#1}{\ttfamily]}}
\providecommand\parg[1]{%
  {\ttfamily(}\meta{#1}{\ttfamily)}}
\raggedbottom

% -----------------------------------
\input{pgfplots.preamble.tex}

\makeatletter
% I want a two-column index just as in pgfmanual styles. This here
% was the best way to get one:
\def\index@prologue{\section*{Index}\addcontentsline{toc}{chapter}{Index}
}
\makeatother

%\RequirePackage[german,english,francais]{babel}

\def\matlabcolormaptext{This colormap is similar to one shipped with Matlab$^\text{\textregistered}$ under a similar name.}

\IfFileExists{tikzlibraryspy.code.tex}{%
\usetikzlibrary{spy}
}{%
    \message{ERROR: tikz SPY library NOT available. The manual will only compile partially.^^J}%
}%

\usepackage{xparse}% for colorbrewer manual

\usetikzlibrary{
    decorations.markings,
    decorations.footprints,
    shapes.arrows,
    matrix,
    positioning,
}

\usepgfplotslibrary{
    fillbetween,
    ternary,
    smithchart,
    patchplots,
    polar,
    colormaps,
    colorbrewer,
    colortol,           % docu not ready yet
}
\pgfqkeys{/codeexample}{%
    every codeexample/.append style={
        /pgfplots/every ternary axis/.append style={
            /pgfplots/legend style={fill=graphicbackground},
        }
    },
    tabsize=4,
}

\pgfplotsmanualenableexternalizationofexpensive

%\usetikzlibrary{external}
%\tikzexternalize[prefix=figures/]{pgfplots}

\title{%
    Manual for Package \PGFPlots{}\\
    {\small 2D/3D Plots in \LaTeX{}, Version \pgfplotsversion}\\
    {\small\url{https://github.com/pgf-tikz/pgfplots}}
    %\\{\small Attention: you are using an unstable development version.}
}

\makeatletter
\long\def\abstractsmuggle{%
    \centering
    \textbf{Abstract}\\[0.5cm]

    \begin{minipage}{12cm}
        \PGFPlots{} draws high-quality function plots in normal or logarithmic
        scaling with a user-friendly interface directly in \TeX{}. The user
        supplies axis labels, legend entries and the plot coordinates for one or
        more plots and \PGFPlots{} applies axis scaling, computes any logarithms
        and axis ticks and draws the plots. It supports line plots, scatter
        plots, piecewise constant plots, bar plots, area plots, mesh and surface
        plots, patch plots, contour plots, quiver plots, histogram plots, box
        plots, polar axes, ternary diagrams, smith charts and some more. It is
        based on Till Tantau's package \PGF{}/\Tikz{}.
    \end{minipage}

}%

\expandafter\date\expandafter{\@date\\[2cm]
    \abstractsmuggle
}%
\makeatother

%\includeonly{pgfplots.reference}


\begin{document}

\def\plotcoords{%
\addplot coordinates {
(5,8.312e-02)    (17,2.547e-02)   (49,7.407e-03)
(129,2.102e-03)  (321,5.874e-04)  (769,1.623e-04)
(1793,4.442e-05) (4097,1.207e-05) (9217,3.261e-06)
};

\addplot coordinates{
(7,8.472e-02)    (31,3.044e-02)    (111,1.022e-02)
(351,3.303e-03)  (1023,1.039e-03)  (2815,3.196e-04)
(7423,9.658e-05) (18943,2.873e-05) (47103,8.437e-06)
};

\addplot coordinates{
(9,7.881e-02)     (49,3.243e-02)    (209,1.232e-02)
(769,4.454e-03)   (2561,1.551e-03)  (7937,5.236e-04)
(23297,1.723e-04) (65537,5.545e-05) (178177,1.751e-05)
};

\addplot coordinates{
(11,6.887e-02)    (71,3.177e-02)     (351,1.341e-02)
(1471,5.334e-03)  (5503,2.027e-03)   (18943,7.415e-04)
(61183,2.628e-04) (187903,9.063e-05) (553983,3.053e-05)
};

\addplot coordinates{
(13,5.755e-02)     (97,2.925e-02)     (545,1.351e-02)
(2561,5.842e-03)   (10625,2.397e-03)  (40193,9.414e-04)
(141569,3.564e-04) (471041,1.308e-04)
(1496065,4.670e-05)
};
}%


\bookmaketitle
\tableofcontents
\include{pgfplots.title_abstract_intro}
\include{pgfplots.preliminaries}
\include{pgfplots.intro}
\include{pgfplots.reference}
\include{pgfplots.libs}
\include{pgfplots.resources}
\include{pgfplots.importexport}
\include{pgfplots.basic.reference}

\printindex

\bibliographystyle{abbrv} %gerapali} %gerabbrv} %gerunsrt.bst} %gerabbrv}% gerplain}
\nocite{pgfplotstable}
\nocite{programmingnotes}
\bibliography{pgfplots}
\end{document}

% -----------------------------------------------------------------------------
% For Stefan Pinnow as reminder on what to look for when editing the manual
% -----------------------------------------------------------------------------
% There should be no line breaks in the following environments
% - |...|
% - \declareandlabel{...}
% - \verbpdfref{...}
% If "MakeTikzPictures" isn't running through check one of the externalized
% LOG files what is the cause of that.
% -----------------------------------------------------------------------------


%%%%%%%%%%%%%%%%%%%%%%%%%%%%%%%%%%%%%%%%%%%%%%%%%%%%%%%%%%%%%%%%%%%%%%%%%%%%%
%
% Package pgfplots.sty documentation.
%
% Copyright 2007/2008 by Christian Feuersaenger.
%
% This program is free software: you can redistribute it and/or modify
% it under the terms of the GNU General Public License as published by
% the Free Software Foundation, either version 3 of the License, or
% (at your option) any later version.
%
% This program is distributed in the hope that it will be useful,
% but WITHOUT ANY WARRANTY; without even the implied warranty of
% MERCHANTABILITY or FITNESS FOR A PARTICULAR PURPOSE.  See the
% GNU General Public License for more details.
%
% You should have received a copy of the GNU General Public License
% along with this program.  If not, see <http://www.gnu.org/licenses/>.
%
%
%%%%%%%%%%%%%%%%%%%%%%%%%%%%%%%%%%%%%%%%%%%%%%%%%%%%%%%%%%%%%%%%%%%%%%%%%%%%%
% SEE pgfplots-macros.tex as well!
%\pdfminorversion=5 % to allow compression
%\pdfobjcompresslevel=2
\documentclass[a4paper,openany]{book}

\let\bookmaketitle=\maketitle

% -----------------------------------
% this here is from ltxdoc:
\usepackage{doc}[=v2]
\AtBeginDocument{\MakeShortVerb{\|}}
%\setlength{\textwidth}{355pt}
%\addtolength\marginparwidth{30pt}
%\addtolength\oddsidemargin{20pt}
%\addtolength\evensidemargin{20pt}
\def\cmd#1{\cs{\expandafter\cmd@to@cs\string#1}}
\def\cmd@to@cs#1#2{\char\number`#2\relax}
\DeclareRobustCommand\cs[1]{\texttt{\char`\\#1}}
\providecommand\marg[1]{%
  {\ttfamily\char`\{}\meta{#1}{\ttfamily\char`\}}}
\providecommand\oarg[1]{%
  {\ttfamily[}\meta{#1}{\ttfamily]}}
\providecommand\parg[1]{%
  {\ttfamily(}\meta{#1}{\ttfamily)}}
\raggedbottom

% -----------------------------------
\input{pgfplots.preamble.tex}

\makeatletter
% I want a two-column index just as in pgfmanual styles. This here
% was the best way to get one:
\def\index@prologue{\section*{Index}\addcontentsline{toc}{chapter}{Index}
}
\makeatother

%\RequirePackage[german,english,francais]{babel}

\def\matlabcolormaptext{This colormap is similar to one shipped with Matlab$^\text{\textregistered}$ under a similar name.}

\IfFileExists{tikzlibraryspy.code.tex}{%
\usetikzlibrary{spy}
}{%
    \message{ERROR: tikz SPY library NOT available. The manual will only compile partially.^^J}%
}%

\usepackage{xparse}% for colorbrewer manual

\usetikzlibrary{
    decorations.markings,
    decorations.footprints,
    shapes.arrows,
    matrix,
    positioning,
}

\usepgfplotslibrary{
    fillbetween,
    ternary,
    smithchart,
    patchplots,
    polar,
    colormaps,
    colorbrewer,
    colortol,           % docu not ready yet
}
\pgfqkeys{/codeexample}{%
    every codeexample/.append style={
        /pgfplots/every ternary axis/.append style={
            /pgfplots/legend style={fill=graphicbackground},
        }
    },
    tabsize=4,
}

\pgfplotsmanualenableexternalizationofexpensive

%\usetikzlibrary{external}
%\tikzexternalize[prefix=figures/]{pgfplots}

\title{%
    Manual for Package \PGFPlots{}\\
    {\small 2D/3D Plots in \LaTeX{}, Version \pgfplotsversion}\\
    {\small\url{https://github.com/pgf-tikz/pgfplots}}
    %\\{\small Attention: you are using an unstable development version.}
}

\makeatletter
\long\def\abstractsmuggle{%
    \centering
    \textbf{Abstract}\\[0.5cm]

    \begin{minipage}{12cm}
        \PGFPlots{} draws high-quality function plots in normal or logarithmic
        scaling with a user-friendly interface directly in \TeX{}. The user
        supplies axis labels, legend entries and the plot coordinates for one or
        more plots and \PGFPlots{} applies axis scaling, computes any logarithms
        and axis ticks and draws the plots. It supports line plots, scatter
        plots, piecewise constant plots, bar plots, area plots, mesh and surface
        plots, patch plots, contour plots, quiver plots, histogram plots, box
        plots, polar axes, ternary diagrams, smith charts and some more. It is
        based on Till Tantau's package \PGF{}/\Tikz{}.
    \end{minipage}

}%

\expandafter\date\expandafter{\@date\\[2cm]
    \abstractsmuggle
}%
\makeatother

%\includeonly{pgfplots.reference}


\begin{document}

\def\plotcoords{%
\addplot coordinates {
(5,8.312e-02)    (17,2.547e-02)   (49,7.407e-03)
(129,2.102e-03)  (321,5.874e-04)  (769,1.623e-04)
(1793,4.442e-05) (4097,1.207e-05) (9217,3.261e-06)
};

\addplot coordinates{
(7,8.472e-02)    (31,3.044e-02)    (111,1.022e-02)
(351,3.303e-03)  (1023,1.039e-03)  (2815,3.196e-04)
(7423,9.658e-05) (18943,2.873e-05) (47103,8.437e-06)
};

\addplot coordinates{
(9,7.881e-02)     (49,3.243e-02)    (209,1.232e-02)
(769,4.454e-03)   (2561,1.551e-03)  (7937,5.236e-04)
(23297,1.723e-04) (65537,5.545e-05) (178177,1.751e-05)
};

\addplot coordinates{
(11,6.887e-02)    (71,3.177e-02)     (351,1.341e-02)
(1471,5.334e-03)  (5503,2.027e-03)   (18943,7.415e-04)
(61183,2.628e-04) (187903,9.063e-05) (553983,3.053e-05)
};

\addplot coordinates{
(13,5.755e-02)     (97,2.925e-02)     (545,1.351e-02)
(2561,5.842e-03)   (10625,2.397e-03)  (40193,9.414e-04)
(141569,3.564e-04) (471041,1.308e-04)
(1496065,4.670e-05)
};
}%


\bookmaketitle
\tableofcontents
\include{pgfplots.title_abstract_intro}
\include{pgfplots.preliminaries}
\include{pgfplots.intro}
\include{pgfplots.reference}
\include{pgfplots.libs}
\include{pgfplots.resources}
\include{pgfplots.importexport}
\include{pgfplots.basic.reference}

\printindex

\bibliographystyle{abbrv} %gerapali} %gerabbrv} %gerunsrt.bst} %gerabbrv}% gerplain}
\nocite{pgfplotstable}
\nocite{programmingnotes}
\bibliography{pgfplots}
\end{document}

% -----------------------------------------------------------------------------
% For Stefan Pinnow as reminder on what to look for when editing the manual
% -----------------------------------------------------------------------------
% There should be no line breaks in the following environments
% - |...|
% - \declareandlabel{...}
% - \verbpdfref{...}
% If "MakeTikzPictures" isn't running through check one of the externalized
% LOG files what is the cause of that.
% -----------------------------------------------------------------------------


%%%%%%%%%%%%%%%%%%%%%%%%%%%%%%%%%%%%%%%%%%%%%%%%%%%%%%%%%%%%%%%%%%%%%%%%%%%%%
%
% Package pgfplots.sty documentation.
%
% Copyright 2007/2008 by Christian Feuersaenger.
%
% This program is free software: you can redistribute it and/or modify
% it under the terms of the GNU General Public License as published by
% the Free Software Foundation, either version 3 of the License, or
% (at your option) any later version.
%
% This program is distributed in the hope that it will be useful,
% but WITHOUT ANY WARRANTY; without even the implied warranty of
% MERCHANTABILITY or FITNESS FOR A PARTICULAR PURPOSE.  See the
% GNU General Public License for more details.
%
% You should have received a copy of the GNU General Public License
% along with this program.  If not, see <http://www.gnu.org/licenses/>.
%
%
%%%%%%%%%%%%%%%%%%%%%%%%%%%%%%%%%%%%%%%%%%%%%%%%%%%%%%%%%%%%%%%%%%%%%%%%%%%%%
% SEE pgfplots-macros.tex as well!
%\pdfminorversion=5 % to allow compression
%\pdfobjcompresslevel=2
\documentclass[a4paper,openany]{book}

\let\bookmaketitle=\maketitle

% -----------------------------------
% this here is from ltxdoc:
\usepackage{doc}[=v2]
\AtBeginDocument{\MakeShortVerb{\|}}
%\setlength{\textwidth}{355pt}
%\addtolength\marginparwidth{30pt}
%\addtolength\oddsidemargin{20pt}
%\addtolength\evensidemargin{20pt}
\def\cmd#1{\cs{\expandafter\cmd@to@cs\string#1}}
\def\cmd@to@cs#1#2{\char\number`#2\relax}
\DeclareRobustCommand\cs[1]{\texttt{\char`\\#1}}
\providecommand\marg[1]{%
  {\ttfamily\char`\{}\meta{#1}{\ttfamily\char`\}}}
\providecommand\oarg[1]{%
  {\ttfamily[}\meta{#1}{\ttfamily]}}
\providecommand\parg[1]{%
  {\ttfamily(}\meta{#1}{\ttfamily)}}
\raggedbottom

% -----------------------------------
\input{pgfplots.preamble.tex}

\makeatletter
% I want a two-column index just as in pgfmanual styles. This here
% was the best way to get one:
\def\index@prologue{\section*{Index}\addcontentsline{toc}{chapter}{Index}
}
\makeatother

%\RequirePackage[german,english,francais]{babel}

\def\matlabcolormaptext{This colormap is similar to one shipped with Matlab$^\text{\textregistered}$ under a similar name.}

\IfFileExists{tikzlibraryspy.code.tex}{%
\usetikzlibrary{spy}
}{%
    \message{ERROR: tikz SPY library NOT available. The manual will only compile partially.^^J}%
}%

\usepackage{xparse}% for colorbrewer manual

\usetikzlibrary{
    decorations.markings,
    decorations.footprints,
    shapes.arrows,
    matrix,
    positioning,
}

\usepgfplotslibrary{
    fillbetween,
    ternary,
    smithchart,
    patchplots,
    polar,
    colormaps,
    colorbrewer,
    colortol,           % docu not ready yet
}
\pgfqkeys{/codeexample}{%
    every codeexample/.append style={
        /pgfplots/every ternary axis/.append style={
            /pgfplots/legend style={fill=graphicbackground},
        }
    },
    tabsize=4,
}

\pgfplotsmanualenableexternalizationofexpensive

%\usetikzlibrary{external}
%\tikzexternalize[prefix=figures/]{pgfplots}

\title{%
    Manual for Package \PGFPlots{}\\
    {\small 2D/3D Plots in \LaTeX{}, Version \pgfplotsversion}\\
    {\small\url{https://github.com/pgf-tikz/pgfplots}}
    %\\{\small Attention: you are using an unstable development version.}
}

\makeatletter
\long\def\abstractsmuggle{%
    \centering
    \textbf{Abstract}\\[0.5cm]

    \begin{minipage}{12cm}
        \PGFPlots{} draws high-quality function plots in normal or logarithmic
        scaling with a user-friendly interface directly in \TeX{}. The user
        supplies axis labels, legend entries and the plot coordinates for one or
        more plots and \PGFPlots{} applies axis scaling, computes any logarithms
        and axis ticks and draws the plots. It supports line plots, scatter
        plots, piecewise constant plots, bar plots, area plots, mesh and surface
        plots, patch plots, contour plots, quiver plots, histogram plots, box
        plots, polar axes, ternary diagrams, smith charts and some more. It is
        based on Till Tantau's package \PGF{}/\Tikz{}.
    \end{minipage}

}%

\expandafter\date\expandafter{\@date\\[2cm]
    \abstractsmuggle
}%
\makeatother

%\includeonly{pgfplots.reference}


\begin{document}

\def\plotcoords{%
\addplot coordinates {
(5,8.312e-02)    (17,2.547e-02)   (49,7.407e-03)
(129,2.102e-03)  (321,5.874e-04)  (769,1.623e-04)
(1793,4.442e-05) (4097,1.207e-05) (9217,3.261e-06)
};

\addplot coordinates{
(7,8.472e-02)    (31,3.044e-02)    (111,1.022e-02)
(351,3.303e-03)  (1023,1.039e-03)  (2815,3.196e-04)
(7423,9.658e-05) (18943,2.873e-05) (47103,8.437e-06)
};

\addplot coordinates{
(9,7.881e-02)     (49,3.243e-02)    (209,1.232e-02)
(769,4.454e-03)   (2561,1.551e-03)  (7937,5.236e-04)
(23297,1.723e-04) (65537,5.545e-05) (178177,1.751e-05)
};

\addplot coordinates{
(11,6.887e-02)    (71,3.177e-02)     (351,1.341e-02)
(1471,5.334e-03)  (5503,2.027e-03)   (18943,7.415e-04)
(61183,2.628e-04) (187903,9.063e-05) (553983,3.053e-05)
};

\addplot coordinates{
(13,5.755e-02)     (97,2.925e-02)     (545,1.351e-02)
(2561,5.842e-03)   (10625,2.397e-03)  (40193,9.414e-04)
(141569,3.564e-04) (471041,1.308e-04)
(1496065,4.670e-05)
};
}%


\bookmaketitle
\tableofcontents
\include{pgfplots.title_abstract_intro}
\include{pgfplots.preliminaries}
\include{pgfplots.intro}
\include{pgfplots.reference}
\include{pgfplots.libs}
\include{pgfplots.resources}
\include{pgfplots.importexport}
\include{pgfplots.basic.reference}

\printindex

\bibliographystyle{abbrv} %gerapali} %gerabbrv} %gerunsrt.bst} %gerabbrv}% gerplain}
\nocite{pgfplotstable}
\nocite{programmingnotes}
\bibliography{pgfplots}
\end{document}

% -----------------------------------------------------------------------------
% For Stefan Pinnow as reminder on what to look for when editing the manual
% -----------------------------------------------------------------------------
% There should be no line breaks in the following environments
% - |...|
% - \declareandlabel{...}
% - \verbpdfref{...}
% If "MakeTikzPictures" isn't running through check one of the externalized
% LOG files what is the cause of that.
% -----------------------------------------------------------------------------


%%%%%%%%%%%%%%%%%%%%%%%%%%%%%%%%%%%%%%%%%%%%%%%%%%%%%%%%%%%%%%%%%%%%%%%%%%%%%
%
% Package pgfplots.sty documentation.
%
% Copyright 2007/2008 by Christian Feuersaenger.
%
% This program is free software: you can redistribute it and/or modify
% it under the terms of the GNU General Public License as published by
% the Free Software Foundation, either version 3 of the License, or
% (at your option) any later version.
%
% This program is distributed in the hope that it will be useful,
% but WITHOUT ANY WARRANTY; without even the implied warranty of
% MERCHANTABILITY or FITNESS FOR A PARTICULAR PURPOSE.  See the
% GNU General Public License for more details.
%
% You should have received a copy of the GNU General Public License
% along with this program.  If not, see <http://www.gnu.org/licenses/>.
%
%
%%%%%%%%%%%%%%%%%%%%%%%%%%%%%%%%%%%%%%%%%%%%%%%%%%%%%%%%%%%%%%%%%%%%%%%%%%%%%
% SEE pgfplots-macros.tex as well!
%\pdfminorversion=5 % to allow compression
%\pdfobjcompresslevel=2
\documentclass[a4paper,openany]{book}

\let\bookmaketitle=\maketitle

% -----------------------------------
% this here is from ltxdoc:
\usepackage{doc}[=v2]
\AtBeginDocument{\MakeShortVerb{\|}}
%\setlength{\textwidth}{355pt}
%\addtolength\marginparwidth{30pt}
%\addtolength\oddsidemargin{20pt}
%\addtolength\evensidemargin{20pt}
\def\cmd#1{\cs{\expandafter\cmd@to@cs\string#1}}
\def\cmd@to@cs#1#2{\char\number`#2\relax}
\DeclareRobustCommand\cs[1]{\texttt{\char`\\#1}}
\providecommand\marg[1]{%
  {\ttfamily\char`\{}\meta{#1}{\ttfamily\char`\}}}
\providecommand\oarg[1]{%
  {\ttfamily[}\meta{#1}{\ttfamily]}}
\providecommand\parg[1]{%
  {\ttfamily(}\meta{#1}{\ttfamily)}}
\raggedbottom

% -----------------------------------
\input{pgfplots.preamble.tex}

\makeatletter
% I want a two-column index just as in pgfmanual styles. This here
% was the best way to get one:
\def\index@prologue{\section*{Index}\addcontentsline{toc}{chapter}{Index}
}
\makeatother

%\RequirePackage[german,english,francais]{babel}

\def\matlabcolormaptext{This colormap is similar to one shipped with Matlab$^\text{\textregistered}$ under a similar name.}

\IfFileExists{tikzlibraryspy.code.tex}{%
\usetikzlibrary{spy}
}{%
    \message{ERROR: tikz SPY library NOT available. The manual will only compile partially.^^J}%
}%

\usepackage{xparse}% for colorbrewer manual

\usetikzlibrary{
    decorations.markings,
    decorations.footprints,
    shapes.arrows,
    matrix,
    positioning,
}

\usepgfplotslibrary{
    fillbetween,
    ternary,
    smithchart,
    patchplots,
    polar,
    colormaps,
    colorbrewer,
    colortol,           % docu not ready yet
}
\pgfqkeys{/codeexample}{%
    every codeexample/.append style={
        /pgfplots/every ternary axis/.append style={
            /pgfplots/legend style={fill=graphicbackground},
        }
    },
    tabsize=4,
}

\pgfplotsmanualenableexternalizationofexpensive

%\usetikzlibrary{external}
%\tikzexternalize[prefix=figures/]{pgfplots}

\title{%
    Manual for Package \PGFPlots{}\\
    {\small 2D/3D Plots in \LaTeX{}, Version \pgfplotsversion}\\
    {\small\url{https://github.com/pgf-tikz/pgfplots}}
    %\\{\small Attention: you are using an unstable development version.}
}

\makeatletter
\long\def\abstractsmuggle{%
    \centering
    \textbf{Abstract}\\[0.5cm]

    \begin{minipage}{12cm}
        \PGFPlots{} draws high-quality function plots in normal or logarithmic
        scaling with a user-friendly interface directly in \TeX{}. The user
        supplies axis labels, legend entries and the plot coordinates for one or
        more plots and \PGFPlots{} applies axis scaling, computes any logarithms
        and axis ticks and draws the plots. It supports line plots, scatter
        plots, piecewise constant plots, bar plots, area plots, mesh and surface
        plots, patch plots, contour plots, quiver plots, histogram plots, box
        plots, polar axes, ternary diagrams, smith charts and some more. It is
        based on Till Tantau's package \PGF{}/\Tikz{}.
    \end{minipage}

}%

\expandafter\date\expandafter{\@date\\[2cm]
    \abstractsmuggle
}%
\makeatother

%\includeonly{pgfplots.reference}


\begin{document}

\def\plotcoords{%
\addplot coordinates {
(5,8.312e-02)    (17,2.547e-02)   (49,7.407e-03)
(129,2.102e-03)  (321,5.874e-04)  (769,1.623e-04)
(1793,4.442e-05) (4097,1.207e-05) (9217,3.261e-06)
};

\addplot coordinates{
(7,8.472e-02)    (31,3.044e-02)    (111,1.022e-02)
(351,3.303e-03)  (1023,1.039e-03)  (2815,3.196e-04)
(7423,9.658e-05) (18943,2.873e-05) (47103,8.437e-06)
};

\addplot coordinates{
(9,7.881e-02)     (49,3.243e-02)    (209,1.232e-02)
(769,4.454e-03)   (2561,1.551e-03)  (7937,5.236e-04)
(23297,1.723e-04) (65537,5.545e-05) (178177,1.751e-05)
};

\addplot coordinates{
(11,6.887e-02)    (71,3.177e-02)     (351,1.341e-02)
(1471,5.334e-03)  (5503,2.027e-03)   (18943,7.415e-04)
(61183,2.628e-04) (187903,9.063e-05) (553983,3.053e-05)
};

\addplot coordinates{
(13,5.755e-02)     (97,2.925e-02)     (545,1.351e-02)
(2561,5.842e-03)   (10625,2.397e-03)  (40193,9.414e-04)
(141569,3.564e-04) (471041,1.308e-04)
(1496065,4.670e-05)
};
}%


\bookmaketitle
\tableofcontents
\include{pgfplots.title_abstract_intro}
\include{pgfplots.preliminaries}
\include{pgfplots.intro}
\include{pgfplots.reference}
\include{pgfplots.libs}
\include{pgfplots.resources}
\include{pgfplots.importexport}
\include{pgfplots.basic.reference}

\printindex

\bibliographystyle{abbrv} %gerapali} %gerabbrv} %gerunsrt.bst} %gerabbrv}% gerplain}
\nocite{pgfplotstable}
\nocite{programmingnotes}
\bibliography{pgfplots}
\end{document}

% -----------------------------------------------------------------------------
% For Stefan Pinnow as reminder on what to look for when editing the manual
% -----------------------------------------------------------------------------
% There should be no line breaks in the following environments
% - |...|
% - \declareandlabel{...}
% - \verbpdfref{...}
% If "MakeTikzPictures" isn't running through check one of the externalized
% LOG files what is the cause of that.
% -----------------------------------------------------------------------------

\include{pgfplots.basic.reference}

\printindex

\bibliographystyle{abbrv} %gerapali} %gerabbrv} %gerunsrt.bst} %gerabbrv}% gerplain}
\nocite{pgfplotstable}
\nocite{programmingnotes}
\bibliography{pgfplots}
\end{document}

% -----------------------------------------------------------------------------
% For Stefan Pinnow as reminder on what to look for when editing the manual
% -----------------------------------------------------------------------------
% There should be no line breaks in the following environments
% - |...|
% - \declareandlabel{...}
% - \verbpdfref{...}
% If "MakeTikzPictures" isn't running through check one of the externalized
% LOG files what is the cause of that.
% -----------------------------------------------------------------------------


%%%%%%%%%%%%%%%%%%%%%%%%%%%%%%%%%%%%%%%%%%%%%%%%%%%%%%%%%%%%%%%%%%%%%%%%%%%%%
%
% Package pgfplots.sty documentation.
%
% Copyright 2007/2008 by Christian Feuersaenger.
%
% This program is free software: you can redistribute it and/or modify
% it under the terms of the GNU General Public License as published by
% the Free Software Foundation, either version 3 of the License, or
% (at your option) any later version.
%
% This program is distributed in the hope that it will be useful,
% but WITHOUT ANY WARRANTY; without even the implied warranty of
% MERCHANTABILITY or FITNESS FOR A PARTICULAR PURPOSE.  See the
% GNU General Public License for more details.
%
% You should have received a copy of the GNU General Public License
% along with this program.  If not, see <http://www.gnu.org/licenses/>.
%
%
%%%%%%%%%%%%%%%%%%%%%%%%%%%%%%%%%%%%%%%%%%%%%%%%%%%%%%%%%%%%%%%%%%%%%%%%%%%%%
% SEE pgfplots-macros.tex as well!
%\pdfminorversion=5 % to allow compression
%\pdfobjcompresslevel=2
\documentclass[a4paper,openany]{book}

\let\bookmaketitle=\maketitle

% -----------------------------------
% this here is from ltxdoc:
\usepackage{doc}[=v2]
\AtBeginDocument{\MakeShortVerb{\|}}
%\setlength{\textwidth}{355pt}
%\addtolength\marginparwidth{30pt}
%\addtolength\oddsidemargin{20pt}
%\addtolength\evensidemargin{20pt}
\def\cmd#1{\cs{\expandafter\cmd@to@cs\string#1}}
\def\cmd@to@cs#1#2{\char\number`#2\relax}
\DeclareRobustCommand\cs[1]{\texttt{\char`\\#1}}
\providecommand\marg[1]{%
  {\ttfamily\char`\{}\meta{#1}{\ttfamily\char`\}}}
\providecommand\oarg[1]{%
  {\ttfamily[}\meta{#1}{\ttfamily]}}
\providecommand\parg[1]{%
  {\ttfamily(}\meta{#1}{\ttfamily)}}
\raggedbottom

% -----------------------------------
\input{pgfplots.preamble.tex}

\makeatletter
% I want a two-column index just as in pgfmanual styles. This here
% was the best way to get one:
\def\index@prologue{\section*{Index}\addcontentsline{toc}{chapter}{Index}
}
\makeatother

%\RequirePackage[german,english,francais]{babel}

\def\matlabcolormaptext{This colormap is similar to one shipped with Matlab$^\text{\textregistered}$ under a similar name.}

\IfFileExists{tikzlibraryspy.code.tex}{%
\usetikzlibrary{spy}
}{%
    \message{ERROR: tikz SPY library NOT available. The manual will only compile partially.^^J}%
}%

\usepackage{xparse}% for colorbrewer manual

\usetikzlibrary{
    decorations.markings,
    decorations.footprints,
    shapes.arrows,
    matrix,
    positioning,
}

\usepgfplotslibrary{
    fillbetween,
    ternary,
    smithchart,
    patchplots,
    polar,
    colormaps,
    colorbrewer,
    colortol,           % docu not ready yet
}
\pgfqkeys{/codeexample}{%
    every codeexample/.append style={
        /pgfplots/every ternary axis/.append style={
            /pgfplots/legend style={fill=graphicbackground},
        }
    },
    tabsize=4,
}

\pgfplotsmanualenableexternalizationofexpensive

%\usetikzlibrary{external}
%\tikzexternalize[prefix=figures/]{pgfplots}

\title{%
    Manual for Package \PGFPlots{}\\
    {\small 2D/3D Plots in \LaTeX{}, Version \pgfplotsversion}\\
    {\small\url{https://github.com/pgf-tikz/pgfplots}}
    %\\{\small Attention: you are using an unstable development version.}
}

\makeatletter
\long\def\abstractsmuggle{%
    \centering
    \textbf{Abstract}\\[0.5cm]

    \begin{minipage}{12cm}
        \PGFPlots{} draws high-quality function plots in normal or logarithmic
        scaling with a user-friendly interface directly in \TeX{}. The user
        supplies axis labels, legend entries and the plot coordinates for one or
        more plots and \PGFPlots{} applies axis scaling, computes any logarithms
        and axis ticks and draws the plots. It supports line plots, scatter
        plots, piecewise constant plots, bar plots, area plots, mesh and surface
        plots, patch plots, contour plots, quiver plots, histogram plots, box
        plots, polar axes, ternary diagrams, smith charts and some more. It is
        based on Till Tantau's package \PGF{}/\Tikz{}.
    \end{minipage}

}%

\expandafter\date\expandafter{\@date\\[2cm]
    \abstractsmuggle
}%
\makeatother

%\includeonly{pgfplots.reference}


\begin{document}

\def\plotcoords{%
\addplot coordinates {
(5,8.312e-02)    (17,2.547e-02)   (49,7.407e-03)
(129,2.102e-03)  (321,5.874e-04)  (769,1.623e-04)
(1793,4.442e-05) (4097,1.207e-05) (9217,3.261e-06)
};

\addplot coordinates{
(7,8.472e-02)    (31,3.044e-02)    (111,1.022e-02)
(351,3.303e-03)  (1023,1.039e-03)  (2815,3.196e-04)
(7423,9.658e-05) (18943,2.873e-05) (47103,8.437e-06)
};

\addplot coordinates{
(9,7.881e-02)     (49,3.243e-02)    (209,1.232e-02)
(769,4.454e-03)   (2561,1.551e-03)  (7937,5.236e-04)
(23297,1.723e-04) (65537,5.545e-05) (178177,1.751e-05)
};

\addplot coordinates{
(11,6.887e-02)    (71,3.177e-02)     (351,1.341e-02)
(1471,5.334e-03)  (5503,2.027e-03)   (18943,7.415e-04)
(61183,2.628e-04) (187903,9.063e-05) (553983,3.053e-05)
};

\addplot coordinates{
(13,5.755e-02)     (97,2.925e-02)     (545,1.351e-02)
(2561,5.842e-03)   (10625,2.397e-03)  (40193,9.414e-04)
(141569,3.564e-04) (471041,1.308e-04)
(1496065,4.670e-05)
};
}%


\bookmaketitle
\tableofcontents

%%%%%%%%%%%%%%%%%%%%%%%%%%%%%%%%%%%%%%%%%%%%%%%%%%%%%%%%%%%%%%%%%%%%%%%%%%%%%
%
% Package pgfplots.sty documentation.
%
% Copyright 2007/2008 by Christian Feuersaenger.
%
% This program is free software: you can redistribute it and/or modify
% it under the terms of the GNU General Public License as published by
% the Free Software Foundation, either version 3 of the License, or
% (at your option) any later version.
%
% This program is distributed in the hope that it will be useful,
% but WITHOUT ANY WARRANTY; without even the implied warranty of
% MERCHANTABILITY or FITNESS FOR A PARTICULAR PURPOSE.  See the
% GNU General Public License for more details.
%
% You should have received a copy of the GNU General Public License
% along with this program.  If not, see <http://www.gnu.org/licenses/>.
%
%
%%%%%%%%%%%%%%%%%%%%%%%%%%%%%%%%%%%%%%%%%%%%%%%%%%%%%%%%%%%%%%%%%%%%%%%%%%%%%
% SEE pgfplots-macros.tex as well!
%\pdfminorversion=5 % to allow compression
%\pdfobjcompresslevel=2
\documentclass[a4paper,openany]{book}

\let\bookmaketitle=\maketitle

% -----------------------------------
% this here is from ltxdoc:
\usepackage{doc}[=v2]
\AtBeginDocument{\MakeShortVerb{\|}}
%\setlength{\textwidth}{355pt}
%\addtolength\marginparwidth{30pt}
%\addtolength\oddsidemargin{20pt}
%\addtolength\evensidemargin{20pt}
\def\cmd#1{\cs{\expandafter\cmd@to@cs\string#1}}
\def\cmd@to@cs#1#2{\char\number`#2\relax}
\DeclareRobustCommand\cs[1]{\texttt{\char`\\#1}}
\providecommand\marg[1]{%
  {\ttfamily\char`\{}\meta{#1}{\ttfamily\char`\}}}
\providecommand\oarg[1]{%
  {\ttfamily[}\meta{#1}{\ttfamily]}}
\providecommand\parg[1]{%
  {\ttfamily(}\meta{#1}{\ttfamily)}}
\raggedbottom

% -----------------------------------
\input{pgfplots.preamble.tex}

\makeatletter
% I want a two-column index just as in pgfmanual styles. This here
% was the best way to get one:
\def\index@prologue{\section*{Index}\addcontentsline{toc}{chapter}{Index}
}
\makeatother

%\RequirePackage[german,english,francais]{babel}

\def\matlabcolormaptext{This colormap is similar to one shipped with Matlab$^\text{\textregistered}$ under a similar name.}

\IfFileExists{tikzlibraryspy.code.tex}{%
\usetikzlibrary{spy}
}{%
    \message{ERROR: tikz SPY library NOT available. The manual will only compile partially.^^J}%
}%

\usepackage{xparse}% for colorbrewer manual

\usetikzlibrary{
    decorations.markings,
    decorations.footprints,
    shapes.arrows,
    matrix,
    positioning,
}

\usepgfplotslibrary{
    fillbetween,
    ternary,
    smithchart,
    patchplots,
    polar,
    colormaps,
    colorbrewer,
    colortol,           % docu not ready yet
}
\pgfqkeys{/codeexample}{%
    every codeexample/.append style={
        /pgfplots/every ternary axis/.append style={
            /pgfplots/legend style={fill=graphicbackground},
        }
    },
    tabsize=4,
}

\pgfplotsmanualenableexternalizationofexpensive

%\usetikzlibrary{external}
%\tikzexternalize[prefix=figures/]{pgfplots}

\title{%
    Manual for Package \PGFPlots{}\\
    {\small 2D/3D Plots in \LaTeX{}, Version \pgfplotsversion}\\
    {\small\url{https://github.com/pgf-tikz/pgfplots}}
    %\\{\small Attention: you are using an unstable development version.}
}

\makeatletter
\long\def\abstractsmuggle{%
    \centering
    \textbf{Abstract}\\[0.5cm]

    \begin{minipage}{12cm}
        \PGFPlots{} draws high-quality function plots in normal or logarithmic
        scaling with a user-friendly interface directly in \TeX{}. The user
        supplies axis labels, legend entries and the plot coordinates for one or
        more plots and \PGFPlots{} applies axis scaling, computes any logarithms
        and axis ticks and draws the plots. It supports line plots, scatter
        plots, piecewise constant plots, bar plots, area plots, mesh and surface
        plots, patch plots, contour plots, quiver plots, histogram plots, box
        plots, polar axes, ternary diagrams, smith charts and some more. It is
        based on Till Tantau's package \PGF{}/\Tikz{}.
    \end{minipage}

}%

\expandafter\date\expandafter{\@date\\[2cm]
    \abstractsmuggle
}%
\makeatother

%\includeonly{pgfplots.reference}


\begin{document}

\def\plotcoords{%
\addplot coordinates {
(5,8.312e-02)    (17,2.547e-02)   (49,7.407e-03)
(129,2.102e-03)  (321,5.874e-04)  (769,1.623e-04)
(1793,4.442e-05) (4097,1.207e-05) (9217,3.261e-06)
};

\addplot coordinates{
(7,8.472e-02)    (31,3.044e-02)    (111,1.022e-02)
(351,3.303e-03)  (1023,1.039e-03)  (2815,3.196e-04)
(7423,9.658e-05) (18943,2.873e-05) (47103,8.437e-06)
};

\addplot coordinates{
(9,7.881e-02)     (49,3.243e-02)    (209,1.232e-02)
(769,4.454e-03)   (2561,1.551e-03)  (7937,5.236e-04)
(23297,1.723e-04) (65537,5.545e-05) (178177,1.751e-05)
};

\addplot coordinates{
(11,6.887e-02)    (71,3.177e-02)     (351,1.341e-02)
(1471,5.334e-03)  (5503,2.027e-03)   (18943,7.415e-04)
(61183,2.628e-04) (187903,9.063e-05) (553983,3.053e-05)
};

\addplot coordinates{
(13,5.755e-02)     (97,2.925e-02)     (545,1.351e-02)
(2561,5.842e-03)   (10625,2.397e-03)  (40193,9.414e-04)
(141569,3.564e-04) (471041,1.308e-04)
(1496065,4.670e-05)
};
}%


\bookmaketitle
\tableofcontents
\include{pgfplots.title_abstract_intro}
\include{pgfplots.preliminaries}
\include{pgfplots.intro}
\include{pgfplots.reference}
\include{pgfplots.libs}
\include{pgfplots.resources}
\include{pgfplots.importexport}
\include{pgfplots.basic.reference}

\printindex

\bibliographystyle{abbrv} %gerapali} %gerabbrv} %gerunsrt.bst} %gerabbrv}% gerplain}
\nocite{pgfplotstable}
\nocite{programmingnotes}
\bibliography{pgfplots}
\end{document}

% -----------------------------------------------------------------------------
% For Stefan Pinnow as reminder on what to look for when editing the manual
% -----------------------------------------------------------------------------
% There should be no line breaks in the following environments
% - |...|
% - \declareandlabel{...}
% - \verbpdfref{...}
% If "MakeTikzPictures" isn't running through check one of the externalized
% LOG files what is the cause of that.
% -----------------------------------------------------------------------------


%%%%%%%%%%%%%%%%%%%%%%%%%%%%%%%%%%%%%%%%%%%%%%%%%%%%%%%%%%%%%%%%%%%%%%%%%%%%%
%
% Package pgfplots.sty documentation.
%
% Copyright 2007/2008 by Christian Feuersaenger.
%
% This program is free software: you can redistribute it and/or modify
% it under the terms of the GNU General Public License as published by
% the Free Software Foundation, either version 3 of the License, or
% (at your option) any later version.
%
% This program is distributed in the hope that it will be useful,
% but WITHOUT ANY WARRANTY; without even the implied warranty of
% MERCHANTABILITY or FITNESS FOR A PARTICULAR PURPOSE.  See the
% GNU General Public License for more details.
%
% You should have received a copy of the GNU General Public License
% along with this program.  If not, see <http://www.gnu.org/licenses/>.
%
%
%%%%%%%%%%%%%%%%%%%%%%%%%%%%%%%%%%%%%%%%%%%%%%%%%%%%%%%%%%%%%%%%%%%%%%%%%%%%%
% SEE pgfplots-macros.tex as well!
%\pdfminorversion=5 % to allow compression
%\pdfobjcompresslevel=2
\documentclass[a4paper,openany]{book}

\let\bookmaketitle=\maketitle

% -----------------------------------
% this here is from ltxdoc:
\usepackage{doc}[=v2]
\AtBeginDocument{\MakeShortVerb{\|}}
%\setlength{\textwidth}{355pt}
%\addtolength\marginparwidth{30pt}
%\addtolength\oddsidemargin{20pt}
%\addtolength\evensidemargin{20pt}
\def\cmd#1{\cs{\expandafter\cmd@to@cs\string#1}}
\def\cmd@to@cs#1#2{\char\number`#2\relax}
\DeclareRobustCommand\cs[1]{\texttt{\char`\\#1}}
\providecommand\marg[1]{%
  {\ttfamily\char`\{}\meta{#1}{\ttfamily\char`\}}}
\providecommand\oarg[1]{%
  {\ttfamily[}\meta{#1}{\ttfamily]}}
\providecommand\parg[1]{%
  {\ttfamily(}\meta{#1}{\ttfamily)}}
\raggedbottom

% -----------------------------------
\input{pgfplots.preamble.tex}

\makeatletter
% I want a two-column index just as in pgfmanual styles. This here
% was the best way to get one:
\def\index@prologue{\section*{Index}\addcontentsline{toc}{chapter}{Index}
}
\makeatother

%\RequirePackage[german,english,francais]{babel}

\def\matlabcolormaptext{This colormap is similar to one shipped with Matlab$^\text{\textregistered}$ under a similar name.}

\IfFileExists{tikzlibraryspy.code.tex}{%
\usetikzlibrary{spy}
}{%
    \message{ERROR: tikz SPY library NOT available. The manual will only compile partially.^^J}%
}%

\usepackage{xparse}% for colorbrewer manual

\usetikzlibrary{
    decorations.markings,
    decorations.footprints,
    shapes.arrows,
    matrix,
    positioning,
}

\usepgfplotslibrary{
    fillbetween,
    ternary,
    smithchart,
    patchplots,
    polar,
    colormaps,
    colorbrewer,
    colortol,           % docu not ready yet
}
\pgfqkeys{/codeexample}{%
    every codeexample/.append style={
        /pgfplots/every ternary axis/.append style={
            /pgfplots/legend style={fill=graphicbackground},
        }
    },
    tabsize=4,
}

\pgfplotsmanualenableexternalizationofexpensive

%\usetikzlibrary{external}
%\tikzexternalize[prefix=figures/]{pgfplots}

\title{%
    Manual for Package \PGFPlots{}\\
    {\small 2D/3D Plots in \LaTeX{}, Version \pgfplotsversion}\\
    {\small\url{https://github.com/pgf-tikz/pgfplots}}
    %\\{\small Attention: you are using an unstable development version.}
}

\makeatletter
\long\def\abstractsmuggle{%
    \centering
    \textbf{Abstract}\\[0.5cm]

    \begin{minipage}{12cm}
        \PGFPlots{} draws high-quality function plots in normal or logarithmic
        scaling with a user-friendly interface directly in \TeX{}. The user
        supplies axis labels, legend entries and the plot coordinates for one or
        more plots and \PGFPlots{} applies axis scaling, computes any logarithms
        and axis ticks and draws the plots. It supports line plots, scatter
        plots, piecewise constant plots, bar plots, area plots, mesh and surface
        plots, patch plots, contour plots, quiver plots, histogram plots, box
        plots, polar axes, ternary diagrams, smith charts and some more. It is
        based on Till Tantau's package \PGF{}/\Tikz{}.
    \end{minipage}

}%

\expandafter\date\expandafter{\@date\\[2cm]
    \abstractsmuggle
}%
\makeatother

%\includeonly{pgfplots.reference}


\begin{document}

\def\plotcoords{%
\addplot coordinates {
(5,8.312e-02)    (17,2.547e-02)   (49,7.407e-03)
(129,2.102e-03)  (321,5.874e-04)  (769,1.623e-04)
(1793,4.442e-05) (4097,1.207e-05) (9217,3.261e-06)
};

\addplot coordinates{
(7,8.472e-02)    (31,3.044e-02)    (111,1.022e-02)
(351,3.303e-03)  (1023,1.039e-03)  (2815,3.196e-04)
(7423,9.658e-05) (18943,2.873e-05) (47103,8.437e-06)
};

\addplot coordinates{
(9,7.881e-02)     (49,3.243e-02)    (209,1.232e-02)
(769,4.454e-03)   (2561,1.551e-03)  (7937,5.236e-04)
(23297,1.723e-04) (65537,5.545e-05) (178177,1.751e-05)
};

\addplot coordinates{
(11,6.887e-02)    (71,3.177e-02)     (351,1.341e-02)
(1471,5.334e-03)  (5503,2.027e-03)   (18943,7.415e-04)
(61183,2.628e-04) (187903,9.063e-05) (553983,3.053e-05)
};

\addplot coordinates{
(13,5.755e-02)     (97,2.925e-02)     (545,1.351e-02)
(2561,5.842e-03)   (10625,2.397e-03)  (40193,9.414e-04)
(141569,3.564e-04) (471041,1.308e-04)
(1496065,4.670e-05)
};
}%


\bookmaketitle
\tableofcontents
\include{pgfplots.title_abstract_intro}
\include{pgfplots.preliminaries}
\include{pgfplots.intro}
\include{pgfplots.reference}
\include{pgfplots.libs}
\include{pgfplots.resources}
\include{pgfplots.importexport}
\include{pgfplots.basic.reference}

\printindex

\bibliographystyle{abbrv} %gerapali} %gerabbrv} %gerunsrt.bst} %gerabbrv}% gerplain}
\nocite{pgfplotstable}
\nocite{programmingnotes}
\bibliography{pgfplots}
\end{document}

% -----------------------------------------------------------------------------
% For Stefan Pinnow as reminder on what to look for when editing the manual
% -----------------------------------------------------------------------------
% There should be no line breaks in the following environments
% - |...|
% - \declareandlabel{...}
% - \verbpdfref{...}
% If "MakeTikzPictures" isn't running through check one of the externalized
% LOG files what is the cause of that.
% -----------------------------------------------------------------------------


%%%%%%%%%%%%%%%%%%%%%%%%%%%%%%%%%%%%%%%%%%%%%%%%%%%%%%%%%%%%%%%%%%%%%%%%%%%%%
%
% Package pgfplots.sty documentation.
%
% Copyright 2007/2008 by Christian Feuersaenger.
%
% This program is free software: you can redistribute it and/or modify
% it under the terms of the GNU General Public License as published by
% the Free Software Foundation, either version 3 of the License, or
% (at your option) any later version.
%
% This program is distributed in the hope that it will be useful,
% but WITHOUT ANY WARRANTY; without even the implied warranty of
% MERCHANTABILITY or FITNESS FOR A PARTICULAR PURPOSE.  See the
% GNU General Public License for more details.
%
% You should have received a copy of the GNU General Public License
% along with this program.  If not, see <http://www.gnu.org/licenses/>.
%
%
%%%%%%%%%%%%%%%%%%%%%%%%%%%%%%%%%%%%%%%%%%%%%%%%%%%%%%%%%%%%%%%%%%%%%%%%%%%%%
% SEE pgfplots-macros.tex as well!
%\pdfminorversion=5 % to allow compression
%\pdfobjcompresslevel=2
\documentclass[a4paper,openany]{book}

\let\bookmaketitle=\maketitle

% -----------------------------------
% this here is from ltxdoc:
\usepackage{doc}[=v2]
\AtBeginDocument{\MakeShortVerb{\|}}
%\setlength{\textwidth}{355pt}
%\addtolength\marginparwidth{30pt}
%\addtolength\oddsidemargin{20pt}
%\addtolength\evensidemargin{20pt}
\def\cmd#1{\cs{\expandafter\cmd@to@cs\string#1}}
\def\cmd@to@cs#1#2{\char\number`#2\relax}
\DeclareRobustCommand\cs[1]{\texttt{\char`\\#1}}
\providecommand\marg[1]{%
  {\ttfamily\char`\{}\meta{#1}{\ttfamily\char`\}}}
\providecommand\oarg[1]{%
  {\ttfamily[}\meta{#1}{\ttfamily]}}
\providecommand\parg[1]{%
  {\ttfamily(}\meta{#1}{\ttfamily)}}
\raggedbottom

% -----------------------------------
\input{pgfplots.preamble.tex}

\makeatletter
% I want a two-column index just as in pgfmanual styles. This here
% was the best way to get one:
\def\index@prologue{\section*{Index}\addcontentsline{toc}{chapter}{Index}
}
\makeatother

%\RequirePackage[german,english,francais]{babel}

\def\matlabcolormaptext{This colormap is similar to one shipped with Matlab$^\text{\textregistered}$ under a similar name.}

\IfFileExists{tikzlibraryspy.code.tex}{%
\usetikzlibrary{spy}
}{%
    \message{ERROR: tikz SPY library NOT available. The manual will only compile partially.^^J}%
}%

\usepackage{xparse}% for colorbrewer manual

\usetikzlibrary{
    decorations.markings,
    decorations.footprints,
    shapes.arrows,
    matrix,
    positioning,
}

\usepgfplotslibrary{
    fillbetween,
    ternary,
    smithchart,
    patchplots,
    polar,
    colormaps,
    colorbrewer,
    colortol,           % docu not ready yet
}
\pgfqkeys{/codeexample}{%
    every codeexample/.append style={
        /pgfplots/every ternary axis/.append style={
            /pgfplots/legend style={fill=graphicbackground},
        }
    },
    tabsize=4,
}

\pgfplotsmanualenableexternalizationofexpensive

%\usetikzlibrary{external}
%\tikzexternalize[prefix=figures/]{pgfplots}

\title{%
    Manual for Package \PGFPlots{}\\
    {\small 2D/3D Plots in \LaTeX{}, Version \pgfplotsversion}\\
    {\small\url{https://github.com/pgf-tikz/pgfplots}}
    %\\{\small Attention: you are using an unstable development version.}
}

\makeatletter
\long\def\abstractsmuggle{%
    \centering
    \textbf{Abstract}\\[0.5cm]

    \begin{minipage}{12cm}
        \PGFPlots{} draws high-quality function plots in normal or logarithmic
        scaling with a user-friendly interface directly in \TeX{}. The user
        supplies axis labels, legend entries and the plot coordinates for one or
        more plots and \PGFPlots{} applies axis scaling, computes any logarithms
        and axis ticks and draws the plots. It supports line plots, scatter
        plots, piecewise constant plots, bar plots, area plots, mesh and surface
        plots, patch plots, contour plots, quiver plots, histogram plots, box
        plots, polar axes, ternary diagrams, smith charts and some more. It is
        based on Till Tantau's package \PGF{}/\Tikz{}.
    \end{minipage}

}%

\expandafter\date\expandafter{\@date\\[2cm]
    \abstractsmuggle
}%
\makeatother

%\includeonly{pgfplots.reference}


\begin{document}

\def\plotcoords{%
\addplot coordinates {
(5,8.312e-02)    (17,2.547e-02)   (49,7.407e-03)
(129,2.102e-03)  (321,5.874e-04)  (769,1.623e-04)
(1793,4.442e-05) (4097,1.207e-05) (9217,3.261e-06)
};

\addplot coordinates{
(7,8.472e-02)    (31,3.044e-02)    (111,1.022e-02)
(351,3.303e-03)  (1023,1.039e-03)  (2815,3.196e-04)
(7423,9.658e-05) (18943,2.873e-05) (47103,8.437e-06)
};

\addplot coordinates{
(9,7.881e-02)     (49,3.243e-02)    (209,1.232e-02)
(769,4.454e-03)   (2561,1.551e-03)  (7937,5.236e-04)
(23297,1.723e-04) (65537,5.545e-05) (178177,1.751e-05)
};

\addplot coordinates{
(11,6.887e-02)    (71,3.177e-02)     (351,1.341e-02)
(1471,5.334e-03)  (5503,2.027e-03)   (18943,7.415e-04)
(61183,2.628e-04) (187903,9.063e-05) (553983,3.053e-05)
};

\addplot coordinates{
(13,5.755e-02)     (97,2.925e-02)     (545,1.351e-02)
(2561,5.842e-03)   (10625,2.397e-03)  (40193,9.414e-04)
(141569,3.564e-04) (471041,1.308e-04)
(1496065,4.670e-05)
};
}%


\bookmaketitle
\tableofcontents
\include{pgfplots.title_abstract_intro}
\include{pgfplots.preliminaries}
\include{pgfplots.intro}
\include{pgfplots.reference}
\include{pgfplots.libs}
\include{pgfplots.resources}
\include{pgfplots.importexport}
\include{pgfplots.basic.reference}

\printindex

\bibliographystyle{abbrv} %gerapali} %gerabbrv} %gerunsrt.bst} %gerabbrv}% gerplain}
\nocite{pgfplotstable}
\nocite{programmingnotes}
\bibliography{pgfplots}
\end{document}

% -----------------------------------------------------------------------------
% For Stefan Pinnow as reminder on what to look for when editing the manual
% -----------------------------------------------------------------------------
% There should be no line breaks in the following environments
% - |...|
% - \declareandlabel{...}
% - \verbpdfref{...}
% If "MakeTikzPictures" isn't running through check one of the externalized
% LOG files what is the cause of that.
% -----------------------------------------------------------------------------


%%%%%%%%%%%%%%%%%%%%%%%%%%%%%%%%%%%%%%%%%%%%%%%%%%%%%%%%%%%%%%%%%%%%%%%%%%%%%
%
% Package pgfplots.sty documentation.
%
% Copyright 2007/2008 by Christian Feuersaenger.
%
% This program is free software: you can redistribute it and/or modify
% it under the terms of the GNU General Public License as published by
% the Free Software Foundation, either version 3 of the License, or
% (at your option) any later version.
%
% This program is distributed in the hope that it will be useful,
% but WITHOUT ANY WARRANTY; without even the implied warranty of
% MERCHANTABILITY or FITNESS FOR A PARTICULAR PURPOSE.  See the
% GNU General Public License for more details.
%
% You should have received a copy of the GNU General Public License
% along with this program.  If not, see <http://www.gnu.org/licenses/>.
%
%
%%%%%%%%%%%%%%%%%%%%%%%%%%%%%%%%%%%%%%%%%%%%%%%%%%%%%%%%%%%%%%%%%%%%%%%%%%%%%
% SEE pgfplots-macros.tex as well!
%\pdfminorversion=5 % to allow compression
%\pdfobjcompresslevel=2
\documentclass[a4paper,openany]{book}

\let\bookmaketitle=\maketitle

% -----------------------------------
% this here is from ltxdoc:
\usepackage{doc}[=v2]
\AtBeginDocument{\MakeShortVerb{\|}}
%\setlength{\textwidth}{355pt}
%\addtolength\marginparwidth{30pt}
%\addtolength\oddsidemargin{20pt}
%\addtolength\evensidemargin{20pt}
\def\cmd#1{\cs{\expandafter\cmd@to@cs\string#1}}
\def\cmd@to@cs#1#2{\char\number`#2\relax}
\DeclareRobustCommand\cs[1]{\texttt{\char`\\#1}}
\providecommand\marg[1]{%
  {\ttfamily\char`\{}\meta{#1}{\ttfamily\char`\}}}
\providecommand\oarg[1]{%
  {\ttfamily[}\meta{#1}{\ttfamily]}}
\providecommand\parg[1]{%
  {\ttfamily(}\meta{#1}{\ttfamily)}}
\raggedbottom

% -----------------------------------
\input{pgfplots.preamble.tex}

\makeatletter
% I want a two-column index just as in pgfmanual styles. This here
% was the best way to get one:
\def\index@prologue{\section*{Index}\addcontentsline{toc}{chapter}{Index}
}
\makeatother

%\RequirePackage[german,english,francais]{babel}

\def\matlabcolormaptext{This colormap is similar to one shipped with Matlab$^\text{\textregistered}$ under a similar name.}

\IfFileExists{tikzlibraryspy.code.tex}{%
\usetikzlibrary{spy}
}{%
    \message{ERROR: tikz SPY library NOT available. The manual will only compile partially.^^J}%
}%

\usepackage{xparse}% for colorbrewer manual

\usetikzlibrary{
    decorations.markings,
    decorations.footprints,
    shapes.arrows,
    matrix,
    positioning,
}

\usepgfplotslibrary{
    fillbetween,
    ternary,
    smithchart,
    patchplots,
    polar,
    colormaps,
    colorbrewer,
    colortol,           % docu not ready yet
}
\pgfqkeys{/codeexample}{%
    every codeexample/.append style={
        /pgfplots/every ternary axis/.append style={
            /pgfplots/legend style={fill=graphicbackground},
        }
    },
    tabsize=4,
}

\pgfplotsmanualenableexternalizationofexpensive

%\usetikzlibrary{external}
%\tikzexternalize[prefix=figures/]{pgfplots}

\title{%
    Manual for Package \PGFPlots{}\\
    {\small 2D/3D Plots in \LaTeX{}, Version \pgfplotsversion}\\
    {\small\url{https://github.com/pgf-tikz/pgfplots}}
    %\\{\small Attention: you are using an unstable development version.}
}

\makeatletter
\long\def\abstractsmuggle{%
    \centering
    \textbf{Abstract}\\[0.5cm]

    \begin{minipage}{12cm}
        \PGFPlots{} draws high-quality function plots in normal or logarithmic
        scaling with a user-friendly interface directly in \TeX{}. The user
        supplies axis labels, legend entries and the plot coordinates for one or
        more plots and \PGFPlots{} applies axis scaling, computes any logarithms
        and axis ticks and draws the plots. It supports line plots, scatter
        plots, piecewise constant plots, bar plots, area plots, mesh and surface
        plots, patch plots, contour plots, quiver plots, histogram plots, box
        plots, polar axes, ternary diagrams, smith charts and some more. It is
        based on Till Tantau's package \PGF{}/\Tikz{}.
    \end{minipage}

}%

\expandafter\date\expandafter{\@date\\[2cm]
    \abstractsmuggle
}%
\makeatother

%\includeonly{pgfplots.reference}


\begin{document}

\def\plotcoords{%
\addplot coordinates {
(5,8.312e-02)    (17,2.547e-02)   (49,7.407e-03)
(129,2.102e-03)  (321,5.874e-04)  (769,1.623e-04)
(1793,4.442e-05) (4097,1.207e-05) (9217,3.261e-06)
};

\addplot coordinates{
(7,8.472e-02)    (31,3.044e-02)    (111,1.022e-02)
(351,3.303e-03)  (1023,1.039e-03)  (2815,3.196e-04)
(7423,9.658e-05) (18943,2.873e-05) (47103,8.437e-06)
};

\addplot coordinates{
(9,7.881e-02)     (49,3.243e-02)    (209,1.232e-02)
(769,4.454e-03)   (2561,1.551e-03)  (7937,5.236e-04)
(23297,1.723e-04) (65537,5.545e-05) (178177,1.751e-05)
};

\addplot coordinates{
(11,6.887e-02)    (71,3.177e-02)     (351,1.341e-02)
(1471,5.334e-03)  (5503,2.027e-03)   (18943,7.415e-04)
(61183,2.628e-04) (187903,9.063e-05) (553983,3.053e-05)
};

\addplot coordinates{
(13,5.755e-02)     (97,2.925e-02)     (545,1.351e-02)
(2561,5.842e-03)   (10625,2.397e-03)  (40193,9.414e-04)
(141569,3.564e-04) (471041,1.308e-04)
(1496065,4.670e-05)
};
}%


\bookmaketitle
\tableofcontents
\include{pgfplots.title_abstract_intro}
\include{pgfplots.preliminaries}
\include{pgfplots.intro}
\include{pgfplots.reference}
\include{pgfplots.libs}
\include{pgfplots.resources}
\include{pgfplots.importexport}
\include{pgfplots.basic.reference}

\printindex

\bibliographystyle{abbrv} %gerapali} %gerabbrv} %gerunsrt.bst} %gerabbrv}% gerplain}
\nocite{pgfplotstable}
\nocite{programmingnotes}
\bibliography{pgfplots}
\end{document}

% -----------------------------------------------------------------------------
% For Stefan Pinnow as reminder on what to look for when editing the manual
% -----------------------------------------------------------------------------
% There should be no line breaks in the following environments
% - |...|
% - \declareandlabel{...}
% - \verbpdfref{...}
% If "MakeTikzPictures" isn't running through check one of the externalized
% LOG files what is the cause of that.
% -----------------------------------------------------------------------------


%%%%%%%%%%%%%%%%%%%%%%%%%%%%%%%%%%%%%%%%%%%%%%%%%%%%%%%%%%%%%%%%%%%%%%%%%%%%%
%
% Package pgfplots.sty documentation.
%
% Copyright 2007/2008 by Christian Feuersaenger.
%
% This program is free software: you can redistribute it and/or modify
% it under the terms of the GNU General Public License as published by
% the Free Software Foundation, either version 3 of the License, or
% (at your option) any later version.
%
% This program is distributed in the hope that it will be useful,
% but WITHOUT ANY WARRANTY; without even the implied warranty of
% MERCHANTABILITY or FITNESS FOR A PARTICULAR PURPOSE.  See the
% GNU General Public License for more details.
%
% You should have received a copy of the GNU General Public License
% along with this program.  If not, see <http://www.gnu.org/licenses/>.
%
%
%%%%%%%%%%%%%%%%%%%%%%%%%%%%%%%%%%%%%%%%%%%%%%%%%%%%%%%%%%%%%%%%%%%%%%%%%%%%%
% SEE pgfplots-macros.tex as well!
%\pdfminorversion=5 % to allow compression
%\pdfobjcompresslevel=2
\documentclass[a4paper,openany]{book}

\let\bookmaketitle=\maketitle

% -----------------------------------
% this here is from ltxdoc:
\usepackage{doc}[=v2]
\AtBeginDocument{\MakeShortVerb{\|}}
%\setlength{\textwidth}{355pt}
%\addtolength\marginparwidth{30pt}
%\addtolength\oddsidemargin{20pt}
%\addtolength\evensidemargin{20pt}
\def\cmd#1{\cs{\expandafter\cmd@to@cs\string#1}}
\def\cmd@to@cs#1#2{\char\number`#2\relax}
\DeclareRobustCommand\cs[1]{\texttt{\char`\\#1}}
\providecommand\marg[1]{%
  {\ttfamily\char`\{}\meta{#1}{\ttfamily\char`\}}}
\providecommand\oarg[1]{%
  {\ttfamily[}\meta{#1}{\ttfamily]}}
\providecommand\parg[1]{%
  {\ttfamily(}\meta{#1}{\ttfamily)}}
\raggedbottom

% -----------------------------------
\input{pgfplots.preamble.tex}

\makeatletter
% I want a two-column index just as in pgfmanual styles. This here
% was the best way to get one:
\def\index@prologue{\section*{Index}\addcontentsline{toc}{chapter}{Index}
}
\makeatother

%\RequirePackage[german,english,francais]{babel}

\def\matlabcolormaptext{This colormap is similar to one shipped with Matlab$^\text{\textregistered}$ under a similar name.}

\IfFileExists{tikzlibraryspy.code.tex}{%
\usetikzlibrary{spy}
}{%
    \message{ERROR: tikz SPY library NOT available. The manual will only compile partially.^^J}%
}%

\usepackage{xparse}% for colorbrewer manual

\usetikzlibrary{
    decorations.markings,
    decorations.footprints,
    shapes.arrows,
    matrix,
    positioning,
}

\usepgfplotslibrary{
    fillbetween,
    ternary,
    smithchart,
    patchplots,
    polar,
    colormaps,
    colorbrewer,
    colortol,           % docu not ready yet
}
\pgfqkeys{/codeexample}{%
    every codeexample/.append style={
        /pgfplots/every ternary axis/.append style={
            /pgfplots/legend style={fill=graphicbackground},
        }
    },
    tabsize=4,
}

\pgfplotsmanualenableexternalizationofexpensive

%\usetikzlibrary{external}
%\tikzexternalize[prefix=figures/]{pgfplots}

\title{%
    Manual for Package \PGFPlots{}\\
    {\small 2D/3D Plots in \LaTeX{}, Version \pgfplotsversion}\\
    {\small\url{https://github.com/pgf-tikz/pgfplots}}
    %\\{\small Attention: you are using an unstable development version.}
}

\makeatletter
\long\def\abstractsmuggle{%
    \centering
    \textbf{Abstract}\\[0.5cm]

    \begin{minipage}{12cm}
        \PGFPlots{} draws high-quality function plots in normal or logarithmic
        scaling with a user-friendly interface directly in \TeX{}. The user
        supplies axis labels, legend entries and the plot coordinates for one or
        more plots and \PGFPlots{} applies axis scaling, computes any logarithms
        and axis ticks and draws the plots. It supports line plots, scatter
        plots, piecewise constant plots, bar plots, area plots, mesh and surface
        plots, patch plots, contour plots, quiver plots, histogram plots, box
        plots, polar axes, ternary diagrams, smith charts and some more. It is
        based on Till Tantau's package \PGF{}/\Tikz{}.
    \end{minipage}

}%

\expandafter\date\expandafter{\@date\\[2cm]
    \abstractsmuggle
}%
\makeatother

%\includeonly{pgfplots.reference}


\begin{document}

\def\plotcoords{%
\addplot coordinates {
(5,8.312e-02)    (17,2.547e-02)   (49,7.407e-03)
(129,2.102e-03)  (321,5.874e-04)  (769,1.623e-04)
(1793,4.442e-05) (4097,1.207e-05) (9217,3.261e-06)
};

\addplot coordinates{
(7,8.472e-02)    (31,3.044e-02)    (111,1.022e-02)
(351,3.303e-03)  (1023,1.039e-03)  (2815,3.196e-04)
(7423,9.658e-05) (18943,2.873e-05) (47103,8.437e-06)
};

\addplot coordinates{
(9,7.881e-02)     (49,3.243e-02)    (209,1.232e-02)
(769,4.454e-03)   (2561,1.551e-03)  (7937,5.236e-04)
(23297,1.723e-04) (65537,5.545e-05) (178177,1.751e-05)
};

\addplot coordinates{
(11,6.887e-02)    (71,3.177e-02)     (351,1.341e-02)
(1471,5.334e-03)  (5503,2.027e-03)   (18943,7.415e-04)
(61183,2.628e-04) (187903,9.063e-05) (553983,3.053e-05)
};

\addplot coordinates{
(13,5.755e-02)     (97,2.925e-02)     (545,1.351e-02)
(2561,5.842e-03)   (10625,2.397e-03)  (40193,9.414e-04)
(141569,3.564e-04) (471041,1.308e-04)
(1496065,4.670e-05)
};
}%


\bookmaketitle
\tableofcontents
\include{pgfplots.title_abstract_intro}
\include{pgfplots.preliminaries}
\include{pgfplots.intro}
\include{pgfplots.reference}
\include{pgfplots.libs}
\include{pgfplots.resources}
\include{pgfplots.importexport}
\include{pgfplots.basic.reference}

\printindex

\bibliographystyle{abbrv} %gerapali} %gerabbrv} %gerunsrt.bst} %gerabbrv}% gerplain}
\nocite{pgfplotstable}
\nocite{programmingnotes}
\bibliography{pgfplots}
\end{document}

% -----------------------------------------------------------------------------
% For Stefan Pinnow as reminder on what to look for when editing the manual
% -----------------------------------------------------------------------------
% There should be no line breaks in the following environments
% - |...|
% - \declareandlabel{...}
% - \verbpdfref{...}
% If "MakeTikzPictures" isn't running through check one of the externalized
% LOG files what is the cause of that.
% -----------------------------------------------------------------------------


%%%%%%%%%%%%%%%%%%%%%%%%%%%%%%%%%%%%%%%%%%%%%%%%%%%%%%%%%%%%%%%%%%%%%%%%%%%%%
%
% Package pgfplots.sty documentation.
%
% Copyright 2007/2008 by Christian Feuersaenger.
%
% This program is free software: you can redistribute it and/or modify
% it under the terms of the GNU General Public License as published by
% the Free Software Foundation, either version 3 of the License, or
% (at your option) any later version.
%
% This program is distributed in the hope that it will be useful,
% but WITHOUT ANY WARRANTY; without even the implied warranty of
% MERCHANTABILITY or FITNESS FOR A PARTICULAR PURPOSE.  See the
% GNU General Public License for more details.
%
% You should have received a copy of the GNU General Public License
% along with this program.  If not, see <http://www.gnu.org/licenses/>.
%
%
%%%%%%%%%%%%%%%%%%%%%%%%%%%%%%%%%%%%%%%%%%%%%%%%%%%%%%%%%%%%%%%%%%%%%%%%%%%%%
% SEE pgfplots-macros.tex as well!
%\pdfminorversion=5 % to allow compression
%\pdfobjcompresslevel=2
\documentclass[a4paper,openany]{book}

\let\bookmaketitle=\maketitle

% -----------------------------------
% this here is from ltxdoc:
\usepackage{doc}[=v2]
\AtBeginDocument{\MakeShortVerb{\|}}
%\setlength{\textwidth}{355pt}
%\addtolength\marginparwidth{30pt}
%\addtolength\oddsidemargin{20pt}
%\addtolength\evensidemargin{20pt}
\def\cmd#1{\cs{\expandafter\cmd@to@cs\string#1}}
\def\cmd@to@cs#1#2{\char\number`#2\relax}
\DeclareRobustCommand\cs[1]{\texttt{\char`\\#1}}
\providecommand\marg[1]{%
  {\ttfamily\char`\{}\meta{#1}{\ttfamily\char`\}}}
\providecommand\oarg[1]{%
  {\ttfamily[}\meta{#1}{\ttfamily]}}
\providecommand\parg[1]{%
  {\ttfamily(}\meta{#1}{\ttfamily)}}
\raggedbottom

% -----------------------------------
\input{pgfplots.preamble.tex}

\makeatletter
% I want a two-column index just as in pgfmanual styles. This here
% was the best way to get one:
\def\index@prologue{\section*{Index}\addcontentsline{toc}{chapter}{Index}
}
\makeatother

%\RequirePackage[german,english,francais]{babel}

\def\matlabcolormaptext{This colormap is similar to one shipped with Matlab$^\text{\textregistered}$ under a similar name.}

\IfFileExists{tikzlibraryspy.code.tex}{%
\usetikzlibrary{spy}
}{%
    \message{ERROR: tikz SPY library NOT available. The manual will only compile partially.^^J}%
}%

\usepackage{xparse}% for colorbrewer manual

\usetikzlibrary{
    decorations.markings,
    decorations.footprints,
    shapes.arrows,
    matrix,
    positioning,
}

\usepgfplotslibrary{
    fillbetween,
    ternary,
    smithchart,
    patchplots,
    polar,
    colormaps,
    colorbrewer,
    colortol,           % docu not ready yet
}
\pgfqkeys{/codeexample}{%
    every codeexample/.append style={
        /pgfplots/every ternary axis/.append style={
            /pgfplots/legend style={fill=graphicbackground},
        }
    },
    tabsize=4,
}

\pgfplotsmanualenableexternalizationofexpensive

%\usetikzlibrary{external}
%\tikzexternalize[prefix=figures/]{pgfplots}

\title{%
    Manual for Package \PGFPlots{}\\
    {\small 2D/3D Plots in \LaTeX{}, Version \pgfplotsversion}\\
    {\small\url{https://github.com/pgf-tikz/pgfplots}}
    %\\{\small Attention: you are using an unstable development version.}
}

\makeatletter
\long\def\abstractsmuggle{%
    \centering
    \textbf{Abstract}\\[0.5cm]

    \begin{minipage}{12cm}
        \PGFPlots{} draws high-quality function plots in normal or logarithmic
        scaling with a user-friendly interface directly in \TeX{}. The user
        supplies axis labels, legend entries and the plot coordinates for one or
        more plots and \PGFPlots{} applies axis scaling, computes any logarithms
        and axis ticks and draws the plots. It supports line plots, scatter
        plots, piecewise constant plots, bar plots, area plots, mesh and surface
        plots, patch plots, contour plots, quiver plots, histogram plots, box
        plots, polar axes, ternary diagrams, smith charts and some more. It is
        based on Till Tantau's package \PGF{}/\Tikz{}.
    \end{minipage}

}%

\expandafter\date\expandafter{\@date\\[2cm]
    \abstractsmuggle
}%
\makeatother

%\includeonly{pgfplots.reference}


\begin{document}

\def\plotcoords{%
\addplot coordinates {
(5,8.312e-02)    (17,2.547e-02)   (49,7.407e-03)
(129,2.102e-03)  (321,5.874e-04)  (769,1.623e-04)
(1793,4.442e-05) (4097,1.207e-05) (9217,3.261e-06)
};

\addplot coordinates{
(7,8.472e-02)    (31,3.044e-02)    (111,1.022e-02)
(351,3.303e-03)  (1023,1.039e-03)  (2815,3.196e-04)
(7423,9.658e-05) (18943,2.873e-05) (47103,8.437e-06)
};

\addplot coordinates{
(9,7.881e-02)     (49,3.243e-02)    (209,1.232e-02)
(769,4.454e-03)   (2561,1.551e-03)  (7937,5.236e-04)
(23297,1.723e-04) (65537,5.545e-05) (178177,1.751e-05)
};

\addplot coordinates{
(11,6.887e-02)    (71,3.177e-02)     (351,1.341e-02)
(1471,5.334e-03)  (5503,2.027e-03)   (18943,7.415e-04)
(61183,2.628e-04) (187903,9.063e-05) (553983,3.053e-05)
};

\addplot coordinates{
(13,5.755e-02)     (97,2.925e-02)     (545,1.351e-02)
(2561,5.842e-03)   (10625,2.397e-03)  (40193,9.414e-04)
(141569,3.564e-04) (471041,1.308e-04)
(1496065,4.670e-05)
};
}%


\bookmaketitle
\tableofcontents
\include{pgfplots.title_abstract_intro}
\include{pgfplots.preliminaries}
\include{pgfplots.intro}
\include{pgfplots.reference}
\include{pgfplots.libs}
\include{pgfplots.resources}
\include{pgfplots.importexport}
\include{pgfplots.basic.reference}

\printindex

\bibliographystyle{abbrv} %gerapali} %gerabbrv} %gerunsrt.bst} %gerabbrv}% gerplain}
\nocite{pgfplotstable}
\nocite{programmingnotes}
\bibliography{pgfplots}
\end{document}

% -----------------------------------------------------------------------------
% For Stefan Pinnow as reminder on what to look for when editing the manual
% -----------------------------------------------------------------------------
% There should be no line breaks in the following environments
% - |...|
% - \declareandlabel{...}
% - \verbpdfref{...}
% If "MakeTikzPictures" isn't running through check one of the externalized
% LOG files what is the cause of that.
% -----------------------------------------------------------------------------


%%%%%%%%%%%%%%%%%%%%%%%%%%%%%%%%%%%%%%%%%%%%%%%%%%%%%%%%%%%%%%%%%%%%%%%%%%%%%
%
% Package pgfplots.sty documentation.
%
% Copyright 2007/2008 by Christian Feuersaenger.
%
% This program is free software: you can redistribute it and/or modify
% it under the terms of the GNU General Public License as published by
% the Free Software Foundation, either version 3 of the License, or
% (at your option) any later version.
%
% This program is distributed in the hope that it will be useful,
% but WITHOUT ANY WARRANTY; without even the implied warranty of
% MERCHANTABILITY or FITNESS FOR A PARTICULAR PURPOSE.  See the
% GNU General Public License for more details.
%
% You should have received a copy of the GNU General Public License
% along with this program.  If not, see <http://www.gnu.org/licenses/>.
%
%
%%%%%%%%%%%%%%%%%%%%%%%%%%%%%%%%%%%%%%%%%%%%%%%%%%%%%%%%%%%%%%%%%%%%%%%%%%%%%
% SEE pgfplots-macros.tex as well!
%\pdfminorversion=5 % to allow compression
%\pdfobjcompresslevel=2
\documentclass[a4paper,openany]{book}

\let\bookmaketitle=\maketitle

% -----------------------------------
% this here is from ltxdoc:
\usepackage{doc}[=v2]
\AtBeginDocument{\MakeShortVerb{\|}}
%\setlength{\textwidth}{355pt}
%\addtolength\marginparwidth{30pt}
%\addtolength\oddsidemargin{20pt}
%\addtolength\evensidemargin{20pt}
\def\cmd#1{\cs{\expandafter\cmd@to@cs\string#1}}
\def\cmd@to@cs#1#2{\char\number`#2\relax}
\DeclareRobustCommand\cs[1]{\texttt{\char`\\#1}}
\providecommand\marg[1]{%
  {\ttfamily\char`\{}\meta{#1}{\ttfamily\char`\}}}
\providecommand\oarg[1]{%
  {\ttfamily[}\meta{#1}{\ttfamily]}}
\providecommand\parg[1]{%
  {\ttfamily(}\meta{#1}{\ttfamily)}}
\raggedbottom

% -----------------------------------
\input{pgfplots.preamble.tex}

\makeatletter
% I want a two-column index just as in pgfmanual styles. This here
% was the best way to get one:
\def\index@prologue{\section*{Index}\addcontentsline{toc}{chapter}{Index}
}
\makeatother

%\RequirePackage[german,english,francais]{babel}

\def\matlabcolormaptext{This colormap is similar to one shipped with Matlab$^\text{\textregistered}$ under a similar name.}

\IfFileExists{tikzlibraryspy.code.tex}{%
\usetikzlibrary{spy}
}{%
    \message{ERROR: tikz SPY library NOT available. The manual will only compile partially.^^J}%
}%

\usepackage{xparse}% for colorbrewer manual

\usetikzlibrary{
    decorations.markings,
    decorations.footprints,
    shapes.arrows,
    matrix,
    positioning,
}

\usepgfplotslibrary{
    fillbetween,
    ternary,
    smithchart,
    patchplots,
    polar,
    colormaps,
    colorbrewer,
    colortol,           % docu not ready yet
}
\pgfqkeys{/codeexample}{%
    every codeexample/.append style={
        /pgfplots/every ternary axis/.append style={
            /pgfplots/legend style={fill=graphicbackground},
        }
    },
    tabsize=4,
}

\pgfplotsmanualenableexternalizationofexpensive

%\usetikzlibrary{external}
%\tikzexternalize[prefix=figures/]{pgfplots}

\title{%
    Manual for Package \PGFPlots{}\\
    {\small 2D/3D Plots in \LaTeX{}, Version \pgfplotsversion}\\
    {\small\url{https://github.com/pgf-tikz/pgfplots}}
    %\\{\small Attention: you are using an unstable development version.}
}

\makeatletter
\long\def\abstractsmuggle{%
    \centering
    \textbf{Abstract}\\[0.5cm]

    \begin{minipage}{12cm}
        \PGFPlots{} draws high-quality function plots in normal or logarithmic
        scaling with a user-friendly interface directly in \TeX{}. The user
        supplies axis labels, legend entries and the plot coordinates for one or
        more plots and \PGFPlots{} applies axis scaling, computes any logarithms
        and axis ticks and draws the plots. It supports line plots, scatter
        plots, piecewise constant plots, bar plots, area plots, mesh and surface
        plots, patch plots, contour plots, quiver plots, histogram plots, box
        plots, polar axes, ternary diagrams, smith charts and some more. It is
        based on Till Tantau's package \PGF{}/\Tikz{}.
    \end{minipage}

}%

\expandafter\date\expandafter{\@date\\[2cm]
    \abstractsmuggle
}%
\makeatother

%\includeonly{pgfplots.reference}


\begin{document}

\def\plotcoords{%
\addplot coordinates {
(5,8.312e-02)    (17,2.547e-02)   (49,7.407e-03)
(129,2.102e-03)  (321,5.874e-04)  (769,1.623e-04)
(1793,4.442e-05) (4097,1.207e-05) (9217,3.261e-06)
};

\addplot coordinates{
(7,8.472e-02)    (31,3.044e-02)    (111,1.022e-02)
(351,3.303e-03)  (1023,1.039e-03)  (2815,3.196e-04)
(7423,9.658e-05) (18943,2.873e-05) (47103,8.437e-06)
};

\addplot coordinates{
(9,7.881e-02)     (49,3.243e-02)    (209,1.232e-02)
(769,4.454e-03)   (2561,1.551e-03)  (7937,5.236e-04)
(23297,1.723e-04) (65537,5.545e-05) (178177,1.751e-05)
};

\addplot coordinates{
(11,6.887e-02)    (71,3.177e-02)     (351,1.341e-02)
(1471,5.334e-03)  (5503,2.027e-03)   (18943,7.415e-04)
(61183,2.628e-04) (187903,9.063e-05) (553983,3.053e-05)
};

\addplot coordinates{
(13,5.755e-02)     (97,2.925e-02)     (545,1.351e-02)
(2561,5.842e-03)   (10625,2.397e-03)  (40193,9.414e-04)
(141569,3.564e-04) (471041,1.308e-04)
(1496065,4.670e-05)
};
}%


\bookmaketitle
\tableofcontents
\include{pgfplots.title_abstract_intro}
\include{pgfplots.preliminaries}
\include{pgfplots.intro}
\include{pgfplots.reference}
\include{pgfplots.libs}
\include{pgfplots.resources}
\include{pgfplots.importexport}
\include{pgfplots.basic.reference}

\printindex

\bibliographystyle{abbrv} %gerapali} %gerabbrv} %gerunsrt.bst} %gerabbrv}% gerplain}
\nocite{pgfplotstable}
\nocite{programmingnotes}
\bibliography{pgfplots}
\end{document}

% -----------------------------------------------------------------------------
% For Stefan Pinnow as reminder on what to look for when editing the manual
% -----------------------------------------------------------------------------
% There should be no line breaks in the following environments
% - |...|
% - \declareandlabel{...}
% - \verbpdfref{...}
% If "MakeTikzPictures" isn't running through check one of the externalized
% LOG files what is the cause of that.
% -----------------------------------------------------------------------------

\include{pgfplots.basic.reference}

\printindex

\bibliographystyle{abbrv} %gerapali} %gerabbrv} %gerunsrt.bst} %gerabbrv}% gerplain}
\nocite{pgfplotstable}
\nocite{programmingnotes}
\bibliography{pgfplots}
\end{document}

% -----------------------------------------------------------------------------
% For Stefan Pinnow as reminder on what to look for when editing the manual
% -----------------------------------------------------------------------------
% There should be no line breaks in the following environments
% - |...|
% - \declareandlabel{...}
% - \verbpdfref{...}
% If "MakeTikzPictures" isn't running through check one of the externalized
% LOG files what is the cause of that.
% -----------------------------------------------------------------------------

\include{pgfplots.basic.reference}

\printindex

\bibliographystyle{abbrv} %gerapali} %gerabbrv} %gerunsrt.bst} %gerabbrv}% gerplain}
\nocite{pgfplotstable}
\nocite{programmingnotes}
\bibliography{pgfplots}
\end{document}

% -----------------------------------------------------------------------------
% For Stefan Pinnow as reminder on what to look for when editing the manual
% -----------------------------------------------------------------------------
% There should be no line breaks in the following environments
% - |...|
% - \declareandlabel{...}
% - \verbpdfref{...}
% If "MakeTikzPictures" isn't running through check one of the externalized
% LOG files what is the cause of that.
% -----------------------------------------------------------------------------

\include{pgfplots.basic.reference}

\printindex

\bibliographystyle{abbrv} %gerapali} %gerabbrv} %gerunsrt.bst} %gerabbrv}% gerplain}
\nocite{pgfplotstable}
\nocite{programmingnotes}
\bibliography{pgfplots}
\end{document}

% -----------------------------------------------------------------------------
% For Stefan Pinnow as reminder on what to look for when editing the manual
% -----------------------------------------------------------------------------
% There should be no line breaks in the following environments
% - |...|
% - \declareandlabel{...}
% - \verbpdfref{...}
% If "MakeTikzPictures" isn't running through check one of the externalized
% LOG files what is the cause of that.
% -----------------------------------------------------------------------------
