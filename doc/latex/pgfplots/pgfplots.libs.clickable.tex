
\section{Clickable Plots}

\begingroup
\def\pgfplotsmanualcurlibrary{clickable}
    \expandafter\ifx\csname pgfplotsclickabledisabled\endcsname\relax
    \else
        \pgfkeysdef{/pgfplots/clickable coords}{}%
        \pgfkeysdef{/pgfplots/clickable coords code}{}%
    \fi

\begin{pgfplotslibrary}{clickable}
    A library which generates small popups whenever one clicks into a plot. The
    popup displays the coordinate under the mouse pointer, supporting the
    optional ``snap to nearest'' |clickable coords| feature with customizable
    displayed information. Furthermore, the library allows to display slopes if
    one holds the mouse pressed and drags it to another point in the plot.

    The library has two purposes: to compute slopes in a simple
    way\footnote{The author is applied mathematician\ldots} and to provide
    related, optional information to single data points which are not important
    enough to be listed in the main text (like prototype parameters or other
    technical things).
\end{pgfplotslibrary}


\subsection{Overview}

It is completely sufficient to write
%
\begin{codeexample}[code only]
\usepgfplotslibrary{clickable}
\end{codeexample}
%
\noindent in the document preamble. This will automatically prepare every plot.

The library works with Acrobat JavaScript and \pdf{} forms: every plot becomes
a push button.

    \includegraphics[height=6cm]{figures/pgfplotsclickable-fig1.png}
    \rlap{\includegraphics[height=6cm]{figures/pgfplotsclickable-fig2.png}}\hfill

\nobreak These screenshots show the result of clicking into the axis range
(left column) and of dragging from one point to another (right column). The
second case shows the result of Drag-and-Drop: it displays start and end points
and the equation for the line segment between between the first point of the
drag and drop and the second point where the mouse has been released. The line
segment is
%
    \[ l(x; x_0,y_0,x_1,y_1) = m \cdot x + n \]
%
where $m = (y_1-y_0) / (x_1-x_0)$ is the slope and $n$ the offset chosen such
that $l(x_0;\dotsc) = y_0$. For logarithmic plots, logarithms will be applied
before computing slopes.

    \noindent
    \hbox to \linewidth{%
    \hspace{-0.5cm}%
    \begin{tikzpicture}
        \node at (8cm,0cm)    {\includegraphics[height=6cm]{figures/pgfplotsclickable-fig4.png}};
        \node at (0cm,0cm)    {\includegraphics[height=6cm]{figures/pgfplotsclickable-fig3.png}};
    \end{tikzpicture}\hss}%

\nobreak These screen shots show the result of drag- and drop for
\emph{logarithmic} axes: the end points show, again, the coordinates (without
logs) and the form field in the middle shows the slope and offset of the linear
equation in log coordinates.

The log basis for any logarithmic axes is usually~$10$, but it respects the
current setting of |log basis x| and |log basis y|. The applied log will always
use the same logarithm which is also used for the axis descriptions (this is
not necessarily the same as used by \PGFPlotstable!).

This document has been produced with the |clickable| library, so it is possible
to load it into Acrobat Reader and simply click into a plot.

    \expandafter\ifx\csname pgfplotsclickabledisabled\endcsname\relax
    \else
    \paragraph{Attention:}
    For this document, the |clickable| library has been deactivated. You may
    find a different version on \url{https://github.com/pgf-tikz/pgfplots}.
    \fi

\begin{pgfplotskey}{clickable coords=\marg{displayed text}}
    Activates a ``snap to nearest'' feature when clicking onto plot
    coordinates. The \meta{displayed text} is the coordinate's $x$ and $y$
    value by default (i.e.\@ you write just |clickable coords| without an equal
    sign).
    %
\begin{codeexample}[]
\begin{tikzpicture}
\begin{loglogaxis}[
    clickable coords={
        Level \thisrow{level} (q=\thisrow{q})
    },
]
    \addplot table [x=dof,y=error] {
        level  dof      error           q
        1      4        2.50000000e-01  48
        2      16       6.25000000e-02  25
        3      64       1.56250000e-02  41
        4      256      3.90625000e-03  8
        5      1024     9.76562500e-04  22
        6      4096     2.44140625e-04  46
        7      16384    6.10351562e-05  40
        8      65536    1.52587891e-05  3
        9      262144   3.81469727e-06  1
        10     1048576  9.53674316e-07  9
    };
\end{loglogaxis}
\end{tikzpicture}
\end{codeexample}
    %
    \noindent Now, clicking onto a data point yields `Level 7 (q=40)' whereas
    clicking besides a data point results in the click coordinates as before,

        \noindent\hbox to \linewidth{\hfill
        \includegraphics[scale=0.4]{figures/pgfplotsclickable-log-snap0}\hfill
        \includegraphics[scale=0.4]{figures/pgfplotsclickable-log-snap2}\hfill
        \includegraphics[scale=0.4]{figures/pgfplotsclickable-log-snap1}.\hfill
        }%

    Note that logarithmic slopes work as before.

    If you want the $(x,y)$ values to be displayed, use the special placeholder
    string `|(xy)|' inside of \meta{displayed text}. As an example, we consider
    again the |scatter/classes| example of
    page~\pageref{pgfplots:scatterclasses}:
    %
\begin{codeexample}[]
\begin{tikzpicture}
\begin{axis}[
    clickable coords={(xy): \thisrow{label}},
    scatter/classes={
        a={mark=square*,blue},
        b={mark=triangle*,red},
        c={mark=o,draw=black}   % <-- don't add comma
    }
]
    \addplot [
        only marks,
        scatter,
        scatter src=explicit symbolic
    ] table [meta=label] {
        x     y      label
        0.1   0.15   a
        0.45  0.27   c
        0.02  0.17   a
        0.06  0.1    a
        0.9   0.5    b
        0.5   0.3    c
        0.85  0.52   b
        0.12  0.05   a
        0.73  0.45   b
        0.53  0.25   c
        0.76  0.5    b
        0.55  0.32   c
    };
\end{axis}
\end{tikzpicture}
\end{codeexample}
    %
    \noindent Here, we find popups like

        \noindent\hbox to \linewidth{\hfill
        \includegraphics[scale=0.4]{figures/pgfplotsclickable-scatter1.png}\hfill
        \includegraphics[scale=0.4]{figures/pgfplotsclickable-scatter2.png}\hfill
        \includegraphics[scale=0.4]{figures/pgfplotsclickable-scatter0.png}.\hfill
        }%

    The \meta{displayed text} is a richtext string displayed with
    \emph{JavaScript}. For most purposes, it is used like an unformatted C
    string: it contains characters, perhaps line breaks with `|\n|' or
    tabulators with `|\t|', but it should not contain \TeX{} formatting
    instructions, especially no math mode (the `|(xy)|' replacement text is
    formatted with |sprintf|, see below). Consider |clickable coords code| in
    case you'd like to preprocess data before displaying it. If you experience
    problems with special characters, try prepending a backslash to them. If
    that doesn't work either, try to prefix the word with `|\\|' and/or with
    `|\string|'. Consider using |clickable coords size| if you intend to work
    with multiline fields and the size allocation needs improvements.

    In fact, \meta{displayed text} can even contain richtext (=XHTML)
    formatting instructions like `|<br/>|' (note the final slash) or
    `|<span style="color:\#7E0000;">text</span>|' (note the backslash before
    `|#|') which changes the color for |text|. The |<span style="">| arguments
    are CSS fields, consider an HTML reference for a list of CSS attributes.

    It is possible to use |clickable coords| together with three dimensional
    axes. Note that dynamic (clickable) features of a three dimensional axis
    without |clickable coords| will be disabled (they appear to be useless).
    Furthermore, three dimensional axes do not support slope calculations; only
    the ``snap to nearest'' feature is available.

    Consider using |annot/snap dist=6| to increase the ``snap to nearest''
    distance.

    The |clickable coords| can be specified for all plots in an axis (as in the
    examples above), but also once for every single |\addplot| commands for
    which the ``snap to nearest'' feature is desired (with different
    \meta{displayed text}).

    If multiple |clickable coords| are on the same position, each click chooses
    the next one (in the order of appearance).
\end{pgfplotskey}

\begin{pgfplotskey}{clickable coords code=\marg{%
    \TeX{} code which defines {\normalfont\ttfamily\textbackslash pgfplotsretval}}%
}
    A variant of |clickable coords| which allows to prepare the displayed
    information before it is handed over to JavaScript.

    The value should be \TeX{} code which defines |\pgfplotsretval| somehow.
    The result is used as simple, unformatted string which is associated to
    coordinates.

    Consider using

    \hspace{2em}|\pgfmathprintnumberto[verbatim]|\marg{number}|\macroname|

    \hspace{2em}|\edef\pgfplotsretval{Number=\macroname}|

    to provide number printing. The |\pgfmathprintnumberto[verbatim]| doesn't
    use math mode to format a number,\footnote{See the \PGFPlotstable{} manual
    for details about number printing.} and it writes its result into
    |\macroname|. The name `|\macroname|' is arbitrary, use anything like
    `|\eps|' or `|\info|'. The |\edef| means ``expanded definition'' and has
    the effect of expanding all macros to determine the value, in our case
    ``Number = \meta{the value}''. The following example uses it twice to
    pretty-print the data:
    %
\begin{codeexample}[]
\begin{tikzpicture}
\begin{loglogaxis}[clickable coords code={
    \pgfmathprintnumberto[verbatim,precision=1]
        {\thisrow{error}}
        \error
    \pgfmathprintnumberto[verbatim,frac]
        {\thisrow{frac}}
        \fraccomp
    \edef\pgfplotsretval{error \error, R=\fraccomp}
}]
    \addplot table [x=dof,y=error] {
        level  dof     error           frac
        1      4       2.50000000e-01  0.5
        2      16      6.25000000e-02  0.75
        3      64      1.56250000e-02  0.1
        4      256     3.90625000e-03  0.2
        5      1024    9.76562500e-04  0.5
        6      4096    2.44140625e-04  0.8
        7      16384   6.10351562e-05  0.125
        8      65536   1.52587891e-05  0.725
        9      262144  3.81469727e-06  0.625
        10     1048576 9.53674316e-07  1
    };
\end{loglogaxis}
\end{tikzpicture}
\end{codeexample}
    %
    \noindent resulting in

        \noindent\hbox to \linewidth{\hfill
        \includegraphics[scale=0.4]{figures/pgfplotsclickable-logcode-snap0.png}\hfill
        \includegraphics[scale=0.4]{figures/pgfplotsclickable-logcode-snap1.png}.\hfill
        }%

    The \meta{\TeX{} code} is evaluated inside of a local scope, all locally
    declared variables are freed afterwards (that's why you can use any names
    you want).
\end{pgfplotskey}

\begin{pgfplotskey}{%
    clickable coords size=\texttt{auto} or \marg{max chars} or
    \marg{max chars x,max chars y} (initially auto)%
}
    This is actually just another name for |annot/popup size snap|, see its
    documentation below.
\end{pgfplotskey}


\subsection{Requirements for the Library}

\begin{itemize}
    \item The library relies on the \LaTeX{} packages |insdljs| (``Insert
        document level JavaScript'') and |eforms| which are both part of the
        freely available |AcroTeX| education
        bundle~\cite{acrotex}.\footnote{These packages rely on \LaTeX{}, so
        the library is only available for \LaTeX{}, not for plain \TeX{} or
        Con\TeX{}t.} The |insdljs| package creates a temporary file with
        extension |.djs|.
    \item At the time of this writing, only Adobe Acrobat Reader interprets
        JavaScript and Forms properly. The library doesn't have any effect if
        the resulting document is used in other viewers (as far as I know).
\end{itemize}
%
Note that although this library has been written for \PGFPlots{}, it can be
used independently of a \PGFPlots{} environment.


\paragraph{Compatibility issues:}

There a several restrictions when using this library. Most of them will vanish
in future versions -- but up to now, I can't do magic.
%
\begin{itemize}
    \item The library does not yet support rotated axes. Use
        |clickable=false| for those axes.
    \item The library works only with |pdflatex|; |dvips| or |dvipdfm| are not
        supported.\footnote{In fact, they should be. I don't really know why
        they don't. Any hint is welcome.}
    \item Up to now, it is \emph{not} possible to use this library together
        with the |external| library and other image externalization methods of
        Chapter~\ref{cha:pgfplots:importexport}.

        To be more precise, you can (with two extra preamble lines, see below)
        get correctly annotated, exported \pdf{} documents, but the
        |\includegraphics| command does not import the dynamic features.

        In case you decide to use this workaround, you need to insert
        %
\begin{codeexample}[code only]
% \maxdeadcycles=10000 % in case you get the error `Output loop---<N> consecutive dead cycles.'
\usepackage[pdftex]{eforms}
\end{codeexample}
        %
        \noindent \emph{before} loading \pgfname{}, \Tikz{} or \PGFPlots{}. The
        |\maxdeadcycles| appears to be necessary for large documents, try it
        out.

        As long as you are working on a draft version of your document, you
        might want to use
        %
\begin{codeexample}[code only]
\pgfkeys{/pgf/images/include external/.code={\href{file:#1}{\pgfimage{#1}}}}
\end{codeexample}
        %
        in your preamble. This will generate hyperlinks around the graphics
        files which link to the exported figures. Clicking on the hyperlinks
        opens the exported figure which, in turn, has been generated with the
        |clickable| library and allows dynamic features.\footnote{This
        special treatment needs the external files in the same base directory
        as the main document, so this approach is most certainly \emph{not}
        suitable for a final document.}
    \item The library automatically calls |\begin{Form}| at
        |\begin{document}| and |\end{Form}| at the end of the document. This
        environment of |hyperref| is necessary for dynamic user interaction
        and should be kept in mind if the document contains other form
        elements.
\end{itemize}

\paragraph{Acknowledgements:}

\begin{itemize}
    \item I have used a JavaScript |sprintf| implementation of Kevin van
        Zonneveld~\cite{phptojs} (the JavaScript API has only a limited set
        of conversions).
\end{itemize}


\subsection{Customization}

It is possible to customize the library with several options.

\begin{pgfplotskey}{clickable=\mchoice{true,false} (initially true)}
    Allows to disable the library for single plots.
\end{pgfplotskey}

\begin{pgfplotskey}{annot/js fillColor=\marg{JavaScript color} (initially ["RGB",1,1,.855])}
    Sets the background (fill) color of the short popup annotations.

    Possible choices are |transparent|, gray, RGB or CMYK color specified as
    four element arrays of the form
    |["RGB", |\meta{red}|,|\meta{green}|,|\meta{blue}|]|. Each color component
    is between $0$ and $1$.

    Again: this option is for JavaScript. It is \emph{not} possible to use
    colors as in \pgfname{}.
\end{pgfplotskey}

\begin{pgfplotskeylist}{%
    annot/point format=\marg{sprintf-format} (initially {(\%.1f,\%.1f)}),
    annot/point format 3d=\marg{sprintf-format} (initially {(\%.1f,\%.1f,\%.1f)})%
}
    Allows to provide an |sprintf| format string which is used to fill the
    annotations with text. The first argument to |sprintf| is the
    $x$-coordinate and the second argument is the $y$-coordinate.

    The |point format 3d| variant is used for any three-dimensional axis
    whereas the |point format| is used (only) for two-dimensional ones.

    The |every semilogx axis|, |every semilogy axis| and |every loglog axis|
    styles have been updated to
    %
\begin{codeexample}[code only]
\pgfplotsset{
    every semilogy axis/.append style={/pgfplots/annot/point format={(\%.1f,\%.1e)}},
    every semilogx axis/.append style={/pgfplots/annot/point format={(\%.1e,\%.1f)}},
    every loglog axis/.append style={/pgfplots/annot/point format={(\%.1e,\%.1e)}},
}
\end{codeexample}
    %
    \noindent such
    %\todosp{is there a special reason, why the last two entries are not in blue? (on previous page they are blue ...}
    % ... a bug in the highlighting code, I suppose. Fixing it is quite involved and I keep it for now
    that every logarithmic coordinate is displayed in scientific
    format.
\end{pgfplotskeylist}

\begin{pgfplotskey}{annot/slope format=\marg{sprintf-format} (initially \%.1f*x \%+.1f)}
    Allows to provide an |sprintf| format string which is used to fill the
    slope annotation with text. The first argument is the slope and the second
    the line offset.
\end{pgfplotskey}

\begin{pgfplotskey}{annot/printable=\mchoice{true,false} (initially false)}
    Allows to configure whether the small annotations will be printed.
    Otherwise, they are only available on screen.
\end{pgfplotskey}

\begin{pgfplotskey}{annot/font=\marg{JavaScript font name} (initially font.Times)}
    Allows to choose a JavaScript font for the annotations. Possible choices
    are limited to what JavaScript accepts (which is \emph{not} the same as
    \LaTeX{}). The default fonts and its names are shown below.

    \begin{center}
        \begin{tabular}{ll}
                \toprule
            Font Name             & Name in JavaScript \\
                \midrule
            Times-Roman           & font.Times         \\
            Times-Bold            & font.TimesB        \\
            Times-Italic          & font.TimesI        \\
            Times-BoldItalic      & font.TimesBI       \\
            Helvetica             & font.Helv          \\
            Helvetica-Bold        & font.HelvB         \\
            Helvetica-Oblique     & font.HelvI         \\
            Helvetica-BoldOblique & font.HelvBI        \\
            Courier               & font.Cour          \\
            Courier-Bold          & font.CourB         \\
            Courier-Oblique       & font.CourI         \\
            Courier-BoldOblique   & font.CourBI        \\
            Symbol                & font.Symbol        \\
            ZapfDingbats          & font.ZapfD         \\
                \bottomrule
        \end{tabular}
    \end{center}
\end{pgfplotskey}

\begin{pgfplotskey}{annot/textSize=\marg{Size in Point} (initially 11)}
    Sets the text size of annotations in points.
\end{pgfplotskey}

\begin{pgfplotskeylist}{%
    annot/popup size generic=\texttt{auto} or \marg{x} or \marg{x,y} (initially auto),
    annot/popup size snap=\texttt{auto} or \marg{x} or \marg{x,y} (initially auto),
    annot/popup size=\marg{value}%
}
    The first key defines the size of popups if you just click into an axis.
    The second key defines the size of popups for the ``snap to nearest''
    feature (i.e.\@ those prepared by |clickable coords|). The third key sets
    both to the same \meta{value}.

    The argument can be |auto| in which case \PGFPlots{} tries to be smart and
    counts characters. This may fail for multiline texts. The choice \meta{x}
    provides the \emph{horizontal} size only, in units of |annot/textSize|.
    Thus, |annot/popup size generic=6| makes the popup $6\cdot 11$ points wide.
    In this case, only one line will be allocated. Finally, \meta{x,y} allows
    to provide horizontal and vertical size, both in units of |annot/textSize|.

    See also |clickable coords size| which is an alias for
    |annot/popup size snap|.
\end{pgfplotskeylist}

\begin{pgfplotskey}{annot/snap dist=\marg{Size in Point} (initially 4)}
    Defines the size within two mouse clicks are considered to be equivalent,
    measured in points (Euclidean distance).
\end{pgfplotskey}

\begin{pgfplotskey}{annot/richtext=\mchoice{true,false} (initially true)}
    Enables or disables richtext formatting in |clickable coords| arguments.
    Richtext is kind of XHTML and allows CSS styles like colors, font changes
    and other CSS attributes, see the documentation for |clickable coords| for
    details.

    The case |annot/richtext=false| is probably more robust.
\end{pgfplotskey}


\subsection{Using the Clickable Library in Other Contexts}

This library provides essentially one command, |\pgfplotsclickablecreate| which
creates a clickable area of predefined size, combined with JavaScript
interaction code. It can be used independently of \PGFPlots{}.

\begin{command}{\pgfplotsclickablecreate\oarg{required key-value-options}}
    Creates an area which is clickable. A click produces a popup which contains
    information about the point under the cursor.

    The complete (!) context needs to be provided using key--value pairs, either
    set before calling this method of inside of
    \oarg{required key-value-options}.

    This command actually creates an AcroForm which invokes JavaScript whenever
    it is clicked. A JavaScript Object is created which represents the context
    (axis limits and options). This JavaScript object is available at runtime.

    This method is public and it is \emph{not} restricted to \PGFPlots{}. The
    \PGFPlots{} hook simply initializes the required key--value pairs.

    This method does not draw anything. It initializes only a clickable area
    and JavaScript code.

    The required key--value pairs are documented below.

    \paragraph{Attention:}

    Complete key--value validation is \emph{not} performed here. It can happen
    that invalid options will produce JavaScript bugs when opened with Acrobat
    Reader. Use the JavaScript console to find them.
\end{command}

\noindent All options described in the following are only interesting for users
who intend to use this library without \PGFPlots{}.

\begin{pgfplotskey}{annot/width=\marg{dimension} (initially -)}
    This required key communicates the area's width to
    |\pgfplotsclickablecreate|. It must be a \TeX{} dimension like |5cm|.
\end{pgfplotskey}

\begin{pgfplotskey}{annot/height=\marg{dimension} (initially -)}
    This required key communicates the area's height to
    |\pgfplotsclickablecreate|. It must be a \TeX{} dimension like |5cm|.
\end{pgfplotskey}

\begin{pgfplotskey}{annot/jsname=\marg{string} (initially -)}
    This required key communicates a unique identifier to
    |\pgfplotsclickablecreate|. This identifier is used to identify the object
    in JavaScript, so there can't be more than one of them. If it is empty, a
    default identifier will be generated.
\end{pgfplotskey}

\begin{pgfplotskeylist}{%
    annot/xmin=\marg{number},
    annot/xmax=\marg{number},
    annot/ymin=\marg{number},
    annot/ymax=\marg{number} (initially empty)%
}
    These required keys communicate the axis limits to
    |\pgfplotsclickablecreate|. They should be set to numbers which can be
    assigned to a JavaScript floating point number (standard IEEE double
    precision).
\end{pgfplotskeylist}

\begin{pgfplotskey}{annot/collected plots=\marg{nested arrays} (initially empty)}
    The low level interface to implement a ``snap to nearest'' feature. The
    value is an array of plots, where each plot is again an array of
    coordinates and each coordinate is an array of three elements, $x$, $y$ and
    text. Please consult the code comments for details and examples.
\end{pgfplotskey}
\endgroup
